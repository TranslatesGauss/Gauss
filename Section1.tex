\documentclass{book}
\usepackage{standalone}%
\usepackage[dvips,text={6.5truein,9truein},left=1truein,top=1truein]{geometry}
\usepackage{amsmath, amsthm}%
\usepackage{titlesec}

% Uncomment to use syncing
%\usepackage{pdfsync}


% Paragraphs
\parindent=5pt
\parskip=5 pt plus 2 pt minus 1pt

\titleformat{\section}
  {\normalfont\large\bfseries\centering}{\thesection.}{1em}{}

\titleformat{\subsection}
  {\normalfont\normalsize\centering}{\thesection.}{1em}{}

\theoremstyle{theorem}
\newtheorem{theorem}{Theorem}
\newtheorem{corollary}{Corollary}
\newtheorem{definition}{Definition}
\newtheorem{proposition}{Proposition}
\newtheorem{lemma}{Lemma}
\newtheorem{problem}{Problem}


\begin{document}

\section*{ On the Congruence of Numbers in General } 

\subsection*{Art. 1\\ \begin{small}\textit{(Congruences, moduli, residues and nonresidues)} \end{small} 
} 
 If the difference between numbers $b$ and $c$ is a multiple of a number $a$, then $b$ and $c$ are said to be congruent modulo $a$, otherwise they are incongruent; $a$ is called the modulus of the congruence.  In the former case, each of the numbers $b$ and $c$ is called a residue of the other, and in the latter case, a non-residue.

These notions apply to all integers, both positive and negative, but they are not to be extended to fractions.  For example, $-9$ and $16$ are congruent modulo $5$; $-7$ and $15$ are residues of each other modulo $11$, but not modulo $3$.  Moreover, since zero is divisible by all numbers, every number must be regarded as congruent to itself according to any modulus.

\subsection*{Art. 2} 

All of the numbers congruent to $a$ modulo $m$ are given by the formula $a+km$ where $k$ denotes an indeterminate integer.  The propositions which we shall give hereafter can be derived from this with no difficulty; but indeed their truth is equally easy to demonstrate directly.

We will denote congruences with the sign $\equiv$, and if necessary the modulus will be in parentheses, e.g. $-16 \equiv 9 \; (\mathrm{mod.} 5)$, $-7 \equiv 15  \;(\mathrm{mod.} 11)$. 

\subsection*{Art. 3} \begin{theorem} Given $m$ consecutive whole numbers
\[ a, a+1, a+2,\dots, a+m-1 \]
and another $A$, one and only one of them will be congruent to $A$ modulo $m$.
\end{theorem}
If indeed $\frac{a-A}{m}$ is an integer, we have $a \equiv A$.  If it is a fraction, let the next largest integer (or the next smallest, if it is negative and sign is not taken into account) be $k$.  Then $A + km$ will be between $a$ and $a+m$, and thus it will be the number sought.  It is clear that all of the quotients $\frac{a-A}{m}$, $\frac{a+1-A}{m}$, $\frac{a+2-A}{m}$, etc. lie between $k-1$ and $k+1$, therefore no more than one of them can be a whole number.

\subsection*{Art. 4 \\ \begin{small}\textit{(Minimal Residues)}\end{small}}  It follows that each number has a residue in the series $0,1,2,\dots,m-1$, and also in the series $0,-1,-2,\dots,1-m$.  We will call it the minimum residue, and it is clear that unless $0$ is the remainder then two will be given, one positive and the other negative.  If they are unequal in size then one will be $< \frac{m}{2}$, otherwise both will be $=\frac{m}{2}$, the sign not being considered.  From this it is clear that any residue not exceeding half of the modulus can be called an absolute minimum.

For example, $-13$ modulo $5$ has minimum positive residue $2$, which is the absolute minimum, and also $-3$, which is the minimum negative residue.  $5$ modulo $7$ is its own minimum positive residue, $-2$ is the negative minimum residue, which is also the absolute minimum.

\subsection*{Art. 5 (Elementary propositions about congruences)}  Having established these notions, let us collect those properties of congruent numbers which present themselves most immediately.

\begin{proposition} Numbers which are congruent according to a composite modulus, are also congruent according to any divisor of the modulus. \end{proposition}

\begin{proposition} If several numbers are each congruent to the same number according to the same modulus, then they will also be congruent to each other (according to that modulus). \end{proposition}

\begin{proposition}  Congruent numbers have the same minimum residues, incongruent ones have different ones. \end{proposition}

\subsection*{Art. 6} Given any number of numbers $A$, $B$, $C$, etc. and as many others $a$, $b$, $c$, etc., and any modulus whatsoever, 
\[ \textit{If  }A \equiv a, B \equiv b \;\mathrm{ etc.} \textit{ then } A + B + C + \;\mathrm{etc.} \equiv a+b+c+\;\mathrm{etc.} \]
\[ \textit{If  }A \equiv a, B \equiv b, \textit{ then } A - B  \equiv a-b \]

\subsection*{Art. 7} 
\[ \textit{If  }A \equiv a, \textit{ then } kA \;\mathrm{etc.} \equiv ka \]
If $k$ is a positive number, this is just a special case of the preceding proposition, in which $A=B=C$ etc and $a=b=c$ etc.  If $k$ is negative, then $-k$ is positive, and $-kA \equiv -ka$ and therefore $kA \equiv ka$.
\[ \textit{If  }A \equiv a, B \equiv b, \textit{ then } AB \equiv ab, \textrm{ since } AB \equiv Ab \equiv ab \]

\subsection*{Art. 8} Given any number of numbers $A$, $B$, $C$, etc. and as many others $a$, $b$, $c$, etc., which are congruent to each other, $A \equiv a$, $B \equiv b$, etc., the products of both will be congruent, $ABC$ etc.$\equiv abc$ etc.

From the preceding article, $AB \equiv ab$, and for the same reason $ABC \equiv abc$; and any number of factors can be treated in the same way.

If all the numbers $A$, $B$, $C$ are assumed to be equal, then so are the corresponding $a$, $b$, $c$, and we obtain the following theorem: \textit{If $A \equiv a$ and $k$ is a positive integer, then $A^k \equiv a^k$}.

\subsection*{Art. 9} 
\begin{proposition} Let $X$ be a function of a variable $x$, of the form
\[ A x^a + B x^b + C x^c + \cdots \]
where $A$, $B$, $C$ etc. are arbitrary integers and $a$, $b$, $c$ are non-negative integers.  Then if the indeterminate $x$ values are congruent according to any modulus, the resulting values of the function $X$ will also be congruent. \end{proposition}

\end{document}