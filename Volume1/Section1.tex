\documentclass{book}
\usepackage{standalone}%
\usepackage[dvips,text={6.5truein,9truein},left=1truein,top=1truein]{geometry}
\usepackage{amsmath, amsthm}%
\usepackage{titlesec}

% Uncomment to use syncing
%\usepackage{pdfsync}


% Paragraphs
\parindent=0pt
\parskip=5 pt plus 2 pt minus 1pt

\titleformat{\section}
  {\normalfont\large\bfseries\centering}{\thesection.}{1em}{}

\titleformat{\subsection}
  {\normalfont\normalsize\centering}{\thesection.}{1em}{}

\theoremstyle{plain}
\newtheorem{theorem}{Theorem}
\newtheorem{proposition}{Proposition}
\newtheorem{lemma}{Lemma}
\newtheorem{problem}{Problem}


\theoremstyle{remark}
\newtheorem*{example}{Example}
\newtheorem*{examples}{Examples}

\begin{document}

\section*{ On the Congruence of Numbers in General } 

\subsection*{ \begin{small}\textit{Congruences, moduli, residues and nonresidues} \end{small} \\ \vspace{2mm} Art. 1
} 
 If a number $a$ divides the difference between two numbers $b$ and $c$, then $b$ and $c$ are said to be congruent modulo $a$, otherwise they are said to be incongruent: the number $a$ is called the modulus of the congruence.  In the former case, the numbers $b$ and $c$ are said to be residues of each other, and in the latter case, they are said to be non-residues.

These notions apply to all integers, both positive and negative\footnote{Of course, the modulus is always absolute, i.e. it is to be taken without regard to sign}  but they are not to be extended to fractions.  

\begin{examples} $-9$ and $16$ are congruent modulo $5$; $-7$ and $15$ are residues of each other modulo $11$, but not modulo $3$. Since zero is divisible by all numbers, every number must be regarded as congruent to itself modulo any other number.  \end{examples} 


\subsection*{Art. 2} 

All of the numbers congruent to $a$ modulo $m$ are given by the formula $a+km$, where $k$ is an indeterminate integer.  The propositions which we shall give hereafter can be derived from this fact with no difficulty; but indeed their truth is equally easy to demonstrate directly.

We will denote congruences with the symbol $\equiv$, and if necessary the modulus will be in parentheses, e.g. \footnote{We have adopted this symbol due to the strong analogy between equality and congruence.   For the same reason, Legendre used the sign of equality for congruity.  We hesitated to follow that convention, lest any ambiguity should arise.}
\begin{align*} -16 &\equiv 9 \; (\mathrm{mod.} \; 5) \\
  -7 &\equiv 15  \;(\mathrm{mod.} \; 11). \end{align*}

\subsection*{Art. 3} \begin{theorem} Given an integer $A$, and $m$ consecutive integers
\[ a, a+1, a+2,\dots, a+m-1, \]
one and only one of them will be congruent to $A$ modulo $m$.
\end{theorem}
For if $\frac{a-A}{m}$ is an integer, then we have $a \equiv A$.  If it is a fraction, let the next largest integer (or the next smallest, if it is negative and sign is not taken into account) be $k$.  Then $A + km$ will be between $a$ and $a+m$, and thus it will be the number sought.  It is clear that all of the quotients $\frac{a-A}{m}$, $\frac{a+1-A}{m}$, $\frac{a+2-A}{m}$, etc. lie between $k-1$ and $k+1$, so at most one of them can be a whole number.

\subsection*{\begin{small}\textit{Minimal Residues}\end{small} \\ \vspace{2mm} Art. 4 }  It follows that each number has a residue in the series $0,1,2,\dots,m-1$, and also in the series $0,-1,-2,\dots,1-m$.  We will call these minimal residues.  Assuming both residues are not $0$, it is clear that there will be two of them, one positive and the other negative.  If the magnitudes of these are not equal, then one will be $< \frac{m}{2}$, and otherwise both will be $=\frac{m}{2}$, the sign not being considered.  From this it is clear that any residue not exceeding half of the modulus can be said to be absolutely minimal.

\begin{examples} $-13$ modulo $5$ has minimal positive residue $2$, which is absolutely minimal, and also $-3$, which is the minimal negative residue.  $5$ modulo $7$ is its own minimal positive residue, $-2$ is the negative minimal residue, which is also absolutely minimal. \end{examples}

\subsection*{ \begin{small}\textit{Elementary propositions about congruences}\end{small}  \\ \vspace{2mm} Art. 5  }  Having established these notions, let us collect those properties of congruent numbers which present themselves most immediately.

\begin{proposition} Numbers which are congruent according to a composite modulus, are also congruent according to any divisor of that modulus. \end{proposition}

\begin{proposition} If several numbers are each congruent to the same number (according to the same modulus), then they will also be congruent to each other (according to that modulus). \end{proposition}

\begin{proposition}  Congruent numbers have the same minimal residues.  Incongruent numbers have different minimal residues. \end{proposition}

\subsection*{Art. 6} Given numbers $A$, $B$, $C$, etc. and as many others $a$, $b$, $c$, etc., and any modulus whatsoever, 
\begin{proposition} If  $A \equiv a$, $B \equiv b$, $C \equiv c$ etc.,  then $A + B + C + \;\mathrm{etc.} \equiv a+b+c+\;\mathrm{etc.} $ \end{proposition}
\begin{proposition}If  $A \equiv a$ and $B \equiv b$, then $ A - B  \equiv a-b $. \end{proposition}

\subsection*{Art. 7} 
\begin{proposition} If $A \equiv a$, then $kA \equiv ka $. \end{proposition}
\begin{proof}If $k$ is a positive number, this is just a special case of the preceding proposition, in which $A=B=C$ etc and $a=b=c$ etc.  If $k$ is negative, then $-k$ is positive, so $-kA \equiv -ka$, and therefore $kA \equiv ka$.\end{proof}
\begin{proposition}If $A \equiv a$ and $B \equiv b$, then $AB \equiv ab$.\end{proposition}\begin{proof} For $AB \equiv Ab \equiv ab$. \end{proof}

\subsection*{Art. 8} 

\begin{proposition} Given any number of numbers $A$, $B$, $C$, etc. and as many others $a$, $b$, $c$, etc., which are congruent, $A \equiv a$, $B \equiv b$, etc., the products of both will be congruent, $ABC$ etc.$\equiv abc$ etc.\end{proposition}
\begin{proof} From the preceding article, $AB \equiv ab$, and for the same reason $ABC \equiv abc$; and any number of factors can be treated in the same way.\end{proof}

If all the numbers $A$, $B$, $C$ are assumed to be equal, then so are the corresponding $a$, $b$, $c$, and we obtain the following theorem: 
\begin{theorem} If $A \equiv a$ and $k$ is a positive integer, then $A^k \equiv a^k$ \end{theorem}

\subsection*{Art. 9} 
\begin{proposition} Let $X$ be an algebraic function of a variable $x$, of the form
\[ A x^a + B x^b + C x^c + \cdots \]
where $A$, $B$, $C$ etc. are arbitrary integers and $a$, $b$, $c$ are non-negative integers.  Then if the indeterminate $x$ values are congruent according to any modulus, the resulting values of the function $X$ will also be congruent. \end{proposition}
\begin{proof} Let $f$, $g$ be congruent values of $x$.  Then from the preceding article $f^a \equiv g^a$ and $Af^a \equiv Ag^a$, and in the same way $Bf^b \equiv Bg^b$ etc.  Thus 
\[ Af^a + Bf^b + Cf^c + \textit{etc}. \equiv Ag^a + Bg^b + Cg^c + \textit{etc}. \]
\end{proof}
It is easy to understand how this theorem can be extended to functions of several variables.

\subsection*{Art. 10}

If, therefore, all consecutive integers are substituted for $x$, and the values of the function $X$ are reduced to their minimum residues, these will constitute a series in which, after an interval of $m$ terms (where $m$ denotes the modulus), the same terms recur again.  That is, the series will be formed from a period of $m$ terms, repeated ad infinitum. 

\begin{example} Let  $X = x^3 - 8x + 6$ and $m=5$.   Then for $x = 0,1,2,3$ etc., the values of $X$ have minimal positive residues $1,4,3,4,3,1,4$ etc., where the first five terms $1,4,3,4,3$ are repeated ad infinitum; and if the series is continued backwards, i.e. $x$ is given negative values, then the same period is produced with an inverted order of terms.  From this it is clear that terms other than those which constitute the period cannot occur anywhere in the series. \end{example}

\subsection*{Art. 11}

In the above example, therefore, $X$ can become neither $\equiv 0$ nor $\equiv 2$, much less $=0$ or $=2$. Hence it follows  that the equations $x^3 -8x + 6 = 0$ and $x^3 - 8x + 4 =2$ cannot be solved by integers, and therefore, as is well known, they cannot be solved by rational numbers.   It is generally clear that the equation $X=0$, when $X$ is a function of the variable $x$, of the form
\[ x^n + Ax^{n-1} + Bx^{n-2} + \textrm{etc.} + N \]
with $A$, $B$, $C$ etc. integers and $n$ a positive integer (to which form it is clear that any algebraic equation can be reduced) can have no rational root, if there is a modulus for which the congruence $X\equiv0$ is impossible.  This criterion, which spontaneously presented itself to us here, will be discussed again in greater detail in Section VIII.  Certainly from this example one obtains a sense of the utility of these investigations.

\subsection*{\begin{small} \textit{Some applications} \end{small} \\ \vspace{2mm} Art. 12  }

Several of the theorems presented in this chapter generalize those usually taught in arithmetic, e.g. the rules for exploring the divisibility of a given number by $9$, $11$, or other numbers.  According to the modulus $9$, all powers of $10$ are congruent to unity: therefore, if the proposed number has the form $a + 10b + 100c +$ etc., then its minimal residue modulo $9$ will be the same as that of $a+b+c+$ etc.  From this it is clear that if the decimal digits of any number are added without regard to the place they occupy, the sum will give the same minimal residue as the original number, so that the latter will be divisible by $9$ if it the former is divisible by 9, and vice versa.  The same also applies to divisibility by $3$.  For the modulus $11$, one has $100 \equiv 1$, so in general $10^{2k} \equiv 1$, $10^{2k+1} \equiv 10 \equiv -1$, and a number of the form $a + 10b + 100c + $ etc. will have the same minimum residue as $a-b+c$ etc.; from which the well-known rule is directly derived.  From the same principle all similar precepts are easily deduced.

Another application of the foregoing is the system of rules which are generally recommended for the verification of arithmetical operations.   If from some given numbers others are to be found by addition, subtraction, multiplication or exponentiation, then instead of the given numbers their minimal residues according to an arbitrary modulus can be substituted (by arbitrary, I really mean 9 or 11, since in our decimal system the remainders according to these moduli can be found so easily).  The numbers derived from this must be congruent with those which were deduced from the proposed numbers; unless this happens, it is concluded that a defect has crept into the calculation.

But since these and the like are abundantly well known, it would be superfluous to dwell on them any longer.


\end{document}