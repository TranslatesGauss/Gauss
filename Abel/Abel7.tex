\documentclass[12pt]{article}
\usepackage{graphicx}
\usepackage{wrapfig}
\usepackage{geometry}
\usepackage{amsmath}
\geometry{legalpaper, margin=1in}
\usepackage{color}
\usepackage{amssymb}
\usepackage{enumitem}
\parindent=0pt
\parskip=8pt
\arraycolsep=2pt
\usepackage{pgfplots}


\begin{document}

\begin{center}
DEMONSTRATION OF THE IMPOSSIBILITY OF AN ALGEBRAIC SOLUTION OF GENERAL EQUATIONS OF DEGREES GREATER THAN FOUR.
\end{center}

It is well-known that one can solve general equations of degree at most four, and equations of higher degree can be solved in some particular cases, but unless I am mistaken, no one has yet addressed the following question in a satisfactory manner: ``Is it possible to solve, in general, equations of degree greater than four?''.  The goal of this memoire will be to answer this question. 

Solving an equation algebraically means nothing more than to express its roots as algebraic functions of the coefficients.  Therefore, we must first consider the general form of an algebraic function, and then determine whether or not a given equation can be satisfied by a function of this form.  

\begin{center} 
\textbf{I.}

\textit{On the general form of algebraic functions.}
\end{center}
Let $x'$, $x''$, $x'''$, \dots be a finite number of unknown quantities.  We say that $v$ is an \emph{algebraic} function of these quantities if it is possible to express $v$ in terms of $x'$, $x''$, $x'''$, \dots and other quantities which do not depend on any of these, using only the following operations: 1) addition, 2) multiplication, 3) division, and 4) extraction of roots of prime order.  Here we have not included subtraction, raising to integer powers, or extraction of roots of composite order, but these are evidently combinations of the four operations we have mentioned.

When a function $v$ can be formed using only the first three operations, it is said to be \emph{algebraic and rational}, or simply \emph{rational}; and if only the first two operations are required, it is said to be \emph{algebraic, rational, and entire}, or simply \emph{polynomial}\footnote{We will always translate \emph{fonction enti\`ere} as \emph{polynomial}, in order to avoid confusion with the modern notion of an entire function.}.

Let $f(x',x'',x''',\dots)$ denote a general function which can be expressed as a sum of a finite number of terms
\[ Ax'^{m_1} x''^{m_2} \cdots, \]
where $A$ is a quantity which does not depend on $x'$, $x''$, etc., and $m_1$, $m_2$, etc. are positive integers; it is clear that operations 1) and 2) are particular cases of the function $f(x',x'',x''',\dots)$.  Therefore, a general polynomial function can be understood as the result of a limited number of repetitions of this operation.  If $v'$, $v''$, $v'''$ etc. denote functions of $x'$, $x''$, $x'''$ \dots which have the same general form as $f(x',x'',\dots)$, then the function $f(v',v'' \dots)$ will evidently have the same general form as $f(x',x'',\dots)$.  Now $f(v',v'',\dots)$ is the general form of a function obtained from two repetitions of the operation $f(x',x'', \dots)$.  We find, therefore, that the result of any number of repetitions will always take the same form.  It follows that any polynomial function of $x', x'', \dots$ can be written as a sum of terms of the form $Ax'^{m_1}x''^{m_2} \cdots$. 

Next we consider the rational functions.  If $f(x',x'',\dots)$ and $\phi(x',x'',\dots)$ are general polynomial functions, it is clear that operations 1), 2), and 3) are particular cases of the operation
\[ \frac{f(x',x'' \dots)}{\phi(x',x'',\dots)} . \]
Therefore, a rational function can be understood as the result of repeating this operation.  If $v'$, $v''$, $v'''$
 are functions of the general form $ \frac{f(x',x'' \dots)}{\phi(x',x'',\dots)}$, then it is easy to see that any function $ \frac{f(v',v'' \dots)}{\phi(v',v'',\dots)}$ can be reduced to the same general form.  It follows that any rational function of $x',x'',\dots$ can always be reduced to the form 
 \[  \frac{f(x',x'' \dots)}{\phi(x',x'',\dots)} \]
 where the numerator and denominator are polynomial functions.  
 
 Finally, we seek the general form of an algebraic function.  If $f(x',x'',\dots)$ denotes a general rational function, it is clear that any algebraic function can be constructed with the help of the general operation $f(x',x'',\dots)$, combined with the operation $\sqrt[m]{r}$, where $m$ is a prime number.  Therefore, if $p'$, $p''$, \dots are rational functions of $x'$, $x''$, \dots,
 \[ p_1 = f(x',x'',\dots,\sqrt[n']{p'},\sqrt[n'']{p''}) \]
is the general form of an algebraic function in which the operation $\sqrt[m]{r}$ is only applied to rational functions.  Functions of the form $p_1$ will be called algebraic functions \emph{of the first order}.  Denoting by $p_1',p_1'',\dots$ any number of quantities of the form $p_1$, the expression 
 \[ p_2 = f(x',x'',\dots \sqrt[n']{p'},\sqrt[n'']{p''} \dots \sqrt[n_1']{p_1'},\sqrt[n_1'']{p_1''} \dots) \]
is the general form of an algebraic function of $x'$, $x''$,\dots, in which the operation $\sqrt[m]{r}$ is only applied to algebraic functions of the first order.  Functions of the form $p_2$ will be called algebraic functions $\emph{of the second order}$.  In the same manner, the expression
\[ p_3 = f(x',x'', \dots \sqrt[n']{p'},\sqrt[n'']{p''} \dots \sqrt[n_1']{p_1'}, \sqrt[n_1'']{p_1''}, \dots \sqrt[n_2']{p_2'}, \sqrt[n_2'']{p_2''} \dots), \]
in which $p_2'$, $p_2''$ \dots are functions of the second order, will be the general form of an algebraic function of $x'$, $x''$, \dots, in which the operation $\sqrt[m]{r}$ is only applied to algebraic functions of the first and second order. 

Continuing in this manner, one obtains algebraic functions of the third, fourth, \dots, $\mu^{th}$ order, and it is clear that the expression for functions of the $\mu^{th}$ order is the \emph{general} form of an algebraic function. 

Therefore let $v$ denote an algebraic function of order $\mu$, 
\[ v = f(r',r'',\dots \sqrt[n']{p'},\sqrt[n'']{p''} \dots), \]
where $p'$, $p''$, \dots are functions of order $\mu -1$; $r'$,$r''$,\dots functions of order less than $\mu-1$, and $n'$, $n''$,\dots are prime numbers; $f$ denoting, as always, a rational function of the quantities inside the parentheses. 

One may evidently assume that it is impossible to express the quantities $\sqrt[n']{p'}$, $\sqrt[n'']{p''}$,\dots by a rational function of the quantities $r'$, $r''$, \dots ; because otherwise, the function $v$ would have the simpler form,
\[ v = f(r',r'',\dots \sqrt[n']{p'},\sqrt[n'']{p''}\dots), \]
where the number of quantities $\sqrt[n']{p'}$, $\sqrt[n'']{p''}$ has been decreased by one.  Reducing as much as possible in this way the expression for $v$, one either arrives at an irreducible expression, or an expression of the form 
\[ v = f(r',r'',r'''\dots); \]
but this function has order $\mu - 1$, contradicting the assumption that $v$ has order $\mu$. 

If in the expression for $v$ the number of quantities $\sqrt[n']{p'},\sqrt[n'']{p''},\dots$ is equal to $m$, we say that the function $v$  is of the $\mu^{th}$ order and the $m^{th}$ degree.  Thus a function of order $\mu$ and degree $0$ is the same thing as a function of order $\mu - 1$, and a function of order $0$ is the same thing as a rational function. 

It follows from this, that one may take
\[ v = f(r',r'',\dots,\sqrt[n]{p}), \]
where $p$ is a function of order $\mu -1$, but $r'$,$r''$ are functions of order $\mu$ and degree at most $m-1$, and that one may always suppose that it is impossible to express $\sqrt[n]{p}$ as a rational function of these quantities. 

In the above discussion, we have seen that a rational function of any number of quantities may always be reduced to the form 
\[ \frac{s}{t}, \]
where $s$ and $t$ are polynomial functions of the same quantities.  We conclude that $v$ can always be expressed as 
\[ v = \frac{\phi(r',r'',\dots,\sqrt[n]{p})}{\tau(r',r'',\dots,\sqrt[n]{p})}, \]
where $\phi$ and $\tau$ are polynomial functions of $r', r'', \dots$ and $\sqrt[n]{p}$.  By virtue of what we have already seen, any polynomial function of the quantities $s$, $r'$, $r''$, \dots can be written in the form
\[ t_0 + t_1s + t_2 s^2 + \dots + t_m s^m, \]
where $t_0,t_1,\dots, t_m$ are polynomial functions of $r'$, $r''$, \dots but not $s$.  We can therefore write
\[ v = \frac{t_0 + t_1 p^{\frac{1}{n}} + t_2p^{\frac{2}{n}} + \cdots + t_m p^{\frac{m}{n}}}{v_0 + v_1 p^{\frac{1}{n}} + v_2p^{\frac{2}{n}} + \cdots + v_m p^{\frac{m}{n}}} = \frac{T}{V} \]
where $t_0,t_1,\dots, t_m$ and $v_0,v_1,\dots, v_m$ are polynomial functions of $r'$,$r''$,$r'''$ etc.

Let $V_1$,$V_2$,\dots $V_{n-1}$ denote the $n-1$ values of $V$ which are found by substituting in turn $\alpha p^{\frac{1}{n}}$, $\alpha^2 p^{\frac{1}{n}}$, $\alpha^3 p^{\frac{1}{n}}$, \dots, $\alpha^{n-1} p^{\frac{1}{n}}$ for $p^{\frac{1}{n}}$, where $\alpha$ is a root of $\alpha^{n} - 1$ which is not equal to $1$.  Multiplying the numberator and denominator of $\frac{T}{V}$ by $V_1V_2V_3 \cdots V_{n-1}$, one finds that
\[ v = \frac{TV_1V_2 \cdots V_{n-1}}{VV_1V_2 \cdots V_{n-1}}. \]
As one knows, the product $VV_1V_2 \cdots V_{n-1}$ can be expressed as a polynomial function of $p$ and the quantities $r'$, $r''$ \dots, and as one can see, the product $TV_1V_2\cdots V_{n-1}$ is a polynomial function of $\sqrt[n]{p}$ and $r'$,$r''$,\dots .  If this product is equal to 
\[  s_0 + s_1p^{\frac{1}{n}} + s_2 p^{\frac{2}{n}} + \cdots s_k p^{\frac{k}{n}}, \]
one finds that 
\[ v = \frac{s_0 + s_1p^{\frac{1}{n}} + s_2 p^{\frac{2}{n}} + \cdots s_k p^{\frac{k}{n}}}{m}, \]
or, writing $q_0$, $q_1$, $q_2$,\dots in place of $\frac{s_0}{m}$, $\frac{s_1}{m}$, $\frac{s_2}{m}$ etc.,
\[ v = q_0 + q_1p^{\frac{1}{n}} + q_2p^{\frac{2}{n}} + \cdots + q_kp^{\frac{k}{n}},\]
where $q_0$, $q_1$, $q_2$ are  all rational functions of $p$, $r'$, $r''$ \dots etc.

If $\mu$ is any whole number, we may always write 
\[ \mu = an + \alpha, \]
where $a$ and $\alpha$ are whole numbers with $\alpha < n$.  It follows that
\[ p^{\frac{\mu}{n}} = p^{\frac{an + \alpha}{n} }= p^ap^{\frac{\alpha}{n}}. \]
Putting this in place of $p^{\frac{\mu}{n}}$ in the expression for $v$, we obtain
\[ v = q_0 + q_1p^{\frac{1}{n}} + q_2p^{\frac{2}{n}} + \cdots + q_{n-1}p^{\frac{n-1}{n}},\]
where $q_0$, $q_1$, $q_2$ are all rational functions of $p$, $r'$, $r''$ ..., and consequently are functions of order $\mu$ and degree at most $m-1$, such that it is impossible to express $p^{\frac{1}{n}}$ as a rational function of these quantities.

In the expression for $v$ above, we may always assume that $ q_1 = 1$.  Indeed, if $q_1$ is nonzero, then by setting $p_1 = p q_1^n$, one obtains
\[ p = \frac{p_1}{q_1^n} \; , \; p^{\frac{1}{n}} = \frac{ p_1^{\frac{1}{n}} }{q_1}, \]
so 
\[ v = q_0 + p_1^{ \frac{1}{n} } + \frac{q_2}{q_1^2} p_1^{ \frac{2}{n} } + \cdots + \frac{ q_{n-1}}{q_1^{n-1}} p_1^{ \frac{n-1}{n} } ,\]
an expression of the same form, but with $q_1 = 1$.  If instead $q_1 = 0$, let $q_\mu$ be one of the quantities $q_1,\dots,q_{n-1}$ which is nonzero, and let $q_{\mu}^n p^{\mu} = p_1$, so that $q_{\mu}^{\alpha} p^{\frac{\alpha \mu}{n}} = p_1^{\frac{\alpha}{n}}$.  Choosing whole numbers $\alpha$, $\beta$, and $\mu'$, which satisfy the equation $\alpha \mu - \beta n = \mu'$,  we have
\[ q_{\mu}^{\alpha} p^{\frac{\beta n + \mu'}{n}} = p_1^{\frac{\alpha}{n}} \mathrm{\; and \;} p^{\frac{\mu'}{n}} = q_{\mu}^{-\alpha}p^{-\beta}p_1^{\frac{\alpha}{n}} .\]
By virtue of this, and remarking that $q_{\mu} p^{\frac{\mu}{n}} = p^{\frac{1}{n}}$, we see that $v$ has the form
\[ v = q_0 + p_1^{\frac{1}{n}} + q_2 p_1^{\frac{2}{n}} + \cdots + q_{n-1}p_1^{\frac{n-1}{n}}. \]

From all of this we conclude: If $v$ is an algebraic function of order $\mu$ and degree $m$, we can always write
\[ v = q_0 + p^{\frac{1}{n}} + q_2p^{\frac{2}{n}} + q_3p^{\frac{3}{n}} + \cdots + q_{n-1} p^{\frac{n-1}{n}}, \]
where $n$ is a prime number, $q_0$, $q_2$,\dots,$q_{n-1}$ are algebraic functions of order $\mu$ and degree at most $m-1$, and $p$ is an algebraic function of order $\mu-1$, such that $p^{\frac{1}{n}}$ cannot be expressed as a rational function of $q_0$, $q_1$, \dots, $q_{n-1}$. 
\begin{center} 
\textbf{II.}

\textit{Properties of algebraic functions which satisfy a given equation.}
\end{center}
 
Let 
\begin{equation} \label{eqn1} c_0 + c_1 y + c_2 y^2 + \cdots + c_{r-1} y^{r-1} + y^r = 0 \end{equation}
 be an equation of degree $r$, where $c_0$, $c_1$,\dots are rational functions of the independent quantities $x'$,$x''$,\dots .  Suppose that this equation is satisfied when we replace $y$ with an algebraic function of $x'$, $x''$, \dots ,
\begin{equation} \label{eqn2} y = q_0 + p^{\frac{1}{n}} + q_2 p^{\frac{2}{n}} + \cdots + q_{n-1} p^{\frac{n-1}{n}} .
\end{equation} 
After substituting this expression for $y$, the proposed equation becomes, by virtue of our previous discussion, an expression of the form 
\begin{equation} \label{eqn3} r_0 + r_1 p^{\frac{1}{n}} + r_2 p^{\frac{2}{n}} + \cdots + r_{n-1} p^{\frac{n-1}{n}} = 0, \end{equation}
 where $r_0$, $r_1$, $r_2$ \dots $r_{n-1}$ are rational functions of the quantities $p$, $q_0$, $q_1$,\dots,$q_{,n-1}$. 
 
 Now I claim that equation (\ref{eqn3}) cannot take place, unless we have separately
 \[ r_0 = 0, r_1 = 0 ,\dots, r_{n-1} = 0 .\]
 Indeed, supposing otherwise, if we set $z = p^{\frac{1}{n}}$, then the two equations
 \[ z^n - p = 0 \]
 and
 \[ r_0 + r_1 z + r_2z^2 + \cdots + r_{n-1} z^n = 0,\]
 will have one or more \emph{common roots}.  Letting $k$ be the number of these roots, it is well-known that we can find an equation which has precisely these $k$ roots, and whose coefficients are rational functions of $p$, $r_0$, $r_1$, \dots, $r_{n-1}$.  Let
 \[ s_0 + s_1 z + s_2 z^2 + \cdots + s_{k-1} z^{k-1} + z^k = 0 \]
 be this equation, and let
 \[ t_0 + t_1 z + t_2 z^2 + \cdots + t_{\mu-1} z^{\mu-1} + z^\mu \]
be any factor of its left hand side, where $t_0$, $t_1$, etc. are rational functions of $p$, $r_0$, $r_1$,\dots,$r_{n-1}$,  such that 
 \[ t_0 + t_1 z + t_2 z^2 + \cdots + t_{\mu-1} z^{\mu-1} + z^\mu = 0 .\]
We may clearly assume that it is impossible to find any equation of the same form with a smaller degree.  This equation shares all of its $\mu$ roots with $z^n - p = 0 $.  But any root of the equation $z^n-p = 0$ takes the form $\alpha z$, where $\alpha$ is a[n $n^{th}$] root of unity.  Therefore, observing that $\mu$ cannot be less than $2$, because it is impossible to express $z$ as a rational function of the quantities $p$, $r_0$, $r_1$, \dots, $r_{n-1}$, it follows that two equations must hold: 
\[ t_0 + t_1 z + t_2 z^2 + \cdots + t_{\mu-1} z^{\mu-1} + z^\mu = 0 ,\]
and 
\[ t_0 + t_1 \alpha z + t_2 \alpha^2 z^2 + \cdots + t_{\mu-1} \alpha^{\mu-1}z^{\mu-1} + \alpha^{\mu}z^\mu = 0 .\]
From these equations we get, after eliminating $z^{\mu}$, 
\[ t_0 (1-\alpha^{\mu}) + t_1(\alpha-\alpha^{\mu})z + \cdots + t_{\mu-1} (\alpha^{\mu-1}-\alpha^{\mu}) z^{\mu-1} = 0. \]
But this equation has degree $\mu - 1$, and the equation
\[ z^{\mu} + t_{\mu-1} z^{\mu-1} + \cdots = 0 \]
is irreducible, and consequently $t_0$ cannot be equal to zero, or we would have $\alpha^{\mu} - 1 = 0$, which cannot be.  Therefore, 
\[ r_0 = 0 , \; r_1 = 0 , \cdots , r_{n-1} = 0. \] 

Now, because of all this, it is clear that the proposed equation must be satisfied by all of the values of $y$ obtained by assigning to $p^{\frac{1}{n}}$ all of the values $\alpha p^{\frac{1}{n}}, \alpha^2 p^{\frac{2}{n}},\cdots \alpha^{n-1} p^{\frac{1}{n}}$. One easily sees that all of these values of $y$ are different from one another, because otherwise there would be an equation of the form (\ref{eqn3}), and we have just seen that such an equation leads to a contradiction. 

Denoting the $n$ distinct roots of equation (\ref{eqn1}) by $y_1$, $y_2$, \dots, $y_n$, we have 
\begin{eqnarray*} y_1 &=& q_0 + p^{\frac{1}{n}} + q_2 p^{\frac{2}{n}} + \cdots + q_{n-1} p^{\frac{n-1}{n}}, \\ 
y_2 &=& q_0 + \alpha p^{\frac{1}{n}} + \alpha^2 q_2 p^{\frac{2}{n}} + \cdots + \alpha^{n-1} q_{n-1} p^{\frac{n-1}{n}}, \\
 && \cdots \cdots \cdots \cdots \cdots \\
 y_n &=& q_0 + \alpha^{n-1} p^{\frac{1}{n}} + \alpha^{n-2} q_2 p^{\frac{2}{n}} + \cdots + \alpha q_{n-1} p^{\frac{n-1}{n}}. \end{eqnarray*} Out of these $n$ equations one can easily extract
\begin{eqnarray*} q_0 &=& \frac{1}{n}\left(y_1 + y_2 + y_3 + \cdots + y_n \right) \\
p^{\frac{1}{n}} &=& \frac{1}{n}\left(y_1 + \alpha^{n-1}y_2 + \alpha^{n-2} y_3 + \cdots + \alpha y_n \right) \\
q_2 p^{\frac{2}{n}} &=& \frac{1}{n}\left(y_1 + \alpha^{n-2} y_2 + \alpha^{n-4} y_3 + \cdots + \alpha^2 y_n \right) \\
 && \cdots \cdots \cdots \cdots \cdots \\
q_{n-1} p^{\frac{n-1}{n}} &=& \frac{1}{n}\left(y_1 + \alpha y_2 + \alpha^2 y_3 + \cdots + \alpha^{n-1} y_n \right). \end{eqnarray*}
From this one sees that the quantities $p^{\frac{1}{n}}$, $q_0$, $q_2$, \dots, $q_{n-1}$ are rational functions of the roots of equation (\ref{eqn1}).  Indeed, one has
\[ q_{\mu} = n^{\mu -1 } \frac{ y_1 + \alpha^{-\mu}y_2 + \alpha^{-2\mu} y_3 + \cdots + \alpha^{-(n-1)\mu} y_n }{\left(y_1 + \alpha^{-1}y_2 + \alpha^{-2} y_3 + \cdots + \alpha^{-(n-1)} y_n \right)^{\mu}} .\]

Now consider the general equation of degree $m$,
\[ 0 = a + a_1x + a_2x^2 + \cdots + a_{m-1}x^{m-1} + x^m, \]
and suppose that it can be solved algebraically.  Let 
\[ x = s_0 + v^{\frac{1}{n}} + s_2 v^{\frac{2}{n}} + \cdots + s_{n-1} v^{\frac{n-1}{n}} ; \]
by virtue of the discussion above, the quantities $v$, $s_0$, $s_2$ etc. can be expressed as rational functions of $x_1$, $x_2$,\dots,$x_m$, the roots of the proposed equation.

Consider one of the quantities $v$, $s_0$, $s_2$,\dots,  for example $v$.  If we denote by $v_1$, $v_2$, \dots, $v_{n'}$ the different values of $v$ which are obtained by interchanging the roots $x_1$, $x_2$, \dots, $x_m$ in all possible ways, we can form an equation of degree $n'$ whose coefficients are rational functions of $a$, $a_1$,\dots,$a_{m-1}$, and whose roots are the quantities $v_1$,$v_2$,\dots,$v_{n'}$, which are rational functions of the quantities $x_1$, $x_2$, \dots, $x_m$. 

Therefore, if we write 
\[ v = t_0 + u^{\frac{1}{\nu}} + t_2 u^{\frac{2}{\nu}} + \cdots + t_{\nu -1} u^{\frac{\nu - 1}{\nu}}, \]
then all of the quantities $u^{\frac{1}{\nu}}$, $t_0$, $t_2$,\dots,$t_{\nu -1}$ will be rational functions of $v_1$, $v_2$, \dots, $v_{n'}$, and consequently $x_1$, $x_2$,\dots $x_m$. Treating the quantities $u$, $t_0$, $t_2$ etc. in a similar manner, we conclude that 
\begin{quote}
If an equation can be solved algebraically, then any one of its roots can be written in such a way that the algebraic functions of which it is composed can all be expressed as rational functions of the roots of the equation.
\end{quote}
\begin{center} 
\textbf{III.}

\textit{On the number of different values that a function of several quantities may acquire, when the quantities of which it is composed are exchanged amongst themselves.}
\end{center}

Let $v$ be a rational function of several independent quantities $x_1$, $x_2$, \dots, $x_n$.   The number of different values that this function takes as we exchange the quantities on which it depends can never be greater than the product $1 \cdot 2 \cdot 3 \cdots n$.  Let $\mu$ denote this product.

Let
\[ v \left( \begin{array}{ccccc} \alpha & \beta & \gamma & \delta & \cdots \\
a & b & c & d & \cdots \end{array} \right ) \]
denote the value taken by $v$ when we substitute $x_a$, $x_b$, $x_c$, $x_d$, etc. for $x_{\alpha}$, $x_{\beta}$, $x_{\delta}$, etc.  If we denote by $A_1,A_2,\dots, A_\mu$ the $\mu$ different permutations of the indices $1, 2,3, \dots, n$, then it is clear that the different values of $v$ are 
\[ v \left( \begin{array}{c} A_1 \\ A_1 \end{array} \right)
v \left( \begin{array}{c} A_1 \\ A_2 \end{array} \right)
v \left( \begin{array}{c} A_1 \\ A_3 \end{array} \right)
\cdots
v \left( \begin{array}{c} A_1 \\ A_\mu \end{array} \right) \]
Suppose that the number of different values of $v$ is less than $\mu$.  Then several of these values will be equal to $v$, for example
\[ v \left( \begin{array}{c} A_1 \\ A_1 \end{array} \right) = 
v \left( \begin{array}{c} A_1 \\ A_2 \end{array} \right) = \cdots = 
v \left( \begin{array}{c} A_1 \\ A_m \end{array} \right). \]
If we apply the substitution $v \left( \begin{array}{c} A_1 \\ A_{m+1} \end{array} \right)$ to each of these quantities, we get a new series of equal values 
\[ v \left( \begin{array}{c} A_1 \\ A_{m+1} \end{array} \right) = 
v \left( \begin{array}{c} A_1 \\ A_{m+2} \end{array} \right) = \cdots = 
v \left( \begin{array}{c} A_1 \\ A_{2m} \end{array} \right), \] 
which are different from the first, but equal in number.  If we instead apply the substitution $v \left( \begin{array}{c} A_1 \\ A_{2m+1} \end{array} \right)$, we get yet a third series of equal values, which are equal and different from the first two.  Continuing this process until we have exhausted all possible permutations, the $\mu$ values of $v$ will be partitioned into several systems, each of which contains $m$ equal values.  It follows that if $\rho$ represents the number of different values of $v$, this number being equal to the number of systems, we have
\[ \phi m = 1 \cdot 2 \cdot 3 \cdots n, \]
that is,

The number of different values that a function of $n$ quantities acquires, through all possible substitutions of these quantities, is necessarily a divisor of the product $1 \cdot 2 \cdot 3 \cdots n$.  

Now let $\left( \begin{array}{c} A_1 \\ A_m \end{array} \right)$ be an arbitrary substitution.  If we apply it several times in succession to the function $v$, we obtain a series of values
\[ v, v_1, v_2, \dots v_{p-1},v_p, \dots \]
and it is clear that $v$ necessarily must be repeated several times.  When $v$ reappears after $p$ substitutions, we say that $\left( \begin{array}{c} A_1 \\ A_m \end{array} \right)$ is a \emph{recurrent substitution of order p}.  We therefore have a periodic series,
\[ v, v_1, v_2, \dots, v_{p-1}, v, v_1, v_2, \dots \]
or, if we write $v \left( \begin{array}{c} A_1 \\ A_m \end{array} \right)^r$ for the value of $v$ obtained by applying the substitution $\left( \begin{array}{c} A_1 \\ A_m \end{array} \right)$ a total of $r$ times in succession, we have the series
\[ v \left( \begin{array}{c} A_1 \\ A_m \end{array} \right)^0,
v \left( \begin{array}{c} A_1 \\ A_m \end{array} \right)^1,
v \left( \begin{array}{c} A_1 \\ A_m \end{array} \right)^2,
\cdots,
v \left( \begin{array}{c} A_1 \\ A_m \end{array} \right)^{p-1},
v \left( \begin{array}{c} A_1 \\ A_m \end{array} \right)^0, \cdots \]
It follows from this that 
\[ \begin{array}{lcl} v \left( \begin{array}{c} A_1 \\ A_m \end{array} \right)^{\alpha p + r} &=& v \left( \begin{array}{c} A_1 \\ A_m \end{array} \right)^r,\\ \\
 v \left( \begin{array}{c} A_1 \\ A_m \end{array} \right)^{\alpha p} &=& v \left( \begin{array}{c} A_1 \\ A_m \end{array} \right)^0 = v. \end{array} \]
 Now suppose $p$ be the largest prime number which is less than $n$.  If the number of different values of $v$ is less than $p$, it follows that among the following $p$ values,
\[ v \left( \begin{array}{c} A_1 \\ A_m \end{array} \right)^0,
v \left( \begin{array}{c} A_1 \\ A_m \end{array} \right)^1,
v \left( \begin{array}{c} A_1 \\ A_m \end{array} \right)^2,
\cdots,
v \left( \begin{array}{c} A_1 \\ A_m \end{array} \right)^{p-1},\]
two of them must be equal to each other.  Suppose, for example, that
\[ v \left( \begin{array}{c} A_1 \\ A_m \end{array} \right)^r = v \left( \begin{array}{c} A_1 \\ A_m \end{array} \right)^{r'}. \]
from which one concludes that 
\[ v \left( \begin{array}{c} A_1 \\ A_m \end{array} \right)^{r+p-r} = v \left( \begin{array}{c} A_1 \\ A_m \end{array} \right)^{r'+p-r}. \]
Writing $r$ in place of $r' + p - r$, and remarking that $ v \left( \begin{array}{c} A_1 \\ A_m \end{array} \right)^p = v$, we have 
\[ v = v \left( \begin{array}{c} A_1 \\ A_m \end{array} \right)^r, \]
where $r$ is evidently not a multiple of $p$.  Therefore, the value of $v$ is neither changed by the substitution
$\left( \begin{array}{c} A_1 \\ A_m \end{array} \right)^r$, nor consequently by any number of repetitions of the same substitution.  We therefore have 
\[ v = v \left( \begin{array}{c} A_1 \\ A_m \end{array} \right)^{r\alpha}, \]
for any whole number $\alpha$.  But since $p$ is a prime number, we can always find two whole numbers $\alpha$ and $\beta$ such that 
\[ r \alpha = p \beta + 1, \]
therefore 
 \[ v = v \left( \begin{array}{c} A_1 \\ A_m \end{array} \right)^{p\beta +1 }, \]
and since 
\[ v = v \left( \begin{array}{c} A_1 \\ A_m \end{array} \right)^{p\beta}, \]
we have 
\[ v = v \left( \begin{array}{c} A_1 \\ A_m \end{array} \right), \]
Therefore, the value of $v$ is unchanged by any recurrent subsitution $\left( \begin{array}{c} A_1 \\ A_m \end{array} \right)$ of order $p$.

Now, it is clear that both 
\[  \left( \begin{array}{ccccccc} \alpha & \beta & \gamma & \delta & \cdots &\zeta &\eta\\
\beta & \gamma & \delta & \epsilon & \cdots &\eta & \alpha \end{array} \right ) \mathrm{\; and \;} \left( \begin{array}{ccccccc} \beta & \gamma & \delta & \epsilon & \cdots &\eta &\alpha\\
\gamma & \alpha & \beta & \delta & \cdots &\zeta & \eta \end{array} \right ) \]
are recurrent substitutions of order $p$, where $p$ is the number of indices $\alpha, \beta,\gamma,\dots,\eta$. The value of $v$ is therefore unchanged by the combination of these two substitutions.  These two substitutions are evidently equivalent to\footnote{Abel is awkwardly trying to tell us that $\left( \begin{array}{ccc} \alpha & \beta & \gamma \\
\beta & \gamma & \alpha \end{array} \right )$ is the \emph{composition} of two permutations of order $p$.} 
\[ \left( \begin{array}{ccc} \alpha & \beta & \gamma \\
\beta & \gamma & \alpha \end{array} \right ), \]
which is equivalent to the two following substitutions, applied successively:
\[ \left( \begin{array}{cc} \alpha & \beta  \\
\beta &  \alpha \end{array} \right ) \mathrm{ \; and \; } \left( \begin{array}{cc} \beta & \gamma  \\
\gamma &  \beta \end{array} \right ).\]
Therefore,
\[ v = v  \left( \begin{array}{cc} \alpha & \beta  \\
\beta &  \alpha \end{array} \right ) \left( \begin{array}{cc} \beta & \gamma  \\
\gamma &  \beta \end{array} \right ), \]
and likewise
\[ v = v  \left( \begin{array}{cc} \beta & \gamma  \\
\gamma &  \beta \end{array} \right ) \left( \begin{array}{cc} \gamma & \delta  \\
\delta &  \gamma \end{array} \right ), \]
from which we get 
\[ v = v \left( \begin{array}{cc} \alpha & \beta  \\
\beta&  \alpha \end{array} \right ) \left( \begin{array}{cc} \gamma & \delta  \\
\delta &  \gamma \end{array} \right ) \]

From this we see that the function $v$ is unchanged by two successive substitutions of the form $\left( \begin{array}{cc} \alpha & \beta  \\
\beta &  \alpha \end{array} \right )$, where $\alpha$ and $\beta$ are arbitrary indices.  Referring to such a substitution as a \emph{transposition}, we can conclude that any value of $v$ will remain unchanged by an even number of transpositions, and consequently any two values of $v$ which result from an odd number of transpositions are equal.  Any exchange of the variables of a function can be carried out with the help of a certain number of transpositions; therefore the function $v$ cannot have more than two different values.  From this we deduce the following theorem:
\begin{quote}
The number of different values that a function of $n$ quantities can obtain, cannot be smaller than the largest prime number less than $n$, unless it is $2$ or $1$.  
\end{quote}
It is therefore impossible to find a function of $5$ quantities which takes on $3$ or $4$ different values. 

The proof of this theorem is taken from a memoir by \emph{Cauchy} in the Journal de l'�cole polytechnique, vol. 17.

Let $v$ and $v'$ be two functions, each of which has two different values.  Denoting these values by $v_1, v_2$ and $v_1', v_2'$, it follows from the above that the expressions
\[ v_1 + v_2 \mathrm{\; and \;} v_1 v_1' + v_2 v_2' \]
are symmetric functions.  Setting
\[ v_1 + v_2 = t \mathrm{\; and \; } v_1v_1' + v_2 v_2' = t_1, \]
we have 
\[ v_1 = \frac{tv_2'-t_1}{v_2' - v_1'}. \]
When the number of quantities $x_1, x_2, \dots, x_m$ is equal to five, the product
\[ \rho = (x_1-x_2)(x_1-x_3)(x_1-x_4)(x_1-x_5)(x_2-x_3)(x_2-x_4)(x_2-x_5)(x_3-x_4)(x_3-x_5)(x_4-x_5) \]
is evidently a function with two different values; its second value is the same function with the opposite sign.  Therefore we may take $v_1' = \rho$ and $v_2' = -\rho$.  The expression for $v_1$ then becomes
\[ v_1 = \frac{t_1 + \rho t}{2\rho}, \]
or equivalently 
\[ v_1 = \frac{1}{2} t + \frac{t_1}{2\rho^2} \rho, \]
where $\frac{1}{2} t$ is a symmetric function.  The two values of $\rho$ differ by a sign, so $\frac{t_1}{2\rho^2}$ is also a symmetric function.   Therefore, setting $\frac{1}{2} t = p$ and $\frac{t_1}{2 \rho^2} = q$, it follows that 
\begin{quote}
Every function of five quantities which has two different values may be put in the form $p+q\rho$, where $p$ and $q$ are both symmetric functions and $\rho = (x_1-x_2)(x_1-x_3) \cdots (x_4 - x_5)$. 
\end{quote}
For the argument we wish to make, we will also need to know the general form taken by functions of five quantities which have five different values.  This can be found as follows:

Let $v$ be a rational function of the quantities $x_1$, $x_2$, $x_3$, $x_4$, $x_5$, with the property that it remains unchanged when we permute four of these quantities, say $x_2$,$x_3$,$x_4$,$x_5$.   In this case, $v$ is evidently symmetric in $x_2$,$x_3$,$x_4$,$x_5$.   We may therefore express $v$ as a rational function of $x_1$ and the symmetric functions of $x_2$,$x_3$,$x_4$,$x_5$.  But any symmetric function of these quantities can be expressed as a rational function of the coefficients of the fourth degree equation whose roots are $x_2$, $x_3$, $x_4$, $x_5$.  That is, if
\[ (x-x_2)(x-x_3)(x-x_4)(x-x_5) = x^4 - px^3 + qx^2 - rx + s, \]
the function $v$ may be expressed as a rational function of $x_1$, $p$, $q$, $r$, $s$.  But if
\[ (x-x_1)(x-x_2)(x-x_3)(x-x_4)(x-x_5) = x^5 - ax^4 + bx^3 - cx^2 + dx - e, \]
then 
\[ (x-x_1)(x^4 - px^3 + qx^2 - rx + s) = x^5 - ax^4 + bx^3 - cx^2 + dx - e\]
\[ = x^5 - (p+x_1)x^4 + (q+px_1)x^3 - (r+qx_1)x^2 + (s+rx_1)x - sx_1, \]
from which we deduce
\begin{eqnarray*} p &=& a-x_1, \\
q = b - ax_1 + x_1^2, \\
r = c - bx_1 + ax_1^2 - x_1^3, \\
s = d - cx_1 + bx_1^2 - a x_1^3 + x_1^4 ;\end{eqnarray*}
the function $v$ may therefore be expressed as a rational function of $x_1$, $a$, $b$, $c$, $d$. 

It follows from this that the function $v$ can be put in the form
\[ v = \frac{t}{\phi x_1}, \]
where $t$ and $\phi x_1$ are polynomial functions of $x_1$, $a$, $b$, $c$, $d$.  Multiplying the numerator and denominator by the expression $\phi x_2 \cdot \phi x_3 \cdot \phi x_4 \cdot \phi x_5$,
\[ v = \frac{t\cdot \phi x_2 \cdot \phi x_3 \cdot \phi x_4 \cdot \phi x_5}{\phi x_1 \cdot \phi x_2 \cdot \phi x_3 \cdot \phi x_4 \cdot \phi x_5} .\]
Now $\phi x_2 \cdot \phi x_3 \cdot \phi x_4 \cdot \phi x_5$ is a symmetric polynomial function of $x_2$, $x_3$, $x_4$, $x_5$.  We may therefore express it as a polynomial function of $p$, $q$, $r$, $s$, and therefore as a polynomial function of $x_1$, $a$, $b$, $c$, $d$.  The numberator of the fraction is therefore a polynomial function of the same quantities; the denominator is a symmetric function of $x_1$, $x_2$, $x_3$, $x_4$, $x_5$ and consequently can be expressed as a rational function of $a$, $b$, $c$, $d$, $e$.  We can therefore write 
\[ v = r_0 + r_1 x_1 + r_2 x_2 + \cdots + r_m x_1^m, \]
[ where $r_0$, $r_1$, $r_2$, \dots $r_m$ are symmetric functions of $a$, $b$, $c$, $d$, $e$.]

Multiplying the equation
\[ x_1^5 = ax_1^4 - bx_1^3 + cx_1^2 - dx_1 + e \]
successively by $x_1$, $x_1^2$, \dots, $x_1^{m-5}$, one obtains $m-4$ equations, which can clearly be used to find expressions for $x_1^5$, $x_1^6$, \dots $x_1^m$ of the form
\[ \alpha + \beta x_1 + \gamma x_1^2 + \delta x_1^3 + \epsilon x_1^4, \]
where $\alpha$, $\beta$, $\gamma$, $\delta$, $\epsilon$ are rational functions of $a$,$b$,$c$,$d$,$e$.

On may therefore reduce $v$ to the form
\begin{equation} \label{eqna} v = r_0 + r_1 x_1 + r_2 x_1^2 + r_3 x_1^3 + r_4 x_1^4, \end{equation}
where $r_0$, $r_1$, $r_2$ etc. are rational functions of $a$, $b$, $c$, $d$, $e$, and are therefore symmetric functions of $x_1$, $x_2$, $x_3$, $x_4$, $x_5$. 

This is the general form of a function which remains unaltered when we exchange the quantities $x_2$, $x_3$, $x_4$, $x_5$.  Such a function either has $5$ values, or it is symmetric.  

Now let $v$ be a rational function of $x_1$, $x_2$, $x_3$, $x_4$, $x_5$, which has the five values $v_1$, $v_2$, $v_3$, $v_4$, $v_5$.  Consider the function $x_1^m v$.   As we exchange the four quantities $x_2$, $x_3$, $x_4$, $x_5$ in all possible ways, the function $x_1^m v$ only takes the following values:
\[ x_1^m v_1 ,x_1^m v_2, x_1^m v_3, x_1^m v_4, x_1^m v_5. \] 

Now, I claim that the number of distinct values of $x_1^mv$ which result from these changes must be strictly less than five.  Indeed, if all five values are attained, then successively exchanging $x_1$ with $x_2$, $x_3$, $x_4$, $x_5$ gives us $20$ new values, which are necessarily different from each other and the values above.  The function therefore has $25$ different values, which is impossible, because $25$ does not divide the product $1 \cdot 2 \cdot 3 \cdot 4 \cdot 5$.  Denoting by $\mu$ the number of values taken by $v$ when we interchange the quantities $x_2$, $x_3$, $x_4$, $x_5$ in all possible ways, we see that $\mu$ must have one of the four values $1$, $2$, $3$, $4$.

\textit{1.} If $\mu = 1$, the discussion above shows that $v$ takes the form (\ref{eqna}).

\textit{2.} If $\mu = 4$, then the sum $v_1+v_2+v_3+v_4$ is a function of the form (\ref{eqna}).  But we have 
$v_5 = ( v_1 + v_2 + v_3 + v_4 + v_5) - (v_1 + v_2 + v_3 + v_4) = $ a symmetric function minus $v_1 + v_2 + v_3 + v_4$, so $v_5$ is of the form (\ref{eqna}).  

\textit{3.} If $\mu = 2$, then $v_1 + v_2$ is a function of the form (\ref{eqna}).  Write
\[ v_1 + v_2 = r_0 + r_1x_1 + r_2  x_1^2 + r_3 x_1^3 + r_4 x_1^4 = \phi x_1. \]
If we exchange $x_1$ with $x_2$, $x_3$, $x_4$, and $x_5$ in turn, we have 
\begin{eqnarray} v_1 + v_2 &=& \phi x_1, \\
v_2 + v_3 &=& \phi x_2, \\
\cdots &  & \cdots \\
v_{m-1} + v_m &=& \phi x_{m-1} ,\\
v_m + v_1 &=& \phi x_m, \end{eqnarray}
where $m$ is one of the numbers $2$, $3$, $4$, $5$.  If $m = 2$, then $\phi x_1 = \phi x_2$, which is impossible, because the number of values of $\phi x_1$ is five.  If $m=3$, then we have 
\[ v_1 + v_2 = \phi x_1, \; v_2 + v_3 = \phi x_2 , \; v+3 + v_1 =  \phi x_3 ,\]
 from which we deduce
 \[ 2v_1 = \phi x_1 - \phi x_2 + \phi x_3. \]
 But the right hand side has more than $5$ values, indeed it has $30$.  One shows in the same way that $m$ cannot be equal to $4$ or $5$.  It follows that $\mu$ is not equal to $2$. 
  
\textit{4.} If $\mu = 3$, then $v_1 + v_2 + v_3$ has five values, and therefore so does $v_4 + v_5 = (v_1 + v_2 + v_3 + v_4 + v_5) - (v_1 + v_2 + v_3)$.  But we have already seen that this is impossible.  Therefore $\mu$ is also not equal to $3$. 

From all of this we deduce the following theorem:
\begin{quote}
Any rational function of five quantities which has five different values necessarily takes the form 
\[ r_0 + r_1 x + r_2 x^2 + r_x x^3 + r_4 x^4, \]
where $r_0$, $r_1$, $r_2$ etc. are symmetric functions, and $x$ is one of the five quantities. 
\end{quote}
From the equation
\[ r_0 + r_1 x + r_2 x^2 + r_x x^3 + r_4 x^4 =  v \]
one can easily deduces, using the equation $x^5 - ax^4 + bx^3 - cx^2 + dx - e = 0$, an expression of the form
\[ x = s_0 + s_1 v + s_2 v^2 + s_3 v^3 + s_4 v^4, \]
where $s_0, s_1, s_2, s_3, s_4$, like $r_0, r_1, r_2, r_3, r_4$, are symmetric functions.

Let $v$ be a rational function which takes $m$ different values $v_1$, $v_2$, $v_3$, \dots, $v_m$. Writing
\[ (v-v_1)(v-v_2)(v-v_3) \cdots (v-v_n) = q_0 + q_1 v + q_2 v^2 + \cdots + q_{m-1}v^{m-1} + v^m = 0, \]
one knows that $q_0, q_1, q_2, \dots$ are symmetric functions, and that the $m$ roots of the equation are $v_1$, $v_2$, $v_3$, \dots, $v_m$.  Now I claim that it is impossible to express the value of $v$ as a root of an equation of the same form but of strictly lower degree.  Indeed, if 
\[ t_0 + t_1 v + t_2 v^2 + \cdots + t_{\mu -1} v^{\mu -1} + t_{\mu}v^{\mu} = 0 \]
is such an equation, with $t_0, t_1,$ etc. being symmetric functions, and if $v_1$ is a value of $v$ which satisfies this equation, we have 
\[ v^{\mu} + t_{\mu-1} v^{\mu-1} + \cdots = (v-v_1) P_1. \] 
Interchanging the variables in the function, we find the following series of equations:
\begin{eqnarray*} v^{\mu} + t_{\mu-1} v^{\mu-1} + \cdots &=& (v-v_1) P_1 \\
 v^{\mu} + t_{\mu-1} v^{\mu-1} + \cdots &=& (v-v_2) P_2 \\
 v^{\mu} + t_{\mu-1} v^{\mu-1} + \cdots &=& (v-v_3) P_3 \\
\cdots \cdots \cdots &=& \cdots \\
v^{\mu} + t_{\mu-1} v^{\mu-1} + \cdots &=& (v-v_m) P_m \\ 
\end{eqnarray*}
We conclude from this that $v-v_1$, $v-v_2$, $v-v_3$, \dots, $v-v_m$ are the factors of $v^{\mu} + t_{\mu-1} v^{\mu-1} + \cdots $ and therefore $\mu$ is equal to $m$.  This gives us the following theorem:
\begin{quote}
When a function of several quantities has $m$ different values, we can always find an equation of degree $m$, whose coefficients are symmetric functions, whose roots are these $m$ values; however it is impossible to find an equation of the same form but of smaller degree which has one or more of these values as a root.   
\end{quote}

\begin{center} 
\textbf{IV.}

\textit{Demonstration of the impossibility of solving the general equation of degree five.}
\end{center}

Using the propositions we have already shown, we can prove the theorem:
\begin{quote} It is impossible find a general solution of equations of degree five. \end{quote}  
From section $II$, all of the algebraic functions of which an algebraic expression for one of the roots is composed  can be expressed as a rational function of the roots.  

Since it is impossible to express any root of an equation as a rational function of the coefficieints, we must have 
\[ R^{\frac{1}{m}} = v, \]
where $m$ is a prime number and $R$ is a rational function of the coefficients, or equivalently a symmetric function of the roots, and $v$ is a rational function of the roots.  From this we conclude,
\[ v^m - R = 0 .\]
As we have seen in section $II$, it is impossible to lower the degree of this equation; by the final theorem of section $III$, $v$ must take $m$ different values.  The number $m$ is necessarily a divisor of $1 \cdot 2 \cdot 3 \cdot 4 \cdot 5$, so it may be equal to $2$, $3$, or $5$.  Now (see section III), no function of five variables can have three values: therefore either $m = 5$ or $m=2$.  If $m=5$, then by the results of the previous section
\[ \sqrt[5]{R} = r_0 + r_1 x + r_2 x^2 + r_3x^3 + r_4x^4, \]
and therefore
\[ x = s_0 + s_1 R^{\frac{1}{5}} + s_2 R^{\frac{2}{5}} + s_3 R^{\frac{3}{5}} + s_4 R^{\frac{4}{5}}. \]
From this we deduce (see section II)
\[ s_1 R^{\frac{1}{5}} = \frac{1}{5} (x_1 + \alpha^4 x_2 + \alpha^3 x_3 + \alpha^2 x_4 + \alpha x_5), \]
where $\alpha^5 = 1$.  But this equation is impossible, because the right hand side has 120 values, yet it is a root of an equation of degree five, $z^5 - s_1^5R = 0$.  We therefore must have $m=2$.

We then have (see section II),
\[ \sqrt{R} = p + qs, \]
where $p$ and $q$ are symmetric functions, and 
\[ s = (x_1 - x_2) \cdots (x_4 - x_5) .\]
Exchanging $x_1$ and $x_2$, we deduce
\[ - \sqrt{R} = p - qs, \]
from which we see that $p = 0$ and $\sqrt{R} = qs$.  We see from this that any algebraic function of the first order which can be found in the expression for the root must necessarily take the form $\alpha + \beta \sqrt{s^2} = \alpha + \beta s$, where $\alpha$ and $\beta$ are symmetric functions.  Now, it is impossible to express the root as a function of the form $\alpha + \beta \sqrt{R}$; we must therefore have an equation of the form
\[ \sqrt[m]{\alpha + \beta \sqrt{s^2}} = v, \]
where $\alpha$ and $\beta$ are nonzero, $m$ is a prime number, $\alpha$ and $\beta$ are symmetric functions, and $r$ is a rational function of the roots.  This gives 
\[ \sqrt[m]{\alpha + \beta s} = v_1, \; \sqrt[m]{\alpha + \beta s} = v_2, \] 
where $v_1$ and $v_2$ are rational functions of the roots.  Multiplying them,
\[ v_1v_2 = \sqrt[m]{\alpha^2 - \beta^2 s^2}. \]
Now $\alpha^2 - \beta^2 s^2$ is a symmetric function.  If $\sqrt[m]{\alpha^2 - \beta^2s^2}$ is not a symmetric function, then the number $m$, as we have seen, must be equal to $2$.  By in this case $v$ would be equal to $\sqrt{\alpha + \beta \sqrt{s^2}}$, in which case it would have $4$ values, which is impossible.  

Therefore, it must be that $\sqrt[m]{\alpha^2 - \beta^2 s^2} $ is a symmetric function.  Denoting this function by $\gamma$, we have 
\[ v_2 v_1 = \gamma, \; \mathrm{and} \; v_2 = \frac{\gamma}{v_1}. \]
Let 
\[ p = v_1 + v_2 = \sqrt[m]{\alpha + \beta \sqrt{s^2}} + \frac{\gamma}{\sqrt[m]{\alpha + \beta \sqrt{s^2}}} = \sqrt[m]{R} + \frac{\gamma}{\sqrt[m]{R}} = R^{\frac{1}{m}} + \frac{\gamma}{R} R^{\frac{m-1}{m}}. \]
Denote by $p_1$, $p_2$, $p_3$,\dots, $p_m$ the different values of $p$ which result from substituting $\alpha R^{\frac{1}{m}}$, $\alpha^2 R^{\frac{1}{m}}$, $\alpha^3 R^{\frac{1}{m}}$, \dots, $\alpha^{m-1} R^{\frac{1}{m}}$ for $R^{\frac{1}{m}}$, where $\alpha$ satisfies
\[ \alpha^{m-1} + \alpha^{m-2} + \cdots + \alpha + 1 = 0, \]
and consider the product
\[ (p-p_1)(p-p_2)\cdots(p-p_m) = p^m - Ap^{m-1} + A_1 p^{m-2} - \cdots = 0. \]
One sees easily that $A$, $A_1$ etc. are rational functions of the coefficients of the given equation, and therefore are symmetric functions of the roots.  Moreover, the equation is evidently irreducible.  By the final theorem of section III, it follows that $p$, considered as a function of the roots, must have $m$ different values.  We conclude that $m = 5$.  Therefore we have 
\[ \sqrt[5]{R} + \frac{\gamma}{\sqrt[5]{R}} = r_0 + r_1 x + r_2 x^2 + r_3 x^3 + r_4 x^4 = p, \]
from which 
\[ x = s_0 + s_1 p + s_2 p^2 + s_3 p^3 + s_4 p^4 , \]
or putting $\sqrt[5]{R}+ \frac{\gamma}{\sqrt[5]{R}}$ in place of $p$,
\[ x = t_0 + t_1 R^{\frac{1}{5}} + t_2 R^{\frac{2}{5}} + t_3 R^{\frac{3}{5}} + t_4 R^{\frac{4}{5}} ,\]
where $t_0$, $t_1$, $t_2$ etc. are rational functions of $R$ and the given equation.  We deduce (see section II) that 
\[ t_1 R^{\frac{1}{5}} = \frac{1}{5}\left(x_1 + \alpha^4 x_2 + \alpha^3 x_3 + \alpha^2 x_4 + \alpha x_5\right) = p', \]
where 
\[ \alpha^4 + \alpha^3 + \alpha^2 + \alpha + 1 = 0. \]
The equation $p' = t_1 R^{\frac{1}{5}}$ implies that $p'^5 = t_1^5 R$.  Since $t_1^5 R$ has the form $u+ u'\sqrt{s^2}$, we have $p'^5 = u + u' \sqrt{s^2}$, which gives
\[ (p'^5 - u)^2 = u'^2 s^2. \]
Thus $p'$ satisfies an equation of degree $10$, whose coefficients are symmetric functions.  But by the final theorem of section III this is impossible, because 
\[ p' = \frac{1}{5}\left(x_1 + \alpha^4 x_2 + \alpha^3 x_3 + \alpha^2 x_4 + \alpha x_5\right) \]
has $120$ different values.  This is a contradiction. 

We conclude that the general equation of degree five has no algebraic solution.  

It follows immediately from this theorem that general equations of degrees greater than $5$ also have no algebraic solution.  Therefore, equations of degrees four and less are the only ones that can be solved algebraically in a completely general manner. 

\end{document}
