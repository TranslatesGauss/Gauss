\documentclass[12pt]{article}
\usepackage{graphicx}
\usepackage{wrapfig}
\usepackage{geometry}
\usepackage{amsmath}
\geometry{legalpaper, margin=1in}
\usepackage{color}
\usepackage{amssymb}
\usepackage{enumitem}
\parindent=0pt
\parskip=8pt
\arraycolsep=2pt
\usepackage{pgfplots}


\begin{document}

\begin{center}
A SIMPLE DEFINITE INTEGRAL EXPRESSING THE FINITE INTEGRAL $\Sigma^n \phi x$
\end{center}

According to the well-known theorem of \emph{Parseval}, one may express the finite integral $\Sigma^n \phi x$ as a definite double integral, but if I am not mistaken, this finite integral has never been expressed as a simple definite integral.   To do so will be the object of this memoire.  

Writing $\phi x$ for an arbitrary function of $x$, it is easy to see that one may always set
\begin{equation} \label{eqn1} \phi x = \int e^{vx} fv . dv, \end{equation}
where the integral is taken between two limits of $v$ which are independent of $x$, and $fv$ denotes a function of $v$ whose form depends on that of $\phi x$.  Setting $\Delta x = 1$, and taking the finite integral of both sides of equation (\ref{eqn1}), one gets
\begin{equation} \label{eqn2} \Sigma \phi x = \int e^{vx} \frac{f v}{e^v-1}dv, \end{equation}
to which must be added an arbitrary constant.  Taking a second finite integral, one obtains
\begin{equation} \label{eqn3} \Sigma^2 \phi x = \int e^{vx} \frac{fv}{\left(e^v-1\right)^2} dv. \end{equation}
In general, one finds that
\begin{equation} \label{eqn4} \Sigma^n \phi x = \int e^{vx} \frac{fv}{\left(e^v-1\right)^n} dv . \end{equation}
To complete the integral one must add to the right hand side a function of the form
\[ C + C_1 x + C_2 x^2 + \cdots + C_{n-1} x^{n-1}, \]
where $C$, $C_1$, $C_2$ etc. are arbitrary constants.

The task is now to find the value of the definite integral $\int e^{vx} \frac{fv}{\left(e^v-1\right)^n} dv$.  For this I will use a theorem of \emph{Lagrange} (Exerc. de calc. int. T. II p. 189):
\[ \frac{1}{4} \frac{e^v + 1}{e^v - 1} - \frac{1}{2v} = \int^{\frac{1}{0}}_0 \frac{dt . \sin(vt)}{e^{2\pi t} -1 },\]
from which follows
\[ \frac{1}{e^v -1} = \frac{1}{v} - \frac{1}{2}  + 2 \int_0^{\frac{1}{0}} \frac{dt . \sin(vt)}{e^{2\pi t} -1 }. \]

Substituting this value of $\frac{1}{e^v-1}$ in equation (\ref{eqn2}), one gets
\[ \Sigma \phi x = \int e^{vx} \frac{fv}{v} dv - \frac{1}{2} \int e^{vx} fv . dv + 2 \int_0^{\frac{1}{0}} \frac{dt}{e^{2\pi t} - 1} \int e^{vx} fv. \sin vt . dv .\]
The integral $\int e^{vx} fv.\sin vt . dv$ can be found in the following manner.  Substituting $x \pm t\sqrt{-1}$ for $x$ in equation (\ref{eqn1}), one gets:
\[ \phi(x + t\sqrt{-1}) = \int e^{vx}e^{vt\sqrt{-1}} f(v) . dv , \]
\[ \phi(x - t\sqrt{-1}) = \int e^{vx}e^{vt\sqrt{-1}} f(v) . dv .\]
Taking the difference and dividing by $2 \sqrt{-1}$, one gets
\[ \int e^{vx} \sin vt . fv. dv = \frac{\phi(x+t \sqrt{-1}) - \phi(x - t \sqrt{-1})}{2\sqrt{-1}}. \]
Therefore, 
\[ \Sigma \phi x = \int \phi x. dx - \frac{1}{2} \phi x + 2 \int_0^{\frac{1}{0}} \frac{dt}{e^{2\pi t} - 1} \frac{\phi(x+t \sqrt{-1}) - \phi(x - t \sqrt{-1})}{2\sqrt{-1}} .\]

Now to find the value of the general integral 
\[ \Sigma^n\phi x = \int e^{vx} . fv . \frac{dv}{\left(e^v-1\right)^n}, \]
one may set 
\[ \frac{1}{\left(e^v - 1\right)^n} = (-1)^n\left( A_{0,n} p + A_{1,n} \frac{dp}{dv} + A_{2,n} \frac{d^2p}{dv^2} + \cdots + A_{n-1,n} \frac{d^{n-1}p}{dv^{n-1}} \right), \]
where $p$ is equal to $\frac{1}{e^v-1}$, and $A_{0,n}, A_{1,n}, \dots$ are numerical coefficients to be determined.  Differentiating the equation above, one gets
\[ \frac{ne^v}{\left(e^v-1\right)^{n+1}} = (-1)^n \left(A_{0,n} \frac{dp}{dv} + A_{1,n} \frac{d^2p}{dv^2} + \cdots + A_{n-1},n \frac{d^np}{dv^n} \right). \]
Now,
\[ \frac{ne^v}{\left(e^v-1\right)^{n+1}} = \frac{n}{\left(e^v-1\right)^n} + \frac{n}{\left(e^v-1\right)^{n+1}} , \]
and therefore, 
\begin{eqnarray*} \frac{ne^v}{\left( e^v -1\right)^{n+1}} &=& n (-1)^{n-1}\left( A_{0,n} p + A_{1,n} \frac{dp}{dv} + A_{2,n} \frac{d^2p}{dv^2} + \cdots + A_{n-1,n} \frac{d^{n-1}p}{dv^{n-1}} \right) \\
&  &+  \; n(-1)^n\left( A_{0,n+1} p + A_{1,n+1} \frac{dp}{dv} + A_{2,n} \frac{d^2p}{dv^2} + \cdots + A_{n,n+1} \frac{d^n p}{dv^n} \right) . \end{eqnarray*}
Comparing these two expressions for $\frac{ne^v}{\left(e^v-1\right)^{n+1}}$, one deduces the following equations:
\[ \begin{array}{rclcrcl} A_{0,n+1} - A_{0,n} &=& 0 & \hspace{2cm} & \Delta A_{0,n} = 0 , \\ \\
A_{1,n+1} - A_{1,n} &=& \frac{1}{n} A_{0,n} &  & \Delta A_{1,n} = \frac{1}{n}A_{0,n} , \\ \\
A_{2,n+1} - A_{2,n} &=& \frac{1}{n}A_{1,n} &  & \Delta A_{0,n} = \frac{1}{n} A_{1,n} , \\ \\
\cdots & \cdots & \cdots & & \cdots\\ \\
A_{n-1,n+1} - A_{n-1,n} &=& \frac{1}{n}A_{n-2,n} &  & \Delta A_{n-1,n} = \frac{1}{n} A_{n-2,n} , \\ \\
A_{n,n+1} &=& \frac{1}{n}A_{n-1,n}, & & \end{array} \]
from which one gets
\[ \begin{array}{c} A_{0,n} = 1, \; A_{1,n} = \Sigma \frac{1}{n}, \; A_{2,n} + \Sigma \frac{1}{n} \Sigma \frac{1}{n}, \; A_{3,n} = \Sigma \frac{1}{n} \Sigma \frac{1}{n} \Sigma \frac{1}{n}, \; \mathrm{etc.} \\ \\
A_{n,n+1} = \frac{1}{n} \cdot \frac{1}{n-1} \cdot \frac{1}{n-2} \cdots \frac{1}{2} \cdot \frac{1}{1} \cdot A_{0,1} = \frac{1}{\Gamma(n +1)} . \end{array} \]
The last equation serves to determine the constants in the expressions for $A_{0,n}$, $A_{1,n}$, $A_{2,n}$ etc.

Having thus determined the coefficients $A_{0,n}$, $A_{1,n}$, $A_{2,n}$ etc., substituting the value of $\frac{1}{\left(e^v-1\right)^n}$ in equation (\ref{eqn3}) gives 
\[ \Sigma^n \phi x = (-1)^{n-1} \int e^{vx} fv.dv \left(A_{0,n} p + A_{1,n} \frac{dp}{dv} + \cdots + A_{n-1,n} \frac{d^{n-1} p}{dv^{n-1}}\right). \]
Now, one has
\[ p = \frac{1}{v} - \frac{1}{2} + 2 \int_0^{\frac{1}{0}} \frac{dt . \sin vt}{e^{2\pi t} - 1} , \]
and differentiating gives
\begin{eqnarray*} \frac{dp}{dv} & = & - \frac{1}{v^2} + 2 \int_0^{\frac{1}{0}} \frac{t dt. \cos vt}{e^{2\pi t} - 1} , \\
\frac{d^2p}{dv^2} & = & \frac{2}{v^2} - 2 \int_0^{\frac{1}{0}} \frac{t dt. \sin vt}{e^{2\pi t} - 1} , \\
\frac{d^3p}{dv^3} & = & \frac{2 \cdot 3}{v^2} - 2 \int_0^{\frac{1}{0}} \frac{t dt. \cos vt}{e^{2\pi t} - 1} \; \mathrm{etc.;} \end{eqnarray*} 
therefore, by substitution
\[ \begin{array}{rcl} \Sigma^n \phi x &=& \displaystyle \int \left( A_{n-1,n} \frac{\Gamma n}{v^n} - A_{n-2,n} \frac{\Gamma(n-1)}{v^{n-1}} + \cdots + (-1)^{n-1} A_{0,n} \frac{1}{v} + (-1)^n . \frac{1}{2}\right) e^{vx} fv.dv \\
& + & 2 (-1)^{n-1} \displaystyle \iint \frac{P \sin vt.dt}{e^{2\pi t}-1} e^{vx} fv.dv + 2 (-1)^{n-1} \iint_0^{\frac{1}{0}} \frac{Q \cos vt . dt}{e^{2\pi t} - 1} e^{vx} fv.dv.\end{array} \]
Integrating the equation $\phi x = \int e^{vx}fv.dv$, one gets:
\begin{eqnarray*} \int \phi x. dx &=& \int e^{vx} fv \frac{dv}{v} ,\\
\int^2 \phi x dx^2 &=& \int e^{vx} fv \frac{dv}{v^2} ,\\
\int^3 \phi x . dx^3 &=& \int e^{vx} fv \frac{dv}{v^3} \;  \; \; \mathrm{etc. } \end{eqnarray*}
Furthermore,
\[ \int e^{vx} \sin vt . fv. dv = \frac{\phi(x+t \sqrt{-1}) - \phi(x - t \sqrt{-1})}{2\sqrt{-1}}. \]
\[ \int e^{vx} \cos vt . fv. dv = \frac{\phi(x+t \sqrt{-1}) + \phi(x - t \sqrt{-1})}{2}, \]
 so by substitution one has
  \begin{eqnarray*} \Sigma^n \phi x &=& A_{n-1,n} \Gamma n \int^n \phi x.dx^n - A_{n-2,n} \Gamma(n-1) \int^{n-1} \phi x.dx^{n-1} + \cdots + (-1)^{n-1} \int \phi x. dx \\
  &+& (-1)^n\cdot \frac{\phi x}{2} +  2 (-1)^{n-1} \int_0^{\frac{1}{0}} \frac{Pdt}{e^{2\pi t} - 1} \frac{\phi(x+t \sqrt{-1}) - \phi(x - t \sqrt{-1})}{2\sqrt{-1}} \\
 &+& 2 (-1)^{n-1} \int_0^{\frac{1}{0}} \frac{Qdt}{e^{2\pi t}-1} \frac{\phi(x+t \sqrt{-1}) + \phi(x - t \sqrt{-1})}{2} \end{eqnarray*}
 where 
 \begin{eqnarray*} P &=& A_{0,n} - A_{2,n}t^2 + A_{4,n}t^4 - \cdots, \\
 Q &=& A_{1,n} t - A_{3,n}t^3 + A_{5,n}t^5 - \cdots  .\end{eqnarray*}
 For example, taking $n=2$, one gets
 \begin{eqnarray*} \Sigma^2\phi x &=& \iint \phi x.dx^2 - \int \phi x .dx + \frac{1}{2} \phi x  \\
 & - & 2 \int_0^{\frac{1}{0}} \frac{dt}{e^{2\pi t}-1} \frac{\phi(x+t \sqrt{-1}) - \phi(x - t \sqrt{-1})}{2\sqrt{-1}} \\
 & - & 2 \int_0^{\frac{1}{0}} \frac{dt}{e^{2\pi t}-1} \frac{\phi(x+t \sqrt{-1}) + \phi(x - t \sqrt{-1})}{2} \end{eqnarray*}  
 and taking $\phi x = e^{ax}$,
 \[ \phi(x \pm t\sqrt{-1}) = e^{ax}e^{\pm at\sqrt{-1}}, \; \int e^{ax}dx = \frac{1}{a}e^{ax}, \; \iint e^{ax}dx^2 = \frac{1}{a^2}e^{ax}. \]
 Substituting and dividing by $e^{ax}$,
\[ \frac{1}{\left(e^a - 1\right)^2} = \frac{1}{2} - \frac{1}{a} + \frac{1}{a^2} - 2 \int_0^{\frac{1}{0}} \frac{dt.\sin at}{e^{2\pi t} -1} - 2 \int_0^{\frac{1}{0}} \frac{tdt.\cos at}{e^{2\pi t}-1} . \]
The most remarkable case is when $n=1$.  Then one has, from what we have seen:
\[ \Sigma \phi x = C + \int \phi x.dx - \frac{1}{2} \phi x+ 2 \int_0^{\frac{1}{0}}\frac{dt}{e^{2 \pi t} -1} \frac{\phi(x+t \sqrt{-1}) - \phi(x - t \sqrt{-1})}{2\sqrt{-1}} . \]
Assuming that the integrals $\Sigma \phi x$ and $\int \phi x dx$ vanish when $x = a$, one clearly has 
\[ C = \frac{1}{2} \phi a - 2 \int_0^{\frac{1}{0}} \frac{dt}{e^{2\pi t} -1} \frac{\phi(a+t \sqrt{-1}) - \phi(a - t \sqrt{-1})}{2\sqrt{-1}}, \]
and therefore
\begin{eqnarray*} \Sigma \phi x &=&  \int \phi x.dx + \frac{1}{2} \left( \phi a-  \phi x \right) + 2 \int_0^{\frac{1}{0}}\frac{dt}{e^{2 \pi t} -1} \frac{\phi(x+t \sqrt{-1}) - \phi(x - t \sqrt{-1})}{2\sqrt{-1}} \\&-& 2 \int_0^{\frac{1}{0}} \frac{dt}{e^{2\pi t} -1} \frac{\phi(a+t \sqrt{-1}) - \phi(a - t \sqrt{-1})}{2\sqrt{-1}}. \end{eqnarray*}
Taking $x = \infty$, and assuming that $\phi x$ and $\int \phi x .dx$ vanish for this value of $x$, one has:
\[ \begin{array}{l} \phi a + \phi(a+1) + \phi(a+2) + \phi(a+3) + \dots \\ \\
   = \displaystyle \int_0^{\frac{1}{0}} \phi x. dx + \frac{1}{2} \phi a - 2 \int_0^{\frac{1}{0}} \frac{dt}{e^{2\pi t}-1} \frac{\phi(a+t \sqrt{-1}) - \phi(a - t \sqrt{-1})}{2\sqrt{-1}} . \end{array} \] 
For example, if $\phi x = \frac{1}{x^2}$, 
\[ \frac{\phi(a+t \sqrt{-1}) - \phi(a - t \sqrt{-1})}{2\sqrt{-1}} = \frac{-2at}{\left(a^2 + t^2\right)^2} , \]
and therefore 
\[ \frac{1}{a^2} + \frac{1}{\left(a+1\right)^2} + \frac{1}{\left(a+2\right)^2} + \cdots = \frac{1}{2a^2} + \frac{1}{a} + 4a \int_0^{\frac{1}{0}} \frac{t dt}{(e^{2\pi t} - 1)(a^2 + t^2)^2} . \]
Setting $a = 1$,
\[ 1 + \frac{1}{4} + \frac{1}{9} + \frac{1}{16} + \frac{1}{25} + \cdots = \frac{\pi^2}{6} = \frac{3}{2} + 4 \int_0^{\frac{1}{0}}  \frac{t dt}{(e^{2\pi t}-1)(1+ t^2)^2 } . \]
  \end{document}