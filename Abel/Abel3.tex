\documentclass[12pt]{article}
\usepackage{graphicx}
\usepackage{wrapfig}
\usepackage{geometry}
\usepackage{amsmath}
\geometry{legalpaper, margin=1in}
\usepackage{color}
\usepackage{amssymb}
\usepackage{enumitem}
\parindent=0pt
\parskip=8pt
\arraycolsep=2pt
\usepackage{pgfplots}


\begin{document}

\begin{center}
MEMOIRE ON ALGEBRAIC EQUATIONS, IN WHICH THE IMPOSSIBILITY OF SOLVING THE GENERAL EQUATION OF DEGREE FIVE IS DEMONSTRATED .
\end{center}

The geometers have occupied themselves intensely with the solution of general algebraic equations, and many of them have sought to prove its impossibility; but if I am not mistaken, none of them have been successful so far.  I permit myself to hope that the geometers will kindly receive this memoire, whose aim is to fill this gap in the theory of algebraic equations. 

Let $y^5 - ay^4 + by^3 - cy^2 + dy - e = 0$ be the general equation of degree $5$, and let us suppose that it can be resolved agebraically, that is, we may express $y$ as a function of the quantities $a$, $b$, $c$, $d$, and $e$, which is formed using radicals.  If this is the case, it is clear that $y$ may be brought into the form
\[ y = p + p_1R^{\frac{1}{m}} + p_2 R^{\frac{2}{m}} + \cdots + p_{m-1}R^{\frac{m-1}{m}}, \]
with $m$ being a prime number and $R$, $p$, $p_1$, $p_2$, etc. being functions of the same form as $y$, and so on until we reach rational functions of the quantities $a$, $b$, $c$, $d$, and $e$.  We may suppose that it is impossible to express $R^{\frac{1}{m}}$ as a rational function of the quantities $a$, $b$, $p$, $p_1$, $p_2$ etc., and putting $\frac{R}{p_1^m}$ in place of $R$ it is clear that we may assume $p_1 = 1$. We then have
\[ y = p + R^{\frac{1}{m}} + p_2 R^{\frac{2}{m}} + \cdots + p_{m-1}R^{\frac{m-1}{m}}. \]
Substituting this value for $y$ in the proposed equation, we obtain, after simplifying, an equation of the form
\[ P = q + q_1R^{\frac{1}{m}} + q_2 R^{\frac{2}{m}} + \cdots + q_{m-1}R^{\frac{m-1}{m}}, \]
with $q$, $q_1$, $q_2$, etc. being polynomial functions of the quantities $a$, $b$, $c$, $d$, $e$, $p$, $p_2$, etc., and $R$.  In order for this equation to hold it is necessary that $q = 0$, $q_1=0$, $q_2=0$ etc. $q_m = 0$.  Indeed, writing $z$ for $R^{\frac{1}{m}}$, we would have both of the equations  
\[ z^m - R = 0 \; \mathrm{and} \; q + q_1z + \cdots + q_mz^{m-1} = 0.\]
Now, if the quantities $q$, $q_1$ etc. are not equal to zero, the equations will necessarily have one or more roots in common.  Letting $k$ be the number of these common roots, we know that it is possible to find an equation of degree $k$ whose roots are the $k$ aforementiond roots, and in which all of the coefficients are rational functions of $R$, $q$, $q_1$, et $q_{m-1}$. Let 
\[ r + r_1 z + r_2 z^2 + \cdots + r_k z^k = 0 \]
be this equation.  It has these roots in common with the equation $z^m-R= 0$; so all of its roots are of the form $\alpha_\mu z$, where $\alpha_\mu$ denotes one of the roots of the equation $\alpha_\mu^m -1 = 0$.  These being substituted, we have the following equations:
\[ r+ r_1 z + r_2 z^2 + \cdots + r_k z^k = 0,\]
\[ r+ \alpha r_1 z + \alpha^2 r_2 z^2 + \cdots + \alpha^k r_k z^k = 0,\]
\[ \cdots \cdots \cdots \cdots \cdots \]
\[ r+ \alpha_{k-2}r_1 z + r_2 z^2 + \cdots + r_k z^k = 0,\]
From these $k$ equations, we can always express $z$ as a rational function of the quantities $r$, $r_1$, $r_2$, etc. $r_k$, and since these quantities are themselves rational functions of $a$, $b$, $c$, $d$, $e$, $R, \dots$ $p$, $p_2$ etc., it follows that $z$ is also a rational function of these quantities; but this is contrary to the hypothesis.  Therefore,
\[ q = 0, \; q_1 = 0 \; \; \mathrm{etc.} \; q_{m-1} = 0 .\]
Now for these equations to hold, it is clear that the proposed equation is satisfied by all the values that are obtained for y by giving $R^{\frac{1}{m}}$ the values 
\[ R^{\frac{1}{m}}, \alpha R^{\frac{1}{m}},\alpha^2 R^{\frac{1}{m}},\alpha^3 R^{\frac{1}{m}}, \; \mathrm{etc.} \; \alpha^{m-1} R^{\frac{1}{m}} , \]
where $\alpha$ is a root of the equation
\[ \alpha^{m-1} + \alpha^{m-2} + \cdots + \alpha + 1 = 0 .\]
One can also see that all these values of $y$ are different; if not then there would be an equation of the same form as the equation $P=0$, and we have seen that any such equation leads to an impossible result.  Therefore the number $m$ cannot be greater than $5$.  

Denoting by $y_1$, $y_2$, $y_3$, $y_4$, $y_5$ the roots of the proposed equation, we have
\[ y_1 = p + R^{\frac{1}{m}} + p_2 R^{\frac{2}{m}} + \cdots + p_{m-1}R^{\frac{m-1}{m}} \]
\[ y_2 = p + \alpha R^{\frac{1}{m}} + p_2 \alpha^2 R^{\frac{2}{m}} + \cdots + \alpha^{m-1} p_{m-1}R^{\frac{m-1}{m}}\]
\[ \cdots \cdots \cdots \cdots \cdots \]
\[ y_m = p + \alpha^{m-1} R^{\frac{1}{m}} + p_2 \alpha^{m-2} R^{\frac{2}{m}} + \cdots + \alpha p_{m-1}R^{\frac{m-1}{m}} . \]
From these equations we easily obtain
\begin{eqnarray*} p &=& \frac{1}{m}\left(y_1 +y_2 + \cdots + y_m\right) \\
 R^{\frac{1}{m}} &=& \frac{1}{m}\left(y_1 + \alpha^{m-1}y_2 + \cdots + \alpha y_m\right) \\
 p_2R^{\frac{2}{m}} &=& \frac{1}{m} \left( y_1 + \alpha^{m-2} y_2 + \cdots + \alpha^2 y_m \right) \\
  && \\
   &&\cdots \cdots \cdots \cdots \cdots \\ 
   && \\   
   p_{m-1}R^{\frac{m-1}{m}} &=& \frac{1}{m} \left(y_1 + \alpha y_2 + \cdots + \alpha^{m-1} y_m \right).\end{eqnarray*} 
We see from this that $p$, $p_2$ etc. $p_{m-1}$, $R$ and $R^{\frac{1}{m}}$ are rational functions of the roots of the proposed equation. 

Let us now consider any one of these quantities, for example $R$.   Let
\[ R = S + v^{\frac{1}{n}} + S_2 v^{\frac{2}{n}} + \cdots + S_{n-1} v^{\frac{n-1}{n}}. \]
Treating this quantity in the same manner as $y$, we obtain a similar result, that $v^{ \frac{1}{n} }$, $v$, $S$, $S_2$ etc. are rational functions of the different values of the function $R$; and since these are rational functions of $y_1$,$y_2$, etc., the functions $v^{\frac{1}{n}}$, $v$, $S$, $S_2$ etc. are as well.  Following this line of reasoning, we conclude that all irrational functions contained in the expression for $y$ are rational functions of the roots of the proposed equation. 

Given this, it is not difficult to complete the proof.  First consider the irrational functions of the form $R^{\frac{1}{m}}$, where $R$ is a rational function of $a$, $b$, $c$, $d$, $e$.  Setting $R^{\frac{1}{m}} = r$, $r$ is a rational function of $y_1$, $y_2$, $y_3$, $y_4$, and $y_5$, and $R$ is a symmetric function of these quantities.  Now, since we are trying to resolve a general equation of degree $5$, it is clear that we may consider $y_1$, $y_2$, $y_3$, $y_4$, and $y_5$ as independent variables; the equation $R^{\frac{1}{m}} = r$ must take place within this supposition.  Consequently, we may interchange the quantities $y_1$, $y_2$, $y_3$, $y_4$, and $y_5$ in the equation $R^{\frac{1}{m}} = r$; through these changes $R^{\frac{1}{m}}$ must take on $m$ different values, since $R$ is a symmetric function.  The function $r$ therefore has that property that it obtains $5$ different values when we permute in all possible ways the $5$ variables it contains.  For this it is necessary that $m= 5$ or $m =2$, since $m$ is a prime number. (See the memoire of \emph{Cauchy} in the Journal de l'\'ecole polytechnique, volume XVII).  

First let us suppose that $m = 5$.  Then the function $r$ has $5$ different values, and can therefore be put in the form
\[ R^{\frac{1}{5}} = r = p + p_1y_1 + p_2 y_1^2 + p_3y_1^3 + p_4 y_1^4, \]
where $p, p_1, p_2, \dots$ are symmetric functions of $y_1$, $y_2$, etc.  Exchanging $y_1$ and $y_2$, this gives
\[ p + p_1 y_1 + p_2y_1^2 + p_3y_1^3 + p_4 y_1^4 = \alpha p + \alpha p_1 y_2 + \alpha p_2y_2^2 + \alpha p_3y_2^3 + \alpha p_4 y_2^4, \]
where 
\[ \alpha^4 + \alpha^3 + \alpha^2 + \alpha + 1 = 0 ; \]
but this equation cannot hold.  The number $m$ must therefore be equal to two.  Letting 
\[ R^{\frac{1}{2}} = r, \]
$r$ must have two different values of opposite sign; therefore (see the memoire of \emph{Cauchy}) 
\[ R^{\frac{1}{2}} = r = v(y_1 - y_2)(y_1 - y_3) \cdots (y_2 - y_3) \cdots (y_4 - y_5) = vS^{\frac{1}{2}}, \]
where $v$ is a symmetric function.  

Now let us consider the irrational functions of the form 
\[ \left( p + p_1 R^{\frac{1}{\nu}} + p_2 R_1^{\frac{1}{\mu}} + \cdots \right)^{\frac{1}{m}}, \]
where $p$, $p_1$, $p_2$, etc., $R$, $R_1$ etc. are rational functions of $a$, $b$, $c$, $d$, and $e$, and consequently are symmetric functions of $y_1$, $y_2$, $y_3$, $y_4$, and $y_5$.  As we have shown, we must have $\nu = \mu =$ etc. $=2$, $R= v^2S$, $R_1 = v_1^2 S$ etc.  The function above can therefore be put in the form 
\[ (p+p_1 S^{\frac{1}{2}})^{\frac{1}{m}}. \]
Let 
\[ r = (p+p_1S^{\frac{1}{2}})^{\frac{1}{m}}, \]
\[ r_1 = (p - p_1S^{\frac{1}{2}})^{\frac{1}{m}} .\] 
Multiplying, we have
\[ rr_1 = (p^2 - p_1S)^{\frac{1}{m}}. \]
Now if $rr_1$ is not a symmetric function, the number $m$ must be equal to two, but in this case $r$ will have four different values, which is impossible; it follows that $rr_1$ is a symmetric function.  Letting $v$ denote this function, consider 
\[ r + r_1 = (p+p_1S^{\frac{1}{2}})^{\frac{1}{m}} + v(p+p_1S^{\frac{1}{2}})^{\frac{-1}{m}} = z. \]
This function has $m$ different values,  and therefore $m=5$, since $m$ is a prime number.  Consequently, 
\[ z = q + q_1y + q_2 y^2 + q_3 y^3 + q_4 y^4 = (p+p_1S^{\frac{1}{2}})^{\frac{1}{m}} + v(p+p_1S^{\frac{1}{2}})^{\frac{-1}{m}}, \]
where $q$, $q_1$, $q_2$, are symmetric functions of $y_1$,$y_2$,$y_3$, etc. and consequently rational functions of $a$, $b$, $c$, $d$, and $e$.  Combining this with the proposed equation, we may express $y$ as a rational function of $z$, $a$, $b$, $c$, $d$, and $e$.  Such a function can always be reduced to the form 
\[ y = P + R^{\frac{1}{5}} + P_2 R^{\frac{2}{5}} + P_3 R^{\frac{3}{5}} + P_4 R^{\frac{4}{5}}, \]
where $P$, $R$, $P_2$, $P_3$, and $P_4$ are functions of the form $p + p_1S^{\frac{1}{2}}$, and $p$,$p_1$, and $S$ are rational functions of $a$, $b$, $c$, $d$, and $e$.  From this expression for $y$ we obtain
\[ R^{\frac{1}{5}} = \frac{1}{5}(y_1 + \alpha^4 y_2 + \alpha^3 y_3 + \alpha^2 y_4 + \alpha y_5) = (p + p_1 S^{\frac{1}{2}})^{\frac{1}{5}}, \]
where 
\[ \alpha^4 + \alpha^3 + \alpha^2 + \alpha + 1 = 0. \]
Now the first of these functions has $120$ different values and the second only has $10$; consequently $y$ cannot have the form we just found; but we have shown that $y$ necessarily has this form, if the proposed equation is solvable.  We conclude that 
\begin{center} \emph{It is impossible to solve the general equation of degree $5$ by radicals.} \end{center}
It follows immediately from this theorem that it is equally impossible to solve the general equations of degrees greater than $5$ by radicals.
  \end{document}