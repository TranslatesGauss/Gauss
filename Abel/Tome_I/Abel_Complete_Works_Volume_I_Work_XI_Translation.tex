\documentclass[oneside, 12 pt, leqno]{memoir}
\usepackage{standalone}
\usepackage[dvips,text={6.2in,8.5in},left=0.9truein,top=1.5truein]{geometry}
\usepackage{amsmath, amssymb, amsthm, amsfonts}
\usepackage{graphicx}
\usepackage{titlesec}
\usepackage{multirow}
\usepackage{wrapfig}
\usepackage{microtype}
\usepackage{indentfirst}
\usepackage[utf8]{inputenc}
\usepackage{exscale}
\usepackage{mlmodern}
\usepackage[OT1]{fontenc}
\usepackage[bottomfloats]{footmisc}
\parindent=2.27em
\parskip=0pt
\nonfrenchspacing
\renewcommand{\baselinestretch}{1.15}
\DeclareMathSizes{12}{12}{8}{6}
\everymath{\displaystyle}
\allowdisplaybreaks
\raggedbottom
\titleformat{\section}
  {\normalfont\centering}{\thesection.}{1em}{}
\titleformat{\subsection}
  {\normalfont\normalsize\centering}{\thesection.}{1em}{}
\titleformat{\subsubsection}
  {\normalfont\normalsize\centering}{\thesection.}{1em}{}
\spaceskip=0.67em plus 0.33em minus 0.33em
\thickmuskip=4mu plus 4mu
\medmuskip=3mu plus 1.5mu minus 3mu
\AtBeginDocument{%
  \mathchardef\stdcomma=\mathcode`,
  \mathcode`,="8000
}
\begingroup\lccode`~=`, \lowercase{\endgroup\def~}{\stdcomma\mspace{\medmuskip}}
%\let\oldfrac\frac
%\def\frac#1#2{\mathchoice{\text{\scalebox{.83}{${\oldfrac{#1}{#2}}$}}}{\text{\scalebox{.83}{${\displaystyle\oldfrac{#1}{#2}}$}}}{\genfrac{}{}{}{2}{#1}{#2}}{\genfrac{}{}{}{3}{#1}{#2}}}
\begin{document}
\setlength{\abovedisplayskip}{0.33\baselineskip plus .16\baselineskip minus .16\baselineskip}
\setlength{\belowdisplayskip}{0.33\baselineskip plus .16\baselineskip minus .16\baselineskip}

\;\\ [3\baselineskip]
\section*{\begin{Large}XI.\end{Large} \\ [\baselineskip]
ON THE INTEGRATION OF THE DIFFERENTIAL FORMULA \(\frac{\varrho dx}{\sqrt{R}}\), WHEN \(R\) AND \(\varrho\) ARE INTEGRAL FUNCTIONS.}
\begin{center}
\rule{2in}{0.1pt}\\ [0.5\baselineskip]
{\tiny Journal für die reine und angewandte Mathematik, herausgegeben von Crelle, Bd. 1, Berlin 1826.\par}
\rule{2in}{0.1pt}\\ [0.5\baselineskip]
\end{center}

\subsection*{1.}

If \(p\), \(q\) and \(R\) are integral functions of a variable quantity \(x\), and we differentiate the expression
\[\tag{1} z=\log \frac{p+q \sqrt{R}}{p-q \sqrt{R}}\]
with respect to \(x\), we obtain
\[d z=\frac{d p+d(q \sqrt{R})}{p+q \sqrt{R}}-\frac{d p-d(q \sqrt{R})}{p-q \sqrt{R}},\]
or
\[d z=\frac{(p-q \sqrt{R})[d p+d(q \sqrt{R})]-(p+q \sqrt{R})[d p-d(q \sqrt{R})]}{p^2-q^2 R},\]
that is,
\[d z=\frac{2 p d(q \sqrt{R})-2 d p. q \sqrt{R}}{p^2-q^2 R}.\]
But
\[d(q \sqrt{R})=d q \sqrt{R}+\frac{1}{2} q \frac{d R}{\sqrt{R}},\]
so by substitution
\[d z=\frac{p q d R+2(p d q-q d p) R}{\left(p^2-q^2 R\right) \sqrt{R}},\] 
and therefore, by setting
\[\tag{2}\begin{gathered}
p q \frac{d R}{d x}+2\left(p \frac{d q}{d x}-q \frac{d p}{d x}\right) R=M, \\
p^2-q^2 R=N,
\end{gathered}\]
we will have
\[\tag{3}d z=\frac{M d x}{N \sqrt{R}},\]
where, as can be easily seen, \(M\) and \(N\) are integral functions of \(x\).
Now, since \(z\) is equal to \(\log \frac{p+q \sqrt{R}}{p-q \sqrt{R}}\), integration yields
\[\tag{4}\int \frac{M d x}{N \sqrt{R}}=\log \frac{p+q \sqrt{R}}{p-q \sqrt{R}}.\]

It follows that in the differential \(\frac{\varrho d x}{\sqrt{R}}\), there can be found an infinity of different forms for the rational function \(\varrho\), which make this differential integrable by logarithms, and specifically by an expression of the form \(\log \frac{p+q \sqrt{R}}{p-q \sqrt{R}}\).  As can be seen from the equations (2), the function \(\varrho\) contains, in addition to \(R\), two indeterminate functions \(p\) and \(q\); it is determined by these functions.

One can reverse the question and ask whether it is possible to find functions \(p\) and \(q\) such that \(\varrho\) or \(\frac{M}{N}\) takes a given specific form. The solution to this problem leads to a wealth of interesting results, which must be viewed as just as many properties of functions of the form \(\int \frac{\varrho d x}{\sqrt{R}}\). In this paper I will limit myself to the case where \(\frac{M}{N}\) is an entire function of \(x\), and attempt to solve the following general problem:

\begin{quote}"Find all differentials of the form \(\frac{\varrho d x}{\sqrt{R}}\), where \(\varrho\) and \(R\) are integral functions of \(x\), whose integrals can be expressed in terms of a function of the form \(\log \frac{p+q \sqrt{R}}{p-q \sqrt{R}}\)." \end{quote}

\subsection*{2.}

By differentiating the equation
\[N=p^2-q^2 R,\]
we obtain
\[d N=2 p d p-2 q d q. R-q^2 d R;\]
thus by multiplying by \(p\),
\[p d N=2 p^2 d p-2 p q d q. R- pq^2 d R,\]
and when we replace \(p^2\) with its value \(N+q^2 R\),
\[p d N=2 N d p+2 q^2 d p. R-2 p q d q. R-p q^2 d R,\]
or
\[p d N=2 N d p-q[2(p d q-q d p) R+p q d R],\]
thus, since (2)
\[2(p d q-q d p) R+p q d R=M d x,\]
we have
\[p d N=2 N d p-q M d x,\]
or
\[q M=2 N \frac{d p}{d x}-p \frac{d N}{d x},\]
thus
\[\tag{5}\frac{M}{N}=\left(2 \frac{d p}{d x}-p \frac{d N}{N d x}\right) : q.\]
Now \(\frac{M}{N}\) must be an integral function of \(x\); denoting this function by \(\varrho\), we have
\[q \varrho=2 \frac{d p}{d x}-p \frac{d N}{N d x}.\]
It follows that \(p \frac{d N}{N d x}\) must be an integral function of \(x\). By setting
\[N=(x+a)^m\left(x+a_1\right)^{m_1} \dots\left(x+a_n\right)^{m_n},\]
we have
\[\frac{d N}{N d x}=\frac{m}{x+a}+\frac{m_1}{x+a_1}+\dots+\frac{m_n}{x+a_n},\] 
thus the expression
\[p\left(\frac{m}{x+a}+\frac{m_1}{x+a_1}+\dots+\frac{m_n}{x+a_n}\right)\]
must also be an integral function, which can only happen if the product \((x+a) \dots\left(x+a_n\right)\) is a factor of \(p\). Therefore, we must have
\[p=(x+a) \dots\left(x+a_n\right) p_1,
\]
with \(p_1\) being an integral function. Now
\[N=p^2-q^2 R,\]
thus
\[(x+a)^m \dots\left(x+a_n\right)^{m_n}=p_1^2(x+a)^2\left(x+a_1\right)^2 \dots\left(x+a_n\right)^2-q^2 R.\]
Since \(R\) does not have a factor of the form \((x+a)^2\), and since we can always assume that \(p\) and \(q\) have no common factors, it is clear that
\[m=m_1=\dots=m_n=1, \]
and that
\[R=(x+a)\left(x+a_1\right) \dots\left(x+a_n\right) R_1,\]
where \(R_1\) is an integral function. We therefore have
\[N=(x+a)\left(x+a_1\right) \dots\left(x+a_n\right), \quad R=N R_1,\]
that is, \(N\) must be a factor of \(R\). We also have \(p=N p_1\). By substituting these values of \(R\) and \(p\) into the equations (2), we find the following two equations
\[\tag{6}\begin{gathered}
p_1^2 N-q^2 R_1=1, \\
\frac{M}{N}=p_1 q \frac{d R}{d x}+2\left(p \frac{d q}{d x}-q \frac{d p}{d x}\right) R_1=\varrho.
\end{gathered}\]
The first of these equations determines the form of the functions \(p_1\), \(q\), \(N\), and \(R_1\); once these are determined, the second equation will then give the function \(\varrho\). This last function can also be found using equation (5).

\subsection*{3.}

Now everything depends on the equation
\[\tag{7} p_1^2 N-q^2 R_1=1.\]
This equation can indeed be solved using the ordinary method of indeterminate coefficients, but the application of this method would be extremely prolix here, and would hardly lead to a general result. I will therefore take another route, similar to the one used for solving indeterminate equations of the second degree with two unknowns. The only difference is that instead of integers, we will have to deal with integral functions. As we will often need to speak of the degree of a function, I will use the letter \(\delta\) to denote this degree, so that \(\delta P\) will denote the degree of the function \(P\).  For example,
\[\begin{gathered}
\delta\left(x^m+a x^{m-1}+\dots\right)=m, \\
\delta\left(\frac{x^5+c x}{x^3+e}\right)=2, \\
\delta\left(\frac{x+e}{x^2+k}\right)=-1, \text{ etc. }
\end{gathered}\]
Moreover, it is clear that the following equations hold:
\[\begin{aligned}
& \delta(P Q)=\delta P+\delta Q, \\
& \delta\left(\frac{P}{Q}\right)=\delta P-\delta Q, \\
& \delta\left(P^m\right)=m \delta P ;
\end{aligned}\]
in addition
\[\delta\left(P+P^{\prime}\right)=\delta P,\]
if \(\delta P^{\prime}\) is smaller than \(\delta P\). Similarly, I will denote, for brevity, the integer part of a rational function \(u\) by \(E u\), so that
\[u=E u+u^{\prime}\]
where \(\delta u^{\prime}\) is negative. It is clear that
\[E\left(s+s^{\prime}\right)=E s+E s^{\prime},\]
thus when \(\delta s'\) is negative,
\[E\left(s+s^{\prime}\right)=E s.\]
In this notation, we will have the following theorem:

\begin{quote}"When the three rational functions \(u\), \(v\), and \(z\) have the property that
\[u^2=v^2+z\]
we will have, if \(\delta z<\delta v\), \(E u= \pm E v\)."
\end{quote}

Indeed, we have by definition
\[\begin{aligned}
& u=E u+u^{\prime}, \\
& v=E v+v^{\prime},
\end{aligned}\]
where \(\delta u^{\prime}\) and \(\delta v^{\prime}\) are negative; therefore, substituting these values into the equation \(u^2=v^2+z\), we obtain
\[(E u)^2+2 u^{\prime} E u+u^{\prime 2}=(E v)^2+2 v^{\prime} E v+v^{\prime 2}+z.\]
It follows that
\[(E u)^2-(E v)^2=z+v^{\prime 2}-u^{\prime 2}+2 v^{\prime} E v-2 u^{\prime} E u=t,\]
or,
\[(E u+E v)(E u-E v)=t.\]
We easily see that \(\delta t<\delta v\); on the other hand, \(\delta(E u+E v)(E u-E v)\) is at least equal to \(\delta v\), provided that \((E u+E v)(E u-E v)\) is not equal to zero. Therefore, it is necessary that \((E u+E v)(E u-E v)\) be zero, which gives
\[E u= \pm E v. \quad \text {q. e. d.}\]

It is clear that equation (7) cannot be sustained unless we have 
\[\delta\left(N p_1^2\right)=\delta\left(R_1 q^2\right), \]
that is, 
\[\delta N+2 \delta p_1=\delta R_1+2 \delta q,\]
from which
\[\delta\left(N R_1\right)=2\left(\delta q-\delta p_1+\delta R_1\right).\]
The highest exponent of the function \(R\) must therefore be an even number. Let \(\delta N=n-m\), \(\delta R_1=n+m\).

\subsection*{4.}

This being said, in place of the equation
\[p_1^2 N-q^2 R_1=1,\]
I will propose the following:
\[\tag{8}p_1^2 N-q^2 R_1=v,\]
where \(v\) is an integral function whose degree is less than \(\frac{\delta N+\delta R_1}{2}\).  This equation is clearly more general, and it can be solved by the same process.

Let \(t\) be the integer part of the fractional function \(\frac{R_1}{N}\), and let \(t'\) be the remainder. With this assumption, we have
\[\tag{9}R_1=N t+t',\]
and it is clear that \(t\) must be of degree \(2 m\), when \(\delta N=n-m\) and \(\delta R_1=n+m\). Substituting this expression for \(R_1\) into equation (8), we obtain
\[\tag{10}\left(p_1^2-q^2 t\right) N-q^2 t'=v.\]
%
Now let
\[\tag{11}t=t_1^2+{t_1}'.\]
We can always determine \(t_1\) in such a way that the degree of \(t_1'\) is less than \(m\). To do this, let us write
\[
\begin{aligned}
t&=\alpha_0+\alpha_1 x+\dots+\alpha_{2m} x^{2m}, \\
t_1&=\beta_0+\beta_1 x+\dots+\beta_m x^m, \\
t_1'&=\gamma_0+\gamma_1 x+\dots+\gamma_{m-1} x^{m-1};
\end{aligned}
\]
then equation (11) gives
\[
\begin{gathered}
\alpha_{2m} x^{2m}+\alpha_{2m-1} x^{2m-1}+\alpha_{2m-2} x^{2m-2}+\dots+\alpha_{m-1} x^{m-1}+\dots+\alpha_1 x+\alpha_0 \\
=\beta_m^2 x^{2m}+2\beta_m\beta_{m-1} x^{2m-1}+(\beta_{m-1}^2+2\beta_m\beta_{m-2}) x^{2m-2}+\dots \\
+\gamma_{m-1} x^{m-1}+\gamma_{m-2} x^{m-2}+\dots+\gamma_1 x+\gamma_0.
\end{gathered}
\]
From this equation, we deduce, by comparing the coefficients, that
\[
\begin{aligned}
\alpha_{2m}&=\beta_m^2, \\
\alpha_{2m-1}&=2\beta_m\beta_{m-1}, \\
\alpha_{2m-2}&=2\beta_m\beta_{m-2}+\beta_{m-1}^2, \\
\alpha_{2m-3}&=2\beta_m\beta_{m-3}+2\beta_{m-1}\beta_{m-2}, \\
\alpha_{2m-4}&=2\beta_m\beta_{m-4}+2\beta_{m-1}\beta_{m-3}+\beta_{m-2}^2, \\
\dots&\dots\dots\dots\dots\dots\dots \\
\alpha_m&=2\beta_m\beta_0+2\beta_{m-1}\beta_1+2\beta_{m-2}\beta_2+\dots, \\
\gamma_{m-1}&=\alpha_{m-1}-2\beta_{m-1}\beta_0-2\beta_{m-2}\beta_1-\dots, \\
\gamma_{m-2}&=\alpha_{m-2}-2\beta_{m-2}\beta_0-2\beta_{m-3}\beta_1-\dots, \\
\dots&\dots\dots\dots\dots\dots\dots \\
\gamma_2&=\alpha_2-2\beta_2\beta_0-\beta_1^2, \\
\gamma_1&=\alpha_1-2\beta_1\beta_0, \\
\gamma_0&=\alpha_0-\beta_0^2.
\end{aligned}
\]
It is easy to see that the first \(m+1\) equations always give the values of the \(m+1\) quantities \(\beta_m\), \(\beta_{m-1},\dots,\beta_0\), and the last \(m\) equations give the values of \(\gamma_0\), \(\gamma_1\), \(\gamma_2,\dots,\gamma_{m-1}\). The equation (11) is therefore always possible.
%
By substituting in equation (10), instead of \(t\), its value obtained from equation (11), we have
\[\tag{12}\left(p_1^2-q^2 t_1^2\right) N-q^2\left(N t_1^{\prime}+t^{\prime}\right)=v;\]
from which we obtain
\[\left(\frac{p_1}{q}\right)^2=t_1^2+t_1^{\prime}+\frac{t^{\prime}}{N}+\frac{v}{q^2 N}.\]
Noting that
\[\delta\left(t_1^{\prime}+\frac{t^{\prime}}{N}+\frac{v}{q^2 N}\right)<\delta t_1,\]
we have, by the above,
\[E\left(\frac{p_1}{q}\right)= \pm E t_1= \pm t_1,\]
thus
\[p_1= \pm t_1 q+\beta, \text { where } \delta \beta<\delta q,\]
or, since we can take \(t_1\) with any sign we like,
\[p_1=t_1 q+\beta.\]
By substituting this expression in place of \(p_1\) in equation (12), it becomes
\[\tag{13}\left(\beta^2+2 \beta t_1 q\right) N-q^2 s=v,\]
where, for brevity, we have defined
\[N t_1^{\prime}+t^{\prime}=s.\]
From this equation, it is easy to deduce
\[\left(\frac{q}{\beta}-\frac{t_1 N}{s}\right)^2=\frac{N\left(t_1^2 N+s\right)}{s^2}-\frac{v}{s \beta^2},\]
or, since \(t_1^2 N+s=R_1\) (because \(R_1=t N+t^{\prime}\), \(s=N {t_1}^{\prime}+t^{\prime}\), and \(t=t_1^2+{t_1}^{\prime}\)),
\[\left(\frac{q}{\beta}-\frac{t_1 N}{s}\right)^2=\frac{R_1 N}{s^2}-\frac{v}{s \beta^2}.\]

Now, letting
\[R_1 N=r^2+r' \text {, where } \delta r'<\delta r',\]
we will have
\[\left(\frac{q}{\beta}-\frac{t_1 N}{s}\right)^2=\left(\frac{r}{s}\right)^2+\frac{r'}{s^2}-\frac{v}{s \beta^2}.\]
But we can easily see that
\[\delta\left(\frac{r'}{s^2}-\frac{v}{s \beta^2}\right)<\delta\left(\frac{r}{s}\right),\]
thus
\[E\left(\frac{q}{\beta}-\frac{t_1 N}{s}\right)=E\left(\frac{r}{s}\right),\]
and therefore
\[E\left(\frac{q}{\beta}\right)=E\left(\frac{r+t_1 N}{s}\right);\]
thus, by setting
\[E\left(\frac{r+t_1 N}{s}\right)=2 \mu,\]
we have
\[q=2 \mu \beta+\beta_1 \text {, where } \delta \beta_1<\delta \beta.\]
Substituting this expression for \(q\) in equation (13), we have
\[\beta^2 N+2 \beta t_1 N\left(2 \mu \beta+\beta_1\right)-s\left(4 \mu^2 \beta^2+4 \mu \beta_1 \beta+\beta_1^2\right)=v,\]
that is,
\[\beta^2\left(N+4 \mu t_1 N-4 s \mu^2\right)+2\left(t_1 N-2 \mu s\right) \beta \beta_1-s \beta_1^2=v.\]
For brevity, we define
\[\tag{14}\begin{aligned}
& s_1=N+4 \mu t_1 N-4 s \mu^2, \\
& t_1 N-2 \mu s=-r_1.
\end{aligned}\]
We then obtain
\[\tag{15}s_1 \beta^2-2 r_1 \beta \beta_1-s \beta_1^2=v.\]
Since \(E\left(\frac{r+t_1 N}{s}\right)=2 \mu\), we have
\[r+t_1 N=2 s \mu+\varepsilon \text {, where } \delta \varepsilon<\delta s,\]
thus the last equation of (14) gives
\[r_1=r-\varepsilon.\]

Multiplying the expression of \(s_1\) by \(s\), we obtain
\[s s_1=N s+4 \mu t_1 N s-4 s^2 \mu^2=N s+t_1^2 N^2-\left(2 s \mu-t_1 N\right)^2.\]
Now \(2 s \mu-t_1 N=r_1\), so
\[s s_1=N s+t_1^2 N^2-r_1^2, \text { and } r_1^2+s s_1=N\left(s+t_1^2 N\right);\]
furthermore, we have
\[s+t_1^2 N=R_1,\]
hence
\[\tag{16} r_1^2+s s_1 = N R_1 = R.\]
According to the above, we have \(R=r^2+r^{\prime}\), therefore
\[r^2-r_1^2=s s_1-r^{\prime},\left(r+r_1\right)\left(r-r_1\right)=s s_1-r^{\prime}.\]
Now since \(\delta r^{\prime}<\delta r\), it follows from this equation that
\[\delta\left(s s_1\right)=\delta\left(r+r_1\right)\left(r-r_1\right),\]
that is, since \(r-r_1=\varepsilon\), where \(\delta \varepsilon<\delta r\),
\[\delta s+\delta s_1=\delta r+\delta \varepsilon.\]
Now \(\delta s > \delta \varepsilon\), therefore
\[\delta s_1 < \delta r.\]
We also have \(s=N {t_1}^{\prime}+t^{\prime}\), where \(\delta t^{\prime}<\delta N\) and \({\delta t_1}^{\prime}<\delta t_1\), therefore
\[\delta s< \delta N+\delta t_1.\]
But \(R=N\left(s+t_1^2 N\right)\), consequently
\[\delta R=2 \delta t_1+2 \delta N,\]
and since \(\delta R=2 \delta r=2 \delta r_1\), we have
\[\delta t_1+\delta N=\delta r_1.\]
It follows that
\[\delta s< \delta r_1.\]

The equation \(p_1^2 N-q^2 R_1=v\) has thus been transformed into the following:
\[s_1 \beta^2-2 r_1 \beta \beta_1-s \beta_1^2=v,\]
where
\[\delta r_1=\frac{1}{2} \delta R=n, \quad \delta \beta_1<\delta \beta, \quad \delta s<n, \quad \delta s_1<n.\]
This equation is obtained, as we have just seen, by setting
\[\tag{17}\begin{aligned}
& p_1=t_1 q+\beta, \\
& q=2 \mu \beta+\beta_1,
\end{aligned}\]
with \(t_1\) being determined by the equation
\[t=t_1^2+{t_1}^{\prime}, \text { where } \delta {t_1}^{\prime}<\delta t_1, \quad t=E\left(\frac{R_1}{N}\right),\]
and \(\mu\) by the equation
\[2 \mu=E\left(\frac{r+t_1 N}{s}\right),\]
where
\[r^2+r^{\prime}=R_1 N, \quad s=N t_1^{\prime}+R_1-N t.\]
Furthermore, we have
\[\tag{18}\left\{\begin{array}{l}
r_1=2 \mu s-t_1 N, \\
s_1=N+4 \mu t_1 N-4 s \mu^2, \\
r_1^2+s s_1=R_1 N=R.
\end{array}\right.\]
We now turn our attention to equation (15).
%
\subsection*{5.\\
{\scriptsize \textit{Solving the equation: \(s_1 \beta^2-2 r_1 \beta \beta_1-s \beta_1^2=v\), where \(\delta s<\delta r_1\), \(\delta s_1<\delta r_1\), \(\delta v<\delta r_1\), \(\delta \beta_1<\delta \beta \).}}}
%
Dividing the equation
\[\tag{19}s_1 \beta^2-2 r_1 \beta_1 \beta_1-s \beta_1^2=v,\]
by \(s_1 \beta_1^2\), we obtain
\[\frac{\beta^2}{\beta_1^2}-2 \frac{r_1}{s_1}\frac{\beta}{\beta_1}-\frac{s}{s_1}=\frac{v}{s_1 \beta_1^2},\]
thus
\[\left(\frac{\beta}{\beta_1}-\frac{r_1}{s_1}\right)^2=\left(\frac{r_1}{s_1}\right)^2+\frac{s}{s_1}+\frac{v}{s_1 \beta_1^2}.\]
From this, noting that \(\delta\left(\frac{s}{s_1}+\frac{v}{s_1 \beta_1^2}\right)<\delta\left(\frac{r_1}{s_1}\right),\)
we deduce
\[E\left(\frac{\beta}{\beta_1}-\frac{r_1}{s_1}\right)= \pm E\left(\frac{r_1}{s_1}\right),\]
thus
\[E\left(\frac{\beta}{\beta_1}\right)=E\left(\frac{r_1}{s_1}\right) \cdot(1 \pm 1),\]
where one must take the + sign, as the other sign would give \(E\left(\frac{\beta}{\beta_1}\right)=0\); thus
\[E\left(\frac{\beta}{\beta_1}\right)=2 E\left(\frac{r_1}{s_1}\right),\]
and consequently, setting
\[E\left(\frac{r_1}{s_1}\right)=\mu_1,\]
we will have
\[\beta=2 \beta_1, \quad u_1+\beta_2 \text {, where } \delta \beta_2<\delta \beta_1.\]
Substituting this value of \(\beta\) into the proposed equation, we have
\[s_1\left(\beta_2^2+4 \beta_1 \beta_2 \mu_1+4 \mu_1^2 \beta_1^2\right)-2 r_1 \beta_1\left(\beta_2+2 \mu_1 \beta_1\right)-s \beta_1^2=v,\]
or
\[\tag{20}s_2 \beta_1^2-2 r_2 \beta_1 \beta_2-s_1 \beta_2^2=-v,\]
where
\[r_2=2 \mu_1 s_1-r_1, \quad s_2=s+4 r_1 \mu_1-4 s_1 \mu_1^2.\]

The equation \(E\left(\frac{r_1}{s_1}\right)=\mu_1\) gives
\[r_1=\mu_1 s_1+\varepsilon_1 \text {, where } \delta \varepsilon_1<\delta s_1.\]
From this, we obtain
\[\begin{aligned}
& r_2=r_1-2 \varepsilon_1, \\
& s_2=s+4 \varepsilon_1 \mu_1,
\end{aligned}\]
thus, as is easy to see,
\[\delta r_2=\delta r_1, \quad \delta s_2<\delta r_2.\]
Equation (19) therefore has the same form as equation (20); we can therefore apply the same operation to it.  That is, by setting
\[\mu_2=E\left(\frac{r_2}{s_2}\right), \quad r_2=s_2 \mu_2+\varepsilon_2, \quad \beta_1=2 \mu_2 \beta_2+\beta_3,\]
we will have
\[s_3 \beta_2^{2}-2 r_3 \beta_2 \beta_3-s_2 \beta_3^2=v,\]
where
\[\begin{gathered}
r_3=2 \mu_2 s_2-r_2=r_2-2 \varepsilon_2, \\
s_3=s_1+4 r_2 \mu_2-4 s_2 \mu_2^2=s_1+4 \varepsilon_2 \mu_2, \\
\delta \beta_3<\delta \beta_2.
\end{gathered}\]

Continuing this process, after \(n-1\) transformations, we obtain the following equation:
\[\tag{21}\begin{gathered}
s_n \beta_{n-1}^2-2 r_n \beta_{n-1} \beta_n-s_{n-1} \beta_n^2=(-1)^{n-1} v, \\
\text { where } \delta \beta_n<\delta \beta_{n-1}.
\end{gathered}\]
The quantities \(s_n\), \(r_n\), \(\beta_n\) are determined by the following equations:
\[
\begin{aligned}
\beta_{n-1} & =2 \mu_n \beta_n+\beta_{n+1}, \\
\mu_n & =E\left(\frac{r_n}{s_n}\right), \\
r_n & =2 \mu_{n-1} s_{n-1}-r_{n-1}, \\
s_n & =s_{n-2}+4 r_{n-1} \mu_{n-1}-4 s_{n-1} \mu_{n-1}^2.
\end{aligned}
\]
To these equations, we can add the following ones:
\[
\begin{aligned}
& r_n=\mu_n s_n+\varepsilon_n, \\
& r_n=r_{n-1}-2 \varepsilon_{n-1}, \\
& s_n=s_{n-2}+4 \varepsilon_{n-1} \mu_{n-1}.
\end{aligned}
\]
Now, since the numbers \(\delta \beta\), \(\delta \beta_1\), \(\delta {\beta}_2 \dots \delta {\beta}_n\), etc. form a decreasing series, we must necessarily, after a certain number of transformations, find a \(\beta_n\) which is equal to zero.  Let 
\[\beta_m=0.\]
Then equation (21) will give, by setting \(n=m\),
\[\tag{22} s_m \beta_{m-1}^2=(-1)^{m-1} v.\]

This is the general condition equation for the solvability of equation (19); \(s_m\) depends on the functions \(s\), \(s_1\), \(r_1\), and \(\beta_{m-1}\) must be chosen so as to satisfy the condition
\[\delta s_m+2 \delta \beta_{m-1}<\delta r.\]
Equation (22) shows that for all \(s\), \(s_1\), and \(r_1\), we can find an infinite number of values of \(v\) that satisfy equation (19).

By substituting in the proposed equation, instead of \(v\), its value \((-1)^{m-1} s_m \beta_{m-1}^2\), we obtain
\[s_1 \beta^2-2 r_1 \beta \beta_1-s \beta_1^2=(-1)^{m-1} s_m \beta_{m-1}^2,\]
an equation that is still solvable. We can easily see that \(\beta\) and \(\beta_1\) have the common factor \(\beta_{m-1}\). Therefore, if we suppose that \(\beta\) and \(\beta_1\) have no common factors, \(\beta_{m-1}\) will be independent of \(x\). We can therefore set \(\beta_{m-1}=1\), from which this equation follows,
\[s_1 \beta^2-2 r_1 \beta \beta_1-s \beta_1^2=(-1)^{m-1} s_m.\]

The functions \(\beta\), \(\beta_1\), \(\beta_2, \dots\) are determined by the equation
\[\beta_{n-1}=2 \mu_n \beta_n+\beta_{n+1},\]
by successively setting \(n=1,2,3, \dots, m-1\) and noting that \({\beta}_m=0\). This yields
\[\begin{aligned}
 \beta_{m-2}&=2 \mu_{m-1} \beta_{m-1}, \\
 \beta_{m-3}&=2 \mu_{m-2} \beta_{m-2}+\beta_{m-1}, \\
 \beta_{m-4}&=2 \mu_{m-3} \beta_{m-3}+\beta_{m-2}, \\
 \dots &\dots \dots \dots \dots \\
 \beta_3&=2 \mu_4 \beta_4+\beta_5, \\
 \beta_2&=2 \mu_3 \beta_3+\beta_4,  \\
 \beta_1&=2 \mu_2 \beta_2+\beta_3, \\
 \beta &=2 \mu_1 \beta_1+\beta_2. \\
\end{aligned}\]
These equations yield
\[\begin{aligned}
& \frac{\beta}{\beta_1}=2 \mu_1+\frac{1}{\frac{\beta_1}{\beta_2}}, \\
& \frac{\beta_1}{\beta_2}=2 \mu_2 +\frac{1}{\frac{\beta_2}{\beta_3}}, \\
& \dots \dots \dots \\
& \frac{\beta_{m-3}}{\beta_{m-2}}=2 \mu_{m-2}+\frac{1}{\frac{\beta_{m-2}}{\beta_{m-1}}}, \\
& \frac{\beta_{m-2}}{\beta_{m-1}}=2 \mu_{m-1}.
\end{aligned}\]
These equations can be solved by successive substitutions:
\[\frac{\beta}{\beta_1}=2 \mu_1+\cfrac{1}{2 \mu_2+\cfrac{1}{2 \mu_3+\cfrac{1}{\ddots+\cfrac{1}{2 \mu_{m-2}+\cfrac{1}{2 \mu_{m-1}}.}}}}\]
Hence, we can obtain the values of \(\beta\) and \(\beta_1\) by converting this continued fraction into an ordinary fraction.
%
\subsection*{6.}
%
Substituting for \(v\) its value \((-1)^{m-1} s_m\) in the equation
\[p_1^2 N-q^2 R_1=v,\]
we will have
\[p_1^2 N-q^2 R_1=(-1)^{m-1} s_m,\]
where
\[\begin{aligned}
& q=2 \mu \beta+\beta_1, \\
& p_1=t_1 q+\beta,
\end{aligned}\]
and thus
\[\frac{p_1}{q}=t_1+\frac{\beta}{q}=t_1+\frac{1}{\frac{q}{\beta}};\]
but
\[\frac{q}{\beta}=2\mu+\frac{\beta_1}{\beta};\]
therefore,
\[\frac{p_1}{q}=t_1+\cfrac{1}{2 \mu+\cfrac{1}{2 \mu_1+\cfrac{1}{2 \mu_2+\cfrac{1}{\ddots+\cfrac{1}{2 u_{m-1}}.}}}}\]
%
The equation
\[p_1^2 N-q^2 R_1=v\]
gives
\[\begin{aligned}
& \left(\frac{p_1}{q}\right)^2=\frac{R_1}{N}+\frac{v}{q^2 N}, \\
& \frac{p_1}{q}=\sqrt{\frac{R_1}{N}+\frac{v}{q^2 N}};
\end{aligned}\]
so assuming \(m\) is infinite
\[\frac{p_1}{q}=\sqrt{\frac{R_1}{N}};\]
thus
\[\sqrt{\frac{R_1}{N}}=t_1+\cfrac{1}{2 \mu+\cfrac{1}{2 \mu_1+\cfrac{1}{2 \mu_2+\cfrac{1}{2 \mu_3+\cfrac{1}{\text { etc. }}}}}}\]
We thus find the values of \(p_1\) and \(q\) by transforming the function \(\sqrt{\frac{R_1}{N}}\) into a continued fraction.\footnote{The equation above does not express an absolute equality. It only indicates in an abbreviated way how one can find the quantities \(t_1\), \(\mu\), \(\mu_1\), \(\mu_2 \dots\). If, however, the continued fraction has a value, it will always be equal to \(\sqrt{\frac{R_1}{N}}\). }

\section*{7.}

Now let \(v=a\).  Then we have
\[s_m=(-1)^{m-1} a.\]
Therefore, if the equation
\[p_1^2 N-q^2 R_1=a,\]
is solvable, at least one of the quantities
\[s, s_1, s_2, \dots, s_m \text{, etc.}\]
must be independent of \(x\).

On the other hand, when one of these quantities is independent of \(x\), it is always possible to find two integral functions \(p_1\) and \(q\) that satisfy this equation. Indeed, when \(s_m=a\), we can find the values of \(p_1\) and \(q\) by transforming the continued fraction
\[\frac{p_1}{q}=t_1+\cfrac{1}{2 \mu+\cfrac{1}{2 \mu_1+\cfrac{1}{2 \mu_2+\cfrac{1}{\ddots+\cfrac{1}{2 \mu_{m-1}}}}}}\]
into an ordinary fraction. It is easy to see that the functions \(s\), \(s_1\), \(s_2\), etc., are generally of degree \(n-1\) when \(N R_1\) is of degree \(2 n\). The condition equation
\[s_m=a,\]
will therefore give \(n-1\) equations between the coefficients of the functions \(N\) and \(R_1\); there are therefore only \(n+1\) of these coefficients that can be chosen arbitrarily, the others are determined by the condition equations.

\subsection*{8.}

From the above, it follows that we can find all the values of \(R_1\) and \(N\) that make the differential \(\frac{\varrho dx}{\sqrt{R_1 N}}\) integrable by an expression of the form
\[\log \frac{p+q \sqrt{R_1 N}}{p-q \sqrt{ R_1 N}},\]
by successively setting the quantities \(s\), \(s_1\), \(s_2 \dots s_m\) to be independent of \(x\).
%
Since \(p=p_1 N\), we have similarly,
\[\int \frac{\varrho d x}{\sqrt{R_1 N}}=\log \frac{p_1 \sqrt{ N }+q \sqrt{R_1}}{p_1 \sqrt{ N }-q \sqrt{R_1}};\]
or
\[\tag{23} \left\{ \begin{array}{lc} & \int \frac{\varrho dx}{\sqrt{R_1N}} = \log \frac{y \sqrt{N} + \sqrt{R_1}}{y\sqrt{N} - \sqrt{R}},\\ \text{where} \\ &y = t_1 + \cfrac{1}{2\mu+\cfrac{1}{2\mu_1 + \cfrac{1}{2\mu_2+\cfrac{1}{\ddots + \cfrac{1}{2\mu_{m-1}},}}}} \end{array} \right.\]
assuming \(s_m\) is equal to a constant.
%
The quantities \(R_1\), \(N\), \(p_1\), and \(q\) being thus determined, we find \(\varrho\) by equation (5). This equation gives, by replacing \(p\) with \(p_1 N\) and \(\frac{M}{N}\) with \(\varrho\),
\[\varrho=\left(p_1 \frac{d N}{d x}+2 N \frac{d p_1}{d x}\right): q.\]
It follows that
\[\delta \varrho=\delta p_1+\delta N-1-\delta q=\delta p-\delta q-1.\]
Now we have seen that \(\delta p-\delta q=n\), so
\[\delta \varrho=n-1.\]
Therefore, if the function \(R\) or \(R_1 N\) is of degree \(2n\), the function \(\varrho\) will necessarily be of degree \(n-1\).
%
\subsection*{9.}
%
We have seen earlier that
\[R=R_1 N;\]
but we can always assume that the function \(N\) is constant. Indeed, we have
\[\int \frac{\varrho d x}{\sqrt{R_1 N}}=\log \frac{p_1 \sqrt{N}+q \sqrt{R_1}}{p_1 \sqrt{N}-q \sqrt{R_1}},\]
and therefore,
\[\int \frac{\varrho d x}{\sqrt{R_1 N}}=\frac{1}{2} \log \left(\frac{p_1 \sqrt{N}+q \sqrt{R_1}}{p_1 \sqrt{N}-q \sqrt{R_1}}\right)^2=\frac{1}{2} \log \frac{p_1^2 N+q^2 R_1+2 p_1 q \sqrt{R_1 N}}{p_1^2 N+q^2 R_1-2 p_1 q \sqrt{R_1 N}};\]
or, by letting \(p_1^2 N+q^2 R_1=p^{\prime}\) and \(2 p_1 q=q^{\prime}\),
\[\int \frac{2 \varrho d x}{\sqrt{R}}=\log \frac{p^{\prime}+q^{\prime} \sqrt{R}}{p^{\prime}-q^{\prime} \sqrt{R}}.\]
It is clear that \(p^{\prime}\) and \(q^{\prime}\) have no common factor; therefore, we can always set
\[N=1.\]
Instead of the equation \(p_1^2 N-q_2 R_1=1\), we then have this one,
\[p^{\prime 2}-q^{\prime 2} R=1,\]
from which we obtain the solution by setting \(N=1\) and replacing \(R_1\) with \(R\).
%
When \(N=1\), we see easily that
\[t=R ; \quad t_1=r ; \quad R=r^2+s;\]
thus
\[\tag{24}
\left\{ \begin{array}{l}
 \frac{p^{\prime}}{q^{\prime}}=r+\cfrac{1}{2 \mu+\cfrac{1}{2 \mu_1+\cfrac{1}{2 \mu_2+\cfrac{1}{\ddots+\cfrac{1}{2 \mu_{m-1}},}}}} \\
\begin{array}{ll} R=r^2+s,& \\
 \mu=E\left(\frac{r}{s}\right), & r=s \mu+\varepsilon, \\
 r_1=r-2 \varepsilon,& s_1=1+4 \varepsilon \mu, \\
 \mu_1=E\left(\frac{r_1}{s_1}\right), & r_1=s_1 \mu_1+\varepsilon_1, \\
 r_2=r_1-2 \varepsilon_1, & s_2=s+4 \varepsilon_1, \mu_1 , \end{array}\\
 \dots \dots \dots \dots \dots \dots \dots\\
\begin{array}{ll} \mu_n=E\left(\frac{r_n}{s_n}\right), & r_n=\mu_n s_n+\varepsilon_n, \\
 r_{n+1}=r_n-2 \varepsilon_n, &s_{n+1}=s_{n-1}+4 \varepsilon_n \mu_n , \end{array} \\
 \dots \dots \dots \dots \dots \dots \dots \dots \dots \\
 \begin{array}{ll} \mu_{m-1}=E\left(\frac{r_{m-1}}{s_{m-1}}\right), & r_{m-1}=\mu_{m-1} s_{m-1}+\varepsilon_{m-1}, \\
 r_m=r_{m-1}-2 \varepsilon_{m-1}, & s_m=s_{m-2}+4 \varepsilon_{m-1} \mu_{m-1}=a. \end{array}
\end{array}\right.\]
Having determined the quantities \(R\), \(r\), \(\mu\), \(\mu_1 \dots \mu_{m-1}\) by these equations, we will have
\[\tag{25}\left\{\begin{array}{lc}
&\int \frac{\varrho d x}{\sqrt{R}}=\log \frac{p^{\prime}+q^{\prime} \sqrt{R}}{p^{\prime}-q^{\prime} \sqrt{R}}, \\
\text{where} &\\
&\varrho=\frac{2}{q^{\prime}} \frac{d p^{\prime}}{d x},
\end{array}\right.\]
which results from equation (5) by setting \(N=1\).

\subsection*{10.}

We can give a simpler form to the expression \(\log \frac{p_1 \sqrt{N}+q \sqrt{R_1}}{p_1 \sqrt{N}-q \sqrt{R_1}}\), namely,
\[\begin{aligned}
\log \frac{p_1 \sqrt{N}+q \sqrt{R_1}}{p_1 \sqrt{N}-q \sqrt{R_1}}=&\log \frac{t_1 \sqrt{\Lambda}+\sqrt{R_1}}{t_1 \sqrt{N}-\sqrt{R_1}} +\log \frac{r_1+\sqrt{R}}{r_1-\sqrt{R}} \\
&+\log \frac{r_2+\sqrt{R}}{r_2-\sqrt{R}}+\dots+\log \frac{r_m+\sqrt{R}}{r_m-\sqrt{R}},
\end{aligned}\]
which can be demonstrated as follows. Let
\[\frac{\alpha_m}{\beta_m}=t_1+\cfrac{1}{2 \mu+\cfrac{1}{2 \mu_1+\cfrac{1}{\ddots+\cfrac{1}{2 \mu_{m-1}},}}}\]
we have by the theory of continued fractions,
\begin{align*}
\tag{a} \alpha_m=\alpha_{m-2}+2 \mu_{m-1} \alpha_{m-1}, \\
\tag{b} \beta_m=\beta_{m-2}+2 \mu_{m-1} \beta_{m-1}.
\end{align*}
From these equations, by eliminating \(\mu_{m-1}\), we obtain
\[\alpha_m \beta_{m-1}-\beta_m \alpha_{m-1}=-\left(\alpha_{m-1} \beta_{m-2}-\beta_{m-1} \alpha_{m-2}\right),\]
hence
\[\alpha_m \beta_{m-1}-\beta_m \alpha_{m-1}=(-1)^{m-1},\]
which is known.

The two equations (a) and (b) still give
\[\begin{aligned}
\alpha_m^2&=\alpha_{m-2}^2+4 \alpha_{m-1} \alpha_{m-2} \mu_{m-1}+4 \mu_{m-1}^2 \alpha_{m-1}^2, \\
\beta_m^2&=\beta_{m-2}^2+4 \beta_{m-1} \beta_{m-2} \mu_{m-1}+4 \mu_{m-1}^2 \beta_{m-1}^2.
\end{aligned}
\]
It follows that
\[\begin{aligned}
\alpha_m^2 N-\beta_m^2 R_1=\alpha_{m-2}^2 N-\beta_{m-2}^2 R_1&+4 \mu_{m-1}\left(\alpha_{m-1} \alpha_{m-2} N-\beta_{m-1} \beta_{m-2} R_1\right) \\
&+4 \mu_{m-1}^2\left(\alpha_{m-1}^2 N-\beta_{m-1}^2 R_1\right).
\end{aligned}\]
Now we have
\[\begin{gathered}
\alpha_m^2 N-\beta_m^2 R_1=(-1)^{m-1} s_m, \\
\alpha_{m-1}^2 N-\beta_{m-1}^2 R_1=(-1)^{m-2} s_{m-1}, \\
\alpha_{m-2}^2 N-\beta_{m-2}^2 R_1=(-1)^{m-3} s_{m-2},
\end{gathered}\]
so, by substituting,
\[s_m=s_{m-2}+4(-1)^{m-1} \mu_{m-1}\left(\alpha_{m-1} \alpha_{m-2} N-\beta_{m-1} \beta_{m-2} R_1\right)-4 \mu_{m-1}^2 s_{m-1} .\]
But, according to the above, we have
\[s_m=s_{m-2}+4 \mu_{m-1} r_{m-1}-4 s_{m-1} \mu_{m-1}^2,\]
so
\[r_{m-1}=(-1)^{m-1}\left(\alpha_{m-1} \alpha_{m-2} N-\beta_{m-1} \beta_{m-2} R_1\right).\]

Letting
\[z_m=\alpha_m \sqrt{N}+\beta_m \sqrt{R_1}, \text { and } z_m^{\prime}=\alpha_m \sqrt{N}-\beta_m \sqrt{R_1},\]
we will obtain, by multiplying,
\[z_m z_{m-1}^{\prime}=\alpha_m \alpha_{m-1} N-\beta_m \beta_{m-1} R_1-\left(\alpha_m \beta_{m-1}-\alpha_{m-1} \beta_m\right) \sqrt{N R_1};\]
but as we have just seen
\[\alpha_m \beta_{m-1}-\alpha_{m-1} \beta_m=(-1)^{m-1}, \quad \alpha_m \alpha_{m-1} N-\beta_m \beta_{m-1} R_1=(-1)^m r_m;\]
we conclude from this
\[z_m z_{m-1}^{\prime}=(-1)^m\left(r_m+\sqrt{R}\right),\]
and in the same way,
\[{z_m}^{\prime} z_{m-1}=(-1)^m\left(r_m-\sqrt{R}\right);\]
dividing, we deduce
\[\frac{z_m}{{z_m}^{\prime}} \frac{z_{m-1}^{\prime}}{z_{m-1}}=\frac{r_m+\sqrt{R}}{r_m-\sqrt{R}};\]
or, multiplying by \(\frac{z_{m-1}}{z_{m-1}^{\prime}}\),
\[\frac{z_m}{z_m^{\prime}}=\frac{r_m+\sqrt{R}}{r_m-\sqrt{R}} \frac{z_{m-1}}{z_{m-1}^{\prime}}.\]

By successively taking \(m=1,2,3 \dots m\), we have,
\[\begin{aligned}
&\frac{z_1}{z_1^{\prime}}=\frac{r_1+\sqrt{R}}{r_1-\sqrt{R}} \frac{z_0}{z_0^{\prime}} \\
&\frac{z_2}{z_2^{\prime}}=\frac{r_2+\sqrt{R}}{r_2-\sqrt{R}} \frac{z_1}{z_1^{\prime}} \\
&\dots \dots \dots \dots \dots \\
&\frac{z_m}{z_m^{\prime}}=\frac{r_m+\sqrt{R}}{r_m-\sqrt{R}} \frac{z_{m-1}}{z_{m-1}^{\prime}}
\end{aligned}\]
from which we deduce,
\[\frac{z_m}{z_{m^{\prime}}}=\frac{z_0}{z_0{ }^{\prime}} \frac{r_1+\sqrt{R}}{r_1-\sqrt{R}} \frac{r_2+\sqrt{R}}{r_2-\sqrt{R}} \frac{r_3+\sqrt{R}}{r_3-\sqrt{R}} \dots \frac{r_m+\sqrt{R}}{r_m-\sqrt{R}}.\]
Now we have
\[\begin{aligned}
& z_0=\alpha_0 \sqrt{N}+\beta_0 \sqrt{R_1}=t_1 \sqrt{N}+\sqrt{R_1}, \\
& z_0^{\prime}=\alpha_0 \sqrt{N}-\beta_0 \sqrt{R_1}=t_1 \sqrt{N}-\sqrt{R_1},
\end{aligned}\]
and
\[\frac{z_m}{{z_m}^{\prime}}=\frac{\alpha_m \sqrt{N}+\beta_m \sqrt{R_1}}{\alpha_m \sqrt{N}-\beta_m \sqrt{R_1}},\]
thus
\[\frac{\alpha_m \sqrt{N}+\beta_m \sqrt{R_1}}{\alpha_m \sqrt{N}-\beta_m \sqrt{R_1}}=\frac{t_1 \sqrt{N}+\sqrt{R_1}}{t_1 \sqrt{N}-\sqrt{R_1}} \cdot \frac{r_1+\sqrt{R}}{r_1-\sqrt{R}} \cdot \frac{r_2+\sqrt{R}}{r_2-\sqrt{R}} \cdots \frac{r_m+\sqrt{R}}{r_m-\sqrt{R}},\]
and by taking logarithms
\begin{gather*}
\tag{26}\log \frac{\alpha_m \sqrt{N}+\beta_m \sqrt{R_1}}{\alpha_m \sqrt{N}-\beta_m \sqrt{R_1}} \\
=\log \frac{t_1 \sqrt{N}+\sqrt{R_1}}{t_1 \sqrt{N}-\sqrt{R_1}}+\log \frac{r_1+\sqrt{R}}{r_1-\sqrt{R}}+\log \frac{r_2+\sqrt{R}}{r_2-\sqrt{R}}+\dots+\log \frac{r_m+\sqrt{R}}{r_m-\sqrt{R}},
\end{gather*}
which is what we needed to prove.

\subsection*{11.}

Differentiating the expression \(z=\log \frac{\alpha_m \sqrt{N}+\beta_m \sqrt{R_1}}{\alpha_m \sqrt{N}-\beta_m \sqrt{R_1}}\), we have, after suitable simplifications,
\[d z=\frac{2\left(\alpha_m d \beta_m-\beta_m d \alpha_m\right) N R_1-\alpha_m \beta_m\left(R_1 d N-N d R_1\right)}{\left(\alpha_m^2 N-\beta_m^2 R_1\right) \sqrt{N R_1}}.\]
Now we also have
\[\alpha_m^2 N-\beta_m^2 R_1=(-1)^{m-1} s_m,\]
thus by setting
\[\tag{27}(-1)^{m-1} \varrho_m=2\left(\alpha_m \frac{d \beta_m}{d x}-\beta_m \frac{d \alpha_m}{d x}\right) N R_1-\alpha_m \beta_m\left(\frac{R_1 d N-N d R_1}{d x}\right),\]
we have
\[d z=\frac{\varrho_m}{s_m} \frac{d x}{\sqrt{N R_1}},\]
and
\[z=\int \frac{\varrho_m}{s_m} \frac{d x}{\sqrt{ N R_1}},\]
thus
\[\int \frac{\varrho_m}{s_m} \frac{d x}{\sqrt{N R_1}}=\log \frac{\alpha_m \sqrt{N}+\beta_m \sqrt{R_1}}{\alpha_m \sqrt{N}-\beta_m \sqrt{ R_1}},\]
or equivalently
\[\tag{28}\int \frac{\varrho_m}{s_m} \frac{d x}{\sqrt{R}}=\log \frac{t_1 \sqrt{N}+\sqrt{R_1}}{t_1 \sqrt{N}-\sqrt{R_1}}+\log \frac{r_1+\sqrt{R}}{r_1-\sqrt{ R}}+\dots+\log \frac{r_m+\sqrt{R}}{r_m-\sqrt{ R}}.\]

In this expression \(s_m\) is at most of degree \((n-1)\) and \(\varrho_m\) is necessarily of degree \((n-1+\delta s_m)\), which can be seen as follows. 
Differentiating the equation 
\[\tag{29}\alpha_m^2 N-\beta_m^2 R_1=(-1)^{m-1} s_m,\]
we obtain the following:
\[2 \alpha_m d \alpha_m N+\alpha_m^2 d N-2 \beta_m d \beta_m R_1-\beta_m^2 d R_1=(-1)^{m-1} d s_m,\]
or, multiplying by \(\alpha_m N\):
\[\alpha_m^2 N\left(2 N d \alpha_m+\alpha_m d N\right)-2 \alpha_m \beta_m d \beta_m N R_1-\beta_m^2 \alpha_m N d R_1=(-1)^{m-1} \alpha_m N d s_m.\]
Substituting the value of \(\alpha_m^2 N\) from equation (29), we have
\[\begin{aligned}
(-1)^{m-1} s_m\left(2 N d \alpha_m+\alpha_m d N\right) +\beta_m [&2 N R_1 \beta_m d \alpha_m+\alpha_m \beta_m R_1 d N \\
&-2 \alpha_m d \beta_m N R_1 -\beta_m \alpha_m N d R_1]
=(-1)^{m-1} \alpha_m N d s_m,
\end{aligned}\]
that is,
\[\begin{aligned}
\beta_m\left[2\left(\alpha_m d \beta_m-\beta_m d \alpha_m\right) N R_1-\alpha_m \beta_m\left(R_1 d N-N d R_1\right)\right] &\\
=(-1)^{m-1}\left[s_m\left(2 N d \alpha_m+\alpha_m d N\right)-\alpha_m N d s_m\right]&.
\end{aligned}\]
By equation (27), the left-hand side of this equation is equal to \(\beta_m(-1)^{m-1} \varrho_m d x\); therefore, we have
\[\tag{30} \beta_m \varrho_m=s_m\left(\frac{2 N d \alpha_m}{d x}+\frac{\alpha_m d N}{d x}\right)-\alpha_m \frac{N d s_m}{d x}.\]
Since \(\delta s_m<n\), the right-hand side of this equation is necessarily of degree \(\left(\delta s_m+\delta N+\delta \alpha_m-1\right)\), as can be easily seen; hence,
\[\delta \varrho_m=\delta s_m+\delta N+\delta \alpha_m-\delta \beta_m-1.\]
From equation (29), it follows that
\[2 \delta \alpha_m+\delta N=2 \delta \beta_m+\delta R_1,\]
so
\[\delta \varrho_m=\delta s_m+\frac{\delta N+\delta R_1}{2}-1;\]
or, since \(\delta N+\delta R_1=2 n\),
\[\delta \varrho_m=\delta s_m+n-1,\]
which means that \(\varrho_m\) is necessarily of degree \(\left(\delta s_m+n-1\right)\). From there, it follows that the function \(\frac{\varrho_m}{s_m}\) is of degree \((n-1)\).

Setting \(N=1\) in formula (28), we have \(t_1=r\), and consequently
\[\tag{31}\int \frac{\varrho_m d x}{s_m \sqrt{R}}=\log \frac{r+\sqrt{R}}{r-\sqrt{R}}+\log \frac{r_1+\sqrt{R}}{r_1-\sqrt{R}}+\dots+\log \frac{r_m+\sqrt{R}}{r_m-\sqrt{R}},\]
where, following equation (30),
\[\beta_m \varrho_m=2 s_m \frac{d \alpha_m}{d x}-\alpha_m \frac{d s_m}{d x}.\]

Equation (28) yields, by setting \( s_m = a \),
\begin{gather*}
\tag{32} \int \frac{\varrho_m dx}{a \sqrt{R}} = \log \frac{t_1 \sqrt{N} + \sqrt{R_1}}{t_1 \sqrt{N} - \sqrt{R_1}} + \log \frac{r_1 + \sqrt{R}}{r_1 - \sqrt{R}} + \dots + \log \frac{r_m + \sqrt{R}}{r_m - \sqrt{R}} \\
\text{where } \beta_m \varrho_m = a \left( 2N \frac{d \alpha_m}{d x} + \alpha_m \frac{dN}{dx} \right),
\end{gather*}
and when \( N = 1 \),
\begin{gather*}
\tag{33} \int \frac{\varrho_m dx}{\sqrt{R}} = \log \frac{r + \sqrt{R}}{r - \sqrt{R}} + \log \frac{r_1 + \sqrt{R}}{r_1 - \sqrt{R}} + \dots + \log \frac{r_m + \sqrt{R}}{r_m - \sqrt{R}} \\
\text{where } \varrho_m = \frac{2}{\beta_m} \frac{da_m}{dx}.
\end{gather*}

From the above, this formula has the same generality as formula (32), and provides all integrals of the form \(\int \frac{\varrho d x}{\sqrt{R}}\), where \(\varrho\) and \(R\) are integral functions, which can be expressed by a logarithmic function of the form \(\log \frac{p+q \sqrt{R}}{p-q \sqrt{R}}\).

\subsection*{12.}

In equation (28), the function \(\frac{\varrho_m}{s_m}\) is given by equation (30). However, we can express this function in a more convenient way using the quantities \(t_1\), \(r_1\), \(r_2\), etc. \(\mu\), \(\mu_1\), \(\mu_2\), etc. Indeed, letting
\[z_m=\log \frac{r_m+\sqrt{R}}{r_m-\sqrt{R}}\]
and differentiation, we will have 
\[d z_m=\frac{d r_m+\frac{1}{2} \frac{d R}{\sqrt{R}}}{r_m+\sqrt{R}}-\frac{d r_m-\frac{1}{2} \frac{d R}{\sqrt{R}}}{r_m-\sqrt{R}},\]
or after simplifying,
\[\tag{33'}d z_m=\frac{r_m d R-2 R d r_m}{r_m^2-R} \frac{1}{\sqrt{R}}.\]
Now we have found earlier that
\[s_m=s_{m-2}+4 \mu_{m-1} r_{m-1}-4 s_{m-1} \mu_{m-1}^2,\]
so, multiplying by \(s_{m-1}\),
\[s_m s_{m-1}=s_{m-1} s_{m-2}+4 \mu_{m-1} s_{m-1} r_{m-1}-4 s_{m-1}^2 \mu_{m-1}^2,\]
that is,
\[s_m s_{m-1}=s_{m-1} s_{m-2}+r_{m-1}^2-\left(2 s_{m-1} \mu_{m-1}-r_{m-1}\right)^2.\]
But we have
\[r_m=2 s_{m-1} \mu_{m-1}-r_{m-1},\]
and by substituting this quantity,
\[s_m s_{m-1}=s_{m-1} s_{m-2}+r_{m-1}^2-r_m^2,\]
from which we deduce by transposition,
\[r_m^2+s_m s_{m-1}=r_{m-1}^2+s_{m-1} s_{m-2}.\]

It follows from this equation that \(r_m^2+s_m s_{m-1}\) has the same value for all \(m\) and therefore
\[r_m^2+s_m s_{m-1}=r_1^2+s s_1;\]
now we have seen above that \(r_1^2+s s_1=R\), and so it follows that
\[\tag{34} R=r_m^2+s_m s_{m-1}.\]
Substituting this expression for \(R\) in equation \(\left(33^{\prime}\right)\), we will have after suitable reductions
\[d z_m=\frac{2 d r_m}{\sqrt{R}}-\frac{d s_m}{s_m} \frac{r_m}{\sqrt{R}}-\frac{d s_{m-1}}{s_{m-1}} \frac{r_m}{\sqrt{R}};\]
but since \(r_m=2 s_{m-1} \mu_{m-1}-r_{m-1}\), the term \(-\frac{d s_{m-1}}{s_{m-1}} \frac{r_m}{\sqrt{R}}\) gets transformed into \(-2 \mu_{m-1} \frac{d s_{m-1}}{\sqrt{R}}+\frac{d s_{m-1}}{s_{m-1}} \frac{r_{m-1}}{\sqrt{R}}\). We obtain therefore
\[d z_m=\left(2 d r_m-2 \mu_{m-1} d s_{m-1}\right) \frac{1}{\sqrt{R}}-\frac{d s_m}{s_m} \frac{r_m}{\sqrt{R}}+\frac{d s_{m-1}}{s_{m-1}} \frac{r_{m-1}}{\sqrt{R}},\]
and upon integrating
\[\tag{35}\int \frac{d s_m}{s_m} \frac{r_m}{\sqrt{R}}=-z_m+\int\left(2 d r_m-2 \mu_{m-1} d s_{m-1}\right) \frac{1}{\sqrt{R}}+\int \frac{d s_{m-1}}{s_{m-1}} \frac{r_{m-1}}{\sqrt{R}}.\]

This expression is, as one can see, a reduction formula for integrals of the form \(\int \frac{d s_m}{s_m} \frac{r_m}{\sqrt{R}}\). Indeed, it gives the integral \(\int \frac{d s_m}{s_m} \frac{r_m}{\sqrt{R}}\) in terms of another integral of the same form and an integral of the form \(\int \frac{t d x}{\sqrt{R}}\) where \(t\) is an integral function. By substituting \(m\) successively with \(m\), \(m-1\), \(m-2 \dots 3\), \(2\), \(1\), we obtain \(m\) similar equations, the sum of which gives the following formula (noting that \(r_0=2 s\mu-r_1=t_1 N\) according to the equation \(r_1+t_1 N=2 s \mu\))
\[\begin{aligned}
\int \frac{d s_m}{s_m} \sqrt{n}=&-\left(z_1+z_2+z_3+\dots+z_m\right)+\int \frac{d s}{s} \frac{t_1 N}{\sqrt{R}} \\
&+\int 2\left(d r_1+d r_2+\dots+d r_m+\mu d s-\mu_1 d s_1-\dots-\mu_{m-1} d s_{m-1}\right) \frac{1}{\sqrt{R}}.
\end{aligned}\]

We can further reduce the integral \(\int \frac{ds}{s} \sqrt{N}\). By differentiating the expression
\[z=\log \frac{t_1 \sqrt{N}+\sqrt{R_1}}{t_1 \sqrt{N}-\sqrt{R_1}},\]
after a few reductions we obtain
\[d z=\frac{-2 d t_1 N R_1-t_1\left(R_1 d N-N d R_1\right)}{\left(t_1^2 N-R_1\right) \sqrt{R}}.\]
Now we have
\[R_1=t_1^2 N+s;\]
substituting this value of \(R_1\) into the above equation, we find
\[d z=\left(2 N d t_1+t_1 d N\right) \frac{1}{\sqrt{R}}-\frac{d s}{s} \frac{t_1 N}{\sqrt{R}},\]
thus by integrating we get
\[\int \frac{d s}{s} \frac{t_1 N}{\sqrt{R}}=-z+\int\left(2 N d t_1+t_1 d N\right) \frac{1}{\sqrt{R}}.\]
The expression for \(\int \frac{d s_m}{s_m} \frac{r_m}{\sqrt{R}}\) is transformed in this way into the following,
\[\begin{gathered}
\int \frac{d s_{m}}{s_m} \frac{r_m}{\sqrt{R}}=-\left(z+z_1+z_2+\dots+z_m\right) \\
+\int \frac{2}{\sqrt{R}}\left(N d t_1+\frac{1}{2} t_1 d N+d r_1+\dots+d r_m-\mu d s-\mu_1 d s_1-\dots-\mu_{m-1} d s_{m-1}\right),
\end{gathered}\]
or, by substituting the values of \(z, z_1, z_2, \dots\),
\[\tag{36}
\begin{gathered}
 \int\frac{ds_m}{s_m} \frac{r_m}{\sqrt{R}} \\
 =\int \frac{2}{\sqrt{R}}\left(N d t_1+\frac{1}{2} t_1 d N+d r_1+\dots+d r_m+\mu d s-\mu_1 d s_1-\dots-\mu_{m-1} d s_{m-1}\right) \\
 -\log \frac{t_1 \sqrt{N}+\sqrt{R_1}}{t_1 \sqrt{N}-\sqrt{R_1}}-\log \frac{r_1+\sqrt{R}}{r_1-\sqrt{R}}- \log \frac{r_2+\sqrt{R}}{r_2-\sqrt{R}}-\dots-\log \frac{r_m+\sqrt{R}}{r_m-\sqrt{R}}.
\end{gathered}\]
This formula is exactly the same as formula (28); it yields
\[\tag{37}\begin{gathered}
\frac{\varrho_m}{s_m} d x=-\frac{r_m d s_m}{s_m} \\
+2\left(N d t_1+\frac{1}{2} t_1 d N+d r_1+\dots+d r_m-\mu d s-\dots-\mu_{m-1} d s_{m-1}\right).
\end{gathered}\]
But the above expression eliminates the need for computing the functions \(\alpha_m\) and \(\beta_m\).

If now \(s_m\) is independent of \(x\), the integral \(\int \frac{d s_m}{s_m} \frac{r_m}{\sqrt{R}}\) disappears and we obtain the following formula:
\[\tag{38} \begin{gathered}
 \int \frac{2}{\sqrt{R}}\left(\frac{1}{2} t_1 d N+N d t_1+d r_1+\dots+d r_m-\mu d s-\dots-\mu_{m-1} d s_{m-1}\right) \\
=  \log \frac{t_1 \sqrt{N}+\sqrt{R_1}}{t_1 \sqrt{N}-\sqrt{R_1}}+\log \frac{r_1+\sqrt{R}}{r_1-\sqrt{R}}+\log \frac{r_2+\sqrt{R}}{r_2-\sqrt{R}}+\dots+\log \frac{r_m+\sqrt{R}}{r_m-\sqrt{R}}.
\end{gathered}\]
If in the expression (36) we set \(N=1\), then we have \(t_1=r\), and consequently
\[\tag{39}\begin{aligned}
\int \frac{d s_m}{s_m} \frac{r_m}{\sqrt{R}} = \int \frac{2}{\sqrt{R}}\left(d r+d r_1+\dots+d r_m-\mu d s-\dots-\mu_{m-1} d s_{m-1}\right) \\
-\log \frac{r+\sqrt{R}}{r-\sqrt{R}}-\log \frac{r_1+\sqrt{R}}{r_1-\sqrt{R}}-\dots-\log \frac{r_m+\sqrt{R}}{r_m-\sqrt{R}}
\end{aligned}\]
and if we set \(s_m=a\):
\[\tag{40}\begin{gathered}
\int \frac{2}{\sqrt{{R}}}\left(d r+d r_1+\dots+d r_m-\mu d s-\mu_1 d s_1-\dots-\mu_{m-1} d s_{m-1}\right) \\
=\log \frac{r+\sqrt{R}}{r-\sqrt{R}}+\log \frac{r_1+\sqrt{R}}{r_1-\sqrt{R}}+\dots+\log \frac{r_m+\sqrt{R}}{r_m-\sqrt{R}}.
\end{gathered}\]

According to the above, this formula has the same generality as (38); therefore, it gives all integrals of the form \(\int \frac{t d x}{\sqrt{R}}\), where \(t\) is an integral function, which can be expressed by a function of the form \(\log \frac{p+q \sqrt{R}}{p-q \sqrt{R}}\).

\subsection*{13.}

We have seen above that
\[\sqrt{\frac{R_1}{N}}=t_1+\cfrac{1}{2 \mu+\cfrac{1}{2 \mu_1+\cfrac{1}{2 \mu_2+\cfrac{1}{2 \mu_3+\cfrac{1}{\ddots}}}}}\]
so, when \(N=1\),
\[\sqrt{R}=r+\cfrac{1}{2 \mu+\cfrac{1}{2 \mu_1+\cfrac{1}{2 \mu_2+\cfrac{1}{2 \mu_3+\cfrac{1}{\ddots}}}}}\]
In general, the quantities \(\mu\), \(\mu_1\), \(\mu_2\), \(\mu_3 \dots\) are different from each other. But when one of the quantities \(s\), \(s_1\), \(s_2 \dots\) is independent of \(x\), the continued fraction becomes \textit{periodic}. This can be demonstrated as follows.

We have
\[r_{m+1}^2+s_m s_{m+1}=R=r^2+s,\]
therefore, when \(s_m=a\),
\[r_{m+1}^2-r^2=s-a s_{m+1}=\left(r_{m+1}+r\right)\left(r_{m+1}-r\right ).\]
Now \(\delta r_{m+1}=\delta r\), \(\delta s<\delta r\), \(\delta s_{m+1}<\delta r\), so this equation cannot be sustained unless we also have
\[r_{m+1}=r, \quad s_{m+1}=\frac{s}{a}.\]
Now, since \(\mu_{m+1}=E\left(\frac{r_{m+1}}{s_{m+1}}\right)\) we likewise have
\[\mu_{m+1}=a E\left(\frac{r}{s}\right);\]
but \(E\left(\frac{r}{s}\right)=\mu\), so
\[\mu_{m+1}=a \mu .\]

We have further
\[s_{m+2}=s_m+4 \mu_{m+1} r_{m+1}-4 \mu_{m+1}^2 s_{m+1},\]
thus assuming \(s_m=a\), \(r_{m+1}=r\), \(\mu_{m+1}=a \mu\) we deduce
\[s_{m+2}=a\left(1+4 \mu r-4 \mu^2 s\right);\]
but \(s_1=1+4 \mu r-4 \mu^2 s\), thus
\[s_{m+2}=a s_1.\]

We similarly have
\[r_{m+2}=2 \mu_{m+1} s_{m+1}-r_{m+1}=2 \mu s-r,\]
so, since \(r_1=2 \mu s-r\),
\[r_{m+2}=r_1,\]
from which we deduce
\[\mu_{m+2}=E\left(\frac{r_{m+2}}{s_{m+2}}\right)=\frac{1}{a} E\left(\frac{r_1}{s_1}\right),\]
thus
\[\mu_{m+2}=\frac{\mu_1}{a}.\]

Continuing this process, we can easily see that in general we will have
\[\tag{41}\left\{\begin{array}{l} r_{m+n}=r_{n-1}, \quad s_{m+n}=a^{\pm1} s_{n-1}, \\
\mu_{m+n}=a^{\mp 1} \mu_{n-1}. \end{array}\right.\]
The upper sign should be taken when \(n\) is even and the lower sign in the contrary case.

Putting into the equation
\[r_m^2+s_{m-1} s_m=r^2+s\]
\(a\) in place of \(s_m\), we have
\[\left(r_m-r\right)\left(r_m+r\right)=s-a s_{m-1}.\]
It follows that
\[ r_m=r, \quad s_{m-1}=\frac{s}{r}.\]
Now we have \(\mu_m=E\left(\frac{r_m}{s_m}\right)\), so
\[\mu_m=\frac{1}{a} E r;\]
that is
\[\mu_m=\frac{1}{a} r.\]
We furthermore have
\[r_m+r_{m-1}=2 s_{m-1} \mu_{m-1},\]
that is, since \(r_m=r\), \(s_{m-1}=\frac{s}{a}\),
\[r+r_{m-1}=\frac{2 s}{a} \mu_{m-1}.\]
But \(r+r_1=2 s \mu\), so
\[r_{m-1}-r_1=\frac{2 s}{a}\left(\mu_{m-1}-a \mu\right).\]
We have
\[r_{m-1}^2+s_{m-1} s_{m-2}=r_1^2+s s_1,\]
that is, since \(s_{m-1}=\frac{s}{a}\),
\[\left(r_{m-1}+r_1\right)\left(r_{m-1}-r_1\right)=\frac{s}{a}\left(a s_1-s_{m-2}\right).\]
But we have seen that
\[r_{m-1}-r_1=\frac{2 s}{a}\left(\mu_{m-1}-a \mu\right),\]
so by substituting,
\[2\left(r_{m-1}+r_1\right)\left(\mu_{m-1}-a \mu\right)=a s_1-s_{m-2}.\]
Noting that \(\delta^{\prime}\left(r_{m-1}+r_1\right)>\delta\left(a s_1-s_{m-2}\right)\), this equation gives
\[\mu_{m-1}=a \mu, s_{m-2}=a s_1,\]
and consequently,
\[r_{m-1}=r_1.\]

By a similar process we will easily find
\[r_{m-2}=r_{2,}, \quad s_{m-3}=\frac{1}{a} s_2, \quad \mu_{m-2}=\frac{\mu_1}{a},\]
and in general
\[\tag{42}\left\{\begin{array}{l}r_{m-n}=r_n, \quad s_{m-n}=a^{ \pm 1} s_{n-1}, \\
\mu_{m-n}=a^{\mp 1} \mu_{n-1}.\end{array}\right.\]

\subsection*{14.}

\begin{center}A. Let \(m\) be an even number, \(2k\).\end{center}
In this case, we easily see, by virtue of equations (41) and (42), that the quantities $r$, $r_1$, $r_2 \dots s$, $s_1$, $s_2 \dots, \mu$, $\mu_1$, $\mu_2 \dots$ form the following series:

{\setlength\arraycolsep{0.25em}\[\begin{array}{cccccccccccc|ccc}
0&1&..&2k-2&2k-1&2k&2k+1&2k+2&..&4k-1&4k&4k+1&4k+2&4k+3&\text{etc.}\\
r&r_1&..&r_2&r_1&r&r&r_1&..&r_2&r_1&r&r&r_1&\text{etc.}\\
s&s_1&..&as_1&\frac{s}{a}&a&\frac{s}{a}&as_1&..&s_1&s&1&s&s_1&\text{etc.}\\
\mu&\mu_1&..&\frac{\mu_1}{a}&a \mu & \frac{r}{a}&a\mu& \frac{\mu_1}{a}&..&\mu_1&\mu&r&\mu&\mu_1&\text{etc.}
\end{array}\]}

\begin{center}B. Let \(m\) be an odd number, \(2k-1\).\end{center}
In this case the equation
\[s_{m-n}=a^{ \pm 1} s_{n-1} \text { or } s_{2 k-n-1}=a^{ \pm 1} s_{n-1}\]
gives, for \(n=k\),
\[s_{k-1}=a^{ \pm 1} s_{k-1}, \text { donc } a=1.\]
The quantities \(r\), \(r_1\) etc. \(s\), \(s_1\) etc.,\(\|,\|_1\) etc. form the following sequences:
\[\begin{array}{cccccccccc|ccc}
0 & 1 & .. & k-2 & k-1 & k & k+1 & .. & 2 k-2 & 2 k-1 & 2 k & 2 k+1 & \text { etc. } \\
r & r_1 & .. & r_{k-2} & r_{k-1} & r_{k-1} & r_{k-2} & .. & r_1 & r & r & r_1 & \text { etc. } \\
s & s_1 & .. & s_{k-2} & s_{k-1} & s_{k-2} & s_{k-3} & .. & s & 1 & s & s_1 & \text { etc. } \\
\mu & \mu_1 & .. & \mu_{k-2} & \mu_{k-1} & \mu_{k-2} & \mu_{k-3} & .. & \mu & r & \mu & \mu_1 & \text { etc. }
\end{array}\]

We see here that, when one of the quantities \(s\), \(s_1\), \(s_2 \dots\) is independent of \(x\), the continued fraction resulting from \(\sqrt{R}\) is always periodic and of the following form, when \(s_m=a\):
\[\sqrt{R} = r+\cfrac{1}{2 \mu+\cfrac{1}{2 \mu_1+\cfrac{1}{\ddots \cfrac{1}{\frac{2\mu_1}{a}+\cfrac{1}{2 a \mu+\cfrac{1}{\frac{2 r}{a}+\cfrac{1}{2 a \mu+\cfrac{1}{\frac{2 \mu_1}{a}+\cfrac{1}{\ddots+\cfrac{1}{2 \mu+\cfrac{1}{2 r+\cfrac{1}{2 \mu+\cfrac{1}{\ddots}}}}}}}}}}}}}\]
When \(m\) is odd, we moreover have \(a=1\), and therefore
\[\sqrt{R}=r+\cfrac{1}{2 \mu+\cfrac{1}{2 \mu_1+\cfrac{1}{\ddots+\cfrac{1}{2 \mu_1+\cfrac{1}{2 \mu+\cfrac{1}{2 r+\cfrac{1}{2 \mu+\cfrac{1}{2 \mu_1+\cfrac{1}{\ddots}}}}}}}}}\]

The converse also holds; that is, when the continued fraction resulting from \(\sqrt{R}\) has the above form, \(s_m\) will be independent of \(x\). Indeed, let 
\[\mu_m=\frac{r}{a},\]
then from the equation \(r_m=s_m \mu_m+\varepsilon_m\), we obtain 
\[r_m=\frac{r}{a} s_m+\varepsilon_m.\]
Now, since \(r_m=r_{m-1}-2 \varepsilon_{m-1}\), where \(\delta \varepsilon_{m-1} < \delta r\), it is clear that
\[r_m=r+\gamma_m,\text{ where }\delta \gamma_m<\delta r.\]
Hence, we obtain 
\[r(1-\frac{s_m}{a})=\varepsilon_m-\gamma_m,\] 
and consequently \(s_m=a\), which was to be demonstrated. Combining this with the above, we find the following proposition:

\begin{quote}"When it is possible to find an integral function \(\varrho\) such that
\[\int \frac{\varrho dx}{\sqrt{R}}=\log \frac{y+\sqrt{R}}{y-\sqrt{R}},\]
the continued fraction resulting from \(\sqrt{R}\) is periodic, and has the following form:
\[\sqrt{R}=r+\cfrac{1}{2 \mu+\cfrac{1}{2 \mu_1+\cfrac{1}{\ddots +\cfrac{1}{2 \mu_1+\cfrac{1}{2 \mu+\cfrac{1}{2 r+\cfrac{1}{2 \mu +\cfrac{1}{2 \mu_1+\text {etc.}}}}}}}}}\]
and conversely, when the continued fraction resulting from \(\sqrt{R}\) has this form, it is always possible to find an integral function\(\varrho\) that satisfies the equation,
\[\int \frac{\varrho dx}{\sqrt{R}}=\log \frac{y+\sqrt{R}}{y-\sqrt{R}}.\]
The function \(y\) is given by the following expression:
\[y=r+\cfrac{1}{2 \mu+\cfrac{1}{2 \mu_1+\cfrac{1}{2 \mu_2+\cfrac{1}{\ddots +\cfrac{1}{2 \mu+\cfrac{1}{2 r}."}}}}}\]\end{quote}

This proposition provides the complete solution of the problem proposed at the beginning of this paper.

\subsection*{15.}

We have just seen that, when \(s_{2 k-1}\) is independent of \(x\), we always have \(s_k=s_{k-2}\), and when \(s_{2 k}\) is independent of \(x\), we have \(s_k=c s_{k-1}\), where \(c\) is a constant. The converse also holds, which we can prove as follows.

I. First, letting \(s_k=s_{k-2}\), we have
\[r_{k-1}^2+s_{k-1} s_{k-2}=r_k^2+s_k s_{k-1};\]
but \(s_k=s_{k-2}\), so
\[r_k=r_{k-1}.\]
Furthermore,
\[\begin{gathered}
r_k=\mu_k s_k+\varepsilon_k, \\
r_{k-2}=\mu_{k-2} s_{k-2}+\varepsilon_{k-2},
\end{gathered}\]
and thus
\[r_k-r_{k-2}=s_k\left(\mu_k-\mu_{k-2}\right)+\varepsilon_k-\varepsilon_{k-2}.\]
But
\[r_k=r_{k-1}, \quad r_{k-2}=r_{k-1}+2 \varepsilon_{k-2},\]
and hence, by substitution, we find
\[0=s_k\left(\mu_k-\mu_{k-2}\right)+\varepsilon_k+\varepsilon_{k-2}.\]
Noting that \(\delta \varepsilon_k<\delta s_k, \quad \delta \varepsilon_{k-2}<\delta s_{k-2}\), this equation yields
\[\mu_k=\mu_{k-2}, \quad \varepsilon_k=-\varepsilon_{k-2}.\]
Now \(r_{k+1}=r_k-2 \varepsilon_k\), and hence, according to this last equation,
\[r_{k+1}=r_{k-1}+2 \varepsilon_{k-2}.\]
Since \(r_{k-1}=r_{k-2}-2 \varepsilon_{k-2}\), we conclude that
\[r_{k+1}=r_{k-2}.\]

We have
\[r_{k+1}^2+s_k s_{k+1}=r_{k-2}^2+s_{k-2} s_{k-3},\]
therefore, since \(r_{k+1}=r_{k-2}\), \(s_k=s_{k-2}\), we also have
\[s_{k+1}=s_{k-3}.\]

Combining this equation with the following ones,
\[r_{k+1}=\mu_{k+1} s_{k+1}+\varepsilon_{k+1}, \quad r_{k-3}=\mu_{k-3} s_{k-3}+\varepsilon_{k-3},\]
we obtain
\[r_{k+1}-r_{k-3}=s_{k+1}\left(\mu_{k+1}-\mu_{k-3}\right)+\varepsilon_{k+1}-\varepsilon_{k-3}.\]
Now we have \(r_{k+1}=r_{k-2}\), and \(r_{k-2}=r_{k-3}-2 \varepsilon_{k-3}\), and therefore
\[0=s_{k+1}\left(\mu_{k+1}-\mu_{k-3}\right)+\varepsilon_{k+1}+\varepsilon_{k-3}.\]
It follows that
\[\mu_{k+1}=\mu_{k-3}, \quad \varepsilon_{k+1}=-\varepsilon_{k-3}.\]

Continuing in this way, we easily see that in general we will have
\[r_{k+n}=r_{k-n-1}, \quad \mu_{k+n}={\mu}_{k-n-2}, \quad s_{k+n}=s_{k-n-2}.\]
By setting \(n=k-1\) in the last equation, we obtain
\[s_{2 k-1}=s_{-1}.\]
Now it is clear that \(s_{-1}\) is the same as 1; because we have in general
\[R=r_m^2+s_m s_{m-1}\]
thus by taking \(m=0\),
\[R=r^2+s s_{-1}\]
but \(R=r^2+s\), so \(s_{-1}=1\), and consequently
\[s_{2 k-1}=1.\]

II. Secondly, letting \(s_k=c s_{k-1}\), we have
\[\begin{gathered}
r_k=\mu_k s_k+\varepsilon_k, \\
r_{k-1}=\mu_{k-1} s_{k-1}+\varepsilon_{k-1},
\end{gathered}\]
thus
\[r_k-r_{k-1}=s_{k-1}\left(c \mu_k-\mu_{k-1}\right)+\varepsilon_k-\varepsilon_{k-1}.\]
Now \(r_k-r_{k-1}=-2 \varepsilon_{k-1}\), so
\[0=s_{k-1}\left(c \mu_k-\mu_{k-1}\right)+\varepsilon_k+\varepsilon_{k-1}.\]
This equation implies that
\[\mu_k=\frac{1}{c} \mu_{k-1}, \quad \varepsilon_k=-\varepsilon_{k-1}.\]
Thus, by adding the equations
\[r_k-r_{k-1}=-2 \varepsilon_{k-1}, \quad r_{k+1}-r_k=-2 \varepsilon_k,\]
we deduce that
\[r_{k+1}=r_{k-1}.\]

We furthermore have
\[r_{k+1}^2+s_k s_{k+1}=r_{k-1}^2+s_{k-1} s_{k-2},\]
and since \(r_{k+1}=r_{k-1}\) and \(s_k=c s_{k-1}\), we conclude
\[s_{k+1}=\frac{1}{c} s_{k-2}.\]

Continuing in this manner, we will have,
\[s_{2 k}=c^{ \pm 1},\]
meaning that \(s_{2 k}\) is independent of \(x\).

This property of the quantities \(s\), \(s_1\), \(s_2\), etc. shows that the equation \(s_{2 k}=a\) is identical to the equation \(s_k=a^{ \pm 1} s_{k-1}\), and that the equation \(s_{2 k-1}=1\) is identical to the equation \(s_k=s_{k-2}\). It follows that, when one seeks the form of \(R\) which satisfies the equation \(s_{2 k}=a\), one can instead impose the equation \(s_k=a^{ \pm 1} s_{k-1}\), and when one seeks the form of \(R\) which satisfies the equation \(s_{2 k-1}=1\), it suffices to take \(s_k=s_{k-2}\).  This greatly simplifies the calculations.

\subsection*{16.}

According to equations (41) and (42), we can simplify expression (40) as follows:

\begin{center}a) When \(m\) is even and equal to \(2k\), we have\end{center}
\[\tag{43}\left\{\begin{array}{l}\int \frac{2}{\sqrt{R}}\left(dr+dr_1+\dots+dr_{k-1}+\frac{1}{2}dr_k-\mu ds-\mu_1 ds_1-\dots-\mu_{k-1}ds_{k-1}\right) \\ =\log \frac{r+\sqrt{R}}{r-\sqrt{R}}+\log \frac{r_1+\sqrt{R}}{r_1-\sqrt{R}}+\dots+\log \frac{r_{k-1}+\sqrt{R}}{r_{k-1}-\sqrt{R}}+\frac{1}{2}\log \frac{r_k+\sqrt{R}}{r_k-\sqrt{R}}.\end{array}\right.\]
\begin{center} b) When \(m\) is odd and equal to \(2k-1\), we have \end{center}
\[\tag{44} \left\{\begin{array}{c}\int \frac{2}{\sqrt{R}}\left(dr+dr_1+\dots+dr_{k-1}-\mu ds-\mu_1 ds_1-\dots-\mu_{k-2}ds_{k-2}-\frac{1}{2}\mu_{k-1}ds_{k-1}\right) \\ =\log \frac{r+\sqrt{R}}{r-\sqrt{R}}+\log \frac{r_1+\sqrt{R}}{r_1-\sqrt{R}}+\dots+\log \frac{r_{k-1}+\sqrt{R}}{r_{k-1}-\sqrt{R}}.\end{array}\right.\]

\subsection*{17.}

To apply the above to an example, let us consider the integral 
\[\int \frac{\varrho d x}{\sqrt{x^4+\alpha x^3+\beta x^2+\gamma x+\delta}}.\]
Here, we have \(\delta R=4\), so the functions \(s\), \(s_1\), \(s_2\), \(s_3 \dots\) are linear functions, and therefore the equation \(s_m=\) const. yields only one condition equation among the quantities \(\alpha\), \(\beta\), \(\gamma\), \(\delta\), \(\varepsilon\).

Setting
\[x^4+\alpha x^3+\beta x^2+\gamma x+\delta=\left(x^2+a x+b\right)^2+c+e x,\]
we will have
\[r=x^2+a x+b, \quad s=c+e x.\]
To simplify the calculation, we will take \(c=0\). In this case, we have \(s=e x\), and therefore,
\[\mu=E\left(\frac{r}{s}\right)=E\left(\frac{x^2+a x+b}{e x}\right);\]
that is,
\[\mu=\frac{x}{e}+\frac{a}{e}, \quad \varepsilon=b.\]
Moreover,
\[\begin{gathered}
r_1=r-2 \varepsilon=x^2+a x+b-2 b=x^2+a x-b,\\
s_1=1+4 \varepsilon \mu=1+4 b \frac{x+a}{e}=\frac{4 b}{e} x+\frac{4 a b}{e}+1, \\
\mu_1=E\left(\frac{r_1}{s_1}\right)=E \frac{x^2+a x-b}{\frac{4 b}{e} x+\frac{4 a b}{e}+1}=\frac{e}{4 b} x-\frac{e^2}{16 b^2}, \\
\varepsilon_1=r_1-\mu_1 s_1=\frac{a e}{4 b}+\frac{e^2}{16 b^2}-b, \\
s_2=s+4 \varepsilon_1 \mu_1=\left(\frac{a e^2}{4 b^2}+\frac{e^3}{16 b^3}\right) x-\frac{e^2}{4 b^2}\left(\frac{a e}{4 b}+\frac{e^2}{16 b^2}-b\right).
\end{gathered}\]

First, let us assume that \(s_1\) is constant. Then the equation
\[s_1=\frac{4 b}{e} x+\frac{4 a b}{e}+1\]
gives
\[b=0,\]
therefore,
\[\begin{gathered}
r=x^2+a x, \\
\int \frac{2}{\sqrt{R}}\left(d r-\frac{1}{2} \mu d s\right)=\log \frac{r+\sqrt{R}}{r-\sqrt{R}},
\end{gathered}\]
or, since \(\mu=\frac{x+a}{e}\), \(s=e x\),
\[\int \frac{(3 x+a) d x}{\sqrt{\left(x^2+a x\right)^2+e x}}=\log \frac{x^2+a x+\sqrt{R}}{x^2+a x-\sqrt{R}}.\]
This integral can also be easily evaluated by dividing the numerator and denominator of the differential by \(\sqrt{x}\).

Second, let \(s_2\) be constant. In this case, formula (43) gives, with \(k\) being equal to one,
\[\int \frac{2}{\sqrt{R}}\left(d r+\frac{1}{2} d r_1-\mu d s\right)=\log \frac{r+\sqrt{R}}{r-\sqrt{R}}+\frac{1}{2} \log \frac{r_1+\sqrt{R}}{r_1-\sqrt{R}}.\]
Now the equation \(s_2=\) const. gives \(s_1=c s\), so
\[\frac{4 b}{e} x+\frac{4 a b}{e}+1=c e x.\]
The condition equation will thus be \(\frac{4 a b}{e}+1=0\), that is
\[e=-4 a b,\]
thus
\[R=\left(x^2+a x+b\right)^2-4 a b x.\]
In addition, having \(\mu=\frac{x+a}{e}\), \(r=x^2+a x+b\), and \(r_1=x^2+a x-b\), we will have the formula,
\[\int \frac{(4 x+a) d x}{\sqrt{\left(x^2+a x+b\right)^2-4 a b x}}=\log \frac{x^2+a x+b+\sqrt{R}}{x^2+a x+b-\sqrt{R}}+\frac{1}{2} \log \frac{x^2+a x^2-b+\sqrt{R}}{x^2+a x-b-\sqrt{R}}.\]

Third, let \(s_3\) be constant.  This gives \(s=s_2\), i.e.
\[\frac{a e}{4 b}+\frac{e^2}{16 b^2}-b=0.\]
From this we deduce
\[e=-2 b\left(a \pm \sqrt{a^2+4 b}\right).\]
Formula (44) consequently gives, since \(k=2\),
\[\int \frac{\left(5 x+\frac{3}{2} a \mp \frac{1}{2} \sqrt{a^2+4 b}\right) d x}{\sqrt{\left(x^2+a x+b\right)^2-2 b x\left(a \pm \sqrt{a^2+4 b}\right)}}
=\log \frac{x^2+a x+b+\sqrt{R}}{x^2+a x+b-\sqrt{R}}+\log \frac{x^2+a x-b+\sqrt{R}}{x^2+a x-b-\sqrt{R}}.\]
If for example \(a=0\), \(b=1\), we will have this integral:
\[\int \frac{(5 x-1) d x}{\sqrt{\left(x^2+1\right)^2-4 x}}=\log \frac{x^2+1+\sqrt{\left(x^2+1\right)^2-4 x}}{x^2+1-\sqrt{\left(x^2+1\right)^2-4 x}}+\log \frac{x^2-1+\sqrt{\left(x^2+1\right)^2-4 x}}{x^2-1-\sqrt{\left(x^2+1\right)^2-4 x}}.\]

Fourth, let \(s_4\) be constant. This gives \(s_2=c s_1\), i.e.
\[\left(\frac{a e^2}{4 b^2}+\frac{e^3}{16 b^3}\right) x-\frac{e^2}{4 b^2}\left(\frac{a e}{4 b}+\frac{e^2}{16 b^2}-b\right)=\frac{4 c b}{e} x+\left(\frac{4 a b}{e}+1\right) c.\]
Comparing the coefficients and then eliminating \(c\), we obtain
\[\begin{gathered}
\frac{e}{16 b^3}(e+4 a b)^2=-\frac{e}{b}\left(\frac{a e}{4 b}+\frac{e^2}{16 b^2}-b\right),\\
(e+4 a b)^2=16 b^3-e(e+4 a b), \\
e^2+6 a b e=8 b^3-8 a^2 b^2, \\
e=-3 a b \mp \sqrt{8 b^3+a^2 b^2}=-b\left(3 a \pm \sqrt{a^2+8 b}\right).
\end{gathered}\]
According to this expression, formula (43) gives,
\[\begin{aligned}
\int \frac{\left(6 x+\frac{3}{2} a-\frac{1}{2} \sqrt{a^2+8 b}\right) d x}{\sqrt{\left(x^2+a x+b\right)^2-b\left(3 a+\sqrt{a^2+8 b}\right) x}}&=\log \frac{x^2+a x+b+\sqrt{R}}{x^2+a x+b-\sqrt{R}} \\
+\log \frac{x^2+a x-b+\sqrt{R}}{x^2+a x-b-\sqrt{R}}&+\frac{1}{2} \log \frac{x^2+a x+\frac{1}{4} a\left(a-\sqrt{a^2+8 b}\right)+\sqrt{R}}{x^2+a x+\frac{1}{4} a\left(a-\sqrt{a^2+8 b}\right)-\sqrt{R}}.
\end{aligned}\]
If we take for example \(a=0\), \(b=\frac{1}{2}\), we obtain
\[ \begin{aligned} \int &\frac{\left(x+\frac{1}{6}\right)dx}{\sqrt{x^4+x^2+x+\frac{1}{4}}} = \frac{1}{6} \log \frac{x^2 + \frac{1}{2} + \sqrt{x^4+x^2+x+\frac{1}{4}}}{x^2 + \frac{1}{2} - \sqrt{x^4+x^2+x+\frac{1}{4}}} \\
&+  \frac{1}{6} \log \frac{x^2 + -\frac{1}{2} + \sqrt{x^4+x^2+x+\frac{1}{4}}}{x^2 - \frac{1}{2} - \sqrt{x^4+x^2+x+\frac{1}{4}}} + \frac{1}{12} \log \frac{x^2 + \sqrt{x^4+x^2+x+\frac{1}{4}}}{x^2 - \sqrt{x^4+x^2+x+\frac{1}{4}}} 
\end{aligned}\]

We can continue this way and find a large number of integrals. For example, the integral
\[\int \frac{\left(x+\frac{\sqrt{5}+1}{14}\right) d x}{\sqrt{\left(x^2+\frac{\sqrt{5}-1}{2}\right)^2+(\sqrt{5}-1)^2 x}}\]
can be expressed in terms of logarithms.
\begin{center}\rule{2in}{0.1pt}\end{center}

We have here sought integrals of the form \(\int \frac{\rho dx}{\sqrt{R}}\) which can be expressed as a logarithmic function of the form \(\log \frac{p+q \sqrt{R}}{p-q \sqrt{R}}\). One could make the problem even more general, and search in general for all integrals of the above form that could be expressed in any way using logarithms; but we would not find any new integrals. In fact, we have this remarkable theorem:

\begin{quote}"When an integral of the form \(\int \frac{\varrho d x}{\sqrt{R}}\), where \(\varrho\) and \(R\) are integral functions of \(x\), can be expressed in terms of logarithms, it can always be expressed in the following way:
\[\int \frac{\varrho d x}{\sqrt{R}}=A \log \frac{p+q \sqrt{R}}{p-q \sqrt{R}},\]
where \(A\) is a constant, and \(p\) and \(q\) are integral functions of \(x\)." \end{quote}

I will prove this theorem on another occasion.

\begin{center}\rule{2in}{0.1pt}\end{center}
\vfill
\end{document}