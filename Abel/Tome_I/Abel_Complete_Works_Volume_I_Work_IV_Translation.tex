\documentclass[oneside, 12 pt, leqno]{memoir}
\usepackage{standalone}
\usepackage[dvips,text={6.2in,8.5in},left=0.9truein,top=1.5truein]{geometry}
\usepackage{amsmath, amssymb, amsthm, amsfonts}
\usepackage{graphicx}
\usepackage{titlesec}
\usepackage{multirow}
\usepackage{wrapfig}
\usepackage{microtype}
\usepackage{indentfirst}
\usepackage[utf8]{inputenc}
\usepackage{exscale}
\usepackage{mlmodern}
\usepackage[OT1]{fontenc}
\usepackage[bottomfloats]{footmisc}
\parindent=2.27em
\parskip=0pt
\nonfrenchspacing
\renewcommand{\baselinestretch}{1.15}
\DeclareMathSizes{12}{12}{8}{6}
\everymath{\displaystyle}
\allowdisplaybreaks
\raggedbottom
\titleformat{\section}
  {\normalfont\centering}{\thesection.}{1em}{}
\titleformat{\subsection}
  {\normalfont\normalsize\centering}{\thesection.}{1em}{}
\titleformat{\subsubsection}
  {\normalfont\normalsize\centering}{\thesection.}{1em}{}
\spaceskip=0.67em plus 0.33em minus 0.33em
\thickmuskip=4mu plus 4mu
\medmuskip=3mu plus 1.5mu minus 3mu
\AtBeginDocument{%
  \mathchardef\stdcomma=\mathcode`,
  \mathcode`,="8000
}
\begingroup\lccode`~=`, \lowercase{\endgroup\def~}{\stdcomma\mspace{\medmuskip}}
\let\oldfrac\frac
\def\frac#1#2{\mathchoice{\text{\scalebox{.83}{${\oldfrac{#1}{#2}}$}}}{\text{\scalebox{.83}{${\displaystyle\oldfrac{#1}{#2}}$}}}{\genfrac{}{}{}{2}{#1}{#2}}{\genfrac{}{}{}{3}{#1}{#2}}}
\begin{document}
\setlength{\abovedisplayskip}{0.33\baselineskip plus .16\baselineskip minus .16\baselineskip}
\setlength{\belowdisplayskip}{0.33\baselineskip plus .16\baselineskip minus .16\baselineskip}
\;\\ [3\baselineskip]
\section*{\begin{Large}IV.\end{Large} \\ [\baselineskip]
THE FINITE INTEGRAL \(\Sigma^n \varphi x\) EXPRESSED BY A SIMPLE DEFINITE INTEGRAL.}
\begin{center}
\rule{2in}{0.1pt}\\ [0.5\baselineskip]
\begin{scriptsize} Magazin for Naturvidenskaberne, Aargang III, Bind 2, Christiania 1825.\par\end{scriptsize}
\rule{2in}{0.1pt}
\end{center}

It is known that one can use the theorem of \textit{Parseval} to express the finite integral \(\Sigma^n \varphi x\) as a definite double integral. However, if I am not mistaken, no one has expressed the same integral as a single definite integral. That is the objective of this memoire.

Letting \(\varphi x \) denote an arbitrary function of \(x\), it is easy to see that we can always assume
\[\tag{1} \varphi(x)=\int e^{vx} f(v) . dv,\]
where the integral is taken between two arbitrary limits of \(v\), independent of \(x\). The function \(f(v)\) is a function of \(v\), whose form depends on that of \(\varphi(x)\). Assuming \(A(x)=1\), by taking the finite integral of both sides of equation (1), we have
\[\tag{2} \Sigma \varphi(x)=\int e^{vx} \frac{f(v)}{e^v-1} dv,\]
where an arbitrary constant must be added. By taking the finite integral again, we obtain
\[\Sigma^2 \varphi(x)=\int e^{vx} \frac{f v}{\left(e^v-1\right)^2} dv.\]
In general, we have
\[\tag{3} \Sigma^n \varphi(x)=\int e^{vx} \frac{f v}{\left(e^v-1\right)^n}dv.\]
To complete this integral, we need to add to the right-hand side a function of the form
\[C+C_1x+C_2x^2+\ldots+C_{n-1}x^{n-1},\]
where \(C\), \(C_1\), \(C_2\), etc. are arbitrary constants.

Now let us find the value of the definite integral \(\int e^{v x} \frac{f v}{\left(e^v-1\right)^n} d v\). For this purpose, I use a theorem due to M. \textit{Legendre} (Exerc. de calc. int. t. II, p. 189), which states that
\[\frac{1}{4} \frac{e^v+1}{e^v-1}-\frac{1}{2 v}=\int_0^{\frac{1}{0}} \frac{d t. \sin v t}{e^{2 \pi t}-1}.\]
We obtain from this equation
\[\tag{4} \frac{1}{e^v-1}=\frac{1}{v}-\frac{1}{2}+2 \int_0^{\frac{1}{0}} \frac{d t. \sin v t}{e^{2 \pi t}-1}.\]
By substituting this value of \(\frac{1}{e^v-1}\) in equation (2), we have
\[\Sigma \varphi x=\int e^{v x} \frac{f v}{v} d v-\frac{1}{2} \int e^{v x} f v. d v+2 \int_0^{\frac{1}{0}} \frac{d t}{e^{2 \pi t}-1} \int e^{v x} f v. \sin v t. d v.\]
The integral \(\int e^{v x} f v. \sin v t. d v\) can be found as follows. By replacing \(x\) successively by \(x+t \sqrt{-1}\) and \(x-t \sqrt{-1}\) in equation (1), we obtain
\[\begin{aligned}
& \varphi(x+t \sqrt{-1})=\int e^{v x} e^{v t \sqrt{-1}} f v. d v, \\
& \varphi(x-t \sqrt{-1})=\int e^{v x} e^{-v t \sqrt{-1}} f v. d v,
\end{aligned}\]
from which we obtain, by subtracting and dividing by \(2 \sqrt{-1}\),
\[\int e^{v x} \sin v t. f v. d v=\frac{\varphi(x+t \sqrt{-1})-\varphi(x-t \sqrt{-1})}{2 \sqrt{-1}}.\]
Thus, the expression for \(\Sigma \varphi x\) becomes
\[\Sigma \varphi x=\int \varphi x. d x-\frac{1}{2} \varphi x+2 \int_0^{\frac{1}{0}} \frac{d t}{e^{2 \pi t}-1} \frac{\varphi(x+t \sqrt{-1})-\varphi(x-t \sqrt{-1})}{2 \sqrt{-1}}.\]

Now to find the value of the general integral
\[\Sigma^n \varphi x=\int e^{v x} f v \frac{d v}{\left(e^v-1\right)^n},\]
let us set
\[\frac{1}{\left(e^v-1\right)^n}=(-1)^{n-1}\left(A_{0, n} p+A_{1, n} \frac{d p}{d v}+A_{2, n} \frac{d^2 p}{d v^2}+\ldots+A_{n-1, n} \frac{d^{n-1} p}{d v^{n-1}}\right)\]
where \(p\) is equal to \(\frac{1}{e^v-1}\), and \(A_{0, n}\), \(A_{1, n} \dots\) are numerical coefficients that must be determined. If we differentiate the previous equation, we have
\[\frac{n e^v}{\left(e^v-1\right)^{n+1}}=(-1)^n\left(A_{0, n} \frac{d p}{d v}+A_{1, n} \frac{d^2 p}{d v^2}+\ldots+A_{n-1, n} \frac{d^n p}{d v^n}\right)\]
Now
\[\frac{n e^v}{\left(e^v-1\right)^{n+1}}=\frac{n}{\left(e^v-1\right)^n}+\frac{n}{\left(e^v-1\right)^{n+1}},\]
so
\[\begin{aligned}
\frac{n e^v}{\left(e^v-1\right)^{n+1}} & =n(-1)^{n-1}\left(A_{0, n} p+A_{1, n} \frac{d p}{d v}+\ldots+A_{n-1, n} \frac{d^{n-1} p}{d v^{n-1}}\right) \\
& +n(-1)^n\left(A_{0, n+1} p+A_{1, n+1} \frac{d p}{d v}+\cdots+A_{n, n+1} \frac{d^n p}{d v^n}\right).
\end{aligned}\]
Comparing these two expressions for \(\frac{n e^v}{\left(e^v-1\right)^{n+1}}\), we obtain the following equations:
\[\begin{aligned}
 &A_{0, n+1}-A_{0, n}=0 && \text { \rotatebox[origin=c]{180}c: }&& \Delta A_{0, n}=0 \text {, } \\
 &A_{1, n+1}-A_{1, n}=\frac{1}{n} A_{0, n} && \text { \rotatebox[origin=c]{180}c: }&&  \Delta A_{1, n}=\frac{1}{n} A_{0, n}, \\
 &A_{2, n+1}-A_{2, n}=\frac{1}{n} A_{1, n} && \text { \rotatebox[origin=c]{180}c: }&& \Delta A_{2, n}=\frac{1}{n} A_{1, n}, \\
 &\ldots \ldots \ldots \ldots \ldots && && \ldots \ldots \ldots &\\
 &A_{n-1, n+1}-A_{n-1, n}=\frac{1}{n} A_{n-2, n} && \text { \rotatebox[origin=c]{180}c: }&& \Delta A_{n-1, n}=\frac{1}{n} A_{n-2, n}, \\
 &\phantom{A_{n-1, n+1}-:} A_{n, n+1}=\frac{1}{n} A_{n-1, n}, &&
\end{aligned}\]
from which we deduce
\[\begin{gathered}
A_{0, n}=1, A_{1, n}=\Sigma \frac{1}{n}, A_{2, n}=\Sigma\left(\frac{1}{n} \Sigma \frac{1}{n}\right), A_{3, n}=\Sigma\left[\frac{1}{n} \Sigma\left(\frac{1}{n} \Sigma \frac{1}{n}\right)\right] \text { etc. } \\
A_{n, n+1} = \frac{1}{n} \frac{1}{n-1} \frac{1}{n-2} \cdots \frac{1}{2}. \frac{1}{1}. A_{0,1}=\frac{1}{\Gamma(n+1)}.
\end{gathered}\]
This last equation can be used to determine the constants that appear in the expressions of \(A_{1, n}\), \(A_{2, n}\), \(A_{3, n}\) etc.

Having thus determined the coefficients \(A_{0, n}\), \(A_{1, n}\), \(A_{2, n}\), etc., we will have, upon substituting the resulting value of \(\frac{1}{\left(e^v-1\right)^n}\) in equation (3),
\[\Sigma^n \varphi x=(-1)^{n-1} \int e^{v x} f v. d v\left(A_{0, n} p+A_{1, n} \frac{d p}{d v}+\ldots+A_{n-1, n} \frac{d^{n-1} p}{d v^{n-1}}\right);\]
now we have 
\[p=\frac{1}{v}-\frac{1}{2}+2 \int_0^{\frac{1}{0}} \frac{d t. \sin v t}{e^{2 \pi t}-1},\]
from which we obtain, by differentiating,
\[\begin{aligned}
 \frac{d p}{d v}&=-\frac{1}{v^2}+2 \int_0^{\frac{1}{0}} \frac{t d t. \cos v t}{e^{2 \pi t}-1}, \\
 \frac{d^2 p}{d v^2}&=\phantom{-}\frac{2}{v^3}-2 \int_0^{\frac{1}{0}} \frac{t^2 d t. \sin v t}{e^{2 \pi t}-1}, \\
\frac{d^3 p}{d v^3}&=-\frac{2.3}{v^4}-2 \int_0^{\frac{1}{0}} \frac{t^3 d t. \cos v t}{e^{2 \pi t}-1} \text { etc.};
\end{aligned}\]
therefore, by substituting 
\[\begin{aligned}
\Sigma^n \varphi x & =\int\left(A_{n-1, n} \frac{\Gamma n}{v^n}-A_{n-2, n} \frac{\Gamma(n-1)}{v^{n-1}}+...+(-1)^{n-1} A_{0, n} \frac{1}{v}+(-1)^n. \frac{1}{2}\right) e^{v x} f v. d v \\
& +2(-1)^{n-1} \iint_0^{\frac{1}{0}} \frac{P \sin v t. d t}{e^{2 \pi t}-1} e^{v x} f v. d v+2(-1)^{n-1} \iint_0^{\frac{1}{0}} \frac{Q \cos v t. d t}{e^{2. T t}-1} e^{v x} f v. d v.
\end{aligned}\]
From the equation \(\varphi x=\int e^{v x} f v. d v\) we obtain by integrating:
\[\begin{aligned}
& \int \varphi x. d x=\int e^{v x} f v \frac{d v}{v},\\
& \int^2 \varphi x. d x^2=\int e^{v x} f v \frac{d v}{v^2},\\
& \int^3 \varphi x. d x^3=\int e^{v x} f v \frac{d v}{v^3} \text { etc.};
\end{aligned}\]
in addition, we have
\[\begin{aligned}
& \int \sin v t. e^{v x} f v. d v=\frac{\varphi(x+t \sqrt{-1})-\varphi(x-t \sqrt{-1})}{2 \sqrt{-1}},\\
& \int \cos v t. e^{v x} f v. d v=\frac{\varphi(x+t \sqrt{-1})+\varphi(x-t \sqrt{-1})}{2},
\end{aligned}\]
therefore we will have, by substituting 
\[\begin{aligned}
\Sigma^n \varphi x & =A_{n-1, n} \Gamma n \int^n \varphi x. d x^n-A_{n-2, n} \Gamma(n-1) \int^{n-1} \varphi x. d x^{n-1}+...+(-1)^{n-1} \int \varphi x. d x \\
& +(-1)^n. \frac{1}{2} \varphi x+2(-1)^{n-1} \int_0^{\frac{1}{0}} \frac{P d t}{e^{2 \pi t}-1} \frac{\varphi(x+t \sqrt{-1})-\varphi(x-t \sqrt{-1})}{2 \sqrt{-1}} \\
& +2(-1)^{n-1} \int_0^{\frac{1}{0}} \frac{Q d t}{e^{2 \pi t}-1} \frac{\varphi(x+t \sqrt{-1})+\varphi(x-t \sqrt{-1})}{2}
\end{aligned}\]
where
\[\begin{aligned}
& P=A_{0, n}\phantom{t}-A_{2, n} t^2+A_{4, n} t^4-\ldots, \\
& Q=A_{1, n} t-A_{3, n} t^3+A_{5, n} t^5-\ldots.
\end{aligned}\]

By letting e.g. \(n=2\), we will have
\[\begin{aligned}
\Sigma^2 \varphi x=\iint \varphi x. d x^2-\int \varphi x. d x+\frac{1}{2} \varphi x-2 \int_0^{\frac{1}{0}} \frac{d t}{e^{2 \pi t}-1} \frac{\varphi(x+t \sqrt{-1})-\varphi(x-t \sqrt{-1})}{2 \sqrt{-1}} &\\
-2 \int_0^{\frac{1}{0}} \frac{t d t}{e^{2. \pi t}-1} \frac{\varphi(x+t \sqrt{-1})+\varphi(x-t \sqrt{-1})}{2}.&
\end{aligned}\]
Setting e.g. \(\varphi x=e^{a x}\), we have
\[\varphi(x \pm t \sqrt{-1})=e^{a x} e^{ \pm a t \sqrt{-1}}, \int e^{a x} d x=\frac{1}{a} e^{a x}, \iint e^{a x} d x^2=\frac{1}{a^2} e^{a x},\]
therefore, by substituting and dividing by \(e^{a x}\),
\[\frac{1}{\left(e^a-1\right)^2}=\frac{1}{2}-\frac{1}{a}+\frac{1}{a^2}-2 \int_0^{\frac{1}{0}} \frac{d t. \sin a t}{e^{2 \pi t}-1}-2 \int_0^{\frac{1}{0}} \frac{t d t. \cos a t}{e^{2\pi t}-1}.\]

The most remarkable case is when \(n=1\). In that case, as we have seen previously:
\[\Sigma \varphi x=C+\int \varphi x \, d x-\frac{1}{2} \varphi x+2 \int_0^{\frac{1}{0}} \frac{d t}{e^{2 \pi t}-1} \frac{\varphi(x+t \sqrt{-1})-\varphi(x-t \sqrt{-1})}{2 \sqrt{-1}}.\]
Assuming that both integrals \(\Sigma \varphi x\) and \(\int \varphi x \, d x\) vanish for \(x=a\), it is clear that we will have:
\[C=\frac{1}{2} \varphi a-2 \int_0^{\frac{1}{0}} \frac{d t}{e^{2 \pi t}-1} \frac{\varphi(a+t \sqrt{-1})-\varphi(a-t \sqrt{-1})}{2 \sqrt{-1}};\]
therefore
\[\begin{aligned}
\Sigma \varphi x=\int \varphi x \, d x-\frac{1}{2}(\varphi x-\varphi a) +2 \int_0^{\frac{1}{0}} \frac{d t}{e^{2 \pi t}-1} \frac{\varphi(x+t \sqrt{-1})-\varphi(x-t \sqrt{-1})}{2 \sqrt{-1}}& \\
 -2 \int_0^{\frac{1}{0}} \frac{d t}{e^{2\pi t}-1} \frac{\varphi(a+t \sqrt{-1})-\varphi(a-t \sqrt{-1})}{2 \sqrt{-1}}.&
\end{aligned}\]

If we let \(x=\infty\), assuming that \(\varphi x\) and \(\int \varphi x.d x\) vanish for this value of \(x\), we will have:
\[\begin{aligned}
& \varphi a+\varphi(a+1)+\varphi(a+2)+\varphi(a+3)+\ldots \text { in inf. } \\
& =\int_a^{\frac{1}{0}} \varphi x. d x+\frac{1}{2} \varphi a-2 \int_0^{\frac{1}{0}} \frac{d t}{e^{2 \pi t}-1} \frac{\varphi(a+t \sqrt{-1})-\varphi(a-t \sqrt{-1})}{2 \sqrt{-1}}.
\end{aligned}\]

Setting e.g. \(\varphi x = \frac{1}{x^2}\), then we have
\[\frac{\varphi(a+t \sqrt{-1})-\varphi(a-t \sqrt{-1})}{2 \sqrt{-1}}=\frac{-2 a t}{\left(a^2+t^2\right)^2},\]
so
\[\frac{1}{a^2}+\frac{1}{(a+1)^2}+\frac{1}{(a+2)^2}+\ldots=\frac{1}{2 a^2}+\frac{1}{a}+4 a \int_0^{\frac{1}{0}} \frac{t d t}{\left(e^{2 \pi t}-1\right)\left(a^2+t^2\right)^2},\]
and by letting \(a=1\)
\[1+\frac{1}{4}+\frac{1}{9}+\frac{1}{16}+\frac{1}{25}+\ldots=\frac{\pi^2}{6}=\frac{3}{2}+4 \int_0^{\frac{1}{0}} \frac{t d t}{\left(e^{2 \pi t}-1\right)\left(1+t^2\right)^2}.\]
\begin{center}\rule{2in}{0.1pt}\end{center}
\end{document}