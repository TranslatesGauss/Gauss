\documentclass[oneside, 12 pt, leqno]{memoir}
\usepackage{standalone}
\usepackage[dvips,text={6.2in,8.5in},left=0.9truein,top=1.5truein]{geometry}
\usepackage{amsmath, amssymb, amsthm, amsfonts}
\usepackage{graphicx}
\usepackage{titlesec}
\usepackage{multirow}
\usepackage{wrapfig}
\usepackage{microtype}
\usepackage{indentfirst}
\usepackage[utf8]{inputenc}
\usepackage{exscale}
\usepackage{mlmodern}
\usepackage[OT1]{fontenc}
\usepackage[bottomfloats]{footmisc}
\parindent=2.27em
\parskip=0pt
\nonfrenchspacing
\renewcommand{\baselinestretch}{1.15}
\DeclareMathSizes{12}{12}{8}{6}
\everymath{\displaystyle}
\allowdisplaybreaks
\raggedbottom
\titleformat{\section}
  {\normalfont\centering}{\thesection.}{1em}{}
\titleformat{\subsection}
  {\normalfont\normalsize\centering}{\thesection.}{1em}{}
\titleformat{\subsubsection}
  {\normalfont\normalsize\centering}{\thesection.}{1em}{}
\spaceskip=0.67em plus 0.33em minus 0.33em
\thickmuskip=4mu plus 4mu
\medmuskip=3mu plus 1.5mu minus 3mu
\AtBeginDocument{%
  \mathchardef\stdcomma=\mathcode`,
  \mathcode`,="8000
}
\begingroup\lccode`~=`, \lowercase{\endgroup\def~}{\stdcomma\mspace{\medmuskip}}
\let\oldfrac\frac
\def\frac#1#2{\mathchoice{\text{\scalebox{.83}{${\oldfrac{#1}{#2}}$}}}{\text{\scalebox{.83}{${\displaystyle\oldfrac{#1}{#2}}$}}}{\genfrac{}{}{}{2}{#1}{#2}}{\genfrac{}{}{}{3}{#1}{#2}}}
\begin{document}
\setlength{\abovedisplayskip}{0.33\baselineskip plus .16\baselineskip minus .16\baselineskip}
\setlength{\belowdisplayskip}{0.33\baselineskip plus .16\baselineskip minus .16\baselineskip}

\;\\[4\baselineskip]
\section*{{\Large VIII.} \\ [\baselineskip]
REMARK ON PAPER NO. 4 IN THE FIRST ISSUE OF CRELLE'S JOURNAL.}
\begin{center}
\rule{2in}{0.1pt}\\
{\tiny Journal für die reine und angewandte Mathematik, edited by \textit{Crelle}, Vol. I, Berlin 1826.}\\
\rule{2in}{0.1pt}
\end{center}

The aim of the study is to find the effect of a force on three given points. The author's results are very much justified when the three points are not located on the same straight line; but not in the case when they are. The three equations by which the three unknowns \(Q, Q', Q''\) are determined are as follows:
\[\tag{1}\left\{\begin{array}{l}
P=Q+Q'+Q'', \\
Q'b\sin\alpha=Q''c\sin\beta, \\
Qa\sin\alpha=-Q''c\sin(\alpha+\beta).
\end{array}\right.\]
These equations hold for any values of \(P\), \(a\), \(b\), \(c\), \(\alpha\), and \(\beta\). In general, they give, as the author found,
\[\tag{2}\left\{\begin{array}{l}
Q=-\frac{bc\sin(\alpha+\beta)}{r}P, \\[0.25\normalbaselineskip] 
Q'=\frac{ac\sin\beta}{r}P, \\[0.25\normalbaselineskip] 
Q''=\frac{ab\sin\alpha}{r}P,
\end{array}\right.\]
where
\[r=ab\sin\alpha+ac\sin\beta-bc\sin(\alpha+\beta).\]
However, the equations (2) are not determined when one or the other of the quantities \(Q\), \(Q'\), \( Q''\) takes the form \(\frac{0}{0}\), which occurs, as is easily seen, for \(\alpha=\beta=180^{\circ}\). In this case, it is necessary to resort to the fundamental equations (1), which then give
\[\begin{aligned}
&P=Q+Q'+Q'', \\
&Q'b\sin 180^{\circ}=Q''c\sin 180^{\circ},\\
&Qa\sin 180^{\circ}=-Q''c\sin 360^{\circ}.\\
\end{aligned}\]
The last two equations are identical since \(\sin 180^{\circ}=\sin 360^{\circ}=0\), thus in the case where \(\alpha=\beta=180^{\circ}\), there exists only one equation, namely \(P=Q+Q'+Q''\), and consequently, the values of \(Q\), \(Q'\), \(Q''\) cannot be derived from the equations established by the author.
\begin{center}\rule{2in}{0.1pt}\end{center}
\vfill
\end{document}