\documentclass[14pt]{memoir}
\usepackage{standalone}
\usepackage[dvips,text={6.5truein,9.1truein},left=0.86truein,right=0.8truein,top=1truein]{geometry}
\usepackage{amsmath, amsthm, amsfonts}
\usepackage{titlesec}
\usepackage{enumitem}
% Uncomment to use syncing
%\usepackage{pdfsync}


% Paragraphs
\usepackage{indentfirst}
\parindent=3em
\parskip=0pt

%font
\usepackage{mlmodern}
%\usepackage[T1]{fontenc}% http://ctan.org/pkg/fontenc
\usepackage{microtype}

\titleformat{\section}
 {\centering}{\thesection.}{0em}{}

\titleformat{\subsection}
 {\normalfont\small\centering}{\thesection.}{0em}{}
\titlespacing*{\subsection}
{0pt}{\baselineskip}{0\baselineskip}

%footnotes
\usepackage[perpage]{footmisc}
\usepackage{etoolbox}
\DefineFNsymbols*{asterisks}{{ *}{ **}{ ***}}
\setfnsymbol{asterisks}
\renewcommand{\thefootnote}{\fnsymbol{footnote}}
\makeatletter
\renewcommand{\@makefnmark}{\mbox{\normalfont\@thefnmark})}
\settowidth{\footnotemargin}{***) }
\patchcmd{\@makefntext}{\hss\@makefnmark}{\hss \@makefnmark\ }{}{}
\makeatother

%Line Spacing
\renewcommand{\baselinestretch}{1.1}
\renewcommand{\footnotelayout}{ \baselineskip=1.02\baselineskip }
\setlength{\skip\footins}{2\baselineskip}

\theoremstyle{plain}
\newtheorem*{theorem}{Theorem}
\newtheorem{proposition}{Proposition}
\newtheorem{lemma}{Lemma}
\newtheorem{problem}{Problem}


\theoremstyle{remark}
\newtheorem*{example}{Example}
\newtheorem*{examples}{Examples}

\begin{document}

\setlength{\abovedisplayskip}{0.5\baselineskip}
\setlength{\belowdisplayskip}{0.5\baselineskip}


\section*{BEWEIS \\[\baselineskip]
\begin{large}EINES ALGEBRAISCHEN LEHRSATZES.\end{large}\\[\baselineskip]
 \rule{0.85in}{0.5pt}}
Der Gegenstand dieses Aufsatzes ist der \textsc{Cartes}ische, gewöhnlich nach \textsc{Harriot} benannte, Lehrsatz über den Zusammenhang der Anzahl der positiven und negativen Wurzeln einer algebraischen Gleichung mit der Anzahl der Abwechselungen und Folgen in den Zeichen der Coëfficienten. Man vermisst an den von verschiedenen Schriftstellern versuchten Beweisen dieses Theorems die Klarheit, Kürze und umfassende Allgemeinheit, die man bei einem so elementarischen Gegenstande mit Recht verlangen kann, und eine neue Behandlung desselben scheint daher nicht überflüssig zu sein.

Es sei \(X\) eine algebraische ganze Function von \(x\) von der Ordnung \(m\), nach absteigenden Potenzen von \(x\) geordnet. Wir nehmen an (ohne Nachtheil für die Allgemeinheit), dass das höchste Glied \(x^m\) sei, und das niedrigste von \(x\) freie Glied nicht fehle; bloss die wirklich vorhandenen Glieder sollen aufgestellt, also nicht die etwa fehlenden mit dem Coëfficienten \(0\) angesetzt sein.

Wenn nicht alle Coëfficienten positiv sind, so werden sie einen oder mehrere Zeichenwechsel darbieten. Es sei \(-Nx^n\) das erste negative Glied, das erste hierauf folgende positive \(+Px^p\) das erste hierauf folgende negative \(-Qx^q\) u.s.w. Es sind mithin \(m\), \(n\), \(p\), \(q\) u.s.w. abnehmende ganze Zahlen; \(N\), \(P\), \(Q\) u.s.w. positiv, und \(X\) erscheint so dargestellt
\[X = x^m + + .. - Nx^n - - ..+Px^p++..-Qx^q-\text{ u.s.w.}\]%pagebreak

Es werde \(X\) mit dem einfachen Factor \(x-\alpha\) multiplicirt, wo \(\alpha\) positiv vorausgesetzt wird. Man sieht leicht, dass in dem Producte, \(x^{n+1}\) einen negativen, \(x^{p+1}\) einen positiven, \(x^{q+1}\) einen negativen Coëfficienten u.s.w., also das Product diese Form erhalten wird:
\[X(x-\alpha) = x^{m+1}\dots-N'x^{n+1}\dots+P'x^{p+1}..-Q'x^{q+1}\dots\]
so dass \(N'\), \(P'\), \(Q'\) u.s.w., positiv werden. Die Zeichen zwischen den aufgestellten Gliedern bleiben zwar unentschieden: allein es ist klar, dass vom höchsten Gliede bis zur Potenz \(x^{n+1}\) wenigstens ein Zeichenwechsel, bis \(x^{q+1}\) wenigstens zwei, bis \(x^{q+1}\) wenigstens drei u.s.w. statt finden. Ist der letzte Zeichenwechsel in \(X\) bei dem Gliede \(\pm Ux^u\), und bezeichnet man den Coëfficienten von \(x^{u+1}\) in \(X(x-\alpha)\) durch \(\pm U'\), so wird \(U'\) positiv sein, und bis zum Gliede \(\pm U'x^{u+1}\) haben dann wenigstens eben so viele Zeichenwechsel, wie in \(X\) sind, statt gefunden. Das letzte Glied in \(X(x-\alpha)\) wird aber das Zeichen \(\mp\) haben; es muss also bis dahin wenigstens noch ein Zeichenwechsel hinzugekommen sein. Wir schliessen also, dass \(X(x-\alpha)\) wenigstens einen Zeichenwechsel mehr hat als \(X\).

Es sei nun \(X\) das Product aller einfachen Factoren, die den negativen und imaginären Wurzeln einer Gleichung \(y=0\) entsprechen, also wenn \(\alpha\), \(\beta\), \(\gamma\) u.s.w. die positiven Wurzeln derselben Gleichung sind,
\[y = X(x-\alpha)(x-\beta)(x-\gamma)\dots\]
Es finden sich also nach vorstehendem Satze, in \(X(x-\alpha)\) wenigstens ein Zeichenwechsel, in \(X(x-\alpha)(x-\beta)\) wenigstens zwei, in \(X(x-\alpha)(x-\beta)(x-\gamma)\) wenigstens drei u.s.w. mehr als in \(X\); folglich werden, auch wenn in \(X\) gar kein Zeichenwechsel vorkommt, in \(y\) wenigstens so viele Zeichenwechsel sein, wie positive Wurzeln. Man sieht von selbst, dass wenn die Gleichung weder negative, noch imaginäre Wurzeln hat, man \(X=1\) zu setzen hat, und dieser Schluss seine Gültigkeit behält.

Es gehe \(y\), wenn den Coëfficienten der Potenzen \(x^{m-1}\), \(x^{m-3}\), \(x^{m-5}\) u.s.w. die entgegengesetzten Zeichen beigelegt werden, in \(y'\) über; sämmtliche Wurzeln der Gleichung \(y'=0\) werden dann den Wurzeln der Gleichung \(y = 0\) entgegengesetzt sein.  Es wird daher in \(y'\) wenigstens eben so viele Zeichenwechsel geben, als die Gleichung \(y = 0\)  negative Wurzeln hat.%pagebreak

Wir haben daher folgenden Lehrsatz:

\textit{Die Gleichung \(y=0\) kann nicht mehr positive Wurzeln haben, als es Zeichenwechsel in \(y\) gibt, und nicht mehr negative Wurzeln, als Zeichenwechsel in \(y'\) sind.}

Diese Einkleidung des Theorems scheint die zweckmässigste zu sein, da sie die grösste Einfachheit mit der umfassendsten Allgemeinheit vereinigt, und alle Gestalten des Satzes, die nur unter besondern Bedingungen gelten, von selbst daraus fliessen.

Will man die Grenze der Anzahl der negativen Wurzeln unmittelbar an den Zeichen der Coëfficienten von \(y\) erkennen, so wird es nothwendig, die \textit{unmittelbaren} Zeichenwechsel und Zeichenfolgen (bei Gliedern, wo die Exponenten von \(x\) um eine Einheit verschieden sind) von den durch fehlende Glieder \textit{unterbrochenen} zu unterscheiden. Offenbar wird jeder unmittelbare und jeder durch eine gerade Anzahl fehlender Glieder unterbrochene Zeichenwechsel in \(y'\) zu einer ähnlichen Zeichenfolge in \(y\), während ein durch eine ungerade Anzahl fehlender Glieder unterbrochener Zeichenwechsel in \(y'\) auch in \(y\) ein ähnlicher Zeichenwechsel bleibt. Der zweite Theil des Theorems lässt sich daher auch so ausdrücken:

Die Anzahl der negativen Wurzeln der Gleichung \(y=0\) kann nicht grösser sein, als die Anzahl der unmittelbaren und der durch eine gerade Anzahl fehlender Glieder unterbrochenen Zeichenfolgen, addirt zu der Anzahl der durch eine ungerade Anzahl fehlender Glieder unterbrochenen Zeichenwechsel in \(y\).

Fehlt in \(y\) gar kein Glied, so ist die Anzahl der negativen Wurzeln nicht grösser, als die Anzahl der Zeichenfolgen.

Bezeichnet man durch \(A\) die Anzahl der unmittelbaren Zeichenwechsel, und durch \(B\) die Anzahl der unmittelbaren Zeichenfolgen in \(y\), so wird, wenn kein Glied fehlt, \(A+B=m\) sein, also der Anzahl aller Wurzeln gleich. Insofern diese Zeichen also bloss lehren, dass die Anzahl der positiven Wurzeln nicht grösser als \(A\), und die der negativen nicht grösser als \(B\) sein kann, bleibt es unentschieden, ob oder wie viel imaginäre Wurzeln vorhanden sind. Weiss man aber anderswoher, dass die Gleichung keine imaginäre Wurzeln hat, so muss nothwendig \(A\) der Anzahl der positiven, und \(B\) der Anzahl der negativen Wurzeln gleich sein.

Anders aber verhält es sich, wenn in \(y\) Glieder fehlen.   Um mit Klarheit %pagebreak
zu übersehen, was sich daraus in Beziehung auf die imaginären Wurzeln schliessen lässt, bezeichnen wir durch \(a\) die Anzahl der durch eine gerade, durch \(c\) die Anzahl der durch eine ungerade Anzahl fehlender Glieder unterbrochenen Zeichenwechsel; durch \(b\) und \(d\) resp. die Anzahl der durch eine gerade und ungerade Anzahl fehlender Glieder unterbrochenen Zeichenfolgen in \(y\). Man sieht leicht, dass \(m-A-B-a-b-c-d\) der Anzahl sämmtlicher fehlender Glieder, die wir durch \(e\) bezeichnen wollen, gleich sein werde. Nun ist nach unserm Lehrsatze die Anzahl der positiven Wurzeln höchstens \(A+a+c\), die Anzahl der negativen höchstens \(B+b+c\), also die Anzahl aller reeellen Wurzeln höchstens
\(A+B+a+b+2c=m+c-d-e\)
Es muss daher die Anzahl der imaginären Wurzeln wenigstens \(e-c+d\) sein. 

Zählt man also alle fehlenden Glieder zusammen, jedoch so, dass man in jeder Lücke zwischen einem Zeichenwechsel eine Einheit weniger, zwischen einer Zeichenfolge aber eine Einheit mehr rechnet, als Glieder fehlen, so oft deren Anzahl ungerade ist, so erhält man eine Zahl, der die Anzahl der imaginären Wurzeln wenigstens gleich kommen muss.

\end{document}