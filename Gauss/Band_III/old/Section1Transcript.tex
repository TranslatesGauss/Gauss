\documentclass[14pt]{memoir}
\usepackage{standalone}
\usepackage[dvips,text={6.5truein,9.1truein},left=0.86truein,right=0.8truein,top=1truein]{geometry}
\usepackage{amsmath, amsthm, amsfonts}
\usepackage{titlesec}

% Uncomment to use syncing
%\usepackage{pdfsync}


% Paragraphs
\usepackage{indentfirst}
\parindent=3em
\parskip=0pt

%font
\usepackage{mlmodern}
%\usepackage[T1]{fontenc}% http://ctan.org/pkg/fontenc
\usepackage{microtype}

\titleformat{\section}
 {\centering}{\thesection.}{0em}{}

\titleformat{\subsection}
 {\normalfont\small\centering}{\thesection.}{0em}{}
\titlespacing*{\subsection}
{0pt}{\baselineskip}{0\baselineskip}

%footnotes
\usepackage[perpage]{footmisc}
\usepackage{etoolbox}
\DefineFNsymbols*{asterisks}{{ *}{ **}{ ***}}
\setfnsymbol{asterisks}
\renewcommand{\thefootnote}{\fnsymbol{footnote}}
\makeatletter
\renewcommand{\@makefnmark}{\mbox{\normalfont\@thefnmark})}
\settowidth{\footnotemargin}{***) }
\patchcmd{\@makefntext}{\hss\@makefnmark}{\hss \@makefnmark\ }{}{}
\makeatother

%Line Spacing
\renewcommand{\baselinestretch}{1.1}
\renewcommand{\footnotelayout}{ \baselineskip=1.02\baselineskip }
\setlength{\skip\footins}{2\baselineskip}

\theoremstyle{plain}
\newtheorem*{theorem}{Theorem}
\newtheorem{proposition}{Proposition}
\newtheorem{lemma}{Lemma}
\newtheorem{problem}{Problem}


\theoremstyle{remark}
\newtheorem*{example}{Example}
\newtheorem*{examples}{Examples}

\begin{document}

\setlength{\abovedisplayskip}{0.5\baselineskip}
\setlength{\belowdisplayskip}{0.5\baselineskip}


\section*{ \hspace{1pt} \\[3\baselineskip]
\begin{small}DEMONSTRATIO NOVA THEOREMATIS\end{small} \\[\baselineskip]
OMNEM FUNCTIONEM ALGEBRAICAM RATIONALEM INTEGRAM \\[\baselineskip]
 \begin{small} UNIUS VARIABILIS \end{small} \\[\baselineskip]
 \begin{scriptsize} IN FACTORES REALES PRIMI VEL SECUNDI GRADUS RESOLVI POSSE. \end{scriptsize}\\[\baselineskip]
 \rule{0.85in}{0.5pt}}
 


\subsection*{1.}

Quaelibet aequatio algebraica determinata reduci potest ad formam \[ x^m + Ax^{m-1} + Bx^{m-2} + \text{etc.} + M = 0 \] ita ut \(m\) sit numerus integer positivus. Si partem primam huius aequationis per \(X\) denotamus, aequationique \(X = 0\) per plures valores inaequales ipsius \(x\) satisfieri supponimus, puta ponendo \(x=\alpha\), \(x =\beta\), \(x = \gamma\) etc., functio \(X\) per productum e factoribus \(x-\alpha\), \(x-\beta\), \(x-\gamma\) etc. divisibilis erit. Vice versa, si productum e pluribus factoribus simplicibus \(x-\alpha\), \(x-\beta\), \(x-\gamma\) etc. functionem \(X\) metitur: aequationi \(X= 0\) satisfiet, aequando ipsam \(x\) cuicunque quantitatum \(\alpha, \beta, \gamma\) etc. Denique si \(X\) producto ex \(m\) factoribus talibus simplicibus aequalis est (sive omnes diversi sint, sive quidam ex ipsis identici): alii factores simplices praeter hos functionem \(X\) metiri non poterunt. Quamobrem aequatio \(m^{ti}\) gradus plures quam \(m\) radices habere nequit; simul vero patet, aequationem \(m^{ti}\) gradus \textit{pauciores} radices habere posse, etsi \(X\) in \(m\) factores simplices resolubilis sit: si enim inter hos factores aliqui sunt identici, multitudo modorum diversorum aequationi satisfaciendi necessario minor erit quam \(m\). Attamen concinnitatis caussa geometrae dicere maluerunt, aequationem in hoc quoque casu \(m\) radices habere, et tantummodo quasdam ex ipsis aequales inter se evadere: quod utique sibi permittere potuerunt.

\subsection*{2.}

Quae hucusque sunt enarrata, in libris algebraicis sufficienter demonstrantur neque rigorem geometricum uspiam offendunt. Sed nimis praepropere et sine praevia demonstratione solida adoptavisse videntur analystae theorema, cui tota fere doctrina aequationum superstructa est: \textit{Quamvis functionem talem ut \(X\) semper in \(m\) factores simplices resolvi posse}, sive hoc quod cum illo prorsus conspirat, \textit{quamvis aequationem \(m^{ti}\) gradus revera habere \(m\) radices}. Quum iam in aequationibus secundi gradus saepissime ad tales casus perveniatur, qui theoremati huic repugnant: algebraistae, ut hos illi subiicerent, coacti fuerunt, fingere quantitatem quandam imaginariam, cuius quadratum sit \(-1\), et tum agnoverunt, si quantitates formae \(a+b\surd{-1}\) perinde concedantur ut reales, theorema non modo pro aequationibus secundi gradus verum esse, sed etiam pro cubicis et biquadraticis. Hinc vero neutiquam inferre licuit, admissis quantitatibus formae \(a+b\surd{-1}\) cuivis aequationi quinti superiorisve gradus satisfieri posse, aut uti plerumque exprimitur (quamquam phrasin lubricam minus probarem) radices cuius vis aequationis ad formam \(a+b\surd{-1}\) reduci posse. Hoc theorema ab eo, quod in titulo huius scripti enunciatum est, nihil differt, si ad rem ipsam spectas, huiusque demonstrationem novam rigorosam tradere, constituit propositum praesentis dissertationis.

Ceterum ex eo tempore, quo analystae comperti sunt, infinite multas aequationes esse, quae nullam omnino radicem haberent, nisi quantitates formae \(a+b\surd{-1}\) admittantur, tales quantitates fictitiae tamquam peculiare quantitatum genus, quas imaginarias dixerunt, ut a realihus distinguerentur, consideratae et in totam analysin introductae sunt; quonam iure? hocloco non disputo. — Demonstrationem meam absque omni quantitatum imaginariarum subsidio absolvam, etsi eadem libertate, qua omnes recentiores analystae usi sunt, etiam mihi uti liceret.

\subsection*{3.}

Quamvis ea, quae in plerisque libris elementaribus tamquam demonstratio theorematis nostri afferuntur, tam levia sint, tantumque a rigore geometrico abhorreant, ut vix mentione sint digna: tamen, ne quid deesse videatur, paucis illa attingam. `Ut demonstrent, quamvis aequationem \[x^m + Ax^{m-1} + Bx^{m-2} + \text{etc.} + M = 0\] sive \(X= 0\), revera habere \(m\) radices, suscipiunt probare, \(X\) in \(m\) factores simplices resolvi posse. Ad hunc finem assumunt \(m\) factores simplices \(x-\alpha\), \(x-\beta\), \(x-\gamma\) etc. ubi \(\alpha\), \(\beta\), \(\gamma\) etc. adhuc sunt incognitae, productumque ex illis aequale ponunt functioni \(X\). Tum ex comparatione co\"efficientium deducunt \(m\) aequationes, ex quibus incognitas \(\alpha\), \(\beta\), \(\gamma\) etc. determinari posse aiunt, quippe quarum multitudo etiam sit \(m\). Scilicet \(m-1\) incognitas eliminari posse, unde emergere aequationem, quae, quam placuerit, incognitam solam contineat.'  Ut de reliquis, quae in tali argumentatione reprehendi possent, taceam, quaeram tantummodo, unde certi esse possimus, ultimam aequationem revera ullam radicem habere? Quidni fieri posset, ut neque huic ultimae aequationi neque propositae ulla magnitudo in toto quantitatum realium atque imaginariarum ambitu satisfaciat? — Ceterum periti facile perspicient, hanc ultimam aequationem necessario cum proposita \textit{omnino identicam} fore, siquidem calculus rite fuerit institutus; scilicet eliminatis incognitis \(\beta\), \(\gamma\) etc. aequationem \[ \alpha^m + A\alpha^{m-1} + B\alpha^{m-2} + \text{etc.} + M = 0 \] prodire debere.  Plura de isto ratiocinio exponere necesse non est.

Quidam auctores, qui debilitatem huius methodi percepisse videntur, tamquam \textit{axioma} assumunt, quamvis aequationem revera habere radices, si non possibiles, impossibiles. Quid sub quantitatibus possibilibus et impossibilibus intelligi velint, haud satis distincte exposuisse videntur. Si quantitates possibiles idem denotare debent ut reales, impossibiles idem ut imaginariae: axioma illud neutiquam admitti potest, sed necessario demonstratione opus habet. Attamen in illo sensu expressiones accipiendae non videntur, sed axiomatis mens haec potius videtur esse: `Quamquam nondum sumus certi, necessario dari \(m\) quantitates reales vel imaginarias, quae alicui aequationi datae \(m^{ti}\) gradus satisfaciant, tamen aliquantisper hoc supponemus; nam si forte contingeret, ut tot quantitates reales et imaginariae inveniri nequeant, certe effugium patebit, ut dicamus reliquas esse impossibiles.' Si quis hac phrasi uti mavult quam simpliciter dicere, aequationem in hoc casu tot radices non habituram, a me nihil obstat: at si tum his radicibus impossibilibus ita utitur tamquam aliquid veri sint, et e.g. dicit, summam omnium radicum aequationis \(x^m + A x^{m-1} + \text{etc.} = 0\), esse \(= -A\), etiamsi impossibiles inter illas sint (quae expressio proprio significat, \textit{etiamsi aliquae deficiant}): hoc neutiquam probare possum.   Nam radices impossibiles, in tali sensu acceptae, tamen sunt radices, et tum axioma illud nullo modo sine demonstratione admitti potest, neque inepte dubitares, annon aequationes exstare possint, quae ne impossibiles quidem radices habeant?\footnote{ Sub quantitate imaginaria hic semper intelligo quantitatem in forma \(a+b\surd{-1}\) contentam, quamdiu \(b\) non est \(= 0\). In hoc sensu expressio illa semper ab omnibus geometris primae notae accepta est, neque audiendos censeo, qui quantitatem \(a+b\surd{-1}\) in eo solo casu imaginariam vocare voluerunt, ubi \(a = o\), impossibilem vero quando non sit \(a = 0\), quum haec distinctio neque necessaria sit neque ullius utilitatis. — Si quantitates imaginariae omnino in analysi retineri debent (quod pluribus rationibus consultius videtur, quam ipsas abolere, modo satis solide stabiliantur): necessario tamquam aeque possibiles ac reales spectandae sunt; quamobrem reales et imaginarias sub denominatione communi \textit{quantitatum possibilium} complecti mallem: contra, \textit{impossibilem} dicerem quantitatem, quae conditionibus satisfacere debeat, quibus ne imaginariis quidem concessis satisfieri potest, attamen ita, ut \textit{phrasis} haec idem significet ac si dicas, talem quantitatem in toto magnitudinum ambitu non dari. Hinc vero genus peculiare quantitatum formare, neutiquam concederem. Quodsi quis dicat, triangulum rectilineum aequilaterum rectangulum impossibile esse, nemo erit qui neget. At si tale triangulum impossibile tamquam novum triangulorum genus contemplari, aliasque triangulorum proprietates ad illud applicare voluerit, ecquis risum teneat? Hoc esset verbis ludere seu potius abuti. — Quamvis vero etiam summi mathematici saepius veritates, quae quantitatum ad quas spectant possibilitatem manifesto supponunt, ad tales quoque applicaverint, quarum possibilitas adhuc dubia erat; neque abnuerim, huiusmodi licentias plerumque ad solam formam et quasi velamen ratiociniorum pertinere, quod veri geometrae acies mox penetrare possit: tamen consultius, scientiaeque, quae tamquam perfectissimum claritatis et certitudinis exemplar merito celebratur, sublimitate magis dignum videtur, tales libertates aut omnino proscribere, aut saltem parcius neque alias ipsis uti, nisi ubi etiam minus exercitati perspicere valeant, rem etiara absque illarum subsidio etsi forsan minus breviter tamen aeque rigorose absolvi potuisse. — Ceterum haud negaverim, ea quae hic contra impossibilium abusum dixi, quodam respectu etiam contra imaginarias obiici posse: sed harum vindicationem nee non totius huius rei expositionem uberiorem ad aliam occasionem mihi reservo.}

\subsection*{4.}

Antequam aliorum geometrarum demonstrationes theorematis nostri recenseam, et quae in singulis reprehendenda mihi videantur, exponam: observo sufficere, si tantummodo ostendatur, omni aequationi quantivis gradus \[x^m + Ax^{m-1} + B x^{m-2} + \text{etc.} + M = 0\]
sive \(X = 0\) (ubi co\"efficientes \(A, B\) etc. reales esse supponuntur) ad minimum uno modo satisfieri posse per valorem ipsius \(x\) sub forma \(a+b\surd{-1}\) contentum. Constat enim, \(X\) tunc divisibilem fore per factorem realem secundi gradus \(xx - 2ax + aa + bb\), si \(b\) non fuerit \(=0\), et per factorem realem simplicem \(x-a\), si \(b = 0\). In utroque casu quotiens erit realis, et inferioris gradus quam \(X\); et quum hic eadem ratione factorem realem primi secundive gradus habere debeat, patet, per continuationem huius operationis functionem \(X\) tandem in factores reales simplices vel duplices resolutum iri, aut, si pro singulis factoribus realibus duplicibus binos imaginarios simplices adhibere mavis, in \(m\) factores simplices.

\subsection*{5.}

Prima theorematis demonstratio illustri geometrae \textsc{d'Alembert} debetur, \textit{Recherches sur le calcul integral, Histoire de l'Acad. de Berlin, Ann\'ee} 1746. p. 182 sqq. Eadem extat in \textsc{Bougainville}, \textit{Trait\'e du calcul integral, \`a Paris} 1754. p. 47 sqq.   Methodi huius praecipua momenta haec sunt.

Primo ostendit, si functio quaecunque \(X\) quantitatis variabilis \(x\) fiat \(= 0\) aut pro \(x = 0\) aut pro \(x = \infty\), atque valorem infinite parvum realem positivum nancisci possit tribuendo ipsi x valorem realem: hanc functionem etiam valorem infinite parvum realem negativum obtinere posse per valorem ipsius \(x\) vel realem vel sub forma imaginaria \(p+q\surd{-1}\) contentum. Scilicet designante \(\Omega\) valorem infinite parvum ipsius \(X\), et \(\omega\) valorem respondentem ipsius \(x\), asserit \(\omega\) per seriem valde convergentem \(a\Omega^{\alpha} + b \Omega^{\beta} + c \Omega^{\gamma} \) etc. exprimi posse, ubi exponentes \(\alpha\), \(\beta\), \(\gamma\) etc. sint quantitates rationales continuo crescentes, et quae adeo ad minimum in distantia certa ab initio positivae evadant, terminosque, in quibus adsint, infinite parvos reddant. Iam si inter omnes hos exponentes nullus occurrat, qui sit fractio denominatoris paris, omnes terminos seriei reales fieri tum pro positive tum pro negative valore ipsius \(\Omega\); si vero quaedam fractiones denominatoris paris inter illos exponentes reperiantur, constare, pro valore negativo ipsius \(\Omega\) terminos respondentes in forma \(p+q\surd{-1}\) contentos esse. Sed propter infinitam seriei convergentiam in casu priori sufiicere, si terminus primus (i.e. maximus) solus retineatur, in posteriori ultra eum terminum, qui partem imaginariam primus producat, progredi opus non esse.

Per similia ratiocinia ostendi posse, si \(X\) valorem negativum infinite parvum ex valore reali ipsius \(x\) assequi possit: functionem illam valorem realem positivum infinite parvum ex valore reali ipsius \(x\) vel ex imaginario sub forma \(p+q\surd{-1}\)  contento adipisci posse.

Hinc secundo concludit, etiam valorem aliquem realem finitum ipsius \(X\) dari, in casu priori negativum, in posteriori positivum, qui ex valore imaginario ipsius \(x\) sub forma \(p+q\surd{-1}\)  contento produci possit.

Hinc sequitur, si \(X\) sit talis functio ipsius \(x\), quae valorem realem \(V\) ex valore ipsius \(x\) reali \(v\) obtineat, atque etiam valorem realem quantitate infinite parva vel maiorem vel minorem ex valore reali ipsius \(x\) assequatur, eandem etiam valorem realem quantitate infinite parva atque adeo finita vel minorem vel maiorem quam \(V\) (resp.) recipere posse, tribuendo ipsi \(x\) valorem sub forma \(p+q\surd{-1}\)  contentum. Hoc nullo negotio ex praecc. derivatur, si pro \(X\) substitui concipitur  \(V + Y\),  et pro \(x\), \(v+y\).

Tandem afiirmat ill. \textsc{d'Alembert}, si \(X\) totum intervallum aliquod inter duos valores reales \(R\), \(S\) percurrere posse supponatur (i.e. tum ipsi \(R\), tum ipsi \(S\), tum omnibus valoribus realibus intermediis aequalis fieri), tribuendo ipsi \(x\) valores semper in forma \(p+q\surd{-1}\)  contentos; functionem X quavis quantitate finita reali adhuc augeri vel diminui posse (prout \(S>R\) vel \(S<R\)), manente \(x\) semper sub forma \(p+q\surd{-1}\). Si enim quantitas realis \(U\) daretur (inter quam et \(R\) supponitur \(S\) iacere), cui \(X\) per talem valorem ipsius \(x\) aequalis fieri non posset, necessario valorem \textit{maximum} ipsius \(X\) dari (scilicet quando \(S>R\); minimum vero, quando \(S<R\)), puta \(T\), quem ex valore ipsius \(x\), \(p+q\surd{-1}\), consequeretur, ita ut ipsi \(x\) nullus valor sub simili forma contentus tribui posset, qui functionem \(X\) vel minimo excessu propius versus \(U\) promoveret. Iam si in aequatione inter \(X\) et \(x\) pro \(x\) ubique substituatur \(p+q\surd{-1}\) , atque tum pars realis, tum pars, quae factorem \(\surd{-1}\) implicet, hocomisso, cifrae aequentur: ex duabus aequationibus hinc prodeuntibus (in quibus \(p\), \(q\) et \(X\) cum constantibus permixtae occurreut) per eliminationem duas alias elici posse, in quarum altera \(p\), \(X\) et constantes reperiantur, altera a \(p\) libera solas \(q\), \(X\) et constantes involvat. Quamobrem quum \(X\) per valores reales ipsarum \(p, q\) omnes valores ab \(R\) usque ad \(T\) percurrerit, per praecc. \(X\) versus valorem \(U\) adhuc propius accedere posse tribuendo ipsius \(p\), \(q\) valores tales \(\alpha + \gamma \surd{-1}\), \(\beta + \delta\surd{-1}\) resp. Hinc vero fieri \(x = \alpha - \delta + (\gamma + \beta)\surd{-1}\), i.e. adhuc sub forma \(p+q\surd{-1}\) esse, contra hyp.

Iam si \(X\) functionem talem ut \(x^m + A x^{m-1} + B x^{m-2} + \text{etc.} + M\) denotare supponitur, nullo negotio perspicitur, ipsi \(x\) tales valores reales tribui posse, ut \(X\) totum aliquod intervallum inter duos valores reales percurrat. Quare \(x\) valorem aliquem sub forma \(p+q\surd{-1}\) contentum talem etiam nancisci poterit, unde \(X\) fiat \(=0\).   Q.E.D.\footnote{Observare convenit, ill. \textsc{d'Alembert} in sua huius demonstrationis expositione considerationes geometricas adhibuisse, atque \(X\) tamquam abscissam, \(x\) tamquam ordinatam curvae spectavisse (secundum morem omnium geometrarum primae huius saeculi partis, apud quos notio functionum minus usitata erat). Quia vero omnia ipsius ratiocinia, si ad ipsorum essentiam solam respicis, nuUis principiis geometricis, sed pure analyticis innituntur, et curva imaginaria, ordinataeque imaginariae expressiones duriores esse lectoremque hodiernum facilius offendere posse videntur, formam repraesentationis mere analyticam hic adhibere malui. Hanc annotationem ideo adieci, ne quis demonstrationem \textsc{d'Alembert}ianam ipsam cum hac succincta expositione comparans aliquid essentiale immutatum esse suspicetur.}

\subsection*{6.}

Quae contra demonstrationem \textsc{d'Alembert}ianam obiici posse videntur, ad haec fere redeunt.

1. Ill. \textsc{d'A}. nullum dubium movet de \textit{existentia} valorum ipsius \(x\), quibus valores dati ipsius \(X\) respondeant, sed illam supponit, solamque \textit{formam} istorum valorum investigat.

Quamvis vero haec obiectio per se gravissima sit, tamen hic ad solam dictionis formam pertinet, quae facile ita corrigi potest, ut illa penitus destruatur.

2. Assertio, \(\omega\) per talem seriem qualem ponit semper exprimi posse, certo est falsa, si \(X\) etiam functionem quamlibet transscendentem designare debet (uti \textsc{d'A.} pluribus locis innuit). Hoc e.g. manifestum est, si ponitur \(X = e^{\frac{1}{x}}\), sive \(X = \frac{1}{\log X}\). Attamen si demonstrationem ad eum casum restringimus, ubi X est functio algebraica ipsius \(x\) (quod in praesenti negotio sufficit), propositio utique est vera. — Ceterum d'A. nihil pro confirmatione suppositionis suae attulit; cel. \textsc{Bougainville} supponit, \(X\) esse functionem algebraicam ipsius \(x\), et ad inventionem seriei parallelogrammum \textsc{Newton}ianum commendat.

3. Quantitatibus infinite parvis liberius utitur, quam cum geometrico rigore consistere potest aut saltem nostra aetate (ubi illae merito male audiunt) ab analysta scrupuloso concederetur, neque etiam saltum a valore infinite parvo ipsius \(\Omega\) ad finitum satis luculenter explicavit. Propositionem suam, \(\Omega\) etiam valorem aliquem finitum consequi posse, non tam ex possibilitate valoris infinite parvi ipsius \(\Omega\) concludere videtur quam inde potius, quod denotante \(\Omega\) quantitatem valde parvam, propter magnam seriei convergentiam, quo plures termini seriei accipiantur, eo propius ad valorem verum ipsius \(\omega\) accedatur, aut, quo plurium partium summa pro \(\omega\) accipiatur, eo exactius aequationi, quae relationem inter \(\omega\) et \(\Omega\) sive \(x\) et \(X\) exhibeat, satisfactum iri. Praeterea quod tota haec argumentatio nimis vaga videtur, quam ut ulla conclusio rigorosa inde colligi possit: observo, utique dari series, quae quantumvis parvus valor quantitati, secundum cuius potestates progrediuntur, tribuatur, nihilominus semper divergant, ita ut si modo satis longe continuentur, ad terminos quavis quantitate data maiores pervenire possis\footnote{Hacce occasione obiter adnoto, ex harum serierum numero plurimas esse, quae primo aspectu maxime convergentes videantur, e.g. ad maximam partem eas, quibus ill. \textsc{Euler} in parte poster. \textit{Inst. Calc. Diff.} Cap.VI. ad summam aliarum serierum quam proxime assignandam utitur p. 441-474 (reliquae enim series p. 475 — 478 revera convergere possunt), quod, quantum scio, a nemine hucusque observatum est. Quocirca magnopere optandum esset, ut dilucide et rigorose ostenderetur, cur huiusmodi series, quae primo citissime, dein paullatim lentius lentiusque convergunt, tandemque magis magisque divergunt, nihilominus summam proxime veram suppeditent, si modo non nimis multi termini capiantur, et quousque talis summa pro exacta tuto haberi possit?}. Hoc evenit, quando co\"efficientes seriei progressionem hypergeometricam constituunt. Quamobrem necessario demonstrari debuisset, talem seriem hypergeometricam in casu praesenti provenire non posse.

Ceterum mihi videtur, ill. \textsc{d'A.} hic non recte ad series infinitas confugisse, hasque ad stabiliendum theorema hoc fundamentale doctrinae aequationum haud idoneas esse.

4. Ex suppositione, \(X\) obtinere posse valorem \(S\) neque vero valorem \(U\), nondum sequitur, inter \(S\) et \(U\) necessario valorem \(T\) iacere, quem \(X\) attingere sed non superare possit. Superest adhuc alius casus: scilicet fieri posset, ut inter \(S\) et \(U\) limes situs sit, ad quem accedere quidem quam prope velis possit \(X\), ipsum vero nihilominus numquam attingere. Ex argumentis ab ill. \textsc{d'A.} allatis tantummodo sequitur, \(X\) omnem valorem, quem attigerit, adhuc quantitate finita superare posse, puta quando evaserit \(= S\), adhuc quantitate aliqua finita \(\Omega\) augeri posse; quo facto, novum incrementum \(\Omega'\) accedere, tunc iterum augmentum \(\Omega''\) etc., ita ut quotcunque incrementa iam adiecta sint, nullum pro ultimo haberi debeat, sed semper aliquod novum accedere possit. At quamvis multitudo incrementorum possibilium nullis limitibus sit circumscripta: tamen utique fieri posset, ut si incrementa \(\Omega\), \(\Omega'\), \(\Omega''\) etc. continuo decrescerent, nihilominus summa \(S +\Omega + \Omega' + \Omega''\) etc. limitem aliquem nunquam attingeret, quotcunque termini considerentur.

Quamquam hic casus occurrere non potest, quando \(X\) designat functionem algebraicam integram ipsius \(x\): tamen sine demonstratione, hoc fieri non posse, mefhodus necessario pro incompleta habenda est. Quando vero \(X\) est functio transscendens, sive etiam algebraica fracta, casus ille utique locum habere potest, e.g. semper quando valori cuidam ipsius \(X\) valor infinite magnus ipsius \(x\) respondet. Tum methodus \textsc{d'Alembert}iana non sine multis ambagibus, et in quibusdam casibus nullo forsan modo, ad principia indubitata reduci posse videtur.


Propter has rationes demonstrationem \textsc{d'Alembert}ianam pro satisfaciente habere nequeo. Attamen hoc non obstante verus demonstrationis nervus probandi per omnes obiectiones neutiquam infringi mihi videtur, credoque eidem fundamento (quamvis longe diversa ratione, et saltem maiori circumspicientia) non solum demonstrationem rigorosam theorematis nostri superstrui, sed ibinde omnia peti posse, quae circa aequationum \textit{transscendentium} theoriam desiderari queant. De qua re gravissima alia occasione fusius agam; conf. interim infra art. 24.

\subsection*{7.}

Post \textsc{d'Alembertum} ill. \textsc{Euler} disquisitiones suas de eodem argumento promulgavit, \textit{Recherches sur les racines imaginaires des equations, Hist. de l'Acad. de Berlin A.} 1749, p. 223 sqq. Methodum duplicem hic tradidit: prioris summa continetur in sequentibus.

Primo ill. \textsc{E.} suspicit demonstrare, si \(m\) denotet quamcunque dignitatem numeri \(2\), functionem \(x^{2m} + Bx^{2m-2} + C x^{2m-3} + \text{etc.} + M = X\) (in qua co\"efficiens termini secundi est \(= 0\)) semper in duos factores reales resolvi posse, in quibus \(x\) usque ad \(m\) dimensiones ascendat.  Ad hunc finem duos factores assumit, \[x^m - u x^{m-1} + \alpha x^{m-2} + \beta x^{m-3} + \text{etc., et} x^m + u x^{m-1}+\lambda x^{m-2} + \mu x^{m-3} + \text{etc.} \] ubi co\"efficientes \(u\), \(\alpha\), \(\beta\), etc. \(\lambda\), \(\mu\) etc. adhuc incogniti sunt, horumque productum aequale ponit functioni \(X\). Tum co\"efficientium comparatio suppeditat \(2m-1\) aequationes, manifestoque demonstrari tantummodo debet, incognitis \(u\), \(\alpha\), \(\beta\), etc. \(\lambda\), \(\mu\) (quarum multitudo etiam est \(2m -1\)) tales valores reales tribui posse, qul aequationibus illis satisfaciant. Iam \textsc{E.} affirmat, si primo \(u\) tamquam cognita consideretur, ita ut multitudo incognitarum unitate minor sit quam multitudo aequationum, his secundum methodos algebraicas notas rite combinatis omnes \(\alpha\), \(\beta\) etc. \(\lambda\), \(\mu\) etc. rationaliter et sine ulla radicum extractione per \(u\) et co\"efficientes \(B, C\) etc. determinari posse, adeoque valores reales nancisci, simulac \(u\) realis fiat. Praeterea vero omnes \(\alpha\), \(\beta\) etc. \(\lambda\), \(\mu\) etc. eliminari poterunt, ita ut prodeat aequatio \(U = 0\), ubi \(U\) erit functio integra solius \(u\) et co\"efficientium cognitorum. Hanc aequationem ipsam per methodum eliminationis vulgarem evolvere, opus immensum foret, quando aequatio proposita \(X = 0\)  est gradus aliquantum alti; et pro gradu indeterminato, plane impossibile (iudice ipso \textsc{E.} p. 239). Attamen hic sufficit, unam illius aequationis proprietatem novisse, scilicet quod terminus ultimus in \(U\) (qui incognitam \(u\) non implicat) necessario est negativus, unde sequi constat, aequationem ad minimum unam radicem realem habere, sive \(u\) et proin etiam  \(\alpha\), \(\beta\) etc. \(\lambda\), \(\mu\) etc. ad minimum uno modo realiter determinari posse: illam vero proprietatem per sequentes reflexiones confirmare licet.  Quum \(x^m - u x^{m-1}+\alpha x^{m-2} +\) etc. supponatur esse factor functionis \(X\):  necessario \(u\) erit summa \(m\) radicum aequationis \(X = 0\), adeoque totidem valores habere debebit, quot modis diversis ex \(2m\) radicibus \(m\) excerpi possunt, sive per principia calculi combinationum \(\frac{2m . 2m-1 . 2m-2 . \dots . m+1}{1.2.3 \dots m}\) valores.   Hic numerus semper erit impariter par (demonstrationem haud difficilem supprimo): si itaque ponitur \(= 2k\), ipsius semissis \(k\) impar erit; aequatio \(U= 0\) vero erit gradus \(2k^{ti}\). Iam quoniam in aequatione \(X= 0\) terminus secundus deest: summa omnium \(2m\) radicum erit \(0\); unde patet, si summa quarumcunque \(m\) radicum fuerit \(+p\), reliquarum summam fore \(-p\), i.e. si \(+p\) est inter valores ipsius \(u\), etiam \(-p\) inter eosdem erit. Hinc \textsc{E.} concludit, \(U\) esse productum ex \(k\) factoribus duplicibus talibus \(uu - pp\), \(uu - qq\), \(uu - rr\) etc., denotantibus \(+p\), \(-p\), \(+q\), \(-q\) etc. omnes \(2k\) radices aequationis \(U=0\),  unde, propter multitudinem iraparem herum factorum, terminus ultimus in \(U\) erit quadratum producti \(pqr\) etc. signo negativo affectum. Productum autem \(pqr\) etc. semper ex co\"efficientibus \(B, C\) etc. rationaliter determinari potest, adeoque necessario erit quantitas realis. Huius itaque quadratum signo negativo affectum certo erit quantitas negativa.   Q.E.D.

Quum hi duo factores reales ipsius \(X\) sint gradus \(m^{ti}\) atque \(m\) potestas numeri \(2\): eadem ratione uterque rursus in duos factores reales \(\tfrac{1}{2}m\) dimensionum resolvi poterit. Quoniam vero per repetitam dimidiationem numeri m necessario tandem ad binarium pervenitur, manifestum est, per continuationem operationis functionem \(X\) tandem in factores reales secundi gradus resolutam haberi.

Quodsi vero functio talis proponitur, in qua terminus secundus non deest, puta \(x^{2m} + A x^{2m-1} + B x^{2m-2} + \text{etc.} + M\), designante etiamnum \(2m\) potestatem binariam, haec per substitutionem \(x = y - \frac{A}{2m}\) transibit in similem functionem termino secundo carentem. Unde facile concluditur, etiam illam functionem in factores reales secundi gradus resolubilem esse.

Denique proposita functione gradus \(n^{ti}\) designante \(n\) numerum, qui non est potestas binaria: ponatur potestas binaria proxime maior quam \(n\), \(= 2m\), multipliceturque functio proposita per \(2 m - n\) factores simplices reales quoscunque. Ex resolubilitate producti in factores reales secundi gradus, nullo negotio derivatur, etiam functionem propositam in factores reales secundi vel primi gradus resolubilem esse debere.

\subsection*{8.}

Contra hanc demonstrationem obiici potest

1.  Regulam, secundum quam \textsc{E.} concludit, ex \(2m-1\) aequationibus \(2m-2\) incognitas \(\alpha\), \(\beta\) etc. \(\lambda\), \(\mu\) etc. omnes rationaliter determinari posse, neutiquam esse generalem, sed saepissime exceptionem pati. Si quis e.g. in art. 3, aliqua incognitarum tamquam cognita spectata, reliquas per hanc et co\"efficientes datos rationaliter exprimere tentat, facile inveniet, hoc esse impossibile, nullamque quantitatum incognitarum aliter quam per aequationem \((m-1)^{ti}\) gradus determinari posse. Quamquam vero hic statim a priori perspici potest, illud necessario ita evenire debuisse: tamen merito dubitari posset, annon etiam in casu praesenti pro quibusdam valoribus m res eodem modo se habeat, ut incognitae \(\alpha\), \(\beta\) etc. \(\lambda\), \(\mu\) etc. ex \(u\), \(B\), \(C\) etc. aliter quam per aequationem gradus forsan maioris quam \(2m\) determineri nequeant. Pro eo casu, ubi aequatio \(X = 0\) est quarti gradus, \textsc{E.} valores rationales co\"efficientium per \(u\) et co\"efficientes datos eruit; idem vero etiam in omnibus aequationibus altioribus fieri posse, utique explicatione ampliori egebat.   Ceterum operae pretium esse videtur, in formulas illas, quae \(\alpha\), \(\beta\) etc. rationaliter per \(u, B, C\) etc. exprimant, profundius et generalissime inquirere; de qua re aliisque ad eliminationis theoriam (argumentum haudquaquam exhaustum) pertinentibus alia occasione fusius agere suscipiam.

2.  Etiamsi autem demonstratum fuerit, cuiusvis gradus sit aequatio \(X = 0\), semper formulas inveniri posse, quae ipsas \(\alpha\), \(\beta\),  etc. \(\lambda\), \(\mu\) etc. rationaliter per \(u\), \(B\), \(C\) etc. exhibeant: tamen certum est, pro valoribus quibusdam determinatis co\"efficientium \(B\), \(C\) etc. formulas illas \textit{indeterminatas} evadere posse, ita ut non solum impossibile sit, incognitas illas rationaliter ex \(u\), \(B\), \(C\) etc. definire, sed adeo revera quibusdam in casibus valori alicui reali ipsius \(u\) nulli valores reales ipsarum \(\alpha\), \(\beta\) etc. \(\lambda\), \(\mu\) etc. respondeant.  Ad confirmationem huius rei brevitatis gratia ablego lectorem ad diss. ipsam \textsc{E.}, ubi p. 236 aequatio quarti gradus fusius explicata est.   Statim quisque videbit, formulas pro co\"efficientibus \(\alpha\), \(\beta\) indeterminatas fieri, si \(C = 0\) et pro \(u\) assumatur valor \(0\), illorumque valores non solum sine extractione radicum assignari non posse, sed adeo ne reales quidem esse, si fuerit \(BB-4D\) quantitas negativa. Quamquam vero in hoc casu \(u\) adhuc alios valores reales habere, quibus valores reales ipsarum \(\alpha\), \(\beta\) respondeant, facile perspici potest: tamen vereri aliquis posset, ne huius difficultatis enodatio (quam \textsc{E.} omnino non attigit) in aequationibus altioribus multo maiorem operam facessat. Certe haec res in demonstratione exacta neutiquam silentio praeteriri debet.

3. Ill. \textsc{E.} supponit tacite, aequationem \(X=0\) habere \(2m\) radices, harumque summam statuit \(= 0\), ideo quod terminus secundus in \(X\) abest. Quomodo de hac licentia (qua omnes auctores de hoc argumento utuntur) sentiam, iam supra art. 3 declaravi. Propositio, summam omnium radicum aequationes alicuius co\"efficienti primo, mutato signo, aequalem esse, ad alias aequationes applicanda non videtur, nisi quae radices habent: iam quum per hanc ipsam demonstrationem evinci debeat, aequationem \(X = 0\) revera radices habere, haud permissum videtur, harum existentiam supponere. Sine dubio ii, qui huius paralogismi fallaciam nondum penetraverunt, respondebunt, \textit{hic non demonstrari, aequationi \(X= 0\) satisfieri posse} (nam hoc dicere vult expressio, eam habere radices), \textit{sed tantummodo, ipsi per valores ipsius \(x\) sub forma \(a + b \surd{-1}\) contentos satisfieri posse; illud vero tamquam axioma supponi}. At quum aliae quantitatum formae, praeter realem et imaginariam \(a + b \surd{-1}\) concipi nequeant, non satis luculentum videtur, quomodo id, quod demonstrari debet, ab eo, quod tamquam axioma supponitur, differat; quin adeo si possibile esset adhuc alias formas quantitatum excogitare, puta formam \(F\), \(F'\), \(F''\) etc.: tamen sine demonstratione admitti non deberet, cuius aequationi per aliquem valorem ipsius \(x\) aut realem, aut sub forma \(a + b \surd{-1}\) aut sub forma \(F\), aut sub \(F'\) etc. contentum satisfieri posse. Quamobrem axioma illud alium sensum habere nequit quam hunc: Cuivis aequationi satisfieri potest \textit{aut} per valorem realem incognitae, \textit{aut} per valorem imaginarium sub forma \(a + b \surd{-1}\) contentum, \textit{aut} forsan per valorem sub forma alia hucusque ignota contentum, \textit{aut} per valorem, qui sub nulla omnino forma continetur. Sed quomodo huiusmodi quantitates, de quibus ne ideam quidem fingere potes — vera umbrae umbra — summari aut multiplicari possint, hoc ea perspicuitate, quae in mathesi semper postulatur, certo non intelligitur\footnote{Tota haec res multum illustrabitur per aliam disquisitlonem sub prelo iam sudantera, ubi in argumento longe quidem diverso, nihilominus tamen analogo, licentiam similem prorsus eodem iure usurpare potuissem, ut hic in aequationibus ab omnibus analystis factum est. Quamquam vero plurium veritatum demonstrationes adiumento talium fictionum paucis verbis absolvere licuisset, quae absque bis perquam difficiles evadunt et subtilissima artificia requirunt, tamen illis omnino abstinere malui, speroque, paucis me satisfacturum fuisse, si analystarum methodum imitatus essem.}.

Ceterum conclusiones, quas \textsc{E.} ex suppositione sua elicuit, per has obiectiones haudquaquam suspectas reddere volo; quin potius certus sum, illas per methodum neque difficilem neque ab \textsc{Euler}iana multum diversam ita comprobari posse, ut nemini vel minimus scrupulus superesse debeat. Solam \textit{formam} reprehendo, quae quamvis in \textit{inveniendis} novis veritatibus magnae utilitatis esse possit, tamen in \textit{demonstrando}, coram publico, minime probanda videtur.

4. Pro demonstratione assertionis, productum \(pqr\) etc. ex co\"efficientibus in \(X\) rationaliter determinari posse, ill. \textsc{E.} nihil omnino attulit. Omnia, quae hac de re in aequationibus \textit{quarti gradus} explicat, haec sunt (ubi \(\mathfrak{a}\), \(\mathfrak{b}\), \(\mathfrak{c}\), \(\mathfrak{d}\) sunt radices aequationis propositae \(x^4 + B xx + C x + D = 0\)):

`On m'objectera sans doute, que j'ai suppos\'e ici, que la quantit\'e \(pqr\) etait une quantit\'e r\'eelle, et que son quarr\'e \(ppqqrr\) \'etait affirmatif; ce qui \'etait encore douteux, vu que les racines \(\mathfrak{a}\), \(\mathfrak{b}\), \(\mathfrak{c}\), \(\mathfrak{d}\) etant imaginaires, il pourrait bien arriver, que le quarr\'e de la quantit\'e \(pqr\), qui en est compos\'ee, fut n\'egatif. Or je r\'eponds \`a cela que ce cas ne saurait jamais avoir lieu; car quelque imaginaires que soient les racines \(\mathfrak{a}\), \(\mathfrak{b}\), \(\mathfrak{c}\), \(\mathfrak{d}\), on sait pourtant, qu'il doit y avoir \(\mathfrak{a} + \mathfrak{b} + \mathfrak{c} + \mathfrak{d} = 0 \); \(\mathfrak{a}\mathfrak{b} + \mathfrak{a}\mathfrak{c} + \mathfrak{a}\mathfrak{d} + \mathfrak{b}\mathfrak{c} + \mathfrak{b}\mathfrak{d} + \mathfrak{c}\mathfrak{d} = B\); \(\mathfrak{a}\mathfrak{b}\mathfrak{c}+\mathfrak{a}\mathfrak{b}\mathfrak{d}+\mathfrak{a}\mathfrak{c}\mathfrak{d}+\mathfrak{b}\mathfrak{c}\mathfrak{d} = -C\) \footnote{\textsc{E.} per errorem habet \(C\), unde etiam postea perperam statuit \(pqr = C\).}; \(\mathfrak{a}\mathfrak{b}\mathfrak{c}\mathfrak{d}=D\), ces quantites \(B\), \(C\), \(D\) \'etant r\'eelles. Mais puisque \(p = \mathfrak{a} + \mathfrak{b}\), \(q =\mathfrak{a} + \mathfrak{c}\), \(r = \mathfrak{a} + \mathfrak{d}\), leur produit \(pqr = (\mathfrak{a} + \mathfrak{b})(\mathfrak{a} + \mathfrak{c})(\mathfrak{a} + \mathfrak{d})\) est d\'eterminable \textit{comme on sait}, par les quantit\'es \(B\), \(C\), \(D\), et sera par cons\'equent r\'eel, tout comme nous avons vu, qu'il est effectivement \(pqr = -C\), et \(ppqqrr = CC\). On reconna\^{\i}tra ais\'ement de m\^eme, que dans les plus hautes \'equations cette m\^eme circonstance doit avoir lieu, et qu'on ne saurait me faire des objections de ce c\^ot\'e.'  Conditionem, productum \(pqr\) etc. rationaliter per \(B\), \(C\) etc. determinari posse, \textsc{E.} nullibi adiecit, attamen semper subintellexisse videtur, quum absque illa demonstratio nullam vim habere possit.  Iam verum quidem est in aequationibus quarti gradus, si productum \( (\mathfrak{a} + \mathfrak{b})(\mathfrak{a} + \mathfrak{c})(\mathfrak{a} + \mathfrak{d})\) evoluatur, obtineri \( \mathfrak{a}\mathfrak{a}(\mathfrak{a} + \mathfrak{b} + \mathfrak{c} + \mathfrak{d}) \mathfrak{a}\mathfrak{b}\mathfrak{c}+\mathfrak{a}\mathfrak{b}\mathfrak{d}+\mathfrak{a}\mathfrak{c}\mathfrak{d}+\mathfrak{b}\mathfrak{c}\mathfrak{d}= - C\), attamen non satis perspicuum videtur, quomodo in omnibus aequationibus superioribus productum rationaliter per co\"efficientes determinari possit. Clar. \textsc{de Foncenex}, qui primus hoc observavit (\textit{Miscell. phil. math. soc. Taurin. T. I. p. 117}), recte contendit, sine demonstratione rigorosa huius propositionis methodum omnem vim perdere, illam vero satis difficilem sibi videri confitetur, et quam viam frustra tentaverit, enarrat\footnote{In hanc expositionem error irrepsisse videtur, scilicet p. 118. 1. 5. loco characteris (on choisissait seulement Celles oü entrait p etc.), necessario legere oportet, une mömc racine quelconque de Pequation in-oposee, aut simile quid, quum illud nullum sensum habeat.}. Attamen haec res haud difficulter per methodum sequentem (cuius summam addigitare tantummodo hic possum) absolvitur: Quamquam in aequationibus quarti gradus non satis clarum est, productum \( (\mathfrak{a} + \mathfrak{b})(\mathfrak{a} + \mathfrak{c})(\mathfrak{a} + \mathfrak{d})\) per co\"efficientes \(B\), \(C\), \(D\) determinabile esse, tamen facile perspici potest, idem productum etiam esse \(= (\mathfrak{b} + \mathfrak{a})(\mathfrak{b} + \mathfrak{c})(\mathfrak{b} + \mathfrak{d})\), nec non \( = (\mathfrak{c} + \mathfrak{a})(\mathfrak{c} + \mathfrak{b})(\mathfrak{c} + \mathfrak{d})\) denique etiam \( = (\mathfrak{d} + \mathfrak{a})(\mathfrak{d} + \mathfrak{b})(\mathfrak{d} + \mathfrak{c})\). Quare productum \(pqr\) erit quadrans summae \((\mathfrak{b} + \mathfrak{a})(\mathfrak{b} + \mathfrak{c})(\mathfrak{b} + \mathfrak{d})+(\mathfrak{c} + \mathfrak{a})(\mathfrak{c} + \mathfrak{b})(\mathfrak{c} + \mathfrak{d})+(\mathfrak{d} + \mathfrak{a})(\mathfrak{d} + \mathfrak{b})(\mathfrak{d} + \mathfrak{c})\), quam, si evolvatur, fore functionem rationalem integram radicum \(\mathfrak{a}\), \(\mathfrak{b}\), \(\mathfrak{c}\), \(\mathfrak{d}\) talem, in quam omnes eadem ratione ingrediantur, nullo negotio a priori praevideri potest.  Tales vero functiones semper rationaliter per co\"efficientes aequationis, cuius radices sunt \(\mathfrak{a}\), \(\mathfrak{b}\), \(\mathfrak{c}\), \(\mathfrak{d}\), exprimi possunt. — Idem etiam manifestum est, si productum \(pqr\)  sub hanc formam redigatur:
\[ \tfrac{1}{2}\left( \mathfrak{a} + \mathfrak{b} - \mathfrak{c} - \mathfrak{d}  \right) \times  \tfrac{1}{2}\left( \mathfrak{a} + \mathfrak{c} - \mathfrak{b} - \mathfrak{d}  \right)  \times  \tfrac{1}{2}\left( \mathfrak{a} + \mathfrak{d} - \mathfrak{b} - \mathfrak{c}  \right)  \]
quod productum evolutum omnes \(\mathfrak{a}\), \(\mathfrak{b}\), \(\mathfrak{c}\), \(\mathfrak{d}\) eodem modo implicaturum esse facile praevideri potest.   Simul periti facile hinc colligent, quomodo hoc ad altiores aequationes applicari debeat.   Completam demonstrationis expositionem, quam hic apponere brevitas non permittit, una cum uberiori disquisitione de functionibus plures variabiles eodem modo involventibus ad aliam occasionem mihi reservo.

Ceterum observo, praeter has quatuor obiectiones, adhuc quaedam alia in demonstratione \textsc{E.} reprehendi posse, quae tamen silentio praetereo, ne forte censor nimis severus esse videar, praesertim quum praecedentia satis ostendere videantur, demonstrationem in ea quidem forma, in qua ab \textsc{E.} proposita est, pro completa neutiquam haberi posse.

Post hanc demonstrationem, \textsc{E.} adhuc aliam viam theorema pro aequationibus, quarum gradus non est potestas binaria, ad talium aequationum resolutionem reducendi ostendit: attamen quum methodus haec pro aequationibus quarum gradus est potestas binaria, nihil doceat, insuperque omnibus obiectionibus praecc. (praeter quartam) aeque obnoxia sit ut demonstratio prima generalis: haud necesse est illam hic fusius explicare.
 
 
\subsection*{9.}

In eadem commentatione ill. \textsc{E.} theorema nostrum adhuc alia via confirmare annixus est p. 263, cuius summa continetur in his: Proposita aequatione \(x^n + Ax^{n-1} + B x^{n-2} \text{ etc.} = 0\), hucusque quidem expressio analytica, quae ipsius radices exprimat, inveniri non potuit, si exponens \(n>4\); attamen certum esse videtur (uti asserit \textsc{E.}), illam nihil aliud continere posse, quam operationes arithmeticas et extractiones radicum eo magis complicatas, quo maior sit \(n\). Si hoc conceditur, \textsc{E.} optime ostendit, quantumvis inter se complicata sint signa radicalia, tamen formulae valorem semper per formam \(M+N\surd{-1}\) repraesentabilem fore, ita ut \(M, N\) sint quantitates reales.

Contra hoc ratiocinium obiici potest, post tot tantorum geometrarum labores perexiguam spem superesse, ad resolutionem generalem aequationum algebraicarum umquam perveniendi, ita ut magis magisque verisimile fiat, talem resolutionem omnino esse impossibilem et contradictoriam. Hoc eo minus paradoxum videri debet, \textit{quum id, quod vulgo resolutio aequationis dicitur, proprie nihil aliud sit quam ipsius reductio ad aequationes puras}. Nam aequationum purarum solutio hinc non docetur sed supponitur, et si radicem aequationis \(x^m = H \) per \(\sqrt[m]{H}\) exprimis, illam neutiquam solvisti, neque plus fecisti, quam si ad denotandam radicem aequationis \(x^n + Ax^{n-1}+ \text{ etc.} = 0\) signum aliquod excogitares, radicemque huic aequalem poneres. Verum est, aequationes puras propter facilitatem ipsarum radices per approximationem inveniendi, et propter nexum elegantem, quem omnes radices inter se habent, prae omnibus reliquis multum praestare, adeoque neutiquam vituperandum esse, quod analystae harum radices per signum peculiare denotaverunt: attamen ex eo, quod hoc signum perinde ut signa arithmetica additionis, subtractionis, multiplicationis, divisionis et evectionis ad dignitatem sub nomine \textit{expressionum analyticarum} complexi sunt, minime sequitur, cuiusvis aequationis radicem per illas exhiberi posse. Seu, missis verbis, sine ratione sufficienti supponitur, cuiusvis aequationis solutionem ad solutionem aequationum purarum reduci posse. Forsan non ita difficile foret, impossibilitatem iam pro quinto gradu omni rigore demonstrare, de qua re alio loco disquisitiones meas fusius proponam. Hic sufficit, resolubilitatem generalem aequationum, in illo sensu acceptam , adhuc valde dubiam esse, adeoque demonstrationem, cuius tota vis ab illa suppositione pendet, in praesenti rei statu nihil ponderis habere.

\subsection*{10.}

Postea etiam clar. \textsc{de Foncenex}, quum in demonstratione prima \textsc{Euleri} defectum animadvertisset (supra art. 8 obiect. 4), quem tollere non poterat, adhuc aliam viam tentavit et in comment. laudata p. 120 in medium protulit\footnote{In tomo secundo eorundem Miscellaneorum p. 3 37 dilucidationes ad hanc commentationem continentur: attamen hae ad disquisitionem praesentem non pertinent, sed ad logarithmos quantitatum negativarum, de quibus in eadem coram. sermo fuerat.}. Quae consistit in sequentibus.

Proposita sit aequatio \(Z = 0\) , designante \(Z\) functionem \(m^{ti}\) gradus incognitae \(z\). Si \(m\) est numerus impar, iam constat, aequationem hanc habere radicem realem; si vero \(m\) est par, clar. \textsc{F.} sequenti modo probare conatur, aequationem ad minimum unam radicem formae \(p+q\surd-1\) habere. Sit \(m=2^n i\), designante \(i\) numerum imparem, supponaturque \(zz + uz + M\) esse divisor functionis\(Z\). Tunc singuli valores ipsius \(u\) erunt summae binarum radicum aequationis \(Z = 0\) (mutato signo), quamobrem \(u\) habebit \(\frac{m . m-1}{1 . 2} = m'\) valores, et si \(u\) per aequationem \(U=0\) determinari supponitur (designante \(U\) functionem integram ipsius \(u\) et co\"efficientium cognitorum in \(Z\)), haec erit gradus \({m'}^{ti}\). Facile vero perspicitur, \(m'\) fore numerum formae \(2^{n-1} i' \), designante \(i'\) numerum imparem. Iam nisi \(m'\) est impar, supponatur iterum, \(uu + u'u + M'\) esse divisorem ipsius \(U\), patetque per similia ratiocinia, \(u'\) determinari per aequationem \(U'= 0\), ubi \(U'\) sit functio \(\frac{m' . m'-1}{1 . 2}^{ti}\) gradus ipsius \(u'\). Posito vero \(\frac{m' . m'-1}{1 . 2} = m''\), erit \(m''\) numerus formae \(2^{n-2}i''\), designante \(i''\) numerum imparem. Iam nisi \(m''\) est impar, statuatur \(u'u'+u''u'+M''\) esse divisorem functionis \(U'\) determinabiturque \(u''\) per aequationem \(U'' = 0\), quae si supponitur esse gradus \({m'''}^{ti}\) erit numerus formae \(2^{n-3}i'''\). Manifestum est, in serie aequationum \(U=0\), \(U'=0\), \(U''=0\) etc. \(n^{tam}\) fore gradus imparis adeoque radicem realem habere. Statuemus brevitatis gratia \(n = 3\), ita ut aequatio \(U'' = 0\) radicem realem \(u''\) habeat, nullo enim negotio perspicitur, pro quovis alio valore ipsius \(n\) idem ratiocinium valere. Tunc co\"efficientem \(M''\) per \(u''\) et co\"efficientes in \(U'\) (quos fore functiones integras co\"efficientium in \(Z\) facile intelligitur), sive per \(u''\) et co\"efficientes in \(Z\) rationaliter determinabilem fore asserit clar. de \textsc{F.}, et proin realem. Hinc sequitur, radices aequationis \(u'u'+u''u'+M''=0\) sub forma \(p+q\surd{-1}\) contentas fore; eaedem vero manifesto aequationi \(U' = 0\) satisfacient: quare dabitur valor aliquis ipsius \(u'\) sub forma \(p+q\surd{-1}\) contentus. Iam co\"efficiens \(M'\) (eodem modo ut ante) rationaliter per \(u'\) et co\"efficientes in \(Z\) determinari potest, adeoque etiam sab forma \(p+q\surd{-1}\) contentus erit; quare aequationis \(uu+u'u+M'\) radices sub eadem forma contentae erunt, simul vero aequationi \(U=0\) satisfacient, i.e. aequatio haec habebit radicem sub forma \(p + \surd{-1}\) contentam. Denique hinc simili ratione sequitur, etiam \(M\) sub eadem forma contineri, nec non radicem aequationis \(zz+uz+M=0\), quae manifesto etiam aequationi propositae \(Z=0\) satisfaciet. Quamobrem quaevis aequatio ad minimum unam radicem formae \(p+q\surd{-1}\) habebit.

\subsection*{11.}

Obiectiones 1, 2, 3, quas contra \textsc{Euleri} demonstrationem primam feci (art. 8), eandem vim contra hanc methodum habent, ea tamen differentia, ut obiectio secunda, cui \textsc{Euleri} demonstratio tantummodo in quibusdam casibus specialibus obnoxia erat, praesentem in omnibus casibus attingere debeat. Scilicet a priori demonstrari potest, etiamsi formula detur, quae co\"efficientem \(M'\) rationaliter per \(u\) et co\"efficientes in \(Z\) exprimat, hanc pro pluribus valoribus ipsius \(u'\) necessario indeterminatam fieri debere; similiterque formulam, quae co\"efficientem \(M''\) per \(u''\) exhibeat, indeterminatam fieri pro quibusdam valoribus ipsius \(u''\) etc. Hoc luculentissime perspicietur, si aequationem quarti gradus pro exemplo assumimus. Ponamus itaque \(m= 4\), sintque radices aequationis \(Z = 0\), hae \(\alpha\), \(\beta\), \(\gamma\), \(\delta\). Tum patet, aequationem \(U = 0\) fore sexti gradus ipsiusque radices \(-(\alpha+\beta)\), \(-(\alpha+\gamma)\), \(-(\alpha+\delta)\), \(-(\beta+\gamma)\), \(-(\beta+\delta)\), \(-(\gamma+\delta)\). Aequatio \(U'=0\) autem erit decimi quinti gradus, et valores ipsius \(u'\) hi \[ \begin{array}{c} \begin{array}{cccccc} 2\alpha+\beta+\gamma, & 2\alpha + \beta + \delta,& 2\alpha + \gamma + \delta, & 2\beta + \alpha + \gamma,& 2\beta + \alpha + \delta, & 2\beta + \gamma + \delta,\\ 2\gamma+ \alpha + \beta, & 2\gamma + \alpha + \delta,& 2 \gamma + \beta + \delta,& 2 \delta + \alpha + \beta, &2\delta + \alpha + \gamma, &2\delta + \beta + \gamma, \end{array} \\
\begin{array}{ccc} \alpha + \beta + \gamma + \delta, & \alpha + \beta + \gamma + \delta, & \alpha + \beta + \gamma + \delta \end{array} \end{array} \] Iam in hac aequatione, quippe cuius gradus est impar, subsistendum erit, habe-
bitque ea revera radicem realem \( \alpha+\beta+\gamma+\delta \) (quae primo co\"efficienti in \(Z\) mutato signo aequalis adeoque non modo realis sed etiam rationalis erit, si co\"efficientes in \(Z\) sunt rationales). Sed nullo negotio perspici potest, si formula detur, quae valorem ipsius \(M'\) per valorem respondentem ipsius \(u'\) rationaliter exhibeat, hanc necessario pro \(u' = \alpha + \beta + \gamma + \delta \) indeterminatam fieri.  Hic enim valor \textit{ter} erit radix aequationis \(U' = 0\), respondebuntque ipsi tres valores ipsius \(M'\), puta \((\alpha+\beta)(\gamma+\delta)\), \((\alpha+\gamma)(\beta+\delta)\) et \((\alpha+\delta)(\beta+\gamma)\), qui omnes irrationales esse possunt. Manifesto autem formula rationalis neque valorem irrationalem ipsius \(M'\) in hoc casu producere posset, neque tres valores diverses. Ex hoc specimine satis colligi potest, methodum clar. \textsc{de Foncenex}ii neutiquam esse satisfacientem, sed si ab omni parte completa reddi debeat, multo profundius in theoriam eliminationis inquiri oportere.

\subsection*{12.}

Denique ill. La Grange de theoremate nostro egit in comm. \textit{Sur la forme des racines imaginaires des \'equations, Nouv. M\'em. de l'Acad. de Berlin 1772, p. 222 sqq}. Magnus hic geometra imprimis operam dedit, defectus in Euleri demonstratione prima supplere et revera praesertim ea, quae supra (art. 8) obiectionem secundam et quartam constituunt, tam profunde perscrutatus est, ut nihil amplius desiderandum restet, nisi forsan in disquisitione anteriori super theoria eliminationis (cui investigatio haec tota innititur) quaedam dubia superesse videantur.—  Attamen obiectionem tertiam omnino non attigit, quin etiam tota disquisitio superstructa est suppositioni, quamvis aequationem \(m^{ti}\) gradus revera \(m\) radices habere.

Probe itaque iis, quae hucusque exposita sunt, perpensis, demonstrationem novam theorematis gravissimi ex principiis omnino diversis petitam peritis haud ingratam fore spero, quam exponere statim aggredior.

\subsection*{13.}

\textsc{Lemma.} \textit{ Denotante m numerum integrum positivum quemcunque, functio \(\sin \varphi . x^m - \sin m \varphi . r^{m-1} x + \sin(m-1)\varphi . r^m\) divisibilis erit per \(xx - 2 \cos \varphi . rx + rr \)}.

\textit{Demonstr.} Pro \(m = 1\) functio illa fit \(= 0\) adeoque per quemcunque factorem divisibilis; pro \(m=2\) quotiens fit \(\sin \varphi\), et pro quovis valore maiori quotiens erit \( \sin \varphi.x^{m-2} + \sin 2\varphi . rx^{m-3} + \sin 3\varphi . rrx^{m-4} + \text{etc.}+\sin(m-1)\varphi . r^{m-2} \). Facile enim confirmatur, multiplicata hac functione per \(xx-2\cos\varphi.rx + rr\), productum functioni propositae aequale fieri.

\subsection*{14.}

\textsc{Lemma.} \textit{Si quantitas \(r\) angulusque \(\varphi\) ita sunt determinati, ut habeantur aequationes}
\begin{align*} r^m \cos m \varphi + A r^{m-1} \cos(m-1) \varphi + B r^{m-2}\cos(m-2) \varphi + \text{etc.} &\\
+ Krr \cos 2\varphi + Lr \cos \varphi + M &= 0  \tag*{[1]} \end{align*}
\begin{align*} r^m \sin m \varphi + A r^{m-1} \sin(m-1) \varphi + B r^{m-2}\sin(m-2) \varphi + \text{etc.}& \\
+ Krr \sin 2\varphi + Lr \sin \varphi & = 0  \tag*{[2]}\end{align*}
\textit{functio \(x^m+Ax^{m-1} + Bx^{m-2}+\text{etc.}+Kxx + Lx + M = X\) divisibilis erit per factorem duplicem \(xx-2\cos\varphi.rx+rr\), si modo \(r\sin\varphi\) non \(=0\); si vero \(r\sin\varphi = 0\), eadem functio divisibilis erit per factorem simplicem \(x - r \cos\varphi\).}

\textit{Demonstr.} I. Ex art. praex. omnes sequentes quantitates divisibiles erunt per \(xx - 2\cos \varphi . rx + rr\):\[\begin{array}{lll} \hspace{1em} \sin \varphi. rx^m &-\hspace{1em} \sin m \varphi . r^m x &+ \sin (m-1) \varphi . r^{m+1} \\ A\sin \varphi. rx^{m-1} &- A\sin (m-1) \varphi . r^{m-1} x &+ \sin (m-2) \varphi . r^{m} \\ B\sin \varphi. rx^{m-2} &- B \sin (m-2) \varphi . r^{m-2} x &+ \sin (m-3) \varphi . r^{m-1} \\ &\text{etc.} & \text{etc.} \\ K\sin \varphi. rx^2 &- K\sin 2 \varphi . rr x &+ \sin (m-1) \varphi . r^{3} \\ L\sin \varphi. rx &- L \sin  \varphi . r x &  \\ M\sin \varphi. r &  &+ M\sin(-\varphi).r \end{array}\]

Quamobrem etiam summa harum quantitatum per \(xx-2\cos\varphi.rx+rr\) divisibilis erit. At singularum partes primae constituunt summam \(\sin\varphi.rX\); secundae additae dant \(0\), propter [2]; tertiarum vero aggregatum quoque evanescere, facile perspicitur, si [1] multiplicatur per \(\sin \varphi\), [2] per \(\cos \varphi\), productumque illud ab hoc subducitur. Unde sequitur, functionem \(\sin \varphi . r X\) divisibilem esse per \(xx - 2 \cos \varphi .rx +rr\), adeoque, nisi fuerit \(r \sin \varphi = 0\), etiam functionem \(X\).  Q.E.P.

II. Si vero \(r \sin \varphi = 0\), erit aut \(r = 0\) aut \(\sin \varphi = 0\). In casu priori erit \(M=0\), propter [1], adeoque \(X\) per \(x\) sive per \(x - r\cos\varphi\) divisibilis; in posteriori erit \(\cos \varphi = \pm 1\), \(\cos 2 \varphi = +1\), \(\cos 3 \varphi = \pm 1\) et generaliter \(\cos n\varphi = \cos \varphi^n\). Quare propter [1] fiet \(X=0\), statuendo \(x=r \cos \varphi\), et proin functio \(X\) per \(x-r\cos\varphi\) erit divisibilis.   Q.E.S.

\subsection*{15.}

Theorema praecedens plerumque adiumento quantitatum imaginariarum demonstratur, vid. \textsc{Euler} \textit{Introd. in Anal. Inf. }T. I. p.110; operae pretium esse duxi, ostendere, quomodo aeque facile absque illarum auxilio erui possit. Manifestum iam est, ad demonstrationem theorematis nostri nihil aliud requiri quam ut ostendatur: \textit{Proposita functione quacunque \(X\) formae \( x^m + Ax^{m-1}+Bx^{m-2}+\text{etc.} +Lx + M\), \(r\)  et \(\varphi\) ita determinari posse, ut aequationes} [1] \textit{et} [2] \textit{locum habeant}. Hinc enim sequetur, \(X\) habere factorem realem primi vel secundi gradus; divisio autem necessario producet quotientem realem inferioris gradus, qui ex eadem ratione quoque factorem primi vel secundi gradus habebit. Per continuationem huius operationis \(X\) tandem in factores reales simplices vel duplices resolvetur. Illud itaque theorema demonstrare, propositum est sequentium disquisitionum.

\subsection*{16.}

Concipiatur planum fixum infinitum (planum tabulae, fig. 1), et in hoc recta fixa infinita \(GC\) per punctum fixum \(C\) transiens. Assumta aliqua longitudine pro unitate ut omnes rectae per numeros exprimi possint, erigatur in quovis puncto plani \(P\), cuius distantia a centro \(C\) est \(r\) angulusque \(GCP = \varphi\), perpendiculum aequale valori expressionis \[ r^m \sin m \varphi + A r^{m-1} \sin(m-1)\varphi + \text{etc.} + L r\sin \phi \] quem brevitatis gratia in sequentibus semper per \(T\) designabo. Distantiam \(r\) semper tamquam positivam considero, et pro punctis, quae axi ab altera parte iacent, angulus \(\varphi\) aut tamquam duobus rectis maior, aut tamquam negativus (quod hic eodem redit) spectari debet. Extremitates horum perpendiculorum (quae pro valore positive ipsius \(T\) supra planum accipiendae sunt, pro negativo infra, pro evanescente in plano ipso) erunt ad superficiem curvam continuam quaquaversum infinitam, quam brevitatis gratia in sequentibus \textit{superficiem primam} vocabo. Prorsus simili modo ad idem planum et centrum eundemque axem referatur alia superficies, cuius altitudo supra quodvis plani punctum sit \[r^m\cos m\varphi + Ar^{m-1}\cos(m-1)\varphi + \text{etc.}+Lr\cos\varphi+M\] quam expressionem brevitatis gratia semper per \(U\) denotabo. Superficiem vero hanc, quae etiam continua et quaquaversum infinita erit, per denominationem \textit{superficiei secundae} a priori distinguam. Tunc manifestum est, totum negotium in eo versari, ut demonstretur, ad minimum unum punctum dari, quod simul in plano, in superficie prima et in superficie secunda iaceat.

\subsection*{17.} 

Facile perspici potest, superficiem primam partim supra planum partim infra planum iacere; patet enim distantiam a centro \(r\) tam magnam accipi posse, ut reliqui termini in \(T\) prae primo \(r^m \sin m \varphi\) evanescant; hic vero, angulo \(\varphi\) rite determinato, tam positivus quam negativus fieri potest. Quare planum fixum necessario a superficie prima secabitur; hanc plani cum superficie prima intersectionem vocabo \textit{lineam primam}; quae itaque determinabitur per aequationem \(T = 0\). Ex eadem ratione planum a superficie secunda secabitur; intersectio constituet curvam per aequationem \(U = 0\) determinatam, quam \textit{lineam secundam} appellabo. Proprie utraque curva ex pluribus ramis constabit, qui omnino seiuncti esse possunt, singuli vero erunt lineae continuae. Quin adeo linea prima semper erit talis, quam complexam vocant, axisque \(GC\) tamquam pars huius curvae spectanda; quicunque enim valor ipsi \(r\) tribuatur, \(T\) semper fiet \(= 0\), quando \(\varphi\) aut \(= 0\) aut \(= 180^o\). Sed praestat complexum cunctorum ramorum per omnia puncta, ubi \(T = 0\), transeuntium tamquam unam curvam considerare (secundum usum in geometria sublimiori generaliter receptum), similiterque cunctos ramos per omnia puncta transeuntes, ubi \(U=0\). Patet iam, rem eo reductam esse, ut demonstretur, ad minimum unum punctum in plano dari, ubi ramus aliquis lineae primae a ramo lineae secundae secetur. Ad hunc finem indolem harum linearum propius contemplari oportebit.

\subsection*{18.}

Ante omnia observo, utramque curvam esse algebraicam, et quidem, si ad coordinatas orthogonales revocetur, ordinis \(m^{ti}\).   Sumto enim initio abscissarum in \(C\), abscissisque \(x\) versus \(G\), applicatis \(y\) versus \(P\), erit \(x = r \cos \varphi\), \(y = r \sin \varphi \), adeoque generaliter, quidquid sit \(n\), \begin{align*} r^n \sin n \varphi = nx^{n-1}y - \tfrac{n . n-1 . n-2}{1 . 2 . 3} x^{n-3} y^3 + \tfrac{n \dots n-4}{1 \dots . 5} x^{n-5}y^5 - \text{etc.}, \\ r^n \cos n \varphi = x^n - \tfrac{n . n-1}{1 . 2} x^{n-2} yy + \tfrac{n . n-1 . n-2 . n-3 }{ 1 . 2 . 3 . 4} x^{n-4}y^4 - \text{etc.} \end{align*} Quamobrem tum \(T\) tum \(U\) constabunt ex pluribus huiusmodi terminis \(a x^{\alpha} y^{\beta} \), denotantibus \(\alpha\), \(\beta\) numeros integros positivos, quorum summa, ubi maxima est, fit \(= m\). Ceterum facile praevideri potest, cunctos terminos ipsius \(T\) factorem \(y\) involvere, adeoque lineam primam proprie ex recta (cuius aequatio \(y = 0\)) et curva ordinis \({m-1}^{ti}\) compositam esse; sed necesse non est ad hanc distinctionem hic respicere.

Maioris momenti erit investigatio, an linea prima et secunda crura infinita habeant, et quot qualiaque. In distantia infinita a puncto \(C\) linea prima, cuius aequatio \(\sin m \varphi + \frac{A}{r} \sin (m-1)\varphi + \frac{B}{rr} \sin (m-2) \varphi \text{ etc.} = 0\), confundetur cum linea, cuius aequatio \(\sin m \varphi = 0\). Haec vero exhibet \(m\) lineas rectas in puncto \(C\) se secantes, quarum prima est axis \(GCG'\), reliquae contra hanc sub angulis \(\frac{1}{m} 180\), \(\frac{2}{m} 180\), \(\frac{3}{m} 180 \) etc. graduum inclinatae.  Quare linea prima \(2m\) ramos infinitos habet, qui peripheriam circuli radio infinite descripti in \(2m\) partes aequales dispertiuntur, ita ut peripheria a ramo primo secetur in concursu circuli et axis, a secundo in distantia \(\frac{1}{m} 180^o\), a tertio in distantia \(\frac{2}{m} 180^o\) etc.  Eodem modo linea secunda in distantia infinita a centro habebit asymptotam per aequationem \(\cos m \phi = 0\) expressam, quae est complexus m rectarum in puncto \(C\) sub aequalibus angulis itidem se secantium, ita tamen, ut prima cum axe \(CG\) constituat angulum \(\frac{1}{m}90^o\) secunda angulum \(\frac{3}{m}90^o\), tertia angulum \(\frac{5}{m}90^o\) etc. Quare linea secunda etiam \(2m\) ramos infinites habebit, quorum singuli medium locum inter binos ramos proximos lineae primae occupabunt, ita ut peripheriam circuli radio infinite magno descripti in punctis, quae \(\frac{1}{m}90^o\), \(\frac{3}{m}90^o\), \(\frac{5}{m}90^o\) etc. ab axe distant, secent. Ceterum palam est, axem ipsum semper duos ramos infinitos lineae primae constituere, puta primum et \({m+1}^{tum}\). Luculentissime hic ramorum situs exhibetur in fig. 2, pro casu \(m = 4\) constructa, ubi rami lineae secundae, ut a ramis lineae primae distinguantur, punctati exprimuntur, quod etiam de figura quarta est tenendum\footnote{Figura quarta constructa est supponendo \(X = x^4 - 2xx + 3x + 10\), in qua itaque lectores disquisitionibus generalibus et abstractis minus assueti situm respectivum utriusque curvae in concreto intueri poterunt.   Longitudo lineae \(CG\) assumta est  = 10  (CN= 1,26255.)}.—  Quum vero hae conclusiones maximi momenti sint, quantitatesque infinite magnae quosdam lectores offendere possint: illas etiam absque infinitorum subsidio in art. sequ. eruere docebo.

\subsection*{19.}

\textsc{Theorema}.  \textit{Manentibus cunctis ut supra, ex centro \(C\) describi poterit circulus, in cuius peripheria sint \(2m\) puncta, in quibus \(T= 0\), totidemque, in quibus \(U=0\), et quidem ita, ut singula posteriora inter bina priorum iaceant.}

Sit summa omnium coefficientium \(A\), \(B\) etc. \(K\), \(L\), \(M\) positive acceptorum \(= S\), accipiaturque \(R\) simul \(>S\surd{2}\) et \(>1\) \footnote{Quando \(S>\surd{\tfrac{1}{2}}\), conditio prima secundam; quando vero \(S<\surd{\tfrac{1}{2}}\), secunda primam implicabit.}: tum dico in circulo radio \(R\) descripto ea, quae in theoremate enunciata sunt, necessario locum habere. Scilicet designato brevitatis gratia eo puncto huius circumferentiae, quod \(\frac{1}{m}45\) gradibus ab ipsius concursu cum laeva parte axis distat, sive pro quo \(\varphi = \frac{1}{m} 45^o\), per (1); similiter eo puncto, quod \(\frac{3}{m} 45^o\) ab hoc concursu distat, sive pro quo \(\varphi = \frac{3}{m} 45^o\), per (3); porro eo, ubi \(\varphi = \frac{5}{m} 45^o\), per (5) etc. usque ad \((8m-1)\), quod \(\frac{8m-1}{m} 45\) gradibus ab illo concursu distat, si semper versus eandem partem progrederis, (aut \(\frac{1}{m}45^o\) a parte opposita), ita ut omnino \(4m\) puncta in peripheria habeantur, aequalibus intervallis dissita: iacebit inter \((8m-1)\) et (1) unum punctum, pro quo \(T= 0\); nee non sita erunt similia puncta singula inter (3) et (5); inter (7) et (9); inter (11) et (13) etc., quorum itaque multitudo \(2m\); eodemque modo singula puncta, pro quibus \(U=0\), iacebunt inter (1) et (3); inter (5) et (7); inter (9) et (11), quorum multitudo igitur etiam \(= 2m\); denique praeter haec \(4m\) puncta alia in tota peripheria non dabuntur, pro quibus vel \(T\) vel \(U\) sit \(= 0\).

\textit{Demonstr.} I. In puncto (1) erit  \(m\varphi = 45^o\) adeoque \[ T = R^{m-1}(R\surd \tfrac{1}{2} + A \sin (m-1)\varphi + \frac{B}{R} \sin (m-2) \varphi + \text{etc.}+\frac{L}{R^{m-2}}\sin \varphi )\] summa vero \(A \sin (m-1) \varphi + \frac{B}{R} \sin (m-2) \phi \) etc. certo non poterit esse maior quam \(S\), adeoque necessario erit minor quam \(R\surd{\tfrac{1}{2}}\): unde sequitur, in hoc puncto valorem ipsius \(T\) certo esse positivum. A potiori itaque \(T\) valorem positivum habebit, quando \(m\varphi\) inter \(45^o\) et \(135^o\) iacet, i.e. a puncto (1) usque ad (3) valor ipsius \(T\) semper positivus erit. Ex eadem ratione \(T\) a puncto (9) usque ad (11) positivum valorem ubique habebit, et generaliter a quovis puncto \((8k+1)\) usque ad \((8k+3)\), denotante \(k\) integrum quemcunque. Simili modo \(T\) ubique inter (5) et (7), inter (13) et (15) etc. et generaliter inter \((8k + 5)\) et \((8k+7)\) valorem negativum habebit, adeoque in omnibus his intervallis nullibi poterit esse \(= 0\). Sed quoniam in (3) hic valor est positivus, in (5) negativus: necessario alicubi inter (3) et (5) erit \(= 0\);  nee non alicubi inter (7) et (9);  inter (11) et (13) etc. usque ad intervallum inter \((8m-1)\) et (1) incl., ita ut omnino in \(2m\) punctis habeatur  \(T = 0\).  Q.E.P.

II. Quod vero praeter haec \(2m\) puncta, alia, hac proprietate praedita, non dantur, ita cognoscitur. Quum inter (1) et (3); inter (5) et (7) etc. nulla sint, aliter fieri non posset, ut plura talia puncta exstent, quam si in aliquo intervallo inter (3) et (5), vel inter (7) et (9) etc. ad minimum duo iacerent. Tum vero necessario in eodem intervallo \(T\) alicubi esset \textit{maximum} vel \textit{minimum}, adeoque \(\frac{dT}{d\varphi} = 0\). Sed \(\frac{dT}{d\varphi} = mR^{m-2}(R\cos m \varphi + \frac{m-1}{m} A \cos(m-1)\varphi + \text{etc.}\)) et \(\cos m\varphi \) inter (3) et (5) semper est negativus et \(>\surd{\tfrac{1}{2}}\). Unde facile perspicitur, in toto hoc intervallo \(\frac{dT}{d\varphi}\) esse quantitatem negativam; eodemque modo inter (7) et (9) ubique positivam; inter (11) et (13) negativam etc., ita ut in nullo herum intervallorum esse possit \(0\), adeoque suppositio consistere nequeat.  Quare etc. Q.E.S.

III. Prorsus simili modo demonstratur, \(U\) habere valorem negativum ubique inter (3) et (5), inter (11) et (13) etc. et generaliter inter \((8k+3)\) et \(8k+5\); positivum vero inter (7) et (9), inter (15) et (17) etc. et generaliter inter \( (8k+7)\) et \((8k+9)\). Hinc statim sequitur, \(U=0\) fieri debere alicubi inter (1) et (3), inter (5) et (7) etc., i.e. in \(2m\) punctis. In nullo vero herum intervallorum fieri poterit \(\frac{dU}{d\varphi}=0\) (quod facile simili modo ut supra probatur): quamobrem plura quam illa \(2m\) puncta in circuli peripheria non dabuntur, in quibus fiat \(U=0\) Q. E. T. et Q.

Ceterum ea theorematis pars, secundum quam plura quam \(2m\) puncta non dantur, in quibus \(T=0\), neque plura quam \(2m\), in quibus \(U = 0\), etiam inde demonstrari potest, quod per aequationes \(T = 0\), \(U=0\) exhibentur curvae \(m^{ti}\) ordinis, quales a circulo tamquam curva secundi ordinis in pluribus quam \(2m\) punctis secari non posse, ex geometria sublimiori constat.

\subsection*{20.}

 Si circulus alius radio maiori quam \(R\) ex eodem centro describitur, eodemque modo dividitur: etiam in hoc inter puncta (3) et (5) iacebit punctum unum, in quo \(T= 0\), itemque inter (7) et (9) etc., perspicieturque facile, quo minus radius huius circuli a radio \(R\) differat, eo propius huiusmodi puncta inter (3) et (5) in utriusque circumferentia sita esse debere. Idem etiam locum habebit, si circulus radio aliquantum minori quam \(R\), attamen maiori quam \(S\surd{2}\) et \(1\), describitur.  Ex bis nullo negotio intelligitur, circuli radio \(R\) descripti circumferentiam in eo puncto inter (3) et (5), ubi \(T=0\), revera \textit{secari} ab aliquo ramo lineae primae; idemque valet de reliquis punctis, ubi \(T= 0\). Eodem modo patet, circumferentiam circuli huius in omnibus \(2m\) punctis, ubi \(U=0\), ab aliquo ramo lineae secundae secari. Hae conclusiones etiam sequenti modo exprimi possunt: Descripto circulo debitae magnitudinis e centro \(C\), in hunc intrabunt \(2m\) rami lineae primae totidemque rami lineae secundae, et quidem ita, ut bini rami proximi lineae primae per aliquem ramum lineae secundae ab invicem separentur.  Vid. fig. 2, ubi circulus iam non infinitae sed finitae magnitudinis erit, numerique singulis ramis adscripti cum numeris, per quos in art. praec. et hoc limites certos in peripheria brevitatis caussa designavi, non sunt confundendi.

\subsection*{21.}

Iam ex hoc situ relative ramorum in circulum intrantium tot modis diversis deduci potest, intersectionem alicuius rami lineae primae cum ramo lineae secundae intra circulum necessario dari, ut, quaenam potissimum methodus prae reliquis eligenda sit, propemodum nesciam. Luculentissima videtur esse haec: Designemus (fig. 2) punctum peripheriae circuli, ubi a laeva axis parte (quae ipsa est unus ex \(2m\) ramis lineae primae) secatur, per \(0\); punctum proximum, ubi ramus lineae secundae intrat, per \(1\); punctum huic proximum, ubi secundus lineae primae ramus intrat, per \(2\), et sie porro usque ad \(4m-1\), ita ut in quovis puncto numero pari signato ramus lineae secundae in circulum intret, contra ramus lineae secundae in omnibus punctis per numerum imparem expressis. Iam ex geometria sublimiori constat, quamvis curvam algebraicam, (sive singulas cuiusvis curvae algebraicae partes, si forte e pluribus composita sit) aut in se redeuntem aut utrimque in infinitum excurrentem esse, adeoque si ramus aliquis curvae algebraicae in spatium definitum intret, eundem necessario ex hoc spatio rursus alicubi exire debere \footnote{ Satis bene certe demonstratum esse videtur, curvam algebraicam neque alicubi subito abrumpi posse (uti e.g. evenit in curva transscendente, cuius aequatio \( y = \frac{1}{\log x} \)), neque post spiras infinitas in aliquo puncto se quasi perdere (ut spiralis logarithmica), quantumque scio nemo dubium contra hanc rem movit. Attamen si quis postulat, demonstrationem nullis dubiis obnoxiam alia occasione tradere suscipiam. In casu praesenti vero manifestum est, si aliquis ramus e.g. 2, ex circulo nullibi exiret (fig. 3), te in circulum inter \(0\) et \(2\) intrare, postea circa totum hunc ramum (qui in circuli spatio se perdere deberet) circummeare, et tandem inter \(2\) et \(4\) rursus ex circulo egredi posse, ita ut nullibi in tota via in lineam primam incideris.   Hoc vero absurdum esse inde patet,  quod in puncto, ubi in circulum ingressus es, superficiem primam supra te habuisti, in egressu, infra; quare necessario alicubi in superficiem primam ipsam incidere debuisti, sive in punctum lineae primae. — Ceterum ex hoc ratiocinio principiis geometriae situs innixo, quae haud minus valida sunt, quam principia geometriae magnitudinis, sequitur tantummodo, si in aliquo ramo lineae primae in circulum intres, te alio loco ex circulo rursus egredi posse, semper in linea prima manendo, neque vero, viam tuam esse lineam continuam in eo sensu, quo in geometria sublimiori accipitur. Sed hic sufficit, viam esse lineam continuam in sensu communi, i.e. nullibi interruptam sed ubique cohaerentem.}. Hinc concluditur facile, quodvis punctum numero pari signatum (seu, brevitatis caussa, quodvis punctum par) per ramum lineae primae cum alio puncto pari intra circulum iunctum esse debere, similiterque quodvis punctum numero impari notatum cum alio simili puncto per ramum lineae secundae.  Quamquam vero haec binorum punctorum connexio secundum indolem functionis \(X\) perquam diversa esse potest, ita ut in genere determinari nequeat, tamen facile demonstrari potest, \textit{quaecunque demum illa sit, semper intersectionem lineae primae cum linea secunda oriri.}

\subsection*{22.}
Demonstratio huius necessitatis commodissime apagogice repraesentari posse videtur. Scilicet supponamus, iunctionem binorum quorumque punctorum parium, et binorum quorumque punctorum imparium ita adornari posse, ut nulla intersectio rami lineae primae cum ramo lineae secundae inde oriatur. Quoniam axis est pars lineae primae, manifesto punctum \(0\) cum puncto \(2m\) iunctum erit. Punctum \(1\) itaque cum nullo puncto ultra axem sito, i.e. cum nullo puncto per numerum maiorem quam \(2m\) expresso iunctum esse potest, alioquin enim linea iungens necessario axem secaret. Si itaque \(1\) cum puncto \(n\) iunctum esse supponitur, erit \(n<2m\). Ex simili ratione, si \(2\) cum \(n'\) iunctum esse statuitur, erit \(n'<n\), quia alioquin ramus \(2...n'\) ramum \(1...n\) necessario secaret. Ex eadem caussa punctum \(3\) cum aliquo punctorum inter \(4\) et \(n'\) iacentium iunctum erit, patetque si \(3\), \(4\), \(5\) etc. iuncta esse supponantur cum \(n''\), \(n'''\), \(n''''\)etc., \(n'''\) iacere inter \(5\) et \(n'', n''''\) inter \(6\) et \(n'''\) etc. Unde perspicuum est, tandem ad aliquod punctum \(h\) perventum iri, quod cum puncto \(h+2\) iunctum sit, et tum ramus, qui in puncto \(h+1\) in circulum intrat, necessario ramum puncta \(h\) et \(h+2\) iungentem secabit. Quia autem alter horum duorum ramorum ad lineam primam, alter ad secundam pertinebit, manifestum iam est, suppositionem esse contradictoriam, adeoque necessario alicubi intersectionem lineae primae cum linea secunda fieri.

Si haec cum praecedentibus iunguntur, ex omnibus disquisitionibus explicatis colligetur, theorema, \textit{quamvis functionem algebraicam rationalem integram unius indeterminatae in factores reales primi vel secundi gradus resolvi posse}, omni rigore demonstratum.


\subsection*{23.} 

Ceterum haud difficile ex iisdem principiis deduci potest, non solum unam sed ad minimum \(m\) intersectiones lineae primae cum secunda dari, quamquam etiam fieri potest, ut linea prima a pluribus ramis lineae secundae in eodem puncto secetur, in quo casu functio \(X\) plures factores aequales habebit. Attamen quum hic sufficiat, unius intersectionis necessitatem demonstravisse, fusius huic rei brevitatis caussa non immoror. Ex eadem ratione etiam alias harum linearum proprietates hic uberius non persequor, e.g. intersectionem semper fieri sub angulis rectis; aut si plura crura utriusque curvae in eodem puncto conveniant, totidem crura lineae primae affore, quot crura lineae secundae, haecque alternatim posita esse, et sub aequalibus angulis se secare etc.

Denique observo, minime impossibile esse, ut demonstratio praecedens, quam hic principiis geometricis superstruxi, etiam in forma mere analytica exhibeatur: sed eam repraesentationem, quam hic explicavi, minus abstractam evadere credidi, verumque nervum probandi hic multo clarius ob oculos poni, quam a demonstratione analytica exspectari possit. 

Coronidis loco adhuc aliam methodum theorema nostrum demonstrandi addigitabo, quae primo aspectu non modo a demonstratione praecedente, sed etiam ab omnibus demonstrationibus reliquis supra enarratis maxime diversa esse videbitur, et quae nihilominus cum \textsc{d'Alembert}iana, si ad essentiam spectas, proprie eadem est. Cum qua illam comparare, parallelismumque inter utramque explorare peritis committo, in quorum gratiam unice subiuncta est.

\subsection*{24.}
 
Supra planum figurae \(4\) relative ad axem \(CG\) punctumque fixum \(C\) descriptas suppono superficiem primam et secundam eodem modo ut supra.  Accipe punctum quodcunque in aliquo ramo lineae primae situm sive ubi \(T= 0\), (e. g. quodlibet punctum \(M\) in axe iacens), et nisi in hoc etiam \(U= 0\), progredere ex hoc puncto in linea prima versus eam partem, versus quam magnitudo absoluta ipsius \(U\) decrescit. Si forte in puncto \(M\) valor absolutus ipsius \(U\) versus utramque partem decrescit, arbitrarium est, quorsum progrediaris; quid vero faciendum sit, si \(U\) versus utramque partem crescat, statim docebo. Manifestum est itaque, dum semper in linea prima progrediaris, necessario tandem te ad punctum perventurum, ubi \(U = 0\), aut ad tale, ubi valor ipsius \(U\) fiat minimum, e.g. punctum \(N\). In priori casu quod quaerebatur, inventum est; in posteriori vero demonstrari potest, in hoc puncto plures ramos lineae primae sese intersecare (et quidem multitudinem parem ramorum), quorum semissis ita comparati sint, ut si in aliquem eorum defiectas (sive huc sive illuc), valor ipsius \(U\) adhucdum decrescere pergat. (Demonstrationem huius theorematis, prolixiorem quam difficiliorem brevitatis gratia supprimere debeo.) In hoc itaque ramo iterum progredi poteris, donec \(U\) aut fiat \(= 0\) (uti in fig. 4 evenit in \(P\)), aut denuo minimum. Tum rursus defiectes, necessarioque tandem ad punctum pervenies, ubi sit  \(U = 0\).

Contra hanc demonstrationem obiici posset dubium, annon possibile sit, ut quantumvis longe progrediaris, et quamvis valor ipsius \(U\) semper decrescat, tamen haec decrementa continuo tardiora fiant, et nihilominus ille valor limitem aliquem nusquam attingat; quae obiectio responderet quartae in art. 6. Sed haud difficile foret, terminum aliquem assignare, quem simulac transieris, valor ipsius \(U\) necessario non modo semper rapidius mutari debeat, sed etiam \textit{decrescere} non amplius possit, ita ut antequam ad hunc terminum perveneris, necessario valor \(0\) iam affuisse debeat. Hoc vero et reliqua, quae in hac demonstratione addigitare tantummodo potui, alia occasione fusius exsequi mihi reservo.

\begin{center}

 \rule{3in}{0.5pt}

\begin{scriptsize}\textit{Principia quibus haecce demonstratio innititur deteximus Initio Octob. 1797.}\end{scriptsize}

 \rule{2in}{0.5pt}

\end{center}
\end{document}