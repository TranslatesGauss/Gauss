\documentclass[twoside,12pt, showframe]{memoir}
\usepackage{standalone}
\usepackage{array}
\usepackage{mlmodern}
\usepackage[utf8]{inputenc}
\usepackage[T1]{fontenc}
\usepackage[dvips,text={6.25in,8.5in},left=1.125truein,top=1.5truein]{geometry}
\renewcommand{\baselinestretch}{1.25}
\usepackage{amsmath}
\usepackage{amsfonts}
\usepackage{amssymb}
\usepackage{graphicx}
\usepackage{CJKutf8}
\usepackage{multirow}
\usepackage{indentfirst}
\usepackage{mathtools}
\graphicspath{ {./2024_01_11_75975a03bcf8b0416cd0g/images/} }
\newtagform{brackets}{[}{]}
\usetagform{brackets}
\geometry{footnotesep=0.5\baselineskip}
\parindent=3em
\parskip=0pt
\usepackage{titlesec}
\titleformat{\section}
  {\normalfont\centering}{\thesection.}{1em}{}
\titleformat{\subsection}
  {\normalfont\normalsize\centering}{\thesection.}{1em}{}
\titleformat{\subsubsection}
  {\normalfont\normalsize\centering}{\thesection.}{1em}{}
  \titlespacing\subsection{0pt}{12pt plus 6pt minus 6pt}{0pt plus 2pt minus 2pt}
\def\equation{\old@equation\small\hskip\textwidth minus \textwidth}
\def\endequation{\leaders\hbox{ . }\hskip \textwidth minus \textwidth\old@endequation}
\spaceskip=0.67em plus 0.33em minus 0.33em
\renewenvironment{quote}%
  {\list{}{\leftmargin=5em\rightmargin=0em}\item[]}%
  {\endlist}
\renewcommand{\pmod}[1]{\;(\textrm{mod.}\;#1)}
\let\oldfrac\frac
\def\frac#1#2{\mathchoice{\tfrac{#1}{#2}}{\oldfrac{#1}{#2}}{\genfrac{}{}{}{2}{#1}{#2\mathstrut}}{\genfrac{}{}{}{3}{#1}{#2\mathstrut}}}
\thickmuskip=4mu plus 4mu
\medmuskip=3mu plus 1.5mu minus 3mu
\arraycolsep=3pt%\def\arraystretch{1.03}
%\setlength{\tabcolsep}{3pt}
\DeclareMathSizes{12}{12}{9}{6}
\DeclareUnicodeCharacter{0131}{$\imath$}
\begin{document}
\setlength{\abovedisplayskip}{0.33\baselineskip plus .25\baselineskip minus .25\baselineskip}
\setlength{\belowdisplayskip}{0.33\baselineskip plus .25\baselineskip minus .25\baselineskip}

\title{DEMONSTRATIO NOVA }


\author{15.}
\date{}


\maketitle
THEOREMATIS

\section*{OMNEM FUNCTIONEM ALGEBRAICAM RATIONALEM INTEGRAM}
EXHIBUIT

C.AROLUS FRIDERICUS GAUSS

\section*{OMNEM FUNCTIONEM ALGEBRAICAM RATIONALEM INTEGRAM }
IN FACTORES REALES PRIMI VEL SECUNDI GRADUS RESOIVI POSSE.

\section*{1.}
Quaelibet aequatio algebraica determinata reduci potest ad formam

\[
x^{m}+A x^{m-1}+B x^{m-2}+\text { etc. }+M=0
\]

ita ut \(m\) sit numerus integer positivus. Si partem primam huius aequationis per \(\boldsymbol{X}\) denotamus, aequationique \(\mathbf{X}=0\) per plures valores inaequales ipsius \(x\) satisfieri supponimus, puta ponendo \(x=\alpha, x=b, x=\gamma\) etc. functio \(\boldsymbol{X}\) per productum e factoribus \(x-\alpha, x-b, x-\gamma\) etc. divisibilis erit. Vice versa, si productum e pluribus factoribus simplicibus \(x-\alpha, x-b, x-\gamma\) etc. functionem \(\boldsymbol{X}\) metitur: aequationi \(\boldsymbol{X}=\mathbf{0}\) satisfiet, aequando ipsam \(x\) cuicunque quantitatum \(\alpha, b, \gamma\) etc. Denique si \(X\) producto ex \(m\) factoribus talibus simplicibus aequalis est (sive omnes diversi sint, sive quidam ex ipsis identici): alii factores simplices praeter hos functionem \(\boldsymbol{X}\) metiri non poterunt. Quamobrem aequatio \(m^{\text {ti }}\) gradus plures quam \(m\) radices habere nequit; simul vero patet, aequationem \(m^{\text {ti }}\) gradus pauciores radices habere posse, etsi \(\mathbf{X}\) in \(m\) factores simplices resolubilis sit: si enim inter hos factores aliqui sunt identici, multitudo modorum diversorum aequationi satisfaciendi necessario minor erit quam \(m\). Attamen concinnitatis caussa geometrae dicere maluerunt, aequationem in hoc quoque casu \(m\) radices habere, et tantummodo quasdam ex ipsis aequales inter se evadere: quod ' utique sibi permittere potuerunt.

2.

Quae hucusque sunt enarrata, in libris algebraicis sufficienter demonstrantur neque rigorem geometricum uspiam offendunt. Sed nimis praepropere et sine praevia demonstratione solida adoptavisse videntur analystae theorema cui tota fere doctrina aequationum superstructa est: Quamvis functionem talem ut \(\boldsymbol{X}\) semper in \(m\) factores simplices resolvi posse, sive hoc quod cum illo prorsus conspirat, quamvis aequationem \(m^{t i}\) gradus revera habere \(m\) radices. Quum iam in aequationibus secundi gradus saepissime ad tales casus perveniatur, qui theoremati huic repugnant: algebraistae, ut hos illi subiicerent, coacti fuerunt, fingere quantitatem quandam imaginariam cuius quadratum sit -1 , et tum agnoverunt, si quantitates formae \(a+b \sqrt{ }-1\) perinde concedantur ut reales, theorema non modo pro aequationibus secundi gradus verum esse, sed etiam pro cubicis et biquadraticis. Hinc vero neutiquam inferre licuit, admissis quantitatibus formae \(a+b \sqrt{ }-1\) cuivis aequationi quinti superiorisve gradus satisfieri posse, aut uti plerumque exprimitur (quamquam phrasin lubricam minus probarem) radices cuiusvis aequationis ad formam \(a+b \sqrt{ }-1\) reduci posse. Hoc theorema ab eo, quod in titulo huius scripti enunciatum est, nihil differt. si ad rem ipsam spectas, huiusque demonstrationem novam rigorosam tradere. constituit propositum praesentis dissertationis.

Ceterum ex eo tempore, quo analystae comperti sunt, infinite multas aequationes esse, quae nullam omnino radicem haberent, nisi quantitates formae \(a+b \sqrt{ }-1\) admittantur, tales quantitates fictitiae tamquam peculiare quantitatum genus, quas imaginarias dixerunt, ut a realibus distinguerentur, consideratae et in totam analysin introductae sunt; quonam iure? hoc loco non disputo. - Demonstrationem meam absque omni quantitatum imaginariarum subsidio absolvam, etsi eadem libertate, qua omnes recentiores analystae usi sunt, etiam mihi uti liceret.

3.

Quamvis ea, quae in plerisque libris elementaribus tamquam demonstratio theorematis nostri afferuntur, tam levia sint, tantumque a rigore geometrico abhorreant, ut vix mentione sint digna: tamen, ne quid deesse videatur, paucis illa attingam. 'Ut demonstrent, quamvis aequationem

\[
x^{m}+A x^{m-1}+B x^{m-2}+\text { etc. }+M=0
\]

sive \(\boldsymbol{X}=0\), revera habere \(m\) radices, suscipiunt probare, \(\boldsymbol{X}\) in \(m\) factores simplices resolvi posse. Ad hunc finem assumunt \(m\) factores simplices \(x-\alpha\), \(x-b, x-\gamma\) etc. ubi \(\alpha, b, \gamma\) etc. adhuc sunt incognitae, productumque ex illis aequale ponunt functioni \(X\). Tum ex comparatione coëfficientium deducunt \(m\) aequationes, ex quibus incognitas \(\alpha, \vec{b}, \gamma\) etc. determinari posse aiunt, quippe quarum multitudo etiam sit \(m\). Scilicet \(m^{\prime \prime}-1\) incognitas eliminari posse, unde emergere aequationem, quae, quam placuerit, incognitam solam contineat.' Ut de reliquis, quae in tali argumentatione reprehendi possent, taceam, quaeram tantummodo, unde certi esse possimus, ultimam aequationem revera ullam radicem habere? Quidni fieri posset, ut neque huic ultimae aequationi neque propositae, ulla magnitudo in toto quantitatum realium atque imaginariarum ambitu satisfaciat? \_ Ceterum periti facile perspicient, hanc ultimam aequationem necessario cum proposita omnino identicam fore, siquidem calculus rite fuerit institutus; scilicet eliminatis incognitis \(b, \gamma\) etc. aequationem

\[
\alpha^{m}+A \alpha^{m-1}+B \alpha^{m-2}+\text { etc. }+M=0
\]

prodire debere. Plura de isto ratiocinio exponere necesse non est.

Quidam auctores, qui debilitatem huius methodi percepisse videntur, tamquam axioma assumunt, quamvis aequationem revera habere radices, si non possibiles, impossibiles. Quid sub quantitatibus possibilibus et impossibilibus intelligi velint, haud satis distincte exposuisse videntur. Si quantitates possibiles idem denotare debent ut reales, impossibiles idem ut imaginariae: axioma illud neutiquam admitti potest, sed necessario demonstratione opus habet. Attamen in illo sensu expressiones accipiendae non videntur, sed axiomatis mens haec potius videtur esse: 'Quamquam nondum sumus certi, necessario dari \(m\) quantitates reales vel imaginarias, quae alicui aequationi datae \(m^{\text {ti }}\) gradus satisfaciant, tamen aliquantisper hoc supponemus; nam si forte contingeret, ut tot quantitates reales et imaginariae inveniri nequeant, certe effugium patebit, ut dicamus reliquas esse impossibiles.' Si quis hac phrasi uti mavult quam simpliciter dicere, aequationem in hoc casu tot radices non habituram, a me nihil obstàt: at si tum his radicibus impossibilibus ita utitur tamquam aliquid veri sint, et e. g. dicit, summam omnium radicum aequationis \(x^{m}+A x^{m-1}+\) etc. \(=0\). esse \(=-A\), etiamsi impossibiles inter illas sint (quae expressio proprie significat, etiamsi aliquae deficiant): hoc neutiquam probare possum. Nam radices impossi-
biles, in tali sensu acceptae, tamen sunt radices, et tum axioma illud nullo modo sine demonstratione admitti potest, neque inepte dubitares, annon aequationes exstare possint, quae ne impossibiles quidem radices habeant?*)

4.

Antequam aliorum geometrarum demonstrationes theorematis nostri recenseam, et quae in singulis reprehendenda mihi videantur, exponam: observo sufficere si tantummodo ostendatur, omni aequationi quantivis gradus

\[
x^{m}+A x^{m-1}+B x^{m-2}+\text { etc. }+M=0
\]

sive \(\boldsymbol{X}=0\) (ubi coëfficientes \(A, B\) etc. reales esse supponuntur) ad minimum uno modo satisfieri posse per valorem ipsius \(x\) sub forma \(a+b \sqrt{ }-1\) contentum. Constat enim, \(\boldsymbol{X}\) tunc divisibilem fore per factorem realem secundi gradus \(x x-2 a x+a a+b b\), si \(b\) non fuerit \(=0\), et per factorem realem simplicem \(x-a\), si \(b=0\). In utroque casu quotiens erit realis, et inferioris gradus quam \(\mathbf{X}\); et quum hic eadem ratione factorem realem primi secundive gradus habere de-

*) Sub quantitate imaginaria hic semper intelligo quantitatem in forma \(a+b \sqrt{ }-1\) contentam, quamdiu \(b\) non est \(=0\). In hoc sensu expressio illa semper ab omnibus geometris primae notae accepta est, neque audiendos censeo, qui quantitatem \(a+b \sqrt{ }-1\) in eo solo casu imaginariam vocare voluerunt ubi \(a=0\), impossibilem vero quando non sit \(a=0\), quum haec distinctio neque necessaria sit neque ullius utilitatis. Si quantitates imaginariae omnino in analysi retineri debent (quod pluribus rationibus consultius videtur, quam ipsas abolere, modo satis solide stabiliantur) : necessario tamquam aeque possibiles ac reales spectandae sunt; quamobrem reales et imaginarias sub denominatione communi quantitatum possibilium complecti mallem: contra, impossibilem dicerem quantitatem, quae conditionibus satisfacere debeat, quibus ne imaginariis quidem concessis satisfieri potest, attamen ita, ut phrasis haec idem significet ac si dicas, talem quantitatem in toto magnitudinum ambitu non dari. Hinc vero genus peculiare quantitatum formare, neutiquam concederem. Quodsi quis dicat, triangulum rectilineum aequilaterum rectangulum impossibile esse, nemo erit qui neget. At si tale triangulum impossibile tamquam novum triangulorum genus contemplari, aliasque triangulorum proprietates ad illud applicare voluerit, ecquis risum teneat? Hoc esset verbis ludere seu potius abuti. - Quamvis vero etiam summi mathematici saepius veritates, quae quantitatum ad quas spectant possibilitatem manifesto supponunt, ad tales quoque applicaverint quarum possibilitas adhuc dubia erat; neque abnuerim, huiusmodi licentias plerumque ad solam formam et quasi velamen ratiociniorum pertinere, quod veri geometrae acies mox penetrare possit: tamen consultius, scientiaeque, quae tamquam perfectissimum claritatis et certitudinis exemplar merito celebratur, sublimitate magis dignum videtur, tales libertates aut omnino proscribere, aut saltem parcius neque alias ipsis uti, nisi ubi etiam minus exercitati perspicere valeant, rem etiam absque illarum subsidio etsi forsan minus breviter tamen aeque rigorose absolvi potuisse. - Ceterum haud negaverim, ea quae hic contra impossibilium abusum dixi, quodam respectu etiam contra imaginarias obiici posse : sed harum vindicationem nec non totius huius rei expositionem uberiorem ad aliam occasionem mihi reservo.
beat, patet, per continuationem huius operationis functionem \(\boldsymbol{X}\) tandem in factores reales simplices vel duplices resolutum iri, aut, si pro singulis factoribus realibus duplicibus binos imaginarios simplices adhibere mavis, in \(m\) factores simplices.

\section*{5.}
Prima theorematis demonstratio illustri geometrae D'Aumabert debetur, \(R e\) cherches sur le calcul intégral, Histoire de l'Acad. de Berlin, Année 1746. p. 182 sqq. Eadem extat in Bodganvilue, Traité du calcul intégral, à Paris 1754. p. 47 sqq. Methodi huius praecipua momenta haec sunt.

Primo ostendit, si functio quaecunque \(\boldsymbol{X}\) quantitatis variabilis \(x\) fiat \(=0\) aut pro \(x=0\) aut pro \(x=\infty\), atque valorem infinite parvum realem positivum nancisci possit tribuendo ipsi \(x\) valorem realem: hanc functionem etiam valorem infinite parvum realem negativum obtinere posse per valorem ipsius \(x\) vel realem vel sub forma imaginaria \(p+q \sqrt{ }-1\) contentum. Scilicet designante \(Q\) valorem infinite parvum ipsius \(\boldsymbol{X}\), et \(\omega\) valorem respondentem ipsius \(x\), asserit \(\omega\) per seriem valde convergentem \(a Q^{\alpha}+b Q^{b}+c Q^{\gamma}\) etc. exprimi posse, ubi exponentes \(\alpha, b, \gamma\) etc. sint quantitates rationales continuo crescentes, et quae adeo ad minimum in distantia certa ab initio positivae evadant, terminosque, in quibus adsint, infinite parvos reddant. Iam si inter omnes hos exponentes nullus occurrat, qui sit fractio denominatoris paris, omnes terminos seriei reales fieri tum pro positivo tum pro negativo valore ipsius \(Q\); si vero quaedam fractiones denominatoris paris inter illos exponentes reperiantur, constare, pro valore negativo ipsius \(Q\) terminos respondentes in forma \(p+q \sqrt{ }-1\) contentos esse. Sed propter infinitam seriei convergentiam in casu priori sufficere, si terminus primus (i. e. maximus) solus retineatur, in posteriori ultra eum terminum, qui partem imaginariam primus producat, progredi opus non esse.

Per similia ratiocinia ostendi posse, si \(\boldsymbol{X}\) valorem negativum infinite parvum ex valore reali ipsius \(x\) assequi possit: functionem illam valorem realem positivum infinite parvum ex valore reali ipsius \(x\) vel ex imaginario sub forma \(p+q \sqrt{ }-1\) contento adipisci posse.

Hinc secundo concludit, etiam valorem aliquem realem finitum ipsius \(\boldsymbol{X}\) dari, in casu priori negativum, in posteriori positivum, qui ex valore imaginario ipsius \(x\) sub forma \(p+q \sqrt{ }-1\) contento produci possit.

Hine sequitur, si \(\boldsymbol{X}\) sit talis functio ipsius \(x\), quae valorem realem \(V\) ex valore ipsius \(x\) reali \(v\) obtineat. atque etiam valorem realem quantitate infinite parva vel maiorem vel minorem ex valore reali ipsius \(x\) assequatur, eandem etiam valorem realem quantitate infinite parva atque adeo finita vel minorem vel maiorem quam \(V\) (resp.) recipere posse, tribuendo ipsi \(x\) valorem sub forma \(p+q \sqrt{ }-1\) contentum. Hoc nullo negotio ex praecc. derivatur, si pro \(\mathrm{X}\) substitui concipitur \(V+Y\), et pro \(x, v+y\).

Tandém affirmat ill. D'ALEMBERT, si \(\boldsymbol{X}\) totum intervallum aliquod inter duos valores reales \(\boldsymbol{R}, \boldsymbol{S}\) percurrere posse supponatur (i. e. tum ipsi \(R\), tum ipsi \(\boldsymbol{S}\), tum omnibus valoribus realibus intermediis aequalis fieri), tribuendo ipsi \(x\) valores semper in forma \(p+q \sqrt{ }-1\) contentos; functionem \(X\) quavis quantitate finita reali adhuc augeri vel diminui posse (prout \(S>R\) vel \(S<R\) ), manente \(x\) semper sub forma \(p+q \sqrt{ }-1\). Si enim quantitas realis \(U\) daretur (inter quam et \(\boldsymbol{R}\) supponitur \(\boldsymbol{S}\) iacere), cui \(\boldsymbol{X}\) per talem valorem ipsius \(x\) aequalis fieri non posset, necessario valorem maximum ipsius \(\mathrm{X}\) dari (scilicet quando \(\boldsymbol{S}>\boldsymbol{R}\); minimum vero, quando \(S<R\) ), puta \(T\), quem ex valore ipsius \(x, p+q \sqrt{ }-1\), consequeretur, ita ut ipsi \(x\) nullus valor sub simili forma contentus tribui posset, qui functionem \(X\) vel minimo excessu propius versus \(U\) promoveret. Iam si in aequatione inter \(\boldsymbol{X}\) et \(x\) pro \(x\) ubique substituatur \(p+q \sqrt{ }-1\). atque tum pars realis, tum pars, quae factorem \(V-1\) implicet, hoc omisso, cifrae aequentur: ex duabus aequationibus hinc prodeuntibus (in quibus \(p, q\) et \(\boldsymbol{X}\) cum constantibus permixtae occurrent) per eliminationem duas alias elici posse, in quarum altera \(p, \boldsymbol{X}\) et constantes reperiantur, altera a \(p\) libera solas \(q, \boldsymbol{X}\) et constantes involvat. Qamobrem quum \(\mathbf{X}\) per valores reales ipsarum \(p, q\) omnes valores \(\mathrm{ab}\) \(R\) usque ad \(T\) percurrerit, per praecc. \(X\) versus valorem \(U\) adhuc propius accedere posse tribuendo ipsius \(p, q\) valores tales \(\alpha+\gamma \sqrt{ }-1, b+\delta V-1\) resp. Hinc vero fieri \(x=\alpha-\delta+(\gamma+b) \sqrt{ }-1\), i. e. adhuc sub forma \(p+q \sqrt{ }-1\) esse, contra hyp.

Iam si \(\boldsymbol{X}\) functionem talem ut \(x^{m}+\boldsymbol{A} x^{m-1}+B x^{m-2}+\) etc. \(+M\) denotare supponitur, nullo negotio perspicitur, ipsi \(x\) tales valores reales tribui posse, ut \(X\) totum aliquod intervallum inter duos valores reales percurrat. Quare \(x\) valorem aliquem sub forma \(p+q \sqrt{ }-1\) contentum talem etiam nancisci poterit, unde \(\boldsymbol{X}\) fiat \(=0\). Q. E. D. \({ }^{*}\) )

*) Observare convenit, ill. D'AlEMBERt in sua huius demonstrationis expositione considerationes geome-

6.

Quae contra demonstrationem D'AyEMBERTianam obiici posse videntur, ad haec fere redeunt.

\begin{enumerate}
  \item Ill. D'A. nullum dubium movet de existentia valorum ipsius \(x\), quibus valores dati ipsius \(X\) respondeant, sed illam supponit, solamque formam istorum valorum investigat.
\end{enumerate}

Quamvis vero haec obiectio per se gravissima sit, tamen hic ad solam dictionis formam pertinet, quae facile ita corrigi potest, ut illa penitus destruatur.

\begin{enumerate}
  \setcounter{enumi}{1}
  \item Assertio, \(\omega\) per talem seriem qualem ponit semper exprimi posse. certo est falsa, si \(\mathbf{X}\) etiam functionem quamlibet transscendentem designare debet (uti D'A. pluribus locis innuit). Hoc e. g. manifestum est, si ponitur \(X=e^{\frac{1}{x}}\), sive \(x=\frac{1}{\log X}\). Attamen si demonstrationem ad eum casum restringimus, ubi \(X\) est functio algebraica ipsius \(x\) (quod in praesenti negotio sufficit), propositio utique est vera. - Ceterum D'A. nihil pro confirmatione suppositionis suae attulit; cel. BovgaINvile supponit, \(\mathrm{X}\) esse functionem algebraicam ipsius \(x\), et ad inventionem seriei parallelogrammum NewTonianum commendat.

  \item Quantitatibus infinite parvis liberius utitur, quam cum geometrico rigore consistere potest aut saltem nostra aetate (ubi illae merito male audiunt) ab analysta scrupuloso concederetur, neque etiam saltum a valore infinite parvo ipsius \(Q\) ad finitum satis luculenter explicavit. Propositionem suam, \(Q\) etiam valorem aliquem finitum consequi posse, non tam ex possibilitate valoris infinite parvi ipsius \(Q\) concludere videtur quam inde potius, quod denotante \(Q\) quantitatem valde parvam, propter magnam seriei convergentiam, quo plures termini seriei accipiantur, eo propius ad valorem verum ipsius \(\omega\) accedatur, aut, quo plurium partium summa pro \(\omega\) accipiatur, eo exactius aequationi, quae relationem inter \(\omega\) et \(Q\) sive \(x\) et \(\boldsymbol{X}\) exhibeat, satisfactum iri. Praeterea quod tota haec argumentatio nimis vaga videtur, quam ut ulla conclusio rigorosa inde colligi possit: observo, utique dari series, quae quantumvis parvus valor quantitati,
\footnotetext{tricas adhibuisse, atque \(X\) tamquam abscissam, \(x\) tamquam ordinatam curvae spectavisse (secundum morem omnium geometrarum primae huius saeculi partis, apud quos notio functionum minus usitata erat). Quia vero omnia ipsius ratiocinia, si ad ipsorum essentiam solam respicis, nullis principiis geometricis, sed pure analyticis innituntur, et curva imaginaria, ordinataeque imaginariae expressiones duriores esse lectoremque hodiernum facilius offendere posse videntur, formam repraesentationis mere analyticam hic adhibere malui. Hanc annotationem ideo-adieci, ne quis demonstrationem D'Alembertianam ipsam cum hac succincta expositione com. parans aliquid essentiale immutatum esse suspicetur.
}
secundum cuius potestates progrediuntur, tribuatur, nihilominus semper diver-gant, ita ut si modo satis longe continuentur, ad terminos quavis quantitate data maiores pervenire possis*). 'Hoc evenit, quando coëfficientes seriei progressionem hypergeometricam constituunt. Quamobrem necessario demonstrari debuisset, talem seriem hypergeometricam in casu praesenti provenire non posse.

\end{enumerate}

Ceterum mihi videtur, ill. D'A. hic non recte ad series infinitas confugisse. hasque ad stabiliendum theorema hoc fundamentale doctrinae aequationum haud idoneas esse.

\begin{enumerate}
  \setcounter{enumi}{3}
  \item Ex suppositione, \(X\) obtinere posse valorem \(S\) neque vero valorem \(U\), nondum sequitur, inter \(S\) et \(U\) necessario valorem \(T\) iacere, quem \(X\) attingere sed non superare possit. Superest adhuc alius casus: scilicet fieri posset, ut inter \(S\) et \(U\) limes situs sit, ad quem accedere quidem quam prope velis possit \(\boldsymbol{X}\), ipsum vero nihilominus numquam attingere. Ex argumentis ab ill. D'A. allatis tantummodo sequitur, \(\boldsymbol{X}\) omnem valorem, quem attigerit, adhuc quantitate finita superare posse, puta quando evaserit \(=S\), adhuc quantitate aliqua finita \(Q\) augeri posse; quo facto, novum incrementum \(Q^{\prime}\) accedere, tunc iterum augmentum \(Q^{\prime \prime}\) etc., ita ut quotcunque incrementa iam adiecta sint. nullum pro ultimo haberi debeat, sed semper aliquod novum accedere possit. At quamvis multitudo incrementorum possibilium nullis limitibus sit circumscripta: tamen utique fieri posset, ut si incrementa \(Q, Q^{\prime}, Q^{\prime \prime}\) etc. continuo decrescerent, nihilominus summa \(S+Q+Q^{\prime}+Q^{\prime \prime}\) etc. limitem aliquem numquam attingeret, quotcunque termini considerentur.
\end{enumerate}

Quamquam hic casus occurrere non potest, quando \(X\) designat functionem algebraicam integram ipsius \(x\) : tamen sine demonstratione, hoc fieri non posse, methodus necessario pro incompleta habenda est. Quando vero \(\boldsymbol{X}\) est functio transscendens, sive etiam algebraica fracta, casus ille utique locum habere potest, e.g. semper quando valori cuidam ipsius \(\boldsymbol{X}\) valor infinite magnus ipsius \(x\) respon-
\footnotetext{*) Hacce occasione obiter adnoto, ex harum serierum numero plurimas esse, quae primo aspectu maxime convergentes videantur, e. g. ad maximam partem eas, quibus ill. Euckr in parte poster. Inst. Calc. Diff. Cap. VI, ad summam aliarum serierum quam proxime assignandam utitur p. 441-474 (reliquae enim series p. 475-478 revera convergere possunt), quod, quantum scio, a nemine hucusque observatum est. Quocirca magnopere optandum esset, ut dilucide et rigorose ostenderetur, cur huiusmodi series, quae primo citissime, dein paullatim lentius lentiusque convergunt, tandemque magis magisque divergunt, nihilominus summam proxime veram suppeditent, si modo non nimis multi termini capiantur, et quousque talis summa pro exacta tuto haberi possit?
}
det. Tum methodus D'Alembertiana non sine multis ambagibus, et in quibusdàm casibus nullo forsan modo, ad principia indubitata reduci posse videtur.

Propter has rationes demonstrationem D'ALEMBERTianam pro satisfaciente habere nequeo. Attamen hoc non obstante verus demonstrationis nervus probandi per omnes obiectiones neutiquam infringi mihi videtur, credoque eidem fundamento (quamvis longe diversa ratione, et saltem maiori circumspicientia) non solum demonstrationem rigorosam theorematis nostri superstrui, sed ibinde omnia peti posse, quae circa aequationum transscendentium theoriam desiderari queant. De qua re gravissima alia occasione fusius agam; conf. interim infra art. 24.

7.

Post d'Alembertum'ill. Euler disquisitiones suas de eodem argumento promulgavit, Recherches sur les racines imaginaires des équations, Hist. de l Acad. de Berlin A.1749, p. 223 sqq. Methodum duplicem hic tradidit: prioris summa continetur in sequentibus.

Primo ill. E. suscipit demonstrare, si \(m\) denotet quamcunque dignitatem numeri 2, functionem \(x^{2 m}+B x^{2 m-2}+C x^{2 m-3}+\) etc. \(+M=X\) (in qua coëfficiens termini secundi est \(=0)\) semper in duos factores reales resolvi posse, in quibus \(x\) usque ad \(m\) dimensiones ascendat. Ad hunc finem duos factores assumit,

\[
x^{m}-u x^{m-1}+\alpha x^{m-2}+b x^{m-3}+\text { etc., et } x^{m}+u x^{m-1}+\lambda x^{m-2}+\mu x^{m-3} \text { etc. }
\]

ubi coëfficientes \(u, \alpha, b\) etc. \(\lambda, \mu\) etc. adhuc incogniti sunt, horumque productum aequale ponit functioni \(X\). Tum coëfficientium comparatio suppeditat \(2 m-1\) aequationes, manifestoque demonstrari tantummodo debet, incognitis \(u, \alpha ; b\) etc. \(\lambda, \mu\) etc. (quarum multitudo etiam est \(2 m-1\) ) tales valores reales tribui posse. qui aequationibus illis satisfaciant. Iam E. affirmat, si primo \(u\) tamquam cognita consideretur, ita ut multitudo incognitarum unitate minor sit quam multitudo aequationum. his secundum methodos algebraicas notas rite combinatis omnes \(\alpha, b\) etc. \(\lambda, \dot{\mu}\) etc. rationaliter et sine ulla radicum extractione per \(u\) et coëfficientes \(B, C\) etc. determinari posse, adeoque valores reales nancisci, simulac \(u\) realis fiat. Praeterea vero omnes \(\alpha, b\) etc. \(\lambda, \mu\) etc. eliminari poterunt, ita ut prodeat aequatio \(U=0\), ubi \(U\) erit functio integra solius \(u\) et coëfficientium cognitorum. Hanc aequationem ipsam per methodum eliminationis vulgarem evolvere, opus immensum foret, quando aequatio proposita \(X=0\) est gradus ali-
quantum alti; et pro gradu indeterminato, plane impossibile (iudice ipso E. p. 239). Attamen hic sufficit, unam illius aequationis proprietatem novisse, scilicet quod terminus ultimus in \(U\) (qui incognitam \(u\) non implicat) necessario est negativus, unde sequi constat, aequationem ad minimum unam radicem realem habere, sive \(u\) et proin etiam \(\alpha, b\) etc. \(\lambda, \mu\) etc. ad minimum uno modo realiter determinari posse: illam vero proprietatem per sequentes reflexiones confirmare licet. Quum \(x^{m}-u x^{m-1}+\alpha x^{m-2}+\) etc. supponatur esse factor functionis \(X\) : necessario \(u\) erit summa \(m\) radicum aequationis \(X=0\), adeoque totidem valores habere debebit, quot modis diversis ex \(2 m\) radicibus \(m\) excerpi possunt, sive per principia calculi combinationum \(\frac{2 m .2 m-1.2 m-2 \ldots \ldots m+1}{1 \cdot 2 \cdot 3 \cdot \cdot m}\) valores. Hic numerus semper erit impariter par (demonstrationem haud difficilem supprimo): si itaque ponitur \(=2 k\), ipsius semissis \(k\) impar erit; aequatio \(U=0\) vero erit gradus \(2 k^{\text {ti }}\). Iam quoniam in aequatione \(X=0\) terminus secundus deest: summa omnium \(2 m\) radicum erit 0 ; unde patet, si summa quarumcunque \(m\) radicum fuerit \(+p\), reliquarum summam fore \(-p\), i. e. si \(+p\) est inter valores ipsius \(u\), etiam \(-p\) inter eosdem erit. Hinc E. concludit, \(U\) esse productum ex \(k\) factoribus duplicibus talibus \(u u-p p, u u-q q, u u-r r\) etc., denotantibus \(+p\), \(-p,+q,-q\) etc. omnes \(2 k\) radices aequationis \(U=0\), unde, propter multitudinem imparem horum factorum, terminus ultimus in \(U\) erit quadratum producti \(p q r\) etc. signo negativo affectum. Productum autem \(p q r\) etc. semper ex coëfficientibus \(B, C\) etc. rationaliter determinari potest, adeoque necessario erit quantitas realis. Huius itaque quadratum signo negativo affectum certo erit quantitas negativa. Q. E. D.

Quum hi duo factores reales ipsius \(X\) sint gradus \(m^{\text {ti }}\) atque \(m\) potestas numeri 2: eadem ratione uterque rursus in duos factores reales \(\frac{1}{2} m\) dimensionum resolvi poterit. Quoniam vero per repetitam dimidiationem numeri \(m\) necessario tandem ad binarium pervenitur, manifestum est, per continuationem operationis functionem \(\boldsymbol{X}\) tandem in factores reales secundi gradus resolutam haberi.

Quodsi vero functio talis proponitur, in qua terminus secundus non deest, puta \(x^{2 m}+A x^{2 m-1}+B x^{2 m-2}+\) etc. \(+M\), designante etiamnum \(2 m\) potestatem binariam, haec per substitutionem \(x=y-\frac{A}{2 m}\) transibit in similem functionem termino secundo carentem. Unde facile concluditur, etiam illam functionem in factores reales secundi gradus resolubilem esse.

Denique proposita functione gradus \(n^{\text {ti }}\), designante \(n\) numerum, qui non
est potestas binaria: ponatur potestas binaria proxime maior quam \(n,=2 m\). multipliceturque functio proposita per \(2 m-n\) factores simplices reales quoscunque. Ex resolubilitate producti in factores reales secundi gradus, nullo negotio derivatur, etiam functionem propositam in factores reales secundi vel primi gradus resolubilern esse debere.

8.

Contra hanc demonstrationem obiici potest

\begin{enumerate}
  \item Regulam, secundum quam E. concludit, ex \(2 m-1\) aequationibus \(2 m-2\) incognitas \(\alpha, b\) etc. \(\lambda, \mu\) etc. omnes rationaliter determinari posse, neutiquam esse generalem, sed saepissime exceptionem pati. Si quis e.g. in art. 3, aliqua incognitarum tamquam cognita spectata, reliquas per hanc et coëfficientes datos rationaliter exprimere tentat, facile inveniet, hoc esse impossibile, nullamque quantitatum incognitarum aliter quam per aequationem \(m-1^{\text {ti }}\) gradus determinari posse. Quamquam vero hic statim a priori perspici potest, illud necessario ita evenire debuisse: tamen merito dubitari posset, annon etiam in casu praesenti pro quibusdam valoribus \(m\) res eodem modo se habeat, ut incognitae \(\alpha, b\) etc. \(\lambda, \mu\) etc. ex \(u, B, C\) etc. aliter quam per aequationem gradus forsan maioris quam \(2 m\) determinari nequeant. Pro eo casu, ubi aequatio \(X=0\) est quarti gradus, E. valores rationales coëfficientium per \(u\) et coëfficientes datos eruit; idem vero etiam in omnibus aequationibus altioribus fieri posse, utique explicatione ampliori egebat. - Ceterum operae pretium esse videtur, in formulas illas, quae \(\alpha, b\) etc. rationaliter per \(u, B, C\) etc. exprimant, profundius et generalissime inquirere; de qua re aliisque ad eliminationis theoriam (argumentum haudquaquam exhaustum) pertinentibus alia occasione fusius agere suscipiam.

  \item Etiamsi autem demonstratum fuerit, cuiusvis gradus sit aequatio \(X=0\), semper formulas inveniri posse, quae ipsaś \(\alpha, b\) etc. \(\lambda, \mu\) etc. rationaliter per \(u, B . C\) etc. exhibeant: tamen certum est, pro valoribus quibusdam determinatis coëfficientium \(B, C\) etc. formulas illas indeterminatas evadere posse, ita ut non solum impossibile sit, incognitas illas rationaliter ex \(u, B, C\) etc. definire, sed adeo revera quibusdam in casibus valori alicui reali ipsius \(u\) nulli valores reales ipsarum \(\alpha, b\) etc. \(\lambda, \mu\) etc. respondeant. Ad confirmationem huius rei brevitatis gratia ablego lectorem ad diss. ipsam E., ubi p. 236 aequatio quarti gradus fusius explicata est. Statim quisque videbit, formulas pro coëfficientibus \(\alpha, b\) indeter-
minatas fieri, si \(C=0\) et pro \(u\) assumatur valor 0 , illorumque valores non solum sine extractione radicum assignari non posse, sed adeo ne reales quidem esse, si fuerit \(B B-4 D\) quantitas negativa. Quamquam vero in hoc casu \(u\) adhuc alios valores reales habere, quibus valores reales ipsarum \(\alpha .6\) respondeant, facile perspici potest: tamen vereri aliquis posset, ne huius difficultatis enodatio (quam E. omnino non attigit) in aequationibus altioribus multo maiorem operam facessat. Certe haec res in demonstratione exacta neutiquam silentio praeteriri debet.

  \item Ill. E. supponit tacite, aequationem \(X=0\) habere \(2 m\) radices, harumque summam statuit \(=0\), ideo quod terminus secundus in \(X\) abest. Quomodo de hac licentia (qua omnes auctores de hoc argumento utuntur) sentiam, iam supra art. 3 declaravi. Propositio, summam omnium radicum aequationis alicuius coëfficienti primo, mutato signo, aequalem esse, ad alias aequationes applicanda non videtur, nisi quae radices habent: iam quum per hanc ipsam demonstrationem evinci debeat, aequationem \(X=0\) revera radices habere, haud permissum videtur, harum existentiam supponere. Sine dubio ii, qui huius paralogismi fallaciam nondum penetraverunt, respondebunt, hic non demonstrari, aequationi \(X=0\) satisfieri posse (nam hoc dicere vult expressio, eam habere radices), sed tantummodo, ipsi per valores ipsius \(x\) sub forma \(a+b \sqrt{ }-1\) contentos satisfieri posse: illud vero tamquam axioma supponi. At quum aliae quantitatum formae, praeter realem et imaginariam \(a+b \vee-1\) concipi nequeant, non satis luculentum videtur, quomodo id, quod demonstrari debet, ab eo, quod tamquam axioma supponitur, differat; quin adeo si possibile esset adhuc alias formas quantitatum excogitare, puta formam \(F, F^{\prime}, F^{\prime \prime}\) etc.: tamen sine demonstratione admitti non deberet. cuius aequationi per aliquem valorem ipsius \(x\) aut realem, aut sub forma \(a+b V-1\), aut sub forma \(F\), aut sub \(F^{\prime}\) etc. contentum satisfieri posse. Quamobrem axioma illud alium sensum habere nequit quam hunc: Cuivis aequationi satisfieri potest aut per valorem realem incognitae, aut per valorem imaginarium sub forma \(a+b \vee-1\) contentum, aut forsan per valorem sub forma alia hucusque ignota contentum, aut per valorem, qui sub nulla omnino forma continetur. Sed quomodo huiusmodi quantitates, de quibus ne ideam quidem fingere potes - vera umbrae umbra - summari aut multiplicari possint, hoc ea perspicuitate, quae in mathesi semper postulatur, certo non intelligitur *).

\end{enumerate}

*) Tota haec res multum illustrabitur per aliam disquisitionem sub prelo iam sudantem, ubi in argu-

Ceterum conclusiones, quas E. ex suppositione sua elicuit, per has obiectiones haudquaquam suspectas reddere volo; quin potius certus sum, illas per methodum neque difficilem neque ab Euxeriana multum diversam ita comprobari posse, ut nemini vel minimus scrupulus superesse debeat. Solam formam reprehendo, quae quamvis in inveniendis novis veritatibus magnae utilitatis esse possit, tamen in demonstrando, coram publico, minime probanda videtur.

\begin{enumerate}
  \setcounter{enumi}{3}
  \item Pro demonstratione assertionis, productum \(p q r\) etc. ex coëfficientibus in \(\mathbf{X}\) rationaliter determinari posse, ill. E. nihil omnino attulit. Omnia, quae hac de re in aequationibus quarti gradus explicat, haec sunt (ubi \(\mathfrak{a}, \mathfrak{b}, \mathfrak{c}, \mathfrak{b}\) sunt radices aequationis propositae \(\left.x^{4}+B x x+C x+D=0\right)\) :
\end{enumerate}

'On m'objectera sans doute, que j'ai supposé ici, que la quantité \(p q r\) était une quantité réelle, et que son quarré \(p p q q r r\) était affirmatif; ce qui était encore douteux, vu que les racines \(\mathfrak{a}, \mathfrak{b}, \mathfrak{c}, \boldsymbol{b}\) étant imaginaires. il pourrait bien arriver, que le quarré de la quantité \(p q r\), qui en est composée, fut négatif. Or je réponds à cela que ce cas ne saurait jamais avoir lieu; car quelque imaginaires que soient les racines \(\mathfrak{a}, \mathfrak{b}, \mathfrak{c}, \mathfrak{b}\), on sait pourtant, qu'il doit y avoir \(\mathfrak{a}+\mathfrak{b}+\mathfrak{c}+\mathfrak{b}=0\) : \(\left.\mathfrak{a} \mathfrak{b}+\mathfrak{a} \mathfrak{c}+\mathfrak{a} \mathfrak{d}+\mathfrak{b} \mathfrak{c}+\mathfrak{b} \mathfrak{b}+\mathfrak{c} \mathfrak{b}=B ; \quad \mathfrak{a} \mathfrak{b} \mathfrak{c}+\mathfrak{a} \mathfrak{b} \mathfrak{b}+\mathfrak{a} \mathfrak{c} \mathfrak{b}+\mathfrak{b} \mathfrak{c} \mathfrak{b}=-C^{*}\right) ; \quad \mathfrak{a} \mathfrak{b} \mathfrak{c} \mathfrak{b}=D\), ces quantités \(B, C, D\) étant réelles. Mais puisque \(p=\mathfrak{a}+\mathfrak{b}, q=\mathfrak{a}+\mathfrak{c}\), \(r=\mathfrak{a}+\mathfrak{b}\), leur produit \(p q \mathfrak{r}=(\mathfrak{a}+\mathfrak{b})(\mathfrak{a}+\mathfrak{c})(\mathfrak{a}+\mathfrak{b})\) est déterminable comme on sait, par les quantités \(B, C, D\), et sera par conséquent réel, tout comme nous avons vu, qu'il est effectivement \(p q r=-C\), et \(p p q q r r=C C\). On reconnaîtra aisément de même, que dans les plus hautes équations cette même circonstance doit avoir lieu. et qu'on ne saurait me faire des objections de ce côté.' Conditionem, productum \(p q r\) etc. rationaliter per \(B, C\) etc. determinari posse, E. nullibi adiecit, attamen semper subintellexisse videtur, quum absque illa demonstratio nullam vim habere possit. Iam verum quidem est in aequationibus quarti gradus, si productum \((\mathfrak{a}+\mathfrak{b})(\mathfrak{a}+\mathfrak{c})(\mathfrak{a}+\mathfrak{b})\) evoluatur, obtineri \(\mathfrak{a} \mathfrak{a}(\mathfrak{a}+\mathfrak{b}+\mathfrak{c}+\mathfrak{d})+\mathfrak{a} \mathfrak{b} \mathfrak{c}+\mathfrak{a} \mathfrak{b} \mathfrak{d}+\mathfrak{a} \mathfrak{c} \mathfrak{b}+\mathfrak{b} \mathfrak{c} \mathfrak{d}=-C\), attamen non satis perspicuum videtur, quomodo in omnibus aequationibus superioribus productum rationaliter

mento longe quidem diverso, nihilominus tamen analogo, licentiam similem prorsus eodem iure usurpare potuissem, ut hic in aequationibus ab omnibus analystis factum est. Quamquam vero plurium veritatum demonstrationes adiumento talium tictionum paucis verbis absolvere licuisset, quae absque his perquam difficiles evadunt et subtilissima artificia requirunt, tamen illis omnino abstinere malui, speroque, paucis me satisfacturum fuisse, si analystarum methodum imitatus essem.

\textit{) E. per errorem habet \(C\), unde etiam postea perperam statuit \(p q r=C\).
per coëfficientes determinari possit. Clar. DE FONCENEx, qui primus hoc observavit (Miscell. phil. math. soc. Taurin. T. I. p. i17), recte contendit, sine demonstratione rigorosa huius propositionis methodum omnem vim perdere, illam vero satis difficilem sibi videri confitetur, et quam viam frustra tentaverit, enarrat}). Attamen haec res haud difficulter per methodum sequentem (cuius summam addigitare tantummodo hic possum) absolvitur: Quamquam in aequationibus quarti gradus non satis clarum est, productum \((\mathfrak{a}+\mathfrak{b})(\mathfrak{a}+\mathfrak{c})(\mathfrak{a}+\mathfrak{b})\) per coëfficientes \(B, C, D\) determinabile esse, tamen facile perspici potest, idem productum etiam esse \(=(\mathfrak{b}+\mathfrak{a})(\mathfrak{b}+\mathfrak{c})(\mathfrak{b}+\mathfrak{b})\), nec non \(=(\mathfrak{c}+\mathfrak{a})(\mathfrak{c}+\mathfrak{b})(\mathfrak{c}+\mathfrak{b})\), denique etiam \(=(\mathfrak{b}+\mathfrak{a})(\mathfrak{b}+\mathfrak{b})(\mathfrak{b}+\mathfrak{c})\). Quare productum \(p q r\) erit quadrans summae \((\mathfrak{a}+\mathfrak{b})(\mathfrak{a}+\mathfrak{c})(\mathfrak{a}+\mathfrak{b})+(\mathfrak{b}+\mathfrak{a})(\mathfrak{b}+\mathfrak{c})(\mathfrak{b}+\mathfrak{b})+(\mathfrak{c}+\mathfrak{a})(\mathfrak{c}+\mathfrak{b})(\mathfrak{c}+\mathfrak{b})+(\mathfrak{b}+\mathfrak{a})(\mathfrak{b}+\mathfrak{b})(\mathfrak{b}+\mathfrak{c})\), quam, si evolvatur, fore functionem rationalem integram radicum \(\mathfrak{a}, \mathfrak{b}, \mathfrak{c}, \mathfrak{b}\) talem, in quam omnes eadem ratione ingrediantur, nullo negotio a priori praevideri potest. Tales vero functiones semper rationaliter per coëfficientes aequationis, cuius radices sunt \(\mathfrak{a}, \mathfrak{b}, \mathfrak{c}, \mathfrak{b}\), exprimi possunt. - Idem etiam manifestum est, si productum \(p q r\) sub hanc formam redigatur:

\[
\frac{1}{2}(\mathfrak{a}+\mathfrak{b}-\mathfrak{c}-\mathfrak{b}) \times \frac{1}{2}(\mathfrak{a}+\mathfrak{c}-\mathfrak{b}-\mathfrak{b}) \times \frac{1}{2}(\mathfrak{a}+\mathfrak{b}-\mathfrak{b}-\mathfrak{c})
\]

quod productum evolutum omnes \(\mathfrak{a}, \mathfrak{b}, \mathfrak{c}, \mathfrak{b}\) eodem modo implicaturum esse facile praevideri potest. Simul periti facile hinc colligent, quomodo hoc ad altiores aequationes applicari debeat. - Completam demonstrationis expositionem, quam hic apponere brevitas non permittit, una cum uberiori disquisitione de functionibus plures variabiles eodem modo involventibus ad aliam occasionem mihi reservo.

Ceterum observo, praeter has quatuor obiectiones, adhuc quaedam alia in demonstratione E. reprehendi posse, quae tamen silentio praetereo, ne forte censor nimis severus esse videar, praesertim quum praecedentia satis ostendere videantur, demonstrationem in ea quidem forma, in qua ab E. proposita est, pro completa neutiquam haberi posse.

Post hanc demonstrationem, E. adhuc aliam viam theorema pro aequationibus, quarum gradus non est potestas binaria. ad talium aequationum resolutionem reducendi ostendit: attamen quum methodus haec pro aequationibus quarum

*) In hanc expositionem error irrepsisse videtur, scilicet p. 118. 1. 5. loco characteris \(p\) (on choisissait seulement celles où entrait \(p\) etc.), necessario legere oportet, une méme racine quelconque de Péquation proposée, aut simile quid, quum illud nullum sensum habeat.
gradus est potestas binaria, nihil doceat, insuperque omnibus obiectionibus praecc. (praeter quartam) aeque obnoxia sit ut demonstratio prima generalis: haud necesse est illam hic fusius explicare.

9.

In eadem commentatione ill. E. theorema nostrum adhuc alia via confirmare annixus est \(p\) 263, cuius summa continetur in his: Proposita aequatione \(x^{n}+A x^{n-1}+B x^{n-2}\) etc. \(=0\), hucusque quidem expressio analytica, quae ipsius radices exprimat, inveniri non potuit, si exponens \(n>4\); attamen certum esse videtur (uti asserit E.), illam nihil aliud continere posse, quam operationes arithmeticas et extractiones radicum eo magis complicatas, quo maior sit \(n\). Si hoc conceditur, E. optime ostendit, quantumvis inter se complicata sint signa radicalia, tamen formulae valorem semper per formam \(M+N V-1\) repraesentabilem fore, ita ut \(M, N\) sint quantitates reales.

Contra hoc ratiocinium obiici potest, post tot tantorum geometrarum labores perexiguam spem superesse, ad resolutionem generalem aequationum algebraicarum umquam perveniendi, ita ut magis magisque verisimile fiat, talem resolutionem omnino esse impossibilem et contradictoriam. Hoc eo minus paradoxum videri debet, quum id, quod vulgo resolutio aequationis dicitur, proprie nihil aliud sit quam ipsius reductio ad aequationes puras. Nam aequationum purarum solutio hinc non docetur sed supponitur, et si radicem aequationis \(x^{m}=H\) per \(\sqrt[m]{H}\) exprimis, illam neutiquam solvisti, neque plus fecisti, quam si ad denotandam radicem aequationis \(x^{n}+A x^{n-1}+\) etc. \(=0\) signum aliquod excogitares, radicemque huic aequalem poneres. Verum est, aequationes puras propter facilitatem ipsarum radices per approximationem inveniendi, et propter nexum elegantem, quem omnes radices inter se habent, prae omnibus reliquis multum praestare, adeoque neutiquam vituperandum esse, quod analystae harum radices per signum peculiare denotaverunt: attamen ex eo, quod hoc signum perinde ut signa arithmetica additionis, subtractionis, multiplicationis, divisionis et evectionis ad dignitatem sub nomine expressionum analyticarum complexi sunt, minime sequitur cuiusvis aequationis radicem per illas exhiberi posse. Seu, missis verbis, sine ratione sufficienti supponitur, cuiusvis aequationis solutionem ad solutionem aequationum purarum reduci posse. Forsan non ita difficile foret, impossibilitatem iam pro quinto gradu omni rigore demonstrare, de qua re alio loco disquisitiones meas fusius proponam.

Hic sufficit, resolubilitatem generalem aequationum, in illo sensu acceptam . adhuc valde dubiam esse, adeoque demonstrationem, cuius tota vis ab illa suppositione pendet, in praesenti rei statu nihil ponderis habere.

\section*{10.}
Postea etiam clar. De Foncenex, quum in demonstratione prima Euleri defectum animadvertisset (supra art. 8 obiect. 4), quem tollere non poterat, adhuc aliam viam tentavit et in comment. laudata p. 120 in medium protulit*). Quae consistit in sequentibus.

Proposita sit aequatio \(Z=0\), designante \(Z\) functionem \(m^{\text {ti }}\) gradus incognitae \(z\). Si \(m\) est numerus impar, iam constat, aequationem hanc habere radicem realem; si vero \(m\) est par, clar. F. sequenti modo probare conatur, aequationem ad minimum unam radicem formae \(p+q \sqrt{ }-1\) habere. Sit \(m=2^{n} i\), designante \(i\) numerum imparem, supponaturque \(z z+u z+M\) esse divisor functionis \(Z\). Tunc singuli valores ipsius \(u\) erunt summae binarum radicum aequationis \(Z=0\) (mutato signo), quamobrem \(u\) habebit \(\frac{m \cdot m-1}{1 \cdot 2}=m^{\prime}\) valores, et si \(u\) per aequationem \(U=0\) determinari supponitur (designante \(U\) functionem integram ipsius \(u\) et coëfficientium cognitorum in \(Z\) ), haec erit gradus \(m^{\text {ti }}\). Facile vero perspicitur, \(m^{\prime}\) fore numerum formae \(2^{n-1} i^{\prime}\), designante \(i^{\prime}\) numerum imparem. Iam nisi \(m^{\prime}\) est impar, supponatur iterum, \(u u+u^{\prime} u+M^{\prime}\) esse divisorem ipsius \(U\), patetque per similia ratiocinia, \(u^{\prime}\) determinari per aequationem \(U^{\prime}=0\). ubi \(U^{\prime}\) sit functio \(\frac{m^{\prime} \cdot m^{\prime}-1 \text { ti }}{1 \cdot 2}\) gradus ipsius \(u^{\prime}\). Posito vero \(\frac{m^{\prime} \cdot m^{\prime}-1}{1 \cdot 2}=m^{\prime \prime}\), erit \(m^{\prime \prime}\) numerus formae \(2^{n-2} i^{\prime \prime}\), designante \(i^{\prime \prime}\) numerum imparem. Iam nisi \(m^{\prime \prime}\) est impar, statuatur \(u^{\prime} u^{\prime}+u^{\prime \prime} u^{\prime}+M^{\prime \prime}\) esse divisorem functionis \(U^{\prime}\), determinabiturque \(u^{\prime \prime}\) per aequationem \(U^{\prime \prime}=0\), quae si supponitur esse gradus \(m^{\prime \prime \prime t}, m^{\prime \prime \prime}\) erit numerus formae \(2^{n-3} i^{\prime \prime \prime}\). Manifestum est, in serie aequationum \(U=0, U^{\prime}=0\), \(U^{\prime \prime}=0\) etc. \(n^{\text {tam }}\) fore gradus imparis adeoque radicem realem habere. Statuemus brevitatis gratia \(n=3\), ita ut aequatio \(U^{\prime \prime}=0\) radicem realem \(u^{\prime \prime}\) habeat, nullo enim negotio perspicitur, pro quovis alio valore ipsius \(n\) idem ratiocinium valere. Tunc coëfficientem \(M^{\prime \prime}\) per \(u^{\prime \prime}\) et coëfficientes in \(U^{\prime}\) (quos fore functiones integras coëfficientium in \(Z\) facile intelligitur), sive per \(u^{\prime \prime}\) et coëfficientes in
\footnotetext{*) In tomo secundo eorundem Miscellaneorum p. 337 dilucidationes ad hanc commentationem continentur: attamen hae ad disquisitionem praesentem non pertinent, sed ad logarithmos quantitatum negativarum, de quibus in eadem comm. sermo fuerat.
}
\(Z\) rationaliter determinabilem fore asserit clar. de F., et proin realem. Hinc sequitur, radices aequationis \(u^{\prime} u^{\prime}+u^{\prime \prime} u^{\prime}+M^{\prime \prime}=0\) sub forma \(p+q \sqrt{ }-1\) contentas fore; eaedem vero manifesto aequationi \(U^{\prime}=0\) satisfacient: quare dabitur valor aliquis ipsius \(u^{\prime}\) sub forma \(p+q \sqrt{ }-1\) contentus. Iam coëfficiens \(M^{\prime}\) (eodem modo ut ante) rationaliter per \(u^{\prime}\) et coëfficientes in \(Z\) determinari potest, adeoque etiam sub forma \(p+q \sqrt{ }-1\) contentus erit; quare aequationis \(u u+u^{\prime} u+M^{\prime}\) radices sub eadem forma contentae erunt, simul vero aequationi \(U=0\) satisfacient, i. e. aequatio haec habebit radicem sub forma \(p+q \sqrt{ }-1\) contentam. Denique hinc simili ratione sequitur, etiam \(M\) sub eadem forma contineri, nec non radicem aequationis \(z z+u z+M=0\), quae manifesto etiam aequationi propositae \(Z=0\) satisfaciet. Quamobrem quaevis aequatio ad minimum unam radicem formae \(p+q \sqrt{ }-1\) habebit.

\section*{11.}
Obiectiones 1, 2, 3, quas contra EuLeri demonstrationem primam feci (art. 8), eandem vim contra hanc methodum habent, ea tamen differentia, ut obiectio secunda, cui EuLeri demonstratio tantummodo in quibusdam casibus specialibus obnoxia erat, praesentem in omnibus casibus attingere debeat. Scilicet a priori demonstrari potest, etiamsi formula detur, quae coëfficientem \(M^{\prime}\) rationaliter per \(u^{\prime}\) et coëfficientes in \(Z\) exprimat, hanc pro pluribus valoribus ipsius \(u^{\prime}\) necessario indeterminatam fieri debere; similiterque formulam, quae coëfficientem \(M^{\prime \prime}\) per \(u^{\prime \prime}\) exhibeat, indeterminatam fieri pro quibusdam valoribus ipsius \(u^{\prime \prime}\) etc. Hoc luculentissime perspicietur, si aequationem quarti gradus pro exemplo assumimus. Ponamus itaque \(m=4\), sintque radices aequationis \(Z=0\), hae \(\alpha, \vec{b}, \gamma, \delta\). Tum patet, aequationem \(U=0\) fore sexti gradus ipsiusque radices \(-(\alpha+b)\), \(-(\alpha+\gamma),-(\alpha+\delta),-(b+\gamma),-(b+\delta),-(\gamma+\delta)\). Aequatio \(U^{\prime}=0\) autem erit decimi quinti gradus, et valores ipsius \(u^{\prime}\) hi

\[
\begin{array}{cc}
2 \alpha+b+\gamma, & 2 \alpha+b+\delta, 2 \alpha+\gamma+\delta, 2 b+\alpha+\gamma, 2 b+\alpha+\delta, 2 b+\gamma+\delta, \\
2 \gamma+\alpha+b, & 2 \gamma+\alpha+\delta, 2 \gamma+b+\delta, 2 \delta+\alpha+b, 2 \delta+\alpha+\gamma, 2 \delta+b+\gamma \\
\alpha+b+\gamma+\delta . \alpha+b+\gamma+\delta, \alpha+b+\gamma+\delta
\end{array}
\]

Iam in hac aequatione, quippe cuius gradus est impar, subsistendum erit, habebitque ea revera radicem realem \(\alpha+b+\gamma+\delta\) (quae primo coëfficienti in \(Z\) mutato signo aequalis adeoque non modo realis sed etiam rationalis erit, si coëfficien-
tes in \(Z\) sunt rationales). Sed nullo negotio perspici potest, si formula detur, quae valorem ipsius \(M^{\prime}\) per valorem respondentem ipsius \(u^{\prime}\) rationaliter exhibeat, hanc necessario pro \(u^{\prime}=\alpha+b+\gamma+\delta\) indeterminatam fieri. Hic enim valor ter erit radix aequationis \(U^{\prime}=0\), respondebuntque ipsi tres valores ipsius \(M^{\prime}\), puta \((\alpha+b)(\gamma+\delta),(\alpha+\gamma)(b+\delta)\) et \((\alpha+\delta)(b+\gamma)\), qui omnes irrationales esse possunt. Manifesto autem formula rationalis neque valorem irrationalem ipsius \(M^{\prime}\) in hoc casu producere posset, neque tres valores diversos. Ex hoc specimine satis colligi potest, methodum clar. DE Foncenexii neutiquam esse satisfacientem, sed si ab omni parte completa reddi debeat, multo profundius in theoriam eliminationis inquiri oportere.

12.

Denique ill. La Gravge de theoremate nostro egit in comm. Sur la forme des racines imaginaires des équations, Nouv. Mém. de l'Acad. de Berlin 1772, p. 222 sqq. Magnus hic geometra imprimis operam dedit, defectus in EUleri demonstratione prima supplere et revera praesertim ea, quae supra (art. 8) obiectionem secundam et quartam constituunt, tam profunde perscrutatus est. ut nihil amplius desiderandum restet, nisi forsan in disquisitione anteriori super theoria eliminationis (cui investigatio haec tota innititur) quaedam dubia superesse videantur. - Attamen obiectionem tertiam omnino non attigit, quin etiam tota disquisitio superstructa est suppositioni, quamvis aequationem \(m^{\text {ti }}\) gradus revera \(m\) radices habere.

Probe itaque iis, quae hucusque exposita sunt, perpensis, demonstrationem novam theorematis gravissimi ex principiis omnino diversis petitam peritis haud ingratam fore spero, quam exponere statim aggredior.

\section*{13.}
\textsc{Lemma.} Denotante \(m\) numerum integrum positivum quemcunque, functio \(\sin \varphi . x^{m}-\sin m \varphi . r^{m-1} x+\sin (m-1) \varphi . r^{m}\) divisibilis erit per \(x x-2 \cos \varphi . r x+r r\).

Demonstr. Pro \(m=1\) functio illa fit \(=0\) adeoque per quemcunque factorem divisibilis; pro \(m=2\) quotiens fit \(\sin \varphi\), et pro quovis valore maiori quotiens erit \(\sin \varphi . x^{m-2}+\sin 2 \varphi . r x^{m-3}+\sin 3 \varphi . r r x^{m-4}+\) etc. \(+\sin (m-1) \varphi \cdot r^{m-2}\). Facile enim confirmatur, multiplicata hac functione per \(x x-2 \cos \varphi \cdot r x+r r\), productum functioni propositae aequale fieri.

14.

Lemma. Si quantitas \(r\) angulusque \(?\) ita sunt determinati, ut habeantur aequationes

\[
\begin{aligned}
& r^{m} \cos m \varphi+A r^{m-1} \cos (m-1) \varphi+B r^{m-2} \cos (m-2) \varphi+\text { etc. } \\
&+K r r \cos 2 \varphi+L r \cos \varphi+M=0 \\
& r^{m} \sin m \varphi+A r^{m-1} \sin (m-1) \varphi+B r^{m-2} \sin (m-2) \varphi+\text { etc. } \\
&+K r r \sin 2 \varphi+L r \sin \varphi=0
\end{aligned}
\]

functio \(x^{m}+A x^{m-1}+B x^{m-2}+\) etc. \(+K x x+L x+M=\mathbf{X}\) divisibilis erit per factorem duplicem \(x x-2 \cos \varphi . r x+r r\), simodo \(r \sin \varphi\) non \(=0\); si vero \(r \sin \varphi=0\), zadem functio divisibilis erit per factorem simplicem \(x-r \cos \varphi\).

Demonstr. I.' Ex art. praec. omnes sequentes quantitates divisibiles erunt per \(x x-2 \cos \varphi \cdot r x+r r\) :

\[
\begin{aligned}
& \sin \varphi \cdot r x^{m}-\sin m \varphi \cdot r^{m} x+\sin (m-1) \varphi \cdot r^{m+1} \\
& A \sin \varphi \cdot r x^{m-1}-A \sin (m-1) \varphi \cdot r^{m-1} x+A \sin (m-2) \varphi \cdot r^{m} \\
& B \sin \varphi \cdot r x^{m-2}-B \sin (m-2) \varphi \cdot r^{m-2} x+B \sin (m-3) \varphi \cdot r^{m-1} \\
& \text { etc. etc. } \\
& K \sin \varphi \cdot r x x-K \sin 2 \varphi \cdot r r x+K \sin \varphi \cdot r^{3} \\
& L \sin \varphi \cdot r x \quad-L \sin \varphi \cdot r x \\
& M \sin \varphi \cdot r \quad * \quad+M \sin (-\varphi) \cdot r
\end{aligned}
\]

Quamobrem etiam summa harum quantitatum per \(x x-2 \cos \varphi \cdot r x+r r\) divisibilis erit. At singularum partes primae constituunt summam \(\sin \varphi . r \mathbf{X}\); secundae additae dant 0 , propter [2]; tertiarum vero aggregatum quoque evanescere, facile perspicitur, si [1] multiplicatur per \(\sin \varphi\), [2] per \(\cos \varphi\), productumque illud ab hoc subducitur. Unde sequitur, functionem \(\sin \varphi . r \boldsymbol{X}\) divisibilem esse per \(x x-2 \cos \varphi . r x+r r\), adeoque, nisi fuerit \(r \sin \varphi=0\), etiam functionem \(\boldsymbol{X}\). Q. E. P.

II. Si vero \(r \sin \varphi=0\), erit aut \(r=0\) aut \(\sin \varphi=0\). In casu priori erit \(M=0\), propter [1]. adeoque \(\boldsymbol{X}\) per \(x\) sive per \(x-r \cos \varphi\) divisibilis; in posteriori erit \(\cos \varphi= \pm 1, \cos 2 \varphi=+1, \cos 3 \varphi= \pm 1\) et generaliter \(\cos n \varphi=\cos \varphi^{n}\). Quare propter [1] fiet \(X=0\), statuendo \(x=r \cos \varphi\), et proin functio \(X\) per \(x-r \cos \varphi\) erit divisibilis. Q. E. S.

\section*{15.}
Theorema praecedens plerumque adiumento quantitatum imaginariarum demonstratur, vid. EuLer Introd. in Anat. Inf. T.I. p. 110 ; operae pretium esse duxi, ostendere, quomodo aeque facile absque illarum auxilio erui possit. Manifestum iam est, ad demonstrationem theorematis nostri nihil aliud requiri quam ut ostendatur: Proposita functione quacunque \(\boldsymbol{X}\) formae \(x^{m}+A x^{m-1}+B x^{m-2}+\) etc. \(+L x+M, r\) et \(\varphi\) ita determinari posse, ut aequationes [1] et [2] locum habeant. Hinc enim sequetur, \(\boldsymbol{X}\) habere factorem realem primi vel secundi gradus; divisio autem necessario producet quotientem realem inferioris gradus, qui ex eadem ratione quoque factorem primi vel secundi gradus habebit. Per continuationem huius operationis \(\boldsymbol{X}\) tandem in factores reales simplices vel duplices resolvetur. Illud itaque theorema demonstrare, propositum est sequentium disquisitionum.

\section*{16.}
Concipiatur planum fixum infinitum (planum tabulae, fig. 1), et in hoc recta fixa infinita \(G C\) per punctum fixum \(C\) transiens. Assumta aliqua longitudine pro unitate ut omnes rectae per numeros exprimi possint, erigatur in quovis puncto plani \(P\), cuius distantia a centro \(C\) est \(r\) angulusque \(G C P=\varphi\), perpendiculum aequale valori expressionis

\[
r^{m} \sin m \varphi+A r^{m-1} \sin (m-1) \varphi+\text { etc. }+L r \sin \varphi
\]

quem brevitatis gratia in sequentibus semper per \(T\) designabo. Distantiam \(r\) semper tamquam positivam considero, et pro punctis, quae axi ab altera parte iacent, angulus \(\varphi\) aut tamquam duobus rectis maior, aut tamquam negativus (quod hic eodem redit) spectari debet. Extremitates horum perpendiculorum (quae pro valore positivo ipsius \(T\) supra planum accipiendae sunt, pro negativo infra, pro evanescente in plano ipso) erunt ad superficiem curvam continuam quaquaversum infinitam, quam brevitatis gratia in sequentibus superficiem primam vocabo. Prorsus simili modo ad idem planum et centrum eundemque axem referatur alia superficies, cuius altitudo supra quodvis plani punctum sit

\[
r^{m} \cos m \varphi+A r^{m-1} \cos (m-1) \varphi+\text { etc. }+L r \cos \varphi+M
\]

quam expressionem brevitatis gratia semper per \(U\) denotabo. Superficiem vero hanc, quae etiam continua et quaquaversum infinita erit, per denominationem
superficiei secundae a priori distinguam. Tunc manifestum est, totum negotium in eo versari, ut demonstretur, ad minimum unum punctum dari, quod simul in plano, in superficie prima et in superficie secunda iaceat.

17.

Facile perspici potest, superficiem primam partim supra planum partim infra planum iacere; patet enim distantiam a centro \(r\) tam magnam accipi posse, ut reliqui termini in \(T\) prae primo \(r^{m} \sin m \varphi\) evanescant; hic vero, angulo \(\varphi\) rite determinato, tam positivus quam negativus fieri potest. Quare planum fixum necessario a superficie prima secabitur; hanc plani cum superficie prima intersectionem vocabo lineam primam; quae itaque determinabitur per aequationem \(T=0\). Ex eadem ratione planum a superficie secunda secabitur; intersectio constituet curvam per aequationem \(U=0\) determinatam, quam lineam secundam appellabo. Proprie utraque curva ex pluribus ramis constabit, qui omnino seiuncti esse possunt, singuli vero erunt lineae continuae. Quin adeo linea prima semper erit talis, quam complexam vocant, axisque \(G C\) tamquam pars huius curvae spectanda; quicunque enim valor ipsi \(r\) tribuatur, \(T\) semper fiet \(=0\), quando \(i\) aut \(=0\) aut \(=180^{\circ}\). Sed praestat complexum cunctorum ramorum per omnia puncta, ubi \(T=0\), transeuntium tamquam unam curvam considerare (secundum usum in geometria sublimiori generaliter receptum), similiterque cunctos ramos per omnia puncta transeuntes, ubi \(U=0\). Patet iam, rem eo reductam esse, ut demonstretur, ad minimum unum punctum in plano dari, ubi ramus aliquis lineae primae a ramo lineae secundae secetur. Ad hunc finem indolem harum linearum propius contemplari oportebit.

18.

Ante omnia observo, utramque curvam esse algebraicam, et quidem, si ad coordinatas orthogonales revocetur, ordinis \(m^{\text {ti }}\). Sumto enim initio abscissarum in \(C\), abscissisque \(x\) versus \(G\), applicatis \(y\) versus \(P\), erit \(x=r \cos \varphi, y=r \sin \varphi\), adeoque generaliter, quidquid sit \(n\),

\[
\begin{aligned}
& r^{n} \sin n \varphi=n x^{n-1} y-\frac{n \cdot n-1 \cdot n-2}{1 \cdot 2 \cdot 3} x^{n-3} y^{3}+\frac{n \cdot \ldots-4}{1 \cdot \ldots 5} x^{n-5} y^{5}-\text { etc. }, \\
& r^{n} \cos n \varphi=x^{n}-\frac{n \cdot n-1}{1 \cdot 2} x^{n-2} y y+\frac{n \cdot n-1 \cdot n-2 \cdot n-3}{1 \cdot 2 \cdot 3 \cdot 4} x^{n-4} y^{4}-\text { etc. }
\end{aligned}
\]

Quamobrem tum \(T\) tum \(U\) constabunt ex pluribus huiusmodi terminis \(a x^{2} y^{6}\),
denotantibus \(\alpha, b\) numeros integros positivos, quorum summa, ubi maxima est, fit \(=m\). Ceterum facile praevideri potest, cunctos terminos ipsius \(T\) factorem \(y\) involvere, adeoque lineam primam proprie ex recta (cuius aequatio \(y=0\) ) et curva ordinis \(m-1^{\text {ti }}\) compositam esse; sed necesse non est ad hanc distinctionem hic respicere.

Maioris momenti erit investigatio, an linea prima et secunda crura infinita habeant, et quot qualiaque. In distantia infinita a puncto \(C\) linea prima, cuius aequatio \(\sin m \varphi+\frac{A}{r} \sin (m-1) \varphi+\frac{B}{r r} \sin (m-2) \varphi\) etc. \(=0\), confundetur cum linea, cuius aequatio \(\sin m \varphi=0\). Haec vero exhibet \(m\) lineas rectas in puncto \(C\) se secantes, quarum prima est axis \(G C G^{\prime}\), reliquae contra hanc sub angulis \(\frac{1}{m} 180, \frac{2}{m} 180, \frac{3}{m} 180\) etc. graduum inclinatae. Quare linea prima \(2 m\) ramos infinitos habet, qui peripheriam circuli radio infinito descripti in \(2 m\) partes aequales dispertiuntur, ita ut peripheria a ramo primo secetur in concursu circuli et axis, a secundo in distantia \(\frac{1}{m} 180^{\circ}\). a tertio in distantia \(\frac{2}{m} 180^{\circ}\) etc. Eodem modo linea secunda in distantia infinita a centro habebit asymptotam per aequationem \(\cos m \varphi=0\) expressam. quae est complexus \(m\) rectarum in puncto \(C\) sub aequalibus angulis itidem se secantium, ita tamen, ut prima cum axe \(C G\) constituat angulum \(\frac{1}{m} 90^{\circ}\), secunda angulum \(\frac{3}{m} 90^{\circ}\), tertia angulum \(\frac{5}{m} 90^{\circ}\) etc. Quare linea secunda etiam \(2 m\) ramos infinitos habebit, quorum singuli medium locum inter binos ramos proximos lineae primae occupabunt, ita ut peripheriam circuli radio infinite magno descripti in punctis, quae \(\frac{1}{m} 90^{\circ}, \frac{3}{m} 90^{\circ}, \frac{5}{m} 90^{\circ}\) etc. ab axe distant, secent. Ceterum palam est, axem ipsum semper duos ramos infinitos lineae primae constituere, puta primum et \(m+1^{\text {tum }}\). Luculentissime hic ramorum situs exhibetur in fig. 2 , pro casu \(m=4\) constructa, ubi rami lineae secundae, ut a ramis lineae primae distinguantur, punctati exprimuntur, quod etiam de figura quarta est tenendum *). - Quum vero hae conclusiones maximi momenti sint, quantitatesque infinite magnae quosdam lectores offendere possint: illas etiam absque infinitorum subsidio in art. sequ. eruere docebo.

19.

\textsc{Theorema}. Manentibus cunctis ut supra, ex centro \(C\) describi poterit circulus,
\footnotetext{*) Figura quarta constructa est supponendo \(X=x^{4}-2 x x+3 x+10\), in qua itaque lectores disquisitionibus generalibus et abstractis minus assueti situm respectivum utriusque curvae in concreto intueri poterunt. Longitudo lineae \(C G\) assumta est \(=10(C N=1,2625\).. \()\)
}
in cuius peripheria sint \(2 m\) puncta, in quibus \(T=0\), totidemque, in quibus \(U=0\), et quidem ita, ut singula posteriora inter bina priorum iaceant.

Sit summa omnium coëfficientium \(A, B\) etc. \(K, L, M\) positive acceptorum \(=S\), accipiaturque \(R\) simul \(>S \sqrt{ } 2\) et \(\left.>1^{*}\right)\) : tum dico in circulo radio \(R\) descripto ea, quae in theoremate enunciata sunt, necessario locum habere. Scilicet designato brevitatis gratia eo puncto huius circumferentiae, quod \(\frac{1}{m} 45\) gradibus ab ipsius concursu cum laeva parte axis distat, sive pro quo \(\varphi=\frac{1}{m} 45^{0}\), per (1); similiter eo puncto, quod \(\frac{3}{m} 45^{\circ}\) ab hoc concursu distat, sive pro quo \(\vartheta=\frac{3}{m} 45^{0}\), per (3); porro eo, ubi \(\varphi=\frac{5}{m} 45^{0}\), per (5) etc. usque ad \((8 m-1)\), quod \(\frac{8 m-1}{m} 45\) gradibus. ab illo concursu distat, si semper versus eandem partem progrederis, (aut \(\frac{1}{m} 45^{\circ}\) a parte opposita), ita ut omnino \(4 m\) puncta in peripheria habeantur, aequalibus intervallis dissita: iacebit inter \((8 m-1)\) et (1) unum punctum, pro quo \(T=0\); nec non sita erunt similia puncta singula inter (3) et (5); inter (7) et (9); inter (11) et (13) etc., quorum itaque multitudo \(2 m\); eodemque modo singula puncta, pro quibus \(U=0\), iacebunt inter (1) et (3); inter (5) et (7); inter (9) et (11), quorum multitudo igitur etiam \(=2 m\); denique praeter haec \(4 m\) puncta alia in tota peripheria non dabuntur, pro quibus vel \(T\) vel \(U\) sit \(=0\).

Demonstr. I. In puncto (1) erit \(m \varphi=45^{\circ}\) adeoque

\[
T=R^{m-1}\left(R \sqrt{\frac{1}{2}}+A \sin (m-1) \varphi+\frac{B}{R} \sin (m-2) \varphi+\text { etc. }+\frac{L}{R^{m-2}} \sin \varphi\right)
\]

summa vero \(A \sin (m-1) \varphi+\frac{B}{R} \sin (m-2) \varphi\) etc. certo non poterit esse maior quam \(S\), adeoque necessario erit minor quam \(R \sqrt{\frac{1}{2}}\) : unde sequitur, in hoc puncto valorem ipsius \(T\) certo esse positivum. A potiori itaque \(T\) valorem positivum habebit, quando \(m \varphi\) inter \(45^{\circ}\) et \(135^{\circ}\) iacet, i. e. a puncto (1) usque ad (3) valor ipsius \(T\) semper positivus erit. Ex eadem ratione \(T\) a puncto (9) usque ad (11) positivum valorem ubique habebit, et generaliter a quovis puncto \((8 k+1)\) usque ad \((8 k+3)\), denotante \(k\) integrum quemcunque. Simili modo \(T\) ubique inter (5) et \((7)\), inter (13) et (15) etc. et generaliter inter \((8 k+5)\) et \((8 k+7)\) valorem negativum habebit, adeoque in omnibus his intervallis nullibi poterit esse \(=0\). Sed quoniam in (3) hic valor est positivus, in (5) negativus: necessario alicubi inter (3) et (5) erit \(=0\); nec non alicubi inter (7) et (9); inter (11) et (13) etc.

*) Quando \(S>\sqrt{\frac{1}{2}}\), conditio prima secundam; quando vero \(s<\sqrt{\frac{1}{2}}\), secunda primam implicabit.
usque ad intervallum inter \((8 m-1)\) et (1) incl., ita ut omnino in \(2 m\) punctis habeatur \(T=0\). Q.E.P.

II. Quod vero praeter haec \(2 m\) puncta, alia, hac proprietate praedita, non dantur. ita cognoscitur. Quum inter (1) et (3); inter (5) et (7) etc. nulla sint, aliter fieri non posset, ut plura talia puncta exstent, quam si in aliquo intervallo inter (3) et (5), vel inter (7) et (9) etc. ad minimum duo iacerent. Tum vero necessario in eodem intervallo \(T\) alicubi esset maximum, vel minimum, adeoque \(\frac{\mathrm{d} T^{\prime}}{\mathrm{d} \varphi}=0 . \quad\) Sed \(\frac{\mathrm{d} T^{\prime}}{\mathrm{d} \varphi}=m \boldsymbol{R}^{m-2}\left(\boldsymbol{R} \cos m \varphi+\frac{m-1}{m} A \cos (m-1) \varphi+\right.\) etc. \()\) et \(\cos m \varphi\) inter (3) et (5) semper est negativus et \(>\sqrt{ } \frac{1}{2}\). Unde facile perspicitur, in toto hoc intervallo \(\frac{\mathrm{d} T}{\mathrm{~d} \varphi}\) esse quantitatem negativam; eodemque modo inter (7) et (9) ubique positivam; inter (11) et (13) negativam etc., ita ut in nullo horum intervallorum esse possit 0, adeoque suppositio consistere nequeat. Quare etc. Q.E.S.

III. Prorsus simili modo demonstratur, \(U\) habere valorem negativum ubique inter (3) et (5), inter (11) et (13) etc. et generaliter inter \((8 k+3)\) et \((8 k+5)\); positivum vero inter \((7)\) et \((9)\), inter (15) et \((17)\) etc. et generaliter inter \((8 k+7)\) et \((8 k+9)\). Hinc statim sequitur, \(U=0\) fieri debere alicubi inter (1) et (3), inter (5) et (7) etc., i. e. in \(2 m\) punctis. In nullo vero horum intervallorum fieri poterit \(\frac{d U}{d \varphi}=0\) (quod facile simili modo ut supra probatur): quamobrem plura quam illa \(2 m\) puncta in circuli peripheria non dabuntur, in quibus fiat \(U=0\). Q. E. T. et \(\mathbf{Q}\).

Ceterum ea theorematis pars, secundum quam plura quam \(2 m\) puncta non dantur, in quibus \(T=0\), neque plura quam \(2 m\), in quibus \(U=0\), etiam inde demonstrari potest, quod per aequationes \(T=0, U=0\) exhibentur curvae \(m^{\text {ti }}\) ordinis, quales a circulo tamquam curva secundi ordinis in pluribus quam \(2 m\) punctis secari non posse, ex geometria sublimiori constat.

20.

Si circulus alius radio maiori quam \(R\) ex eodem centro describitur, eodemque modo dividitur: etiam in hoc inter puncta (3) et (5) iacebit punctum unum, in quo \(\boldsymbol{T}=0\), itemque inter (7) et (9) etc., perspicieturque facile, quo minus radius huius circuli a radio \(R\) differat, eo propius huiusmodi puncta inter (3) et (5) in utriusque circumferentia sita esse debere. Idem etiam locum habebit, si circulus radio aliquantum minori quam \(R\), attamen maiori quam \(S \sqrt{2}\) et 1 , describitur. Ex his nullo negotio intelligitur, circuli radio \(\boldsymbol{R}\) descripti circumferen-
tiam in eo puncto inter (3) et (5), ubi \(T=0\), revera secari ab aliquo ramo lineae primae; idemque valet de reliquis punctis, ubi \(T=0\). Eodem modo patet. circumferentiam circuli huius in omnibus \(2 m\) punctis, ubi \(U=0\), ab aliquo ramo lineae secundae secari. Hae conclusiones etiam sequenti modo exprimi possunt: Descripto circulo debitae magnitudinis e centro \(C\), in hunc intrabunt \(2 m\) rami lineae primae totidemque rami lineae secundae, et quidem ita, ut bini rami proximi lineae primae per aliquem ramum lineae secundae ab invicem separentur. Vid. fig. 2, ubi circulus iam non infinitae sed finitae magnitudinis erit, numerique singulis ramis adscripti cum numeris, per quos in art. praec. et hoc limites certos in peripheria brevitatis caussa designavi, non sunt confundendi.

21.

Iam ex hoc situ relativo ramorum in circulum intrantium tot modis diversis deduci potest, intersectionem alicuius rami lineae primae cum ramo lineae secundae intra circulum necessario dari, ut, quaenam potissimum methodus prae reliquis eligenda sit, propemodum nesciam. Luculentissima videtur esse haec: Designemus (fig. 2) punctum peripheriae circuli, ubi a laeva axis parte (quae ipsa est unus ex \(2 m\) ramis lineae primae) secatur, per 0 ; punctum proximum, ubi ramus lineae secundae intrat, per 1; punctum huic proximum, ubi secundus lineae primae ramus intrat, per 2 , et sic porro usque ad \(4 m-1\), ita ut in quovis puncto numero pari signato ramus lineae secundae in circulum intret, contra ramus lineae secundae in omnibus punctis per numerum imparem expressis. Iam ex geometria sublimiori constat, quamvis curvam algebraicam, (sive singulas cuiusvis curvae algebraicae partes, si forte e pluribus composita sit) aut in se redeuntem aut utrimque in infinitum excurrentem esse, adeoque si ramus aliquis curvae algebraicae in spatium definitum intret, eundem necessario ex hoc spatio rursus alicubi exire debere*). Hinc concluditur facile, quodvis punctum numero pari signa-

*) Satis bene certe demonstratum esse videtur, curvam algebraicam neque alicubi subito abrumpi posse (uti e.g. evenit in curva transscendente, cuius aequatio \(y=\frac{1}{\log x}\) ), neque post spiras infinitas in aliquo puncto se quasi perdere (ut spiralis logarithmica), quantumque scio nemo dubium contra hanc rem movit. Attamen si quis postulat, demonstrationem nullis dubiis obnoxiam alia occasione tradere suscipiam. In casu praesenti vero manifestum est, si aliquis ramus e. g. 2 , ex circulo nullibi exiret (fig. 3), te in circulum inter 0 et 2 intrare, postea circa totum hunc ramum (qui in circuli spatio se perdere deberet) circummeare, et tandem inter 2 et 4 rursus ex circulo egredi posse, ita ut nullibi in tota via in lineam primam incideris. Hoc vero absurdum esse inde patet, quod in puncto, ubi in circulum ingressus es, superficiem
'tum (seu, brevitatis caussa, quodvis punctum par) per ramum lineae primae cum alio puncto pari intra circulum iunctum esse debere, similiterque quodvis punctum numero impari notatum cum alio simili puncto per ramum lineae secundae. Quamquam vero haec binorum punctorum connexio secundum indolem functionis \(\boldsymbol{X}\) perquam diversa esse potest, ita ut in genere determinari nequeat, tamen facile demonstrari potest, quaecunque demum illa sit, semper intersectionem lineae primae cum linea secunda oriri.

22.

Demonstratio huius necessitatis commodissime apagogice repraesentari posse videtur. Scilicet supponamus, iunctionem binorum quorumque punctorum parium, et binorum quorumque punctorum imparium ita adornari posse, ut nulla intersectio rami lineae primae cum ramo lineae secundae inde oriatur. Quoniam axis est pars lineae primae, manifesto punctum 0 cum puncto \(2 m\) iunctum erit. Punctum 1 itaque cum nullo puncto ultra axem sito, i. e. cum nullo puncto per numerum maiorem quam \(2 m\) expresso iunctum esse potest, alioquin enim linea iungens necessario axem secaret. Si itaque 1 cum puncto \(n\) iunctum esse supponitur, erit \(n<2 m\). Ex simili ratione si 2 cum \(n^{\prime}\) iunctum esse statuitur, erit \(n^{\prime}<n\), quia alioquin ramus \(2 \ldots n^{\prime}\) ramum \(1 \ldots n\) necessario secaret. Ex eadem caussa punctum 3 cum aliquo punctorum inter 4 et \(n^{\prime}\) iacentium iunctum erit, patetque si \(3,4,5\) etc. iuncta esse supponantur cum \(n^{\prime \prime}, n^{\prime \prime \prime}, n^{\prime \prime \prime}\) etc., \(n^{\prime \prime \prime}\) iacere inter 5 et \(n^{\prime \prime}, n^{\prime \prime \prime \prime}\) inter 6 et \(n^{\prime \prime \prime}\) etc. Unde perspicuum est, tandem ad aliquod punctum \(h\) perventum iri, quod cum puncto \(h+2\) iunctum sit, et tum ramus. qui in puncto \(h+1\) in circulum intrat, necessario ramum puncta \(h\) et \(h+2\) iungentem secabit. Quia autem alter horum duorum ramorum ad lineam primam, alter ad secundam pertinebit, manifestum iam est, suppositionem esse contradictoriam, adeoque necessario alicubi intersectionem lineae primae cum linea secunda fieri.

primam supra te habuisti, in egressu, infra; quare necessario alicubi in superficiem primam ipsam incidere debuisti, sive in punctum lineae primae. - Ceterum ex hoc ratiocinio principiis geometriae situs innixo, quae haud minus valida sunt, quam principia geometriae magnitudinis, sequitur tantummodo, si in aliquo ramo lineae primae in circulum intres, te alio loco ex circulo rursus egredi posse, semper in linea prima manendo, neque vero, viam tuam esse lineam continuam in eo sensu, quo in geometria sublimiori accipitur. Sed hic sufficit, viam esse lineam continuam in sensu communi, i. e. nullibi interruptam sed ubique cohaerentem.

Si haec cum praecedentibus iunguntur. ex omnibus disquisitionibus explicatis colligetur, theorema, quamvis functionem algebraicam rationalem integram unius indeterminatae in factores reales primi vel secundi gradus resolvi posse, omni rigore demonstratum.

23.

Ceterum haud difficile ex iisdem principiis deduci potest, non solum unam sed ad minimum \(m\) intersectiones lineae primae cum secunda dari, quamquam etiam fieri potest, ut linea prima a pluribus ramis lineae secundae in eodem puncto secetur, in quo casu functio \(\boldsymbol{X}\) plures factores aequales habebit. Attamen quum hic sufficiat, unius intersectionis necessitatem demonstravisse, fusius huic rei brevitatis caussa non immoror. Ex eadem ratione etiam alias harum linearum proprietates hic uberius non persequor, e. g. intersectionem semper fieri sub angulis rectis; aut si plura crura utriusque curvae in eodem puncto conveniant, totidem crura lineae primae affore, quot crura lineae secundae, haecque alternatim posita esse, et sub aequalibus angulis se secare etc.

Denique observo, minime impossibile esse, ut demonstratio praecedens, quam hic principiis geometricis superstruxi, etiam in forma mere analytica exhibeatur: sed eam repraesentationem, quam hic explicavi, minus abstractam evadere credidi, verumque nervum probandi hic multo clarius ob oculos poni, quam a demonstratione analytica exspectari possit.

Coronidis loco adhuc aliam methodum theorema nostrum demonstrandi addigitabo, quae primo aspectu non modo a demonstratione praecedente, sed etiam ab omnibus demonstrationibus reliquis supra enarratis maxime diversa esse videbitur, et quae nihilominus cum D'ALEmBERTiana, si ad essentiam spectas proprie eadem est. Cum qua illam comparare, parallelismumque inter utramque explorare peritis committo, in quorum gratian unice subiuncta est.

24.

Supra planum figurae 4 relative ad axem \(C G\) punctumque fixum \(C\) descriptas suppono superficiem primam et secundam eodem modo ut supra. Accipe punctum quodcunque in aliquo ramo lineae primae situm sive ubi \(T=0\), (e. g. quodlibet punctum \(M\) in axe iacens), et nisi in hoc etiam \(U=0\), progredere ex hoc puncto in linea prima versus eam partem, versus quam magnitudo abso-
luta ipsius \(U\) decrescit. Si forte in puncto \(M\) valor absolutus ipsius \(U\) versus utramque partem decrescit, arbitrarium est, quorsum progrediaris; quid vero faciendum sit, si \(U\) versus utramque partem crescat, statim docebo. Manifestum est itaque, dum semper in linea prima progrediaris, necessario tandem te ad punctum perventurum, ubi \(U=0\), aut ad tale, ubi valor ipsius \(U\) fiat minimum, e.g. punctum \(N\). In priori casu quod quaerebatur, inventum est; in posteriori vero demonstrari potest, in hoc puncto plures ramos lineae primae sese intersecare (et quidem multitudinem parem ramorum), quorum semissis ita comparati sint, ut si in aliquem eorum deflectas (sive huc sive illuc) valor ipsius \(U\) adhucdum decrescere pergat. (Demonstrationem huius theorematis, prolixiorem quam difficiliorem brevitatis gratia supprimere debeo.) In hoc itaque ramo iterum progredi poteris, donec \(U\) aut fiat \(=0\) (uti in fig. 4 evenit in \(P\) ), aut denuo minimum. Tum rursus deflectes, necessarioque tandem ad punctum pervenies, ubi sit \(U=0\).

Contra hanc demonstrationem obiici posset dubium, annon possibile sit, ut quantumvis longe progrediaris, et quamvis valor ipsius \(U\) semper decrescat, tamen haec decrementa continuo tardiora fiant, et nihilominus ille valor limitem aliquem nusquam attingat; quae obiectio responderet quartae in art. 6. Sed haud difficile foret, terminum aliquem assignare, quem simulac transieris, valor ipsius \(U\) necessario non modo semper rapidius mutari debeat, sed etiam decrescere non amplius possit, ita ut antequam ad hunc terminum perveneris, necessario valor 0 iam affuisse debeat. Hoc vero et reliqua, quae in hac demonstratione addigitare tantummodo potui, alia occasione fusius exsequi mihi reservo.

Principia quibus haecce demonstratio innititur deteximus Initio Octob. 1797.

\begin{center}
%\includegraphics[max width=\textwidth]{2024_01_11_75975a03bcf8b0416cd0g-030}
\end{center}

Gauss Whater Rand III Shever \(30^{\circ}\)

\section*{DEMONSTRATIO NOVA ALTERA}
THEOREMATIS

\section*{OMNEM FUNCTIONEM ALGEBRAICAM RATIONALEM INTEGRAM}
UNIUS VARIABILIS

\section*{IN FACTORES REALES PRIMI VEL SECUNDI GRADUS}
RESOLVI POSSE

A U C T OR E

CAROLO FRIDERICO GAUSS

SOCIETATI REGIAE SCIENIIARUM TRADITA 1815. DEC. 7.

Commentationes societatis regiae scientiarum Gottingensis recentiores. Vol. III.

Gottingae mocccxvr.

\section*{OMNEM FUNCTIONEM ALGEBRAICAM RATIONALEM INTEGRAM}
\section*{UNIUS VARIABILIS}
IN FACTORES REALES PRIMI VEL SECUNDI GRADUS RESOIVI POSSE.

1.

Quamquam demonstratio theorematis de resolutione functionum algebraicarum integrarum in factores, quam in commentatione sedecim abhinc annis promulgata tradidi, tum respectu rigoris tum simplicitatis nihil desiderandum relinquere videatur, tamen haud ingratum fore geometris spero, si iterum ad eandem quaestionem gravissimam revertar, atque e principiis prorsus diversis demonstrationem alteram haud minus rigorosam adstruere coner. Pendet scilicet illa demonstratio prior, partim saltem, a considerationibus geometricis: contra ea, quam hic exponere aggredior, principiis mere analyticis innixa erit. Methodorum analyticarum, per quas usque ad illud quidem tempus alii geometrae theorema nostrum demonstrare susceperunt, insigniores loco citato recensui, et quibus vitiis laborent copiose exposui. Quorum gravissimum ac vere radicale omnibus illis conatibus, perinde ac recentioribus, qui quidem mihi innotuerunt, commune: quod tamen neutiquam inevitabile videri in demonstratione analytica, iam tunc declaravi. Esto iam penes peritos iudicium, an fides olim data per has novas curas plene sit liberata.

2.

Disquisitioni principali quaedam praeliminares praemittentur, tum ne quid deesse videatur, tum quod ipsa forsan tractatio iis quoque, quae ab aliis iam de-
libata fuerant, novam qualemcunque lucem affundere poterit. Ac primo quidem de altissimo divisore communi duarum functionum algebraicarum integrarum unius indeterminatae agemus. Ubi praemonendum, hic semper tantum de functionibus integris sermonem esse: e qualibus duabus si productum confletur, utraque huius divisor vocatur. Divisoris ordo ex exponente summae potestatis indeterminatae quam continet diiudicatur, nulla prorsus coëfficientium numericorum ratione habita. Ceterum quae ad divisores communes functionum pertinent, eo brevius absolvere licet, quod iis, quae ad divisores communes numerorum spectant, omnino sunt analoga.

Propositis duabus functionibus \(Y, \boldsymbol{Y}^{\prime}\) indeterminatae \(x\), quarum prior sit ordinis altioris aut saltem non inferioris quam posterior, formabimus aequationes sequentes

\[
\begin{gathered}
Y=q Y^{\prime}+Y^{\prime \prime} \\
Y^{\prime}=q^{\prime} Y^{\prime \prime}+Y^{\prime \prime \prime} \\
Y^{\prime \prime}=q^{\prime \prime} Y^{\prime \prime \prime}+Y^{\prime \prime \prime} \\
\text { etc. usque ad } \\
Y^{(\mu-1)}=q^{(\mu-1)} Y^{(\mu)}
\end{gathered}
\]

ea scilicet lege, ut primo \(Y\) dividatur sueto more per \(Y^{\prime}\); dein \(Y^{\prime}\) per residuum primae divisionis \(Y^{\prime \prime}\), quod erit ordinis inferioris quam \(Y^{\prime}\); tunc rursus residuum primum per secundum \(Y^{\prime \prime \prime}\) et sic porro, donec ad divisionem absque residuo perveniatur, quod tandem necessario evenire debere inde patet, quod ordo functionum \(Y^{\prime}, Y^{\prime \prime}, Y^{\prime \prime \prime}\) etc. continuo decrescit. Quas functiones perinde atque quotientes \(q, q^{\prime}, q^{\prime \prime}\) etc. esse functiones integras ipsius \(x\), vix opus est monere. His praemissis, manifestum est,

I. regrediendo ab ultima istarum aequationum ad primam, functionem \(Y^{(\mu)}\) esse divisorem singularum praecedentium, adeoque certo divisorem communem propositarum \(Y, \boldsymbol{Y}^{\prime}\).

II. Progrediendo a prima aequatione ad ultimam, elucet, quemlibet divisorem communem functionum \(\boldsymbol{Y}, \boldsymbol{Y}^{\prime}\) etiam metiri singulas sequentes, et proin etiam ultimam \(Y^{(\mu)}\). Quamobrem functiones \(Y, Y^{\prime}\) habere nequeunt ullum divisorem communem altioris ordinis quam \(Y^{(\mu)}\), omnisque divisor communis eiusdem ordinis ut \(\boldsymbol{Y}^{(\mu)}\) erit ad hunc in ratione numeri ad numerum, unde hic ipse pro divisore communi summo erit habendus.

III. Si \(Y^{(\mu)}\) est ordinis 0 , i. e. numerus, nulla functio indeterminatae \(x\) proprie sic dicta ipsas \(Y, \boldsymbol{Y}^{\prime}\) metiri potest: in hoc itaque casu dicendum est, has functiones divisorem communem non habere.

IV. Excerpamus ex aequationibus nostris penultimam; dein ex hac eliminemus \(Y^{(\mu-1)}\) adiumento aequationis antepenultimae; tunc iterum eliminemus \(\boldsymbol{Y}^{(\mu-2)}\) adiumento aequationis praecedentis et sic porro: hoc pacto habebimus

\[
\begin{aligned}
Y^{(\mu)} & =+k Y^{(\mu-2)}-k^{\prime} Y^{(\mu-1)} \\
& =-k^{\prime} Y^{(\mu-3)}+k^{\prime \prime} Y^{(\mu-2)} \\
& =+k^{\prime \prime} Y^{(\mu-4)}-k^{\prime \prime \prime} Y^{(\mu-3)} \\
& =-k^{\prime \prime \prime} Y^{(\mu-5)}+k^{\prime \prime \prime} Y^{(\mu-4)} \\
& \text { etc. }
\end{aligned}
\]

si functiones \(k, k^{\prime}, k^{\prime \prime}\) etc. ex lege sequente formatas supponamus

\[
\begin{aligned}
& k=1 \\
& k^{\prime}=q^{(\mu-2)} \\
& k^{\prime \prime}=q^{(\mu-3)} k^{\prime}+k \\
& k^{\prime \prime \prime}=q^{(\mu-4)} k^{\prime \prime}+k^{\prime} \\
& k^{\prime \prime \prime \prime}=q^{(\mu-5)} k^{\prime \prime \prime}+k^{\prime \prime} \\
& \text { etc. }
\end{aligned}
\]

Erit itaque

\[
\pm k^{(\mu-2)} Y \mp k^{(\mu-1)} Y^{\prime}=Y^{(\mu)}
\]

valentibus signis superioribus pro \(\mu\) pari, inferioribus pro impari. In eo itaque casu, ubi \(Y\) et \(Y^{\prime}\) divisorem communem non habent, invenire licet hoc modo duas functiones \(Z, Z^{\prime}\) indeterminatae \(x\) tales, ut habeatur

\[
Z Y+Z Y^{\prime} Y^{\prime}=1
\]

V. Haec propositio manifesto etiam inversa valet, puta, si satisfieri potest aequationi

\[
Z Y+Z^{\prime} Y^{\prime}=1
\]

ita, ut \(Z, Z^{\prime}\) sint functiones integrae indeterminatae \(x\), ipsae \(Y\) et \(Y^{\prime}\) certo divisorem communem habere nequeunt.

\[
5^{*}
\]

3.

Disquisitio praeliminaris altera circa transformationem functionum symmetricarum versabitur. Sint \(a, b, c\) etc. quantitates indeterminatae, ipsarum multitudo \(m\), designemusque per \(\lambda^{\prime}\) illarum summam, per \(\lambda^{\prime \prime}\) summam productorum \(\mathrm{e}\) binis, per \(\lambda^{\prime \prime \prime}\) summam productorum e ternis etc., ita ut ex evolutione producti

\[
(x-a)(x-b)(x-c) \ldots
\]

oriatur

\[
x^{m}-\lambda^{\prime} x^{m-1}+\lambda^{\prime \prime} x^{m-2}-\lambda^{\prime \prime \prime} x^{m-3}+\text { etc. }
\]

Ipsae itaque \(\lambda^{\prime}, \lambda^{\prime \prime}, \lambda^{\prime \prime \prime}\) etc. sunt functiones symmetricae indeterminatarum \(a, b, c\) etc., i. e. tales, in quibus hae indeterminatae eodem modo occurrunt, sive clarius, tales, quae per qualemcunque harum indeterminatarum inter se permutationem non mutantur. Manifesto generalius, quaelibet functio integra ipsarum \(\lambda^{\prime}, \lambda^{\prime \prime}, \lambda^{\prime \prime \prime}\) etc. (sive has solas indeterminatas implicet, sive adhuc alias ab \(a, b, c\) etc. independentes contineat) erit functio symmetrica integra indeterminatarum \(a, b, c\) etc.

4.

Theorema inversum paullo minus obvium. Sit \(\rho\) functio symmetrica indeterminatarum \(a, b, c\) etc., quae igitur composita erit e certo numero terminorum formae

\[
M a^{\alpha} b^{\circ} c^{\gamma} \ldots
\]

denotantibus \(\alpha, b, \gamma\) etc. integros non negativos, atque \(M\) coëfficientem vel determinatum vel saltem ab \(a, b, c\) etc. non pendentem (si forte aliae adhuc indeterminatae praeter \(a, b, c\) etc. functionem \(\rho\) ingrediantur). Ante omnia inter singulos hos terminos ordinem certum stabiliemus, ad quem finem primo ipsas indeterminatas \(a, b, c\) etc. ordine certo per se quidem prorsus arbitrario disponemus, e. g. ita, ut \(a\) primum locum obtineat, \(b\) secundum, \(c\) tertium etc. Dein e duobus terminis

\[
M a^{\alpha} b^{b} c^{\gamma} \ldots \text { et } M a^{\alpha^{\prime}} b^{\varepsilon^{\prime}} c^{i^{\prime}} \ldots
\]

priori ordinem altiorem tribuemus quam posteriori, si fit

vel \(\alpha>\alpha^{\prime}\), vel \(\alpha=\alpha^{\prime}\) et \(b>b^{\prime}\), vel \(\alpha=\alpha^{\prime}, b=b^{\prime}\) et \(\gamma>\gamma^{\prime}\), vel etc.
i. e. si e differentiis \(\alpha-\alpha^{\prime}, b-b^{\prime}, \gamma-\gamma^{\prime}\) etc. prima, quae non evanescit, positiva evadit. Quocirca quum termini eiusdem ordinis non differant nisi respectu coëfficientis \(M\), adeoque in terminum unum conflari possint, singulos terminos functionis \(\rho\) ad ordines diversos pertinere supponemus.

Iam observamus, si \(M a^{\alpha} b^{\ell} c^{\gamma} \ldots\) sit ex omnibus terminis functionis \(\rho\) is, cui ordo altissimus competat, necessario \(\alpha\) esse maiorem, vel saltem non minorem, quam \(b\). Si enim esset \(b>\alpha\), terminus \(M a^{b} b^{\alpha} c^{\gamma} \ldots\), quem functio \(\rho\), utpote symmetrica, quoque involvet, foret ordinis altioris quam \(M a^{\alpha} b^{b} c^{\gamma} \ldots\). contra hyp. Simili modo \(b\) erit maior vel saltem non minor quam \(\gamma\); porro \(\gamma\) non minor quam exponens sequens \(\delta\) etc.: proin singulae differentiae \(a-b\), \(b-\gamma, \gamma-\delta\) etc. erunt integri non negativi.

Secundo perpendamus, si e quotcunque functionibus integris indeterminatarum \(a, b, c\) etc. productum confletur, huius terminum altissimum necessario esse ipsum productum e terminis altissimis illorum factorum. Aeque manifestum est, terminos altissimos functionum \(\lambda^{\prime}, \lambda^{\prime \prime}, \lambda^{\prime \prime \prime}\) etc. resp. esse \(a, a b, a b c\) etc. Hinc colligitur, terminum altissimum e producto

\[
p=M \lambda^{\prime \alpha-b} \lambda^{\prime \prime b-\gamma} \lambda^{\prime \prime \prime \gamma-\hat{o}} \ldots
\]

prodeuntem esse \(M a^{\alpha} b^{\varepsilon} c^{\gamma} \ldots\); quocirca statuendo \(\rho-p=\rho^{\prime}\), terminus altissimus functionis \(\rho^{\prime}\) certo erit ordinis inferioris quam terminus altissimus functionis \(\rho\). Manifesto autem \(p\), et proin etiam \(\rho^{\prime}\), fiunt functiones integrae symmetricae ipsarum \(a, b, c\) etc. Quamobrem \(\rho^{\prime}\) perinde tractata, ut antea \(\rho\), discerpetur in \(p^{\prime}+\rho^{\prime \prime}\), ita ut \(p^{\prime}\) sit productum e potestatibus ipsarum \(\lambda^{\prime}, \lambda^{\prime \prime}, \lambda^{\prime \prime \prime}\) etc. in coëfficientem vel determinatum vel saltem ab \(a, b, c\) etc. non pendentem, \(\rho^{\prime \prime}\) vero functio integra symmetrica ipsarum \(a, b, c\) etc. talis, ut ipsius terminus altissimus pertineat ad ordinem inferiorem, quam terminus altissimus functionis \(p^{\prime}\). Eodem modo continuando, manifesto tandem \(\rho\) ad formam \(p+p^{\prime}+p^{\prime \prime}+p^{\prime \prime \prime}\) etc. redacta, i. e. in functionem integram ipsarum \(\lambda^{\prime}, \lambda^{\prime \prime}, \lambda^{\prime \prime \prime}\) etc. transformata erit.

5.

Theorema in art. praec. demonstratum etiam sequenti modo enunciare possumus: Proposita functione quacunque indeterminatarum \(a, b, c\) etc. integra symmetrica \(\rho\), assignari potest functio integra totidem aliarum indeterminatarum \(l^{\prime}, l^{\prime \prime}, l^{\prime \prime \prime}\) etc. talis, quae per substitutiones \(l^{\prime}=\lambda^{\prime}, l^{\prime \prime}=\lambda^{\prime \prime}, l^{\prime \prime \prime}=\lambda^{\prime \prime \prime}\) etc. transeat
in \(\rho\). Facile insuper ostenditur, hoc unico tantum modo fieri posse. Supponamus enim, e duabus functionibus diversis indeterminatarum \(l^{\prime}, l^{\prime \prime}, l^{\prime \prime \prime}\) etc. puta tum ex \(r\), tum ex \(r^{\prime}\) post substitutiones \(l^{\prime}=\lambda^{\prime}, l^{\prime \prime}=\lambda^{\prime \prime}, l^{\prime \prime \prime}=\lambda^{\prime \prime \prime}\) etc. resultare eandem functionem ipsarum \(a, b, c\) etc. Tunc itaque \(r-r^{\prime}\) erit functio ipsarum \(l^{\prime}, l^{\prime \prime}, l^{\prime \prime \prime}\) etc. per se non evanescens, sed quae identice destruitur post illas substitutiones. Hoc vero absurdum esse, facile perspiciemus, si perpendamus, \(r-r^{\prime}\) necessario compositam esse e certo numero partium formae

\[
M l^{\prime \alpha} l^{\prime \prime} l^{\prime \prime \prime} \gamma \ldots
\]

quarum coëfficientes \(M\) non evanescant, et quae singulae respectu exponentium inter se diversae sint, adeoque terminos altissimos e singulis istis partibus prodeuntes exhiberi per

\[
M a^{\alpha+b+\gamma+\text { etc. }} b^{b+\gamma+\text { etc. }} c^{\gamma+\text { etc. }} . . . .
\]

et proin ad ordines diversos referendos esse, ita ut terminus absolute altissimus nullo modo destrui possit.

Ceterum ipse calculus pro huiusmodi transformationibus pluribus compendiis insigniter abbreviari posset, quibus tamen hoc loco non immoramur, quum ad propositum nostrum sola transformationis possibilitas iam sufficiat.

6.

Consideremus productum ex \(m(m-1)\) factoribus

\[
\begin{aligned}
& (a-b)(a-c)(a-d) \ldots \\
\times & (b-a)(b-c)(b-d) \ldots \\
\times & (c-a)(c-b)(c-d) \ldots \\
\times & (d-a)(d-b)(d-c) \ldots \\
\quad & \text { etc. }
\end{aligned}
\]

quod per \(\pi\) denotabimus, et, quum indeterminatas \(a, b, c\) etc. symmetrice involvat, in formam functionis ipsarum \(\lambda^{\prime}, \lambda^{\prime \prime}, \lambda^{\prime \prime \prime}\) etc. redactum supponemus. Transeat haec functio in \(p\), si loco ipsarum \(\lambda^{\prime}, \lambda^{\prime \prime}, \lambda^{\prime \prime \prime}\) etc. resp. substituuntur \(l^{\prime}, l^{\prime \prime}, l^{\prime \prime \prime}\) etc. His ita factis, ipsam \(p\) vocabimus determinantem functionis

\[
y=x^{m}-l^{\prime} x^{m-1}+l^{\prime \prime} x^{m-2}-l^{\prime \prime \prime} x^{m-3}+\text { etc. }
\]

Ita e. g. pro \(m=2\) habemus

\[
p=-l^{\prime 2}+4 l^{\prime \prime}
\]

Perinde pro \(m=3\) invenitur

\[
p=-l^{\prime 2} l^{\prime \prime 2}+4 l^{\prime 3} l^{\prime \prime \prime}+4 l^{\prime \prime 3}-1.8 l^{\prime} l^{\prime \prime} l^{\prime \prime \prime}+27 l^{\prime \prime 2}
\]

Determinans functionis \(y\) itaque est functio coëfficientium \(l^{\prime}, l^{\prime \prime}, l^{\prime \prime \prime}\) etc. talis, quae per substitutiones \(l^{\prime}=\lambda^{\prime}, l^{\prime \prime}=\lambda^{\prime \prime}, l^{\prime \prime \prime}=\lambda^{\prime \prime \prime}\) etc. transit in productum ex omnibus differentiis inter binas quantitatum \(a, b, c\) etc. In casu eo, ubi \(m=1\), i. e. ubi unica tantum indeterminata \(a\) habetur, adeoque nullae omnino adsunt differentiae, ipsum numerum 1 tamquam determinantem functionis \(y\) adoptare conveniet.

In stabilienda notione determinantis, coëfficientes functionis \(y\) tamquam quantitates indeterminatas spectare oportuit. Determinans functionis cum coëfficientibus determinatis

\[
Y=x^{m}-L^{\prime} x^{m-1}+L^{\prime \prime} x^{m-2}-L^{\prime \prime \prime} x^{m-3}+\text { etc. }
\]

erit numerus determinatus \(P\), puta valor functionis \(p\) pro \(l^{\prime}=L^{\prime}, l^{\prime \prime}=L^{\prime \prime}\), \(l^{\prime \prime \prime}=L^{\prime \prime \prime}\) etc. Quodsi itaque supponimus, \(Y\) resolvi posse in factores simplices

\[
Y=(x-A)(x-B)(x-C) \ldots
\]

sive \(\boldsymbol{Y}\) oriri ex

\[
v=(x-a)(x-b)(x-c)
\]

statuendo \(a=A, b=B, c=C\) etc., adeoque per easdem substitutiones \(\lambda^{\prime}, \lambda^{\prime \prime}, \lambda^{\prime \prime \prime}\) etc. resp. fieri \(L^{\prime}, L^{\prime \prime}, L^{\prime \prime}\) etc., manifesto \(P\) aequalis erit producto e factoribus

\[
\begin{aligned}
& (A-B)(A-C)(A-D) \ldots \\
\times & (B-A)(B-C)(B-D) \cdots \\
\times & (C-A)(C-B)(C-D) \cdots \\
\times & (D-A)(D-B)(D-C) \cdots \\
\quad & \quad \text { etc. }
\end{aligned}
\]

Patet itaque, si fiat \(P=0\), inter quantitates \(A, B, C\) etc. duas saltem aequales reperiri debere; contra, si non fuerit \(\boldsymbol{P}=\mathbf{0}\), cunctas \(\boldsymbol{A}, \boldsymbol{B}, \boldsymbol{C}\) etc. necessario inaequales esse. Iam observamus, si statuamus \(\frac{\mathrm{d} Y}{\mathrm{~d} x}=Y^{\prime}\), sive

\[
Y^{\prime}=m x^{m-1}-(m-1) L^{\prime} x^{m-2}+(m-2) L^{\prime \prime} x^{m-3}-(m-3) L^{\prime \prime} x^{m-4}+\text { etc. }
\]

haberi

\[
\begin{aligned}
Y^{\prime}= & (x-B)(x-C)(x-D) \ldots \\
& +(x-A)(x-C)(x-D) \ldots \\
& +(x-A)(x-B)(x-D) \ldots \\
& +(x-A)(x-B)(x-C) \ldots \\
& + \text { etc. }
\end{aligned}
\]

Si itaque duae quantitatum \(A, B, C\) etc. aequales sunt, e.g. \(A=B, Y^{\prime}\) per \(x-A\) divisibilis erit, sive \(Y\) et \(Y^{\prime}\) implicabunt divisorem communem \(x-A\). Vice versa, si \(Y^{\prime}\) cum \(Y\) ullum divisorem communem habere supponitur, necessario \(Y^{\prime}\) aliquem factorem simplicem ex his \(x-A, x-B, x-C\) etc. implicare debebit, e. g. primum \(x-A\), quod manifesto fieri nequit, nisi \(A\) alicui reliquarum \(B, C, D\) etc. aequalis fuerit. Ex his omnibus itaque colligimus duo Theoremata:

I. Si determinans functions \(Y\) fit \(=0\), certo \(Y\) cum \(Y^{\prime}\) divisorem communem habet, adeoque, si \(Y\) et \(Y^{\prime}\) divisorem communem non habent, determinans functionis \(Y\) nequit esse \(=0\).

II. Si determinans functionis \(Y\) non est \(=0\), certo \(Y\) et \(Y^{\prime}\) divisorem communem habere nequeunt; vel, si \(Y\) et \(Y^{\prime}\) divisorem communem habent, necessario determinans functionis \(Y\) esse debet \(=0\).

7.

At probe notandum est, totam vim huius demonstrationis simplicissimae inniti suppositioni, functionem \(Y\) in factores simplices resolvi posse: quae ipsa suppositio, hocce quidem loco, ubi de demonstratione generali huius resolubilitatis agitur, nihil esset nisi petitio principii. Et tamen a paralogismis huic prorsus similibus non sibi caverunt omnes, qui demonstrationes analyticas theorematis principalis tentaverunt, cuius speciosae illusionis originem iam in ipsa disquisitionis enunciatione animadvertimus, quum omnes in formam tantum radicum aequationum inquisiverint, dum existentiam temere suppositam demonstrare oportuisset. Sed de tali procedendi modo, qui nimis a rigore et claritate abhorret, satis iam in commentatione supra citata dictum est. Quamobrem iam theoremata art. praec.,
quorum altero saltem ad propositum nostrum non possumus carere, solidiori fundamento superstruemus: a secundo, tamquam faciliori initium faciemus.

\section*{8.}
Denotemus per \(\rho\) functionem

\[
\begin{aligned}
& \frac{\pi(x-b)(x-c)(x-d) \cdots}{(a-b)^{2}(a-c)^{2}(a-d)^{2} \cdots} \cdots \\
+ & \frac{\pi(x-a)(x-c)(x-d) \cdots}{(b-a)^{2}(b-c)^{2}(b-d)^{2} \cdots \cdots} \\
+ & \frac{\pi(x-a)(x-b)(x-d) \cdots}{(c-a)^{2}(c-b)^{2}(c-d)^{2} \cdots \cdots} \\
+ & \frac{\pi(x-a)(x-b)(x-c) \ldots \cdots}{(d-a)^{2}(d-b)^{2}(d-c)^{2} \cdots \cdots} \\
+ & \text { etc. }
\end{aligned}
\]

quae, quoniam \(\pi\) per singulos denominatores est divisibilis, fit functio integra indeterminatarum \(x, a, b, c\) etc. Statuamus porro \(\frac{\mathrm{d} v}{\mathrm{~d} x}=v^{\prime}\), ita ut habeatur

\[
\begin{aligned}
v^{\prime}= & (x-b)(x-c)(x-d) \\
& +(x-a)(x-c)(x-d) \\
& +(x-a)(x-b)(x-d) \\
& +(x-a)(x-b)(x-c) \\
& + \text { etc. }
\end{aligned}
\]

Manifesto pro \(x=a\), fit \(\rho v^{\prime}=\pi\), unde concludimus, functionem \(\pi\) - \(\rho v^{\prime}\) indefinite divisibilem esse per \(x-a\), et perinde per \(x-b, x-c\) etc., nec non per productum \(v\). Statuendo itaque

\[
\frac{\pi-p u^{\prime}}{v}=\sigma
\]

erit \(\sigma\) functio integra indeterminatarum \(x, a, b, c\) etc., et quidem, perinde ut \(\rho\), symmetrica ratione indeterminatarum \(a, b, c\) etc. Erui poterunt itaque functiones duae integrae \(r, s\), indeterminatarum \(x, l^{\prime}, l^{\prime \prime}, l^{\prime \prime \prime}\) etc., tales quae per substitutiones \(l^{\prime}=\lambda^{\prime}, l^{\prime \prime}=\lambda^{\prime \prime}, l^{\prime \prime \prime}=\lambda^{\prime \prime \prime}\) etc. 'transeantin \(\rho, \sigma\) resp. Quodsi itaque analogiam sequentes, functionem

\[
m x^{m-1}-(m-1) l^{\prime} x^{m-2}+(m-2) l^{\prime \prime} x^{m-3}-(m-3) l^{\prime \prime \prime} x^{m-4}+\text { etc. }
\]

i. e. quotientem differentialem \(\frac{\mathrm{d} y}{\mathrm{~d} x}\) per \(y^{\prime}\) denotemus, ita ut \(y^{\prime}\) per easdem illas
substitutiones transeat in \(v^{\prime}\), patet, \(p-s y-r y^{\prime}\) per easdem substitutiones transire in \(\pi-\sigma 0-\rho v^{\prime}\), i. e. in 0 , adeoque necessario iam per se identice evanescere debere (art. 5): habemus proin aequationem identicam

\[
p=s y+r y^{\prime}
\]

Hinc si supponamus, ex substitutione \(l^{\prime}=L^{\prime}, l^{\prime \prime}=L^{\prime \prime}, l^{\prime \prime \prime}=L^{\prime \prime \prime}\) etc. prodire \(\boldsymbol{r}=\boldsymbol{R}, \boldsymbol{s}=\boldsymbol{S}\), erit etiam identice

\[
P=S Y+R Y^{\prime}
\]

ubi quum \(\boldsymbol{S}, \boldsymbol{R}\) sint functionis integrae ipsius \(x, \boldsymbol{P}\) vero quantitas determinata seu numerus, sponte patet, \(\boldsymbol{Y}\) et \(\boldsymbol{Y}^{\prime}\) divisorem communem habere non posse, nisi fuerit \(\boldsymbol{P}=\mathbf{0}\). Quod est ipsum theorema posterius art. 6 .

9.

Demonstrationem theorematis prioris ita absolvemus, ut ostendamus, in casu eo, ubi \(\boldsymbol{Y}\) et \(\boldsymbol{Y}^{\prime}\) non habent divisorem communem, certo fieri non posse \(P=0\). Ad hunc finem primo, per praecepta art. 2 erutas supponimus duas functiones integras indeterminatae \(x\), puta \(f x\) et \(\varphi x\), tales, ut habeatur aequatio identica

\[
f x \cdot Y+\varphi x \cdot Y^{\prime}=1
\]

quam hic ita exhibemus:

\[
f x \cdot v+\varphi x \cdot v^{\prime}=1+f x \cdot(u-Y)+\varphi x \cdot \frac{\mathrm{d}(u-Y)}{\mathrm{d} x}
\]

sive, quoniam habemus

\[
\begin{aligned}
v^{\prime} & =(x-b)(x-c)(x-d) \ldots \\
& +(x-a) \cdot \frac{\mathrm{d}[(x-b)(x-c)(x-d) \ldots]}{\mathrm{d} x}
\end{aligned}
\]

in forma sequente:

\[
\begin{aligned}
& \varphi x \cdot(x-b)(x-c)(x-d) \ldots \\
+ & \varphi x \cdot(x-a) \cdot \frac{\mathrm{d}[(x-b)(x-c)(x-d) \ldots . .]}{\mathrm{d} x} \\
+ & f x \cdot(x-a)(x-b)(x-c)(x-d) \ldots=1+f x \cdot(0-Y)+\varphi x \cdot \frac{\mathrm{d}(\nu-Y)}{\mathrm{d} x}
\end{aligned}
\]

Exprimamus brevitatis caussa

\[
f x \cdot(y-Y)+\varphi x \cdot \frac{\mathrm{d}(y-Y)}{\mathrm{d} x}
\]

quae est functio integra indeterminatarum \(x, l^{\prime}, l^{\prime \prime}, l^{\prime \prime \prime}\) etc.

\[
\text { per } F\left(x, l^{\prime}, l^{\prime \prime}, l^{\prime \prime \prime} \text { etc. }\right)
\]

unde erit identice

\[
1+f x \cdot(u-Y)+\varphi x \cdot \frac{\mathrm{d}(0-Y)}{\mathrm{d} x}=1+F\left(x, \lambda^{\prime}, \lambda^{\prime \prime}, \lambda^{\prime \prime \prime} \text { etc. }\right)
\]

Habebimus itaque aequationes identicas [1]

\[
\begin{aligned}
& \varphi a .(a-b)(a-c)(a-d) \ldots=1+F\left(a, \lambda^{\prime}, \lambda^{\prime \prime}, \lambda^{\prime \prime \prime} \text { etc. }\right) \\
& \varphi b .(b-a)(b-c)(b-d) \ldots=1+F\left(b, \lambda^{\prime}, \lambda^{\prime \prime}, \lambda^{\prime \prime \prime} \text { etc. }\right) \\
& \varphi c .(c-a)(c-b)(c-d) \ldots=1+F\left(c, \lambda^{\prime}, \lambda^{\prime \prime}, \lambda^{\prime \prime \prime} \text { etc. }\right)
\end{aligned}
\]

etc.

Supponendo itaque, productum ex omnibus

\[
\begin{aligned}
& 1+\boldsymbol{F}\left(a, l^{\prime}, l^{\prime \prime}, l^{\prime \prime \prime} \text { etc. }\right) \\
& 1+\boldsymbol{F}\left(b, l^{\prime}, l^{\prime \prime}, l^{\prime \prime \prime} \text { etc. }\right) \\
& 1+F\left(c, l^{\prime}, l^{\prime \prime}, l^{\prime \prime \prime} \text { etc. }\right)
\end{aligned}
\]

etc.

quod erit functio integra indeterminatarum \(a, b, c\) etc.. \(l^{\prime}, l^{\prime \prime}, l^{\prime \prime \prime}\) etc. et quidem functio symmetrica respectu ipsarum \(a, b, c\) etc., exhiberi per

\[
\oint\left(\lambda^{\prime}, \lambda^{\prime \prime}, \lambda^{\prime \prime \prime} \text { etc } c_{i}, l^{\prime}, l^{\prime \prime}, l^{\prime \prime \prime} \text { etc. }\right)
\]

e multiplicatione cunctarum aequationum [1] resultabit aequatio identica nova [2]

\[
\pi \varphi a . \varphi b . \varphi c \ldots .=\psi\left(\lambda^{\prime}, \lambda^{\prime \prime}, \lambda^{\prime \prime \prime} \text { etc. }, \lambda^{\prime}, \lambda^{\prime \prime}, \lambda^{\prime \prime \prime} \text { etc. }\right)
\]

Porro patet, quum productum \(\varphi a . \varphi b . \varphi c \ldots\). indeterminatas \(a, b, c\) etc. symmetrice involvat, inveniri posse functionem integram indeterminatarum \(l^{\prime}, l^{\prime \prime}, l^{\prime \prime}\) etc. talem, quae per substitutiones \(l^{\prime}=\lambda^{\prime}, l^{\prime \prime}=\lambda^{\prime \prime}, l^{\prime \prime \prime}=\lambda^{\prime \prime \prime}\) etc. transeat in \(\wp a . \varphi b c . . \ldots\) Sit \(t\) illa functio, eritque etiam identice [3]

\[
p t=\psi\left(l^{\prime}, l^{\prime \prime}, l^{\prime \prime \prime} \text { etc. }, l^{\prime}, l^{\prime \prime}, l^{\prime \prime \prime}\right)
\]

quoniam haec aequatio per substitutiones \(l^{\prime}=\lambda^{\prime}, l^{\prime \prime}=\lambda^{\prime \prime}, l^{\prime \prime \prime}=\lambda^{\prime \prime}\) etc. in identicam \([2]\) transit.

Iam ex ipsa definitione functionis \(F\) sequitur, identice haberi

\[
F\left(x, L^{\prime}, L^{\prime \prime}, L^{\prime \prime \prime} \text { etc. }\right)=0
\]

Hinc etiam identice erit

\[
\begin{aligned}
& 1+F\left(a, L^{\prime}, L^{\prime \prime}, L^{\prime \prime \prime} \text { etc. }\right)=1 \\
& 1+F\left(b, L^{\prime}, L^{\prime \prime}, L^{\prime \prime \prime} \text { etc. }\right)=1 \\
& 1+F\left(c, L^{\prime}, L^{\prime \prime}, L^{\prime \prime \prime} \text { etc. }\right)=1
\end{aligned}
\]

etc.

et proin erit etiam identice

\[
\psi\left(\lambda^{\prime}, \lambda^{\prime \prime}, \lambda^{\prime \prime \prime} \text { etc., } L^{\prime}, L^{\prime \prime}, L^{\prime \prime \prime} \text { etc. }\right)=1
\]

adeoque etiam identice \([4]\)

\[
\oint\left(l^{\prime}, l^{\prime \prime}, l^{\prime \prime \prime} \text { etc., } L^{\prime} . L^{\prime \prime}, L^{\prime \prime \prime} \text { etc. }\right)=1
\]

Quamobrem e combinatione aequationum [3] et [4], et substituendo \(l^{\prime}=L^{\prime}\), \(l^{\prime \prime}=L^{\prime \prime}, l^{\prime \prime \prime}=L^{\prime \prime \prime}\) etc. habebimus \([5]\)

\[
P T=1
\]

si per \(T\) denotamus valorem functionis \(t\) illis substitutionibus respondentem. Qui valor quum necessario fiat quantitas finita, \(P\) certo nequit esse \(=0\). Q. E. D.

10.

E praecedentibus iam perspicuum est, quamlibet functionem integram \(\boldsymbol{Y}\) unius indeterminatae \(x\), cuius determinans sit \(=0\), decomponi posse in factores, quorum nullus habeat determinantem 0 . Investigato enim divisore communi altissimo functionum \(Y\) et \(\frac{\mathrm{d} Y}{\mathrm{~d} x}\), illa iam in duos factores resoluta habebitur. Si quis horum factorum *) iterum habet determinantem 0 , eodem modo in duos factores resolvetur, eodemque pacto continuabimus, donec \(Y\) in factores tales tandem resoluta habeatur, quorum nullus habeat determinantem 0 .

Facile porro perspicietur, inter hos factores, in quos \(Y\) resolvitur, ad mi-
\footnotetext{*) Revera quidem non nisi factor iste, qui est ille divisor communis, determinantem 0 habere potest. Sed demonstratio huius propositionis hocce loco in quasdam ambages perduceret; neque etiam hic necessaria est, quum factorem alterum, si et huius determinans evanescere posset, eodem modo tractare, ipsumque in factores resolvere liceret.
}
nimum unum reperiri debere ita comparatum. ut inter factores numeri, qui eius ordinem exprimit, binarius saltem non pluries occurrat, quam inter factores numeri \(m\), qui exprimit ordinem functionis \(Y:\) puta, si statuatur \(m=k .2^{\mu}\), denotante \(k\) numerum imparem, inter factores functionis \(Y\) ad minimum unus reperietur ad ordinem \(k^{\prime} \cdot 2^{\nu}\) referendus, ita ut etiam \(k^{\prime}\) sit impar, atque vel \(\nu=\mu\), vel \(\nu<\mu . \quad\) Veritas huius assertionis sponte sequitur inde, quod \(m\) est aggregatum numerorum, qui ordinem singulorum factorum ipsius \(\boldsymbol{Y}\) exprimunt.

11.

Antequam ulterius progrediamur, expressionem quandam explicabimus, cuius introductio in omnibus de functionibus symmetricis disquisitionibus maximam utilitatem affert, et quae nobis quoque peropportuna erit. Supponamus, \(M\) esse functionem quarundam ex indeterminatis \(a, b, c\) etc., et quidem sit \(\mu\) multitudo earum, quae in expressionem \(M\) ingrediuntur, nullo respectu habito aliarum indeterminatarum, si quas forte implicet ipsa \(M\). Permutatis illis \(\mu\) indeterminatis omnibus quibus fieri potest modis tum inter se tum cum \(m-\mu\) reliquis ex \(a, b, c\) etc., orientur ex \(M\) aliae expressiones ipsi \(M\) similes, ita ut omnino habeantur

\[
m(m-1)(m-2)(m-3) \ldots .(m-\mu+1)
\]

expressiones, ipsa \(M\) inclusa, quarum complexum simpliciter dicemus complexum omnium. \(M\). Hinc sponte patet, quid significet aggregatum omnium \(M\), productum ex omnibus \(M\) etc. Ita e.g. \(\pi\) dicetur productum ex omnibus \(a-b\), v productum ex omnibus \(x-a\), v' aggregatum omnium \(\frac{u}{x-a}\) etc.

Si forte \(M\) est functio symmetrica respectu quarundam ex \(\mu\) indeterminatis, quas continet, istarum permutationes inter se functionem \(M\) non variant, quamobrem in complexu omnium \(M\) quilibet terminus pluries, et quidem 1.2.3 ....v vicibus reperietur, si \(\nu\) est multitudo indeterminatarum, quarum respectu \(M\) est symmetrica. Si vero \(M\) non solum respectu \(\nu\) indeterminatarum symmetrica est, sed insuper respectu \(v^{\prime}\) aliarum, nec non respectu \(v^{\prime \prime}\) aliarum etc., ipsa \(M\) non variabitur. sive binae e primis \(\nu\) indeterminatis inter se permutentur, sive binae e secundis \(\gamma^{\prime}\), sive binae e tertiis \(\gamma^{\prime \prime}\) etc., ita ut semper

\[
\text { 1.2.3 .... } 1.2 .3 \ldots . v^{\prime} . \quad 1.2 .3 \ldots \ldots v^{\prime \prime} \text { etc. }
\]

permutationes terminis identicis respondeant. Quare si ex his terminis identicis semper unicum tantum retineamus, omnino habebimus

\[
\frac{m(m-1)(m-2)(m-3) \ldots \ldots(m-\mu+1)}{1.2 \cdot 3 \ldots \ldots v_{0} 1.2 .3 \ldots \ldots v^{\prime} \cdot 1.2 \cdot 3 \ldots \ldots v^{\prime \prime} \text { etc. }}
\]

terminos, quorum complexum dicemus complexum omnium \(M\) exclusis repetitionibus, ut a complexu omnium \(M\) admissis repetitionibus distinguatur. Quoties nihil expressis verbis monitum fuerit, repetitiones admitti semper subintelligemus.

Ceterum facile perspicietur, aggregatum omnium \(M\), vel productum ex omnibus \(M\), vel generaliter quamlibet functionem symmetricam omnium \(M\) semper fieri functionem symmetricam indeterminatarum \(a, b, c\) etc., sive admittantur repetitiones, sive excludantur.

12.

Iam considerabimus, denotantibus \(u, x\) indeterminatas, productum ex omnibus \(u-(a+b) x+a b\), exclusis repetitionibus, quod per \(\zeta\) designabimus. Erit itaque \(\zeta\) productum ex \(\frac{1}{2} m(m-1)\) factoribus his

\[
\begin{gathered}
u-(a+b) x+a b \\
u-(a+c) x+a c \\
u-(a+d) x+a d \\
\text { etc. } \\
u-(b+c) x+b c \\
u-(b+d) x+b d \\
\text { etc. } \\
u-(c+d) x+c d \\
\text { etc. etc. }
\end{gathered}
\]

Quae functio quum indeterminatas \(a, b, c\) etc. symmetrice implicet, assignari poterit functio integra indeterminatarum \(u, x, l^{\prime}, l^{\prime \prime}, l^{\prime \prime \prime}\) etc., per \(\boldsymbol{z}\) denotanda, quae transeat in \(\zeta\), si loco indeterminatarum \(l^{\prime}, l^{\prime \prime}, l^{\prime \prime \prime}\) etc. substituantur \(\lambda^{\prime}, \lambda^{\prime \prime}, \lambda^{\prime \prime \prime}\) etc. Denique designemus per \(Z\) functionem solarum indeterminatarum \(u, x\), in quam \(z\) transit, si indeterminatis \(l^{\prime}, l^{\prime \prime}, l^{\prime \prime \prime}\) etc. tribuamus valores determinatos \(L^{\prime}, L^{\prime \prime}, L^{\prime \prime \prime}\) etc.

Hae tres functiones \(\zeta, z, Z\) considerari possunt tamquam functiones integrae ordinis \(\frac{1}{2} m(m-1)\) indeterminatae \(u\) cum coëfficientibus indeterminatis, qui
quidem coëfficientes erunt

\[
\begin{aligned}
& \text { pro } \zeta, \text { functiones indeterminatarum } x, a, b, c \text { etc. } \\
& \text { pro } z, \text { functiones indeterminatarum } x, l^{\prime}, l^{\prime \prime}, l^{\prime \prime \prime} \text { etc. } \\
& \text { pro } Z \text {, functiones solius indeterminatae } x .
\end{aligned}
\]

Singuli vero coëfficientes ipsius \(z\) transibunt in coëfficientes ipsius \(\zeta\) per substitutiones \(l^{\prime}=\lambda^{\prime}, l^{\prime \prime}=\lambda^{\prime \prime}, l^{\prime \prime \prime}=\lambda^{\prime \prime \prime}\) etc. nec non in coëfficientes ipsius \(Z\) per substitutiones \(l^{\prime}=L^{\prime}, l^{\prime \prime}=L^{\prime \prime}, l^{\prime \prime \prime}=L^{\prime \prime \prime}\) etc. Eadem, quae modo de coëfficientibus diximus, etiam de determinantibus functionum \(\zeta, z, Z\) valebunt. Atque in hos ipsos iam propius inquiremus, et quidem eum in finem, ut demonstretur

Theorema. Quoties non est \(P=0\), determinans functionis \(Z\) certo nequit esse identice \(=0\).

\section*{13.}
Perfacilis quidem esset demonstratio huius theorematis. si supponere liceret, \(\boldsymbol{Y}\) resolvi posse in factores simplices

\[
(x-A)(x-B)(x-C)(x-D) \ldots
\]

Tunc enim certum quoque esset, \(\boldsymbol{Z}\) esse productum ex omnibus \(\boldsymbol{u}-(\boldsymbol{A}+\boldsymbol{B}) \boldsymbol{x}+\boldsymbol{A} \boldsymbol{B}\), atque determinantem functionis \(Z\) productum e differentiis inter binas quantitatum

\[
\begin{aligned}
& (A+B) x-A B \\
& (A+C) x-A C \\
& (A+D) x-A D
\end{aligned}
\]

etc.

\[
(B+C) x-B C
\]

\[
(B+D) x-B D
\]

etc.

\[
(C+D) x-C D
\]

etc. etc

Hoc vero productum identice evanescere nequit, nisi aliquis factorum per se identice fiat \(=0\), unde sequeretur, duas quantitatum \(A, B, C\) etc. aequales esse, adeoque determinantem \(P\) functionis \(Y\) fieri \(=0\), contra hyp.

At seposita tali argumentatione, quam ad instar art. 6 a petitione principii proficisci manifestum est, statim ad demonstrationem stabilem theorematis art. 12 explicandam progredimur.

\section*{14.}
Determinans functionis \(\zeta\) erit productum ex omnibus differentiis inter binas \((a+b) x-a b\), quarum differentiarum multitudo est

\[
\frac{1}{2} m(m-1)\left(\frac{1}{2} m(m-1)-1\right)=\frac{1}{4}(m+1) m(m-1)(m-2)
\]

Hic numerus itaque indicat ordinem determinantis functionis \(\zeta\) respectu indeterminatae \(x\). Determinans functionis \(z\) quidem ad eundem ordinem pertinebit: contra determinans functionis \(Z\) utique ad ordinem inferiorem pertinere potest, quoties scilicet quidam coëfficientes inde ab altissima potestate ipsius \(x\) evanescunt. Nostrum iam est demonstrare, in determinante functionis \(Z\) omnes certo coëfficientes evanescere non posse.

Propius considerando differentias illas, quarum productum est determinans functionis \(\zeta\), deprehendemus, partem ex ipsis (puta differentias inter binas \((a+b) x\)-ab tales, quae elementum commune habent) suppeditare

\[
\text { productum ex omnibus }(a-b)(x-c)
\]

e reliquis vero (puta e differentiis inter binas \((a+b) x-a b\) tales, quarum elementa diversa sunt/ oriri

productum ex omnibus \((a+b-c-d) x-a b+c d\), exclusis repetitionibus.

Productum prius factorem unumquemque \(a-b\) manifesto \(m-2\) vicibus continebit, quemvis factorem \(x-c\) autem \((m-1)(m-2)\) vicibus, unde facile concludimus, hocce productum fieri

\[
=\pi^{m-2} v^{(m-1)(m-2)}
\]

Quodsi ita productum posterius per \(\rho\) designamus, determinans functionis \(\zeta\) erit

\[
=\pi^{m-2} v^{(m-1)(m-2)} \rho
\]

Denotando porro per \(r\) functionem indeterminatarum \(x, l^{\prime}, l^{\prime \prime}, l^{\prime \prime \prime}\) etc. eam, quae transit in \(\rho\) per substitutiones \(l^{\prime}=\lambda^{\prime}, l^{\prime \prime}=\lambda^{\prime \prime}, l^{\prime \prime \prime}=\lambda^{\prime \prime \prime}\) etc., nec non per \(\boldsymbol{R}\)
functionem solius \(x\), eam, in quam transit \(r\) per substitutiones \(l^{\prime}=L^{\prime}, l^{\prime \prime}=L^{\prime \prime}\), \(l^{\prime \prime \prime}=L^{\prime \prime \prime}\) etc., patet determinantem functionis \(z\) fieri

\[
=p^{m-2} y^{(m-1)(m-2)} r
\]

determinantem functionis \(Z\) autem

\[
=P^{m-2} Y^{(m-1)(m-2)} R
\]

Quare quum per hypothesin \(\boldsymbol{P}\) non sit \(=0\), res iam in eo vertitur, ut demonstremus, \(\boldsymbol{R}\) certo identice evanescere non posse.

\section*{15.}
Ad hunc finem adhuc aliam indeterminatam \(w\) introducemus, atque productum ex omnibus

\[
(a+b-c-d) w+(a-c)(a-d)
\]

exclusis repetitionibus considerabimus, quod quum ipsas \(a, b, c\) etc. symmetrice involvat, tamquam functio integra indeterminatarum \(w, \lambda^{\prime}, \lambda^{\prime \prime}, \lambda^{\prime \prime \prime}\) etc. exhiberi poterit. Denotabimus hanc functionem per \(f\left(w, \lambda^{\prime}, \lambda^{\prime \prime}, \lambda^{\prime \prime \prime}\right.\) etc.). Multitudo illorum factorum \((a+b-c-d) w+(a-c)(a-d)\) erit

\[
=\frac{1}{2} m(m-1)(m-2)(m-3)
\]

unde facile colligimus, fieri

\[
f\left(0, \lambda^{\prime}, \lambda^{\prime \prime}, \lambda^{\prime \prime \prime} \text { etc. }\right)=\pi^{(m-2)(m-3)}
\]

et proin etiam

\[
f\left(0, l^{\prime}, l^{\prime \prime}, l^{\prime \prime \prime} \text { etc. }\right)=p^{(m-2)(m-3)}
\]

nec non

\[
f\left(0, L^{\prime}, L^{\prime \prime}, L^{\prime \prime \prime} \text { etc. }\right)=P^{(m-2)(m-3)}
\]

Functio \(f\left(w, L^{\prime}, L^{\prime \prime}, L^{\prime \prime}\right.\) etc.) generaliter quidem loquendo ad ordinem

\[
\frac{1}{2} m(m-1)(m-2)(m-3)
\]

referenda erit: at in casibus specialibus utique ad ordinem inferiorem pertinere potest, si forte contingat, ut quidam coëfficientes inde ab altissima potestate ipsius \(w\) evanescant: impossibile autem est, ut illa functio tota sit identice \(=0\), quum
aequatio modo inventa doceat, functionis saltem terminum ultimum non evanescere. Supponemus, terminum altissimum functionis \(f\left(w, L^{\prime}, L^{\prime \prime}, L^{\prime \prime \prime}\right.\) etc.), qui quidem coëfficientem non evanescentem habeat, esse \(\boldsymbol{N} w^{\nu}\). Si igitur substituimus \(w=x-a\), patet, \(f\left(x-a, L^{\prime}, L^{\prime \prime}, L^{\prime \prime \prime}\right.\) etc.) esse functionem integram indeterminatarum \(x, a\), sive quod idem est, functionem ipsius \(x\) cum coëfficientibus ab indeterminata \(a\) pendentibus, ita tamen ut terminus altissimus sit \(N x^{\nu}\), et proin coëfficientem determinatum ab \(a\) non pendentem habeat, qui non sit \(=0\). Perinde \(f\left(x-b, L^{\prime}, L^{\prime \prime}, L^{\prime \prime}\right.\) etc.), \(f\left(x-c, L^{\prime}, L^{\prime \prime}, L^{\prime \prime \prime}\right.\) etc.) erunt functiones integrae indeterminatae \(x\), tales ut singularum terminus altissimus sit \(N x^{\nu}\), terminorum sequentium autem coëfficientes resp. a \(b, c\) etc. pendeant. Hinc productum ex \(m\) factoribus

\[
\begin{gathered}
f\left(x-a, L^{\prime}, L^{\prime \prime}, L^{\prime \prime \prime} \text { etc. }\right) \\
f\left(x-b, L^{\prime}, L^{\prime \prime}, L^{\prime \prime \prime} \text { etc. }\right) \\
f\left(x-c, L^{\prime}, L^{\prime \prime}, L^{\prime \prime \prime} \text { etc. }\right) \\
\text { etc. }
\end{gathered}
\]

erit functio integra ipsius \(x\), cuius terminus altissimus erit \(N^{m} x^{m v}\), dum terminorum sequentium coëfficientes pendent ab indeterminatis \(a, b, c\) etc.

Consideremus iam porro productum ex \(m\) factoribus his

\[
\begin{aligned}
& f\left(x-a, l^{\prime}, l^{\prime \prime}, l^{\prime \prime \prime} \text { etc. }\right) \\
& f\left(x-b, l^{\prime}, l^{\prime \prime}, l^{\prime \prime \prime} \text { etc. }\right) \\
& f\left(x-c, l^{\prime}, l^{\prime \prime}, l^{\prime \prime \prime} \text { etc. }\right)
\end{aligned}
\]

etc.

quod quum sit functio indeterminatarum, \(x, a, b, c\) etc., \(l^{\prime}, l^{\prime \prime}, l^{\prime \prime \prime}\) etc., et quidem symmetrica respectu ipsarum \(a, b, c\) etc., exhiberi poterit tamquam functio indeterminatarum \(x, \lambda^{\prime}, \lambda^{\prime \prime}, \lambda^{\prime \prime \prime}\) etc. \(l^{\prime}, l^{\prime \prime}, l^{\prime \prime \prime}\) etc. per

denotanda. Erit itaque

\[
?\left(x, \lambda^{\prime}, \lambda^{\prime \prime}, \lambda^{\prime \prime \prime} \text { etc., } l^{\prime}, l^{\prime \prime}, l^{\prime \prime \prime} \text { etc. }\right)
\]

\[
\varphi\left(x, \lambda^{\prime}, \lambda^{\prime \prime}, \lambda^{\prime \prime \prime} \text { etc., } \lambda^{\prime}, \lambda^{\prime \prime}, \lambda^{\prime \prime \prime} \text { etc. }\right)
\]

productum ex factoribus

\[
\begin{aligned}
& f\left(x-a, \lambda^{\prime}, \lambda^{\prime \prime}, \lambda^{\prime \prime \prime} \text { etc. }\right) \\
& f\left(x-b, \lambda^{\prime}, \lambda^{\prime \prime}, \lambda^{\prime \prime \prime} \text { etc. }\right) \\
& f\left(x-c, \lambda^{\prime}, \lambda^{\prime \prime}, \lambda^{\prime \prime \prime} \text { etc. }\right)
\end{aligned}
\]

etc.
et proin indefinite divisibilis per \(\rho\), quum facile perspiciatur, quemlibet factorem ipsius \(\rho\) in aliquo illorum factorum implicari. Statuemus itaque

\[
\varphi\left(x, \lambda^{\prime}, \lambda^{\prime \prime}, \lambda^{\prime \prime \prime} \text { etc., } \lambda^{\prime}, \lambda^{\prime \prime}, \lambda^{\prime \prime \prime} \text { etc. }\right)=\rho \psi\left(x, \lambda^{\prime}, \lambda^{\prime \prime}, \lambda^{\prime \prime \prime} \text { etc. }\right)
\]

ubi characteristica \& functionem integram exhibebit. Hinc vero facile deducitur, etiam identice esse

\[
\varphi\left(x, L^{\prime}, L^{\prime \prime}, L^{\prime \prime \prime} \text { etc., } L^{\prime}, L^{\prime \prime}, L^{\prime \prime} \text { etc. }\right)=R \biguplus\left(x, L^{\prime}, L^{\prime \prime}, L^{\prime \prime \prime} \text { etc. }\right)
\]

Sed supra demonstravimus, productum e factoribus

\[
\begin{gathered}
f\left(x-a, L^{\prime}, L^{\prime \prime}, L^{\prime \prime \prime} \text { etc. }\right) \\
f\left(x-b, L^{\prime}, L^{\prime \prime}, L^{\prime \prime \prime} \text { etc. }\right) \\
f\left(x-c, L^{\prime}, L^{\prime \prime}, L^{\prime \prime \prime} \text { etc. }\right) \\
\text { etc. }
\end{gathered}
\]

quod erit \(=\varphi\left(x, \lambda^{\prime}, \lambda^{\prime \prime}, \lambda^{\prime \prime \prime}\right.\) etc., \(L^{\prime}, L^{\prime \prime}, L^{\prime \prime \prime}\) etc. \()\) habere terminum altissimum \(N^{m} x^{m v}\); eundem proin terminum altissimum habebit functio \(\wp\left(x, L^{\prime}, L^{\prime \prime}, L^{\prime \prime \prime}\right.\) etc., \(L^{\prime}, L^{\prime \prime}, L^{\prime \prime \prime}\) etc.) adeoque certo non est identice \(=0\). Quocirca etiam \(R\) nequit esse identice \(=0\), neque adeo etiam determinans functionis \(Z\). Q. E. D.

\section*{16.}
Theorema. Denotet \(\left.\varphi(u, x)^{*}\right)\) productum ex quotcunque factoribus talibus, in quos indeterminatae \(u, x\) lineariter tantum ingrediuntur, sive qui sint formae

\[
\begin{gathered}
\alpha+b u+\gamma x \\
\alpha^{\prime}+b^{\prime} u+\gamma^{\prime} x \\
\alpha^{\prime \prime}+b^{\prime \prime} u+\gamma^{\prime \prime} x \\
\quad \text { etc. }
\end{gathered}
\]

sit porro \(w\) alia indeterminata. Tunc functio

\[
\varphi\left(u+w \cdot \frac{\mathrm{d} \varphi(u, x)}{\mathrm{d} x}, x-w \cdot \frac{\mathrm{d} \varphi(u, x)}{\mathrm{d} u}\right)=Q
\]

indefinite erit divisibilis per \(\varphi(u, x)\).
\footnotetext{*) Vel nobis non monentibus quisque videbit, signa in art. praec. introducta restringi ad istum solum articulum, et proin significationem characterum \(\varphi, w\) praesentem non esse confundendam cum pristina.
}

Dem. Statuendo

\[
\begin{aligned}
\varphi(u, x)= & (\alpha+b u+\gamma x) Q \\
= & \left(\alpha^{\prime}+b^{\prime} u+\gamma^{\prime} x\right) Q^{\prime} \\
= & \left(\alpha^{\prime \prime}+b^{\prime \prime} u+\gamma^{\prime \prime} x\right) Q^{\prime \prime} \\
& \text { etc. }
\end{aligned}
\]

erunt \(Q, Q^{\prime}, Q^{\prime \prime}\) etc. functiones integrae indeterminatarum \(u, x, \alpha, b, \gamma, \alpha^{\prime}, b^{\prime}, \gamma^{\prime}\), \(\alpha^{\prime \prime}, b^{\prime \prime}, \gamma^{\prime \prime}\) etc. atque

\[
\begin{aligned}
\frac{\mathrm{d} \varphi(u, x)}{\mathrm{d} x} & =\gamma Q+(\alpha+b u+\gamma x) \cdot \frac{\mathrm{d} Q}{\mathrm{~d} x} \\
& =\gamma^{\prime} Q^{\prime}+\left(\alpha^{\prime}+b^{\prime} u+\gamma^{\prime} x\right) \cdot \frac{\mathrm{d} Q^{\prime}}{\mathrm{d} x} \\
& =\gamma^{\prime \prime} Q^{\prime \prime}+\left(\alpha^{\prime \prime}+b^{\prime \prime} u+\gamma^{\prime \prime} x\right) \cdot \frac{\mathrm{d} Q^{\prime \prime}}{\mathrm{d} x} \\
& \text { etc. } \\
\frac{\mathrm{d} \varphi(u, x)}{\mathrm{d} u} & =b Q+(\alpha+b u+\gamma x) \cdot \frac{\mathrm{d} Q}{\mathrm{~d} u} \\
& =b^{\prime} Q^{\prime}+\left(\alpha^{\prime}+b^{\prime} u+\gamma^{\prime} x\right) \cdot \frac{\mathrm{d} Q^{\prime}}{\mathrm{d} u} \\
& =b^{\prime \prime} Q^{\prime \prime}+\left(\alpha^{\prime \prime}+b^{\prime \prime} u+\gamma^{\prime \prime} x\right) \cdot \frac{\mathrm{d} Q^{\prime \prime}}{\mathrm{d} u} \\
& \text { etc. }
\end{aligned}
\]

Substitutis hisce valoribus in factoribus, e quibus conflatur productum \(Q\), puta in

\[
\begin{aligned}
& \alpha+b u+\gamma x+b w \cdot \frac{\mathrm{d} \varphi(u, x)}{\mathrm{d} x}-\gamma w \cdot \frac{\mathrm{d} \varphi(u, x)}{\mathrm{d} u} \\
& \alpha^{\prime}+b^{\prime} u+\gamma^{\prime} x+b^{\prime} w \cdot \frac{\mathrm{d} \varphi(u, x)}{\mathrm{d} x}-\gamma^{\prime} w \cdot \frac{\mathrm{d} \varphi(u, x)}{\mathrm{d} u} \\
& \alpha^{\prime \prime}+b^{\prime \prime} u+\gamma^{\prime \prime} x+b^{\prime \prime} w \cdot \frac{\mathrm{d} \varphi(u, x)}{\mathrm{d} x}-\gamma^{\prime \prime} w \cdot \frac{\mathrm{d} \varphi(u, x)}{\mathrm{d} u}
\end{aligned}
\]

etc. resp.

hi obtinent valores sequentes

\[
\begin{gathered}
(\alpha+b u+\gamma x)\left(1+b w \cdot \frac{\mathrm{d} Q}{\mathrm{~d} x}-\gamma w \cdot \frac{\mathrm{d} Q}{\mathrm{~d} u}\right) \\
\left(\alpha^{\prime}+b^{\prime} u+\gamma^{\prime} x\right)\left(1+b^{\prime} w \cdot \frac{\mathrm{d} Q^{\prime}}{\mathrm{d} x}-\gamma^{\prime} w \cdot \frac{\mathrm{d} Q^{\prime}}{\mathrm{d} u}\right) \\
\left(\alpha^{\prime \prime}+b^{\prime \prime} u+\gamma^{\prime \prime} x\right)\left(1+b^{\prime \prime} w \cdot \frac{\mathrm{d} Q^{\prime \prime}}{\mathrm{d} x}-\gamma^{\prime \prime} w \cdot \frac{\mathrm{d} Q^{\prime \prime}}{\mathrm{d} u}\right) \\
\text { etc. }
\end{gathered}
\]

quapropter \(\Omega\) erit productum ex \(\varphi(u, x)\) in factores

\[
\begin{aligned}
& 1+b w \cdot \frac{\mathrm{d} Q}{\mathrm{~d} x}-\gamma w \cdot \frac{\mathrm{d} Q}{\mathrm{~d} u} \\
& 1+b^{\prime} w \cdot \frac{\mathrm{d} Q^{\prime}}{\mathrm{d} x}-\gamma^{\prime} w \cdot \frac{\mathrm{d} Q^{\prime}}{\mathrm{d} u} \\
& 1+b^{\prime \prime} w \cdot \frac{\mathrm{d} Q^{\prime \prime}}{\mathrm{d} x}-\gamma^{\prime \prime} w \cdot \frac{\mathrm{d} Q^{\prime \prime}}{\mathrm{d} u}
\end{aligned}
\]

etc. i. e. ex \(\varphi(u, x)\) in functionem integram indeterminatarum \(u, x, w, \alpha, \boldsymbol{b}, \gamma\), \(\alpha^{\prime}, b^{\prime}, \gamma^{\prime}, \alpha^{\prime \prime}, b^{\prime \prime}, \gamma^{\prime \prime}\) etc. Q. E. D.

\section*{17.}
Theorema art. praec. manifesto applicabile est ad functionem \(\zeta\), quam abhinc per

\[
f\left(u, x, \lambda^{\prime}, \lambda^{\prime \prime}, \lambda^{\prime \prime \prime} \text { etc. }\right)
\]

exhiberi supponemus, ita ut

\[
f\left(u+w \cdot \frac{\mathrm{d} \zeta}{\mathrm{d} x}, \quad x-w \cdot \frac{\mathrm{d} \zeta}{\mathrm{d} u}, \lambda^{\prime}, \lambda^{\prime \prime}, \lambda^{\prime \prime \prime} \text { etc. }\right)
\]

indefinite divisibilis evadat per \(\zeta:\) quotientem, qui erit functio integra indeterminatarum \(u, x, w, a, b, c\) etc., symmetrica respectu ipsarum \(a, b, c\) etc., exhibebimus per

\[
\Psi\left(u, x, w, \lambda^{\prime}, \lambda^{\prime \prime}, \lambda^{\prime \prime \prime} \text { etc. }\right)
\]

Hinc concludimus, fieri etiam identice

\[
f\left(u+w \cdot \frac{\mathrm{d} z}{\mathrm{~d} x}, x-w \cdot \frac{\mathrm{d} z}{\mathrm{~d} u}, l^{\prime}, l^{\prime \prime}, l^{\prime \prime} \text { etc. }\right)=z \psi\left(u, x, w, l^{\prime}, l^{\prime \prime}, l^{\prime \prime \prime} \text { etc. }\right)
\]

nec non

\[
f\left(u+w \cdot \frac{\mathrm{d} Z}{\mathrm{~d} x}, x-w \cdot \frac{\mathrm{d} Z}{\mathrm{~d} u}, L^{\prime}, L^{\prime \prime}, L^{\prime \prime} \text { etc. }\right)=Z \psi\left(u, x, w, L^{\prime}, L^{\prime \prime}, L^{\prime \prime \prime} \text { etc. }\right)
\]

Quodsi itaque functionem \(Z\) simpliciter exhibemus per \(\boldsymbol{F}(\boldsymbol{u}, x)\), ita ut habeatur

\[
f\left(u, x, L^{\prime}, L^{\prime \prime}, L^{\prime \prime \prime} \text { etc. }\right)=F(u, x)
\]

erit identice

\[
\boldsymbol{F}\left(u+w \cdot \frac{\mathrm{d} Z}{\mathrm{~d} x}, x-w \cdot \frac{\mathrm{d} Z}{\mathrm{~d} u}\right)=Z \psi\left(u, x, w, L^{\prime}, L^{\prime \prime}, L^{\prime \prime \prime} \text { etc. }\right)
\]

18.

Si itaque e valoribus determinatis ipsarum \(u, x\), puta ex \(u=U, x=\mathbf{X}\), prodire supponimus

\[
\frac{\mathrm{d} Z}{\mathrm{~d} x}=X^{\prime}, \quad \frac{\mathrm{d} Z}{\mathrm{~d} u}=U^{\prime}
\]

erit identice

\[
\boldsymbol{F}\left(U+w \boldsymbol{X}^{\prime}, \boldsymbol{X}-w U^{\prime}\right)=\boldsymbol{F}(U, \boldsymbol{X}) \cdot \psi\left(U, X, w, L^{\prime}, L^{\prime \prime}, L^{\prime \prime \prime} \text { etc. }\right)
\]

Quoties \(U^{\prime}\) non evanescit, statuere licebit

\[
w=\frac{X-x}{U^{\prime}}
\]

unde emergit

\[
F\left(U+\frac{X X^{\prime}}{U^{\prime}}-\frac{X^{\prime} x}{U^{\prime}}, x\right)=F(U, X) \cdot \psi\left(U, X, \frac{X-x}{U^{\prime}}, L^{\prime}, L^{\prime \prime}, L^{\prime \prime \prime} \text { etc. }\right)
\]

quod etiam ita enunciare licet:

Si in functione \(Z\) statuitur \(u=U+\frac{X X^{\prime}}{U^{\prime}}-\frac{X^{\prime} x}{U^{\prime}}\), transibit ea in

\[
\boldsymbol{F}(U, \boldsymbol{X}) \cdot \psi\left(U, \boldsymbol{X}, \frac{X-x}{U^{\prime}}, L^{\prime}, L^{\prime \prime}, L^{\prime \prime \prime} \text { etc. }\right)
\]

19.

Quum in casu eo, ubi non est \(\boldsymbol{P}=\mathbf{0}\), determinans functionis \(\boldsymbol{Z}\) sit functio indeterminatae \(x\) per se non evanescens , manifesto multitudo valorum determinatorum ipsius \(x\), per quos hic determinans valorem 0 nancisci potest, erit numerus finitus, ita ut infinite multi valores determinati ipsius \(x\) assignari possint, qui determinanti illi valorem a 0 diversum concilient. Sit \(\boldsymbol{X}\) talis valor ipsius \(x\) (quem insuper realem supponere licet). Erit itaque determinans functionis \(\boldsymbol{F}(\boldsymbol{u}, \boldsymbol{X})\) non \(=0\), unde sequitur, per theorema II. art. 6 , functiones

\[
\boldsymbol{F}(\boldsymbol{u}, \boldsymbol{X}) \text { et } \frac{\mathrm{d} \boldsymbol{F}(u, X)}{\mathrm{d} u}
\]

habere non posse divisorem ullum communem. Supponamus porro, exstare aliquem valorem determinatum ipsius \(u\), puta \(U\) (sive realis sit, sive imaginarius i. e. sub forma \(g+h \sqrt{ }-1\) contentus), qui reddat \(F(u, \boldsymbol{X})=0\), i. e. esse \(\boldsymbol{F}(\boldsymbol{U}, \boldsymbol{X})=0\). Erit itaque \(u-U\) factor indefinitus functionis \(\boldsymbol{F}(u, \boldsymbol{X})\), et proin functio \(\frac{\mathrm{d} F(u, X)}{\mathrm{d} u}\) certo per \(u-U\) non divisibilis. Supponendo itaque, hanc functionem
\(\frac{\mathrm{d} F(u, X)}{\mathrm{d} u}\) nancisci valorem \(U^{\prime}\), si statuatur \(u=U\), certo esse nequit \(U^{\prime}=0\). Mànifesto autem \(U^{\prime}\) erit valor quotientis differentialis partialis \(\frac{\mathrm{d} Z}{\mathrm{~d} u}\) pro \(u=U, x=\boldsymbol{X}\) : quodsi itaque insuper pro iisdem valoribus ipsarum \(u, x\) valorem quotientis differentialis partialis \(\frac{\mathrm{d} Z}{\mathrm{~d} x}\) per \(X^{\prime}\) denotemus, perspicuum est per ea quae in art. praec. demonstrata sunt, functionem \(Z\) per substitutionem

\[
u=U+\frac{X X^{\prime}}{U^{\prime}}-\frac{X^{\prime} x}{U^{\prime}}
\]

identice evanescere, adeoque per factorem

\[
u+\frac{X^{\prime}}{\bar{U}^{\prime}} x-\left(U+\frac{X X^{\prime}}{U^{\prime}}\right)
\]

indefinite esse divisibilem. Quocirca statuendo \(u=x x\), patet, \(\boldsymbol{F}(x x, x)\) divisibilem esse per

\[
x x+\frac{X^{\prime}}{U^{\prime}} x-\left(U+\frac{x X^{\prime}}{U^{\prime}}\right)
\]

adeoque obtinere valorem 0 , si pro \(x\) accipiatur radix aequationis

\[
x x+\frac{X^{\prime}}{U^{\prime}} x-\left(U+\frac{X X^{\prime}}{U^{\prime}}\right)=0
\]

i. e. si statuatur

\[
x=\frac{-X^{\prime} \pm \sqrt{ }\left(4 U U^{\prime} U^{\prime}+4 X X^{\prime} U^{\prime}+X^{\prime} X^{\prime}\right)}{2 U^{\prime}}
\]

quos valores vel reales esse vel sub forma \(g+h \sqrt{ }-1\) contentos constat.

Facile iam demonstratur, per eosdem valores ipsius \(x\) etiam functionem \(\boldsymbol{Y}\) evanescere debere. Manifesto enim \(f\left(x x, x, \lambda^{\prime}, \lambda^{\prime \prime}, \lambda^{\prime \prime \prime}\right.\) etc.) est productum ex omnibus \((x-a)(x-b)\) exclusis repetitionibus, et proin \(=v^{m-1}\). Hinc sponte sequitur

\[
\begin{aligned}
& f\left(x x, x, l^{\prime}, l^{\prime \prime}, l^{\prime \prime \prime} \text { etc. }\right)=y^{m-1} \\
& f\left(x x, x, L^{\prime}, L^{\prime \prime}, L^{\prime \prime \prime} \text { etc. }\right)=Y^{m-1}
\end{aligned}
\]

sive \(\boldsymbol{F}(\boldsymbol{x} x, x)=Y^{m-1}\), cuius itaque valor determinatus evanescere nequit, nisi simul evanescat valor ipsius \(Y\).

20.

Adiumento disquisitionum praecedentium reducta est solutio aequationis \(Y=0\), i. e. inventio valoris determinati ipsius \(x\), vel realis vel sub forma \(g+h \bigvee-1\) contenti, qui illi satisfaciat, ad solutionem aequationis \(\boldsymbol{F}(\boldsymbol{u}, \mathbf{X})=\mathbf{0}\),
siquidem determinans functionis \(Y\) non fuerit \(=0\). Observare convenit, si omnes coëfficientes in \(Y\), i. e. numeri \(L^{\prime}, L^{\prime \prime}, L^{\prime \prime \prime}\) etc. sint quantitates reales, etiam omnes coëfficientes in \(F(u, X)\) reales fieri, siquidem, quod licet, pro \(X\) quantitas realis accepta fuerit. Ordo aequationis secundariae \(F(u, X)=0\) exprimitur per numerum \(\frac{1}{2} m(m-1)\) : quoties igitur \(m\) est numerus par formae \(2^{\mu} k\) denotante \(k\) indefinite numerum imparem, ordo aequationis secundariae exprimitur per numerum formae \(2^{\mu-1} k\).

In casu eo, ubi determinans functionis \(Y\) fit \(=0\), assignari poterit per art. 10 functio alia \(\mathfrak{Y}\) ipsam metiens, cuius determinans non sit \(=0\), et cuius ordo exprimatur per numerum formae \(2^{\nu} k\), ita ut sit vel \(\nu<\mu\), vel \(\nu=\mu\). Quaelibet solutio aequationis \(\mathfrak{Y}=0\) etiam satisfaciet aequationi \(Y=0\) : solutio aequationis \(\mathfrak{Y}=0\) iterum reducetur ad solutionem alius aequationis, cuius ordo exprimetur per numerum formae \(2^{v-1} k\).

Ex his itaque colligimus, generaliter solutionem cuiusvis aequationis, cuius ordo exprimatur per numerum parem formae \(2^{\mu} k\), reduci posse ad solutionem alius aequationis, cuius ordo exprimatur per numerum formae \(2^{\mu^{\prime}} k\), ita ut sit \(\mu^{\prime}<\mu\). Quoties hic numerus etiamnum par est, i. e. \(\mu^{\prime}\) non \(=0\), eadem methodus denuo applicabitur, atque ita continuabimus, donec ad aequationem perveniamus, cuius ordo exprimatur per numerum imparem; et huius aequationis coëfficientes omnes erunt reales, siquidem omnes coëfficientes aequationis primitivae reales fuerunt. Talem vero aequationem ordinis imparis certo solubilem esse constat, et quidem per radicem realem, unde singulae quoque aequationes antecedentes solubiles erunt, sive per radices reales sive per radices formae \(g+h \sqrt{ }-1\).

Evictum estitaque, functionem quamlibet \(Y\) formae \(x^{m}-L^{\prime} x^{m-1}+L^{\prime \prime} x^{m-2}-\) etc., ubi \(L^{\prime}, L^{\prime \prime}\) etc. sunt quantitates determinatae reales, involvere factorem indefinitum \(x-A\), ubi \(A\) sit quantitas vel realis vel sub forma \(g+h \bigvee-1\) contenta. In casu posteriori facile perspicitur, \(Y\) nancisci valorem 0 etiam per substitutionem \(x=g-h \sqrt{ }-1\), adeoque etiam divisibilem esse per \(x-(\dot{g}-h \sqrt{ }-1)\), et proin etiam per productum \(x x-2 g x+g g+h h\). Quaelibet itaque functio \(Y\) certo factorem indefinitum realem primi vel secundi ordinis implicat, et quum idem iterum de quotiente valeat, manifestum est, \(\boldsymbol{Y}\) in factores reales primi vel secundi ordinis resolvi posse. Quod demonstrare erat propositum huius commentationis.

\section*{THEOREMATIS }
\section*{DE RESOLUBILITATE FUNCTIONUM ALGEBRAICARUM INTEGRARUM}


\section*{DEMONSTRATIO TERTIA}
\section*{SUPPLEMENTUM COMMENTATIONIS PRAECEDENTIS}
A U C TOR E

CAROLO FRIDERICO GAUSS

SOCIETATI REGIAE SCIEN'IARUM TRADITUM 1816. JAN. 30.

Commentationes societatis regiae scientiarum Gottingensis recentiores. Vol. III. Gottingae MDCCCXVI.

Postquam commentatio praecedens typis iam expressa esset, iteratae de eodem argumento meditationes ad novam theorematis demonstrationem perduxerunt, quae perinde quidem ac praecedens pure analytica est, sed principiis prorsus diversis innititur, et respectu simplicitatis illi longissime praeferenda videtur. Huic itaque tertiae demonstrationi pagellae sequentes dicatae sunto.

1.

Proposita sit functio indeterminatae \(x\) haecce:

\[
\mathbf{X}=x^{m}+A x^{m-1}+B x^{m-2}+C x^{m-3}+\text { etc. }+L x+M
\]

in qua coëfficientes \(A, B, C\) etc. sunt quantitates reales determinatae. Sint \(r, \varphi\) aliae indeterminatae, statuamusque

\[
\begin{gathered}
r^{m} \cos m \varphi+A r^{m-1} \cos (m-1) \varphi+B r^{m-2} \cos (m-2) \varphi \\
+C r^{m-3} \cos (m-3) \varphi+\text { etc. }+L r \cos \varphi+M=t \\
r^{m} \sin m \varphi+A r^{m-1} \sin (m-1) \varphi+B r^{m-2} \sin (m-2) \varphi \\
+C r^{m-3} \sin (m-3) \varphi+\text { etc. }+L r \sin \varphi=u \\
8^{*}
\end{gathered}
\]

\[
\begin{gathered}
m r^{m} \cos m \varphi+(m-1) A r^{m-1} \cos (m-1) \varphi+(m-2) B r^{m-2} \cos (m-2) \varphi \\
+(m-3) C r^{m-3} \cos (m-3) \varphi+\text { etc. }+L r \cos \varphi=t^{\prime} \\
m r^{m} \sin m \varphi+(m-1) A r^{m-1} \sin (m-1) \varphi+(m-2) B r^{m-2} \sin (m-2) \varphi \\
+(m-3) C r^{m-3} \sin (m-3) \varphi+\text { etc. }+L r \sin \varphi=u^{\prime} \\
m m r^{m} \cos m \varphi+(m-1)^{2} A r^{m-1} \cos (m-1) \varphi+(m-2)^{2} B r^{m-2} \cos (m-2) \varphi \\
+(m-3)^{2} C r^{m-3} \cos (m-3) \varphi+\text { etc. }+L r \cos \varphi=t^{\prime \prime} \\
m m r^{m} \sin m \varphi+(m-1)^{2} A r^{m-1} \sin (m-1) \varphi+(m-2)^{2} B r^{m-2} \sin (m-2) \varphi \\
+(m-3)^{2} C r^{m-3} \sin (m-3) \varphi+\text { etc. }+L r \sin \varphi=u^{\prime \prime} \\
\frac{(t t+u u)\left(t t^{\prime \prime}+u u^{\prime \prime}\right)+\left(t u^{\prime}-u t^{\prime}\right)^{2}-\left(t t^{\prime}+u u^{\prime}\right)^{2}}{r(t t+u u)^{2}}=y
\end{gathered}
\]

Factorem \(r\) manifesto e denominatore formulae ultimae tollere licet, quum \(t^{\prime}, u^{\prime}, t^{\prime \prime}, u^{\prime \prime}\) per illum sint divisibiles. Denique sit \(R\) quantitas positiva determinata, arbitraria quidem, attamen maior maxima quantitatum

\[
m A \sqrt{ } 2, \sqrt{ }(m B \sqrt{ } 2), \sqrt[3]{ }(m C \sqrt{ } 2), \sqrt[4]{ }(m D \sqrt{ } 2) \text { etc. }
\]

abstrahendo a signis quantitatum \(A, B, C\) etc., i. e. mutatis negativis, si quae adsint, in positivas. His ita praeparatis, dico, \(t t^{\prime}+u u^{\prime}\) certo nancisci valorem positivum, si statuatur \(r=R\), quicunque valor (realis) ipsi \(\varphi\) tribuatur.

\section*{Demonstratio. Statuamus}
\[
R^{m} \cos 45^{0}+A R^{m-1} \cos \left(45^{0}+\varphi\right)+B R^{m-2} \cos \left(45^{0}+2 \varphi\right)
\]

\(+C R^{m-3} \cos \left(45^{\circ}+3 \varphi\right)+\) etc. \(+L R \cos \left(45^{\circ}+(m-1) \varphi\right)+M \cos \left(45^{\circ}+m \varphi\right)=T\) \(R^{m} \sin 45^{0}+A R^{m-1} \sin \left(45^{0}+\varphi\right)+B R^{m-2} \sin \left(45^{0}+2 \varphi\right)\) \(+C R^{m-3} \sin \left(45^{\circ}+3 \varphi\right)+\) etc. \(+L R \sin \left(45^{\circ}+(m-1) \varphi\right)+M \sin \left(45^{\circ}+m \varphi\right)=U\) \(m R^{m} \cos 45^{0}+(m-1) A R^{m-1} \cos \left(45^{\circ}+\varphi\right)+(m-2) B R^{m-2} \cos \left(45^{\circ}+2 \varphi\right)\)

\[
+(m-3) C R^{m-3} \cos \left(45^{\circ}+3 \varphi\right)+\text { etc. }+L R \cos \left(45^{\circ}+(m-1) \varphi\right)=T^{\prime}
\]

\(m R^{m} \sin 45^{\circ}+(m-1) A R^{m-1} \sin \left(45^{\circ}+\varphi\right)+(m-2) B R^{m-2} \sin \left(45^{\circ}+2 \varphi\right)\)

\[
+(m-3) C R^{m-3} \sin \left(45^{\circ}+3 \varphi\right)+\text { etc. }+L R \sin \left(45^{\circ}+(m-1) \varphi\right)=U^{\prime}
\]

patetque

I. T compositam esse e partibus

\[
\begin{aligned}
& \frac{R^{m-1}}{m \sqrt{ } 2}\left[R+m A \sqrt{ } 2 \cdot \cos \left(45^{\circ}+\varphi\right)\right] \\
+ & \frac{R^{m-2}}{m \sqrt{2}}\left[R R+m B \sqrt{ } 2 \cdot \cos \left(45^{\circ}+2 \varphi\right)\right] \\
+ & \frac{R^{m-3}}{m \sqrt{2}}\left[R^{3}+m C \sqrt{ } 2 \cdot \cos \left(45^{0}+3 \varphi\right)\right] \\
+ & \frac{R^{m-4}}{m \sqrt{2}}\left[R^{4}+m D \sqrt{ } 2 \cdot \cos \left(45^{\circ}+4 \varphi\right)\right] \\
+ & \text { etc. }
\end{aligned}
\]

quas singulas, pro valore quolibet determinato reali ipsius \(\varphi\), positivas evadere facile perspicitur: hinc \(T\) necessario valorem positivum obtinet. Simili modo probatur, etiam \(U, T^{\prime}, U^{\prime}\) fieri positivas, unde etiam \(T T^{\prime}+U U^{\prime}\) necessario fit quantitas positiva.

II. Pro \(r=R\) functiones \(t, u, t^{\prime}, u^{\prime}\) resp. transeunt in

\[
\begin{aligned}
& T \cos \left(45^{\circ}+m \varphi\right)+U \sin \left(45^{\circ}+m \varphi\right) \\
& T \sin \left(45^{\circ}+m \varphi\right)-U \cos \left(45^{0}+m \varphi\right) \\
& T^{\prime} \cos \left(45^{\circ}+m \varphi\right)+U^{\prime} \sin \left(45^{\circ}+m \varphi\right) \\
& T^{\prime} \sin \left(45^{\circ}+m \varphi\right)-U^{\prime} \cos \left(45^{\circ}+m \varphi\right)
\end{aligned}
\]

uti evolutione facta facile probatur. Hinc vero valor functionis \(t t^{\prime}+u u^{\prime}\), pro \(r=R\), derivatur \(=T T^{\prime}+U U^{\prime}\), adeoque est quantitas positiva. Q. E. D.

Ceterum ex iisdem formulis colligimus valorem functionis \(t t+u u\), pro \(r=R\), esse \(T T+U U\), adeoque positivum, unde concludimus, pro nullo valore ipsius \(r\), singulis \(m A \sqrt{ } 2, \sqrt{ }(m B \sqrt{ } 2), \sqrt[3]{ }(m C \sqrt{ } 2)\) etc. maiori, simul fieri posse \(t=0, u=0\).

2.

Theorema. Intra limites \(r=0\) et \(r=R\), atque \(\varphi=0\) et \(\varphi=360^{\circ}\) certo exstant valores tales indeterminatarum \(r, \varphi\), pro quibus fiat simul \(t=0\) et \(u=0\).

Demonstratio. Supponamus theorema non esse verum, patetque, valorem ipsius \(t t+u u\) pro cunctis valoribus indeterminatarum intra limites assignatos fieri debere quantitatem positivam, et proin valorem ipsius \(y\) semper finitum. Consideremus integrale duplex

\[
\iint y \mathrm{~d} r \mathrm{~d} \varphi
\]

ab \(r=0\) usque ad \(r=R\), atque a \(\varphi=0\) usque ad \(\varphi=360^{\circ}\) extensum, quod igitur valorem finitum plene determinatum nanciscitur. Hic valor, quem
per \(Q\) denotabimus, idem prodire debebit, sive integratio primo instituatur secundum \(\varphi\) ac dein secundum \(r\), sive ordine inverso. At habemus indefinite, considerando \(r\) tamquaim constantem,

\[
\int y \mathrm{~d} \varphi=\frac{t u^{\prime}-u t^{\prime}}{r(t t+u u)}
\]

uti per differentiationem secundum \(\varphi\) facile confirmatur. Constans non adiicienda, siquidem integrale \(a \quad \wp=0\) incipiendum supponamus, quoniam pro \(i=0\) fit \(\frac{t u^{\prime}-u t^{\prime}}{r(t t+u u)}=0\). Quare quum manifesto \(\frac{t u^{\prime}-u t^{\prime}}{r(t t+u u)}\) etiam evanescat pro \(\varphi=360^{\circ}\), integrale \(\int y\) d \(\varphi\) a \(\varphi=0\) usque ad \(\varphi=360^{\circ}\) fit \(=0\), manente \(r\) indefinita. Hinc autem sequitur \(Q=0\).

Perinde habemus indefinite, considerando \(\varphi\) tamquam constantem,

\[
\int y \mathrm{~d} r=\frac{t t^{\prime}+u u^{\prime}}{t t+u u}
\]

uti aeque facile per differentiationem secundum \(r\) confirmatur: hic quoque constans non adiicienda, integrali \(a b=0\) incipiente. Quapropter integrale \(a b\) \(r=0\) usque ad \(r=R\) extensum fit per ea, quae in art. praec. demonstrata sunt, \(=\frac{T T^{\prime}+U U^{\prime}}{T T+U}\) adeoque per theorema art. praec. semper quantitas positiva pro quolibet valore reali ipsius \(\). Hinc etiam \(Q\), i. e. valor integralis

\[
\int \frac{T T^{\prime}+U U^{\prime}}{T T+U U} \mathrm{~d} \varphi
\]

a \(\varphi=0\) usque ad \(\varphi=360^{\circ}\), necessario fit quantitas positiva*). Quod est absurdum, quoniam eandem quantitatem antea invenimus \(=0\) : suppositio itaque consistere nequit, theorematisque veritas hinc evicta est.

3.

Functio \(X\) per substitutionem \(x=r(\cos \varphi+\sin \varphi . V-1)\) transitin \(t+u V-1\), nec non per substitutionem \(x=r(\cos \varphi-\sin \varphi \cdot \sqrt{ }-1)\) in \(t-u \sqrt{ }-1\). Quodsi igitur pro valoribus determinatis ipsarum \(r, \varphi\), puta pro \(r=g, \varphi=G\), simul provenit \(t=0, u=0\) (quales valores exstare in art. praec. demonstratum est), \(\boldsymbol{X}\) per utramque substitutionem

*) Uti iam per se manifestum est. Ceterum integrale indefinitum facile eruitur \(=m \varphi+45^{\circ}-\operatorname{arc\cdot tang} \frac{U}{T}\), atque aliunde demonstrari potest (per se enim nondum obvium est, quemnam valorem ex infinite multis functioni multiformi arc. tang. \(\frac{U}{T}\) competentibus pro \(\varphi=360^{\circ}\) adoptare oporteat), huius valorem usque ad \(\varphi=360^{\circ}\) extensum statui debere \(=m \times 360^{\circ}\) sive \(=2 m \pi\). Sed hoc ad institutum nostrum non est necessarium.

\[
x=g(\cos G+\sin G \cdot V-1), \quad x=g(\cos G-\sin G \cdot V-1)
\]

valorem 0 obtinet, et proin indefinite per

\(x-g(\cos G+\sin G \cdot V-1)\), nec non per \(x-g(\cos G-\sin G \cdot V-1)\)

divisibilis erit. Quoties non est \(\sin G=0\), neque \(g=0\), hi divisores sunt inaequales, et proin \(\boldsymbol{X}\) etiam per illorum productum

\[
x x-2 g \cos G \cdot x+g g
\]

divisibilis erit, quoties autem vel \(\sin G=0\) adeoque \(\cos G= \pm 1\), vel \(g=0\), illi factores sunt identici scilicet \(=x \mp g\). Certum itaque est, functionem \(\mathbf{X}\) involvere divisorem realem secundi vel primi ordinis, et quum eadem conclusio rursus de quotiente valeat, \(\boldsymbol{X}\) in tales factores complete resolubilis erit. Q.E.D.

\section*{4.}
Quamquam in praecedentibus negotio quod propositum erat, iam plene perfuncti simus, tamen haud superfluum erit, adhuc quaedam de ratiocinatione art. 2 adiicere. A suppositione, \(t\) et \(u\) pro nullis valoribus indeterminatarum \(r, \varphi\) intra limites illic assignatos simul evanescere, ad contradictionem inevitabilem delapsi sumus, unde ipsius suppositionis falsitatem conclusimus. Haec igitur contradictio cessare debet, si revera adsunt valores ipsarum \(r, \varphi\), pro quibus \(t\) et \(u\) simul fiunt \(=0\). Quod ut magis illustretur, observamus, pro talibus valoribus fieri \(t t+u u=0\), adeoque ipsam \(y\) infinitam, unde haud amplius licebit, integrale duplex \(\iint y \mathrm{~d} r \mathrm{~d} \varphi\) tamquam quantitatem assignabilem tractare. Generaliter quidem loquendo, denotantibus \(\xi, \eta, \zeta\) indefinite coordinatas punctorum in spatio, integrale \(\iint y \mathrm{~d} r \mathrm{~d} \varphi\) exhibet volumen solidi, quod continetur inter quinque plana, quorum aequationes sunt

\[
\xi=0, \eta=0, \zeta=0, \xi=R, \eta=360^{\circ}
\]

atque superficiem, cuius aequatio \(\zeta=y\), considerando eas partes tamquam negativas, in quibus coordinatae \(\zeta\) sunt negativae. Sed tacite hic subintelligitur, superficiem sextam esse continuam, qua conditione cessante, dum \(y\) evadit infinita, utique fieri potest, ut conceptus ille sensu careat. In tali casu de integrali \(\iint y \mathrm{~d} r \mathrm{~d} \varphi\) colligendo sermo esse nequit, neque adeo mirandum est, operationes analyticas coeco calculo ad inania applicatas ad absurda perducere.

Integratio \(\int y \mathrm{~d} \varphi=\frac{t u^{\prime}-u t^{\prime}}{r(t t+u u)}\) eatenus tantum est integratio vera, i. e. summatio, quatenus inter limites, per quos extenditur, \(y\) ubique est quantitas finita, absurda autem, si inter illos limites \(y\) alicubi infinita evadit. Si integrale tale \(\int \eta d \xi\), quod generaliter loquendo exhibet aream inter lineam abscissarum atque curvam, cuius ordinata \(=\eta\) pro abscissa \(\xi\), secundum regulas suetas evolvimus, continuitatis immemores, saepissime contradictionibus implicamur. E. g. statuendo \(\eta=\frac{1}{\xi \xi}\), analysis suppeditat integrale \(=C-\frac{1}{\xi}\), quo area recte definitur, quamdiu curva continuitatem servat; qua pro \(\xi=0\) interrupta, si quis magnitudinem areae inde ab abscissa negativa usque ad positivam inepte rogat, responsum absurdum a formula feret, eam esse negativam. Quid autem sibi velint haec similiaque analyseos paradoxa, alia occasione fusius persequemur.

Hic unicam observationem adiicere liceat. Propositis absque restrictione quaestionibus, quae certis casibus absurdae evadere possunt, saepissime ita sibi consulit analysis, ut responsum ex parte vagum reddat. Ita pro valore integralis \(\iint y \mathrm{~d} r \mathrm{~d} \varphi\) ab \(r=e\) usque ad \(r=f\), atque a \(\varphi=\boldsymbol{E}\) usque ad \(\varphi=\boldsymbol{F}\) extendendi, si valor ipsius \(\frac{u}{t}\)

\[
\begin{aligned}
& \text { pro } r=e, \varphi=E \text { designatur per } \theta \\
& r=e, \varphi=\boldsymbol{F} \text {. . . . . . . } \theta^{\prime} \\
& r=f, \varphi=E \text {. . . . . . } \theta^{\prime \prime} \\
& r=f, \varphi=F \text {. . . . . . . } \theta^{\prime \prime \prime}
\end{aligned}
\]

per operationes analyticas facile obtinetur

Arc. \(\operatorname{tang} \theta\) - Arc. \(\operatorname{tang} \theta^{\prime}\) - Arc. \(\operatorname{tang} \theta^{\prime \prime}+\) Arc. \(\operatorname{tang} \theta^{\prime \prime \prime}\)

Revera quidem integrale tunc tantum valorem certum habere potest, quoties \(y\) inter limites assignatos semper manet finita: hic valor sub formula tradita utique contentus, tamen per eam nondum ex asse definitur, quoniam Arc. tang. est functio multiformis, seorsimque per alias considerationes (haud quidem difficiles) decidere oportebit, quinam potissimum functionis valores in casu determinato sint adhibendi. Contra quoties \(y\) alicubi inter limites assignatos infinita evadit, quaestio de valore integralis \(\iint y \mathrm{~d} r \mathrm{~d} \varphi\) absurda est: quo non obstante si responsum ab analysi extorquere obstinaveris, pro methodorum diversitate modo hoc modo illud reddetur, quae tamen singula sub formula generali ante tradita contenta erunt.

\section*{B E W E IS }


\section*{ALGEBRAISCHEN LEHRSATZES.}
Journal für die reine und ang. Mathematik herausg. von CRELle. Band III. Berlin 1828.

\section*{EINES ALGEBRAISCHEN LEHRSATZES.}
Der Gegenstand dieses Aufsatzes ist der CaRtesische, gewöhnlich nach HARRIoT benannte, Lehrsatz über den Zusammenhang der Anzahl der positiven und negativen Wurzeln einer algebraischen Gleichung mit der Anzahl der Abwechselungen und Folgen in den Zeichen der Coëfficienten. Man vermisst an den von verschiedenen Schriftstellern versuchten Beweisen dieses Theorems die Klarheit, Kürze und umfassende Allgemeinheit, die man bei einem so elementarischen Gegenstande mit Recht verlangen kann, und eine neue Behandlung desselben scheint daher nicht überflüssig zu sein.

Es sei \(X\) eine algebraische ganze Function von \(x\) von der Ordnung \(m\), nach absteigenden Potenzen von \(x\) geordnet. Wir nehmen an (ohne Nachtheil für die Allgemeinheit), dass das höchste Glied \(x^{m}\) sei, und das niedrigste von \(x\) freie Glied nicht fehle; bloss die wirklich vorhandenen Glieder sollen aufgestellt, also nicht die etwa fehlenden mit dem Coëfficienten 0 angesetzt sein.

Wenn nicht alle Coëfficienten positiv sind, so werden sie einen oder mehrere Zeichenwechsel darbieten. Es sei \(-\boldsymbol{N} x^{n}\) das erste negative Glied, das erste hierauf folgende positive \(+P x^{p}\), das erste hierauf folgende negative \(-Q x^{q}\) u.s.w. Es sind mithin \(m, n, p, q\) u.s.w. abnehmende ganze Zahlen; \(N, P, Q\) u.s.w. positiv, and \(\mathbf{X}\) erscheint so dargestellt

\[
\mathbf{X}=x^{m}++\ldots-\boldsymbol{N} x^{n}-\ldots+\boldsymbol{P} x^{p}++\ldots-Q x^{q}-\text { u.s.w. }
\]

Es werde \(\boldsymbol{X}\) mit dem einfachen Factor \(x-\alpha\) multiplicirt, wo \(\alpha\) positiv vorausgesetzt wird. Man sieht leicht, dass in dem Producte, \(x^{n+1}\) einen negativen, \(x^{p+1}\) einen positiven, \(x^{q+1}\) einen negativen Coëfficienten u.s.w., also das Product diese Form erhalten wird:

\[
X(x-\alpha)=x^{m+1} \ldots-N^{\prime} x^{n+1} \ldots+P^{\prime} x^{p+1} \ldots-Q^{\prime} x^{q+1} \ldots
\]

so dass \(N^{\prime}, P^{\prime}, Q^{\prime}\) u.s. w. positiv werden. Die Zeichen zwischen den aufgestellten Gliedern bleiben zwar unentschieden: allein es ist klar, dass vom höchsten Gliede bis zur Potenz \(x^{n+1}\) wenigstens ein Zeichenwechsel, bis \(x^{p+1}\) wenigstens zwei, bis \(x^{q+1}\) wenigstens drei u.s. w. statt finden. Ist der letzte Zeichenwechsel in \(\boldsymbol{X}\) bei dem Gliede \(\pm U x^{u}\), und bezeichnet man den Coëfficienten von \(x^{u+1}\) in \(X(x-\alpha)\) durch \(\pm U^{\prime}\), so wird \(U^{\prime}\) positiv sein, und bis zum Gliede \(\pm U^{\prime} x^{u+1}\) haben dann wenigstens eben so viele Zeichenwechsel, wie in \(\mathbf{X}\) sind, statt gefunden. Das letzte Glied in \(\boldsymbol{X}(x-\alpha)\) wird aber das Zeichen \(\mp\) haben: es muss also bis dahin wenigstens noch ein Zeichenwechsel hinzugekommen sein. Wir schliessen also, dass \(\boldsymbol{X}(x-\alpha)\) wenigstens einen Zeichenwechsel mehr hat als \(\boldsymbol{X}\).

Es sei nun \(\boldsymbol{X}\) das Product aller einfachen Factoren, die den negativen und imaginären Wurzeln einer Gleichung \(y=0\) entsprechen, also wenn \(\alpha, \boldsymbol{b}, \gamma\) u.s.w. die positiven Wurzeln derselben Gleichung sind,

\[
y=X(x-\alpha)(x-b)(x-\gamma) \ldots
\]

Es finden sich also nach vorstehendem Satze, in \(\boldsymbol{X}(x-\alpha)\) wenigstens ein Zeichenwechsel, in \(X(x-\alpha)(x-b)\) wenigstens zwei, in \(X(x-\alpha)(x-b)(x-\gamma)\) wenigstens drei \(u\).s. w. mehr als in \(\boldsymbol{X}\); folglich werden, auch wenn in \(\boldsymbol{X}\) gar kein Zeichenwechsel vorkommt, in \(y\) wenigstens so viele Zeichenwechsel sein, wie positive Wurzeln. Man sieht von selbst, dass wenn die Gleichung weder negative, noch imaginäre Wurzeln hat, man \(X=1 \mathrm{zu}\) setzen hat, und dieser Schluss seine Gültigkeit behält.

Es gehe \(y\), wenn den Coëfficienten der Potenzen \(x^{m-1}, x^{m-3}, x^{m-5}\) u. s.w. die entgegengesetzten Zeichen beigelegt werden, in \(y^{\prime}\) über; sämmtliche Wurzeln der Gleichung \(y^{\prime}=0\) werden dann den Wurzeln der Gleichung \(y=0\) entgegengesetzt sein. Es wird daher in \(y^{\prime}\) wenigstens eben so viele Zeichenwechsel geben, als die Gleichung \(y=0\) negative Wurzeln hat.

Wir haben daher folgenden Lehrsatz:

Die Gleichung \(y=0\) kann nicht mehr positive Wurzeln haben, als es Zei- chenwechsel in y gibt, und nicht mehr negative Wurzeln, als Zeichenwechsel in \(y^{\prime}\) sind.

Diese Einkleidung des Theorems scheint die zweckmässigste zu sein, da sie die grösste Einfachheit mit der umfassendsten Allgemeinheit vereinigt, und alle Gestalten des Satzes, die nur unter besondern Bedingungen gelten, von selbst daraus fliessen.

Will man die Grenze der Anzahl der negativen Wurzeln unmittelbar an den Zeichen der Coëfficienten von \(y\) erkennen, so wird es nothwendig, die unmittelbaren Zeichenwechsel und Zeichenfolgen (bei Gliedern, wo die Exponenten von \(x\) um eine Einheit verschieden sind) von den durch fehlende Glieder unterbrochenen zu unterscheiden. Offenbar wird jeder unmittelbare und jeder durch eine gerade Anzahl fehlender Glieder unterbrochene Zeichenwechsel in \(y^{\prime}\) zu einer ähnlichen Zeichenfolge in \(y\), während ein durch eine ungerade Anzahl fehlender Glieder unterbrochener Zeichenwechsel in \(y^{\prime}\) auch in \(y\) ein ähnlicher. Zeichenwechsel bleibt. Der zweite Theil des Theorems lässt sich daher auch so ausdrücken:

Die Anzahl der negativen Wurzeln der Gleichung \(y=0\) kann nicht grösser sein, als die Anzahl der unmittelbaren und der durch eine gerade Anzahl fehlender Glieder unterbrochenen Zeichenfolgen, addirt zu der Anzahl der durch eine ungerade Anzahl fehlender Glieder unterbrochenen Zeichenwechsel in \(y\).

Fehlt in \(y\) gar kein Glied, so ist die Anzahl der negativen Wurzeln nicht grösser, als die Anzahl der Zeichenfolgen.

Bezeichnet man durch \(A\) die Anzahl der unmittelbaren Zeichenwechsel, und durch \(B\) die Anzahl der unmittelbaren Zeichenfolgen in \(y\), so wird, wenn kein Glied fehlt, \(A+B=m\) sein, also der Anzahl aller Wurzeln gleich. Insofern diese Zeichen also bloss lehren, dass die Anzahl der positiven Wurzeln nicht grösser als \(A\), und die der negativen nicht grösser als \(B\) sein kann. bleibt es unentschieden, ob oder wie viele imaginäre Wurzeln vorhanden sind. Weiss man aber anders woher, dass die Gleichung keine imaginäre. Wurzeln hat, so muss nothwendig \(A\) der Anzahl der positiven, und \(B\) der Anzahl der negativen Wurzeln gleich sein.

Anders aber verhält es sich, wenn in \(y\) Glieder fehlen. Um mit Klarheit
zu übersehen, was sich daraus in Beziehung auf die imaginären Wurzeln schliessen lässt, bezeichnen wir durch \(a\) die Anzahl der durch eine gerade, durch \(c\) die Anzahl der durch eine ungerade Anzahl fehlender Glieder unterbrochenen Zeichenwechsel; durch \(b\) und \(d\) resp. die Anzahl der durch eine gerade und ungerade Anzahl fehlender Glieder unterbrochenen Zeichenfolgen in \(y\). Man sieht leicht, dass \(m-A-B-a-b-c-d\) der Anzahl sämmtlicher fehlender Glieder, die wir durch \(e\) bezeichnen wollen, gleich sein werde. Nun ist nach unserm Lehrsatze die Anzahl der positiven Wurzeln höchstens \(A+a+c\), die Anzahl der negativen höchstens \(B+b+c\), also die Anzahl aller reellen Wurzeln höchstens

\[
A+B+a+b+2 c=m+c-d-e
\]

Es muss daher die Anzahl der imaginären Wurzeln wenigstens \(e-c+d\) sein.

Zählt man also alle fehlenden Glieder zusammen, jedoch so, dass man in jeder Lücke zwischen einem Zeichenwechsel eine Einheit weniger, zwischen einer Zeichenfolge aber eine Einheit mehr rechnet, als Glieder fehlen, so oft deren Anzahl ungerade ist, so erhält man eine Zahl, der die Anzahl der imaginären Wurzeln wenigstens gleich kommen muss.

\section*{BEITRÄGE ZUR THEORIE}
DER

\section*{ALGEBRAISCHEN GLEICHUNGEN}
VON

\section*{CARL FRIEDRICH GAUSS}
Vorgelesen in der Sitzung der Königl. Gesellschaft der Wissenschaften am 16. Juli 1849.

Abhandlungen der Königl. Gesellschaft der Wissenschaften zu Göttingen. Band Iv. Göttingen, 1850.

\section*{DER ALGEBRAISCHEN GLEICHUNGEN.}
Es werden in dieser Denkschrift zwei verschiedene die algebraischen Gleichungen betreffende Gegenstände behandelt. Zuerst stelle ich den vor funfzig Jahren von mir gegebenen Beweis des Grundlehrsatzes der Theorie der algebraischen Gleichungen in einer veränderten Gestalt und mit erheblichen Zusätzen auf. Der zweite Theil ist einer speciellen Behandlung der algebraischen Gleichungen mit drei Gliedern gewidmet, und enthält Methoden, nicht bloss die reellen, sondern auch die imaginären Wurzeln solcher Gleichungen mit Leichtigkeit zu bestimmen.

\section*{ERSTE ABTHEILUNG.}
Die im Jahre 1799 erschienene Denkschrift, Demonstratio nova theorematis, omnem functionem algebraicam rationalem integram unius variabilis in factores reales primi vel secundi gradus resolvi posse, hatte einen doppelten Zweck, nemlich erstens, zu zeigen, dass sämmtliche bis dahin versuchte Beweise dieses wichtigsten Lehrsatzes der Theorie der algebraischen Gleichungen ungenügend und illusorisch sind, und zweitens, einen neuen vollkommen strengen Beweis zu geben. Es ist unnöthig, auf den erstern Gegenstand noch einmal zurückzukommen. Dem dort gegebenen neuen Beweise habe ich selbst später noch zwei andere folgen lassen, und ein vierter ist zuerst von CACCBY aufgestellt. Diese vier Beweise beru-
hen alle auf eben so vielen verschiedenen Grundlagen, aber darin kommen sie alle überein, dass durch jeden derselben zunächst nur das Vorhandensein Eines Factors der betreffenden Function erwiesen wird. Der Strenge der Beweise thut dies allerdings keinen Eintrag: denn es ist klar, dass wenn von der vorgegebenen Function dieser eine Factor abgelöset wird, eine ähnliche Function von niederer Ordnung zurückbleibt, auf welche der Lehrsatz aufs neue angewandt werden kann, und dass durch Wiederholung des Verfahrens zületzt eine vollständige Zerlegung der ursprünglichen Function in Factoren der bezeichneten Art hervorgehen wird. Indessen gewinnt ohne Zweifel jede Beweisführung eine höhere Vollendung, wenn nachgewiesen wird, dass sie geeignet ist, das Vorhandensein der sämmtlichen Factoren unmittelbar anschaulich zu machen. Dass der erste Beweis in diesem Fall ist, habe ich bereits in der gedachten Denkschrift angedeutet (Art. 23), ohne es dort weiter auszuführen: dies soll jetzt ergänzt werden, und ich benutze zugleich diese Gelegenheit, die Hauptmomente des ganzen Beweises in einer abgeänderten und, wie ich glaube, eine vergrösserte Klarheit darbietenden Gestalt zu wiederholen. Was dabei die äussere Einkleidung des Lehrsatzes selbst betrifft, so war die 1799 gebrauchte, dass die Function \(x^{n}+A x^{n-1}+B x^{n-2}+\) u.s. w. sich in reelle Factoren erster oder zweiter Ordnung zerlegen lässt, damals deshalb gewählt, weil alle Einmischung imaginärer Grössen vermieden werden sollte. Gegenwärtig, wo der Begriff der complexen Grössen jedermann geläufig ist, scheint es angemessener, jene Form fahren zu lassen und den Satz so auszusprechen, dass jene Function sich in \(n\) einfache Factoren zerlegen lasse, wo dann die constanten Theile dieser Factoren nicht eben reelle Grössen zu sein brauchen, sondern für dieselben auch jede complexen Werthe zulässig sein müssen. Bei dieser Einkleidung gewinnt selbst der Satz noch an Allgemeinheit, weil dann die Beschränkung auf reelle Werthe auch bei den Coëfficienten \(A, B\) u.s.w. nicht vorausgesetzt zu werden braucht, vielmehr jedwede Werthe für dieselben zulässig bleiben.

1.

Wir betrachten demnach die Function der unbestimmten Grösse \(x\)

\[
x^{n}+A x^{n-1}+B x^{n-2}+\text { u.s. w. }+M x+N=X
\]

wo \(A, B \ldots M, N\) bestimmte reelle oder imaginäre Coëfficienten vorstellen.

Aus der Elementaralgebra ist der Zusammenhang zwischen den Wurzeln der Gleichung \(\boldsymbol{X}=0\) und den einfachen Factoren von \(\boldsymbol{X}\) bekannt. Geschieht nemlich jener Gleichung durch die Substitution \(x=p\) Genüge, so ist \(x-p\) ein Factor von \(X\), und gibt es \(n\) verschiedene Arten, jener Gleichung Genüge zu leisten, nemlich durch \(x=p, x=p^{\prime}, x=p^{\prime \prime}\) u.s. w., so wird das Product \((x-p)\) \(\left(x-p^{\prime}\right)\left(x-p^{\prime \prime}\right) \ldots\) mit \(\boldsymbol{X}\) identisch sein. Unter besondern Umständen kann aber auch eine Auflösung, wie \(x=p\), in \(X\) den Factor \((x-p)^{2}\), oder \((x-p)^{3}\) oder irgend eine höhere Potenz bedingen, in welçhen Fällen man die Wurzel \(p\) wie zweimal, dreimal u.s.w. vorhanden betrachtet.

Verlangt man also nur den Beweis, dass die Function \(\boldsymbol{X}\) gewiss tinen einfachen Factor zulasse, so ist es zureichend, nur das Vorhandensein irgend einer Wurzel der Gleichung \(\boldsymbol{X}=0\) nachzuweisen. Soll aber die vollständige Zerlegbarkeit der Function in einfache Factoren auf Einmal bewiesen werden, so muss gezeigt werden, dass der Gleichung \(X=0\) Genüge geleistet werden kann, entweder durch \(n\) ungleiche Werthe von \(x\), oder durch eine zwar geringere Anzahl ungleicher Auflösungen, wovon aber ein Theil die Charactere der mehrfach geltenden gleichen Wurzeln dergestalt an sich trägt, dass die Zusammenzähiung aller ungleichen und gleichen die Totalsumme \(=n\) hervorbringt.

2.

Das ganze Gebiet der complexen Grössen, in welchem die der Gleichung \(\boldsymbol{X}=0\) genügenden Werthe von \(x\) gesucht werden sollen, ist ein Unendliches von zwei Dimensionen, indem, wenn ein solcher Werth \(x=t+i u\) gesetzt wird (wo \(i\) immer die imaginäre Einheit \(\sqrt{ }-1\) bedeutet), für \(t\) und \(u\) alle reellen Werthe von \(-\infty\) bis \(+\infty\) zulässig sind. Wir haben nun zuvörderst aus diesem unendlichen Gebiete ein abgegrenztes endliches auszuscheiden, ausserhalb dessen gewiss keine Wurzel der bestimmten Gleichung \(\mathbf{X}=\mathbf{0}\) liegen kann. Dies kann auf mehr als Eine Art geschehen: unserm Zweck am meisten gemäss scheint die folgende zu sein.

Anstatt der Form \(t+i u\) gebrauche man diese

\[
x=r(\cos \rho+i \sin \rho)
\]

wonach zur Umfassung des ganzen unendlichen Gebiets der complexen Grössen| \(r\) durch alle positiven Werthe von 0 bis \(+\infty\), und \(\rho\) von 0 bis \(360^{\circ}\), oder, was
dasselbe ist, von einem beliebigen Anfangswerthe bis an einen um \(360^{\circ}\) grössern Endwerth ausgedehnt werden muss.

Um für \(r\) eine Grenze zu, erhalten, über welche hinaus kein Werth mehr einer Wurzel der Gleichung \(X=0\) entsprechen kann, setze ich zuvörderst die Coëfficienten der einzelnen Glieder von \(X\) in eine ähnliche Form, wie \(x\), nemlich

\[
\begin{aligned}
& A=a(\cos \alpha+i \sin \alpha) \\
& B=b(\cos b+i \sin b) \\
& C=c(\cos \gamma+i \sin \gamma) \text { u.s.w. }
\end{aligned}
\]

wo also \(a, b, c\) bestimmte positive Grössen bedeuten sollen, abgesehen davon, dass auch eine oder die andere darunter \(=0\) sein kann. Ich betrachte sodann die Gleichung

\[
r^{n}-\sqrt{ } 2 \cdot\left(a r^{n-1}+b r^{n-2}+c r^{n-3}+\text { u.s.w. }\right)=0
\]

welche, wie man leicht sieht, eine positive Wurzel hat, und zwar (HARRIOTs Lehrsatz zufolge) nur Eine solche. Es sei \(R\) diese Wurzel, wo dann von selbst klar ist, dass für jeden positiven Werth von \(r\), der grösser ist als \(R\), der Werth von \(r^{n}-\sqrt{ }\) 2. \(\left(a r^{n-1}+b r^{n-2}+c r^{n-3}+\right.\) u.s.w. \()\) positiv sein, und dass dasselbe auch von der Function

\[
n r^{n}-\sqrt{ } 2 .\left((n-1) a r^{n-1}+(n-2) b r^{n-2}+(n-3) c r^{n-3}+\text { u.s. w. }\right)
\]

gelten wird, da dieselbe das \(n\) fache der erstern Function um

\[
\sqrt{ } 2 .\left(a r^{n-1}+2 b r^{n-2}+3 c r^{n-3}+\text { u.s.w. }\right)
\]

also um eine positive Differenz übertrifft.

3.

Ich behaupte nun, dass die Grösse \(R\) geeignet ist, eine solche Grenze für die Werthe von \(r\), wie im vorhergehenden Artikel gefordert ist, abzugeben. Der Beweis dieses Satzes ist auf folgende Art zu führen.

Ich setze allgemein \(X=T+i U\), wo selbstredend \(T\) und \(U\) reelle Grössen bedeuten, und zwar wird

\[
\begin{gathered}
T=r^{n} \cos n \rho+a r^{n-1} \cos ((n-1) \rho+\alpha)+b r^{n-2} \cos ((n-2) \rho+b) \\
+c r^{n-3} \cos ((n-3) \rho+\gamma)+\text { u. s. w. }
\end{gathered}
\]

\[
\begin{aligned}
U=r^{n} \sin n \rho & +a r^{n-1} \sin ((n-1) \rho+\alpha)+b r^{n-2} \sin ((n-2) \rho+b) \\
& +c r^{n-3} \sin ((n-3) \rho+\gamma)+\text { u. s. w. }
\end{aligned}
\]

Man übersieht leicht, dass wenn für \(\boldsymbol{r}\) irgend ein positiver Werth grösser als \(\boldsymbol{R}\) gewählt wird, \(T\) nothwendig dasselbe Zeichen haben wird wie \(\cos n \rho\), so oft dieser Cosinus absolut genommen nicht kleiner ist als \(\sqrt{ } \frac{1}{2}\). Man braucht nemlich nur \(T\) in folgende Form zu setzen

\[
\begin{aligned}
\pm T= & \sqrt{ } \frac{1}{2} \cdot r^{n}-a r^{n-1}-b r^{n-2}-c r^{n-3}-\text { u.s.w. } \\
& +\left[ \pm \cos n \rho-\sqrt{\frac{1}{2}}\right] r^{n} \\
& +[1 \pm \cos ((n-1) \rho+\alpha)] a r^{n-1} \\
& +[1 \pm \cos ((n-2) \rho+6)] b r^{n-2} \\
& +[1 \pm \cos ((n-3) \rho+\gamma)] c r^{n-3} \\
& + \text { u.s.w. }
\end{aligned}
\]

wo die obern Zeichen für den Fall eines positiven, die untern für den Fall eines negativen \(\cos n \rho\) gelten sollen, und wo der erste Theil des Ausdrucks auf der rechten Seite positiv ist, in Folge des im vorhergehenden Artikel gegebenen Satzes, von den folgenden aber wenigstens keiner negativ werden kann. Auf ganz ähnliche Weise erhellet (indem man in obiger Formel nur \(U\) anstatt \(T\) und durchgehends Sinus anstatt Cosinus schreibt), dass unter gleicher Voraussetzung in Beziehung auf \(r\), allemal \(U\) dasselbe Zeichen hat wie \(\sin n \rho\), so oft dieser Sinus absolut genommen nicht kleiner ist als \(\sqrt{\frac{1}{2}}\). Es hat demnach in allen Fällen wenigstens die eine der beiden Grössen \(T, U\) ein voraus bestimmtes positives oder negatives Zeichen, und es kann folglich für keinen Werth von \(\rho\) die Function \(\boldsymbol{X}=0\) werden. W. Z. B. W.

4.

Um das Verhalten von \(T\) und \(U\) in Beziehung auf die Zeichen und deren Wechsel (bei einem bestimmten, \(\boldsymbol{R}\) überschreitenden, Werthe von \(r\) ) noch mehr ins Licht zu setzen, lasse man \(\rho\) alle Werthe zwischen zwei um \(360^{\circ}\) verschiedenen Grenzen durchlaufen, wozu jedoch nicht 0 und \(360^{\circ}\), sondern, indem zur Abkürzuing

\[
\frac{45^{\circ}}{n}=\omega
\]

gesetzt wird, - \(\omega\) und \((8 n-1) \omega\) gewählt werden sollen. Den ganzen Zwischenraum theile ich in \(4 n\) gleiche Theile, so dass der erste sich von \(-\omega\) bis \(\omega\), der zweite von \(\omega\) bis \(3 \omega\), der dritte von \(3 \omega\) bis \(5 \omega\) u.s.w. erstreckt. Zuvörderst hat man auch noch die Werthe der Differentialquotienten \(\frac{\mathrm{d} T}{\mathrm{~d} \rho}, \frac{\mathrm{d} U}{\mathrm{~d} \rho}\) in Betracht zu ziehen, wofür man hat

\[
\begin{aligned}
& \left.\frac{\mathrm{d} T}{\mathrm{~d} \rho}=-n r^{n} \sin n \rho-(n-1) a r^{n-1} \sin ((n-1) \rho+\alpha)-(n-2) b r^{n-2} \sin ((n-2) \rho+b)\right) \\
& -(n-3) c r^{n-3} \sin ((n-3) \rho+\gamma)-\text { u.s.w. } \\
& \left.\frac{\mathrm{d} U}{\mathrm{~d} \rho}=n r^{n} \cos n \rho+(n-1) a r^{n-1} \cos ((n-1) \rho+\alpha)+(n-2) b r^{n-2} \cos ((n-2) \rho+b)\right) \\
& +(n-3) c r^{n-3} \cos ((n-3) \rho+\gamma)+\text { u.s.w. }
\end{aligned}
\]

Man erkennt daraus leicht, durch ähnliche Schlüsse wie im vorhergehenden Artikel und unter Zuziehung des Satzes am Schlusse von Art. 2, dass \(\frac{\mathrm{d} T}{\mathrm{~d} \rho}\) immer das entgegengesetzte Zeichen von \(\sin n \rho\) hat, so oft dieser Sinus absolut genommen nicht kleiner ist als \(V_{\frac{1}{2}}\), dass hingegen \(\frac{\mathrm{d} U}{\mathrm{~d} \rho}\) immer dasselbe Zeichen wie \(\cos n \rho\) hat, so oft der absolute Werth dieses Cosinus nicht kleiner ist als \(\sqrt{\frac{1}{2}}\). Hieraus zieht man folgende Schlüsse.

In dem ersten Intervalle, d. i. von \(\rho=-\omega\) bis \(\rho=+\omega\), ist \(T\) stets positiv, \(U\) hingegen für den Anfangswerth negativ, für den Endwerth positiv, mithin dazwischen gewiss einmal \(=0\), und zwar nur einmal, weil in dem ganzen Intervalle \(\frac{d}{d \rho}\) positiv ist.

In dem zweiten Intervalle ist \(U\) stets positiv, \(T\) zu Anfang positiv, am Ende negativ, dazwischen einmal \(\boldsymbol{T}=0\) und zwar nur einmal, weil in dem ganzen Intervalle \(\frac{\mathrm{d} T}{\mathrm{~d} \rho}\) negativ ist.

In dem dritten Intervalle ist \(T\) stets negativ, \(U\) einem Zeichenwechsel unterworfen, so dass einmal \(U=0\) wird.

Im vierten Intervalle ist \(U\) stets negativ, \(T\) einmal \(=0\).

In den folgenden Intervallen wiederholen sich in gleicher Ordnung diese Verhältnissse, so dass das fünfte dem ersten, das sechste dem zweiten u.s.f. gleichsteht.

5.

Aus der im vorhergehenden Artikel erörterten Folgeordnung der positiven und negativen Werthe von \(T\) und \(U\), die bei jedem über \(R\) hinausgehenden

Werthe von \(r\) Statt findet*), lässt sich nun folgern, dass innerhalb des Gebiets . der kleinern Werthe von \(r\) gewisse Kreuzungen in diesen Anordnungen vorhanden sein müssen, die das Wesen unsers zu beweisenden Lehrsatzes in sich schliessen. Ich werde die Beweisführung in einer der Geometrie der Lage entnommenen Einkleidung darstellen, weil jene dadurch die grösste Anschaulichkeit und Einfachheit gewinnt. Im Grunde gehört aber der eigentliche Inhalt der ganzen Argumentation einem höhern von Räumlichem unabhängigen Gebiete der allgemeinen abstracten Grössenlehre an, dessen Gegenstand die nach der Stetigkeit zusammenhängenden Grössencombinationen sind, einem Gebiete, welches zur Zeit noch wenig angebauet ist, und in welchem man sich auch nicht bewegen kann ohne eine von räumlichen Bildern entlehnte Sprache.

\[
6 .
\]

Das ganze Gebiet der complexen Grössen wird vertreten durch eine unbegrenzte Ebene, in welcher jeder Punkt, dessen Coordinaten in Beziehung auf zwei einander rechtwinklig schneidende Achsen \(t, u\) sind, als der complexen Grösse \(x=t+i u\) entsprechend betrachtet wird: bringt man diese complexe Grösse in die Form \(x=r(\cos \rho+i \sin \rho)\), so bedeuten \(r, \rho\) die Polarcoordinaten des entsprechenden Punkts. Der Inbegriff aller complexen Grössen, für welche \(r\) einerlei bestimmten Werth hat, wird demnach durch einen Kreis repräsentirt, dessen Halbmesser dieser Werth, und dessen Mittelpunkt der Anfangspunkt der Coordinaten ist. Denjenigen dieser Kreise, für welchen \(r\) um eine nach Belieben gewählte Differenz grösser als \(\boldsymbol{R}\) ist, will ich mit \(K\) bezeichnen, und mit (1), (2), (3) ... ( \(2 n\) ) diejenigen Punkte auf demselben, welchen die beziehungsweise zwischen \(\omega\) und \(3 \omega\), zwischen \(5 \omega\) und \(7 \omega\), zwischen \(9 \omega\) und \(11 \omega\) u.s.f. bis zwischen \((8 n-3) \omega\) und \((8 n-1) \omega\) liegenden Werthe von \(\rho\) entsprechen, für welche nach dem 4. Artikel \(T=0\) wird. Man bemerke dabei, dass für die Punkte (1), (3), (5) u.s.w. \(U\) positiv, für die Punkte (2), (4), (6) u.s. w. hingegen negativ sein wird.

*) Es ist leicht zu zeigen, dass auch für den Werth \(r=R\) selbst eine gleiche Folgeordnung noch gültig bleibt, nur mit der Einschrănkung, dass dann in ganz speciellen Fällen ein Uebergangswerth von \(\rho\), (d. i. ein solcher, für welchen \(T\) oder \(U=0\) wird) mit einer der Grössen - \(\omega, \omega, 3 \omega, 5 \omega\) u. s.w. zusammenfallen kann, während für alle grösseren Werthe von \(r\) jeder Uebergangswerth von \(\rho\) zwischen zweien dieser Grössen liegen muss. Ich halte mich jedoch dabei nicht auf, da für unsern Zweck zareicht, das Bestehen jener Folgeordnung, von irgend einem Werthe von \(r\) an, nachgewiesen zu haben.

7.

Die Gesammtheit derjenigen Punkte in unserer Ebene, für welche \(T\) positiv ist, bildet zusammenhängende Flächentheile, wie schon von selbst erhellet, wenn man erwägt, dass bei einem stetigen Uebergange von einem Punkte zu einem andern \(T\) sich nach der Stetigkeit ändert. Eben so bilden sämmtliche Punkte, für welche \(T\) negativ wird, zusammenhängende Flächentheile. Zwischen den Flächentheilen der ersten Art und denen der zweiten liegen Punkte, in welchen \(T=0\) wird, und nach der Natur der Function \(T\) können diese Punkte nicht auch Flächenstücke, sondern nur Linien bilden, welche einerseits die einen, andererseits die andern Flächentheile begrenzen.

Der ausserhalb \(K\) liegende Raum enthält \(n\) Flächen der ersten Art, die mit eben so vielen der zweiten Art abwechseln, und wovon jede, von einem Stück der Kreislinie \(K\) an, zusammenhängend sich ins Unendliche erstreckt. Zugleich aber ist klar, dass jedes dieser Flächenstücke sich über die Kreislinie hinaus in den innern Raum fortsetzt, und dass in Beziehung auf die weitere Gestaltung folgende Fälle Statt finden können.

\begin{enumerate}
  \item Das betreffende von einem Theile von \(K\) anfangende Flächenstück endigt sich isolirt innerhalb der Kreisfläche; seine peripherische Begrenzung besteht dann nur aus zwei zusammenhängenden Stücken, wovon eines ein Bestandtheil von \(\boldsymbol{K}\) ist, das andere innerhalb des Kreisraumes liegt. In der beigefügten Figur, welche sich auf eine Gleichung fünften Grades bezieht und wo die Zeichen von \(T\) in den verschiedenen Flächentheilen eingeschrieben sind, finden sich drei der Flächen mit positivem \(T\) in diesem Falle; die eine hat die Grenzlinien 10.1 und 1.11.10; die zweite diese 4.5 und 5.12.4; die dritte 6.7 und 7.13.6. Flächentheile ähnlicher Art mit negativem \(T\) finden sich zwei vor.

  \item Das Flächenstück durchsetzt einfach die Kreisfläche dergestalt, dass es mit einem an einer andern Stelle eintretenden Eine zusammenhängende Fläche bildet. Die ganze peripherische Begrenzungslinie wird dann aus vier Stücken bestehen, von denen zwei der Kreislinie \(K\) angehören, und die beiden andern dem innern Raume. In unserer Figur findet sich dieser Fall bei dem durch 2.3; 3.0.8;8.9; 9.11.2 begrenzten Flächenstück.

  \item Das Flächenstück spaltet sich im innern Kreisraume einmal oder mehreremale dergestalt, dass es mit noch zweien oder mehrern an andern Stellen eintretenden eine zusammenhängende Fläche bildet, deren ganze peripherische Be-
grenzung dann aus sechs, acht oder mehrern Stücken in gerader Zahl bestehen wird, die abwechselnd der Kreislinie und dem innern Raume angehören. In unserer Figur tritt dies ein bei einem Flächentheile, dessen Begrenzung durch die sechs Stücke \(3.4 ; 4.12 .5 ; 5.6 ; 6.13 .7 ; 7.8 ; 8.0 .3\) gebildet wird, in welchem aber \(T\) negativ ist.

\end{enumerate}

\section*{8.}
Bei einer vollständigen Aufzählung aller denkbaren Gestaltungen der in den innern Kreisraum eintretenden Flächentheile würden den angegebenen Fällen noch anderweitige Modificationen beigefügt werden müssen. Wenn z. B. ein solcher Flächentheil sich zwar in zwei Aeste spaltet, diese aber im innern Raume sich wieder vereinigen, so würde dieser Fall, jenachdem nach der Vereinigung die Fläche im Innern ihren Abschluss findet, oder (ohne neue Theilung) sich bis zu einer andern Stelle der Kreislinie fortsetzt, dem ersten oder zweiten Falle des vorhergehenden Artikels zugerechnet werden können, indem die Gestaltung der Fläche nur durch das Einschliessen einer nicht zu ihr gehörenden Insel modificirt sein würde. Uebrigens würde es nicht schwer sein, strenge zu beweisen, dass bei der besondern Beschaffenheit der Function \(T\) Modificationen dieser Art gar nicht möglich sind: für unsern Zweck ist dies jedoch unnöthig, indem es nur auf die Folge der Stücke der äussern Begrenzung jedes der in Rede stehenden Flächentheile (d.i. derjenigen, in welchen \(T\) positiv ist) ankommt.

Wir haben nemlich schon bemerklich gemacht, dass die Anzahl dieser Stücke allemal gerade ist (zwei im ersten Falle des vorhergehenden Artikels, vier im zweiten, sechs oder mehrere im dritten), wovon wechselsweise eines der Kreislinie \(K\), eines dem innern Raume angehört. Ferner ist klar, dass wenn jene äussere Begrenzungslinie immer in einerlei Sinn durchlaufen wird, wozu hier derjenige gewählt werden soll, in welchem die Bezifrungen der Punkte von \(\boldsymbol{K}\) wachsen (also, Beispiels halber in unserer Figur so, dass die Fläche immer rechts von der Begrenzungslinie liegt), der Anfangspunkt und der Endpunkt eines der Kreislinie angehörenden Stücks beziehungsweise durch eine gerade und die um eine Einheit grössere ungerade Zahl bezeichnet sein wird, mithin der Anfangspunkt und der Endpunkt jedes den innern Raum durchlaufenden Stücks allemal beziehungsweise durch eine ungerade und eine gerade Zahl.

Es steht also fest, dass von den \(n\) an einem mit einer ungeraden Zahl be-
zeichneten Punkte von \(K\) in den innern Raum eintretenden Linien, in denen überall \(T=0\) ist, eine jede auf eine ganz bestimmte Art*) diesen Raum zusammenhängend durchläuft, bis sie an einer andern mit einer geraden Zahl bezeichneten Stelle wieder austritt. Da nun, wie schon oben (Schluss des 6 . Art.) bemerkt ist, in ihrem Anfangspunkte den Werth von \(U\) positiv, am Endpunkte negativ ist, so muss wegen der Stetigkeit der Werthänderung nothwendig in einem Zwischenpunkte \(U=0\) werden. Dieser Punkt repräsentirt dann eine Wurzel der Gleichung \(X=0\); und da die Anzahl solcher Linien \(=n\) ist, so ergeben sich auf diese Weise allemal \(n\) Wurzeln jener Gleichung.

9.

Wenn die gedachten Linien durch den Kreisraum gehen ohne ein Zusammentreffen mit einander, so ist klar, dass die so erhaltenen \(n\) Wurzeln nothwendig ungleich sind. Ein solches freies Durchgehen findet sich in unsrer Figur bei den Linien von 3 nach 8 , von 5 nach 4 und von 7 nach 6 , und es gehören dazu die durch dia Punkte 0, 12, 13 repräsentirten Wurzeln. Wenn hingegen zwei solcher Linien, oder mehrere, einen Punkt gemeinschaftlich haben, so ist zwar darum noch nicht nothwendig, aber doch möglich, dass dieser Punkt zugleich derjenige ist, in welchem \(U=0\) wird, in welchem Falle dann zwei oder mehrere Wurzeln in Eine zusammenfallen, oder, wie es gewöhnlich ausgedrückt wird, unter sich gleich sein werden. In unsrer Figur treffen die Linien 1.10 und 9.2 in dem Punkte 11 zusammen, und in demselben wird zugleich \(U=0\); die Gleichung hat also ausser den schon aufgeführten drei ungleichen noch zwei gleiche Wurzeln.

\section*{10.}
Es bleibt nur noch übrig, nachzuweisen, dass wenn der eine Wurzel \(=p\)
\footnotetext{*) Dass sie allemal einen ganz bestimmten Lauf hat, beruhet darauf, dass sie einen Theil der aussern Abgrenzung einer Fläche, für welche \(\boldsymbol{T}\) ein bestimmtes Zeichen hat, ausmachen soll: ich habe das positive Zeichen gewåhlt, was an sich ganz willkürlich ist. So verstanden setzt sich z. B. die in 1 eintretende Linie durch 11 nach 10 fort: als Theil der Grenzlinie einer Fläche, worin \(T\) negativ ist, würde die Linie 1.11 nach 2 fortgesetzt werden müssen. Spricht man hingegen nur von einer Linie worin \(T=0\) ist, ohne sie als Theil der Begrenzung einer bestimmten Fläche zu betrachten, so würde eher 11.9 als natürliche Fortsetzung von 1.11 gelten können. Der hier gewählte Gesichtspunkt unterscheidet mein gegenwärtiges Verfahren von dem von 1799, und trägt'wesentlich zur Vereinfachung der Beweisführung bei.
}
repräsentirende Punkt \(P\) in zweien oder mehrern Linien \(\boldsymbol{T}=0\) zugleich liegt, das Quadrat von \(x-p\) oder die der Anzahl jener concurrirenden Linien entsprechende höhere Potenz in \(\boldsymbol{X}\) als Factor enthalten sein wird. Der Beweis davon beruhet auf folgenden Sätzen.

Man führe anstatt der unbestimmten Grösse \(x\) eine andere \(z\) ein, indem man \(x=z+p\) setzt. Es gehe durch diese Substitution \(X\) in \(Z\) über, wo also \(Z\) eine Function von \(z\) von gleicher Ordnung wie \(\mathbf{X}\) von \(x\) sein wird, deren constantes Glied aber fehlt. Indem man dieselbe nach aufsteigenden Potenzen von \(\boldsymbol{z}\) ordnet, sei das niedrigste nicht verschwindende Glied

\[
=K z^{m} \text { und } Z=K z^{m}(1+\zeta)
\]

wo \(\zeta\) die Form \(L z+L^{\prime} z z+L^{\prime \prime} z^{3}+\) u.s.w. \(+\frac{1}{\bar{K}} z^{n-m}\) haben wird; endlich setze man

\[
z=s(\cos \psi+i \sin \psi)
\]

Der reelle und der imaginäre Bestandtheil von \(z\) drücken die Lage jedes unbestimmten Punkts der Ebene als rechtwinklige Coordinaten, und die Grössen \(s, \psi\) die Polarcoordinaten ganz eben so relativ gegen den Punkt \(P\) aus, wie die Bestandtheile von \(x\), und die Grössen \(r, \varphi\) die relative Lage gegen den ursprünglichen Anfangspunkt bezeichnen. Die Verbindung eines bestimmten Werthes von \(s\) mit allen Werthen von \(\psi\) in einer Ausdehnung von \(360^{\circ}\) stellt also die Punkte einer Kreislinie dar, die ihren Mittelpunkt in \(P\) hat und deren Halbmesser \(=s\) ist.

Setzt man nun \(K=k(\cos x+i \sin \%)\), und folglich

\[
K z^{m}=k s^{m}(\cos (m \Downarrow+x)+i \sin (m \Downarrow+x))
\]

so wird für ein unendlich kleines \(s\) die Grösse \(\zeta\), die wenigstens von derselben Ordnung ist wie \(s\), neben der 1 vernachlässigt, und mithin gesetzt werden dürfen

\[
T=k s^{m} \cos \left(m \psi+x_{)}\right.
\]

woraus erhellet, dass während \(\oint\) um \(360^{\circ}\) wächst, das Zeichen von \(T\) in \(m\) Stücken der Kreisperipherie positiv, und in eben so vielen mit jenen abwechselnden negativ ist, oder dass \(T\) in \(2 m\) Punkten \(=0\) wird, nemlich für \(\downarrow=\frac{1}{m}\left(x-90^{\circ}\right), \frac{1}{m}\left(x+90^{\circ}\right), \frac{1}{m}\left(x+270^{\circ}\right)\) u.s.w. Es gehen demnach von \(\boldsymbol{P} \mathrm{zu}-\)
sammen \(2 m\) Linien aus, in denen \(T=0\) ist, oder wenn man sie paarweise so verbindet, dass jede, wo, bei wachsendem \(\psi\), das Zeichen aus - in + übergeht, zusammen mit der nächstfolgenden, wo der entgegengesetzte Uebergang Statt findet, wie die Begrenzungslinie eines Flächentheils mit positivem \(T\) betrachtet wird, so treffen in \(P\) überhaupt \(m\) dergleichen Begrenzungslinien zusammen.

Von der andern Seite ist klar, dass so wie \(\boldsymbol{Z}\) unbestimmt durch \(z^{m}\) und durch keine höhere Potenz von \(z\) theilbar ist, \(X\) den Factor \((x-p)^{m}\), aber keine höhere Potenz von \(x-p\) enthalten wird. Es ist also allemal, wenn \(p\) irgend eine Wurzel der Gleichung \(\boldsymbol{X}=0\) bedeutet, der Exponent der höchsten Potenz von \(x-p\), durch welche \(\mathbf{X}\) theilbar ist, der Anzahl der in \(P\) zusammentreffenden Begrenzungslinien für Flächen mit positivem \(T\) gleich, oder was dasselbe ist, der Anzahl solcher an \(P\) zusammentreffender Flächen.

Uebrigens ist es leicht, der Beweisführung eine von Einmischung unendlich kleiner Grössen ganz unabhängige Einkleidung zu geben, und zwar ganz analog der Schlussreihe in den Art. 3 und 4. Es lässt sich nemlich ein Werth von \(s\) nachweisen, für welchen, so wie für jeden kleinern, der ganze Cyklus aller Werthe von \(\psi\) dieselbe abwechselnde Folge von \(m\) Stücken mit positivem \(T\) und ebensovielen mit negativem darbietet. Diese Eigenschaft hat die positive Wurzel der Gleichung

\[
0=m \sqrt{ } \frac{1}{2}-(m+1) l s-(m+2) l^{\prime} s s-(m+3) l^{\prime \prime} s^{3}-\text { u.s.w. }
\]

wo \(l, l^{\prime}, l^{\prime \prime}\) u.s.w. die positiven Quadratwurzeln aus den Normen dér complexen Grössen \(L, L^{\prime}, L^{\prime \prime}\) u.s.w. bedeuten, oder wo

\[
\begin{aligned}
& L=l(\cos \lambda+i \sin \lambda) \\
& L^{\prime}=l^{\prime}\left(\cos \lambda^{\prime}+i \sin \lambda^{\prime}\right) \\
& L^{\prime \prime}=l^{\prime \prime}\left(\cos \lambda^{\prime \prime}+i \sin \lambda^{\prime \prime}\right) \text { u. s. w. }
\end{aligned}
\]

gesetzt ist. Ich glaube jedoch, die sehr leichte Entwicklung dieses Satzes hier übergehen zu können.

Schliesslich mag noch bemerkt werden, dass bei der Beweisführung in der Abhandlung von 1799 die Betrachtung zweier Systeme von Linien erforderlich war, das eine die Linien wo \(T=0\), das andere diejenigen wo \(U=0^{\prime}\) enthaltend, während in unserm jetzigen Verfahren die Betrachtung Eines Systems aus-
gereicht hat; ich habe dazu das System der Begrenzungslinien der Flächentheile mit positivem \(T\) gewählt, es hätte aber eben so gut zu demselben Zweck die Betrachtung der Begrenzungslinien der Flächen mit positivem (oder negativem). \(U\) dienen können.

ZWEITE ABTHEILUNG.

11.

Zur numerischen Bestimmung der Wurzeln solcher algebraischen Gleichungen, die nur aus drei Gliedern bestehen, lassen sich verschiedene Methoden anwenden, die hier einer Eleganz und Bequemlichkeit fähig werden, gegen welche die mühsamen bei Gleichungen von weniger einfacher Gestalt unvermeidlichen Operationen weit zurückstehen. Solche Methoden verdienen also wohl eine besondere Darstellung, zumal da Gleichungen von jener Form häufig genug vorkommen.

Es gilt dies zunächst won der Entwicklung der Wurzeln in unendliche Reihen. In der That lässt sich jede, gleichviel ob reelle oder imaginäre, Wurzel einer Gleichung mit drei Gliedern durch eine convergente Reihe von einfachem Fortschreitungsgesetz ausdrücken. Ich werde jedoch diese Auflösungsart aus mehrern Gründen von meiner gegenwärtigen Betrachtung ganz ausschliessen, und bemerke hier nur, dass der Grad der Convergenz von dem gegenseitigen Verhalten der Coëfficienten abhängig, dass sie desto langsamer ist, je näher dies Verhalten demjenigen kommt, bei welchem die Gleichung zwei gleiche Wurzeln hat, und dass in diesem Grenzfalle selbst sie schwächer ist, als bei irgendwelcher fallenden geometrischen Progression. So bemerkenswerth auch diese Reihen in allgemeiner theoretischer Rücksicht sind, so wird man doch, abgesehen von dem Falle, wo ihre Convergenz eine sehr schnelle wird, in praktischer Beziehung immer den indirecten Methoden den Vorzug geben, welche in den nachfolgenden Artikeln entwickelt werden sollen.

12.

\begin{center}
%\includegraphics[max width=\textwidth]{2024_01_11_75975a03bcf8b0416cd0g-081}
\end{center}

Zur Auffindung der reellen Wurzeln benutze ich meine im Jahre 1810 zuerst gedruckte Hülfstafel für Logarithmen von Summen und Differenzen, oder, wo eine grössere Genauigkeit verlangt wird, als Logarithmen. mit fünf Zifern geben
können, die ähnliche aber erweiterte Tafel von MatTHESSEN. Ich habe ein paar specielle Anwendungen dieses Verfahrens schon früher bekannt gemacht, nemlich zur Auflösung der quadratischen Gleichungen bei der 1840 erschienenen zwanzigsten Ausgabe von VEGA's logarithmischem Handbuch, und zur Auflösung der cubischen Gleichung; welche bei der parabolischen Bewegung zur Bestimmung der wahren Anomalie dient, in Nro. 474 der Astronomischen Nachrichten. An letzterm Orte ist auch bereits die allgemeine Anwendbarkeit des Verfahrens auf alle algebraischen Gleichungen mit drei Gliedern bemerklich gemacht. Obgleich nun die Ausführung dieses ganz elementarischen Gegenstandes gar keine Schwierigkeiten hat, so wird man doch, bei der ziemlich grossen Mannigfaltigkeit der Fälle, einer übersichtlichen Sonderung derselben, und der Zusammenstellung der gebrauchfertigen Vorschriften ein paar Seiten gern eingeräumt sehen.

Anstatt jener logarithmischen Hülfstafeln kann man sich auch der gewöhnlichen logarithmisch-trigonometrischen Tafeln bedienen: allein theils sind jene im Allgemeinen für den gegenwärtigen Zweck von bequemerm Gebrauch, theils gewähren sie doppelt so grosse Genauigkeit als die letztern. Ich würde daher die Benutzung der trigonometrischen Tafeln für das in Rede stehende Geschäft auf den seltenen Fall beschränken, wo man die durch siebenzifrige Logarithmen erreichbare Genauigkeit noch zu überschreiten wünscht und dazu die bekannten zehnzifrigen Logarithmen in VLACQ's oder VEGA's Thesaurus verwenden kann. Uebrigens sind, wenn man sich der Hülfslogarithmen bedient, doppelt so viele Fälle zu unterscheiden, als wenn die trigonometrischen Logarithmen gebraucht werden. Als ein Nachtheil darf dies jedoch nicht angesehen werden: denn wern einmal die vollständige allgemeine Classification vorliegt, ist es leicht, jedem concreten Falle sein Fach anzuweisen, und das eigentliche indirecte Geschäft ist so viel leichter auszuführen, wenn das ganze Fach nur den halben Umfang hat. Aber gerade aus jenem Grunde ist für die Auflösung durch trigonometrische Logarithmen die allgemeine Classification kürzer und bequemer darzustellen, und ich werde sie daher vorausschicken, da sodann die Classification für die andere Auflösungsform sich daraus von selbst ergibt.

13.

Die Ausführung der Methode wird, unmittelbar, nur auf Bestimmung der positiven Wurzeln einer vorgegebenen Gleichung gerichtet; die negativen ergeben
sich, indem man dasselbe Verfahren auf diejenige Gleichung anwendet, welche aus jener durch Einführung der der ursprünglichen Unbekannten entgegengesetzten Grösse entsteht.

Die Gleichung setze ich in die Form

\[
x^{m+n} \pm e x^{m} \pm f=0
\]

wo \(m, n, e, f\) gegebene positive Grössen bedeuten. Diese Form umfasst eigentlich, nach Verschiedenheit der Combination der Zeichen, vier verschiedene Fälle, wovon aber der erste, wo beidemal die oberen Zeichen gewählt werden, ausfällt, da offenbar die Gleichung

\[
x^{m+n}+e x^{m}+f=0
\]

keine positive Wurzel haben kann. Uebrigens ist verstattet, vorauszusetzen, dass \(m\) und \(n\) (worunter ganze Zahlen verstanden werden, obwohl die Anwendbarkeit der Methode an sich davon unabhängig ist) keinen gemeinschaftlichen Divisor haben, indem auf diesen Fall jeder andere leicht zurückzuführen ist. Endlich werde ich zur Abkürzung schreiben

\[
\begin{gathered}
\frac{f^{n}}{e^{m+n}}=\lambda \\
\text { Erste Form. } \\
x^{m+n}+e x^{m}-f=0
\end{gathered}
\]

Indem man einen immer im ersten Quadranten zu nehmenden Winkel \(\theta\) einführt, so dass

\[
\frac{x^{m+n}}{f}=\sin \theta^{2}, \quad \frac{e x^{m}}{f}=\cos \theta^{2}
\]

wird, also (I)

\[
x^{m+n}=f \sin \theta^{2}, \quad x^{m}=\frac{f \cos \theta^{2}}{e}, \quad x^{n}=e \operatorname{tang} \theta^{2}
\]

findet sich durch Elimination von \(x\) die Gleichung

\[
\lambda=\frac{\sin \theta^{2 m}}{\cos \theta^{2 m+2 n}}
\]

aus welcher \(\theta\) bestimmt werden muss. Man erkennt leicht, dass der zweite Theil dieser Gleichung als Function einer unbestimmten Grösse \(\theta\) betrachtet, von 0
bis \(\infty\) wächst, während \(\theta\) alle Werthe von 0 bis \(90^{\circ}\) durchläuft, und dass es also einen, und nur einen Werth von \(\theta\) gibt, der jener Gleichung Genüge leistet. Nachdem derselbe gefunden ist, erhält man \(x\) aus einer der Formeln I. Man bemerke, dass \(\theta=45^{0}\) wird für \(\lambda=2^{n}\), und dass folglich \(\theta\) im ersten Octanten zu suchen ist wenn \(\lambda\) kleiner, im zweiten wenn \(\lambda\) grösser ist als \(2^{n}\).

\[
\begin{gathered}
\text { Zweite Form. } \\
x^{m+n}-e x^{m}-f=0
\end{gathered}
\]

Man wird hier setzen

\[
f x^{-m-n}=\sin \theta^{2}, \quad e x^{-n}=\cos \theta^{2}
\]

oder (I)

\[
x^{m+n}=\frac{f}{\sin \theta^{2}}, \quad x^{n}=\frac{e}{\cos \theta^{2}}, \quad x^{m}=\frac{f \operatorname{cotang} \theta^{2}}{e}
\]

wonach also \(\theta\) aus der Gleichung

\[
\lambda=\frac{\sin \theta^{2 n}}{\cos \theta^{2 m+2 n}}
\]

zu bestimmen sein wird, was auf eine und nur auf eine Art geschehen kann: der Werth von \(x\) findet sich sodann durch eine der Gleichungen I. Im ersten oder zweiten Octanten liegt \(\theta\), jenachdem \(\lambda\) kleiner oder grösser ist als \(2^{m}\).

\[
\begin{gathered}
\text { Dritte Form. } \\
x^{m+n}-e x^{m}+f=0
\end{gathered}
\]

Hier wird man setzen

oder (I)

\[
\frac{x^{n}}{e}=\sin \theta^{2}, \quad \frac{f x^{-m}}{e}=\cos \theta^{2}
\]

\[
x^{m+n}=f \operatorname{tang} \theta^{2}, \quad x^{m}=\frac{f}{e \cos \theta^{2}}, \quad x^{n}=e \sin \theta^{2}
\]

von welchen Formeln eine zur Bestimmung von \(x\) dienen wird, sobald der Werth von \(\theta\) gefunden ist. Dieser ergibt sich durch Auflösung der Gleichung

\[
\lambda=\cos \theta^{2 n} \sin \theta^{2 m}
\]

Da das auf der rechten Seite stehende Glied dieser Gleichung, als Function einer unbestimmten Grösse \(\theta\) betrachtet, sowohl für \(\theta=0\) als für \(\theta=90^{\circ}\) verschwindet, so muss dazwischen ein grösster Werth liegen, und da das Differential des

Logarithmen dieser Function \(=(2 m \operatorname{cotg} \theta-2 n \operatorname{tang} \theta) d \theta\) ist, so findet der grösste Werth Statt für \(\theta=\theta^{*}\), wenn man \(\sqrt{ } \frac{m}{n}=\operatorname{tang} \theta^{*}\) setzt. Es wird demnach jene Function von 0 bis zu ihrem grössten Werthe, welcher offenbar

\[
=\frac{m^{m} n^{n}}{(m+n)^{m+n}}
\]

ist, zunehmen, und von da bis 0 abnehmen, während \(\theta\) von \(0 \mathrm{zu} \theta^{*}\) und von da bis \(90^{\circ}\) zunimmt. Der Maximumwerth ist daher jedenfalls grösser als der Werth für \(\theta=45^{0}\), d. i. grösser als \(\frac{1}{2^{m+n}}\), den Fall ausgenommen, wo \(m=n\), und also \(\frac{1}{2^{m+n}}\) selbst der Maximumwerth ist.

Man schliesst hieraus, dass jenachdem \(\lambda\) grösser ist als

\[
\frac{m^{m} n^{n}}{(m+n)^{m+n}}
\]

oder kleiner, der Gleichung \(\lambda=\cos \theta^{2 n} \sin \theta^{2 m}\) gar nicht oder durch zwei verschiedene Werthe von \(\theta\) wird Genüge geleistet werden können. Im erstern Falle hat die Gleichung \(x^{m+n}-e x^{m}+f=0\) gar keine (positive) Wurzel, im andern zwei. In dem speciellen Falle, wo

\[
\lambda=\frac{m^{m} n^{n}}{(m+n)^{m+n}}
\]

ist, fallen beide Auflösungen zusammen, und die Gleichung hat zwei gleiche Wurzeln, wofür man nach Gefallen eine der drei Formeln benutzen kann

\[
x^{m+n}=\frac{f m}{n}, \quad x^{m}=\frac{f(m+n)}{e n}, \quad x^{n}=\frac{e m}{m+n}
\]

Was übrigens in dem Falle, wo zwei Auflösungen wirklich vorhanden sind, die Octanten betrifft, in welche die Werthe von \(\theta\) fallen, so sieht man leicht, dass wenn \(\lambda\) grösser ist als \(\frac{1}{2^{n+n}}\), beide Werthe von \(\theta\) mit \(\theta^{*}\) in demselben Octanten liegen, nemlich im ersten oder zweiten, jenachdem \(m\) kleiner oder grösser ist als \(n\) : ist hingegen \(\lambda\) kleiner als \(\frac{1}{2^{m+n}}\), so wird der eine Werth von \(\theta\) im ersten, der andere im zweiten Octanten zu suchen sein. In dem speciellen Falle, wo \(\lambda=\frac{1}{2^{m+n}}\), ist \(45^{0}\) selbst der eine Werth von \(\theta\), und der andere liegt in demselben Octanten wie \(\theta^{*}\).

Es mag noch die aus dieser Zergliederung aller drei Formen sich leicht ergebende Folge bemerkt werden, dass unsere Gleichung (insofern wir annehmen, dass \(m\) und \(n\) keinen gemeinschaftlichen Divisor haben) nicht mehr als drei reelle Wurzeln haben kann, was auch aus andern Gründen bekannt ist.

14.

Die vorstehenden Vorschriften werden nun leicht in diejenigen umgeschmolzen, die der Anwendung der Hülfslogarithmen entsprechen, da diese \(A=\log a\). \(B=\log b, C=\log c\), betrachtet werden können wie die Logarithmen der Quadrate der Tangenten, Cosecanten und Secanten der von \(45^{\circ}\) bis \(90^{\circ}\) zunehmenden, oder, was dasselbe ist, wie die Logarithmen der Quadrate der Cotangenten, Secanten und Cosecanten der von \(45^{\circ}\) bis 0 abnehmenden Winkel, also

\[
a=\operatorname{tang} \theta^{2}, \quad \frac{1}{b}=\sin \theta^{2}, \quad \frac{1}{c}=\cos \theta^{2}
\]

für die Werthe von \(\theta \mathrm{im}\) zweiten Octanten, oder

\[
\frac{1}{a}=\operatorname{tang} \theta^{2}, \quad \frac{1}{c}=\sin \theta^{2}, \quad \frac{1}{b}=\cos \theta^{2}
\]

für die Werthe von \(\theta\) im ersten Octanten.

Die vollständigen Vorschriften vereinige ich in folgendem Schema, wo eben so wie oben

\[
\lambda=\frac{f^{n}}{e^{m+n}}
\]

gesetzt ist.

\[
\begin{gathered}
\text { Erste Form. } \\
x^{m+n}+e x^{m}-f=0
\end{gathered}
\]

Erster Fall. \(\lambda>2^{n}\)

\[
\begin{gathered}
\lambda=a^{m+n} b^{n}=a^{m} c^{n}=\frac{c^{m+n}}{b^{n+}} \\
x^{m+n}=\frac{f}{b}, \quad x^{m}=\frac{f}{e c}, \quad x^{n}=e a
\end{gathered}
\]

Zweiter Fall. \(\lambda<2^{n}\)

\[
\begin{gathered}
\lambda=\frac{b^{n}}{a^{m}}=\frac{c^{n}}{a^{m+n}}=\frac{b^{m+n}}{c^{m}} \\
x^{m+n}=\frac{f}{c}, \quad x^{m}=\frac{f}{e b}, \quad x^{n}=\frac{e}{a} \\
\text { Zweite Form. } \\
x^{m+n}-e x^{m}-f=0
\end{gathered}
\]

Erster Fall. \(\lambda>2^{m}\)

\[
\begin{aligned}
\lambda=a^{m+n} b^{m} & =a^{n} c^{m}=\frac{e^{m+n}}{b^{n}} \\
x^{m+n}=f b, \quad x^{m} & =\frac{f}{e a}, \quad x^{n}=e c
\end{aligned}
\]

Zweiter Fall. \(\quad \lambda<2^{m}\)

\[
\begin{gathered}
\lambda=\frac{b^{m}}{a^{n}}=\frac{e^{m}}{a^{m+n}}=\frac{b^{m+n}}{c^{n}} \\
x^{m+n}=f c, \quad x^{m}=\frac{f a}{e}, \quad x^{n}=e b \\
\text { Dritte Form. } \\
x^{m+n}-e x^{m}+f=0
\end{gathered}
\]

Erster Fall. \(\quad \frac{1}{\lambda}<\frac{(m+n)^{m+n}}{m^{m} n^{n}}\)

Gar keine Auflösung.

Zweiter Fall. \(\quad \frac{1}{\lambda}=\frac{(m+n)^{m+n}}{m^{m} n^{n}}\)

Zwei gleiche Wurzeln, zu deren Bestimmung eine der Gleichungen

dient.

\[
x^{m+n}=\frac{f m}{n}, \quad x^{m}=\frac{f(m+n)}{e n}, \quad x^{n}=\frac{e m}{m+n}
\]

Dritter Fall. \(\quad \frac{1}{\lambda}\) grösser als \(\frac{(m+n)^{m+n}}{m^{m} n^{n}}\) aber nicht grösser als \(2^{m+n}\), und zugleich \(m\) grösser als \(n\).

Zwei Wurzeln, für welche

\[
\begin{gathered}
\frac{1}{\lambda}=a^{n} b^{m+n}=\frac{c^{m_{2}+n}}{a^{m}}=b^{m} c^{n} \\
x^{m+n}=f a, \quad x^{m}=\frac{f c}{e}, \quad x^{n}=\frac{e}{b}
\end{gathered}
\]

Vierter Fall. Für \(\frac{1}{\lambda}\) dieselben Grenzen, wie im dritten Fall, aber \(m\) kleiner als \(n\).

Zwei Wurzeln, für welche

\[
\begin{gathered}
\frac{1}{\lambda}=a^{m} b^{m+n}=\frac{c^{m+n}}{a^{n}}=b^{n} c^{m} \\
x^{m+n}=\frac{f}{a}, \quad x^{m}=\frac{f b}{e}, \quad x^{n}=\frac{e}{c}
\end{gathered}
\]

Fünfter Fall. \(\frac{1}{\lambda}\) grösser als \(2^{m+n}\).

Zwei Wurzeln, wovon die eine durch die Formeln des dritten Falles, die andere durch die des vierten bestimmt wird.

Es mag noch bemerkt werden, dass im dritten Falle der Werth von \(a\), welcher der einen Wurzel entspricht, kleiner als \(\frac{m}{n}\), der zur andern Wurzel gehörende grösser als \(\frac{m}{n}\) ist; im vierten Falle verhalten sich die beiden Werthe von \(a\) auf ähnliche Weise gegen \(\frac{n}{m}\).

\section*{15.}
Ueber die Anwendung dieser Vorschriften ist noch folgendes beizufügen.

Zur Bestimmung jeder Wurzel sind zwei Operationen auszuführen : zuerst aus \(\lambda\) den dazu gehörenden Werth von \(a\) (und damit zugleich den von \(b\) oder \(c\) ) abzuleiten; sodann, aus diesem.den Werth von \(x\) zu berechnen. Für jede dieser beiden Operationen kann man unter drei Formeln wählen; ich ziehe in den meisten Fällen die zuerst angesetzten vor. Bei allen diesen Rechnungen hat man es gar nicht mit den Grössen \(\lambda, a, b, c\) selbst, sondern nur mit ihren Logarithmen zu thun. Die erste Operation ist eine indirecte, und beruhet demnach in der Regel auf mehrern stufenweise fortschreitenden Annäherungen, wobei es bequem gefunden werden wird, zu Anfang Tafeln mit einer geringern Anzahl von Zifern zu gebrauchen. Matthiessens Tafel hat bekanntlich sieben Decimalen: die meinige fünf; EvCKE und Ursir haben sie mit vier Zifern abdrucken lassen, und wenn man beim Anfange der Arbeit noch gar keine Kenntniss einer ersten groben Annäherung mitbringt, wird man es vielleicht vortheilhaft finden, einen noch kürzern Extract der Tafeln mit nur drei Zifern auf einem besondern Blättchen vor sich zu haben, etwa so:

\begin{center}
\begin{tabular}{l|c|c|c|c|c}
\(A\) & \(B\) & \(A\) & \(B\) & \(A\) & \(B\) \\
\hline
0 & 0,301 & 1,0 & 0,041 & 2,0 & 0,004 \\
0,1 & 0,254 & 1,1 & 0,033 & 2,1 & 0,003 \\
0,2 & 0,212 & 1,2 & 0,027 & 2,2 & 0,003 \\
0,3 & 0,176 & 1,3 & 0,021 & 2,3 & 0,002 \\
0,4 & 0,146 & 1,4 & 0,017 & 2,4 & 0,002 \\
0,5 & 0,119 & 1,5 & 0,014 & 2,5 & 0,001 \\
0,6 & 0,097 & 1,6 & 0,011 & 2,9 & 0,001 \\
0,7 & 0,079 & 1,7 & 0,009 & 3,0 & 0,000 \\
0,8 & 0,064 & 1,8 & 0,007 &  &  \\
0,9 & 0,051 & 1,9 & 0,005 &  &  \\
\end{tabular}
\end{center}

16.

Als Beispiel mag die Gleichung

\[
x^{7}+28 x^{4}-480=0
\]

dienen, wo

\[
\lambda=\frac{6750}{\$ 23543}, \quad \log \frac{1}{\lambda}=2,0863825
\]

wird. Die Gleichung hat die erste Form, mithin eine positive Wurzel, und gehört, da \(\lambda\) kleiner ist als 8, zum zweiten Fall. Die erste Operation besteht darin, dass der Gleichung \(\log \frac{1}{\lambda}=4 A-3 B\) Genüge geschehe, also, wenn man die Rechnung mit drei Decimalen anfängt, dieser

\[
2,086=4 A-3 B
\]

Ein flüchtiger Blick auf obige Tafel zeigt schon, dass \(A\) zwischen 0,5 und 0,6 zu suchen sei. Es wird nemlich

\begin{center}
\begin{tabular}{c|c|c}
\(A\) & \(4 A-3 B\) & Fehler \\
\hline
0,5 & 1,643 & \(-0,443\) \\
0,6 & 2,109 & \(+0,023\) \\
\end{tabular}
\end{center}

woraus sich auf einen genauern Werth 0,595 schliessen lässt. Eine neue Rechnung nach den Tafeln mit fünf Decimalen, wo also \(\log \frac{1}{\lambda}=2,08638\) zu setzen ist, gibt

\begin{center}
\begin{tabular}{c|c|c}
\(A\) & \(4 A-3 B\) & Fehler \\
\hline
0,595 & 2,08501 & \(-0,00137\) \\
0,596 & 2,08961 & \(+0,00323\) \\
\end{tabular}
\end{center}

woraus der noch genauere Werth 0,5953 erkannt wird. Endlich für sieben Decimalen hat man

\begin{center}
\begin{tabular}{c|c|c}
\(A\) & \(4 A-3 B\) & Fehler \\
\hline
0,5952 & 2,0859279 & \(-0,0004546\) \\
0,5953 & 2,0863885 & \(+0,0000060\) \\
\end{tabular}
\end{center}

Zu dem Werthe \(A=0,5953\) muss also noch die Correction \(-\frac{60}{4606}\) Einheiten der vierten Decimale hinzukommen, in welcher Form ich sie beibehalte, da es, wenn zur Bestimmung von \(x\) die erste Formel

\[
x^{7}=\frac{f}{e}
\]

gebraucht werden soll, nur darauf ankommt, den entsprechenden Werth von \(C\) zu finden. Diesen erhält man, indem man zu dem neben \(A=0,5953\) stehenden Werthe \(C=0,6935705\) die Correction \(-\frac{60}{4606} \times 798\) hinzufügt, letztere wie Einheiten der siebenten Decimale betrachtet, also

\[
\begin{aligned}
C & =0,6935695 \\
\log f & =2,6812412 \\
\hline 7 \log x & =1,9876717 \\
\log x & =0,2839531 \\
x & =1,9228841
\end{aligned}
\]

Zur Auffindung der negativen Wurzeln wird man \(x=-y\) schreiben und die positiven Wurzeln der Gleichung

\[
y^{7}-28 y^{4}+480=0
\]

aufsuchen. Diese gehört zur dritten Form, und da \(\frac{1}{\lambda}=\frac{\$ 23543}{6750}\) grösser ist als \(\frac{7^{7}}{3^{3} 4^{4}}=\frac{\$ 23543}{6912}\), aber kleiner als \(2^{7}=128\), zugleich auch \(m\) grösser ist als \(n\), so gilt der dritte Fall, oder es finden zwei Wurzeln Statt, zu deren Ausmittlung der Gleichung

\[
2,0863825=3 A+7 B
\]

genügt werden muss. Aus der Schlussbemerkung des 14. Art. weiss man, dass der eine Werth von \(A\) kleiner, der andere grösser sein muss als \(\log _{\frac{4}{3}}=0,12494\). Auch ergeben sich die Grenzen der Werthe von \(A\) sofort aus der obigen Tafel mit dreizifrigen Logarithmen, nach welchen man erhält:

\begin{center}
\begin{tabular}{c|c|c}
\(A\) & \(3 A+7 B\) & Fehler \\
\hline
0,0 & 2,107 & \(+0,021\) \\
0,1 & 2,078 & \(-0,008\) \\
0,2 & 2,084 & -0.002 \\
0,3 & 2,132 & \(+0,046\) \\
\end{tabular}
\end{center}

Will man zur nähern Bestimmung zuerst vierzifrige Logarithmen gebrauchen, so hat man zunächst für die erste Auflösung

\begin{center}
\begin{tabular}{c|c|c}
\(A\) & \(3 A+7 B\) & Fehler \\
\hline
0,05 & 2,0869 & \(+0,0005\) \\
0,06 & 2,0847 & \(-0,0017\) \\
\end{tabular}
\end{center}

Sodann ergeben die fünfzifrigen Tafeln

\[
\begin{array}{l|l|c}
0,052 & 2,08667 & +0,00029 \\
0,053 & 2,08638 & 0
\end{array}
\]

Endlich die siebenzifrigen

Hienach wird

\[
\begin{array}{l|l|l}
0,0529 & 2,0863943 & +0,0000118 \\
0,0530 & 2,0863660 & -0,0000165
\end{array}
\]

\[
\begin{aligned}
A & =0,0529417 \\
\log f & =2,6812412 \\
7 \log y & =2,7341829 \\
\log y & =0,3905976 \\
-y & =x=-2,4580892
\end{aligned}
\]

Für die zweite Auflösung steht die Rechnung, auf ähnliche Weise geführt, folgendermaassen :

\[
\begin{array}{c|c|c}
A & 3 A+7 B & \text { Fehler } \\
\hline 0,19 & 2,0843 & -0,0021 \\
0,20 & 2,0868 & +0,0004 \\
0,197 & 2,08627 & -0,00011 \\
0,198 & 2,08654 & +0,00016 \\
0,1975 & 2,0863805 & -0,0000020 \\
0,1976 & 2,0864082 & +0,0000257 \\
A=0,1975072 \\
\log f=2,6812412 \\
\hline 7 \log y=2,8787484 \\
\log y=0,4112498 \\
x=-2,5778036
\end{array}
\]

-Die Gleichung, welche uns hier als Beispiel gedient hat, ist absichtlich so gewählt, dass zwei ihrer Wurzeln wenig verschieden sind. In einem solchen Falle sind, wie schon oben im Art. 11 bemerkt ist, die Reihen wegen ihrer sehr langsamen Convergenz wenig brauchbar: auch bei der indirecten Auflösung ist davon wenigstens eine schwache Analogie erkennbar, indem das Fortschreiten der successiven Annäherungen bei den beiden negativen Wurzeln (welche eben die wenig ungleichen sind) etwas träger ist, als bei der positiven. Ein wesentlicher Unterschied ist aber der, dass die sehr langsame Convergenz der Reihen für sämmtliche Wurzeln eintritt, während bei dem indirecten Verfahren die, auch nur in geringem Grade fühlbare, langsamere Annäherung lediglich bei den zwei wenig verschiedenen Wurzeln vorkommt.

17.

Ganz verschieden von dem in den vorhergehenden Artikeln gelehrten Verfahren ist dasjenige, welches zur Bestimmung der imaginären Wurzeln angewandt werden muss. Im Allgemeinen ist die Bestimmung der imaginären Wurzeln auf indirectem Wege deswegen weit schwieriger, als die der reellen, weil jene aus einem unendlichen Gebiet von zwei Dimensionen herausgesucht werden müssen, diese nur aus einem Unendlichen von Einer Dimension, und gerade darum verdient ein sehr umfassender besonderer Fall, wo man jene Schwierigkeit umgehen und die Frage in dasselbe Gebiet versetzen kann, zu welchem die Aufsuchung der reellen Wurzeln gehört, eine eigne Ausführung. Einen solchen Fall bieten die Gleichungen mit drei Gliedern dar.

Da die Methode mit gleicher Leichtigkeit angewandt werden kann, die Coëfficienten der Gleichung mögen reell oder imaginär sein, so lege ich sofort die allgemeine Form der Gleichung zum Grunde

\[
\mathrm{X}=x^{m+n}+e(\cos \varepsilon+i \sin \varepsilon) x^{m}+f(\cos \varphi+i \sin \varphi)=0
\]

wo \(\epsilon\) und \(f\) positive Grössen bedeuten : für einen reellen, positiven oder negativen, Coëfficienten ist dann der betreffende Winkel ( \(\varepsilon\) oder \(\varphi\) ) entweder 0 oder \(180^{\circ}\). Die Voraussetzung, dass \(m\) und \(n\) keinen gemeinschaftlichen Divisor haben, wird ohne Beeinträchtigung der Allgemeinheit auch hier beibehalten bleiben können. Eine der Gleichung Genüge leistende imaginäre Wurzel \(x=t+i u\) setzt man in die Form \(r(\cos \rho+i \sin \rho)\), wobei es für unsern gegenwärtigen Zweck
vortheilhafter ist, die sonst gewöhnliche Bedingung, dass \(r\) positiv sein soll, hier nicht zu machen, sondern anstatt derselben die, dass \(\rho\) immer zwischen den Grenzen 0 und \(180^{\circ}\) genommen werden soll. In dem Fall, wo die Coëfficienten der Gleichung beide reell sind, kann man den Umfang der Werthe von \(\rho\) noch weiter auf die Hälfte verengen: denn da bekanntlich von den imaginären Wurzeln einer solchen Gleichung je zwei zusammengehören, wie \(t+i u\) und \(t-i u\), so wird offenbar für die eine Wurzel jedes Paars der Werth von \(\rho\) zwischen 0 und \(90^{\circ}\) fallen, und man braucht durch das indirecte Verfahren nur diese zu bestimmen, indem daraus die andere von selbst folgt durch Vertauschung von \(\rho\) mit \(180^{\circ}-\rho\) und von \(r\) mit \(-r\).

18.

Das Wesen der Methode besteht in der Aufstellung einer Gleichung, welche bloss \(\rho\) ohne \(r\) enthält. Um dazu zu gelangen, setze man die Gleichung \(X=0\) durch Division mit ihrem ersten Gliede in die Form

\[
1+e(\cos \varepsilon+i \sin \varepsilon) x^{-n}+f(\cos \varphi+i \sin \varphi) x^{-m-n}=0
\]

oder

\(1+e r^{-n}(\cos (n \rho-\varepsilon)-i \sin (n \rho-\varepsilon))+f r^{-m-n}(\cos [(m+n) \rho-\varphi]-i \sin [(m+n) \rho-\varphi])=0\)

Da nun hier die imaginären Theile einander aufheben müssen, so hat man (I)

\[
r^{m}=-\frac{f \sin ((m+n) \rho-\varphi)}{e \sin (n \rho-\varepsilon)}
\]

Auf ähnliche Art erhält man, wenn die Gleichung \(\boldsymbol{X}=0\) mit ihrem zweiten oder dritten Gliede dividirt, und erwägt, dass in beiden Fällen die imaginären Theile der neuen Gleichungen einander aufheben müssen, die Gleichungen

\[
\left.\begin{array}{l}
r^{m+n}=\frac{f \sin (m \rho+\varepsilon-\varphi)}{\sin (n \rho-\varepsilon)} \\
r^{n}=-\frac{e \sin (m \rho+\varepsilon-\varphi)}{\sin ((m+n) \rho-\varphi)}
\end{array}\right\}
\]

Man sieht, dass jede der drei Gleichungen (I) auch schon aus der Verbindung der beiden andern abgeleitet werden kann. Eliminirt man aber \(r\) aus Verbindung zweier, so erhält man (II)

\[
\lambda=(-1)^{m+n} \frac{\sin (m \rho+\varepsilon-\varphi)^{m} \sin (n \rho-\varepsilon)^{n}}{\sin ((m+n) \rho-\varphi)^{m+n}}
\]

wo zur Abkürzung (eben so wie oben)

\[
\frac{f^{n}}{e^{m+n}}=\lambda
\]

gesetzt ist. Aus dieser Gleichung hat man die verschiedenen Werthe von \(\rho \mathrm{zu}\) bestimmen; den Werth von \(r\), welcher jedem Werthe von \(\rho\) entspricht, findet man sodann aus einer der Gleichungen (I), am besten aus der zweiten, rücksichtlich der absoluten Grösse, wobei jedoch in dem Falle, wo \(m+n\) gerade ist, noch eine der beiden andern Gleichungen zur Entscheidung des Zeichens hinzugezogen werden muss.

19.

Die Auflösung der Gleichung II auf indirectem Wege wird man immer mit Leichtigkeit beschaffen können, wozu noch die Berücksichtigung der folgenden Bemerkungen beitragen wird.

\begin{enumerate}
  \item Die Werthe von \(\rho\) liegen zwischen 0 und \(180^{\circ}\); in dem Falle, wo die Coëfficienten der vorgegebenen Gleichung reell sind, braucht man nur die halbe Anzahl, nemlich die zwischen 0 und \(90^{\circ}\) liegenden. einzeln aufzusuchen.

  \item In dem einen wie in dem andern Falle wird man zuerst das betreffende Intervall in die verschiedenen Unterabtheilungen scheiden, die sich durch die Zeichenabwechslungen in den Werthen der auf der rechten Seite der Gleichung II stehenden Function von \(\rho\) bilden. Die Uebergangswerthe von \(\rho\) können offenbar nur solche sein, wo einer der Winkel \(m \rho+\varepsilon-\varphi, n \rho-\varepsilon,(m+n) \rho-\varphi\) durch \(180^{\circ}\) theilbar, und also jene Function selbst entweder 0 oder unendlich wird. Von jenen Unterabtheilungen bleiben dann diejenigen, in welchen der Werth der Function negativ wird, schon von selbst aus der weitern Untersuchung ausgeschlossen.

  \item Falls man nicht schon auf andern Wegen genäherte Werthe von \(\rho\) erlangen kann, wird man sich das indirecte Durchsuchen der geeigneten Intervalle dadurch sehr erleichtern, dass man auf ähnliche Weise, wie aus den Beispielen des 16. Artikels zu ersehen ist, die ersten Versuche nach abgekürzten Tafeln mit wenigen Zifern ausführt, und in manchen Fällen möchte man wohl bequem fin-
den, zuerst nur die Sinuslogarithmen mit drei Zifern auf einem Blättchen\(\cdot\) etwa von Grad zu Grad verzeichnet zu diesem Zweck zu verwenden.

\end{enumerate}

20.

Zu weiterer Erläuterung mag die Berechnung der imaginären Wurzeln der oben behandelten Gleichung

\[
x^{7}+28 x^{4}-480=0
\]

als Beispiel dienen. Nach der Bezeichnung des Art. 17 haben wir hier zuvörderst, wie oben, \(m=4, n=3, e=28, f=480\), und sodann weiter \(\varepsilon=0, \varphi=180^{\circ}\). Die Formeln I des Art. 18 werden demnach

\[
\begin{aligned}
& r^{4}=\frac{480 \sin 7 \rho}{28 \sin 3 \rho} \\
& r^{7}=-\frac{480 \sin 4 \rho}{\sin 3 \rho} \\
& r^{3}=-\frac{28 \sin 4 \rho}{\sin 7 \rho}
\end{aligned}
\]

und die Formel II

\[
\frac{1}{\lambda}=\frac{\$ 23543}{6750}=\frac{\sin 7 \rho^{7}}{\sin 3 \rho^{3} \sin 4 \rho^{4}}
\]

aus welcher Gleichung zwei zwischen 0 und \(90^{\circ}\) liegende Werthe von \(\rho\) zu bestimmen sind, da die Gleichung \(\boldsymbol{X}=\mathbf{0}\) neben ihren drei bereits ermittelten reellen Wurzeln noch zwei Paare zusammengehöriger imaginärer hat. Innerhalb dieser Grenzen wird \(\sin 7 \rho\) dreimal \(=0\), nemlich für \(\rho=25 \frac{5}{7}\) Grad, \(51 \frac{3}{7}\) Grad und \(77 \frac{1}{7}\) Grad, wobei \(\sin 7 \rho^{7}\) jedesmal sein Zeichen ändert; \(\sin 3 \rho\) wird einmal \(=0\) für \(\rho=60^{\circ}\) gleichfalls mit Zeichenwechsel von \(\sin 3 \rho^{3}\); endlich \(\sin 4 \rho\) wird einmal \(=0\) für \(\rho=45^{\circ}\), aber ohne Zeichenwechsel für \(\sin 4 \rho^{4}\). Erwägt man nun noch, dass der Werth von \(\frac{\sin 7 \rho^{7}}{\sin 3 \rho^{3} \sin 4 \rho^{2}}\) für \(\rho=0\) dem Grenzwerthe \(\frac{7^{7}}{3^{3} 4^{4}}\) gleich zu setzen ist, so wird das Verhalten der Werthe jener Function in den sechs Unterabtheilungen des Zwischenraumes von 0 bis \(90^{\circ}\) in folgender Uebersicht zusammengefasst:

\begin{center}
%\includegraphics[max width=\textwidth]{2024_01_11_75975a03bcf8b0416cd0g-096}
\end{center}

Man erkennt hieraus, dass sowohl im vierten als im sechsten Zwischenraume nothwendig ein der Formel II Genüge leistender Werth von \(\rho\) liegen muss, und eines Mehrern bedarf es für unsern Zweck nicht, da schon von vorne her fest steht, dass es nur zwei solche Werthe gibt. Die Gleichung II setze ich in die Form

\[
7 \log \sin 7 \rho-3 \log \sin 3 \rho-4 \log \sin 4 \rho=S=2,0863825
\]

Die Auffindung des zwischen \(51 \frac{3}{7}\) und 60 Grad liegenden Werthes durch allmählige Annäherung vermittelst der Tafeln mit 3, 4, 5, 7 Zifern zeigt folgendes Schema:

\[
\begin{array}{l|l|l}
\rho & S & \text { Fehler } \\
57^{0} & 1,527 & -0,559 \\
58 & 2,354 & +0,268 \\
57^{0} 40^{\prime} & 2,0624 & -0,0240 \\
5750 & 2,2057 & +0,1193 \\
57^{0} 41^{\prime} & 2,07658 & -0,00980 \\
5742 & 2,09074 & +0,00436 \\
57^{0} 41^{\prime} 41^{\prime \prime} & 2,0862962 & -0,0000863 \\
574142 & 2,0865320 & +0,0001495
\end{array}
\]

\begin{center}
%\includegraphics[max width=\textwidth]{2024_01_11_75975a03bcf8b0416cd0g-096(1)}
\end{center}

\begin{center}
\begin{tabular}{ll}
\(\log \sin 4 p\) & \(=9,8891425 n\) \\
Compl. \(\log \sin 3 p\) & \(=0,9193523\) \\
\(\log (-480)\) & \(=2,6812412 n\) \\
\hline
\(7 \log r\) & \(=3,4897360\) \\
\(\log r\) & \(=0,4985337\) \\
\end{tabular}
\end{center}

und damit

\[
x=+1,6843159+2,6637914 i
\]

so wie die andere dazu gehörige Wurzel

\[
x=+1,6843159-2,6637914 i
\]

Der andere zwischen \(77 \frac{1}{7}\) und \(90^{\circ}\) liegende Werth von \(\rho\) wird durch Anwendung von Tafeln mit drei Decimalen als zwischen \(86^{\circ}\) und \(87^{\circ}\) liegend erkannt. Die Rechnung in gleicher Gestalt wie im vorhergehenden Falle steht so:

\[
\begin{aligned}
& \begin{array}{c|c|c}
\rho & \boldsymbol{S} & \text { Fehler } \\
86^{\circ} \\
87 & 1,885 & -0,201 \\
2,533 & +0,447
\end{array} \\
& \begin{array}{l|l|l}
86^{\circ} 10^{\prime} & 1,9907 & -0,0957 \\
8620 & 2,0946 & +0,0082
\end{array} \\
& \begin{array}{ll|l|l}
86 & 19 & 2,08409 & -0,00229 \\
86 & 20 & 2,09447 & +0,00809
\end{array} \\
& 86^{0} 19^{\prime} 13^{\prime \prime}|\quad 2,0863229|-0,0000596 \\
& \begin{array}{l|l|l}
861914 & 2,0864970 & +0,0001145
\end{array} \\
& \rho=86^{\circ} 19^{\prime} 13^{\prime \prime} 342 \\
& \log \sin 4 \rho \quad=9,4049540 n \\
& \text { Compl. } \log \sin 3 \rho=0,0081108 n \\
& \begin{array}{ll}
\log (-480) & =2,6812412 n \\
\hline 7 \log r & =2,0943060 n
\end{array} \\
& \log r \quad=0,2991866 n
\end{aligned}
\]

Zieht man vor, \(r\) positiv zu haben, so braucht man nur zugleich für \(\rho\) den um \(180^{\circ}\) vergrösserten Werth \(266^{\circ} 19^{\prime} 13^{\prime \prime} 342\) anzusetzen. Die Wurzel selbst ist

\[
x=-0,1278113-1,9874234 i
\]

und die andere dazu gehörige nur im Zeichen des imaginären Theils davon verschieden.

Die sämmtlichen Wurzeln der Gleichung \(x^{7}+28 x^{4}-480=0\) sind demnach

\[
\begin{aligned}
& +1,9228841 \\
& -2,4580892 \\
& -2,5778036 \\
& +1,6843159+2,6637914 i \\
& +1,6843159-2,6637914 i \\
& -0,1278113+1,9874234 i \\
& -0,1278113-1,9874234 i
\end{aligned}
\]

Die Summe der Wurzeln \(+0,0000005\) ist so genau mit dem wahren Werthe 0 übereinstimmend, wie nur von dem Gebrauch siebenzifriger Logarithmen erwartet werden durfte. In der andern Form hat man

\begin{center}
\begin{tabular}{c|r}
\(\log r\) & \multicolumn{1}{|c}{\(\rho\)} \\
\cline { 2 - 2 }
0,2839531 & \multicolumn{1}{|l}{0} \\
0,3905976 & \(180^{\circ}\) \\
0,4112498 & 180 \\
0,4985337 & \(57.41^{\prime} 41^{\prime \prime} 366\) \\
0,4985337 & 3021818,634 \\
0,2991866 & 934046,658 \\
0,2991866 & 2661913,342 \\
\end{tabular}
\end{center}

Die Summe der Logarithmen der Werthe von \(r\) findet sich \(=2,6812411\), gleichfalls befriedigend genau mit dem Logarithmen von 480 übereinstimmend.

Es wird übrigens kaum nöthig sein zu erinnern, dass die in diesem so wie die im 16. Artikel aufgestellten Rechnungen nur dazu bestimmt sind, den Gang der Arbeit nach ihren Hauptmomenten zu erläutern, keinesweges aber für die Form des kleinen Mechanismus der Operationen maassgebend sein sollen. Geübtere Rechner werden meistens vorziehen, nicht so viele Zwischenstufen anzuwenden, als in jenen Beispielen geschehen ist. Ueberhaupt wird jeder in dergleichen Arbeiten einigermaassen erfahrne die Einzelnheiten des Geschäfts leicht selbst in diejenige Gestalt bringen, die den jedesmaligen Umständen und seiner eignen individuellen Gewöhnung am meisten angemessen ist, und es kann hier nicht der Ort sein, in solche Einzelnheiten weiter einzugehen.

\begin{center}
%\includegraphics[max width=\textwidth]{2024_01_11_75975a03bcf8b0416cd0g-099}
\end{center}

Gouns AVenter SBand III Shetes 102

\section*{DISQUISITIONES GENERALES}
\section*{CIRCA SERIEM INFINITAM}
\[
1+\frac{\alpha b}{1 \cdot \gamma} x+\frac{\alpha(\alpha+1) b(b+1)}{1 \cdot 2 \cdot \gamma(\gamma+1)} x x+\frac{\alpha(\alpha+1)(\alpha+2) b(b+1)(b+2)}{1 \cdot 2 \cdot 3 \cdot \gamma(\gamma+1)(\gamma+2)} x^{3}+\text { etc. }
\]

\section*{PARS PRIOR}
A U C T OR E

\section*{CAROLO FRIDERICO GAUSS}
SOCIETATI REGIAE SCIEN'IARUM TRADITA 1812. JAN. 30.

Commentationes societatis regiae scientiarum Gottingensis recentiores. Vol. II.

Gottingae MDcccxm.

\section*{DISQUISITIONES GENERALES}
\section*{CIRCA SERIEM INFINITAM \(1+\frac{a b}{1-\gamma} x+\frac{\alpha(\alpha+1) b(b+1)}{1 \cdot 2 \cdot \gamma(\gamma+1)} x x+\frac{\alpha(\alpha+1)(\alpha+2) b(b+1)(b+2)}{1 \cdot 2 \cdot 3 \cdot \gamma(\gamma+1)(\gamma+2)} x^{3}+\) etc.}
PARS PRIOR.

INTRODUCTIO.

1.

Series, quam in hac commentatione perscrutari suscipimus, tamquam functio quatuor quantitatum \(\alpha, b, \gamma, x\) spectari potest, quas ipsius elementa vocabimus, ordine suo elementum primum \(\alpha\), secundum 6 . tertium \(\gamma\), quartum \(x\) distinguentes. Manifesto elementum primum cum secundo permutare licet: quodsi itaque brevitatis caussa seriem nostram hoc signo \(F(\alpha, b, \gamma, x)\) denotamus, habebimus \(\boldsymbol{F}(\boldsymbol{b}, \alpha, \gamma, x)=\boldsymbol{F}(\alpha, \boldsymbol{b}, \gamma, x)\).

\section*{2.}
'Tribuendo elementis \(\boldsymbol{\alpha}, \boldsymbol{b}, \boldsymbol{\gamma}\) valores determinatos, series nostra in functionem unicae variabilis \(x\) transit, quae manifesto post terminum \(1-\alpha^{\text {tum }}\) vel \(1-b^{\text {tum }}\) abrumpitur, si \(a-1\) vel \(b-1\) est numerus integer negativus, in casibus reliquis vero in infinitum excurrit. In casu priori series exhibet functionem algebraicam rationalem, in posteriori autem plerumque functionem transscendentem. Elementum tertium \(\gamma\) debet esse neque numerus negativus integer neque \(=0\), ne ad terminos infinite magnos delabamur.

\section*{3.}
Coëfficientes potestatum \(x^{m}, x^{m+1}\) in serie nostra sunt ut

\[
1+\frac{\gamma+1}{m}+\frac{\gamma}{m m}: 1+\frac{a+b}{m}+\frac{\alpha \varepsilon}{m m}
\]

adeoque ad rationem aequalitatis eo magis accedunt, quo maior assumitur \(m\). Si itaque etiam elemento quarto \(x\) valor determinatus tribuitur, ab huius indole convergentia seu divergentia pendebit. Quoties scilicet ipsi \(x\) tribuitur valor realis, positivus seu negativus, unitate minor, series certo, si non statim ab initio, tamen post certum intervallum, convergens erit, atque ad summam finitam ex asse determinatam perducet. Idem eveniet per valorem imaginarium ipsius \(x\) formae \(a+b V-1\), quoties \(a a+b b<1\). Contra pro valore ipsius \(x\) reali unitateque maiori, vel pro imaginario formae \(a+b \sqrt{ }-1\), quoties \(a a+b b>1\), series si non statim tamen post certum intervallum necessario divergens erit, ita ut de ipsius summa sermo esse nequeat. Denique pro valore \(x=1\) (seu generalius pro valore formae \(a+b \sqrt{ }-1\), quoties \(a a+b b=1\) ) seriei convergentia seu divergentia ab ipsarum \(\alpha, b, \gamma\) indole pendebit, de qua, atque in specie de summa seriei pro \(x=1\), in Sect. tertia loquemur.

Patet itaque, quatenus functio nostra tamquam summa seriei definita sit, disquisitionem natura sua restrictam esse ad casus eos, ubi series revera convergat, adeoque quaestionem ineptam esse, quinam sit valor seriei pro valore ipsius \(x\) unitate maiori. Infra autem, inde a Sectione quarta, functionem nostram altiori principio superstruemus, quod applicationem generalissimam patiatur.

4.

Differentiatio seriei nostrae, considerando solum elementum quartum \(x\) tamquam variabile, ad functionem similem perducit, quum manifesto habeatur

\[
\frac{\mathrm{d} F(\alpha, \ell, \gamma, x)}{\mathrm{d} x}=\frac{\alpha b}{\gamma} F(\alpha+1, b+1, \gamma+1, x)
\]

Idem valet de differentiationibus repetitis.

5.

Operae pretium erit, quasdam functiones, quas ad seriem nostram reducere licet, quarumque usus in tota analysi est frequentissimus, hic apponere.

I.

\[
(t+u)^{n}=t^{u} F\left(-n, b, b,-\frac{u}{t}\right)
\]

ubi elementum \(b\) est arbitrarium.

II.

III.

\[
\begin{aligned}
(t+u)^{n}+(t-u)^{n} & =2 t^{n} F\left(-\frac{1}{2} n,-\frac{1}{2} n+\frac{1}{2}, \frac{1}{2}, \frac{u u}{t t}\right) \\
(t+u)^{n}+t^{n} & =2 t^{n} F\left(-n, \omega, 2 \omega,-\frac{u}{t}\right)
\end{aligned}
\]

denotante \(\omega\) quantitatem infinite parvam.

IV.

V.

VI.

\[
(t+u)^{n}-(t-u)^{n}=2 n t^{n-1} u F\left(-\frac{1}{2} n+\frac{1}{2},-\frac{1}{2} n+1, \frac{3}{2}, \frac{u u}{t t}\right)
\]

VII.

\[
(t+u)^{n}-t^{n}=n t^{n-1} u F\left(1-n, 1.2,-\frac{u}{t}\right)
\]

VIII. \(e^{t}=F\left(1, k, 1, \frac{t}{k}\right)=1+t F\left(1, k, 2, \frac{t}{k}\right)=1+t+\frac{1}{2} t t F\left(1, k, 3, \frac{t}{k}\right)\) etc.

denotante \(e\) basin logarithmorum hyperbolicorum, \(k\) numerum infinite magnum.

IX. \(\quad e^{t}+e^{-t}=2 \boldsymbol{F}\left(k, k^{\prime}, \frac{1}{2}, \frac{t t}{4 k k^{\prime}}\right)\)

denotantibus \(k, k^{\prime}\) numeros infinite magnos.

X. \(\quad e^{t}-e^{-t}=2 t F\left(k, k^{\prime}, \frac{3}{2}, \frac{t t}{4 k k^{\prime}}\right)\)

XI. \(\quad \sin t=t F\left(k, k^{\prime}, \frac{3}{2},-\frac{t t}{4 k k^{\prime}}\right)\)

XII. \(\quad \cos t=F\left(k, k^{\prime}, \frac{1}{2},-\frac{t t_{i}}{4 k k^{\prime}}\right)\)

XIII. \(\quad t=\sin t . F\left(\frac{1}{2}, \frac{1}{2}, \frac{3}{2}, \sin t^{2}\right)\)

XIV. \(\quad t=\sin t \cdot \cos t . F\left(1,1, \frac{3}{2}, \sin t^{2}\right)\)

XV. \(\quad t=\operatorname{tang} t . F\left(\frac{1}{2}, 1, \frac{3}{2},-\operatorname{tang} t^{2}\right)\)

XVI. \(\quad \sin n t=n \sin t . F\left(\frac{1}{2} n+\frac{1}{2},-\frac{1}{2} n+\frac{1}{2}, \frac{3}{2}, \sin t^{2}\right)\)

XVII. \(\quad \sin n t=n \sin t \cdot \cos t . F\left(\frac{1}{2} n+1,-\frac{1}{2} n+1, \frac{3}{2}, \sin t^{2}\right)\)

XVIII. \(\quad \sin n t=n \sin t \cdot \cos t^{n-1} F\left(-\frac{1}{2} n+1,-\frac{1}{2} n+\frac{1}{2}, \frac{3}{2},-\operatorname{tang} t^{2}\right)\)

XIX. \(\quad \sin n t=n \sin t \cdot \cos t^{-n-1} F\left(\frac{1}{2} n+1, \frac{1}{2} n+\frac{1}{2}, \frac{3}{2},-\operatorname{tang} t^{2}\right)\)

XX. \(\quad \cos n t=F\left(\frac{1}{2} n,-\frac{1}{2} n, \frac{1}{2}, \sin t^{2}\right)\)

XXI. \(\quad \cos n t=\cos t . F\left(\frac{1}{2} n+\frac{1}{2},-\frac{1}{2} n+\frac{1}{2}, \frac{1}{2}, \sin t^{2}\right)\)

XXII. \(\quad \cos n t=\cos t^{n} F\left(-\frac{1}{2} n,-\frac{1}{2} n+\frac{1}{2}, \quad 1+\operatorname{tang} t^{2}\right)\)

XXIII. \(\quad \cos n t=\cos ^{-n} F\left(\frac{1}{2} n+\frac{1}{2}, \frac{1}{2} n, \frac{1}{2},-\operatorname{tang} t^{2}\right)\)

6.

Functiones praecedentes sunt algebraicae atque transscendentes a logarithmis circuloque pendentes. Neutiquam vero harum caussa disquisitionem nostram generalem suscipimus, sed potius in gratiam theoriae functionum transscendentium altiorum promovendae, quarum genus amplissimum series nostra complectitur. Huc, inter infinita alia, pertinent coëfficientes ex evolutione functionis \((a a+b b-2 a b \cos \varphi)^{-n}\) in seriem secundum cosinus angulorum \(\varphi, 2 \varphi, 3 \varphi\) etc. progredientem orti, de quibus in specie alia occasione fusius agemus. Ad formam seriei nostrae autem illi coëfficientes pluribus modis reduci possunt. Scilicet statuendo

\[
(a a+b b-2 a b \cos \varphi)^{-n}=Q=A+2 A^{\prime} \cos \varphi+2 A^{\prime \prime} \cos 2 \varphi+2 A^{\prime \prime \prime} \cos 3 \varphi+\text { etc. }
\]

habemus primo

\[
\begin{aligned}
& A=a^{-2 n} F\left(n, n, 1, \frac{b b}{a a}\right) \\
& A^{\prime}=n a^{-2 n-1} b F\left(n, n+1,2, \frac{b b}{a a}\right) \\
& A^{\prime \prime}=\frac{n(n+1)}{1 \cdot 2} a^{-2 n-2} b b F\left(n, n+2,3, \frac{b b}{a a}\right) \\
& A^{\prime \prime \prime}=\frac{n(n+1)(n+2)}{1 \cdot 2 \cdot 3} a^{-2 n-3} b^{3} F\left(n, n+3,4 \cdot \frac{b b}{a a}\right) \\
& \text { etc. }
\end{aligned}
\]

Si enim \(a a+b b-2 a b \cos \varphi\) consideratur tamquam productum ex \(a-b r\) in \(a-b r^{-1}\) (designante \(r\) quantitatem \(\left.\cos \varphi+\sin \varphi \cdot \sqrt{ }-1\right)\), fit \(Q\) aequalis producto ex \(a^{-2 n}\)

\[
\begin{aligned}
& \text { in } \quad 1+n \frac{b r}{a}+\frac{n(n+1)}{1 \cdot 2} \cdot \frac{b b r r}{a a}+\frac{n(n+1)(n+2)}{1 \cdot 2 \cdot 3} \cdot \frac{b^{3} r^{3}}{a^{3}}+\text { etc. } \\
& \text { in } \quad 1+n \frac{b r^{-1}}{a}+\frac{n(n+1)}{1 \cdot 2} \cdot \frac{b b r^{-2}}{a a}+\frac{n(n+1)(n+2)}{1 \cdot 2 \cdot 3} \cdot \frac{b^{3} r^{-3}}{a^{3}}+\text { etc. }
\end{aligned}
\]

Quod productum quum identicum esse debeat cum

\[
A+A^{\prime}\left(r+r^{-1}\right)+A^{\prime \prime}\left(r r+r^{-2}\right)+A^{\prime \prime \prime}\left(r^{3}+r^{-3}\right)
\]

valores supra dati sponte prodeunt.

Porro habemus secundo

\[
\begin{aligned}
& A=(a a+b b)^{-n} F\left(\frac{1}{2} n, \frac{1}{2} n+\frac{1}{2} \cdot 1, \frac{4 a a b b}{(a a+b b)^{2}}\right) \\
& A^{\prime}=n(a a+b b)^{-n-1} a b F\left(\frac{1}{2} n+\frac{1}{2}, \frac{1}{2} n+1,2, \frac{4 a a b b}{(a a+b b)^{2}}\right) \\
& A^{\prime \prime}=\frac{n(n+1)}{1 \cdot 2}(a a+b b)^{-n-2} a a b b F\left(\frac{1}{2} n+1, \frac{1}{2} n+\frac{3}{2}, 3, \frac{4 a a b b}{(a a+b b)^{2}}\right) \\
& A^{\prime \prime \prime}=\frac{n(n+1)(n+2)}{1 \cdot 2 \cdot 3}(a a+b b)^{-n-3} a^{3} b^{3} F\left(\frac{1}{2} n+\frac{3}{2}, \frac{1}{2} n+2,4, \frac{4 a a b b}{(a a+b b)^{2}}\right) \\
& \text { etc. }
\end{aligned}
\]

qui valores facile deducuntur ex

\(Q(a a+b b)^{n}=1+n\left(r+r^{-1}\right) \frac{a b}{a a+b b}+\frac{n(n+1)}{1 \cdot 2}\left(r+r^{-1}\right)^{2} \frac{a a b b}{(a a+b b)^{2}}+\) etc.

Tertio fit

\[
\begin{gathered}
A=(a+b)^{-2 n} F\left(n, \frac{1}{2}, 1, \frac{4 a b}{(a+b)^{2}}\right) \\
A^{\prime}=n(a+b)^{-2 n-2} a b F\left(n+1, \frac{3}{2}, 3, \frac{4 a b}{(a+b)^{2}}\right) \\
A^{\prime \prime}=\frac{n(n+1)}{1 \cdot 2}(a+b)^{-2 n-4} a a b b F\left(n+2, \frac{5}{2}, 5, \frac{4 a b}{(a+b)^{2}}\right) \\
A^{\prime \prime \prime}=\frac{n(n+1)(n+2)}{1 \cdot 2 \cdot 3}(a+b)^{-2 n-6} a^{3} b^{3} F\left(n+3, \frac{7}{2}, 7, \frac{4 a b}{(a+b)^{2}}\right) \\
\text { etc. }
\end{gathered}
\]

Denique fit quarto

\[
\begin{gathered}
A=(a-b)^{-2 n} F\left(n, \frac{1}{2}, 1,-\frac{4 a b}{(a-b)^{2}}\right) \\
A^{\prime}=n(a-b)^{-2 n-2} a b F\left(n+1, \frac{3}{2}, 3,-\frac{4 a b}{(a-b)^{2}}\right) \\
A^{\prime \prime}=\frac{n(n+1)}{1 \cdot 2}(a-b)^{-2 n-4} a a b b F\left(n+2, \frac{5}{2}, 5,-\frac{4 a b}{(a-b)^{2}}\right) \\
A^{\prime \prime \prime}=\frac{n(n+1)(n+2)}{1 \cdot 2 \cdot 3}(a-b)^{2 n-6} a^{3} b^{3} F\left(n+3, \frac{7}{2}, 7,-\frac{4 a b}{(a-b)^{2}}\right) \\
\text { etc. }
\end{gathered}
\]

Valores illi atque hi facile eruuntur ex

\[
\begin{aligned}
Q(a+b)^{2 n} & =\left(1-\frac{4 a b \cos \frac{1}{2} \varphi^{2}}{(a+b)^{2}}\right)^{-n} \\
& =1+n \frac{a b}{(a+b)^{2}}\left(r^{\frac{1}{2}}+r^{-\frac{1}{2}}\right)^{2}+\frac{n(n+1)}{1 \cdot 2} \cdot \frac{a a b b}{(a+b)^{2}}\left(r^{\frac{1}{2}}+r^{-\frac{1}{2}}\right)^{2}+\text { etc. } \\
Q(a-b)^{2 n} & =\left(1+\frac{4 a b \sin \frac{1}{2} \varphi^{2}}{(a-b)^{2}}\right)^{-n} \\
& =1+n \frac{a b}{(a-b)^{2}}\left(r^{\frac{1}{2}}-r^{-\frac{1}{2}}\right)^{2}+\frac{n(n+1)}{1 \cdot 2} \cdot \frac{a a b b}{(a-b)^{2}}\left(r^{\frac{1}{2}}-r^{-\frac{1}{2}}\right)^{4}+\text { etc. }
\end{aligned}
\]

\section*{\(S E C T I O P R I M A\).}
Relationes inter functiones contiguas:

7.

Functionem ipsi \(F^{\prime}(\alpha, b, \dot{\gamma}, x)\) contiguam vocamus, quae ex illa oritur, dum elementum primum, secundum, vel tertinm unitate vel augetur vel diminuitur, manentibus tribus reliquis elementis. Functio itaque primaria \(F(\alpha . b, \gamma, x)\) sex contiguas suppeditat, inter quarum binas ipsamque primariam aequatio persimplex linearis datur. Has aequationes, numero quindecim, hic in conspectum producimus, brevitatis gratia elementum quartum quod semper subintelligitur \(=x\) omittentes, functionemque primariam simpliciter per \(F\) denotantes.

[1] \(\quad 0=(\gamma-2 \alpha-(b-\alpha) x) F+\alpha(1-x) F(\alpha+1, b, \gamma)-(\gamma-\alpha) F(\alpha-1, b, \gamma)\)

[2] \(0=(b-\alpha) F+\alpha F(\alpha+1, b, \gamma)-b F(\alpha, b+1, \gamma)\)

[3] \(0=(\gamma-\alpha-b) F+\alpha(1-x) F(\alpha+1, b, \gamma)-(\gamma-b) F(\alpha, b-1, \gamma)\)

[4] \(0=\gamma(\alpha-(\gamma-b) x) F-\alpha \gamma(1-x) F(\alpha+1, b, \gamma)+(\gamma-\alpha)(\gamma-b) \times F(\alpha, b, \gamma+1)\)

[5] \(\quad 0=(\gamma-\alpha-1) F+\alpha F(\alpha+1, b, \gamma)-(\gamma-1) F(\alpha, b, \gamma-1)\)

[6] \(\quad 0=(\gamma-\alpha-b) F-(\gamma-\alpha) F(\alpha-1, b, \gamma)+b(1-x) F(\alpha, b+1, \gamma)\)

[7] \(0=(b-\alpha)(1-x) F-(\gamma-\alpha) F(\alpha-1, b, \gamma)+(\gamma-b) F(\alpha, b-1, \gamma)\)

[8] \(0=\gamma(1-x) F-\gamma F(\alpha-1, b, \gamma)+(\gamma-b) x F(\alpha, b, \gamma+1)\)

[9] \(0=(\alpha-1-(\gamma-b-1) x) F+(\gamma-\alpha) F(\alpha-1, b, \gamma)-(\gamma-1)(1-x) F(\alpha, b, \gamma-1)\)

\([10] 0=(\gamma-2 b+(b-\alpha) x) F+b(1-x) F(\alpha, b+1, \gamma)-(\gamma-b) F(\alpha, b-1, \gamma)\)

[11] \(0=\gamma(b-(\gamma-\alpha) x) F-b \gamma(1-x) F(\alpha, b+1, \gamma)-(\gamma-\alpha)(\gamma-b) F(\alpha, b, \gamma+1)\)

[12] \(0=(\gamma-b-1) F+b F(\alpha, b+1, \gamma)-(\gamma-1) F(\alpha, b, \gamma-1)\)

[13] \(0=\gamma(1-x) F-\gamma F(\alpha, b-1, \gamma)+(\gamma-\alpha) x F(\alpha, b, \gamma+1)\)

[14] \(0=(b-1-(\gamma-\alpha-1) x) F+(\gamma-b) F(\alpha, b-1, \gamma)-(\gamma-1)(1-x) F(\alpha, b, \gamma-1)\)

[15] \(0=\gamma(\gamma-1-(2 \gamma-\alpha-b-1) x) F+(\gamma-\alpha)(\gamma-b) x F(\alpha, b, \gamma+1)\)

\(-\gamma(\gamma-1)(1-x) \boldsymbol{F}(\alpha, b, \gamma-1)\)

8.

Ecce iam demonstrationem harum formularum. Statuendo

\[
\frac{(\alpha+1)(\alpha+2) \ldots(\alpha+m-1)^{6}(b+1) \ldots(b+m-2)}{1 \cdot 2 \cdot 3 \cdot \ldots \cdot(\gamma \cdot \gamma(\gamma+1) \ldots \cdot(\gamma+m-1)}=M
\]

CIRCA SERIEM INFINTTAM \(1+\frac{a b}{1 \cdot \gamma} x+\) ETC.

erit coëfficiens potestatis \(x^{m}\)

\[
\begin{aligned}
& \text { in } F \cdot . \quad \cdot \quad \cdot \quad \alpha(b+m-1) M \\
& \text { in } F(\alpha, b-1, \gamma) \cdot \cdot(\alpha(b-1) M \\
& \text { in } F(\alpha+1, b, \gamma) \cdot .(\alpha+m)(b+m-1) M \\
& \text { in } F(\alpha, b ; \gamma-1) \cdot . \quad \frac{(b+m-1)(\gamma+m-1) M}{\gamma-1}
\end{aligned}
\]

coëfficiens autem potestatis \(x^{m-1}\) in \(F(\alpha+1, b, \gamma)\), seu coëfficiens potestatis \(x^{m}\) in \(x F(\alpha+1, b, \gamma)\)

\[
=m(\gamma+m-1) M
\]

Hinc statim demanat veritas formularum 5 et 3 ; permutando \(\alpha\) cum \(b\), oritur ex 5 formula 12, atque ex his duabus per eliminationem 2. Perinde per eandem permutationem ex 3 oritur 6 ; ex 6 et 12 combinatis oritur 9 , hinc per permutationem 14, quibus combinatis habetur 7; denique ex 2 et 6 eruitur 1 , atque hinc permutando 10. Formula 8 simili modo ut supra formulae 5 et 3 , e consideratione coëfficientium derivari potest (eodemque modo, si placeret, omnes 15 formulae erui possent), vel elegantius ex iam notis sequenti modo. Mutando in formula 5 elementum \(\alpha\) in \(\alpha-1\), atque \(\gamma\) in \(\gamma+1\), prodit

\[
0=(\gamma-\alpha+1) F(\alpha-1, b, \gamma+1)+(\alpha-1) F(\alpha, b, \gamma+1)-\gamma F(\alpha-1, b, \gamma)
\]

Mutando vero in formula 9 tantummodo \(\gamma\) in \(\gamma+1\), fit

\(0=(\alpha-1-(\gamma-b) x) F(\alpha, b, \gamma+1)+(\gamma-\alpha+1) F(\alpha-1, b, \gamma+1)-\gamma(1-x) F(\alpha, b, \gamma)\)

E subtractione harum formularum statim oritur 8 , atque hinc per permutationem 13. Ex 1 et 8 prodit 4, hincque permutando 11. Denique ex 8 et 9 deducitur 15.

9.

Si \(\alpha^{\prime}-\alpha, b^{\prime}-b, \gamma^{\prime}-\gamma\), nec non \(\alpha^{\prime \prime}-\alpha, b^{\prime \prime}-b . \gamma^{\prime \prime}-\gamma\) sunt numeri integri (positivi seu negativi), a functione \(F(\alpha, b, \gamma)\) ad functionem \(F\left(\alpha^{\prime}, b^{\prime}, \gamma^{\prime}\right)\), et perinde ab hac usque ad functionem \(F\left(\alpha^{\prime \prime}, b^{\prime \prime}, \gamma^{\prime \prime}\right)\) transire licet per seriem similium functionum, ita ut quaelibet contigua sit antecedenti et consequenti, mutando scilicet primo elementum unum e.g. \(\alpha\) continuo unitate, donec a \(F(\alpha, b, \gamma)\) perventum sit ad \(F\left(\alpha^{\prime}, b, \gamma\right)\), dein mutando elementum secundum, donec perven-
tum sit ad \(F\left(\alpha^{\prime}, b^{\prime}, \gamma\right)\), denique mutando elementum tertium, donec perventum sit ad \(F\left(\alpha^{\prime}, b^{\prime}, \gamma^{\prime}\right)\), et perinde ab hac usque ad \(F\left(\alpha^{\prime \prime}, b^{\prime \prime}, \gamma^{\prime \prime}\right)\). Quum itaque per art. 7 habeantur aequationes lineares inter functionem primam, secundam atque tertiam, et generaliter inter ternas quascunque consequentes huius seriei, facile perspicitur, hinc per eliminationem deduci posse aequationem linearem inter functiones \(F(\alpha, b, \gamma), F\left(\alpha^{\prime}, b^{\prime}, \gamma^{\prime}\right), F\left(\alpha^{\prime \prime}, b^{\prime \prime}, \gamma^{\prime \prime}\right)\), ita ut generaliter loquendo e duabus functionibus, quarum tria elementa prima numeris integris differunt, quamlibet aliam functionem eadem proprietate gaudentem derivare liceat, siquidem elementum quartum idem maneat. Ceterum hic nobis sufficit, hanc veritatem insignem generaliter stabilivisse, neque hic compendiis immoramur, per quae operationes ad hunc finem necessariae quam brevissimae reddantur.

\section*{10.}
Propositae sint e. g. functiones

\[
F(\alpha, b, \gamma), F(\alpha+1, b+1, \gamma+1), F(\alpha+2, b+2, \gamma+2)
\]

inter quas aequationem linearem invenire oporteat. Iungamus ipsas per functiones contiguas sequenti modo:

\[
\begin{aligned}
& F(\alpha, b, \gamma)=F \\
& F(\alpha+1, b, \gamma)=F^{\prime} \\
& F(\alpha+1, b+1, \gamma)=F^{\prime \prime} \\
& F(\alpha+1, b+1, \gamma+1)=F^{\prime \prime \prime} \\
& F(\alpha+2, b+1, \gamma+1)=F^{\prime \prime \prime} \\
& F(\alpha+2, b+2, \gamma+1)=F^{\prime \prime \prime \prime} \\
& F(\alpha+2, b+2, \gamma+2)=F^{\prime \prime \prime \prime \prime}
\end{aligned}
\]

Habemus itaque quinque aequationes lineares (e formulis \(6,13,5\) art. 7):
I. \(\quad 0=(\gamma-\alpha-1) F-(\gamma-\alpha-1-b) F^{\prime}-b(1-x) F^{\prime \prime}\)

II. \(0=\gamma F^{\prime}-\gamma(1-x) F^{\prime \prime}-(\gamma-\alpha-1) x F^{\prime \prime \prime}\)

III. \(0=\gamma \boldsymbol{F}^{\prime \prime}-(\gamma-\alpha-1) \boldsymbol{F}^{\prime \prime \prime}-(\alpha+1) \boldsymbol{F}^{\prime \prime \prime}\)

IV. \(0=(\gamma-\alpha-1) \boldsymbol{F}^{\prime \prime \prime}-(\gamma-\alpha-2-b) \boldsymbol{F}^{\prime \prime \prime \prime}-(b+1)(1-x) \boldsymbol{F}^{\prime \prime \prime \prime}\)
V. \(0=(\gamma+1) \boldsymbol{F}^{\prime \prime \prime}-(\gamma+1)(1-x) \boldsymbol{F}^{\prime \prime \prime \prime}-(\gamma-\alpha-1) x \boldsymbol{F}^{\prime \prime \prime \prime}\)

Ex I et II prodit, eliminando \(F^{\prime}\)

\[
\text { CIRCA SERIEM INFINITAM } 1+\frac{\alpha 6}{1 \cdot \gamma} x+\text { ETC. }
\]

VI. \(0=\gamma \boldsymbol{F}-\gamma(1-x) \boldsymbol{F}^{\prime \prime}-(\gamma-\alpha-b-1) x \boldsymbol{F}^{\prime \prime \prime}\)

Hinc atque ex III, eliminando \(\boldsymbol{F}^{\prime \prime}\)

VII.

\[
0=\gamma \boldsymbol{F}-(\gamma-\alpha-1-b x) \boldsymbol{F}^{\prime \prime \prime}-(\alpha+1)(1-x) \boldsymbol{F}^{\prime \prime \prime}
\]

Porro ex IV atque V, eliminando \(\boldsymbol{F}^{\prime \prime \prime \prime}\)

VIII. \(0=(\gamma+1) \boldsymbol{F}^{\prime \prime \prime}-(\gamma+1) \boldsymbol{F}^{\prime \prime \prime}+(b+1) x \boldsymbol{F}^{\prime \prime \prime \prime \prime}\)

Hinc atque ex VII, eliminando \(\boldsymbol{F}^{\prime \prime \prime \prime}\),

IX. \(0=\gamma(\gamma+1) F-(\gamma+1)(\gamma-(\alpha+b+1) x) \boldsymbol{F}^{\prime \prime \prime}-(\alpha+1)(b+1) x(1-x) \boldsymbol{F}^{\prime \prime \prime \prime \prime}\)

\section*{11.}
Si omnes relationes inter ternas functiones \(\boldsymbol{F}(\alpha, b, \gamma), \boldsymbol{F}(\alpha+\lambda, b+\mu, \gamma+\nu)\), \(F\left(\alpha+\lambda^{\prime}, b+\mu^{\prime}, \gamma+\nu^{\prime}\right)\), in quibus \(\lambda, \mu, \nu, \lambda^{\prime}, \mu^{\prime}, \nu^{\prime}\) vel \(=0\) vel \(=+1\) vel \(=-1\), exhaurire vellemus, formularum multitudo usque ad 325 ascenderet. Haud inutilis foret talis collectio, saltem simpliciorum ex his formulis: hoc vero loco sufficiat, paucas tantummodo apposuisse, quas vel ex formulis art.7, vel si magis placet, simili modo ut duae priores ex illis in art. 8 erutae sunt, quivis nullo negotio sibi demonstrare poterit.

\[
\begin{aligned}
& F(\alpha, b, \gamma)-F(\alpha, b, \gamma-1)=-\frac{\alpha b x}{\gamma(\gamma-1)} F(\alpha+1, b+1, \gamma+1) \\
& F(\alpha, b+1, \gamma)-F(\alpha, b, \gamma)=\frac{\alpha x}{\gamma} F(\alpha+1, b+1, \gamma+1) \\
& F(\alpha+1, b, \gamma)-F(\alpha, b, \gamma)=\frac{b x}{\gamma} F(\alpha+1, b+1, \gamma+1) \\
& F(\alpha, b+1, \gamma+1)-F(\alpha, b, \gamma)=\frac{\alpha(\gamma-b) x}{\gamma(\gamma+1)} F(\alpha+1, b+1, \gamma+2) \\
& F(\alpha+1, b, \gamma+1)-F(\alpha, b, \gamma)=\frac{b(\gamma-\alpha) x}{\gamma(\gamma+1)} F(\alpha+1, b+1, \gamma+2) \\
& F(\alpha-1, b+1, \gamma)-F(\alpha, b, \gamma)=\frac{(\alpha-b-1) x}{\gamma} F(\alpha, b+1, \gamma+1) \\
& F(\alpha+1, b-1, \gamma)-F(\alpha, b, \gamma)=\frac{(b-\alpha-1) x}{\gamma} F(\alpha+1, b, \gamma+1) \\
& \text { ] } \\
& F(\alpha-1, b+1, \gamma)-F(\alpha+1, b-1, \gamma)=\frac{(\alpha-b) x}{\gamma} F(\alpha+1, b+1, \gamma+1)
\end{aligned}
\]

SECTIO SECUNDA.

Fractiones continuae.

12.

Designando

\[
\frac{F(\alpha, b+1, \gamma+1, x)}{F(\alpha, b, \gamma, x)} \text { per } G(\alpha, b, \gamma ; x)
\]

fit

\[
\frac{F(\alpha+1, b, \gamma+1, x)}{F(\alpha, b, \gamma, x)}=\frac{F(b, \alpha+1, \gamma+1, x)}{F(b, \alpha, \gamma, x)}=G(b, \alpha, \gamma, x)
\]

et proin, dividendo aequationem 19 per \(\boldsymbol{F}(\alpha, b+1, \gamma+1, x)\),

\[
1-\frac{1}{G(\alpha, b, \gamma, x)}=\frac{\alpha(\gamma-\ell)}{\gamma(\gamma+1)} x G(b+1, \alpha, \gamma+1, x)
\]

sive

\[
G(\alpha, b, \gamma, x)=\frac{1}{1-\frac{\alpha(\gamma-b)}{\gamma(\gamma+1)} x G(b+1, \alpha, \gamma+1, x)}
\]

et quum perinde fiat

\[
G(b+1, \alpha, \gamma+1, x)=\frac{1}{1-\frac{(b+1)(\gamma+1-\alpha)}{(\gamma+1)(\gamma+2)} x G(\alpha+1, b+1, \gamma+2, x)}
\]

etc., resultabit pro \(G(\alpha, b, \gamma, x)\) fractio continua

ubi

\[
\frac{F(\alpha, b+1, \gamma+1, x)}{F(\alpha, b, \gamma, x)}=\frac{1}{1-\frac{a x}{1-\frac{b x}{1-\frac{c x}{1-\frac{d x}{1-\text { etc. }}}}}}
\]

\[
\begin{array}{ll}
a=\frac{\alpha(\gamma-b)}{\gamma(\gamma+1)} & b=\frac{(b+1)(\gamma+1-\alpha)}{(\gamma+1)(\gamma+2)} \\
c=\frac{(\alpha+1)(\gamma+1-b)}{(\gamma+2)(\gamma+3)} & d=\frac{(b+2)(\gamma+2-a)}{(\gamma+3)(\gamma+4)} \\
e=\frac{(\alpha+2)(\gamma+2-b)}{(\gamma+4)(\gamma+5)} & f=\frac{(b+3)(\gamma+3-a)}{(\gamma+5)(\gamma+6)}
\end{array}
\]

etc., cuius lex progressionis obvia est.

Porro ex aequationibus 17, 18, 21, 22 sequitur

CIRCA SERIEM INFINTTAM \(1+\frac{\mu 6}{1 \cdot \gamma} x+\) ETC.

\([26]\)

\[
\begin{gathered}
\frac{F(\alpha, b+1, \gamma, x)}{F(\alpha, b, \gamma, x)}=\frac{1}{1-\frac{\alpha x}{\gamma} G(b+1, \alpha, \gamma, x)} \\
\frac{F(\alpha+1, b, \gamma, x)}{F(\alpha, b, \gamma, x)}=\frac{1}{1-\frac{b x}{\gamma} G(\alpha+1, b, \gamma, x)} \\
\frac{F(\alpha-1, b+1, \gamma, x)}{F(\alpha, b, \gamma, x)}=\frac{1}{1-\frac{(\alpha-b-1) x}{\gamma} G(b+1, \alpha-1, \gamma, x)} \\
\frac{F(\alpha+1, b-1, \gamma, x)}{F(\alpha, b, \gamma, x)}=\frac{1}{1-\frac{(b-\alpha-1) x}{\gamma} G(\alpha+1, b-1, \gamma, x)}
\end{gathered}
\]

unde, substitutis pro functione \(G\) eius valoribus in fractionibus continuis, totidem fractiones continuae novae prodeunt.

Ceterum sponte patet, fractionem continuam in formula 25 abrumpi, si e numeris \(\alpha, b, \gamma-\alpha, \gamma-b\) aliquis fuerit integer negativus, alioquin vero in infinitum excurrere.

\section*{13.}
Fractiones continuae in art. praec. erutae maximi sunt momenti, asserique potest, vix ullas fractiones continuas secundum legem obviam progredientes ab analystis hactenus erutas esse, quae sub nostris tamquam casus speciales non sint contentae. Imprimis memorabilis est casus is, ubi in formula 25 statuitur \(b=0\), unde \(F(\alpha, \hat{b}, \gamma, x)=1\), adeoque, scribendo \(\gamma-1\) pro \(\gamma\)

\[
\begin{aligned}
& F(\alpha, 1, \gamma)=1+\frac{\alpha}{\gamma} x+\frac{\alpha(\alpha+1)}{\gamma(\gamma+1)} x x+\frac{\alpha(\alpha+1)(\alpha+2)}{\gamma(\gamma+1)(\gamma+2)} x^{3}+\text { etc. }
\end{aligned}
\]

\begin{center}
%\includegraphics[max width=\textwidth]{2024_01_11_75975a03bcf8b0416cd0g-111}
\end{center}

ubi

\[
\begin{aligned}
& a=\frac{\alpha}{\gamma} \quad b=\frac{\gamma-\alpha}{\gamma(\gamma+1)} \\
& c=\frac{(\alpha+1) \gamma}{(\gamma+1)(\gamma+2)} \quad d=\frac{2(\gamma+1-a)}{(\gamma+2)(\gamma+3)} \\
& e=\frac{(\alpha+2)(\gamma+1)}{(\gamma+3)(\gamma+4)} \quad f=\frac{3(\gamma+2-\alpha)}{(\gamma+4)(\gamma+5)} \\
& \text { etc. }
\end{aligned}
\]

14.

Operae pretium erit, quosdam casus speciales huc adscripsisse. Ex formula I art. 5 sequitur, statuendo \(t=1, b=1\)

\[
\begin{aligned}
& (1+u)^{n}=\frac{1}{1-\frac{n u}{}} \\
& 1+\frac{\frac{n+1}{2} u}{n} \\
& 1-\frac{n-1}{2.3} u \\
& 1+\frac{\frac{2(n+2)}{3 \cdot 4} u}{1-\frac{2(n-2)}{1+5} u}
\end{aligned}
\]

E formulis VI, VII art. 5 sequitur

\begin{center}
%\includegraphics[max width=\textwidth]{2024_01_11_75975a03bcf8b0416cd0g-112}
\end{center}

\[
\begin{aligned}
& \log \frac{1+t}{1-t}=\frac{2 \dot{t}}{1-\frac{\frac{1}{3} t t}{\frac{2.2}{t} t t}} \\
& 1-\frac{\frac{2.2}{3.5} t t}{1-\frac{\frac{3.3}{5.7} t t}{1-\frac{\frac{4.4}{7.9} t t}{1-\text { etc. }}}}
\end{aligned}
\]

Mutando hic signa - in + prodit fractio continua pro arc. tang \(t\).

Porro habemus

\[
\begin{aligned}
& e^{t}=\frac{1}{1-\frac{t}{1+\frac{\frac{1}{3} t}{1-\frac{1}{6} t}}} \\
& 1+\frac{\frac{1}{6} t}{1-\frac{1}{10} t} \\
& 1+\frac{\frac{1}{10} t}{1-\text { etc }}
\end{aligned}
\]

CIRCA SERIEM INFINITAM \(1+\frac{\alpha \delta}{1 \cdot \gamma} x+\) etc.

\[
\begin{aligned}
& t=\frac{\sin t \cos t}{\frac{1.2}{1.3} \sin t^{2}} \\
& 1-\frac{1.2}{3.5} \sin t^{2}
\end{aligned}
\]

\begin{center}
%\includegraphics[max width=\textwidth]{2024_01_11_75975a03bcf8b0416cd0g-113}
\end{center}

Statuendo \(\alpha=3, \gamma=\frac{5}{2}\), e formula 30 sponte sequitur fractio continua in Theoria motus corporum coelestium art. 90 proposita. Ibidem duae aliae fractiones continuae prolatae sunt, quarum evolutionem hacce occasione supplere visum est. Statuendo

\[
Q=1-\frac{\frac{5.8}{7.9} x}{1-\frac{\frac{1.4}{9.11} x}{1-\frac{7.10}{11.13} x \text { etc. }}}
\]

fit 1. c. \(x-\xi=\frac{x}{1+\frac{2 x}{35 Q}}=\frac{x Q}{Q+\frac{2}{35} x}\), adeoque

\[
\xi=\frac{\frac{2}{35} x x}{Q+\frac{2}{35} x}
\]

quae est formula prior: posterior sequenti modo eruitur. Statuendo

\[
\begin{aligned}
& R=1-\frac{\frac{1.4}{7.9} x}{5.8}
\end{aligned}
\]

\begin{center}
%\includegraphics[max width=\textwidth]{2024_01_11_75975a03bcf8b0416cd0g-113(1)}
\end{center}

erit per formulam 25

\[
\frac{1}{R}=G\left(\frac{1}{2}, \frac{3}{2}, \frac{7}{2}, x\right), \quad \text { atque } \quad \frac{1}{Q}=G\left(\frac{5}{2},-\frac{1}{2}, \frac{7}{2}, x\right)
\]

Hinc

\[
\begin{aligned}
& R F\left(\frac{1}{2}, \frac{5}{2}, \frac{9}{2}, x\right)=F\left(\frac{1}{2}, \frac{8}{2}, \frac{7}{2}, x\right) \\
& Q F\left(\frac{5}{2}, \frac{1}{2}, \frac{9}{2}, x\right)=F\left(\frac{5}{2},-\frac{1}{2}, \frac{7}{2}, x\right)
\end{aligned}
\]

sive permutando elementum primum cum secundo

\[
Q F\left(\frac{1}{2}, \frac{5}{2}, \frac{9}{2}, x\right)=F\left(-\frac{1}{2}, \frac{5}{2}, \frac{7}{2}, x\right)
\]

Sed per aequationem 21 habemus

\[
F\left(-\frac{1}{2}, \frac{5}{2}, \frac{7}{2}, x\right)-F\left(\frac{1}{2}, \frac{3}{2}, \frac{7}{2}, x\right)=-\frac{4}{7} x F\left(\frac{1}{2}, \frac{5}{2}, \frac{9}{2}, x\right)
\]

unde fit \(Q=\boldsymbol{R}-\frac{4}{7} x\), quo valore in formula supra data substituto prodit

\[
\xi=\frac{\frac{2}{35} x x}{R-\frac{1}{35} x}
\]

quae est formula posterior.

Statuendo in formula \(30, \alpha=\frac{m}{n}, x=-\gamma n t\), fit pro valore infinite magno ipsius \(\gamma\)

\[
\begin{aligned}
& F\left(\frac{m}{n}, 1, \gamma,-\gamma n t\right)=1-m t+m(m+n) t t-m(m+n)(m+2 n) t^{3}+\text { etc. }
\end{aligned}
\]

\begin{center}
%\includegraphics[max width=\textwidth]{2024_01_11_75975a03bcf8b0416cd0g-114}
\end{center}

\section*{SECTIO T'ERTIA.}
De summa seriei nostrae statuendo elementum quartum \(=1\), ubi simul quaedam aliae functiones transscendentes discutiuntur.

15.

Quoties elementa \(\alpha, b, \gamma\) omnia sunt quantitates positivae, omnes coëfficientes potestatum elementi quarti \(x\) positivi evadunt: quoties vero ex illis elementis unum alterumve negativum est, saltem inde ab aliqua potestate \(x^{m}\) omnes coëfficientes eodem signo affecti erunt, si modo \(m\) accipitur maior quam valor absolutus elementi negativi maximi. Porro hinc sponte patet, seriei șummam pro

\[
\text { CIRCA SERIEM ANFINITAM } 1+\frac{a^{\ell}}{1 \cdot \gamma} x+\text { ETC. }
\]

\(x=1\) finitam esse non posse, nisi coëfficientes saltem post certum terminum in infinitum decrescant, vel, ut secundum morem analystarum loquamur, nisi coëfficiens termini \(x^{\infty}\) sit \(=0\). Ostendemus autem, et quidem, in gratiam eorum, qui methodis rigorosis antiquorum geometrarum favent, omni rigore,

primo, coëfficientes (siquidem series non abrumpatur), in infinitum crescere, quoties fuerit \(a+b-\gamma-1\) quantitas positiva.

secundo, coëfficientes versus limitem finitum continuo convergere, quoties fuerit \(\alpha+b-\gamma-1=0\).

tertio, coëfficientes in infinitum decrescere, quoties fuerit \(\alpha+b-\gamma-1\) quantitas\_negativa.

quarto, summam seriei nostrae pro \(x=1\), non obstante convergentia in casu tertio, infinitam esse, quoties fuerit \(\alpha+b-\gamma\) quantitas positiva vel \(=0\).

quinto, summam vero finitam esse, quoties \(\alpha+b-\gamma\) fuerit quantitas negativa.

\section*{16.}
Hanc disquisitionem generalius adaptabimus seriei infinitae \(M, M^{\prime}, M^{\prime \prime}, M^{\prime \prime \prime}\) etc. ita formatae, ut quotientes \(\frac{M^{\prime}}{\bar{M}}, \frac{M^{\prime \prime}}{\bar{M}^{\prime}}, \frac{M^{\prime \prime \prime}}{M^{\prime \prime}}\) etc. resp. sint valores fractionis

\[
\frac{t^{\lambda}+A t^{\lambda-1}+B t^{\lambda-2}+C t^{\lambda-3}+\text { etc. }}{t^{\lambda}+a t^{\lambda-1}+b t^{\lambda-2}+c t^{\lambda-3}+\text { etc. }}
\]

pro \(t=m, t=m+1, t=m+2\) etc. Brevitatis caussa huius fractionis numeratorem per \(P\), denominatorem per \(p\) denotabimus: praeterea supponemus, \(P, p\) non esse identicas, sive differentias \(A-a, B-b, C-c\) etc. non omnes simul evanescere.

I. Quoties e differentiis \(A-a, B-b, C-c\) etc. prima quae non evanescit est positiva, assignari poterit limes aliquis \(l\), quem simulac egressus est valor ipsius \(t\), valores functionum \(P\) et \(p\) certo semper evadent positivi, atque \(P>p\). Manifestum est, hoc evenire, si pro \(l\) accipiatur radix maxima realis aequationis \(p(P-p)=0\); si vero haec aequatio nullas omnino radices reales habeat, proprietatem illam pro omnibus valoribus ipsius \(t\) locum habere. Quapropter in serie \(\frac{M^{\prime}}{\bar{M}}, \frac{M^{\prime \prime}}{M^{\prime}}, \frac{M^{\prime \prime \prime}}{M^{\prime \prime}}\) etc. saltem post certum intervallum (si non ab initio) omnes termini erunt positivi atque maiores unitate; quodsi itaque nullus neque \(=0 \mathrm{ne}-\) que infinite magnus evadit, perspicuum est,
seriem \(M, M^{\prime}, M^{\prime \prime}, M^{\prime \prime \prime}\) etc. si non ab initio tamen post certum intervallum omnes suos terminos eodem signo affectos continuoque crescentes habituram esse.

Eadem ratione, si e differentiis \(A-a, B-b, C-c\) etc. prima quae non evanescit est negativa, series \(M, M^{\prime}, M^{\prime \prime}, M^{\prime \prime \prime}\) etc. si non ab initio tamen post certum intervallum omnes suos terminos eodem signo affectos continuoque decrescentes habebit.

II. Si iam coëfficientes \(A, a\) sunt inaequales, termini seriei \(M, M^{\prime}, M^{\prime \prime}, M^{\prime \prime \prime}\) etc. ultra omnes limites sive in infinitum vel crescent vel decrescent, prout differentia \(A-a\) est positiva vel negativa: hoc ita demonstramus. Si \(A-a\) est quantitas positiva, accipiatur numerus integer \(h\) ita, ut fiat \(h(A-a)>1\), statuaturque \(\frac{M^{h}}{m}=N, \frac{M^{\prime h}}{m+1}=N^{\prime}, \frac{M^{\prime \prime h}}{m+2}=N^{\prime \prime}, \frac{M^{\prime \prime \prime h}}{m+3}=N^{\prime \prime \prime}\) etc., nec non \(t P^{h}=Q\), \((t+1) p^{h}=q . \quad\) Tunc patet, \(\frac{N^{\prime}}{N^{\prime}}, \overline{N^{\prime \prime}}, \frac{N^{\prime \prime \prime}}{N^{\prime \prime}}\) etc. esse valores fractionis \(\frac{Q}{q}\) ponendo \(t=m, t=m+1, t=m+2\) etc., ipsas \(Q, q\) vero esse functiones algebraicas formae huius

\[
\begin{aligned}
& Q=t^{\lambda h+1}+h A t^{\lambda h}+\text { etc. } \\
& q=t^{\lambda h+1}+(h a+1) t^{\lambda h}+\text { etc. }
\end{aligned}
\]

Quare quum per hyp. differentia \(h A-(h a+1)\) sit quantitas positiva, termini seriei \(N, N^{\prime}, N^{\prime \prime}, N^{\prime \prime \prime}\) etc. si non ab initio tamen post certum intervallum continuo crescent (per I); hinc termini seriei \(m \boldsymbol{N},(m+1) \boldsymbol{N}^{\prime},(m+2) \boldsymbol{N}^{\prime \prime}\), \((m+3) N^{\prime \prime \prime}\) etc. necessario ultra omnes limites crescent, et proin etiam termini seriei \(M, M^{\prime}, M^{\prime \prime}, M^{\prime \prime}\) etc., quippe quorum potestates exponente \(h\) illis sunt aequales. Q.E.P.

Si \(A-a\) est quantitas negativa, accipere oportet integrum \(h\) ita, ut \(h(a-A)\) fiat maior quam 1, unde per ratiocinia similia termini seriei

\[
m M^{h}, \quad(m+1) M^{\prime h}, \quad(m+2) M^{\prime \prime h}, \quad(m+3) M^{m \prime h} \quad \text { etc. }
\]

post certum intervallum continuo decrescent. Quamobrem termini seriei \(M^{h}\), \(M^{\prime \prime}, M^{\prime \prime h}\) etc. adeoque etiam termini huius \(M, M^{\prime}, M^{\prime \prime}, M^{\prime \prime \prime}\) etc. necessario in infinitum decrescent. Q. E. S.

III. Si vero coëfficientes primi \(A, a\) sunt aequales, termini seriei \(M, M^{\prime}, M^{\prime \prime}, M^{\prime \prime \prime}\) etc. versus limitem finitum continuo convergent, quod ita demonstramus. Supponamus primo, terminos seriei post certum intervallum continuo crescere, sive e differentiis \(B-b, C-c\) etc. primam quae non.evanescat
esse positivam. Sit \(h\) integer talis, ut \(h+b-B\) fiat quantitas positiva, statuamusque

\[
M\left(\frac{m}{m-1}\right)^{h}=N, \quad M^{\prime}\left(\frac{m+1}{m}\right)^{h}=N^{\prime}, \quad M^{\prime \prime}\left(\frac{m+2}{m+1}\right)^{h}=N^{\prime \prime} \text { etc. }
\]

atque \((t t-1)^{h} P=Q, t^{2 h} p=q\), ita ut \(\frac{N^{\prime}}{N}, \frac{N^{\prime \prime}}{N^{\prime}}\) etc. sint valores fractionis \(\frac{Q}{q}\) ponendo \(t=m, t=m+1\) etc. Quum itaque habeatur

\[
\begin{aligned}
& Q=t^{\lambda+2 h}+A t^{\lambda+2 h-1}+(B-h) t^{\lambda+2 h-2} \text { etc. } \\
& q=t^{\lambda+2 h}+A t^{\lambda+2 h-1}+b t^{\lambda+2 h-2} \text { etc. }
\end{aligned}
\]

atque per hyp. \(B-h-b\) sit quantitas negativa, termini seriei \(N, N^{\prime}, N^{\prime \prime}, N^{\prime \prime \prime}\) etc. post certum saltem intervallum continuo decrescent, adeoque termini seriei \(M, M^{\prime}, M^{\prime \prime}, M^{\prime \prime \prime}\) etc., qui illis resp. semper sunt minores, dum continuo crescunt, tantummodo versus limitem finitum convergere possunt. Q. E. P.

Si termini seriei \(M, M^{\prime}, M^{\prime \prime}, M^{\prime \prime \prime}\) etc. post certum intervallum continuo decrescunt, accipere oportet pro \(h\) integrum talem, ut \(h+B-b\) sit quantitas positiva, evinceturque per ratiocinia prorsus similia, terminos seriei

\[
M\left(\frac{m-1}{m}\right)^{h}, \quad M^{\prime}\left(\frac{m}{m+1}\right)^{h}, \quad M^{\prime \prime}\left(\frac{m+1}{m+2}\right)^{h} \text { etc. }
\]

post certum intervallum continuo crescere, unde termini seriei \(M, M^{\prime}, M^{\prime \prime}\) etc., qui illis resp. semper sunt maiores, dum continuo decrescunt, necessario tantummodo versus limitem finitum decrescere possunt. Q. E. S.

IV. Denique quod attinet ad summam seriei, cuius termini sunt \(M, M^{\prime}\), \(M^{\prime \prime}, M^{\prime \prime \prime}\) etc., in casu eo, ubi hi in infinitum decrescunt, supponamus primo, \(A-a\) cadere inter 0 et -1 , sive \(A+1-a\) esse vel quantitatem positivam vel \(=0\). Sit \(h\) integer positivus, arbitrarius in casu eo, ubi \(A+1-a\) est quantitas positiva, vel talis qui reddat quantitatem \(h+m+A+B-b\) positivam in casu eo ubi \(A+1-a=0\). Tunc erit

\[
\begin{aligned}
P(t-(m+h-1)) & =t^{\lambda+1}+(A+1-m-h) t^{\lambda}+(B-A(m+h-1)) t^{\lambda-1} \text { etc. } \\
p(t-(m+h)) & =t^{\lambda+1}+(a-m-h) t^{\lambda}+(b-a(m+h)) t^{\lambda-1} \text { etc. }
\end{aligned}
\]

ubi vel \(A+1-m-h-(a-m-h)\) erit quantitas positiva, vel, si haec fit \(=0\), saltem \(B-A(m+h-1)-(b-a(m+h))\) positiva erit. Hinc (per I) pro quantitate \(t\) assignari poterit valor aliquis \(l\), quem simulac transgressa est, valores fractionis \(\frac{P(t-(m+h-1))}{p(t-(m+h))}\) semper fient positivi atque unitate maiores. Sit \(n\) integer
maior quam \(l\) simulque maior quam \(h\), sintque termini seriei \(M, M^{\prime}, M^{\prime \prime}, M^{\prime \prime \prime}\) etc. qui respondent valoribus \(t=m+n, t=m+n+1, t=m+n+2\) etc., hi \(N, N^{\prime}, N^{\prime \prime}, N^{\prime \prime \prime}\) etc. Erunt itaque

\[
\frac{(n+1-h) N^{\prime}}{(n-h) \boldsymbol{N}}, \quad \frac{(n+2-h) N^{\prime \prime}}{(n+1-h) N^{\prime}}, \quad \frac{(n+3-h) N^{\prime \prime \prime}}{(n+2-h) N^{\prime \prime}} \text { etc. }
\]

quantitates positivae unitate maiores, unde

\[
N^{\prime}>\frac{(n-h) N}{n+1-\bar{h}}, \quad N^{\prime \prime}>\frac{(n-h) N}{n+2-h}, \quad N^{\prime \prime \prime}>\frac{(n-h) N}{n+3-h} \text { etc. }
\]

adeoque summa seriei \(N+N^{\prime}+N^{\prime \prime}+N^{\prime \prime \prime}+\) etc. maior summa seriei

\[
(n-h) N\left(\frac{1}{n-h}+\frac{1}{n+1-h}+\frac{1}{n+2-h}+\frac{1}{n+3-h}+\text { etc. }\right)
\]

quotcunque termini colligantur. Sed posterior series, crescente terminorum numero in infinitum, omnes limites egreditur, quum summa seriei \(1+\frac{1}{2}+\frac{1}{3}+\frac{1}{4}+\) etc. quam infinitam esse constat etiam infinita maneat, si ab initio termini \(1+\frac{1}{2}+\frac{1}{3}\) +etc. \(+\frac{1}{n-1-h}\) rescindantur. Quare summa seriei \(N+N^{\prime}+N^{\prime \prime}+N^{\prime \prime \prime}+\) etc., adeoque etiam summa huius \(M+M^{\prime}+M^{\prime \prime}+M^{\prime \prime \prime}+\) etc., cuius pars est illa, ultra omnes limites crescit.

V. Quoties autem ' \(A-a\) est quantitas negativa absolute maior quam unitas, summa seriei \(M+M^{\prime}+M^{\prime \prime}+M^{\prime \prime \prime}+\) etc. in infinitum continuatae certo erit finita. Sit enim \(h\) quantitas positiva minor quam \(a-A-1\), demonstrabiturque per ratiocinia similia, assignari posse valorem aliquem \(l\) quantitatis \(t\), ultra quem fractio \(\frac{p t}{p(t-h-1)}\) semper adipiscatur valores positivos unitate minores. Quodsi iam pro \(n\) accipitur numerus integer ipsis \(l, m, h+1\) maior, terminique seriei \(M, M^{\prime}, M^{\prime \prime}, M^{\prime \prime \prime}\) etc., valoribus \(t=n, t=n+1, t=n+2\) etc. respondentes, designantur per \(N, N^{\prime}, N^{\prime \prime}\) etc., erit

\[
N^{\prime}<\frac{n-h-1}{n} \cdot N, \quad N^{\prime \prime}<\frac{(n-h-1)(n-h)}{n(n+1)} \cdot N^{\prime} \text { etc. }
\]

adeoque summa seriei \(N+N^{\prime}+N^{\prime \prime}+\) etc., quotcunque termini colligantur, minor producto ex \(N\) in summam totidem terminorum seriei

\[
1+\frac{n-h-1}{n}+\frac{(n-h-1)(n-h)}{n(n+1)}+\frac{(n-h-1)(n-h)(n-h+1)}{n(n+1)(n+2)} \text { etc. }
\]

Huius vero summa pro quolibet terminorum numero facile assignari potest; est scilicet

terminus primus

\[
=\frac{n-1}{h}-\frac{n-h-1}{h}
\]

sümma duorum terminorum \(=\frac{n-1}{h}-\frac{(n-h-1)(n-h)}{h n}\)

summa trium terminorum \(=\frac{n-1}{h}-\frac{(n-h-1)(n-h)(n-h+1)}{h n(n+1)}\) etc.

et quum pars altera (per II) formet seriem ultra omnes limites decrescentem, summailla in infinitum continuata statui debet \(=\frac{n-1}{h}\). Hinc \(N+N^{\prime}+N^{\prime \prime}\) etc. in infinitum continuata semper manebit minor quam \(\frac{N(n-1)}{h}\), et proin \(M+M^{\prime}+M^{\prime \prime}\) etc. certo ad summam finitam converget.

Q. E. D.

VI. Ut ea, quae generaliter de serie \(M, M^{\prime}, M^{\prime \prime}\) etc. demonstravimus, ad coëfficientes potestatum \(x^{m}, x^{m+1}, x^{m+2}\) etc. in serie \(\boldsymbol{F}(\alpha, b, \gamma, x)\), applicentur, statuere oportebit \(\lambda=2, A=\alpha+b, B=\alpha b, a=\gamma+1, b=\gamma\), unde quinque assertiones in art. praec. propositae sponte demanant.

17.

Disquisitio itaque de indole summae seriei \(F(\alpha, b, \gamma, 1)\) natura sua restringitur ad casum, quo \(\gamma-\alpha-b\) est quantitas positiva, ubi illa summa semper exhibet quantitatem finitam. Praemittimus autem observationem sequentem. Si coëfficientes seriei \(1+a x+b x x+c x^{3}+\) etc. \(=S\) inde a certo termino ultra omnes limites decrescunt, productum

\[
(1-x) S=1+(a-1) x+(b-a) x x+(c-b) x^{3}+\text { etc. }
\]

pro \(x=1\) statuere oportet \(=0\), etiamsi summa ipsius seriei \(\mathbb{S}\) infinite magna evadat. Quoniam enim collectis duobus terminis summa fit \(=a\), collectis tribus \(=b\), collectis quatuor \(=c\) etc., limes summae in infinitum continuatae est \(=0\). Quoties itaque \(\gamma-\alpha-b\) est quantitas positiva, statuere oportet \((1-x) F(\alpha, b, \gamma-1, x)=0\) pro \(x=1\), unde per aequationem 15 art. 7 habebimus

\[
0=\gamma(\alpha+b-\gamma) F(\alpha, b, \gamma, 1)+(\gamma-\alpha)(\gamma-b) F(\alpha, b, \gamma+1.1), \text { sive }
\]

\[
F(\alpha, b, \gamma, 1)=\frac{(\gamma-\alpha)(\gamma-b)}{\gamma(\gamma-\alpha-b)} F(\alpha, b, \gamma+1,1)
\]

Quare quum perinde fiat

\[
\begin{aligned}
& F(\alpha, b, \gamma+1,1)=\frac{(\gamma+1-\alpha)(\gamma+1-b)}{(\gamma+1)(\gamma+1-\alpha-b)} F(\alpha, b, \gamma+2,1) \\
& F(a, b, \gamma+2,1)=\frac{(\gamma+2-\alpha)(\gamma+2-b)}{(\gamma+2)(\gamma+2-\alpha-\varnothing)} F(\alpha, b, \gamma+3,1)
\end{aligned}
\]

et sic porro, erit generaliter, \(k\) denotante integrum positivum quemcunque

\[
\begin{aligned}
& F(\alpha . b, \gamma, 1) \text { aequalis producto ex } \boldsymbol{F}(\alpha, b, \gamma+k, 1) \\
\text { in } & (\gamma-\alpha)(\gamma+1-\alpha)(\gamma+2-\alpha) \ldots(\gamma+k-1-\alpha) \\
\text { in } & (\gamma-b)(\gamma+1-b)(\gamma+2-b) \ldots(\gamma+k-1-b)
\end{aligned}
\]

diviso per productum

ex \(\quad \gamma(\gamma+1)(\gamma+2) \ldots(\gamma+k-1)\)

in \(\quad(\gamma-\alpha-b)(\gamma+1-\alpha-b)(\gamma+2-\alpha-b) \ldots(\gamma+k-1-\alpha-b)\)

18.

Introducamus abhinc sequentem notationem :

\([38]\)

\[
\Pi(k, z)=\frac{1 \cdot 2 \cdot 3 \ldots \ldots k}{(z+1)(z+2)(z+3) \ldots(z+k)} k^{z}
\]

ubi \(k\) natura sua subintelligitur designare integrum positivum, qua restrictione \(\Pi(k, z)\) exhibet functionem duarum quantitatum \(k, z\) prorsus determinatam. Hoc modo facile intelligetur, theorema in fine art. praec. propositum ita exhiberi posse

\[
F(\alpha, b, \gamma, 1)=\frac{\Pi(k, \gamma-1) \cdot \Pi(k, \gamma-\alpha-b-1)}{\Pi(k, \gamma-\alpha-1) \cdot \Pi(k, \gamma-b-1)} \cdot F(\alpha, b, \gamma+k, 1)
\]

\section*{19.}
Operae pretium erit, indolem functionis \(\Pi(k, z)\) accuratius perpendere. Quoties \(z\) est integer negativus, functio manifesto valorem infinite magnum obtinet, simulac ipsi \(k\) tribuitur valor satis magnus. Pro valoribus integris ipsius \(\boldsymbol{z}\) non negativis autem habemus

\[
\begin{aligned}
& \Pi(k, 0)=1 \\
& \Pi(k, 1)=\frac{1}{1+\frac{1}{k}} \\
& \Pi(k, 2)=\frac{1 \cdot 2}{\left(1+\frac{1}{k}\right)\left(1+\frac{2}{k}\right)} \\
& \Pi(k, 3)=\frac{1 \cdot 2 \cdot 3}{\left(1+\frac{1}{k}\right)\left(1+\frac{2}{k}\right)\left(1+\frac{3}{k}\right)}
\end{aligned}
\]

etc. sive generaliter

\[
\Pi(k, z)=\frac{1}{\left(1+\frac{1}{k}\right)\left(1+\frac{2}{k}\right)\left(1+\frac{3}{k}\right) \ldots\left(1+\frac{z}{k}\right)}
\]

\[
\text { CIRCA SERIEM INFINITAM } 1+\frac{\alpha b}{1 \cdot \gamma} x+\text { ETC. }
\]

Generaliter autem pro quovis valore ipsius \(\boldsymbol{z}\) habemus

\[
\begin{aligned}
& \Pi(k, z+1)=\Pi(k, z) \cdot \frac{1+z}{1+\frac{1+z}{k}} \\
& \Pi(k+1, z)=\Pi(k, z) \cdot\left\{\frac{\left(1+\frac{1}{k}\right)^{z+1}}{1+\frac{1+z}{k}}\right\}
\end{aligned}
\]

adeoque, quum \(\Pi(1, z)=\frac{1}{z+1}\),

\[
\Pi(k, z)=\frac{1}{z+1} \cdot \frac{2^{z+1}}{1^{z} \cdot(2+z)} \cdot \frac{3^{z+1}}{2^{z}(3+z)} \cdot \frac{4^{z+1}}{3^{z}(4+z)} \cdots \cdot \frac{k^{z+1}}{(k-1)^{z}(k+z)}
\]

20.

Imprimis vero attentione dignus est limes, ad quem pro valore dato ipsius \(z\) functio \(\Pi(k, z)\) continuo converget, dum \(k\) in infinitum crescit. Sit primo \(h\) valor finitus ipsius \(k\) maior quam \(z\), patetque, si \(k\) transire supponatur ex \(h\) in \(h+1\), logarithmum ipsius \(\Pi(k, z)\) accipere incrementum, quod per seriem convergentem sequentem exprimatur

\[
\frac{z(1+z)}{2(h+1)^{2}}+\frac{z(1-z z)}{3(h+1)^{3}}+\frac{z\left(1+z^{3}\right)}{4(h+1)^{2}}+\frac{z\left(1-z^{2}\right)}{5(h+1)^{3}}+\text { etc. }
\]

Si itaque \(k\) e valore \(h\) transit in \(h+n\), logarithmus ipsius \(\Pi(k, z)\) accipiet incrementum

\[
\begin{aligned}
& \quad \frac{1}{2} z(1+z)\left(\frac{1}{(h+1)^{2}}+\frac{1}{(h+2)^{2}}+\frac{1}{(h+3)^{2}}+\text { etc. }+\frac{1}{(h+n)^{2}}\right) \\
& +\frac{1}{3} z(1-z z)\left(\frac{1}{(h+1)^{2}}+\frac{1}{(h+2)^{3}}+\frac{1}{(h+3)^{3}}+\text { etc. }+\frac{1}{(h+n)^{3}}\right) \\
& +\frac{1}{4} z\left(1+z^{3}\right)\left(\frac{1}{(h+1)^{2}}+\frac{1}{(h+2)^{2}}+\frac{1}{(h+3)^{2}}+\text { etc. }+\frac{1}{(h+n)^{2}}\right) \\
& + \text { etc. }
\end{aligned}
\]

quod semper finitum manere, etiamsi \(n\) in infinitum crescat, facile demonstrari potest. Quare nisi iam in \(\Pi(h, z)\) factor infinitus affuerit, i. e nisi \(z\) sit numerus integer negativus, limes ipsius \(\Pi(k, z)\) pro \(k=\infty\) certo erit quantitas finita. Manifesto itaque \(\Pi(\infty, z)\) tantummodo a \(z\) pendet, sive functionem ipsius \(z\) ex asse determinatam exhibet, quae abhinc simpliciter per \(\Pi z\) denotabitur. Definimus itaque functionem \(\Pi z\) per valorem producti

\[
\frac{1 \cdot 2 \cdot 3 \cdot \ldots \cdot k \cdot k^{z}}{(z+1)(z+2)(z+3) \cdots \cdot(z+k)}
\]

pro \(k=\infty\), aut si mavis per limitem producti infiniti

\[
\frac{1}{z+1} \cdot \frac{2^{z+1}}{1^{z}(2+z)} \cdot \frac{3^{z+1}}{2^{z}(3+z)} \cdot \frac{4^{z+1}}{3^{z}(4+z)} \text { etc. }
\]

21.

Ex aequatione 41 statim sequitur aequatio fundamentalis

\[
\Pi(z+1)=(z+1) \Pi z
\]

unde generaliter, designante \(n\) integrum positivum quemcunque

\[
\Pi(z+n)=(z+1)(z+2)(z+3) \ldots \ldots(z+n) \Pi z
\]

Pro valore integro negativo ipsius \(z\) erit valor functionis \(\Pi z\) infinite magnus; pro valoribus integris non negativis habemus

\[
\begin{aligned}
& \Pi_{0}=1 \\
& \Pi_{1}=1 \\
& \Pi_{2}=2 \\
& \Pi_{3}=6 \\
& \Pi_{4}=24 \text { etc. }
\end{aligned}
\]

atque generaliter

\[
\Pi z=1.2 .3 \ldots z
\]

Sed male haec proprietas functionis nostrae tamquam ipsius definitio venditaretur, quippe quae natura sua ad valores integros restringitur, et praeter functionem nostram infinitis aliis (e. g. \(\cos 2 \pi z . \Pi z, \cos \pi z^{2 n} \Pi z\) etc., denotante \(\pi\) semiperipheriam circuli, cuius radius \(=1\) ) communis est.

22.

Functio \(\Pi(k, z)\), etiamsi generalior videatur quam. \(\Pi z\), tamen abhinc nobis superflua erit, quum facile ad posteriorem reducatur. Colligitur enim e combinatione aequationum \(38,45,46\).

\[
\Pi(k, z)=\frac{k^{z} \Pi k \cdot \Pi z}{\Pi(k+z)}
\]

Ceterum nexus harum functionum cum is, quas clar. KraMP facultates nu-
mericas nominavit, per se obvius est. Scilicet facultas numerica, quam hic auctor per \(a^{b I_{c}}\) designat, in signis nostris est

\[
=\frac{c^{b} b^{\frac{a}{c}-1}}{\Pi\left(b, \frac{a}{c}-1\right)}=\frac{c^{b} \Pi\left(\frac{a}{c}+b-1\right)}{\Pi\left(\frac{a}{c}-1\right)}
\]

Sed consultius videtur, functionem unius variabilis in analysin introducere, quam functionem trium variabilium, praesertim quum hanc ad illam reducere liceat.

Continuitas functionis \(\Pi z\) interrumpitur, quoties ipsius valor fit infinite magnus, i. e. pro valoribus intégris negativis ipsius \(z\). Erit itaque illa positiva a \(z=-1\) usque ad \(z=\infty\), et quum pro utroque limite \(\Pi z\) obtineat valorem infinite magnum, inter ipsos dabitur valor minimus, quem esse \(=\mathbf{0 , 8 8 5 6 0 2 4}\) atque respondere valori \(z=0,4616321\) invenimus. Inter limites \(z=-1\) et \(z=-2\), valor functionis \(\Pi z\) fit negativus, inter \(z=-2\) atque \(z=-3\) iterum positivus et sic porro, uti ex aequ. 44 sponte sequitur. Porro patet, si omnes valores functionis \(\Pi z\) inter limites arbitrarios unitate differentes e. g. a \(z=0\) usque ad \(z=1\) pro notis habere liceat, valorem functionis pro quovis alio valore reali ipsius \(z\) adiumento aequationis 45 facile inde deduci posse. Ad hunc finem construximus tabulam, ad ealcem huius sectionis annexam, quae ad figuras viginti exhibet logarithmos briggicos functionis \(\Pi z\), pro \(z=0\) usque ad \(z=1\) per singulas partes centesimas summa cura computatos, ubi tamen monendum, figuram ultimam vigesimam interdum una duabusve unitatibus erroneam esse posse.

\section*{24.}
Quum limes functionis \(F(\alpha, b, \gamma+k, 1)\), crescente \(k\) in infinitum, manifesto sit unitas, aequatio 39 transit in hanc

\[
F(\alpha, b, \gamma, 1)=\frac{\Pi(\gamma-1) \cdot \Pi(\gamma-\alpha-b-1)}{\Pi(\gamma-\alpha-1) \cdot \Pi(\gamma-b-1)}
\]

quae formula exhibet solutionem completam quaestionis, quae obiectum huius sectionis constituit. Sponte hine sequuntur aequationes elegantes:

\[
\begin{aligned}
& F(\alpha, b, \gamma, 1)=F(-\alpha,-b, \gamma-\alpha-b, 1) \\
& F(\alpha . b, \gamma, 1) \cdot F(-\alpha, b, \gamma-\alpha, 1)=1 \\
& F(\alpha, b, \gamma, 1) \cdot F(\alpha,-b, \gamma-b, 1)=1
\end{aligned}
\]

in quarum prima \(\gamma\), in secunda \(\gamma-b\), in tertia \(\gamma-\alpha\) debet esse quantitas positiva.

25.

Applicemus formulam 48 ad quasdam ex aequationibus art. 5. Formula XIII, statuendo \(t=90^{0}=\frac{1}{2} \pi\), fit \(\frac{1}{2} \pi=F\left(\frac{1}{2}, \frac{1}{2}, \frac{3}{2}, 1\right)\), sive aequivalet aequationi notae

\[
\frac{1}{2} \pi=1+\frac{1.1}{2.3}+\frac{1.1 .3}{2.4 .5}+\frac{1.1 .3 .5}{2.4 \cdot 6.7}+\text { etc. }
\]

Quare quum per formulam 48 habeatur \(F\left(\frac{1}{2}, \frac{1}{2}, \frac{3}{2}, 1\right)=\frac{\Pi_{\frac{1}{3}} . \Pi\left(-\frac{1}{2}\right)}{\Pi_{0} \cdot \Pi_{0}}\), atque sit \(\Pi 0=1, \Pi_{\frac{1}{2}}=\frac{1}{2} \Pi\left(-\frac{1}{2}\right)\), fit \(\pi=\left(\Pi\left(-\frac{1}{2}\right)\right)^{2}\) sive

\[
\begin{aligned}
& \Pi\left(-\frac{1}{2}\right)=V \pi \\
& \Pi_{\frac{1}{2}}=\frac{1}{2} V \pi
\end{aligned}
\]

Formula XVI art. 5 , quae aequivalet aequationi notae

\[
\sin n t=n \sin t-\frac{n(n n-1)}{2 \cdot 3} \sin t^{3}+\frac{n(n n-1)(n n-9)}{2 \cdot 3 \cdot 4 \cdot 5} \sin t^{5}-\text { etc. }
\]

atque generaliter pro quovis valore ipsius \(n\) locum habet, si modo \(t\) limites \(-90^{\circ}\) et \(+90^{\circ}\) non transgrediatur, dat pro \(t=\frac{1}{2} \pi\)

\[
\cdot \sin \frac{n \pi}{2}=\frac{n \Pi_{\frac{1}{2}} \cdot \Pi\left(-\frac{1}{2}\right)}{\Pi\left(-\frac{1}{2} n\right) \cdot \Pi_{\frac{1}{2} n}}
\]

unde deducitur formula elegans

\[
\Pi_{\frac{1}{2}} n \cdot \Pi\left(-\frac{1}{2} n\right)=\frac{\frac{1}{2} n \pi}{\sin \frac{1}{2} n \pi} \text {, sive statuendo } n=2 z
\]

\[
\begin{aligned}
\Pi(-z) \cdot \Pi(+z) & =\frac{z \pi}{\sin z \pi} \\
\Pi(-z) \cdot \Pi(z-1) & =\frac{\pi}{\sin z \pi}
\end{aligned}
\]

nec non scribendo \(z+\frac{1}{2}\) pro \(z\)

\[
\Pi\left(-\frac{1}{2}+z\right) \cdot \Pi\left(-\frac{1}{2}-z\right)=\frac{\pi}{\cos z \pi}
\]

CIRCA SERLEM INFINTTAM, \(1+\frac{a q}{1 \cdot \gamma} x+\) ETC.

E combinatione formulae 54 cum definitione functionis \(\Pi\) sequitur, \(\frac{z \pi}{\sin z \pi}\) esse limitem producti infiniti

\[
\frac{(1 \cdot 2 \cdot 3 \cdot 4 \ldots \ldots k)^{2}}{(1-z z)(4-z z)(9-z z) \ldots \ldots(k \overline{k-z z)}}
\]

crescente \(k\) in infinitum, adeoque

\[
\sin z \pi=z \pi(1-z z)\left(1-\frac{z z}{4}\right)\left(1-\frac{z z}{9}\right) \text { etc. in inf. }
\]

similique modo ex 56 deducitur

\[
\cos z \pi=(1-4 z z)\left(1-\frac{4 z z}{9}\right)\left(1-\frac{4 z z}{25}\right) \text { etc. in inf. }
\]

formulae notissimae, quae ab analystis per methodos prorsus diversas erui solent.

\section*{26.}
Designante \(n\) numerum integrum, valor expressionis

\[
\frac{n^{n z} \Pi(k, z) \cdot \Pi\left(k, z-\frac{1}{n}\right) \cdot \Pi\left(k, z-\frac{2}{n}\right) \ldots \ldots \Pi\left(k, z-\frac{n-1}{n}\right)}{\Pi(n k, n z)}
\]

rite collectus invenitur

\[
=\frac{(1 \cdot 2 \cdot 3 \ldots k)^{n} n^{n k}}{1 \cdot 2 \cdot 3 \cdot \ldots n k \cdot k^{\frac{1}{2}(n-1)}}
\]

adeoque a \(z\) est independens, sive idem manebit,' quicunque valor ipsi \(z\).tribuatur. Exhiberi poterit itaque, quoniam \(\Pi(k, 0)=\Pi(n k, 0)=1\), per productum

\[
\Pi\left(k,-\frac{1}{n}\right) \cdot \Pi\left(k,-\frac{2}{n}\right) \cdot \Pi\left(k,-\frac{3}{n}\right) \ldots \Pi\left(k,-\frac{n-1}{n}\right)
\]

Crescente igitur \(k\) in infinitum, nanciscimur

\[
\frac{n^{n z} \Pi z \cdot \Pi\left(z-\frac{1}{n}\right) \cdot \Pi\left(z-\frac{2}{n}\right) \ldots \Pi\left(z-\frac{n-1}{n}\right)}{\Pi n z}=\Pi\left(-\frac{1}{n}\right) \cdot \Pi\left(-\frac{2}{n}\right) \cdot \Pi\left(-\frac{3}{n}\right) \ldots \Pi\left(-\frac{n-1}{n}\right)
\]

Productum ad dextram, in se ipsum ordine factorum inverso multiplicatum, producit, per form. 55 ,

\[
\frac{\pi}{\sin \frac{1}{n} \pi} \cdot \frac{\pi}{\sin \frac{2}{n} \pi} \cdot \frac{\pi}{\sin \frac{3}{n} \pi} \cdots \frac{\pi}{\sin \frac{n-1}{n} \pi}=\frac{(2 \pi)^{n-1}}{n}
\]

Unde habemus theorema elegans

\[
\frac{n^{n z} \Pi z \cdot \Pi\left(z-\frac{1}{n}\right) \cdot \Pi\left(z-\frac{2}{n}\right) \ldots \ldots\left(z-\frac{n-1}{n}\right)}{\Pi n z}=\frac{(2 \pi)^{\frac{1}{2}(n-1)}}{\sqrt{n}}
\]

27.

Integrale \(\int x^{\lambda-1}\left(1-x^{\mu}\right)^{\nu} \mathrm{d} x\), ita acceptum, ut evanescat pro \(x=0\), exprimitur per seriem sequentem, siquidem \(\lambda, \mu\) sunt quantitates positivae:

\[
\frac{x^{2}}{\lambda}-\frac{v x^{\mu+\lambda}}{\mu+\lambda}+\frac{\nu(\nu-1) x^{2 \mu+\lambda}}{1 \cdot 2 \cdot(2 \mu+\lambda)}-\text { etc. }=\frac{x^{\lambda}}{\lambda} F\left(-\nu, \frac{\lambda}{\mu}, \frac{\lambda}{\mu}+1, x^{\mu}\right)
\]

Hinc ipsius valor pro \(x=1\) erit

\[
=\frac{\Pi \frac{\lambda}{\mu} \cdot \Pi \nu}{\lambda \Pi\left(\frac{\lambda}{\mu}+v\right)}
\]

Ex hoc theoremate omnes relationes, quas ill. EuLer olim multo labore evolvit, sponte demanant. Ita e. g. statuendo

\[
\int \frac{\mathrm{d} x}{\sqrt{\left(1-x^{2}\right)}}=A, \quad \int \frac{x x d x}{\sqrt{\left(1-x^{2}\right)}}=B
\]

erit \(A=\frac{\Pi \frac{1}{2} \cdot \Pi\left(-\frac{1}{2}\right)}{\Pi\left(-\frac{1}{2}\right)}, B=\frac{\Pi_{\frac{3}{2}} \cdot \Pi\left(-\frac{1}{2}\right)}{3 \Pi_{\frac{1}{2}}}=\frac{\Pi\left(-\frac{1}{2}\right) \cdot \Pi\left(-\frac{1}{2}\right)}{4 \Pi_{\frac{1}{4}}}\), adeoque \(A B=\frac{1}{4} \pi\). Simul hinc sequitur, quoniam \(\Pi_{\frac{1}{4}} \cdot \Pi\left(-\frac{1}{4}\right)=\frac{\frac{1}{2} \pi}{\sin \frac{1}{2} \pi}=\frac{\pi}{\sqrt{8}}\),

\[
\Pi_{\frac{1}{4}}=\sqrt[4]{ }\left(\frac{1}{8} \pi A A\right)=\sqrt[4]{ } \frac{\pi^{3}}{128 B B}, \quad \Pi\left(-\frac{1}{4}\right)=\sqrt[4]{ } \frac{\pi^{3}}{8 A A}=\sqrt[4]{ }(2 \pi B B)
\]

Valor numericus ipsius \(A\), computante Stirling, habetur \(=1,31102877714605987\), valor ipsius \(B\), secundum eundem auctorem, \(=0,59907011736779611\), ex nostro calculo, artificio peculiari innixo, \(=0,59907011736779610372\).

Generaliter facile ostendi potest, valorem functionis \(\Pi_{z}\), si \(z\) sit quantitas rationalis \(=\frac{m}{\mu}\), denotantibus \(m, \mu\) integros, ex \(\mu-1\) valoribus determinatis talium integralium pro \(x=1\) deduci posse, et quidem permultis modis diversis. Accipiendo enim pro \(\lambda\) numerum integrum atque pro \(\vee\) fractionem, cuius denominator \(=\mu\), valor illius integralis semper reducitur ad tres \(\Pi z\), ubi \(z\) est fractio cum denominatore \(=\mu\); quodvis vero huiusmodi \(\Pi z\) vel ad \(\Pi\left(-\frac{1}{\mu}\right)\), vel ad \(\Pi\left(-\frac{2}{\mu}\right)\), vel ad \(\Pi\left(-\frac{3}{\mu}\right)\) etc. vel ad \(\Pi\left(-\frac{\mu-1}{\mu}\right)\) reduci potest per formulam 45 , siquidem \(z\) revera est fractio; si enim \(z\) est integer, \(\Pi \boldsymbol{z}\) per se constat. Ex illis vero integralium valoribus, generaliter loquendo, quodvis \(\Pi\left(-\frac{m}{\mu}\right)\),
si \(m<\mu\), per eliminationem erui potest \({ }^{*}\) ). Quin adeo semissis talium integralium sufficiet, si formulam 54 simul in auxilium vocamus. Ita e. g. statuendo

\[
\begin{aligned}
& \text { - } \int \frac{\mathrm{d} x}{\sqrt[3]{\left(1-x^{5}\right)}}=C, \int \frac{\mathrm{d} x}{\sqrt[3]{\left(1-x^{5}\right)^{2}}}=D, \quad \int \frac{\mathrm{d} x}{\sqrt[5]{\left(1-x^{5}\right)^{3}}}=E, \quad \int \frac{\mathrm{d} x}{\sqrt[5]{\left(1-x^{5}\right)^{2}}}=F, \text { erit } \\
& C=\Pi_{\frac{1}{5}} \cdot \Pi\left(-\frac{1}{5}\right), \quad D=\frac{\Pi_{\frac{1}{5}} \cdot \Pi\left(-\frac{3}{3}\right)}{\Pi\left(-\frac{1}{5}\right)}, \quad E=\frac{\Pi_{\frac{1}{5}} \cdot \Pi\left(-\frac{3}{3}\right)}{\Pi\left(-\frac{2}{5}\right)}, \quad F=\frac{\Pi_{\frac{1}{5}} \cdot \Pi_{\left(-\frac{2}{3}\right)}}{\Pi\left(-\frac{3}{3}\right)}
\end{aligned}
\]

Hinc propter \(\Pi_{\frac{1}{5}}=\frac{1}{5} \Pi\left(-\frac{4}{5}\right)\), habemus

\[
\begin{aligned}
& \Pi\left(-\frac{1}{5}\right)=\sqrt[5]{\frac{5}{D E F}}, \quad \Pi\left(-\frac{2}{5}\right)=\sqrt[5]{\frac{5}{E E C^{3} D^{3}}}, \quad \Pi\left(-\frac{3}{5}\right)=\sqrt[5]{\frac{125 C C D D E E}{F^{3}}}, \\
& \left.\Pi\left(-\frac{4}{5}\right)=\sqrt[5]{(625 C D E F}\right)
\end{aligned}
\]

Formulae 54, 55 adhuc suppeditant

\[
C=\frac{\pi}{\sin \frac{1}{8} \pi}, \quad \frac{D}{F}=\frac{\sin \frac{3}{3} \pi}{\sin \frac{2}{3} \pi}
\]

ita ut duo integralia \(D, E\), vel \(E\) et \(\boldsymbol{F}\) sufficiant, ad omnes valores \(\Pi\left(-\frac{1}{3}\right)\), \(\Pi\left(-\frac{2}{5}\right)\) etc. computandos.

28.

Statuendo \(y=v x\), atque \(\mu=1, \frac{\Pi \lambda . \Pi v}{\lambda \Pi(\lambda+v)}\) erit valor integralis \(\int \frac{y^{2-1}\left(1-\frac{y}{v}\right)^{\nu} \mathrm{d} y}{\nu^{2}}\) ab \(y=0\) usque ad \(y=\nu\), sive valor integralis \(\int y^{\lambda-1}\left(1-\frac{y}{v}\right)^{v} \mathrm{~d} y\) inter eosdem limites \(=\frac{\nu^{2} \Pi \lambda \cdot \Pi v}{\lambda \Pi(\lambda+\nu)}=\frac{\Pi(\nu, \lambda)}{\lambda}\) (form. 47), siquidem \(\nu\) denotet integrum. Iam crescente \(\nu\) in infinitum, limes ipsius \(\Pi(\nu, \lambda)\) erit \(=\Pi \lambda\), limes ipsius \(\left(1-\frac{y}{v}\right)^{\nu}\) autem \(e^{-y}\), denotante \(e\) basin logarithmorum hyperbolicorum. Quamobrem si \(\lambda\) est positiva, \(\frac{\Pi \lambda}{\lambda}\) sive \(\Pi(\lambda-1)\) exprimet integrale \(\int y^{\lambda-1} e^{-y} \mathrm{~d} y\) ab \(y=0\) usque ad \(y=\infty\), sive scribendo \(\lambda\) pro \(\lambda-1\), \(\Pi \lambda\) est valor integralis \(\int y^{\lambda} e^{-y} \mathrm{~d} y\) ab \(y=0\) usque ad \(y=\infty\), si \(\lambda+1\) est quantitas positiva.

Generalius statuendo \(y=z^{\alpha}, \alpha \lambda+\alpha-1=b\), transit \(\int y^{\lambda} e^{-y} \mathrm{~d} y\) in \(\int \alpha z^{6} e^{-z^{\alpha}} \mathrm{d} z\), quod itaque inter limites \(z=0\) atque \(z=\infty\) sumtum exprimetur per \(\Pi\left(\frac{b+1}{a}-1\right)\)

sive

Valor integralis \(\int z^{b} e^{-z^{\alpha}} \mathrm{d} z\), a \(z=0\) usque ad \(z=\infty\) fit \(=\frac{\Pi\left(\frac{6+1}{a}-1\right)}{\alpha}=\frac{\Pi \frac{\varepsilon+1}{a}}{b+1}\) si modo \(\alpha\) atque \(b+1\) ' sunt quantitates positivae (si utraque est negativa, in-

*) Haec eliminatio, si pro quantitatibus ipsis logarithmos introducimus, aequationibus tantammodo linearibus applicanda erit.
tegrale per \(-\frac{n \frac{b+1}{a}}{b+1}\) exprimetur). Ita e. g. pro \(b=0, \alpha=2\), valor integralis \(\int e^{-z z} \mathrm{~d} z\) invenitur \(=\Pi_{\frac{1}{2}}=\frac{1}{2} \vee \tau\).

IIl. Euler pro summa \(\operatorname{logarithmorum} \log 1+\log 2+\log 3+\) etc. \(+\log z\) eruit seriem \(\left(z+\frac{1}{2}\right) \log z-z+\frac{1}{2} \log 2 \pi+\frac{\mathfrak{A}}{1.2 z}-\frac{\mathfrak{B}}{3.4 z^{3}}+\frac{\mathbb{C}}{5.6 z^{3}}-\) etc. ubi \(\mathfrak{A}=\frac{1}{6}, \mathfrak{b}=\frac{1}{30}, \quad \mathfrak{C}=\frac{1}{42}\) etc. sunt numeri Bernodludani. Per hanc.itaque seriem exprimitur \(\log \Pi z\); etiamsi enim primo aspectu haec conclusio ad valores integros restricta videatur, tamen rem propius contemplando invenietur, evolutionem ab Eulero adhibitam (Instit. Calc. Diff. Cap.vi. 159) saltem ad valores positivos fractos eodem iure applicari posse, quo ad integros: supponit enim tantummodo. functionem ipsius \(\boldsymbol{z}\), in seriem evolvendam, esse talem, ut ipsius diminutio, si \(\boldsymbol{z}\) transeat in \(\boldsymbol{z}-1\), exhiberi possit per theorema TAYLORI, simulque ut eadem diminutio sit \(=\log z\). Conditio prior innititur continuitati functionis, adeoque locum non habet pro valoribus negativis ipsius \(\boldsymbol{z}\), ad quos proin seriem illam extendere non licet: conditio posterior autem functioni \(\log \Pi z\) generaliter competit sine restrictione ad valores integros ipsius \(\boldsymbol{z}\). Statuemus itaque

[58] \(\log \Pi z=\left(z+\frac{1}{2}\right) \log z-z+\frac{1}{2} \log 2 \pi+\frac{\mathfrak{A}}{1.2 z}-\frac{\mathfrak{B}}{3.4 z^{3}}+\frac{\mathfrak{C}}{5.6 z^{3}}-\frac{\mathfrak{D}}{7.8 z^{7}}+\) etc.

Quum hinc quoque habeatur

\(\log \Pi 2 z=\left(2 z+\frac{1}{2}\right) \log 2 z-2 z+\frac{1}{2} \log 2 \pi+\frac{\mathfrak{A}}{1.2 .2 z}-\frac{\mathfrak{B}}{3.4 .9 z^{3}}+\frac{\mathfrak{C}}{3.6 .32 z^{5}}-\frac{\mathfrak{D}}{7.8 .128 z^{7}}+\) etc. atque per formulam 57 , statuendo \(n=2\),

\(\log \Pi\left(z-\frac{1}{2}\right)=\log \Pi 2 z-\log \Pi z-\left(2 z+\frac{1}{2}\right) \log 2+\frac{1}{2} \log 2 \pi\), fit

[59] \(\log \Pi\left(z-\frac{1}{2}\right)=z \log z-z+\frac{1}{2} \log 2 \pi-\frac{\mathfrak{A}}{1.2 .2 z}+\frac{7 \mathfrak{B}}{3.4 .8 z^{3}}-\frac{31 \mathfrak{C}}{5.6 .32 z^{5}}+\frac{127 \mathfrak{D}}{7.8 .128 z^{7}}\) - etc.

Hae duae series pro valoribus magnis ipsius \(\boldsymbol{z}\) ab initio satis promte convergunt, ita ut summam approximatam commode satisque exacte colligere liceat: attamen probe notandum est, pro quovis valore dato ipsius \(z\), quantumvis magno, praecisionem limitatam tantummodo obtineri posse, quum numeri BERNOULLIANI seriem hypergeometricam constituant, adeoque series illae, si modo satis longe extendantur, certo e convergentibus divergentes evadant. Ceterum negari nequit, theoriam talium serierum divergentium adhuc quibusdam difficultatibus premi, de quibus forsan alia occasione pluribus commentabimur.

CLRCA SERIEM INFINITAM \(1+\frac{28}{1 \cdot y} x+\) ETC.

30.

E formula 38 sequitur

\[
\frac{\Pi(k, z+\infty)}{\Pi(k, z)}=\frac{z+1}{z+1+\omega} \cdot \frac{z+2}{z+2+\omega} \cdot \frac{z+3}{z+3+\infty} \cdots \cdot \frac{z+k}{z+k+\infty} \cdot k^{\infty}
\]

unde sumtis logarithmis, in series infinitas evolutis, prodit

\([60] \log \Pi(k, z+\omega)=\log \Pi(k, z)\)

\[
\begin{aligned}
& +\omega\left(\log k-\frac{1}{z+1}-\frac{1}{z+2}-\frac{1}{z+3}-\text { etc. }-\frac{1}{z+k}\right) \\
& +\frac{1}{2} \omega \omega\left(\frac{1}{(z+1)^{2}}+\frac{1}{(z+2)^{2}}+\frac{1}{(z+3)^{3}}+\text { etc. }+\frac{1}{(z+k)^{2}}\right) \\
& -\frac{1}{3} \omega^{3}\left(\frac{1}{(z+1)^{3}}+\frac{1}{(z+2)^{3}}+\frac{1}{(z+3)^{3}}+\text { etc. }+\frac{1}{(z+k)^{3}}\right) \\
& + \text { etc. in inf. }
\end{aligned}
\]

Series, hic in \(\omega\) multiplicata, quae, si magis placet, ita etiam exhiberi potest,

\(-\frac{1}{z+1}+\log 2=\frac{1}{z+2}+\log \frac{3}{2}-\frac{1}{z+3}+\log \frac{1}{3}-\frac{1}{z+4}+\log \frac{3}{z}\)-etc. \(+\log _{\frac{k}{k-1}}^{\frac{1}{z+k}}\)

e terminorum multitudine finita constat, crescente autem \(k\) in infinitum, ad limitem certum converget, qui novam functionum transscendentium speciem nobis sistit, in posterum per \(\Psi z\) denotandam.

Designando porro summas serierum sequentium, in infinitum extensarum,

\[
\begin{aligned}
& \frac{1}{(z+1)^{2}}+\frac{1}{(z+2)^{2}}+\frac{1}{(z+3)^{2}}+\text { etc. } \\
& \frac{1}{(z+1)^{3}}+\frac{1}{(z+2)^{3}}+\frac{1}{(z+3)^{3}}+\text { etc. } \\
& \frac{1}{(z+1)^{2}}+\frac{1}{(z+2)^{2}}+\frac{1}{(z+3)^{2}}+\text { etc. } \\
& \text { etc. }
\end{aligned}
\]

resp. per \(\boldsymbol{P}, \boldsymbol{Q}, \boldsymbol{R}\) etc. (pro quibus signa functionalia introducere minus necessarium videtur), habebimus

[61] \(\log \Pi(z+\omega)=\log \Pi z+\omega \Psi z+\frac{1}{2} \omega \omega P-\frac{1}{3} \omega^{3} Q+\frac{1}{4} \omega^{4} R-\) etc.

Manifesto functio \(\Psi z\) erit functio derivata prima functionis \(\log \Pi z\), adeoque

\[
\frac{\mathrm{d} \Pi z}{\mathrm{~d} z}=\Pi z \cdot \Psi z
\]

Perinde erit \(P=\frac{\mathrm{d} \Psi z}{\mathrm{~d} z}, Q=-\frac{\mathrm{dd} \Psi_{z}}{2 \mathrm{~d} z^{2}}, R=+\frac{\mathrm{d}^{3} \Psi z}{2 \cdot 3 \mathrm{~d} z^{3}}\) etc.

31.

Functio \(\Psi z\) aeque fere memorabilis est atque functio \(\Pi z\), quapropter insigniores relationes ad illam spectantes hic colligemus. E differentiatione aequationis 44 fit

unde

\[
\Psi(z+1)=\Psi z+\frac{1}{z+1}
\]

\[
\Psi(z+n)=\Psi z+\frac{1}{z+1}+\frac{1}{z+2}+\frac{1}{z+3}+\text { etc. }+\frac{1}{z+n}
\]

Huius adiumento a valoribus minoribus ipsius \(z\) ad maiores progredi, vel a maioribus ad minores regredi licet: pro valoribus maioribus positivis ipsius \(\boldsymbol{z}\) functionis valores numerici satis commode per formulas sequentes e differentiatione aequationum 58, 59 oriundas computantur, de quibus tamen eadem sunt tenenda, quae in art. 29 circa formulas 58 et 59 monuimus:

\[
\begin{aligned}
\Psi z & =\log z+\frac{1}{2 z}-\frac{\mathfrak{A}}{2 z z}+\frac{\mathfrak{B}}{4 z^{2}}-\frac{\mathfrak{E}}{6 z^{\mathrm{e}}}+\text { etc. } \\
\Psi\left(z-\frac{1}{2}\right) & =\log z+\frac{\mathfrak{A}}{2.2 z z}-\frac{7 \mathfrak{B}}{4 \cdot 8 z^{4}}+\frac{31 \mathcal{E}}{6.32 z^{6}}-\text { etc. }
\end{aligned}
\]

Ita pro \(z=10\) computavimus

\[
\Psi z=2,3517525890 \quad 6672110764 \quad 743
\]

unde regredimur ad

\[
\left.\Psi_{0}=-0,57721566490153286060 \quad 653^{*}\right)
\]

Pro valore integro positivo ipsius \(\boldsymbol{z}\) fit generaliter

\[
\Psi z=\Psi 0+1+\frac{1}{2}+\frac{1}{3}+\text { etc. }+\frac{1}{z}
\]

Pro valore integro negativo autem manifesto \(\Psi z\) fit quantitas infinite magna.

*) Quum hic valor inde a figura vigesima discrepet ab eo quem computavit clar. Mascheronr in Adnotat. ad Euleri Calculum Integr., adhortatus sum Fridericum Bernhardí Gothofredum Nicurar, iuvenem in calculo indefessum, at computum illum repeteret ulteriusque extenderet. Invenit itaque per calculum duplicem, scilicet descendens tum a \(z=50\) tum a \(z=100\),

\[
\Psi_{0}=-0,57721560490153286060 \quad 65120900824024310421
\]

Eidem calculatori exercitatissimo etiam debetur tabulae ad finem huius Sectionis annexae pars altera, exhibens valores functionis \(\Psi z\) ad 18 figuras (quarum ultima haud certa), pro omnibus valoribus ipsius \(z\) a 0 usque ad 1 per singulas partes centesimas. Ceterum methodi, per quas utraque tabula constructa est, innituntur partim theorematibus quae hic traduntur, partim calculi artificis singularibus, quae alia occasione proferemus.

32.

Formula 55 nobis suppeditat \(\log \Pi(-z)+\log \Pi(z-1)=\log \pi-\log \sin z \pi\), unde fit per differentiationem

\[
\Psi(-z)-\Psi(z-1)=\pi \operatorname{cotang} z \pi
\]

Et quum e definitione functionis \(\Psi\) generaliter habeatur

\[
\Psi x-\Psi y=-\frac{1}{x+1}+\frac{1}{y+1}-\frac{1}{x+2}+\frac{1}{y+2}-\frac{1}{x+3}+\text { etc. }
\]

oritur series nota

\[
\pi \text { cotang } z \pi=\frac{1}{z}-\frac{1}{1-z}+\frac{1}{1+z}-\frac{1}{2-z}+\frac{1}{2+z}-\frac{1}{3-z}+\text { etc. }
\]

Simili modo e differentiatione formulae 57 prodit

\[
\Psi z+\Psi\left(z-\frac{1}{n}\right)+\Psi\left(z-\frac{2}{n}\right)+\text { etc. }+\Psi\left(z-\frac{n-1}{n}\right)=n \Psi n z-n \log n
\]

adeoque statuendo \(z=0\)

[71] \(\Psi\left(-\frac{1}{n}\right)+\Psi\left(-\frac{2}{n}\right)+\Psi\left(-\frac{3}{n}\right)+\) etc. \(+\Psi\left(-\frac{n-1}{n}\right)=(n-1) \Psi 0-n \log n\) Ita e. g. habetur

\[
\begin{aligned}
& \Psi\left(-\frac{1}{2}\right)=\Psi 0-2 \log 2=-1,9635100260 \quad 2142347944 \quad 099, \text { unde porro } \\
& \Psi_{\frac{1}{2}}=+0,03648997397857652055901 .
\end{aligned}
\]

33.

Sicuti in art. praec. \(\Psi\left(-\frac{1}{2}\right)\) ad \(\Psi 0\) et logarithmum reduximus, ita generaliter \(\Psi\left(-\frac{m}{n}\right)\), designantibus \(m, n\) integros, quorum minor \(m\), ad \(\Psi 0\) et logarithmos reducemus. Statuamus \(\frac{2 \pi}{n}=\omega\), sitque \(\varphi\) alicui angulorum \(\omega, 2 \omega\), \(3 \omega \ldots(n-1) \omega\) aequalis; unde \(1=\cos n \varphi=\cos 2 n \varphi=\cos 3 n \varphi\) etc., \(\cos \varphi=\cos (n+1) \varphi=\cos (n+2) \varphi\) etc., \(\cos 2 \varphi=\cos (n+2) \varphi\) etc., nec non \(\cos \varphi+\cos 2 \varphi+\cos 3 \varphi+\) etc. \(+\cos (n-1) \varphi+1=0\). Habemus itaque \(\cos \varphi \cdot \Psi \frac{1-n}{n}=-n \cos \varphi+\cos \varphi \cdot \log 2-\frac{n}{n+1} \cos (n+1) \varphi+\cos \varphi \cdot \log \frac{3}{2}-\) etc. \(\cos 2 \varphi \cdot \Psi \frac{2-n}{n}=-\frac{n}{2} \cos 2 \varphi+\cos 2 \varphi \cdot \log 2-\frac{n}{n+2} \cos (n+2) \varphi+\cos 2 \varphi \cdot \log \frac{3}{2}-\) etc. \(\cos 3 \varphi \cdot \Psi \frac{3-n}{n}=-\frac{n}{3} \cos 3 \varphi+\cos 3 \varphi \cdot \log 2-\frac{n}{n+3} \cos (n+3) \varphi+\cos 3 \varphi \cdot \log \frac{3}{2}-\) etc. etc. usque ad
\(\cos (n-1) \varphi . \Psi\left(-\frac{1}{n}\right)=-\frac{n}{n-1} \cos (n-1) \varphi+\cos (n-1) \varphi \cdot \log 2-\frac{n}{2 n-1} \cos (2 n-1) \varphi\)
\(+\cos (n-1) \varphi \cdot \log \frac{3}{2}-\) etc.

\(\Psi 0=-\frac{n}{n} \cos n \varphi+\log 2-\frac{n}{2 n} \cos 2 n \varphi+\log \frac{3}{2}-\) etc.

atque per summationem

\(\cos \varphi . \Psi \frac{1-n}{n}+\cos 2 \varphi . \Psi \frac{2-n}{n}+\cos 3 \varphi \cdot \Psi \frac{3-n}{n}+\) etc. \(+\cos (n-1) \varphi . \Psi\left(-\frac{1}{n}\right)+\Psi 0\) \(=-n\left(\cos \varphi+\frac{1}{2} \cos 2 \varphi+\frac{1}{3} \cos 3 \varphi+\frac{1}{4} \cos 4 \varphi+\right.\) etc. in infin. \()\)

Sed habetur generaliter, pro valore ipsius \(x\) unitate non maiori,

\[
\log (1-2 x \cos \varphi+x x)=-2\left(x \cos \varphi+\frac{1}{2} x x \cos 2 \varphi+\frac{1}{3} x^{3} \cos 3 \varphi+\text { etc. }\right)
\]

quae quidem series facile sequitur ex evolutione \(\log (1-r x)+\log \left(1-\frac{x}{r}\right)\), denotante \(r\) quantitatem \(\cos \varphi+\sqrt{ }-1 \cdot \sin \varphi\). Hinc fit aequatio praecedens

[72] \(\cos \varphi . \Psi \frac{1-n}{n}+\cos 2 \varphi \cdot \Psi \frac{2-n}{n}+\cos 3 \varphi \cdot \Psi \frac{3-n}{n}+\) etc. \(+\cos (n-1) \varphi \cdot \Psi\left(-\frac{1}{n}\right)\)

\[
=-\Psi 0+\frac{1}{2} n \log (2-2 \cos \varphi)
\]

Statuatur in hac aequatione deinceps \(\varphi=\omega, \varphi=2 \omega, \varphi=3 \omega\) etc. usque ad \(\vartheta=(n-1) \omega\), multiplicentur singulae hae aequationes ordine suo per \(\cos m \omega\), \(\cos 2 m \omega, \cos 3 m \omega\) etc. usque ad \(\cos (n-1) m \omega\), productorumque aggregato adiiciatur aequatio 71

\[
\Psi \frac{1-n}{n}+\Psi \frac{2-n}{n}+\Psi \frac{3-n}{n}+\text { etc. }+\Psi\left(-\frac{1}{n}\right)=(n-1) \Psi 0-n \log n
\]

Quodsi iam perpenditur, esse

\(1+\cos m \omega \cdot \cos k \omega+\cos 2 m \omega \cdot \cos 2 k \omega+\cos 3 m \omega \cdot \cos 3 k \omega\)

\[
+ \text { etc. }+\cos (n-1) m \omega \cdot \cos (n-1) k \omega=0
\]

denotante \(k\) aliquem numerorum \(1,2,3 \ldots(n-1)\) exceptis his duobus \(m\) atque \(n-m\), pro quibus summa illa fit \(=\frac{1}{2} n\), patebit, ex summatione illarum aequationum prodire, post divisionem per \(\frac{n}{2}\),

\[
\Psi\left(-\frac{m}{n}\right)+\Psi\left(-\frac{n-m}{m}\right)=
\]

\(2 \Psi 0-2 \log n+\cos m \omega \cdot \log (2-2 \cos \omega)+\cos 2 m \omega \cdot \log (2-2 \cos 2 \omega)\)

\(+\cos 3 m \omega \cdot \log (2-2 \cos 3 \omega)+\) etc. \(+\cos (n-1) m \omega \cdot \log (2-2 \cos (n-1) \omega)\)

Manifesto terminus ultimus huius aequationis fit \(=\cos m \omega \cdot \log (2-2 \cos \omega)\), pen-
ultimus \(=\cos 2 m \omega \cdot \log (2-2 \cos 2 \omega)\) etc., ita ut bini termini semper sint aequales, excepto, si \(n\) est par, termino singulari \(\cos \frac{n}{2} \cdot m \omega \log \left(2-2 \cos \frac{n}{2} \omega\right)\), qui fit \(=+2 \log 2\) pro \(m\) pari, vel \(=-2 \log 2\) pro \(m\) impari. Combinando iam cum aequatione 73 hanc

\[
\Psi\left(-\frac{m}{n}\right)-\Psi\left(-\frac{n-m}{n}\right)=\pi \operatorname{cotang} \frac{m}{n} \pi
\]

habemus, pro valore impari ipsius \(n\), siquidem \(m\) est integer positivus minor quam \(n\)

\[
\begin{aligned}
\Psi\left(-\frac{m}{n}\right)= & \Psi 0+\frac{1}{2} \pi \operatorname{cotang} \frac{m \pi}{n}-\log n+\cos \frac{2 m \pi}{n} \cdot \log \left(2-2 \cos \frac{2 \pi}{n}\right) \\
& +\cos \frac{4 m \pi}{n} \cdot \log \left(2-2 \cos \frac{4 \pi}{n}\right)+\cos \frac{6 m \pi}{n} \cdot \log \left(2-2 \cos \frac{6 \pi}{n}\right)+\text { etc. } \\
& +\cos \frac{(n-1) m \pi}{n} \cdot \log \left(2-2 \cos \frac{(n-1) \pi}{n}\right)
\end{aligned}
\]

Pro valore pari ipsius \(n\) autem

[75] \(\Psi\left(-\frac{m}{n}\right)=\Psi 0+\frac{1}{2} \pi \operatorname{cotang} \frac{m \pi}{n}-\log n+\cos \frac{2 m \pi}{n} \log \left(2-2 \cos \frac{2 \pi}{n}\right)\)

\[
\begin{aligned}
& +\cos \frac{4 m \pi}{n} \log \left(2-2 \cos \frac{4 \pi}{n}\right)+\text { etc. }+\cos \frac{(n-2) m \pi}{n} \log \left(2-2 \cos \frac{(n-2) \pi}{n}\right) \\
& \pm \log 2
\end{aligned}
\]

ubi signum superius valet pro \(m\) pari, inferius pro impari. Ita e. g. invenitur

\[
\begin{array}{ll}
\Psi\left(-\frac{1}{4}\right)=\Psi 0+\frac{1}{2} \pi-3 \log 2, & \Psi\left(-\frac{3}{4}\right)=\Psi 0-\frac{1}{2} \pi-3 \log 2 \\
\Psi\left(-\frac{1}{3}\right)=\Psi 0+\frac{1}{2} \pi \sqrt{\frac{1}{3}}-\frac{3}{2} \log 3, & \Psi\left(-\frac{2}{3}\right)=\Psi 0-\frac{1}{2} \pi \sqrt{\frac{1}{3}}-\frac{3}{2} \log 3
\end{array}
\]

Ceterum combinatis his aequationibus cum aequatione 64 sponte patet, \(\Psi z\) generaliter pro quovis valore rationali ipsius \(z\), positivo seu negativo per \(\Psi_{0}\) atque logarithmos determinari posse, quod theorema sane maxime est memorabile.

\[
34 .
\]

Quum, per art. 28, \(\Pi \lambda\) sit valor integralis \(\int y^{\lambda} e^{-y} \mathrm{~d} y\), ab \(y=0\) usque ad \(y=\infty\), siquidem \(\lambda+1\) est quantitas positiva, fit differentiando secundum \(\lambda\)

\[
\frac{\mathrm{d} \Pi \lambda}{\mathrm{d} \lambda}=\frac{\mathrm{d} \int y^{\lambda} e^{-y} \mathrm{~d} y}{\mathrm{~d} \lambda}=\int y^{\lambda} e^{-y} \log y \mathrm{~d} y
\]

sive

\[
\Pi \lambda . \Psi \lambda=\int y^{\lambda} e^{-y} \log y \cdot \mathrm{d} y, \quad \text { ab } y=0 \text { usque ad } y=\infty
\]

Generalius statuendo \(y=z^{\alpha}, a \lambda+a-1=b\), valor integralis \(\int z^{\varepsilon} e^{-z^{a}} \log z \mathrm{~d} z\), a \(z=0\) usque ad \(z=\infty\), fit

\[
=\frac{1}{\alpha a} \Pi\left(\frac{b+1}{\alpha}-1\right) \cdot \Psi\left(\frac{b+1}{\alpha}-1\right)=\frac{1}{\alpha(b+1)} \Pi \frac{b+1}{\alpha} \cdot \Psi \frac{b+1}{a}-\frac{1}{(b+1)^{2}} \Pi \frac{b+1}{a}
\]

siquidem simul \(b+1\) atque \(\alpha\) sunt quantitates positivae, vel aequalis eidem quantitati cum signo opposito, si utraque \(b+1, \alpha\) est negativa.

35.

At non solum productum \(\Pi \lambda . \Psi \lambda\), verum etiam ipsa functio \(\Psi \lambda\) per integrale determinatum exhiberi potest. Designante \(k\) integrum positivum, patet valorem integralis \(\int \frac{x^{2}-x^{2+k}}{1-x} \cdot \mathrm{d} x\), ab \(x=0\) usque ad \(x=1\) esse

\[
=\frac{1}{\lambda+1}+\frac{1}{\lambda+2}+\frac{1}{\lambda+3}+\text { etc. }+\frac{1}{\lambda+k}
\]

Porro quum valor integralis \(\int\left(\frac{1}{1-x}-\frac{k x^{k-1}}{1-x^{k}}\right) \mathrm{d} x\) generaliter sit \(=\) Const. \(+\log \frac{1-x^{k}}{1-x}\), idem inter limites \(x=0\) atque \(x=1\) erit \(=\log k\), unde patet, valorem integralis \(S=\int\left(\frac{1-x^{2}+x^{2+k}}{1-x}-\frac{k x^{k-1}}{1-x^{k}}\right) \mathrm{d} x\) inter eosdem limites esse

\[
=\log k-\frac{1}{\lambda+1}-\frac{1}{\lambda+2}-\frac{1}{\lambda+3}-\text { etc. }-\frac{1}{\lambda+k}
\]

quam expressionem denotabimus per \(Q\). Discerpamus integrale \(S\) in duas partes

\[
\int\left(\frac{1-x^{\lambda}}{1-x}\right) \mathrm{d} x+\int\left(\frac{x^{\lambda+k}}{1-x}-\frac{k x^{k-1}}{1-x^{k}}\right) \mathrm{d} x
\]

Pars prima \(\int \frac{1-x^{\lambda}}{1-x} \cdot \mathrm{d} x\), statuendo \(x=y^{k}\) mutatur in

\[
\int \frac{k y^{k-1}-k y^{2 k+k-1}}{1-y^{k}} \mathrm{~d} y
\]

unde sponte patet, illius valorem ab \(x=0\) usque ad \(x=1\), aequalem esse valori integralis

\[
\int \frac{k x^{k-1}-k x^{\lambda k+k-1}}{1-x^{k}} \mathrm{~d} x
\]

inter eosdem limites, quum manifesto literam \(y\) sub hac restrictione in \(x\) mutare liceat. Hinc fit integrale \(\boldsymbol{S}\), inter eosdem limites

\[
=\int\left(\frac{x^{\lambda+k}}{1-x}-\frac{k x^{\lambda k+k-1}}{1-x^{k}}\right) \mathrm{d} x
\]

Hoc vero integrale, statuendo \(x^{k}=z\), transit in

CIRCA SERIEM INFINTTAM \(1+\frac{2 b}{1 \cdot y} x+\) ETC.

\[
\int\left(\frac{z^{\frac{2+1}{k}}}{k(1-z)^{\frac{1}{k}}}-\frac{z^{2}}{1-z}\right) \mathrm{d} z
\]

quod itaque inter limites \(z=0\) atque \(z=1\) sumtum aequale est ipsi \(Q\). Sed crescente \(k\) in infinitum, limes ipsius \(Q\) est \(\Psi \lambda\), limes ipsius \(\frac{\lambda+1}{k}\) est 0 . limes ipsius \(k(1-z)^{\frac{1}{k}}\) vero est \(\log \frac{1}{z}\) sive \(-\log z\). Quare habemus

\[
\Psi \lambda=\int\left(\frac{1}{\log \frac{1}{z}}-\frac{z^{\lambda}}{1-z}\right) \mathrm{d} z=\int\left(-\frac{1}{\log z}-\frac{z^{\lambda}}{1-z}\right) \mathrm{d} z
\]

a \(z=0\) usque ad \(z==1\).

36.

Integralia determinata, per quae supra expressae sunt functiones \(\Pi \lambda, \Pi \lambda . \Psi \lambda\), restringere oportuit ad valores ipsius \(\lambda\) tales, ut \(\lambda+1\) evadat quantitas positiva: haec restrictio ex ipsa deductione demanavit, reveraque facile perspicitur, pro aliis valoribus ipsius \(\lambda\) illa integralia semper fieri infinita, etiamsi functiones \(\Pi \lambda, \Pi \lambda . \Psi \lambda\) finitae manere possint. Veritati formula \(7 \mathbf{7}\) certo eadem conditio subesse debet, ut \(\lambda+\mathbf{i}\) sit quantitas positiva (alioquin enim integrale certo infinitum evadit, etiamsi functio \(\Psi \lambda\) maneat finita): sed deductio formulae primo aspectu generalis nullique restrictioni obnoxia esse videtur. Sed propius attendenti facile patebit, ipsi analysi, per quam formula eruta est, hanc restrictionem iam inesse. Scilicet tacite supposuimus, integrale \(\int \frac{1-x^{2}}{1-x} \mathrm{~d} x\) cui aequale \(\int \frac{k x^{k-1}-k x^{\lambda k+k-1}}{1-x^{k}} \mathrm{~d} x\) substituimus, habere valorem finitum, quae conditio requirit, ut \(\lambda+1\) sit quantitas positiva. Ex analysi nostra quidem sequitur, haec duo integralia semper esse aequalia, si hoc extendatur ab \(x=0\) usque ad \(x=1-\omega\), illud ab \(x=0\) usque ad \(x=(1-\omega)^{k}\), quantumvis parva sit quantitas \(\omega\), modo non sit \(=0\) : sed hoc non obstante in casu eo. ubi \(\lambda+1\) non est quantitas positiva, duo integralia \(\mathrm{ab} x=0\) usque ad eundem terminum \(x=1-\omega\) extensa neutiquam ad aequalitatem convergunt, sed potius tunc ipsorum differentia, decrescente \(\omega\) in infinitum, in infinitum crescet. Hocce exemplum monstrat, quanta circumspicientia opus sit in tractandis quantitatibus infinitis, quae in ratiociniis analyticis nostro iudicio eatenus tantum sunt admittendae, quatenus ad theoriam limitum reduci possunt.

\section*{37:}
Statuendo in formula 77, \(z=e^{-u}\), patet, illam etiam ita exhiberi posse \(\Psi \lambda=-\int\left(\frac{e^{-u}}{u}-\frac{e^{-u \lambda-u}}{1-e^{-u}}\right) \mathrm{d} u, \quad\) ab \(u=\infty\) usque ad \(u=0\), i. e.

\[
\Psi \lambda=\int\left(\frac{e^{-u}}{u}-\frac{e^{-2 u}}{e^{u}-1}\right) \mathrm{d} u, \quad \text { ab } u=0 \text { usque ad } u=\infty
\]

(Perinde valor ipsius \(\Pi \lambda\) in art. 28 allatus, mutatur statuendo \(e^{-y}=0\), in sequentem

\[
\left.\Pi \lambda=\int\left(\log \frac{1}{v}\right)^{\lambda} d v, \quad \text { a } v=0 \text { usque ad } v=1\right)
\]

Porro patet e formula 77, esse

\[
\Psi \lambda-\Psi \mu=\int \frac{z^{\mu}-z^{\lambda}}{1-z} \mathrm{~d} z, \quad \text { a } z=0 \text { usque ad } z=1
\]

ubi praeter \(\lambda+1\) etiam \(\mu+1\) debet esse quantitas positiva.

Statuendo in eadem formula 77, \(z=u^{\alpha}\), designante \(\alpha\) quantitatem positivam, fit

\[
\Psi \lambda=\int\left(-\frac{u^{\alpha-1}}{\log u}-\frac{a u^{\alpha \lambda+\alpha-1}}{1-u^{\alpha}}\right) \mathrm{d} u, \quad \text { ab } u=0 \text { usque ad } u=1
\]

et quum perinde statui possit, pro valore positivo ipsius \(b\),

patet, fieri

\[
\Psi \lambda=\int\left(-\frac{u^{6-1}}{\log u}-\frac{6 u^{6 \lambda+6-1}}{1-u^{6}}\right) \mathrm{d} u
\]

sive

\[
0=\int\left(\frac{u^{\alpha-1}-u^{6-1}}{\log u}+\frac{\alpha u^{\alpha \lambda+\alpha-1}}{1-u^{\alpha}}-\frac{6 u^{6 \lambda+6-1}}{1-u^{\sigma^{-}}}\right) \mathrm{d} u
\]

\[
\int \frac{u^{\alpha-1}-u^{6-1}}{\log u} \mathrm{~d} u=\int\left(\frac{b^{6 \lambda+6-1}}{1-u^{6}}-\frac{\alpha u^{\alpha \lambda+\alpha-1}}{1-u^{\alpha}}\right) \mathrm{d} u
\]

integralibus semper ab \(u=0\) usque ad \(u=1\) extensis. Sed ponendo \(\lambda=0\), integrale posterius indefinite assignari potest; est scilicet \(=\log \frac{1-u^{\alpha}}{1-u^{6}}\), si evanescere debet pro \(u=0\); quare quum pro \(u=1\). statuere oporteat \(\frac{1-u^{\alpha}}{1-u^{6}}=\frac{\alpha}{\varepsilon}\), erit integrale \(\log \frac{a}{6}=\int \frac{u^{\alpha-1}-u^{6-1}}{\log u} \mathrm{~d} u\), ab \(u=0\) usque ad \(u=1\), quod theorema olim ab ill. EuLer per alias methodos erutum est.

CIRCA SERIEM INFINITAM \(1+\frac{\alpha \ell}{1 \cdot \gamma}+\) ETC.

\begin{center}
%\includegraphics[max width=\textwidth]{2024_01_11_75975a03bcf8b0416cd0g-137}
\end{center}

\begin{center}
\begin{tabular}{|c|c|c|}
\hline
\(z\) & \(\log \Pi z\) & \(\Psi_{z}\) \\
\hline
\begin{tabular}{l}
0.50 \\
0.51 \\
\end{tabular} &% \includegraphics[max width=\textwidth]{2024_01_11_75975a03bcf8b0416cd0g-138}
 & \begin{tabular}{r}
+0.036489973978576520 \\
0.04579678956 \\
\end{tabular} \\
\hline
0.52 & \(9.9479426085749455035 \mathrm{I}\) & 0.055022114579551622 \\
\hline
0.53 & 9.94820145387500065798 & 0.064167307366077154 \\
\hline
0.54 & 9.94849984464251966174 & 0.073233693645365776 \\
\hline
0.55 & 9.94883744144659973817 & 0.082222567539644344 \\
\hline
0.56 & 9.94921391040978143536 & \(0.091135192540635 \times 89\) \\
\hline
0.57 & 9.94962892307706494873 & 0.099972802444444623 , \\
\hline
0.58 & \(9.9500821562 \quad 8891076887\) & 0.108736602251781439 \\
\hline
0.59 & \(9.9505732920580773^{8191}\) & 0.117427769035011042 , \\
\hline
0.60 & 9.95110201745015512544 & 0.126047452773476253 \\
\hline
\(0.6 \mathbf{I}\) & \(9.95166802446766 \mathrm{r} 36244\) & 0.134596777158445210 \\
\hline
0.62 & 9.95227100993756789859 & c. \(1430768403 \quad 6898 \mathrm{c}\) \\
\hline
0.63 & 9.95291067540213704917 & \(0.151_{4887 \times 58} \times 9958383\) \\
\hline
0.64 & 9.9535867270 I 294797674 & 0.159833452883415463 \\
\hline
0.65 & 9.95429887542799988466 & \(0.168 \times 12077584327804\) \\
\hline
0.66 & \(9.9550468357 \mathrm{Ir} 78337730\) & 0.176325593271894293 \\
\hline
0.67 & \(9.955^{8} 3032723821579829\) & 0.184474981267329607 \\
\hline
0.68 & 9.95664907359634064632 & 0.192561201489132418 \\
\hline
0.69 & \(9.9575028024986952535 \mathrm{I}\) & 0.200585193056747012 \\
\hline
0.70 & 9.95839124569225480685 & 0.208547874873493948 \\
\hline
\(0.7 \mathrm{r}\) & 9.95931413887186450668 & \(0.216450146 \mathrm{r} 89604789\) \\
\hline
0.72 & \(9.96027 \times 22 \times 596075 \times 9880\) & \(0.224292887146 \mathrm{r}_{57521}\) \\
\hline
0.73 & \(9.96 \times 26223720530119641\) & 0.232076959300672792 \\
\hline
0.74 & 9.96228693274222223320 & \(0.239803206 x 35096466\) \\
\hline
0.75 & 9.96334505887435456829 & \(0.24747245354686 \times 164\) \\
\hline
0.76 & \(9.9644363698 \quad 1871920339\) & 0.255085510323688336 \\
\hline
0.77 & \(9.9655606232 \quad 6853798084\) & 0.262643168602762795 \\
\hline
0.78 & \(9.9667 \times 758032189 r_{4} r_{4} 7\) & 0.270146204314883540 \\
\hline
0.79 & 9.96790700541227146665 & 0.2775953776 14168016 \\
\hline
0.80 & \(9.969128666240976 \times 4416\) & \(0.28499 \times 43329386 r_{542}\) \\
\hline
\(0.8 \mathbf{I}\) & 9.97038233371127271250 & 0.292335 rorx 88779580 \\
\hline
0.82 & \(9.9716677818 \quad 6428658993\) & 0.299627096564887544 \\
\hline
0.83 & 9.9729847878 1655271065 & 0.306868120496501033 \\
\hline
0.84 & \(9.97433313 \times 9917940601\) & 0.314058860231568639 \\
\hline
0.85 & 9.97571259659857361442 & 0.321199989545479708 \\
\hline
0.86 & 9.97712296849867851092 & 0.328292169083820641 \\
\hline
0.87 & \(9.9785640362 \quad 246764477 x\) & 0.33533604669448 \\
\hline
0.88 & 9.980035591388 III82I62 & 0.342332257749528903 , \\
\hline
0.89 & \(9.9815374283 \quad 3339013630\) & 0.349281425457135499 \\
\hline
0.90 & 9.9830693440856 IIIIO78 & \(0.356 \times 8416 \times 1 \times 64059720\) \\
\hline
0.91 & \(9.984631 \times 382996952032 x\) & 0.363041064648881123 \\
\hline
0.92 & \(9.9862226 x_{32} 10764373^{81}\) & \(0.36985272440640 r_{4} 69\) \\
\hline
0.93 & 9.98784357358573930651 & 0.376619717923498793 \\
\hline
0.94 & \(9.989493826676 \mathrm{rr} 664682\) & 0.3833426 II9 46740214 \\
\hline
0.95 & \(9.9911731821 \quad 7189109803\) & 0.390021962742043086 \\
\hline
0.96 & 9.99288 r452I \(_{5658844947}\) & 0.396658316346662402 \\
\hline
0.97 & \(9.9946 \times 84510 \quad 6337679375\) & 0.403252208813771306 \\
\hline
0.98 & 9.99638399563222432515 & 0.409804166449890838 \\
\hline
0.99 & \(9.9981779048680732016 \mathrm{r}\) & \(0.4163 \times 4706045414956\) \\
\hline
\(1 \infty\) & 0.00000000000000000000 & 0.422784335098467139 \\
\hline
\end{tabular}
\end{center}

\section*{METHODUS NOVA}
\section*{INTEGRALIUM VALORES. 
 PER APPROXIMATIONEM INVENIENDI}
A \(\quad\) U C T O O R E

CAROLO FRIDERICO GAUSS

SOCIETATI REGLAE SCIENTIARUM EXHIBITA 1814. SEPT. 16.

Commentationes societatis regiae seientiarum Gottingensis recentiores. Vol. III. Gottingae mocccxvi.

\section*{165}
METHODUS NOVA

\section*{INTEGRALIUM VALORES 
 PER APPROXIMATIONEM INVENIENDI.}
1.

Inter methodos ad determinationem numericam approximatam integralium propositas insignem tenent locum regulae, quas praeeunte summo NewTon evolutas dedit Cotes. Scilicet si requiritur valor integralis \(\int y \mathrm{~d} x\) ab \(x=g\) usque ad \(x=h\) sumendus, valores ipsius \(y\) pro his valoribus extremis ipsius \(x\) et pro quotcunque aliis intermediis a primo ad ultimum incrementis aequalibus progredientibus, multiplicandi sunt per certos coëfficientes numericos, quo facto productorum aggregatum in \(h-g\) ductum integrale quaesitum suppeditabit, eo maiore praecisione, quo plures termini in hac operatione adhibentur. Quum principia huius methodi, quae a geometris rarius quam par est in usum vocari videtur, nusquam quod sciam plenius explicata sint, pauca de his praemittere ab instituto nostro haud alienum erit.

2.

Sit \(n+1\) multitudo terminorum, quos in usum vocare placuit, statuamusque \(h-g=\Delta\), ita ut valores ipsius \(x\) sint \(g, g+\frac{\Delta}{n}, g+\frac{2 \Delta}{n}, g+\frac{3 \Delta}{n}\) etc. usque ad \(g+\Delta\), respondeantque iisdem resp. valores ipsius \(y\) hi \(A, A^{\prime}, A^{\prime \prime}, A^{\prime \prime \prime}\) etc. usque ad \(A^{(n)}\) : denique ponatur indefinite \(x=g+\Delta t\), ita ut \(y\) etiam spectari possit tamquam functio ipsius \(t\). Designemus per \(\boldsymbol{Y}\) functionem sequentem

\[
\begin{aligned}
& \text { A. } \frac{(n t-1)(n t-2)(n t-3) \ldots(n t-n)}{(-1)(-2) \cdot(-3) \ldots \cdot(-n)} \\
& +A^{\prime} \cdot \frac{n t \cdot(n t-2) \cdot(n t-3) \cdots \cdots \cdot(n t-n)}{1 \cdot(-1)(-2) \cdots \cdots(1-n)} \\
& +A^{\prime \prime} \cdot \frac{n t \cdot(n t-1) \cdot(n t-3) \ldots \ldots(n t-n)}{2 \cdot 1 . \quad(-1) \ldots \ldots(2-n)}
\end{aligned}
\]

\begin{center}
%\includegraphics[max width=\textwidth]{2024_01_11_75975a03bcf8b0416cd0g-141}
\end{center}

\[
\begin{aligned}
& \text { + etc. } \\
& +A^{(n)} \cdot \frac{n t(n t-1) \cdot(n t-2) \ldots \ldots(n t-n+1)}{n \cdot(n-1)(n-2) \ldots \ldots \cdot 1}
\end{aligned}
\]

sive \(\Sigma \frac{A^{(\mu)} T^{(\mu)}}{M^{(\mu)}}\), ubi repraesentante \(\mu\) singulos integros \(0,1,2,3 \ldots n\),

\[
\begin{aligned}
& T^{(\mu)}=\frac{n t(n t-1)(n t-2)(n t-3) \ldots(n t-n)}{n t-\mu} \\
& M^{(\mu)} \text { valor ipsius } T \text { pro } n t=\mu .
\end{aligned}
\]

Manifestum erit, \(\boldsymbol{Y}\) exhibere functionem algebraicam integram ipsius \(t\) ordinis \(n\), atque eius valores pro singulis \(n+1\) valoribus ipsius \(t\), puta \(0, \frac{1}{n}, \frac{2}{n}, \frac{3}{n} \ldots 1\) aequales esse valoribus ipsius \(y\). Porro patet, si \(Y^{\prime}\) sit functio alia integra pro iisdem valoribus cum \(y\) conspirans, \(\boldsymbol{Y}^{\prime}-\boldsymbol{Y}\) pro iisdem evanescere, adeoque per factores \(t, t-\frac{1}{n}, t-\frac{2}{n}, t-\frac{3}{n} \ldots . t-1\) et proin etiam per eorum productum (quod est ordinis: \(n+1\) ) divisibilem esse, unde patet, \(Y^{\prime}\), nisi prorsus identica sit cum \(\boldsymbol{Y}\), certo ad altiorem ordinem ascendere debere, sive \(\boldsymbol{Y}\) ex omnibus functionibus integris ordinem \(n\) haud egredientibus unicam esse, quae pro illis \(n+1\) valoribus cum \(y\) conspiret. Quodsi itaque \(y\), in seriem sécundum potestates ipsius \(t\) progredientem evoluta, ante terminum qui implicat \(t^{n+1}\) omnino abrumpitur, cum \(\boldsymbol{Y}\) identica erit: si vero saltem tam cito convergit, ut terminos sequentes spernere liceat, functio \(Y\) inter limites \(t=0, t=1\) sive \(x=g, x=h\) ipsius \(y\) vice fungi poterit.

\section*{3.}
Iam integrale nostrum \(\int y \mathrm{~d} x\) transit in \(\Delta \int y \mathrm{~d} t\) a \(t=0\) psque ad \(t=1\) sumendum, cuius loco per ea, quae modo monuimus, adoptabimus \(\Delta \int Y \mathrm{~d} t\). Evolvendo itaque \(T^{(\mu)}\) in

\[
\alpha t^{n}+b t^{n-1}+\gamma t^{n-2}+\delta t^{n-3}+\text { etc. }
\]

erit \(\int T^{(\underline{\mu})} \mathbf{d} t, \quad a: t=0\) usque ad \(t=1\),

\[
=\frac{a}{n+1}+\frac{b}{n}+\frac{\gamma}{n-1}+\frac{\delta}{n-2}+\text { etc. }
\]

qua quantitate posita \(=M^{(\mu)} R^{(\mu)}\), erit integrale quaesitum

\[
=\Delta\left(A R+A^{\prime} R^{\prime}+A^{\prime \prime} R^{\prime \prime}+A^{\prime \prime \prime} R^{\prime \prime \prime}+\text { etc. }+A^{(n)} R^{(n)}\right)
\]

Exempli caussa apponemus computum coëfficientis \(R^{\prime \prime}\) pro \(n=5\). Fit hic

\[
\begin{aligned}
& T^{\prime \prime}=5^{5} t^{5}-13.5^{4} t^{4}+59.5^{3} t^{3}-107.5^{2} t t+60.5 . t \\
& M^{\prime \prime}=2 \times 1 \times(-1) \times(-2) \times(-3)=-12
\end{aligned}
\]

Hinc \(-12 R^{\prime \prime}=\frac{3125}{6}-1625+\frac{7375}{4}-\frac{2675}{3}+150=-\frac{25}{12}\), adeoque \(R^{\prime \prime}=\frac{25}{144}\).

Computus aliquanto brevior evadit, statuendo \(2 t-1=u\). Tunc fit

\[
T^{(\mu)}=\frac{(n u+n)(n u+n-2)(n u+n-4) \ldots(n u-n+4)(n u-n+2)(n u-n)}{2^{n}(n u+n-2 \mu)}
\]

\section*{Ponamus}
\[
\frac{(n n u u-n n) \cdot\left(n n u u-(n-2)^{2}\right) \cdot\left(n n u u-(n-1)^{2}\right) \cdot\left(n n u u-(n-6)^{2}\right) \ldots}{n n u u-(n-2 \mu)^{2}}=U^{(\mu)}
\]

ubi numerator desinere debet in ... (nnuu-9)(nnuu-1), si \(n\) est impar, vel in ... \((n n u u-4) n u\), si \(n\) est par, eritque

\[
T^{(\mu)}=\frac{(n u-n+2 \mu) U(\mu)}{2^{n}}
\]

Iam integrale \(\int T^{(\mu)} \mathrm{d} t\) a \(t=0\) usque ad \(t=1\) acceptum aequale est integrali

\[
\int \frac{1}{2} T^{(\mu)} \mathrm{d} u=\int \frac{n u U^{(\mu)} \mathrm{d} u}{2^{n+1}}+\int \frac{(2 \mu-n) U^{(\mu)} \mathrm{d} u}{2^{n+1}}
\]

ab \(u=-1\) usque ad \(u=+1\).

Statuendo itaque

\[
U^{(u)}=\alpha u^{n-1}+b u^{n-3}+\gamma u^{n-5}+\delta u^{n-7}+\text { etc. }
\]

(sponte enim patet, potestates \(u^{n-2}, u^{n-4}, u^{n-6}\) etc. abesse), integralis pars \(\int \frac{n u U\left({ }^{n}\right) \mathrm{d} u}{2^{n+1}}\) evanescet pro valore impari ipsius \(n\), pars altera \(\int \frac{(2 \mu-n) U\left({ }^{\mu}\right) \mathrm{d} u}{2^{n+1}}\) vero pro valore pari, unde integrale \(\int T^{(\mu)} \mathrm{d} t\) fiet pro \(n\) pari

\[
\because \because " 5-1=\frac{n}{2^{n}}\left(\frac{\alpha}{n+1}+\frac{6}{n-1}+\frac{\gamma}{n-3}+\frac{\delta}{n-5}+\text { etc. }\right)
\]

pro \(n\) impari autem

\[
=\frac{2 \mu-n}{2^{n}}\left(\frac{\alpha}{n}+\frac{6}{n-2}+\frac{\gamma}{n-4}+\frac{\delta}{n-6}+\text { etc. }\right)
\]

In exemplo nostro habetur

\[
\begin{aligned}
& U^{\prime \prime}=(25 u u-25)(25 u u-9)=625 u^{4}-850 u u+225, \text { adeoque } \\
& -12 R^{\prime \prime}=-\frac{1}{32}\left(125-\frac{850}{3}+225\right)=-\frac{25}{12} \text { ut supra. }
\end{aligned}
\]

Observare convenit, fieri \(U^{(n-\mu)}=U^{(\mu)}\), adeoque \(\int T^{(n-\mu)} \mathrm{d} t= \pm \int T^{(\mu)} \mathrm{d} t\). signo superiore valente pro \(n\) pari, inferiore pro impari. Quare quum facile perspiciatur, perinde haberi \(M^{(n-\mu)}= \pm M^{(\mu)}\), semper erit \(R^{(n-\mu)}=R^{(\mu)}\), sive \(\mathrm{e}\) coëfficientibus \(R, R^{\prime}, R^{\prime \prime} \ldots R^{(n)}\) ultimus primo aequalis, penultimus secundo et sic porro.

4.

Valores numericos horum coëfficientium a Cotesıo usque ad \(n=10\) computatos ex Harmonia Mensurarum huc adscribimus.

Pro \(n=1\) sive terminis duobus.

\(R=R^{\prime}=\frac{1}{2}\)

Pro \(n=2\) sive terminis tribus.

\(R=R^{\prime \prime}=\frac{1}{6}, R^{\prime}=\frac{2}{3}\)

Pro \(n=3\) sive terminis quatuor.

\(R=R^{\prime \prime \prime}=\frac{1}{8}, R^{\prime}=R^{\prime \prime}=\frac{3}{8}\)

Pro \(n=4\) sive terminis quinque.

\(\boldsymbol{R}=\boldsymbol{R}^{\prime \prime \prime}=\frac{7}{90}, R^{\prime}=R^{\prime \prime \prime}=\frac{16}{45}, R^{\prime \prime}=\frac{2}{15}\)

Pro \(n=5\) sive terminis sex.

\(\boldsymbol{R}=\boldsymbol{R}^{\mathbf{v}}=\frac{19}{288}, \boldsymbol{R}^{\prime}=\boldsymbol{R}^{\prime \prime \prime}=\frac{25}{9}, \boldsymbol{R}^{\prime \prime}=\boldsymbol{R}^{\prime \prime \prime}=\frac{25}{1 \frac{5}{44}}\)

Pro \(n=6\) sive terminis septem.

\(R=R^{\mathrm{v}}=\frac{41}{846}, R^{\prime}=R^{\mathrm{v}}=\frac{9}{35}, R^{\prime \prime}=R^{\mathrm{N}}=\frac{9}{280}, R^{\prime \prime \prime}=\frac{34}{105}\)

Pro \(n=7\) sive terminis octo.

\begin{center}
%\includegraphics[max width=\textwidth]{2024_01_11_75975a03bcf8b0416cd0g-143}
\end{center}

Pro \(n=8\) sive terminis novem.

\(R=R^{\mathrm{VIII}}=\frac{999}{2 \frac{89}{350}}, R^{\prime}=R^{\mathrm{VII}}=\frac{2944}{14175}, R^{\prime \prime}=R^{\mathrm{VI}}=-\frac{464}{14175}, R^{\prime \prime \prime}=R^{\gamma}=\frac{5248}{1417_{5}}\), \(R^{\mathrm{V}}=-\frac{454}{2835}\)

Pro \(n=9\) sive terminis decem.

\begin{center}
%\includegraphics[max width=\textwidth]{2024_01_11_75975a03bcf8b0416cd0g-144}
\end{center}

\[
\begin{aligned}
& R^{\mathrm{IV}}=R^{V}=\frac{2889}{44800}
\end{aligned}
\]

Pro \(n=10\) sive terminis undecim.

\[
R=R^{\mathrm{X}}=\frac{16067}{5987529}, R^{\prime}=R^{\mathrm{IX}}=\frac{265575}{149688} . R^{\prime \prime}=R^{\prime I I I}=-\frac{16175}{199584},
\]

\[
R^{\prime \prime \prime}=R^{V I I}=\frac{5675}{12474}, R^{I V}=R^{V I}=-\frac{4825}{110^{8} 8}, R^{V}=\frac{17807}{2 \frac{78}{4948}} .
\]

5.

Quum formula \(\Delta\left(A R+A^{\prime} R^{\prime}+A^{\prime} R^{\prime}+A^{\prime \prime} R^{\prime \prime}+\right.\) etc. \(\left.+A^{(n)} R^{(n)}\right)\) integrale \(\int y \mathrm{~d} x\) ab \(x=g\) usque ad \(x=g+\Delta\), sive integrale \(\Delta \int y \mathrm{~d} t\) a \(t=0\) usque ad \(t=1\) exacte quidem exhibeat, quoties \(y\) in seriem evoluta potestatem \(t^{n}\) non transscendit, sed approximate tantum, quoties \(y\) ultra progreditur, superest, ut errorem, quem inducunt termini proxime sequentes, assignare doceamus. Designemus generaliter per \(k^{(m)}\) differentiam inter valorem verum integralis \(\int t^{m} \mathrm{dt}\) a \(t=0\) usque ad \(t=1\), atque valorem ex formula prodeuntem, ita ut sit

\[
\begin{aligned}
& k=1-R-R^{\prime}-R^{\prime \prime}-R^{\prime \prime \prime}-\text { etc. }-R^{(n)} \\
& k^{\prime}=\frac{1}{2}-\frac{1}{n}\left(R^{\prime}+2 R^{\prime \prime}+3 R^{\prime \prime \prime}+\text { etc. }+n R^{(n)}\right) \\
& k^{\prime \prime}=\frac{1}{3}-\frac{1}{n n}\left(R^{\prime}+4 R^{\prime \prime}+9 R^{\prime \prime \prime}+\text { etc. }+n n R^{(n)}\right) \\
& k^{\prime \prime \prime}=\frac{1}{4}-\frac{1}{n^{2}}\left(R^{\prime}+8 R^{\prime \prime}+27 R^{\prime \prime \prime}+\text { etc. } n^{3} R^{(n)}\right)
\end{aligned}
\]

etc. Patet igitur, si \(y\) evolvatur in seriem

\[
K+K^{\prime} t+K^{\prime \prime} t t+K^{\prime \prime \prime} t^{3}+\text { etc. }
\]

differentiam inter valorem verum integralis \(\int y \mathrm{~d} t\) atque valorem approximatum formulae exprimi per

\[
K k+K^{\prime} k^{\prime}+K^{\prime \prime} k^{\prime \prime}+K^{\prime \prime \prime} k^{\prime \prime \prime}+\text { etc. }
\]

Sed manifesto \(k, k^{\prime}, k^{\prime \prime}\) etc. usque ad \(k^{(n)}\) sponte fiunt \(=0\) : correctio itaque formulae approximatae èrit

\[
\boldsymbol{K}^{(n+1)} \boldsymbol{k}^{(n+1)}+\boldsymbol{K}^{(n+2)} \boldsymbol{k}^{(n+2)}+\boldsymbol{K}^{(n+3)} \boldsymbol{k}^{(n+3)}+\text { etc. }
\]

Indolem quantitatum \(k^{(n+1}, k^{(n+2)}\) etc. infra accuratius perscrutabimur; hic sufficiat, valores numericos primae aut secundaé, pro singulis valoribus ipsius \(n\), apposuisse, ut gradus praecisionis, quam formula approximata affert, inde aestimari possit.

Pro \(n=1\) habemus \(k^{\prime \prime}=-\frac{1}{6}, k^{\prime \prime \prime}=-\frac{1}{4}, k^{\prime \prime \prime \prime}=-\frac{3}{10}\)

Pro \(n=2\) invenimus \(k^{\prime \prime \prime}=0, k^{\prime \prime \prime \prime}=-\frac{1}{120}, k^{\mathrm{v}}=-\frac{1}{48}\)

Pro \(n=3\) fit \(k^{\prime \prime \prime \prime}=-\frac{1}{270}, \quad k^{\mathrm{v}}=-\frac{1}{108}\)

Pro \(n=4 \ldots k^{\mathrm{v}}=0, k^{\mathrm{VI}}=-\frac{1}{2688}, k^{\mathrm{VII}}=-\frac{1}{768}\)

15 Pro \(n=5 \ldots k^{\mathrm{VI}}=-\frac{1}{52} \overline{1}_{000}^{00}, k^{\mathrm{vII}}=-\frac{11}{15000}\)

Pro \(n=6 \ldots k^{\mathrm{nII}}=0, k^{\mathrm{vHI}}=-\frac{1}{38880}, k^{1 \mathrm{I}}=-\frac{1}{8640}\)

Pro \(n=7 \ldots k^{\mathrm{VIII}}=-\frac{167}{105^{16} 8410}, k^{\mathrm{IT}}=-\frac{16}{2352980}\)

10 Pro \(n=8 \ldots k^{\mathrm{IX}}=0, k^{\mathrm{X}}=-\frac{37}{173 \sqrt{1804}}, k^{\mathrm{XI}}=-\frac{3 \gamma}{314588}\)

Pro \(n=9 \ldots k^{\mathrm{x}}=-\frac{865}{631351908}, k^{\mathrm{xI}}=-\frac{865}{114791256}\)

Pro \(n=10 \ldots k^{\mathrm{XI}}=0, k^{\mathrm{XII}}=-\frac{26927}{13650000000}, k^{\mathrm{xII}}=-\frac{269927}{2100000000}\)

Pro valore pari ipsius \(n\) ubique hic fieri animadvertimus \(k^{(n+1)}=0\), ac praeterea \(k^{(n+3)}=\frac{n+3}{2} k^{(n+2)}\); pro valore impari ipsius \(n\) autem ubique prodit \(k^{(n+2)}=\frac{n+2}{2} k^{(n+1)}\). Ratio horum eventuum facile e considerationibus sequentibus depromitur.

Designemus generaliter per \(l^{(m)}\) differentiam inter valorem verum huius integralis \(\int\left(t-\frac{1}{2}\right)^{m} \mathrm{~d} t\) a \(t=0\) usque ad \(t=1\), atque valorem eum, quem formula approximata profert, ita ut habeatur

\(l^{(m)}=\int\left(t-\frac{1}{2}\right)^{m} \mathrm{~d} t-\left[\left(-\frac{1}{2}\right)^{m} R+\left(\frac{1}{n}-\frac{1}{2}\right)^{m} R^{\prime}+\left(\frac{2}{n}-\frac{1}{2}\right)^{m} R^{\prime \prime}+\left(\frac{3}{n}-\frac{1}{2}\right)^{m} R^{\prime \prime \prime}+\right.\) etc. \(\left.+\left(\frac{1}{2}-\frac{1}{n}\right)^{m} R^{(n-1)}+\left(\frac{1}{2}\right)^{m} R^{(n)}\right]\)

integrali a \(t=0\) usque ad \(t=1\) accepto. Manifesto pro valore impari ipsius \(m\) evanescet tum valor verus integralis tum valor approximatus: erit itaque \(l^{\prime}=0\), \(l^{\prime \prime \prime}=0, l^{\mathrm{v}}=0, l^{\mathrm{VII}}=0\) etc. sive generaliter \(l^{(m)}=0\) pro valore impari ipsius \(m\). Pro valore pari autem ipsius \(m\), formulae tribuimus formam hancce

\[
\begin{gathered}
l^{(m)}=\frac{1}{2^{m}(m+1)}-\frac{2}{n^{m}}\left(\left(\frac{1}{2} n\right)^{m} R+\left(\frac{1}{2} n-1\right)^{m} R^{\prime}+\left(\frac{1}{2} n-2\right)^{m} R^{\prime \prime}+\right.\text { etc. } \\
\left.+2^{m} R^{\left(\frac{1}{2} n-2\right)}+R^{\left(\frac{1}{2} n-1\right)}\right)
\end{gathered}
\]

si simul fuerit \(n\) par; vel hanc

\[
\begin{aligned}
& l^{(m)}=\frac{1}{2^{n}}\left(\frac{1}{m+1}-\frac{2}{n^{n}}\left(n^{m} R+(n-2)^{m} R^{\prime}+(n-4)^{m} R^{\prime \prime}+\right.\text { etc. }\right. \\
& \left.\left.+3^{m} R^{\left(\frac{1}{2} n-\frac{3}{2}\right)}+R^{\left(\frac{1}{2} n-\frac{1}{2}\right)}\right)\right)
\end{aligned}
\]

si simul fuerit \(n\) impar.

Si igitur per evolutionem ipsius \(y\) in seriem secundum potestates ipsius \(t-\frac{1}{2}\) progredientem prodit

\[
y=L+L^{\prime}\left(t-\frac{1}{2}\right)+L^{\prime \prime}\left(t-\frac{1}{2}\right)^{2}+L^{\prime \prime \prime}\left(t-\frac{1}{2}\right)^{3}+\text { etc. }
\]

correctio valori approximato integralis \(\int y \mathrm{~d} t\) a \(t=0\) usque ad \(t=1\) applicanda erit

\[
L l+L^{\prime \prime} l^{\prime \prime}+L^{\prime \prime \prime} l^{\prime \prime \prime}+L^{v 1} l^{v 1}+\text { etc. }
\]

aut potius, quum \(l^{(m)}\) necessario evanescat pro valore quovis integro ipsius \(m\) haud maiori quam \(n\), correctio erit

\[
L^{(n+2)} l^{(n+2)}+L^{(n+4)} l^{(n+4)}+L^{(n+6)} l^{(n+6)}+\text { etc. }
\]

pro \(n\) pari, vel

\[
L^{(n+1)} l^{(n+1)}+L^{(n+3)} l^{(n+3)}+L^{(n+5)} l^{(n+5)}+\text { etc. }
\]

pro \(n\) impari.

Facillime iam correctiones \(l^{(m)}\) ad \(k^{(m)}\) reducuntur et vice versa. Quum enim habeatur

erit

\[
\left(t-\frac{1}{2}\right)^{m}=t^{m}-\frac{1}{2} m \cdot t^{m-1}+\frac{1}{4} \cdot \frac{m(m-1)}{1.2} t^{m-2}+\text { etc. }
\]

Et perinde fit

\[
l^{(m)}=k^{(m)}-\frac{1}{2} m k^{(m-1)}+\frac{1}{4} \cdot \frac{m(m-1)}{1.2} k^{(m-2)}+\text { etc. }
\]

\[
k^{(m)}=l^{(m)}+\frac{1}{2} m l^{(m-1)}+\frac{1}{4} \cdot \frac{m(m-1)^{\prime}}{1 \cdot 2} l^{(m-1)}+\text { etc. }
\]

Ex posteriori formula eiicientur termini, ubi \(l\) afficitur indice impari: utraque autem continuanda est tantummodo usque ad indicem \(n+1\) (inclus.). Manifesto itaque habebimus

pro \(n\) pari

\[
\begin{aligned}
& k^{(n+1)}=0 \\
& k^{(n+2)}=l^{(n+2)} \\
& k^{(n+3)}=\frac{n+3}{2} \cdot l^{(n+2)}
\end{aligned}
\]

pro \(n\) impari

\[
\begin{aligned}
& k^{(n+1)}=l^{(n+1)} \\
& k^{(n+2)}=\frac{n+2}{2} \cdot l^{(n+1)}
\end{aligned}
\]

unde demanant observationes supra indicatae

6.

Generaliter itaque loquendo praestabit, in applicanda methodo Cotesiana ipsi \(n\) tribuere valorem parem, seu terminorum multitudinem imparem in usum vocare. Perparum scilicet praecisio augebitur, si loco valoris paris ipsius \(n\) ad imparem proxime maiorem ascendamus, quum error maneat eiusdem ordinis, licet coëfficiente aliquantulum minori affectus. Contra ascendendo a valore impari ipsius \(n\) ad parem proxime sequentem, error duobus ordinibus promovebitur, insuperque coëfficiens notabilius imminutus praecisionem augebit. Ita si quinque termini adhibentur, sive pro \(n=4\), error proxime exprimitur per \(-\frac{1}{2688} K^{6}\) vel per \(-\frac{1}{2688} L^{6}\); si statuitur \(n=5\), error erit proxime \(-\frac{11}{3 \frac{1}{500}} K^{6}\) vel \(-\frac{11}{52500} L^{6}\), adeoque ne ad semissem quidem prioris depressus: contra faciendo \(n=6\), error fit proxime \(=-\frac{1}{38880} K^{8}\) vel \(=-\frac{1}{3-8880} L^{8}\), praecisioque tanto magis aucta, quo citius series, in quam functio evolvitur, iam per se convergit.

7.

Postquam haecce circa methodum Cotesii praemisimus, ad disquisitionem generalem progredimur, abiiciendo conditionem, ut valores ipsius \(x\) progressione arithmetica procedant. Problema itaque aggredimur, determinare valorem integralis \(\int y \mathrm{~d} x\) inter limites datos ex aliquot valoribus datis ípsius \(y\), vel exacte vel quam proxime. Supponamus, integrale sumendum esse ab \(x=g\) usque ad \(x=g+\Delta\), introducamusque loco ipsius \(x\) aliam variabilem \(t=\frac{x-g}{\Delta}\), ita ut integrale \(\Delta \int y \mathrm{~d} t\) a \(t=0\) usque ad \(t=1\) investigare oporteat. Respondeant \(n+1\) valores dati ipsius \(y\) hi \(A, A^{\prime}, A^{\prime \prime}, A^{\prime \prime \prime} \ldots A^{(n)}\) valoribus ipsius \(t\) inaequalibus his \(a, a^{\prime}, a^{\prime \prime}, a^{\prime \prime \prime} \ldots a^{(n)}\), designemusque per \(\boldsymbol{Y}\) functionem algebraicam integram ordinis \(n\) hances:

\[
\begin{aligned}
& A \frac{\left(t-a^{\prime}\right)\left(t-a^{\prime \prime}\right)\left(t-a^{\prime \prime \prime}\right) \ldots\left(t-a^{(n)}\right)}{\left(a-a^{\prime}\right)\left(a-a^{\prime \prime}\right)\left(a-a^{\prime \prime \prime}\right) \ldots\left(a-a^{(n)}\right)} \\
&+A^{\prime} \frac{(t-a)\left(t-a^{\prime \prime}\right)\left(t-a^{\prime \prime \prime}\right) \ldots\left(t-a^{(n)}\right)}{\left(a^{\prime}-a\right)\left(a^{\prime}-a^{\prime \prime}\right)\left(a^{\prime}-a^{\prime \prime \prime}\right) \ldots\left(a^{\prime}-a^{(n)}\right)} \\
&+A^{\prime \prime} \frac{(t-a)\left(t-a^{\prime}\right)\left(t-a^{\prime \prime \prime}\right) \ldots\left(t-a^{(n)}\right)}{\left(a^{\prime \prime}-a\right)\left(a^{\prime \prime}-a^{\prime}\right)\left(a^{\prime \prime}-a^{\prime \prime}\right) \ldots\left(a^{\prime \prime}-a^{(n)}\right)} \\
&+ \text { etc. } \\
&+A^{(n)} \frac{(t-a)\left(t-a^{\prime}\right)}{\left(a^{(n)}-a\right)\left(a^{(n)}-a^{\prime}\right)\left(a^{(n)}-a^{\prime \prime}\right) \ldots\left(a^{(n)}-a^{\left(n^{\prime \prime-1}\right)}\right)}
\end{aligned}
\]

Manifesto valores huius functionis, si \(t\) alicui quantitatum \(a, a^{\prime}, a^{\prime \prime}, a^{\prime \prime \prime} \ldots a^{(m)}\) aequalis ponitur, coincidunt cum valoribus respondentibus functionis \(y\), unde per-
inde at in art. 2. concludimus, \(Y\) cum \(y\) identicam esse, quoties \(y\) quoque sit functio algebraica integra ordinem \(n\) non transscendens, aut saltem ipsius \(y\) vice fungi posse, si \(y\) in seriem secundum potestates ipsius \(t\) progredientem conversa tantam convergentiam exhibeat, ut terminos altiorum ordinum negligere liceat.

\section*{8.}
Iam ad eruendum integrale \(\int \boldsymbol{Y} \mathrm{d} t\) singulas partes ipsius \(\boldsymbol{Y}\) consideremus. Designemus productum,

\[
(t-a)\left(t-a^{\prime}\right)\left(t-a^{\prime \prime}\right)\left(t-a^{\prime \prime \prime}\right) \ldots\left(t-a^{(n)}\right)
\]

per \(T\), fiatque per evolutionem huius producti

\[
\text { - } T=t^{n+1}+\alpha t^{n}+\alpha^{\prime} t^{n-1}+\alpha^{n} t^{n-2}+\text { etc. }+\alpha^{(n)}
\]

Numerator fractionīs, per quam, in parte prima ipsius \(\boldsymbol{Y}\), multiplicata est \(A\), fit \(=\frac{T^{\prime}}{t-a} ;\) numeratores in partibus sequentibus perinde sunt \(\frac{T}{t-a^{\prime}}, \frac{T}{t-a^{\prime \prime}}, \frac{T}{t-a^{\prime \prime \prime}}\) etc. Denominatores vero nihil aliud sunt, nisi valores determinati horum numeratorum, si resp. statuitur \(t=a, t=a^{\prime}, t=a^{\prime \prime}, t=a^{\prime \prime \prime}\) etc.: denotemus hos denominatores resp. per \(M, M^{\prime}, M^{\prime \prime}, M^{\prime \prime \prime}\) etc., ita ut sit

\[
\dot{Y}=\frac{A T}{M(t-a)}+\frac{A^{\prime} T}{M^{\prime}\left(t-a^{\prime}\right)}+\frac{A^{\prime \prime} T^{\prime}}{M^{\prime \prime}\left(t-a^{\prime \prime}\right)}+\text { etc. }+\frac{A^{(n)} T^{\prime}}{M^{(n)}\left(t-a^{(n)}\right)}
\]

Quum fiat \(T=0\), pro \(t=a\), habemus aequationem identicam

\[
a^{n+1}+\alpha a^{n}+\alpha^{\prime} a^{n-1}+\alpha^{\prime \prime} a^{n-2}+\text { etc. }+\alpha^{(n)}=0
\]

adeoque

\[
\begin{aligned}
T=t^{n+1}-a^{n+1}+\alpha\left(t^{n}-a^{n}\right) & +\alpha^{\prime}\left(t^{n-1}-a^{n-1}\right)+\alpha^{\prime \prime}\left(t^{n-2}-a^{n-2}\right)+\text { etc. } \\
& +\alpha^{(n-1)}(t-a)
\end{aligned}
\]

Hinc dividendo per \(t-a\) fit

\[
\begin{aligned}
& \frac{r}{t-a}=t^{n}+a t^{n-1}+a a t^{n-2}+a^{3} t^{n-3}+\text { etc. }+a^{n} \\
& +\alpha t^{n-1}+\alpha a t^{(n-2)}+\alpha a a t^{n-3}+\text { etc. }+\alpha a^{n-1} \\
& +\alpha^{\prime} t^{n-2}+\alpha^{\prime} a t^{n-3}+\text { etc. }+\alpha^{\prime} a^{n-2} \\
& +\alpha^{\prime \prime} t^{n-3^{-}}+\text {etc. }+\alpha^{\prime \prime} a^{n-3} \\
& \text { + etc. etc. } \\
& +\alpha^{(n-1)}
\end{aligned}
\]

Valor huius functionis pro \(t=\boldsymbol{a}\) colligitur

\[
=(n+1) a^{n}+n \alpha a^{n-1}+(n-1) \alpha^{\prime} a^{n-2}+(n-2) \alpha^{\prime \prime} a^{n-3}+\text { etc. }+\alpha^{(n-1)}
\]

Hinc \(M\) aequalis valori ipsius \(\frac{\mathrm{d} T}{\mathrm{~d} t}\) pro \(t=a\), uti etiam aliunde constat. Perinde \(M^{\prime}, M^{\prime \prime}, M^{\prime \prime \prime}\), etc. erunt valores ipsius \(\frac{\mathrm{d} T}{\mathrm{~d} t}\) pro \(t=a^{\prime}, t=a^{\prime \prime}, t=a^{\prime \prime \prime}\) etc.

Porro invenimus valorem integralis \(\int \frac{T \mathrm{~d} t}{t-a}\), a \(t=0\) usque ad \(t=1\),

\[
\begin{aligned}
=\frac{1}{n+1}+\frac{a}{n}+\frac{a a}{n-1}+\frac{a^{3}}{n-2}+\text { etc. } & +a^{n} \\
+\frac{\alpha}{n}+\frac{\alpha a}{n-1}+\frac{\alpha a a}{n-2} & + \text { etc. }+\alpha a^{n-1} \\
+\frac{\alpha^{\prime}}{n-1}+\frac{\alpha^{\prime} a}{n-2} & + \text { etc. }+\alpha^{\prime} a^{n-2} \\
+\frac{\alpha^{\prime \prime}}{n-2} & + \text { etc. }+\alpha^{\prime \prime} a^{n-3} \\
& + \text { etc. etc. } \\
& +\alpha^{(n-1)}
\end{aligned}
\]

quos terminos ordine sequenti disponemus:

\[
\begin{aligned}
& a^{n}+\alpha a^{n-1}+\alpha^{\prime} a^{n-2}+\alpha^{\prime \prime} a^{n-3}+\text { etc. }+\alpha^{(n-1)} \\
& +\frac{1}{2}\left(a^{n-1}+\alpha a^{n-2}+\alpha^{\prime} a^{n-3}+\text { etc. }+\alpha^{(n-2)}\right) \\
& +\frac{1}{3}\left(a^{n-2}+\alpha a^{n-3}+\alpha^{\prime} a^{n-4}+\text { etc. }+\alpha^{(n-3)}\right) \\
& +\frac{1}{4}\left(a^{n-3}+\alpha a^{n-4}+\alpha^{\prime} a^{n-5}+\text { etc. }+\alpha^{(n-4)}\right) \\
& + \text { etc. } \\
& +\frac{1}{n-1}\left(a a+\alpha a+\alpha^{\prime}\right) \\
& +\frac{1}{n}(a+\alpha) \\
& +\frac{1}{n+1}
\end{aligned}
\]

Sed manifesto eadem quantitas prodit, si in producto e multiplicatione functionis \(T\) in seriem infinitam

\[
t^{-1}+\frac{1}{2} t^{-2}+\frac{1}{3} t^{-3}+\frac{1}{4} t^{-4} \text { etc. }
\]

orto, reiectis omnibus terminis, qui implicant potestates ipsius \(t\) exponentibus negativis (sive brevius, in producti parte ea, quae est functio integra ipsius \(t\) ) pro \(t\) scribitur \(a\). Supponamus itaque. fieri \({ }^{*}\) )

\[
T\left(t^{-1}+\frac{1}{2} t^{-2}+\frac{1}{3} t^{-3}+\frac{1}{4} t^{-4}+\text { etc. }\right)=T^{\prime}+T^{\prime \prime}
\]

") Vix opus erit monere, characteres \(T, T^{\prime}, T^{\prime \prime}\) alio sensu hic accipi, quam in art. 2.
ita ut \(T^{\prime}\) sit functio integra ipsius \(t\) in hoc producto contenta, \(T^{\prime \prime}\) vero pars altera, scilicet series descendens in infinitumque excurrens. Quo facto valor integralis \(\int \frac{T \mathrm{~d} t}{t-a}\) a \(t=0\) usque ad \(t=1\) aequalis erit valori functionis \(T^{\prime}\) pro \(t=a\). Quodsi itaque valores determinatos functionis

\[
\frac{T^{\prime}}{\left(\frac{\mathrm{d} T}{\mathrm{~d} t}\right)}
\]

pro \(t=a, t=a^{\prime}, t=a^{\prime \prime}, t=a^{\prime \prime \prime}\) etc. usque ad \(t=a^{(n)}\) resp. per \(R, R^{\prime}, R^{\prime \prime}\), \(R^{\prime \prime \prime} \ldots R^{(n)}\) denotamus, integrale \(\int Y \mathrm{~d}\) a \(t=0\) usque ad \(t=1\) fiet

\[
=R A+R^{\prime} A^{\prime}+R^{\prime \prime} A^{\prime \prime}+\text { etc. }+R^{(n)} A^{(n)}
\]

quod per \(\Delta\) multiplicatum exhibebit valorem vel verum vel approximatum integralis \(\int y \mathrm{~d} x\) ab \(x=g\) usque ad \(x=g+\Delta\).

9.

Hae operationes aliquanto facilius perficiuntur, si loco indeterminatae \(t\) introducitur alia \(u=2 t-1\). Scribimus quoque brevitatis caussa \(b=2 a-1\), \(b^{\prime}=2 a^{\prime}-1, b^{\prime \prime}=2 a^{\prime \prime}-1\) etc. Transeat \(T\), substituto pro \(t\) valore \(\frac{1}{2} u+\frac{1}{2}\), in \(\frac{U}{2^{n+1}}\), sive sit

\[
U=(u-b)\left(u-b^{\prime}\right)\left(u-b^{\prime \prime}\right) \ldots\left(u-b^{n}\right)
\]

Erit itaque \(\frac{\mathrm{d} T}{\mathrm{~d} t}=\frac{1}{2^{n}} \cdot \frac{\mathrm{d} U}{\mathrm{~d} u}\), adeoque \(M, M^{\prime}, M^{\prime \prime}\) etc. valores determinati ipsius \(\frac{1}{2^{n}} \cdot \frac{\mathrm{d} U}{\mathrm{~d} u}\), si deinceps statuitur \(u=b, u=b^{\prime}, u=b^{\prime \prime}\) etc.

Quum series \(t^{-1}+\frac{1}{2} t^{-2}+\frac{1}{3} t^{-3}+\frac{1}{4} t^{-4}+\) etc. nihil aliud sit quam \(\log \cdot \frac{1}{1-t^{-1}}=\log \frac{1+u^{-1}}{1-u^{-1}}\) : per substitutionem \(t=\frac{1}{2} u+\frac{1}{2}\) necessario transibit in \(2 u^{-1}+\frac{2}{3} u^{-3}+\frac{2}{5} u^{-5}+\frac{2}{7} u^{-7}+\) etc. Quodsi itaque statuimus

\[
U\left(u^{-1}+\frac{1}{3} u^{-3}+\frac{1}{3} u^{-5}+\frac{1}{7} u^{-7}+\text { etc. }\right)=U^{\prime}+U^{\prime \prime}
\]

ita ut \(U^{\prime}\) sit functio integra ipsius \(u\) in hoc producto contenta, \(U^{\prime \prime}\) vero pars altera, puta series descendens infinita, patet esse

\[
T^{\prime}+T^{\prime \prime}=\frac{1}{2^{n}}\left(U^{\prime}+U^{\prime \prime}\right)
\]

Sed manifesto \(T^{\prime}\), tamquam functio integra ipsius \(t\), per substitutionem \(t=\frac{1}{2} u+\frac{1}{2}\) necessario functionem integram ipsius \(u\) producet: contra \(T^{\prime \prime}\), quae non continet nisi potestates negativas ipsius \(t\), per eandem substitutionem tantummodo potesta-
tes negativas ipsius \(u\) gignet. Quam ob rem \(U^{\prime}\) nihil aliud erit quam \(2^{n} T^{\prime}\) per hanc substitutionem transformata, ac perinde \(U^{\prime \prime}\) producta erit ex \(2^{n} T^{\prime \prime}\). Nihil itaque intererit, sive in \(\frac{T^{\prime}}{\left(\frac{\mathrm{d} T}{\mathrm{~d} t}\right)}\) substituamus \(t=a\), sive in \(\frac{U^{\prime}}{\left(\frac{\mathrm{d} U}{\mathrm{~d} u}\right)}\) faciamus \(u=b\), unde colligimus, \(\boldsymbol{R}, \boldsymbol{R}^{\prime}, \boldsymbol{R}^{\prime \prime}, \boldsymbol{R}^{\prime \prime \prime}\) etc. etiam esse valores determinatos functionis \(\frac{U^{\prime}}{\left(\frac{\mathrm{d} U}{\mathrm{~d} u}\right)}\) pro \(u=b . u=b^{\prime}, u=b^{\prime \prime}, u=b^{\prime \prime \prime}\) etc.

10.

Antequam ulterius progrediamur, haecce praecepta per exemplum illustrabimus. Sit \(n=5\), statuamusque \(a=0, a^{\prime}=\frac{1}{5}, a^{\prime \prime}=\frac{2}{5}, a^{\prime \prime \prime}=\frac{3}{5}, a^{\prime \prime \prime \prime}=\frac{4}{5}, a^{\prime \prime \prime \prime}=1\). Hinc fit

\[
T=t^{6}-3 t^{5}+\frac{17}{5} t^{4}-\frac{9}{5} t^{3}+\frac{274}{625} t t--_{62 \frac{4}{2} 5}^{2} t
\]

Multiplicando per \(t^{-1}+\frac{1}{2} t^{-2}+\frac{1}{3} t^{-3}+\frac{1}{4} t^{-4}+\) etc. obtinemus

\[
T^{\prime}=t^{5}-\frac{5}{2} t^{4}+\frac{67}{30} t^{3}-\frac{17}{20} t t+\frac{913}{7500} t-\frac{19}{7500}
\]

Valores itaque coëfficientium \(\boldsymbol{R}, \boldsymbol{R}^{\prime}, \boldsymbol{R}^{\prime \prime}, \boldsymbol{R}^{\prime \prime \prime}, \boldsymbol{R}^{\prime \prime \prime}, \boldsymbol{R}^{\prime \prime \prime \prime}\) exprimuntur per functionem fractam

\begin{center}
%\includegraphics[max width=\textwidth]{2024_01_11_75975a03bcf8b0416cd0g-151}
\end{center}

in qua pro \(t\) deinceps substituendi sunt valores \(0, \frac{1}{5}, \frac{2}{5}, \frac{3}{5}, \frac{4}{5}, 1\). Aliquanto brevior est methodus altera, quae suppeditat \(b=-1, b^{\prime}=-\frac{3}{5}, b^{\prime \prime}=-\frac{1}{5}, b^{\prime \prime \prime}=\frac{1}{5}\), \(b^{\prime \prime \prime}=\frac{3}{5}, b^{\prime \prime \prime \prime}=1\)

\[
\begin{aligned}
& U=u^{6}-\frac{7}{5} u^{4}+\frac{259}{625} u u-\frac{9}{625} \\
& U^{\prime}=u^{5}-\frac{16}{15} u^{3}+\frac{27}{1875} u
\end{aligned}
\]

unde \(R . R^{\prime}, R^{\prime \prime}\) etc. erunt valores functionis fractae

\[
\frac{u^{2}-\frac{1}{1} \frac{6}{5} u u+\frac{27}{18} \frac{7}{5}}{6 u^{2}-\frac{28}{5} u u+\frac{31}{62} \frac{8}{3}}
\]

pro \(u=-1, u=-\frac{3}{5}, u=-\frac{1}{5}\) etc. Utraque methodus eosdem numeros profert, quos in art. 4. ex Harmonia Mensurarum tradidimus. Ceterum in casu tali, qualem hocce exemplum sistit, ubi \(a, a^{\prime}, a^{\prime \prime}\) etc. sunt quantitates rationales, valores denominatoris \(\frac{\mathrm{d} T}{\mathrm{~d} t}\) commodius in forma primitiva computantur, puta \(\left(a-a^{\prime}\right)\left(a-a^{\prime \prime}\right) \cdot\left(a-a^{\prime \prime \prime}\right) \ldots\left(a-a^{(n)}\right)\) pro \(t=a\) ac perinde de reliquis. Idem valet de denominatore \(\frac{\mathrm{d} U}{\mathrm{~d} u}\), qui pro \(u=b\) fit \(=\left(b-b^{\prime}\right)\left(b-b^{\prime \prime}\right)\left(b-b^{\prime \prime \prime}\right) \ldots\left(b-b^{(n)}\right)\).

11.

Quoties \(a, a^{\prime}, a^{\prime \prime}\) etc. vel ex parte vel omnes sant irrationales, utilis erit transformatio functionis fractae, ex qua numeros \(R, R^{\prime}, R^{\prime \prime}\) etc. derivamus, in functionem integram: principia talis transformationis, quum in libris algebraicis non inveniantur. hoc loco breviter explicabimus. Propositis scilicet tribus functionibus integris \(Z, \zeta, \zeta^{\prime}\) indeterminatae \(z\), quaeritur functio integra, quae fractae \(\frac{Z}{\zeta}\) vice fungi possit, quatenus pro \(z\) accipitur radix quaecunque aequationis \(\zeta^{\prime}=0\). Supponemus autem, \(\zeta\) pro nullo horum valorum ipsius \(z\) evanescere, sive quod eodem redit, \(\zeta\) atque \(\zeta^{\prime}\).nullum divisorem communem indeterninatum implicare. Exponentes potestatum altissimarum ipsius \(z\) in \(\zeta\) atque \(\zeta^{\prime}\) per \(k, k^{\prime}\) denotabimus.

Dividatur sueto more \(\zeta\) per \(\zeta^{\prime}\), donec residui ordo infra \(k^{\prime}\) depressus sit; statuatur residuum \(=\frac{\zeta^{\prime \prime}}{\lambda}\), eiusque ordo \(=k^{\prime \prime}\), ita ut \(\frac{1}{\lambda} z^{k^{\prime \prime}}\) sit residui terminus altissimus; divisionis quotientem ponemus \(=\frac{p}{\lambda}\). Perinde ex divisione functionis \(\zeta^{\prime}\) per \(\zeta^{\prime \prime}\). prodeat residuum \(\frac{\zeta^{\prime \prime \prime}}{\lambda^{\prime}}\) ordinis \(k^{\prime \prime \prime}\), quotiens \(\frac{p^{\prime}}{\lambda^{\prime}}\); dein rursus e divisione functionis \(\zeta^{\prime \prime}\) per \(\zeta^{\prime \prime \prime}\) prodeat residuum \(\frac{\zeta^{\prime \prime \prime \prime}}{\lambda^{\prime \prime}}\) ordinis \(k^{\prime \prime \prime \prime}\) atque quotiens \(\frac{p^{\prime \prime}}{\lambda^{\prime \prime}}\) et sic porro, donec in serie functionum \(\zeta^{\prime \prime}, \zeta^{\prime \prime \prime}\). \(\zeta^{\prime \prime \prime \prime}\) etc., quae singulae terminum suum altissimum coëfficiente 1 affectum habebunt, perveniatur ad \(\zeta^{(m)}=1\). Hoc tandem evenire debere facile perspicitur, quum quaelibet functionum \(\zeta, \zeta^{\prime} \zeta^{\prime \prime}, \zeta^{\prime \prime \prime}\) etc. cum praecedenti divisorem communem indeterminatum habere nequeat, adeoque certo divisio absque residuo fieri nequeat, quamdiu divisor fuerit ordinis maioris quam 0. Habebimus igitur seriem aequationum

\[
\begin{aligned}
& \zeta^{\prime \prime}=\lambda \zeta-p \zeta^{\prime} \\
& \zeta^{\prime \prime \prime}=\lambda^{\prime} \zeta^{\prime}-p^{\prime} \zeta^{\prime \prime} \\
& \zeta^{\prime \prime \prime \prime}=\lambda^{\prime \prime} \zeta^{\prime \prime}-p^{\prime \prime} \zeta^{\prime \prime \prime} \\
& \zeta^{\prime \prime \prime \prime}=\lambda^{\prime \prime \prime} \zeta^{\prime \prime \prime}-p^{\prime \prime \prime} \zeta^{\prime \prime \prime \prime} \\
& \text { etc. usque ad } \\
& \zeta^{(m)}=\lambda^{(m-2)} \zeta^{(m-2)}-p^{(m-2)} \zeta^{(m-1)}
\end{aligned}
\]

ubi \(\zeta^{\prime \prime}, \zeta^{\prime \prime \prime}, \zeta^{\prime \prime \prime \prime} \ldots \zeta^{(m)}\) sunt functiones integrae ipsius \(z\) ordinis \(k^{\prime \prime}, k^{\prime \prime \prime}, k^{\prime \prime \prime \prime} \ldots k^{(m)}\); numeri \(k^{\prime}, k^{\prime \prime}, k^{\prime \prime \prime} \ldots k^{(m)}\) continuo decrescentes usque ad ultimum \(k^{(m)}=0\); \(p, p^{\prime}, p^{\prime \prime}, p^{\prime \prime \prime}\) etc. quoque functiones integrae ipsius \(z\) ordinis \(k-k^{\prime}, k^{\prime}-k^{\prime \prime}, k^{\prime \prime}-k^{\prime \prime \prime}\), \(k^{\prime \prime \prime}-k^{\prime \prime \prime \prime}\) etc. (excepto casu, ubi \(k<k^{\prime}\), in quo manifesto statui debet \(p=0\) ).

His ita praeparatis formamus secundam seriem functionum integrarum ipsius \(z\), puta \(\eta, \eta^{\prime}, \eta^{\prime \prime}, \eta^{\prime \prime \prime} \ldots \eta^{(m)}\). Et quidem statuemus \(\eta=1, \eta^{\prime}=0\), reliquas vero
singulas e binis praecedentibus per eandem legem derivamus, per quam functiones \(\zeta, \zeta^{\prime}, \zeta^{\prime \prime}, \zeta^{\prime \prime \prime}\) etc. inter se nexae sunt, scilicet per aequationes

\[
\begin{aligned}
\eta^{\prime \prime} & =\lambda \eta-p \eta^{\prime} \\
\eta^{\prime \prime \prime} & =\lambda^{\prime} \eta^{\prime}-p^{\prime} \eta^{\prime \prime} \\
\eta^{\prime \prime \prime \prime} & =\lambda^{\prime \prime} \eta^{\prime \prime}-p^{\prime \prime} \eta^{\prime \prime \prime} \\
\eta^{\prime \prime \prime \prime} & =\lambda^{\prime \prime \prime} \eta^{\prime \prime \prime}-p^{\prime \prime \prime} \eta^{\prime \prime \prime \prime} \text { etc. usque ad } \\
\eta^{(m)} & =\lambda(m-2) \eta^{(m-2)}-p^{(m-2)} \eta^{(m-1)}
\end{aligned}
\]

Manifesto \(\eta^{\prime \prime}=\lambda\) hic est ordinis \(0 ; \eta^{\prime \prime \prime}=-\lambda p^{\prime}\) ordinis \(k^{\prime}-k^{\prime \prime}\), et perinde sequentes \(\eta^{\prime \prime \prime \prime}, \eta^{\prime \prime \prime \prime}\) etc. resp. ordinis \(k^{\prime}-k^{\prime \prime \prime}, k^{\prime}-k^{\prime \prime \prime \prime}\) etc., ita ut ultima \(\eta^{(m)}\) ascendat ad ordinem \(k^{\prime}-k^{(m-1)}\).

Porro consideremus tertiam functionum seriem, \(\zeta-\zeta \eta\), \(\zeta^{\prime}-\zeta \eta^{\prime}, \zeta^{\prime \prime}-\zeta \eta^{\prime \prime}\), \(\zeta^{\prime \prime \prime}-\zeta \eta^{\prime \prime \prime}\) etc., inter cuius terminos quosvis ternos consequentes manifesto similis relatio intercedet, scilicet

\[
\begin{aligned}
& \zeta^{\prime \prime}-\zeta \eta^{\prime \prime}=\lambda(\zeta-\zeta \eta)-p\left(\zeta^{\prime}-\zeta \eta^{\prime}\right) \\
& \zeta^{\prime \prime \prime}-\zeta \eta^{\prime \prime \prime}=\lambda^{\prime}\left(\zeta^{\prime}-\zeta \eta^{\prime}\right)-p^{\prime}\left(\zeta^{\prime \prime}-\zeta \eta^{\prime \prime}\right) \\
& \zeta^{\prime \prime \prime \prime}-\zeta \eta^{\prime \prime \prime \prime}=\lambda^{\prime \prime}\left(\zeta^{\prime \prime}-\zeta \eta^{\prime \prime}\right)-p^{\prime \prime}\left(\zeta^{\prime \prime \prime}-\zeta \eta^{\prime \prime \prime}\right)
\end{aligned}
\]

Iam prima harum functionum fit \(=0\), secunda \(=\zeta^{\prime}\) : hinc facile colligitur, singulas per \(\zeta^{\prime}\) divisibiles fore.

Hinc autem nullo negotio sequitur, loco fractionis \(\frac{Z}{\zeta}\) adoptari posse functionem integram \(Z \eta^{(m)}\), quatenus quidem ipsi \(\boldsymbol{z}\) non tribuantur alii valores nisi qui sint radices aequationis \(\zeta^{\prime}=0\) : manifesto enim differentia \(\frac{Z\left(1-\zeta \eta^{(m)}\right)}{\zeta}\) pro tali valore ipsius \(z\) necessario evanescit, quum \(1-\zeta \eta^{(m)}=\zeta^{(m)}-\zeta \eta^{(m)}\) per \(\zeta^{\prime}\) sit divisibilis.

Loco functionis \(Z \eta^{(m)}\) etiam adoptari poterit eius residuum ex divisione per \(\zeta^{\prime}\) ortum, cuius ordo erit inferior ordine functionis \(\zeta^{\prime}\).

Ceterum hocce residuum commodius per algorithmum sequentem immediate eruere licet. Formentur aequationes sequentes

\[
\begin{aligned}
Z & =q^{\prime} \zeta^{\prime}+Z^{\prime} \\
Z^{\prime} & =q^{\prime \prime} \zeta^{\prime \prime}+Z^{\prime \prime} \\
Z^{\prime \prime} & =q^{\prime \prime \prime} \zeta^{\prime \prime \prime}+Z^{\prime \prime \prime} \\
Z^{\prime \prime \prime} & =q^{\prime \prime \prime \prime} \zeta^{\prime \prime \prime}+Z^{\prime \prime \prime \prime} \text { etc. usque ad } \\
Z^{(m-1)} & =q^{(m)} \zeta^{(m)}+Z^{(m)}
\end{aligned}
\]

scilicet deinceps dividendo \(Z\) per \(\zeta^{\prime}\), dein residuum primae divisionis \(Z^{\prime}\) per \(\zeta^{\prime \prime}\), tum residuum secundae divisionis per \(\zeta^{\prime \prime \prime}\) ac sie porro. Quum residuum semper ad ordinem inferiorem pertineat quam divisor, ordo functionum \(Z^{\prime}, Z^{\prime \prime}, Z^{\prime \prime \prime}, Z^{\prime \prime \prime}\) etc. erit resp. inferior quam \(k^{\prime}, k^{\prime \prime}, k^{\prime \prime \prime}, k^{\prime \prime \prime \prime}\) etc.; ultima vero \(Z^{(m)}\) necessario fit \(=0\), quum divisor \(\zeta^{(m)}\) sit \(=1\). Habemus itaque

\[
Z=q^{\prime} \zeta^{\prime}+q^{\prime \prime} \zeta^{\prime \prime}+q^{\prime \prime \prime} \zeta^{\prime \prime \prime}+q^{\prime \prime \prime \prime} \zeta^{\prime \prime \prime \prime}+\text { etc. }+q^{(m)} \zeta^{(m)}
\]

Quatenus autem pro \(z\) solae radices aequationis \(\zeta^{\prime}=0\) accipiuntur, fit \(\zeta^{\prime}=0\), \(\zeta^{\prime \prime}=\zeta \eta^{\prime \prime}, \zeta^{\prime \prime \prime}=\zeta \eta^{\prime \prime \prime}, \zeta^{\prime \prime \prime \prime}=\zeta \eta^{\prime \prime \prime \prime}\) etc., unde sub eadem restrictione erit

\[
\frac{Z}{\zeta}=q^{\prime \prime} \eta^{\prime \prime}+q^{\prime \prime \prime} \eta^{\prime \prime \prime}+q^{\prime \prime \prime \prime} \eta^{\prime \prime \prime \prime}+\text { etc. }+q^{(m)} \eta^{(m)}
\]

Ordo vero huius expressionis necessario erit infra \(k^{\prime}\) : quum enim ordo quotientium \(q^{\prime \prime}, q^{\prime \prime \prime}, q^{\prime \prime \prime}\) etc. esse debeat infra \(k^{\prime}-k^{\prime \prime}, k^{\prime \prime}-k^{\prime \prime \prime}, k^{\prime \prime \prime}-k^{\prime \prime \prime}\) etc., ordo singularum partium \(q^{\prime \prime} \eta^{\prime \prime}, q^{\prime \prime \prime} \eta^{\prime \prime \prime}, q^{\prime \prime \prime \prime} \eta^{\prime \prime \prime}\) etc. erit infra \(k^{\prime}-k^{\prime \prime}, k^{\prime}-k^{\prime \prime \prime}, k^{\prime}-k^{\prime \prime \prime}\) etc.

Denique adhuc observamus, si forte inter valores indeterminatae \(z\), quos in fractione \(\frac{Z}{\zeta}\) substituere oporteat, rationales cum irrationalibus mixti reperiantur, magis e re fore, illos ab his separare atque hos solos in aequatione \(\zeta^{\prime}=0\) comprehendere. Pro rationalibus enim valoribus calculi compendio opus non erit; pro irrationalibus autem calculus tanto simplicior erit, quo minor fuerit gradus functionis integrae, ad quam fractam reducere licet.

12.

Ecce nunc exemplum transformationis in art. praec explicatae. Proposita sit functio fracta

\[
\frac{z^{6}-\frac{50}{39} z^{4}+\frac{283}{718} z z-\frac{255}{150515}}{7 z^{6}-\frac{106}{13} z^{2}+\frac{315}{128} z z-\frac{35}{129}}
\]

in qua \(z\) indefinite repraesentat radices aequationis

\[
z^{7}-\frac{2}{1} \cdot \frac{1}{3} z^{5}+\frac{105}{14 \frac{5}{4}} z^{3}-\frac{35}{429} z=0
\]

Si hic omnes septem radices complecti vellemus, ad functionem integram sexti ordinis delaberemur. Manifesto autem pro valore rationali \(z=0\) computus fractionis obvius est, datque valorem \(\frac{356}{1 \frac{35}{2} 5}\) : quapropter seposita hac radice in aequatione sexti gradus subsistemus:

\[
z^{6}-\frac{21}{13} z^{4}+\frac{105}{14 \frac{5}{3}} z z-\frac{35}{429}=0
\]

quo pacto facile praevidemus orturam esse functionem integram quarti ordinis. Iam ex applicatione praeceptorum praecedentium prodeunt sequentia:

\[
\begin{aligned}
& \zeta=7 z^{6}-\frac{105}{13} z^{4}+\frac{315}{143} z z-\frac{35}{429} \\
& \zeta^{\prime}=z^{6}-\frac{21}{13} z^{4}+\frac{105}{143} z z-\frac{35}{429} \\
& \zeta^{\prime \prime}=z^{4}-\frac{10}{11} z z+\frac{5}{33} \\
& \zeta^{\prime \prime \prime}=z z-\frac{3}{7} \\
& \zeta^{\prime \prime \prime \prime}=1 \\
& \lambda=\frac{1}{4} \frac{3}{2} \quad p=\frac{13}{6} \\
& \lambda^{\prime}=-\frac{4719}{280} \quad p^{\prime}=-\frac{4719}{280} z z+\frac{3333}{280} \\
& \lambda^{\prime \prime}=-{ }^{1 \frac{4}{8}} 7 \quad p^{\prime \prime}=-{ }^{1 \frac{4}{8}} z z+\frac{777}{8} z \\
& \eta=1 \\
& \eta^{\prime}=0 \\
& \eta^{\prime \prime}=\frac{13}{42} \\
& \eta^{\prime \prime \prime}=\frac{20449}{3920} z z-\frac{14443}{3920} \\
& \eta^{\prime \prime \prime \prime}=6 \frac{134}{640} 7 z^{4}-1 \frac{27413}{1120} z z+1 \frac{20-263}{480} 3 \\
& Z=z^{6}-\frac{50}{39} z^{4}+\frac{28}{71} \frac{3}{5} z z-\frac{256}{150 \frac{5}{15}} ; \quad q^{\prime}=1 \\
& Z^{\prime}=\frac{1}{3} z^{4}-\frac{22}{6} z z+\frac{323}{5005} \quad q^{\prime \prime}=\frac{1}{3} \\
& Z^{\prime \prime}=-\frac{76}{2145} z z+\frac{632}{45045} \quad q^{\prime \prime \prime}=-\frac{76}{2145} \\
& Z^{\prime \prime \prime}=-\frac{4}{3465} \quad q^{\prime \prime \prime \prime}=-\frac{4}{3465}
\end{aligned}
\]

Hinc tandem derivatur functio integra fractioni propositae aequivalens:

\section*{\(-\frac{1859}{16800} z^{4}-\frac{1573}{29400} z z+\frac{7947}{39200}\)}
13.

Ad determinandum gradum praecisionis, qua formula nostra integralis \(\boldsymbol{R} A+\boldsymbol{R}^{\prime} A^{\prime}+\boldsymbol{R}^{\prime \prime} A^{\prime \prime}+\) etc. \(+\boldsymbol{R}^{(n)} A^{(n)}\) gaudet, statuamus generaliter

\[
\boldsymbol{R} \boldsymbol{a}^{m}+\boldsymbol{R}^{\prime} \boldsymbol{a}^{\prime m}+\boldsymbol{R}^{\prime \prime} \boldsymbol{a}^{\prime \prime m}+\text { etc. }+\boldsymbol{R}^{(n)} a^{(n) m}=\frac{1}{m+1} \div k^{(m)}
\]

ita ut \(k^{(m)}\) sit differentia inter integralis \(\int t^{m} \mathrm{~d} t\) a \(t=0\) usque ad \(t=1\) sumti valorem verum atque approximatum. Habebimus itaque, singulis fractionibus in series evolutis,

\[
\begin{aligned}
& \frac{R}{t-a}+\frac{R^{\prime}}{t-a^{\prime}}+\frac{R^{\prime \prime}}{t-a^{\prime \prime}}+\text { etc. }+\frac{R^{(n)}}{t-a^{(n)}} \\
& \quad=(1-k) t^{-1}+\left(\frac{1}{2}-k^{\prime}\right) t^{-2}+\left(\frac{1}{3}-k^{\prime \prime}\right) t^{-3}+\left(\frac{1}{4}-k^{\prime \prime \prime}\right) t^{-4}+\text { etc. } \\
& \quad=t^{-1}+\frac{1}{2} t^{-2}+\frac{1}{3} t^{-3}+\frac{1}{4} t^{-4}+\text { etc. }-\theta
\end{aligned}
\]

si statuimus

\[
\theta=k t^{-1}+k^{\prime} t^{-2}+k^{\prime \prime} t^{-3}+k^{\prime \prime \prime} t^{-4}+\text { etc. }
\]

sive potius (quum iam sciamus, \(k, k^{\prime}, k^{\prime \prime}, k^{\prime \prime \prime}\) etc. usque ad \(k^{(n)}\) sponte evanescere debere)

\[
\theta=k^{(n+1)} t^{-(n+2)}+k^{(n+2)} t^{-(n+3)}+k^{(n+3)} t^{-(n+4)}+\text { etc. }
\]

Multiplicando per \(T\) fit

\[
T\left(\frac{R}{t-a}+\frac{R^{\prime}}{t-a^{\prime}}+\frac{R^{\prime \prime}}{t-a^{\prime \prime}}+\text { etc. }+\frac{R\left({ }^{n}\right)}{t-a^{(n)}}\right)=T^{\prime}+T^{\prime \prime}-T \theta
\]

Pars prior huius aequationis est functio integra ipsius \(t\) ordinis \(n\), eiusque valores determinati pro \(t=a, t=a^{\prime}, t=a^{\prime \prime}\) etc. resp. fiunt \(M R, M^{\prime} R^{\prime}, M^{\prime \prime} R^{\prime \prime}\) etc.: quapropter, quum eadem valeant de functione \(T^{\prime}\), uti ex ipso modo numeros \(R, R^{\prime}, R^{\prime \prime}\) etc. determinandi perspicuum est, necessario illa pars prior aequationis identica esse debet cum \(T^{\prime}\), adeoque \(T^{\prime \prime}=T \theta\). Oritur itaque \(\theta\) ex evolutione fractionis \(\frac{T^{\prime \prime}}{T}\), quo pacto coëfficientes \(k^{(n+1)}, k^{(n+2)}\) etc. quousque libet determinari poterunt. Quibus inventis correctio valoris nostri approximati integralis \(\int y \mathrm{~d} t\) erit

\[
=k^{(n+1)} K^{(n+1)}+k^{(n+2)} K^{(n+2)}+\text { etc. }
\]

si series, in quam evolvitur \(y\), est

\[
y=K+K^{\prime} t+K^{\prime \prime} t t+K^{\prime \prime \prime} t^{3}+\text { etc. }
\]

14.

Si magis placet, correctionem exprimere per coëfficientes seriei secundum potestates ipsius \(t-\frac{1}{2}\) progredientis

\[
y=L+L^{\prime}\left(t-\frac{1}{2}\right)+L^{\prime \prime}\left(t-\frac{1}{2}\right)^{2}+L^{\prime \prime \prime}\left(t-\frac{1}{2}\right)^{3}+\text { etc. }
\]

illa erit

\[
=l^{(n+1)} L^{(n+1)}+l^{(n+2)} L^{(n+2)}+l^{(n+3)} L^{(n+3)}+\text { etc. }
\]

si generaliter per \(l^{(m)}\) exprimimus correctionem valoris approximati integralis \(\int\left(t-\frac{1}{2}\right)^{m} \mathrm{~d} t\). Hae correctiones \(l^{(m)}\) cum correctionibus \(k^{(m)}\) nexae erunt per aequationem

\[
l^{(m)}=k^{(m)}-\frac{1}{2} m k^{(m-1)}+\frac{1}{4} \cdot \frac{m \cdot m-1}{1 \cdot 2} k^{(m-2)}-\frac{1}{8} \cdot \frac{m \cdot m-1 \cdot m-2}{1 \cdot 2 \cdot 3} k^{(m-3)}+\text { etc. }
\]

Quo vero illas independenter eruere possimus, perpendamus, functionem \(\theta\) per substitutionem \(t=\frac{1}{2} u+\frac{1}{2}\) transire in

\[
\begin{aligned}
& 2 k\left(u^{-1}-u^{-2}+u^{-3}-u^{-4}+\text { etc. }\right) \\
+ & 4 k^{\prime}\left(u^{-2}-2 u^{-3}+3 u^{-4}-4 u^{-5}+\text { etc. }\right) \\
+ & 8 k^{\prime \prime}\left(u^{-3}-3 u^{-4}+6 u^{-5}-10 u^{-6}+\text { etc. }\right) \\
+ & 16 k^{\prime \prime \prime}\left(u^{-4}-4 u^{-5}+10 u^{-6}-20 u^{-7}+\text { etc. }\right) \\
+ & \text { etc. }
\end{aligned}
\]

sive in

\[
\begin{gathered}
2 k u^{-1}+4\left(k^{\prime}-\frac{1}{2}\right) u^{-2}+8\left(k^{\prime \prime}-\frac{1}{2} \cdot 2 k^{\prime}+\frac{1}{4} k\right) u^{-3} \\
\quad+16\left(k^{\prime \prime \prime}-\frac{1}{2} \cdot 3 k^{\prime \prime}+\frac{1}{4} \cdot 3 k^{\prime}-\frac{1}{8} k\right) u^{-4}+\text { etc. }
\end{gathered}
\]

sive in

\[
2 l u^{-1}+4 l^{\prime} u^{-2}+8 l^{\prime \prime} u^{-3}+16 l^{\prime \prime \prime} u^{-4}+\text { etc. }
\]

sive denique, quum a priori sciamus, \(l, l^{\prime}, l^{\prime \prime}, l^{\prime \prime \prime}\) etc. usque ad \(l^{(n)}\) sponte evanescere, in

\[
2^{n+2} l^{(n+1)} u^{-(n+2)}+2^{n+3} l^{(n+2)} u^{-(n+3)}+2^{n+4} l^{(n+4)} u^{-(n+4)}+\text { etc. }
\]

At \(\theta=\frac{T^{\prime \prime}}{T}\); quare quum \(T, T^{\prime \prime}\) per substitutionem \(t=\frac{1}{2} u+\frac{1}{2}\) transeant in \(\frac{U}{2^{n+1}}, \frac{U^{\prime \prime}}{2^{n}}\), (art. 9), functio \(\theta\) per eandem substitutionem transibit in \(\frac{2 U^{\prime \prime}}{U}\). Quodsi itaque seriem ex evolutione fractionis \(\frac{U^{\prime \prime}}{U}\) oriundam per \(Q\) designamus, erit

\[
Q=2^{n+1} l^{(n+1)} u^{-(n+2)}+2^{n+2} l^{(n+2)} u^{-(n+3)}+2^{n+3} l^{(n+3)} u^{-(n+4)}+\text { etc. }
\]

quo pacto coëfficientes \(l^{(n+1)}, l^{(n+2)}\) etc. quousque lubet erui poterunt.

Ita in exemplo art. 10 invenimus

\[
\begin{aligned}
& U^{\prime \prime}=-\frac{176}{13125} u^{-1}-\frac{304}{281125} u^{-3}-\frac{25576}{309375} u^{-5}-\text { etc. } \\
& Q=-\frac{176}{13125} u^{-7}-\frac{83}{28125} u^{-9}-\frac{189856}{42968875} u^{-11}-\text { etc. }
\end{aligned}
\]

adeoque correctio valoris approximati integralis

\[
=-\frac{11}{52500} L^{\mathrm{VI}}-\frac{13}{112500} L^{\mathrm{VIII}}-\frac{5933}{137500000} L^{\mathrm{x}}-\text { etc. }
\]

15.

Coëfficiens \(K^{(m)}\) functionis \(y\) in seriem evolutae fit, per theorema TAYLorI, aequalis valori ipsius

\[
\frac{1}{1.2 .3 \ldots \ldots m} \cdot \frac{\mathrm{d}^{m} y}{\mathrm{~d} t^{m}} \quad \text { sive } \frac{\Delta^{m}}{1.2 .3 \ldots . m} \cdot \frac{\mathrm{d}^{m} y}{\mathrm{~d} x^{m}}
\]

pro \(t=0\) sive \(x=g\); perinde coëfficiens \(L^{(m)}\) est valor eiusdem expressionis pro \(t=\frac{1}{2}\) sive \(u=0\) sive \(x=g+\frac{1}{2} \Delta\) : utrique coëfficienti ordinem \(m\) tribuemus. Generaliter itaque loquendo integratio nostra usque ad ordinem \(n\) inclus. exacta erit, quicunque valores pro \(a, a^{\prime}, a^{\prime \prime} \ldots a^{(n)}\) accipiantur. Attamen hinc nihil obstat, quominus pro valoribus harum quantitatum scite electis praecisio ad altiorem gradum evehatur. Ita iam supra vidimus, in methodo \textsc{Cotesii} i. e. pro \(a=0, a=\frac{1}{n}, a^{\prime \prime}=\frac{2}{n}, a^{\prime \prime \prime}=\frac{3}{n}\) etc. praecisionem sponte ad ordinem \(n+1\) inclus. extendi, quoties \(n\) sit numerus par. Generaliter patet, si valores \(a, a^{\prime}, a^{\prime \prime}, a^{\prime \prime \prime}\) etc. ita fuerint electi, ut in functione \(T^{\prime \prime}\) vel \(U^{\prime \prime}\) ab initio excidat terminus unus pluresve, praecisionem totidem gradibus ultra ordinem \(n\) promotum iri, quot termini exciderint. Hinc facile colligitur, quum multitudo quantitatum quas eligere conceditur sit \(n+1\), per idoneam earum determinationem praecisionem semper ad ordinem \(2 n+1\) inclus. evehi posse, quo pacto adiumento \(n+1\) terminorum eundem praecisionis ordinem assequi licebit, ad quem attingendum \(2 n+1\) vel \(2 n+2\) terminos in usum vocare oporteret, si \textsc{Cotesii} methodum sequeremur.

16.

Totum hoc negotium in eo vertitur, ut pro quovis valore dato ipsius \(n\) functionem \(T\) eruamus formae \(t^{n+1}+\alpha t^{n}+\alpha^{\prime} t^{n-1}+\alpha^{\prime \prime} t^{n-2}\) etc. itaque comparatam, ut in producto

\[
T\left(t^{-1}+\frac{1}{2} t^{-2}+\frac{1}{3} t^{-3}+\frac{1}{4} t^{-4}+\text { etc. }\right)
\]

evoluto potestates \(t^{-1}, t^{-2}, t^{-3} \ldots t^{-(n+1)}\) omnes nanciscantur coëfficientem 0 ; aut si magis placet, functionem \(U\) formae \(u^{n+1}+b u^{n}+b^{\prime} u^{n-1}+b^{\prime \prime} u^{n-2}+\) etc., cuius productum per \(u^{-1}+\frac{1}{3} u^{-3}+\frac{1}{3} u^{-5}+\frac{1}{7} u^{-7}+\) etc. liberum evadat a potesta-
tibus \(u^{-1}, u^{-2}, u^{-3}, u^{-4} \ldots u^{-(n+1)}\). Modus posterior aliquanto simplicior erit: quum enim facile perspiciatur, coëfficientes ipsius \(U\), ut conditioni praescriptae satisfiat, alternatim evanescere debere, sive statui \(b=0, b^{\prime \prime}=0, b^{\prime \prime \prime \prime}=0\) etc., laboris dimidia fere pars iam absoluta censenda erit. Evolvamus casus quosdam simpliciores.

I. Pro \(n=0\), coëfficiens unicus ipsius \(t^{-1}\) in producto

\[
(t+\alpha)\left(t^{-1}+\frac{1}{2} t^{-2}+\frac{1}{3} t^{-3}+\text { etc. }\right)
\]

evanescere debet. Qui quum fiat \(=\frac{1}{2}+\alpha\), habemus \(\alpha=-\frac{1}{2}\), sive \(T=t-\frac{1}{2}\). Perinde \(U=u\).

II. Pro \(n=1\), determinatio ipsius \(T\) pendet a duabus aequationibus

\[
\begin{aligned}
& 0=\frac{1}{3}+\frac{1}{2} \alpha+\alpha^{\prime} \\
& 0=\frac{1}{4}+\frac{1}{3} \alpha+\frac{1}{2} \alpha^{\prime}
\end{aligned}
\]

unde deducimus \(\alpha=-1, \alpha^{\prime}=+\frac{1}{6}\), sive \(T=t t-t+\frac{1}{6}\). Determinatio functionis \(U\) unicam aequationem affert

\[
0=\frac{1}{3}+b^{\prime}
\]

unde \(b^{\prime}=-\frac{1}{3}\), sive \(U=u u-\frac{1}{3}\).

III. Pro \(n=2\), functio \(T\) determinatur adiumento trium aequationum

\[
\begin{aligned}
& 0=\frac{1}{4}+\frac{1}{3} \alpha+\frac{1}{2} \alpha^{\prime}+\alpha^{\prime \prime} \\
& 0=\frac{1}{5}+\frac{1}{4} \alpha+\frac{1}{3} \alpha^{\prime}+\frac{1}{2} \alpha^{\prime \prime} \\
& 0=\frac{1}{6}+\frac{1}{5} \alpha+\frac{1}{4} \alpha^{\prime}+\frac{1}{3} \alpha^{\prime \prime}
\end{aligned}
\]

unde nanciscimur \(\alpha=-\frac{3}{2}, \alpha^{\prime}=\frac{3}{5}, \alpha^{\prime \prime}=-\frac{1}{20}\), adeoque \(T=t^{3}-\frac{3}{2} t t+\frac{3}{5} t-\frac{1}{20}\). Ad determinandam \(U\) unica aequatio sufficit

\[
0=\frac{1}{5}+\frac{1}{3} b^{\prime}
\]

unde \(b^{\prime}=-\frac{3}{5}\) sive \(U=u^{3}-\frac{3}{5} u\).

Attamen hunc modum, qui calculos continuo molestiores adducit, hic ulterius non persequemur, sed ad fontem genuinum solutionis generalis progrediemur.

17.

Proposita fractione continua

\begin{center}
%\includegraphics[max width=\textwidth]{2024_01_11_75975a03bcf8b0416cd0g-160}
\end{center}

constat, fractiones continuo magis appropinquantes inveniri per algorithmum sequentem. Formentur duae quantitatum series, \(V, V^{\prime}, V^{\prime \prime}, V^{\prime \prime \prime}\) etc., \(W, W^{\prime}, W^{\prime \prime}, W^{\prime \prime \prime}\) etc. per hasce formulas

\[
\begin{array}{ll}
V=0 & W=1 \\
V^{\prime}=v & W^{\prime}=w W \\
V^{\prime \prime}=w^{\prime} V^{\prime}+v^{\prime} V & W^{\prime \prime}=w^{\prime} W^{\prime}+v^{\prime} W \\
V^{\prime \prime \prime}=w^{\prime \prime} V^{\prime \prime}+v^{\prime \prime} V^{\prime} & W^{\prime \prime \prime}=w^{\prime \prime} W^{\prime \prime}+v^{\prime \prime} W^{\prime} \\
V^{\prime \prime \prime}=w^{\prime \prime \prime} V^{\prime \prime \prime}+v^{\prime \prime \prime} V^{\prime \prime} & W^{\prime \prime \prime \prime}=w^{\prime \prime \prime} W^{\prime \prime \prime}+v^{\prime \prime \prime} W^{\prime \prime}
\end{array}
\]

etc. eritque

\[
\begin{aligned}
& \frac{V}{W}=0 \\
& \frac{V^{\prime}}{W^{\prime}}=\frac{v}{w} \\
& \frac{V^{\prime \prime}}{W^{\prime \prime}}=\frac{v}{w+\frac{v^{\prime}}{w^{\prime}}} \\
& \bar{V}^{\prime \prime \prime}=\frac{v}{w+\frac{v^{\prime}}{w^{\prime}+\frac{v^{\prime \prime}}{w^{\prime \prime}}}}
\end{aligned}
\]

et sic porro. Praeterea constat, vel facile ex ipsis aequationibus praecedentibus confirmatur, esse

\[
\begin{aligned}
& V W^{\prime}-V^{\prime} W=-v \\
& V^{\prime} W^{\prime \prime}-V^{\prime \prime} W^{\prime}=+v v^{\prime} \\
& V^{\prime \prime} W^{\prime \prime \prime}-V^{\prime \prime \prime} W^{\prime \prime}=-v v^{\prime} v^{\prime \prime} \\
& V^{\prime \prime \prime} W^{\prime \prime \prime}-V^{\prime \prime \prime \prime} W^{\prime \prime \prime}=+v v^{\prime} v^{\prime \prime} v^{\prime \prime \prime}
\end{aligned}
\]

etc. Hinc perspicuum est, seriei

\[
\frac{v}{W W^{\prime}}-\frac{v v^{\prime}}{W^{\prime} W^{\prime \prime}}+\frac{v v^{\prime} v^{\prime \prime}}{W^{\prime \prime} W^{\prime \prime \prime}}-\frac{v v^{\prime} v^{\prime \prime} v^{\prime \prime \prime}}{W^{\prime \prime \prime} W^{\prime \prime \prime}}+\text { etc. }
\]

terminum primum esse \(=\frac{V^{\prime}}{W^{\prime}}\)

summam duorum terminorum primorum \(=\frac{V^{\prime \prime}}{W^{\prime \prime}}\)

summam trium terminorum primorum \(=\frac{V^{\prime \prime \prime}}{W^{\prime \prime \prime}}\)

summam quatuor terminorum primorum \(=\frac{V^{\prime \prime \prime \prime}}{W^{\prime \prime \prime \prime}}\)

et sic porro; quocirca series ipsa vel in infinitum vel usque dum abrumpatur continuata ipsam fractionem continuam \(\varphi\) exprimet. Simul hinc habetur differentia inter \(\varphi\) atque singulas fractiones appropinquantes \(\frac{V^{\prime}}{W^{\prime}}, \frac{V^{\prime \prime}}{W^{\prime \prime}}, \frac{V^{\prime \prime \prime}}{W^{\prime \prime \prime}}\) etc.

E formula 33 art. 14 Disquisitionum generalium circa seriem infinitam mutando \(t\) in \(\frac{1}{u}\), facile obtinemus transformationem seriei

\[
\varphi=u^{-1}+\frac{1}{3} u^{-3}+\frac{1}{5} u^{-5}+\frac{1}{7} u^{-7}+\text { etc. }
\]

in fractionem continuam sequentem

\[
\frac{\frac{1}{u-\frac{1}{3}}}{u-\frac{\frac{2.2}{3.5}}{u-\frac{\frac{5.7}{3.3}}{\frac{4.4}{7.9}}}}
\]

ita ut habeatur

\[
\begin{aligned}
& v=1, v^{\prime}=-\frac{1}{3}, v^{\prime \prime}=-\frac{4}{15}, v^{\prime \prime \prime}=-\frac{9}{35}, v^{\prime \prime \prime \prime}=-\frac{16}{63} \text { etc. } \\
& w=w^{\prime}=w^{\prime \prime}=w^{\prime \prime \prime}=w^{\prime \prime \prime \prime} \text { etc. }=u .
\end{aligned}
\]

Hinc pro \(V, V^{\prime}, V^{\prime \prime}, V^{\prime \prime \prime}\) etc. \(W, W^{\prime}, W^{\prime \prime}, W^{\prime \prime \prime}\) etc. nanciscimur valores sequentes

\[
\begin{aligned}
& V=0 \\
& W=1 \\
& V^{\prime}=1 \\
& W^{\prime}=u \\
& V^{\prime \prime}=u \\
& W^{\prime \prime}=u u-\frac{1}{3} \\
& V^{\prime \prime \prime}=u u-\frac{4}{15}, \\
& W^{\prime \prime \prime}=u^{3}-\frac{3}{5} u \\
& V^{m \prime \prime}=u^{3}-\frac{11}{21} u, \\
& W^{\prime \prime \prime \prime}=u^{4}-\frac{6}{7} u u+\frac{3}{35} \\
& V^{v}=u^{4}-\frac{7}{9} u u+\frac{64}{945} \\
& W^{v}=u^{5}-{ }_{9}^{19} u^{3}+\frac{5}{21} u \\
& V^{\mathrm{VI}}=u^{5}-\frac{34}{3} \frac{4}{3} u^{3}+\frac{1}{5} u, \\
& W^{V 1}=u^{6}-\frac{15}{11} u^{4}+\frac{5}{11} u u-\frac{5}{231} \\
& V^{\mathrm{VII}}=u^{6}-\frac{50}{39} u^{4}+\frac{28}{715} u u-\frac{256}{150} \frac{6}{15}, \\
& W^{\mathrm{Vu}}=u^{7}-\frac{21}{13} u^{5}+\frac{105}{143} u^{3}-\frac{35}{429} u \text { etc }
\end{aligned}
\]

Leviattentione adhibita elucet, singulas \(V, \boldsymbol{V}^{\prime}, V^{\prime \prime}, V^{\prime \prime \prime}\) etc. \(\boldsymbol{W}, \boldsymbol{W}^{\prime}, W^{\prime \prime}, \boldsymbol{W}^{\prime \prime}\) etc. fieri functiones integras indeterminatae \(u\); terminum altissimum in \(\nabla^{(m)}\) fieri \(u^{m-1}\), potestatesque \(u^{m-2}, u^{m-4}, u^{m-6}\) etc. abesse; terminum altissimam vero in \(W^{(m)}\) fieri \(u^{m}\), atque abesse potestates \(u^{m-1}, u^{m-3}, u^{m-5}\) etc. Per ea autem, quae supra demonstravimus, erit

\(\varphi=\frac{1}{W W^{\prime}}+\frac{1}{3 W^{\prime} W^{\prime \prime}}+\frac{2 \cdot 2}{3 \cdot 3 \cdot 5 W^{\prime \prime} W^{\prime \prime \prime}}+\frac{2 \cdot 2 \cdot 3 \cdot 3}{3 \cdot 3 \cdot 5 \cdot 5 \cdot 7 W^{\prime \prime \prime} W^{\prime \prime \prime}}+\frac{2 \cdot 2 \cdot 3 \cdot 3 \cdot 4 \cdot 4}{3 \cdot 3 \cdot 5 \cdot 5 \cdot 7 \cdot 7 \cdot 9 W^{\prime \prime \prime} W^{\prime \prime \prime}}+\) etc. ac proin generaliter

\[
\begin{aligned}
\varphi-\frac{V^{(m)}}{W^{(m)}}= & \frac{2 \cdot 2 \cdot 3 \cdot 3 \ldots \ldots \cdot m}{3 \cdot 3 \cdot 5 \cdot 5 \ldots(2 m-1)(2 m+1) W^{(m)} W^{(m+1)}} \\
& +\frac{2 \cdot 2 \cdot 3 \cdot 3 \ldots \ldots(m+1)(m+1)}{3 \cdot 3 \cdot 5 \cdot 5 \cdots \cdots(2 m+1)(2 m+3) W^{(m+1)} W^{(m+2)}} \\
& + \text { etc. }
\end{aligned}
\]

Si igitur \(\mathfrak{Q}-\frac{V^{(m)}}{W^{(m)}}\) in seriem descendentem convertitur, eius terminus primus erit

\[
=\frac{2 \cdot 2 \cdot 3 \cdot 3 \ldots m \cdot m u^{-(2 m+1)}}{3 \cdot 3 \cdot 5 \cdot 5 \cdots \cdot(2 m-1)(2 m+1)}
\]

Productum vero \(\wp W^{(m)}\) compositum erit e functione integra \(V^{(m)}\) atque serie infinita, cuius terminus primus

\[
=\frac{2 \cdot 2 \cdot 3 \cdot 3 \ldots \ldots m m u^{-(m+1)}}{3 \cdot 3 \cdot 5 \cdot 5 \ldots \ldots(2 m-1)(2 m+1)}
\]

Hinc igitur sponte inventa est functio \(U\) ordinis \(n+1\); quae conditioni in art. praec. stabilitae satisfacit, scilicet ut productum \(\varphi U\) liberum evadat a potestatibus \(u^{-1}, u^{-2}, u^{-3} \ldots . . u^{-(n+1)}\). Scilicet non est alia quam \(W^{(n+1)}\), simulque patet, \(U^{\prime}\) aequalem fieri ipsi \(V^{(m+1)}\), nec non terminum primum ipsius \(U^{\prime \prime}\) esse

\[
=\frac{2 \cdot 2 \cdot 3 \cdot 3 \ldots \ldots(n+1)(n+1)}{3 \cdot 3 \cdot 5 \cdot 5 \cdots \cdots(2 n+1)(2 n+3)} \cdot u^{-(n+2)}
\]

Quodsi igitur pro \(b, b^{\prime}, b^{\prime \prime} \ldots . b^{(n)}\) accipiuntur radices aequationis \(W^{(n+1)}=0\), valoresque coëfficientium \(R, R^{\prime}, R^{\prime \prime} \ldots R^{(n)}\) per praecepta supra tradita eruuntur, formula nostra integralis praecisione gaudebit ad ordinem \(2 n+1\) ascendente, eiusque correctio exprimetur proxime per

\[
\frac{1}{2^{2 n+2}} \cdot \frac{2 \cdot 2 \cdot 3 \cdot 3 \ldots \ldots(n+1)(n+1)}{3 \cdot 3 \cdot 5 \cdot 5 \ldots \ldots(2 n+1)(2 n+3)} L^{(2 n+2)}=\frac{1 \cdot 1 \cdot 2 \cdot 2 \cdot 3 \cdot 3 \ldots \ldots(n+1)(n+1)}{2 \cdot 6 \cdot 6 \cdot 10 \cdot 10 \cdot 11 \ldots \ldots(4 n+2)(4 n+6)} L^{(2 n+2)}
\]

18.

Disquisitiones art. praec. functiones idoneas \(U\). pro singulis valoribus numeri \(n\) invenire quidem docent, sed successive tantum, dum a valoribus minoribus ad maiores transeundum est. Facile autem animadvertimus, has functiones generaliter exprimi per

\[
\begin{aligned}
u^{n+1}-\frac{(n+1) n}{2 \cdot(2 n+1)} u^{n-1} & +\frac{(n+1) n(n-1)(n-2)}{2 \cdot 4(2 n+1)(2 n-1)} u^{n-3}-\frac{(n+1) n(n-1)(n-2)(n-3)(n-1)}{2 \cdot 4 \cdot 6 \cdot(2 n+1)(2 n-1)(2 n-3)} u^{n-5} \\
& + \text { etc. }
\end{aligned}
\]

sive etiam, si characteristica \(F\) ad normam commentationis supra citatae utimur, per

\[
u^{n+1} F\left(-\frac{1}{2} n,-\frac{1}{2}(n+1),-\left(n+\frac{1}{2}\right), u^{-2}\right)
\]

Haecce inductio facile in demonstrationem rigorosam convertitur per methodum vulgo notam, aut, si ita videtur, adiumento formulae 19 in comment. cit. Functio \(U\), si magis placet, etiam ordine terminorum inverso, exprimi potest per

\[
\pm \frac{3 \cdot 5 \cdot 7 \ldots(n+1)}{(n+3)(n+5) \ldots(2 n+1)} \cdot u F\left(-\frac{1}{2} n, \frac{1}{2}(n+3), \frac{3}{2}, u u\right)
\]

pro \(n\) pari, valente signo superiori vel inferiori, prout \(\frac{1}{2} n\) par est vel impar aut per

\[
\pm \frac{1.3 \cdot 5 \ldots n}{(n+2)(n+4) \ldots(2 n+1)} F\left(-\frac{1}{2}(n+1), \frac{1}{2} n+1, \frac{1}{2}, u u\right)
\]

pro \(n\) impari, valente signo superioxi vel inferiori, prout \(\frac{1}{2}(n+1)\) par est vel impar.

Functio \(U^{\prime}\) expressionem generalem aeque simplicem non admittit: facile tamen ex ipsa genesi quantitatum \(V, V^{\prime}, V^{\prime \prime}\) etc. colligitur, terminum ultimum ipsius \(U^{\prime}\) pro \(n\) pari fieri

\[
= \pm \frac{2.2 .4 \cdot 4 \cdot 6 \cdot 6 \ldots \ldots n . n}{3.5 .7 \cdot 9.11 .13 \ldots \ldots(2 n-1)(2 n+1)}
\]

signo superiori vel inferiori valente, prout \(\frac{1}{2} n\) par est vel impar.

Functio \(U^{\prime \prime}=\varphi W^{(n+1)}-V^{(n+1)}\), cuius terminum primum iam in art. praec. assignare docuimus, etiam per algorithmum recurrentem evolvi potest, quum manifesto generaliter habeatur

\[
\begin{aligned}
& \varphi W^{\prime \prime}-V^{\prime \prime}=w^{\prime}\left(\varphi W^{\prime}-V^{\prime}\right)+v^{\prime}(\varphi W-V) \\
& \varphi W^{\prime \prime \prime}-V^{\prime \prime \prime}=w^{\prime \prime}\left(\varphi W^{\prime \prime}-V^{\prime \prime}\right)+v^{\prime \prime}\left(\varphi W^{\prime}-V^{\prime}\right) \\
& \varphi W^{\prime \prime \prime \prime}-V^{\prime \prime \prime \prime}=w^{\prime \prime \prime}\left(\varphi W^{\prime \prime \prime}-V^{\prime \prime \prime}\right)+v^{\prime \prime \prime}\left(\varphi W^{\prime \prime}-V^{\prime \prime}\right)
\end{aligned}
\]

etc. adeoque eo quem tractamus casu

\[
\varphi W^{(m+2)}-V^{(m+2)}=u\left(\varphi W^{(m+1)}-V^{(m+1)}\right)-\frac{(m+1)^{2}}{(2 m-1)(2 m+1)}\left(\varphi W^{(m)}-V^{(m)}\right)
\]

Ita invenimus

\[
\begin{aligned}
& \varphi W-V=u^{-1}+\frac{1}{3} u^{-3}+\frac{1}{5} u^{-5}+\frac{1}{7} u^{-7}+\text { etc. } \\
& \varphi W^{\prime}-V^{\prime}=\frac{1}{3} u^{-2}+\frac{1}{5} u^{-4}+\frac{1}{7} u^{-6}+\frac{1}{9} u^{-8}+\text { etc. } \\
& \varphi W^{\prime \prime}-V^{\prime \prime}=\frac{4}{45} u^{-3}+\frac{8}{105} u^{-5}+\frac{4}{63} u^{-7}+\frac{11}{20 \frac{2}{9}} u^{-9}+\text { etc. } \\
& \varphi W^{\prime \prime \prime}-V^{\prime \prime \prime}=\frac{4}{175} u^{-4}+\frac{8}{315} u^{-6}+\frac{6}{165} u^{-8}+\frac{16}{715} u^{-10}+\text { etc. }
\end{aligned}
\]

etc. quas series ita quoque exhibere licet

\[
\begin{aligned}
& \varphi W-V=u^{-1}\left(1+\frac{1.2}{2.3} u^{-2}+\frac{1.2 \cdot 3.4}{2.4 .3 .5} u^{-4}+\frac{1.2 \cdot 3.4 \cdot 5.6}{2.4 .6 .3 .5 \cdot 7} u^{-6}+\text { etc. }\right) \\
& \varphi W^{\prime}-V^{\prime}=\frac{1}{3} u^{-2}\left(1+\frac{2.3}{2.5} u^{-4}+\frac{2 \cdot 3 \cdot 4.5}{2.4 \cdot 5.7} u^{-4}+\frac{2.3 \cdot 4 \cdot 5 \cdot 6 \cdot 7}{2 \cdot 4 \cdot 6 \cdot 5 \cdot 7 \cdot 9} u^{-6}+\text { etc. }\right) \\
& \varphi W^{\prime \prime}-V^{\prime \prime}=\frac{4}{45} u^{-3}\left(1+\frac{3.4}{2.7} u^{-2}+\frac{\cdot 3 \cdot 4 \cdot 5 \cdot 6}{2 \cdot 4 \cdot 7 \cdot 9} u^{-4}+\frac{3.4 \cdot 5 \cdot 6 \cdot 7 \cdot 8}{2 \cdot 4 \cdot 6 \cdot 7 \cdot 9 \cdot 11} u^{-6}+\text { etc. }\right) \\
& ? W^{\prime \prime \prime}-V^{\prime \prime \prime}=\frac{4}{175} u^{-4}\left(1+\frac{4.5}{2.9} u^{-2}+\frac{4.5 .6 .7}{2.4 .9 .11} u^{-4}+\frac{4.5 \cdot 6 \cdot 7.8 .9}{2.4 \cdot 6.9 .11 .13} u^{-6}+\text { etc. }\right)
\end{aligned}
\]

etc. Hanc inductionem sequentes habebimus generaliter

\[
\begin{aligned}
U^{\prime \prime}= & \varphi W^{(n+1)}-V^{(n+1)} \text { aequalem producto ex } \\
& \frac{2 \cdot 2 \cdot 3 \cdot 3 \cdot 4 \cdot 4 \ldots \ldots(n+1) \cdot(n+1)}{3 \cdot 3 \cdot 5 \cdot 5 \cdot 7 \cdot 7 \cdot 9 \ldots(2 n+1)(2 n+3)} u^{-(n+2)}
\end{aligned}
\]

in seriem infinitam

\[
1+\frac{(n+2)(n+3)}{2(2 n+5)} u^{-2}+\frac{(n+2)(n+3)(n+4)(n+5)}{2 \cdot 4 \cdot(2 n+5)(2 n+7)} u^{-4}+\text { etc. }
\]

aut si magis placet in \(F\left(\frac{1}{2} n+1, \frac{1}{2} n+\frac{3}{2}, n+\frac{5}{2}, u^{-2}\right)\). Haec quoque inductio facillime ad plenam certitudinem evehitur vel per methodum vulgo notam vel adiumento formulae 19 in commentatione saepius citatae.

19.

Quum sufficiat, functionum \(T, U\) alterutram nosse, posterioris determinationem tamquam simpliciorem praetulimus. Quae quemadmodum evolutioni seriei \(u^{-1}+\frac{1}{3} u^{-3}+\frac{1}{5} u^{-5}+\) etc. in fractionem continuam innixa est, per ratiocinia similia ex evolutione seriei \(t^{-1}+\frac{1}{2} t^{-2}+\frac{1}{3} t^{-3}+\frac{1}{4} t^{-4}+\) etc. in fractionem continuam

\begin{center}
%\includegraphics[max width=\textwidth]{2024_01_11_75975a03bcf8b0416cd0g-165}
\end{center}

derivare potuissemus algorithmum ad determinandam functionem \(T\) pro valoribus successivis numeri \(n\). Ad eandem vero conclusionem pervenimus perpendendo, \(T\) nihil aliud esse quam \(\frac{U}{2^{n+1}}\) seu \(\frac{\left.W^{n+1}\right)}{2^{n+1}}\), si pro \(u\) scribitur \(2 t-1\), quo pacto functiones successive pro \(T\) adoptandae habebuntur per algorithmum sequentem:

\[
\begin{aligned}
W & =1 \\
\frac{1}{2} W^{\prime} & =t-\frac{1}{2} \\
\frac{1}{4} W^{\prime \prime} & =\left(t-\frac{1}{2}\right) \cdot \frac{1}{2} W^{\prime}-\frac{1 \cdot 1}{2 \cdot 6} W=t t-t+\frac{1}{6} \\
\frac{1}{8} W^{\prime \prime \prime} & =\left(t-\frac{1}{2}\right) \cdot \frac{1}{4} W^{\prime \prime}-\frac{2 \cdot 2}{6 \cdot 10} \cdot \frac{1}{2} W^{\prime}=t^{3}-\frac{3}{2} t t+\frac{3}{5} t-\frac{1}{20} \\
\frac{1}{10} W^{\prime \prime \prime \prime} & =\left(t-\frac{1}{2}\right) \cdot \frac{1}{8} W^{\prime \prime \prime}-\frac{3.3}{10 \cdot 14} \cdot \frac{1}{4} W^{\prime \prime}=t^{4}-2 t^{3}+\frac{9}{7} t t-\frac{2}{7} t+\frac{1}{70}
\end{aligned}
\]

etc. Per inductionem hinc resultat generaliter

\(T=t^{n+1}-\frac{(n+1)^{2}}{1 \cdot(2 n+2)} t^{n}+\frac{(n+1)^{2} \cdot n n}{1 \cdot 2 \cdot(2 n+2)(2 n+1)} t^{n-1}-\frac{(n+1)^{2} \cdot n n \cdot(n-1)^{2}}{1 \cdot 2 \cdot 3 \cdot(2 n+2)(2 n+1) \cdot 2 n} t^{n-2}+\) etc.

sive \(T=t^{n+1} F\left(-(n+1),-(n+1),-2(n+1), t^{-1}\right)\), cui inductioni facile est demonstrationis vim conciliare. Si magis arridet, \(T\) ordine terminorum inverso etiam per

\[
\pm \frac{1.2 \cdot 3 \cdot 4 \ldots(n+1)}{2 \cdot 6 \cdot 10 \cdot 14 \ldots .(4 n+2)} F(n+2,-(n+1), 1, t)
\]

exprimi potest, ubi signum superius valet pro \(n\) impari, inferius pro pari. Simili denique modo generaliter \(T^{\prime \prime}\) aequalis invenitur producto ex

\[
\frac{1 \cdot 1 \cdot 2 \cdot 2 \cdot 3 \cdot 3 \ldots \ldots(n+1) \cdot(n+1)}{2 \cdot 6 \cdot 6 \cdot 10 \cdot 10 \cdot 14 \ldots \cdot(4 n+2) \cdot(4 n+6)} t^{-(n+2)}
\]

in seriem infinitam

\[
1+\frac{(n+2)^{2}}{1 \cdot(2 n+4)} t^{-1}+\frac{(n+2)^{2}(n+3)^{2}}{1 \cdot 2 \cdot(2 n+4)(2 n+5)} t^{-2}+\frac{(n+2)^{2} \cdot(n+3)^{2}(n+4)^{2}}{1 \cdot 2 \cdot 3 \cdot(2 n+4)(2 n+5)(2 n+6)} t^{-3}+\text { etc. }
\]

sive in \(F\left(n+2, n+2,2 n+4, t^{-1}\right)\)

20.

Quum in functione \(U\) potestates \(u^{n}, u^{n-2}, u^{n-4}\) etc. absint, e radicibus aequationis \(U=0\) binae semper erunt magnitudine aequales signis oppositae, quibus pro valore pari ipsius \(n\) adhuc associare oportet radicem singularem 0 . Inventis radicibus, valores coëfficientium \(R, R^{\prime}, R^{\prime \prime}\) etc. secundum methodum art. 11 habebuntur per functionem integram ipsius \(u\), quae pro valore impari ipsius \(n\) erit formae

\[
\gamma u^{n-1}+\gamma^{\prime} u^{n-3}+\gamma^{\prime \prime} u^{n-5}+\text { etc. }
\]

pro valore pari autem, si excluditur coëfficiens radici \(u=0\) respondens, formae

\[
\gamma u^{n-2}+\gamma^{\prime} u^{n-4}+\gamma^{\prime \prime} u^{n-6}+\text { etc. }
\]

Exemplum art. 12 ipsam hanc reductionem exhibet pro \(n=6\). Manifesto igitur valoribus oppositis ipsius \(u\) semper respondent coëfficientes aequales. Ceterum in casu eo, ubi \(n\) est par, coëfficiens radici \(u=0\) respondens facile generaliter a priori assignari potest. Habebitur hic coëfficiens, si in \(\frac{U^{\prime}}{\left(\frac{d U}{d u}\right)}\) substituitur \(u=0\). Valorem numeratoris \(U^{\prime}\) pro \(u=0\) iam in art. 18 tradidimus, valor denominatoris autem ibinde erit

\[
= \pm \frac{3 \cdot 5 \cdot 7 \ldots . .(n+1)}{(n+3)(n+5) \ldots .(2 n+1)}= \pm \frac{3 \cdot 3 \cdot 5 \cdot 5 \cdot 7 \cdot 7 \ldots .(n+1)(n+1)}{3 \cdot 5 \cdot 7 \cdot 9 \cdot 11 \ldots .(2 n+1)}
\]

adeoque coëfficiens quaesitus

\[
=\left(\frac{2 \cdot 4 \cdot 6 \cdot 8 \ldots \ldots n}{3 \cdot 5 \cdot 7 \cdot 9 \ldots \cdots(n+1)}\right)^{2}
\]

21.

Functio integra ipsius \(u\) coëfficientes \(\boldsymbol{R}, \boldsymbol{R}^{\prime}, \boldsymbol{R}^{\prime \prime}\) etc. repraesentans in eo quem hic tractamus casu etiam independenter a methodo generali art. 11 erui potest sequenti modo. Differentiando aequationem

\[
\varphi-\frac{U^{\prime}}{U}=\frac{U^{\prime \prime}}{U}
\]

substituendo dein \(\frac{\mathrm{d} \varphi}{\mathrm{d} u}=\frac{1}{1-u u}\), ac multiplicando per \(U U(u u-1)\), obtinemus

\[
(u u-1) U^{\prime} \frac{\mathrm{d} U}{\mathrm{~d} u}-U\left(\frac{\mathrm{d} U^{\prime}}{\mathrm{d} u} \cdot(u u-1)+U\right)=(u u-1) U U \frac{\mathrm{d}\left(\frac{U^{\prime \prime}}{U}\right)}{\mathrm{d} u}
\]

Termini huius aequationis ad laevam manifesto constituunt functionem integram ipsius \(u\) : itaque necessario in parte ad dextram coëfficientes potestatum ipsius \(u\) cum exponentibus negativis sese destruere debent.

Sed \(\frac{\mathrm{d} \frac{V^{\prime \prime}}{U}}{\mathrm{~d} u}\) producit seriem infinitam incipientem a termino

\[
-\left(\frac{1.2 \cdot 3 \cdot 4 \ldots \ldots(n+1)}{1 \cdot 3 \cdot 5 \cdot 7 \ldots .(2 n+1)}\right)^{2} u^{-(2 n+4)}
\]

qua igitur per \((u u-1) U U\) multiplicata nihil aliud prodire poterit nisi quantitas constans

\[
-\left(\frac{1 \cdot 2 \cdot 3 \cdot 4 \ldots(n+1)}{1.3 \cdot 5 \cdot 7 \ldots(2 n+1)}\right)^{2}
\]

Hinc colligimus *)

\[
(u u-1) U^{\prime} \frac{\mathrm{d} U}{\mathrm{~d} u}+\left(\frac{1.2 \cdot 3 \cdot 4 \ldots .(n+1)}{1 \cdot 3 \cdot 5 \cdot 7 \ldots(2 n+1)}\right)^{2}
\]

divisibilem esse per \(U\), quamobrem functioni fractae \(\frac{U^{\prime}}{\left(\frac{\mathrm{d} U}{\mathrm{~d} u}\right)}\), quae coëfficientes \(\boldsymbol{R}, \boldsymbol{R}^{\prime}, \boldsymbol{R}^{\prime \prime}\) etc. suggerit, aequivalebit functio integra

\[
-\left(\frac{1 \cdot 3 \cdot 5 \cdot 7 \cdots \cdot(2 n+1)}{1 \cdot 2 \cdot 3 \cdot 4 \cdot \cdots(n+1)} U^{\prime}\right)^{2} \cdot(u u-1)
\]

Loco huius functionis, quae est ordinis \(2 n+2\), manifestoque solas potestates pares ipsius \(u\) implicat, adoptari poterit residuum ex eius divisione per \(U\) ortum, quod erit ordinis \(n\), seu \(n-1\), prout \(n\) par est seu impar. Si vero in casu priori coëfficientem eum, qui respondet radici \(u=0\), excludere malumus, loco illius functionis eius residuum ex divisione per \(\frac{U}{u}\) ortum adoptabimus, quod tantummodo ad ordinem \(n-2\) ascendet.

\section*{22.}
Ut praesto sint, quae ad applicationem methodi hucusque expositae requiruntur. adiungere visum est, pro valoribus successivis numeri \(n\), valores numericos tum quantitatum \(a, a^{\prime}, a^{\prime \prime}\) etc., tum coëfficientium \(\boldsymbol{R}, \boldsymbol{R}^{\prime} . \boldsymbol{R}^{\prime \prime}\) etc. ad sedecim figuras computatos, una cum horum logarithmis ad decem figuras.
\footnotetext{*) Simul hinc, petitur demonstratio, quod \(U\) cum \(\frac{\mathrm{d} U}{\mathrm{~d} u}\) divisorem indeterminatum communem habere nequit, neque adeo aequatio \(U=0\) radices aequales.
}

I. Terminus unus, \(n=0\).

\[
\begin{aligned}
& U=u, U^{\prime}=1, T=t-\frac{1}{2}, T^{\prime}=1 \\
& a=0,5 \\
& R=1
\end{aligned}
\]

Correctio formulae integralis proxime \(=\frac{1}{12} L^{\prime \prime}\).

II. Termini \(d u o, n=1\).

\[
\begin{aligned}
& U=u u-\frac{1}{2}, U^{\prime}=u \\
& T=t t-t+\frac{1}{6}, T^{\prime}=t-\frac{1}{2} \\
& a=0,2113248654051871 \\
& a^{\prime}=0,7886751345948129 \\
& R=R^{\prime}=\frac{1}{2}
\end{aligned}
\]

Correctio proxime \(=\frac{1}{180} L^{\prime \prime \prime}\)

III. Termini tres, \(n=2\).

\(U=u^{3}-\frac{3}{5} u, U^{\prime}=u u-\frac{4}{15}\)

\(T=t^{3}-\frac{3}{2} t t+\frac{3}{5} t-\frac{1}{20}, T^{\prime}=t t-t+\frac{11}{60}\)

\(a=0,1127016653792583\)

\(a^{\prime}=0,5\)

\(a^{\prime \prime}=\$, 8872983346 \quad 207417\)

\(R=\bar{R}^{\prime \prime}=\frac{5}{18}\)

\(R^{\prime}=\frac{4}{9}\)

Correctio proxime \(=\frac{1}{28_{00}} L^{\mathrm{vr}}\).

IV. Termini quatuor, \(n=3\).

\[
\begin{aligned}
& U=u^{4}-\frac{6}{7} u u+\frac{3}{3^{3}} \\
& U^{\prime}=u^{3}-\frac{1}{2} \frac{1}{1} u \\
& T=t^{4}-2 t^{3}+\frac{9}{7} t t-\frac{2}{7} t+\frac{1}{70} \\
& T^{\prime}=t^{3}-\frac{3}{2} t t+\frac{13}{2} t-\frac{5}{84} \\
& a=0,0694318442 \quad 029754 \\
& a^{\prime}=0,3300094782 \quad 075677 \\
& a^{\prime \prime}=0,6699905217924323 \\
& a^{\prime \prime \prime}=0,9305681557 \quad 970246
\end{aligned}
\]

\(R=R^{\prime \prime}=0,1739274225687284 \mathrm{log} .=9,2403680612\)

\(R^{\prime}=R^{\prime \prime}=0,3260725774312716 \log .=9,5133142764\)

Horum coëfficientium expressio generalis \(-\frac{35}{144} u u+\frac{17}{48}\)

Correctio proxime \(=\frac{1}{44100} L^{\mathrm{ymI}}\)

V. Termini quinque, \(n=4\).

\[
\begin{aligned}
& U=u^{5}-\frac{10}{9} u^{3}+\frac{5}{21} u \\
& U^{\prime}=u^{4}-\frac{7}{3} u u+\frac{64}{945} \\
& T=t^{5}-\frac{5}{2} t^{4}+\frac{20}{9} t^{3}-\frac{5}{6} t t+\frac{5}{42} t-\frac{1}{252} \\
& T^{\prime}=t^{4}-2 t^{3}+\frac{47}{36} t t-\frac{11}{36} t+\frac{137}{75600} \\
& a=0,0469100770306680 \\
& a^{\prime}=0,2307653449 \quad 471585 \\
& a^{\prime \prime}=0,5 \\
& a^{\prime \prime \prime}=0,7692346550 \quad 528415 \\
& a^{\prime \prime \prime \prime}=0,9530899229693320 \\
& \boldsymbol{R}=\boldsymbol{R}^{\prime \prime \prime}=0,1184634425280945 \text { log. }=9,0735843490 \\
& \boldsymbol{R}^{\prime}=\boldsymbol{R}^{\prime \prime \prime}=0,2393143352 \quad 496832 \quad 9,3789687142 \\
& R^{\prime \prime}=\frac{64}{22^{4}}=0,2844444444 \quad 444444 \quad 9,4539974559
\end{aligned}
\]

Expressio generalis horum coëfficientium, excluso \(\boldsymbol{R}^{\prime \prime}\),

Correctio proxime \(=\frac{1}{698544} L^{x}\)

\[
-\frac{91}{400} u u+\frac{1099}{3600}
\]

VI. Termini sex, \(n=5\).

\[
\begin{aligned}
U= & u^{6}-\frac{15}{1} \frac{5}{1} u^{4}+\frac{5}{11} u u-\frac{5}{231} \\
U^{\prime}= & u^{5}-\frac{3}{3} \frac{4}{3} u^{3}+\frac{1}{5} u \\
T= & t^{6}-3 t^{5}+\frac{75}{2} t^{4}-\frac{20}{11} t^{3}+\frac{5}{11} t t-\frac{1}{22} t+\frac{1}{924} \\
T^{\prime}= & t^{5}-\frac{5}{2} t^{4}+\frac{7}{33} t^{3}-\frac{19}{22} t t+\frac{29}{220} t-\frac{7}{1320} \\
& \quad a=0,0337652428 \quad 984240 \\
& a^{\prime}=0,1693953067 \quad 668678 \\
& a^{\prime \prime}=0,3806904069 \quad 584015 \\
& a^{\prime \prime \prime}=0,6193095930 \quad 415985 \\
& a^{\prime \prime \prime \prime}=0,8306046932 \quad 331322 \\
& a^{\prime \prime \prime \prime \prime}=0,9662347571 \quad 015760 \\
R= & R^{\prime \prime \prime \prime \prime}==0,0856622461 \quad 895852 \log .=8,9327894580 \\
R^{\prime}= & R^{\prime \prime \prime \prime}=0,1803807865 \quad 240693 \\
R^{\prime \prime}= & R^{\prime \prime \prime}=0,2339569672863455
\end{aligned}
\]

Coëfficientium expressio generalis

\[
-\frac{77}{80} u^{4}-\frac{7}{75} u u+\frac{23}{36}
\]

Correctio proxime \(=\frac{1}{1109988} L^{\mathrm{XII}}\)

VII. Termini septem, \(n=6\).

\[
\begin{aligned}
& U=u^{7}-\frac{2}{13} u^{5}+\frac{10 \times}{14 \frac{x}{3}} u^{3}-\frac{3}{42} \frac{5}{9} u \\
& U^{\prime}=u^{6}-\frac{\times}{3} 8 u^{4}+\frac{28}{713} u u-\frac{286}{15 \frac{8}{15}} \\
& T=t^{7}-\frac{7}{2} t^{6}+\frac{63}{13} t^{5}-\frac{17}{5} 2^{4} t^{4}+\frac{17}{143} t^{3}-\frac{63}{286} t t+\frac{7}{499} t-\frac{1}{3 \frac{1}{32}} \\
& T^{\prime}=t^{6}-3 t^{5}+\frac{5}{1} \frac{35}{56} t^{4}-\frac{1}{5} \frac{1}{8} t^{3}+\frac{1}{2} \frac{3}{8} \frac{7}{60} t t-\frac{22}{2} 9^{3} \frac{3}{8} t+\frac{323}{240 \frac{3}{2} 40} \\
& a=0,0254460438286202 \\
& a^{\prime}=0,1292344072 \quad 003028 \\
& a^{\prime \prime}=0,2970774243 \quad 113015 \\
& a^{\prime \prime \prime}=0,5 \\
& a^{\prime \prime \prime \prime}=0,7029225756 \quad 886985 \\
& a^{\prime \prime \prime \prime \prime}=0,8707655927 \quad 996972 \\
& a^{\prime \prime \prime \prime \prime \prime}=0,9745539561 \quad 713798 \\
& R=R^{\prime \prime \prime \prime}=0,0647424830844348 \text { log. }=8,8111893529 \\
& R^{\prime}=R^{\prime \prime \prime \prime}=0,1398526957446384 \quad \cdot 9,1456708421 \\
& R^{\prime \prime}=R^{\prime \prime \prime}=0,1909150252525595 \quad 9,2808401093 \\
& R^{\prime \prime \prime}=\frac{x_{1}^{256}}{2 \frac{5}{25}}=0,2089795918 \quad 367347 . \quad 9,3201038766
\end{aligned}
\]

Horum coëfficientium, \(R^{\prime \prime \prime}\) excluso, expressio generalis

Correctio proxime \(={ }_{176679360}^{1} L^{\mathrm{xVV}}\)

\begin{center}
%\includegraphics[max width=\textwidth]{2024_01_11_75975a03bcf8b0416cd0g-170}
\end{center}

\section*{23.}
Coronidis loco methodi nostrae efficaciam ob oculos ponemus computando valorem integralis

\[
\int \frac{d x}{\log x}
\]

ab \(x=100000\) usque ad \(x=200000\).

I. Ex termino uno habemus \(\Delta \boldsymbol{R} \boldsymbol{A}=8390,394608\)

II. Ex terminis duobus fit..\(\left\{\begin{array}{l}\Delta R A=4271,810097 \\ \Delta R^{\prime} A^{\prime}=4134,144502 \\ \hline \text { Summa }=8405,954599\end{array}\right.\).

III. Ex terminis tribus .... \(\left\{\begin{array}{l}\Delta R A=2390.572772 \\ \Delta R^{\prime} A^{\prime}=3729,064270 \\ \frac{\Delta R^{\prime \prime} A^{\prime \prime}=2286,599733}{\text { Summa }=8406,236775}\end{array}\right.\)

IV. Ex terminis quatuor \(\ldots \ldots\left\{\begin{array}{l}\Delta R A=1501,957053 \\ \Delta R^{\prime} A^{\prime}=2763,769240 \\ \Delta R^{\prime \prime} A^{\prime \prime}=2711,454637 \\ \Delta R^{\prime \prime} A^{\prime \prime \prime}=1429,062040 \\ \text { Summa }=8406,242970\end{array}\right.\)

V. Ex terminis quinque \(\ldots . .\left\{\begin{array}{l}\Delta R A=1024,879445 \\ \Delta R^{\prime} A^{\prime}=2041,833335 \\ \Delta R^{\prime \prime} A^{\prime \prime}=2386,601133 \\ \Delta R^{\prime \prime \prime} A^{\prime \prime \prime}=1980,509616 \\ \Delta R^{\prime \prime \prime} A^{\prime \prime \prime \prime}=972,419588 \\ \hline \text { Summa }=8406,243117\end{array}\right.\)

\begin{center}
%\includegraphics[max width=\textwidth]{2024_01_11_75975a03bcf8b0416cd0g-171}
\end{center}

VII. Ex terminis septem.... \(\left\{\begin{array}{l}\Delta R A=561,1213804 \\ \Delta R^{\prime} A^{\prime}=1202,0551998 \\ \Delta R^{\prime \prime} A^{\prime \prime}=1621,6290819 \\ \Delta R^{\prime \prime \prime} A^{\prime \prime \prime}=1753,4212406 \\ \Delta R^{\prime \prime \prime} A^{\prime \prime \prime \prime}=1584,9790252 \\ \Delta R^{\mathrm{v}} A^{\mathrm{v}}=1152,0681116 \\ \Delta R^{\mathrm{vi}} A^{\mathrm{vi}}=530,9690816 \\ \hline \text { Summa }=8406,2431211\end{array}\right.\)

\(\mathrm{E}\) calculis clar. BesseL valor eiusdem integralis inventus est \(=8406,24312\).

\section*{331}
\section*{DETERMINATIO ATTRACTIONIS}
QUAM IN PUNCTUM QUODVIS POSITIONIS DATAE

\section*{EXERCERET PLANETA SI EIUS MASSA}
PER TOTAM ORBITAM

\section*{RATIONE TEMPORIS QUO SINGULAE PARTES DESCRIBUNTUR}
UNIFORMITER ESSET DISPERTITA

A U C T O R E

CAROLO FRIDERICO GAUSS

SOCIETATI REGIAE SCIENTIARUM EXHIBITA 1818. IAN. 17.

Commentationes societatis regiae scientiarum Gottingensis recentiores. Vol. Iv.

Gottingae MDcccxviII.

\section*{DETERMINATIO ATTRACTIONIS QUAM IN PUNCTUM QUODVIS POSITIONIS DATAE EXERCERET PLANETA SIENUSMASSA PER TOTAM ORBITAM RATIONE TEMPORIS QUO SINGULAE PARTES DESCRIBUNTUR UNIFORMITER ESSET DISPERTTTA.}
\section*{1.}
Variationes saeculares, quas elementa orbitae planetariae a perturbatione alius planetae patiuntur, ab huius positione in orbita sunt independentes, atque eaedem forent, sive planeta perturbans in orbita elliptica secundum KEPLERI leges incedat, sive ipsius massa per orbitam eatenus aequabiliter dispertita concipiatur, ut orbitae partibus, alias aequali temporis intervallo descriptis, iam aequales massae partes tribuantur, siquidem tempora revolutionum planetae perturbati et perturbantis non sint commensurabilia. Theorema hoc elegans, si. a nemine hucusque disertis verbis propositum est, saltem perfacile ex astronomiae physicae principiis demonstratur. Problema itaque se offert tum per se, tum propter plura artificia, quae eius solutio requirit, attentione perdignum: attractionem orbitae planetariae, aut si mavis, annuli elliptici, cuius crassities infinite parva, atque secundum legem modo explicatam variabilis, in punctum quodlibet positione datum exacte determinare.

\section*{2.}
Denotando excentricitatem orbitae per \(e\), atque puncti cuiusvis in ipsa anomaliam excentricam per \(E\), huius elemento d \(E\) respondebit elementum anomaliae mediae \((1-e \cos E) \mathrm{d} E\); quamobrem elementum massae ei orbitae portiunculae, cui respondent illa elementa, tribuendum, erit ad massam integram, quam pro unitate accipiemus, ut \((1-e \overline{\cos } \boldsymbol{E}) \mathrm{d} \boldsymbol{E}\) ad \(2 \pi\), exprimente \(\pi\) semicircumfe-
rentiam circuli pro radio 1. Statuendo itaque distantiam puncti attracti a puncto orbitae \(=\rho\), attractio ab orbitae elemento producta erit

\[
=\frac{(1-e \cos E) \mathrm{d} E}{2 \pi \rho \rho}
\]

Designabimus semiaxem maiorem per \(a\), semiaxem minorem per \(b\), atque illum tamquam lineam abscissarum, centrumque ellipsis tamquam initium adoptabimus. Hinc erit \(a a-b b=a a e e\), abscissa puncti orbitae \(=a \cos E\), ordinata \(=b \sin E\). Denique distantiam puncti attracti a plano orbitae denotabimus per \(C\), atque coordinatas reliquas axi maiori et minori parallelas per \(A\) et \(B\). His ita praeparatis, attractio elementi orbitae decomponetur in duas axi maiori et minori parallelas atque tertiam plano orbitae normalem, puta

\[
\begin{aligned}
\frac{(A-a \cos E)(1-e \cos E) \mathrm{d} E}{2 \pi \rho^{3}} & =\mathrm{d} \xi \\
\frac{(B-h \sin E)(1-e \cos E) \mathrm{d} E}{2 \pi \rho^{3}} & =\mathrm{d} \eta \\
\frac{C(1-e \cos E) \mathrm{d} E}{2 \pi \rho^{3}} & =\mathrm{d} \zeta
\end{aligned}
\]

ubi \(\rho=\sqrt{ }\left((A-a \cos E)^{2}+(B-b \sin E)^{2}+C C\right)\).

Integratis hisce differentialibus ab \(E=0\) usque ad \(E=360^{\circ}\), prodibunt attractiones partiales \(\xi, \eta, \zeta\) secundum directiones, directionibus coordinatarum oppositas, e quibus attractio integra composita erit. et quas per methodum notam ad quaslibet alias directiones referre licebit.

3.

Rei summa iam in eo versatur, ut introducta loco ipsius \(E\) alia variabili, quantitas radicalis in formam simpliciorem redigatur. Ad hunc finem statuemus

\[
\cos E=\frac{a+\alpha^{\prime} \cos T+a^{\prime \prime} \sin T}{\gamma+\gamma^{\prime} \cos T+\gamma^{\prime \prime} \sin T}, \quad \sin E=\frac{b+b^{\prime} \cos T+b^{\prime \prime} \sin T}{\gamma+\gamma^{\prime} \cos T+\gamma^{\prime \prime} \sin T}
\]

ubi autem novem coëfficientes \(\alpha, \alpha^{\prime}, \alpha^{\prime \prime}\) etc. manifesto non sunt penitus arbitrarii, sed certis conditionibus satisfacere debent, quas ante omnia perscrutari oportet. Primo observamus, substitutionem eandem manere, si omnes coëfficientes per eundem factorem multiplicentur, ita ut absque generalitatis detrimento uni ex ipsis valorem determinatum tribuere, e. g. statuere liceret \(\gamma=1\) : attamen concinnitatis caussa omnes novem aliquantisper indefiniti maneant. Porro monemus, ex-
cludi debere valores tales, ubi \(\alpha, \alpha^{\prime}, \alpha^{\prime \prime}\) vel \(b, b^{\prime}, b^{\prime \prime}\) ipsis \(\gamma, \gamma^{\prime}, \gamma^{\prime \prime}\) resp. proportionales essent: alioquin enim \(E\) haud amplius indeterminata maneret. Nequeunt igitur \(\gamma^{\prime} \alpha^{\prime \prime}-\gamma^{\prime \prime} \alpha^{\prime}, \gamma^{\prime \prime} \alpha-\gamma \alpha^{\prime \prime}, \gamma \alpha^{\prime}-\gamma^{\prime} \alpha\) simul evanescere.

Manifesto coëfficientes \(\alpha, \alpha^{\prime}, \alpha^{\prime \prime}\) etc. ita comparati esse debent, ut fiat indefinite

\[
\left.\begin{array}{r}
\left(\alpha+\alpha^{\prime} \cos T+\alpha^{\prime \prime} \sin T\right)^{2} \\
+\left(b+b^{\prime} \cos T+b^{\prime \prime} \sin T\right)^{2} \\
-\left(\gamma+\gamma^{\prime} \cos T+\gamma^{\prime \prime} \sin T\right)^{2}
\end{array}\right\}=0
\]

unde necessario haec functio habere debet formam

\[
k\left(\cos T^{2}+\sin T^{2}-1\right)
\]

Hinc colligimus sex aequationes conditionales

\[
\left.\begin{array}{lr}
-\alpha \alpha-b b+\gamma \gamma= & k \\
-\alpha^{\prime} \alpha^{\prime}-b^{\prime} b^{\prime}+\gamma^{\prime} \gamma^{\prime}= & -k \\
-\alpha^{\prime \prime} \alpha^{\prime \prime}-b^{\prime \prime} b^{\prime \prime}+\gamma^{\prime \prime} \gamma^{\prime \prime}= & -k \\
-\alpha^{\prime} \alpha^{\prime \prime}-b^{\prime} b^{\prime \prime}+\gamma^{\prime} \gamma^{\prime \prime}= & 0 \\
-\alpha^{\prime \prime} \alpha-b^{\prime \prime} b+\gamma^{\prime \prime} \gamma= & 0 \\
-\alpha \alpha^{\prime}-b b^{\prime}+\gamma \gamma^{\prime}= & 0
\end{array}\right\}
\]

\(\mathrm{Ab}\) his aequationibus pendent plures aliae, quas evolvere operae pretium erit. Statuendo brevitatis caussa

\[
\alpha b^{\prime} \gamma^{\prime \prime}+\alpha^{\prime} b^{\prime \prime} \gamma+\alpha^{\prime \prime} b \gamma^{\prime}-\alpha b^{\prime \prime} \gamma^{\prime}-\alpha^{\prime} b \gamma^{\prime \prime}-\alpha^{\prime \prime} b^{\prime} \gamma=\varepsilon
\]

e combinatione aequationum (I) facile derivantur novem sequentes:

\[
\left.\begin{array}{l}
\varepsilon \alpha=-k\left(b^{\prime} \gamma^{\prime \prime}-\gamma^{\prime} b^{\prime \prime}\right) \\
\varepsilon b=-k\left(\gamma^{\prime} \alpha^{\prime \prime}-\alpha^{\prime} \gamma^{\prime \prime}\right) \\
\varepsilon \gamma=+k\left(\alpha^{\prime} b^{\prime \prime}-b^{\prime} \alpha^{\prime \prime}\right) \\
\varepsilon \alpha^{\prime}=+k\left(b^{\prime \prime} \gamma-\gamma^{\prime \prime} b\right) \\
\varepsilon b^{\prime}=+k\left(\gamma^{\prime \prime} \alpha-\alpha^{\prime \prime} \gamma\right) \\
\varepsilon \gamma^{\prime}=-k\left(\alpha^{\prime \prime} b-b^{\prime \prime} \alpha\right) \\
\varepsilon \alpha^{\prime \prime}=+k\left(b \gamma^{\prime}-b^{\prime}\right) \\
\varepsilon b^{\prime \prime}=+k\left(\gamma \alpha^{\prime}-\alpha \gamma^{\prime}\right) \\
\varepsilon \gamma^{\prime \prime}=-k\left(\alpha b^{\prime}-b \alpha^{\prime}\right)
\end{array}\right\}
\]

E tribus primis harum aequationum rursus deducimus hanc:

\[
\begin{aligned}
& \varepsilon \alpha\left(b^{\prime} \gamma^{\prime \prime}-\gamma^{\prime} b^{\prime \prime}\right)+\varepsilon b\left(\gamma^{\prime} \alpha^{\prime \prime}-\alpha^{\prime} \gamma^{\prime \prime}\right)+\varepsilon \gamma\left(\alpha^{\prime} b^{\prime \prime}-b^{\prime} \alpha^{\prime \prime}\right) \\
&=-k\left(b^{\prime} \gamma^{\prime \prime}-\gamma^{\prime} b^{\prime \prime}\right)^{2}-k\left(\gamma^{\prime} \alpha^{\prime \prime}-\alpha^{\prime} \gamma^{\prime \prime}\right)^{2}+k\left(\alpha^{\prime} b^{\prime \prime}-b^{\prime} \alpha^{\prime \prime}\right)^{2}
\end{aligned}
\]

cui aequivalens est haee:

\[
\varepsilon \varepsilon=k\left(-\alpha^{\prime} \alpha^{\prime}-b^{\prime} b^{\prime}+\gamma^{\prime} \gamma^{\prime}\right)\left(-\alpha^{\prime \prime} \alpha^{\prime \prime}-b^{\prime \prime} b^{\prime \prime}+\gamma^{\prime \prime} \gamma^{\prime \prime}\right)-k\left(-\alpha^{\prime} \alpha^{\prime \prime}-b^{\prime} b^{\prime \prime}+\gamma^{\prime} \gamma^{\prime \prime}\right)^{2}
\]

quae adiumento aequationum \(2,3,4\) in (I) mutatur in hanc:

\[
\varepsilon \varepsilon=k^{3} \ldots \ldots \ldots(\mathrm{IV})
\]

Aeque facile ex aequationibus (I) derivantur hae:

\[
\left.\begin{array}{l}
\left(b^{\prime} \gamma^{\prime \prime}-\gamma^{\prime} b^{\prime \prime}\right)^{2}=-k\left(k-\alpha^{\prime} \alpha^{\prime}-\alpha^{\prime \prime} \alpha^{\prime \prime}\right) \\
\left(\gamma^{\prime} \alpha^{\prime \prime}-\alpha^{\prime} \gamma^{\prime \prime}\right)^{2}=-k\left(k-b^{\prime} b^{\prime}-b^{\prime \prime} b^{\prime \prime}\right) \\
\left(\alpha^{\prime} b^{\prime \prime}-b^{\prime} \alpha^{\prime \prime}\right)^{2}=+k\left(k+\gamma^{\prime} \gamma^{\prime}+\gamma^{\prime \prime} \gamma^{\prime \prime}\right) \\
\left(b^{\prime \prime} \gamma-\gamma^{\prime \prime} b\right)^{2}=+k\left(k+\alpha \alpha-\alpha^{\prime \prime} \alpha^{\prime \prime}\right) \\
\left(\gamma^{\prime \prime} \alpha-\alpha^{\prime \prime} \gamma\right)^{2}=+k\left(k+b b-b^{\prime \prime} b^{\prime \prime}\right) \\
\left(\alpha^{\prime \prime} b-b^{\prime \prime} \alpha\right)^{2}=-k\left(k-\gamma \gamma+\gamma^{\prime \prime} \gamma^{\prime \prime}\right) \\
\left(b \gamma^{\prime}-\gamma b^{\prime}\right)^{2}=+k\left(k+\alpha \alpha-\alpha^{\prime} \alpha^{\prime}\right) \\
\left(\gamma \alpha^{\prime}-\alpha \gamma^{\prime}\right)^{2}=+k\left(k+b b-b^{\prime} b^{\prime}\right) \\
\left(\alpha b^{\prime}-b \alpha^{\prime}\right)^{2}=-k\left(k-\gamma \gamma+\gamma^{\prime} \gamma^{\prime}\right)
\end{array}\right\}
\]

Exempli caussa evolutionem primae adscribimus, ad cuius instar reliquae facile formabuntur. Aequationes 4, 2, 3 in (I) scilicet suppeditant

\[
\left(\gamma^{\prime} \gamma^{\prime \prime}-b^{\prime} b^{\prime \prime}\right)^{2}-\left(\gamma^{\prime} \gamma^{\prime}-b^{\prime} b^{\prime}\right)\left(\gamma^{\prime \prime} \gamma^{\prime \prime}-b^{\prime \prime} b^{\prime \prime}\right)=\alpha^{\prime} \alpha^{\prime} \alpha^{\prime \prime} \alpha^{\prime \prime}-\left(\alpha^{\prime} \alpha^{\prime}-k\right)\left(\alpha^{\prime \prime} \alpha^{\prime \prime}-k\right)
\]

quae aequatio evoluta protinus ipsam primam in (V) sistit.

Ex his aequationibus (V) concludimus, valorem \(k=0\) in disquisitione nostra haud admissibilem esse; hinc enim omnes novem quantitates \(b^{\prime} \gamma^{\prime \prime}-\gamma^{\prime} b^{\prime \prime}\) etc. necessario evanescerent, i. e. coëfficientes \(\alpha, \alpha^{\prime}, \alpha^{\prime \prime}\) tum ipsis \(b, b^{\prime}, b^{\prime \prime}\), tum ipsis \(\gamma, \gamma^{\prime}, \gamma^{\prime \prime}\) proportionales evaderent. Hinc etiam, propter aequationem IV, quantitas \(\varepsilon\) evanescere nequit; quamobrem \(k\) necessario debet esse quantitas positiva, siquidem omnes coëfficientes \(\alpha, \alpha^{\prime}, \alpha^{\prime \prime}\) etc. debent esse reales. Combinatis tribus aequationibus primis in (III) cum tribus primis in (V), hae novae prodeunt, quae manifesto a valore ipsius \(k\) non evanescente pendent:

\[
\left.\begin{array}{l}
\alpha \alpha-\alpha^{\prime} \alpha^{\prime}-\alpha^{\prime \prime} \alpha^{\prime \prime}=-k \\
b b-b^{\prime} b^{\prime}-b^{\prime \prime} b^{\prime \prime}=-k \\
\gamma \gamma-\gamma^{\prime} \gamma^{\prime}-\gamma^{\prime \prime} \gamma^{\prime \prime}=+k
\end{array}\right\}
\]

Combinatio reliquarum easdem produceret. His denique adiungimus tres sequentes:

\[
\left.\begin{array}{l}
b \gamma-b^{\prime} \gamma^{\prime}-b^{\prime \prime} \gamma^{\prime \prime}=0 \\
\gamma \alpha-\gamma^{\prime} \alpha^{\prime}-\gamma^{\prime \prime} \alpha^{\prime \prime}=0 \\
\alpha b-\alpha^{\prime} b^{\prime}-\alpha^{\prime \prime} b^{\prime \prime}=0
\end{array}\right\}
\]

quae facile ex aequationibus III derivantur; e. g. secunda, quinta et octava suppeditant:

\[
\varepsilon b \gamma-\varepsilon b^{\prime} \gamma^{\prime}-\varepsilon b^{\prime \prime} \gamma^{\prime \prime}=-k \gamma\left(\gamma^{\prime} \alpha^{\prime \prime}-\alpha^{\prime} \gamma^{\prime \prime}\right)-k \gamma^{\prime}\left(\gamma^{\prime \prime} \alpha-\alpha^{\prime \prime} \gamma\right)-k \gamma^{\prime \prime}\left(\gamma \alpha^{\prime}-\alpha \gamma^{\prime}\right)=0
\]

Manifesto hae quoque aequationes ab exclusione valoris \(k=0\) sunt dependentes \({ }^{*}\). Quoniam, ut iam supra monuimus, omnes coëfficientes \(\alpha, \alpha^{\prime}, \alpha^{\prime \prime}\) etc. per eundem factorem multiplicare licet, unde valor ipsius \(k\) per quadratum eiusdem factoris multiplicatus prodibit, abhinc semper supponemus

\[
k=1
\]

quo pacto necessario quoque erit vel \(\varepsilon=+1\) vel \(\varepsilon=-1\). Patet itaque, novem coëfficientes \(\alpha, \alpha^{\prime}, \alpha^{\prime \prime}\) etc., inter quos sex aequationes conditionales adsunt, ad tres quantitates ab invicem independentes reducibiles esse debere, quod quidem commodissime per tres angulos sequenti modo efficitur:

\[
\begin{aligned}
& \alpha=\cos L \operatorname{tang} N \\
& b=\sin L \operatorname{tang} N \\
& \gamma=\sec N \\
& \alpha^{\prime}=\cos L \cos M \sec N \pm \sin L \sin M \\
& b^{\prime}=\sin L \cos M \sec N \mp \cos L \sin M \\
& \gamma^{\prime}=\cos M \operatorname{tang} N \\
& \alpha^{\prime \prime}=\cos L \sin M \sec N \mp \sin L \cos M \\
& b^{\prime \prime}=\sin L \sin M \sec N \pm \cos L \cos M \\
& \gamma^{\prime \prime}=\sin M \operatorname{tang} N
\end{aligned}
\]

*) Forsan haud superfluum erit monere, nos analysin praecedentem consulto elegisse atque alii derivationi relationum III-VII praetulisse, quae quamquam aliquantulum elegantior videretur, tamen, accurate examinata, quibusdam dubiis obnoxia inventa est, quae non sine ambagibus removere licuisset.
ubi signorum ambiguorum superiora referuntur ad casum \(\varepsilon=+1\), inferiora ad casum \(\varepsilon=-1\). Attamen tractatio analytica ad maximam partem elegantius sine usu horum angulorum absolvitur. Ceterum haud difficile foret, significationem geometricam tum horum angulorum, tum reliquarum quantitatum auxiliarium in hac disquisitione occurrentium assignare; hanc vero interpretationem ad institutum nostrum haud necessariam lectori perito explicandam linquimus.

4.

Si iam in expressione distantiae \(\rho\) pro \(\cos E\) et \(\sin E\) valores supra assumti substituuntur, illa in hanc formam transibit:

\[
\rho=\frac{\sqrt{ }\left(G+G^{\prime} \cos T^{2}+G^{\prime \prime} \sin T^{2}+2 H \cos T \sin T+2 H^{\prime} \sin T+2 H^{\prime \prime} \cos T\right)}{\gamma+\gamma^{\prime} \cos T^{\prime}+\gamma^{\prime \prime} \sin T}
\]

ubi coëfficientes \(\alpha, \alpha^{\prime}, \alpha^{\prime \prime}\) etc. ita determinabimus, ut salvis sex aequationibus conditionalibus

\[
\left.\begin{array}{lr}
-\alpha \alpha-b b+\gamma \gamma= & 1 \\
-\alpha^{\prime} \alpha^{\prime}-b^{\prime} b^{\prime}+\gamma^{\prime} \gamma^{\prime}= & -1 \\
-\alpha^{\prime \prime} \alpha^{\prime \prime}-b^{\prime \prime} b^{\prime \prime}+\gamma^{\prime \prime} \gamma^{\prime \prime}= & -1 \\
-\alpha^{\prime} \alpha^{\prime \prime}-b^{\prime} b^{\prime \prime}+\gamma^{\prime} \gamma^{\prime \prime}= & 0 \\
-\alpha^{\prime \prime} \alpha-b^{\prime \prime} b+\gamma^{\prime \prime} \gamma= & 0 \\
-\alpha \alpha^{\prime}-b b^{\prime}+\gamma^{\prime}= & 0
\end{array}\right\}
\]

adeoque etiam reliquis inde demanantibus, fiat

\[
H=0, \quad H^{\prime}=0, \quad H^{\prime \prime}=0
\]

quo pacto problema generaliter loquendo erit determinatum. Quodsi itaque denominatorem ipsius \(\rho\) per \(t\) denotamus, transire debet functio trium quantitatum \(t, t \cos E, t \sin E\) haec

\((A A+B B+C C) t t+a a(t \cos E)^{2}+b b(t \sin E)^{2}-2 a A t \cdot t \cos E-2 b B t . t \sin E\) per substitutionem

\[
\begin{aligned}
t \cos E & =\alpha+\alpha^{\prime} \cos T+\alpha^{\prime \prime} \sin T \\
t \sin E & =b+b^{\prime} \cos T+b^{\prime \prime} \sin T \\
t & =\gamma+\gamma^{\prime} \cos T+\gamma^{\prime \prime} \sin T
\end{aligned}
\]

in

\[
G+G^{\prime} \cos T^{2}+G^{\prime \prime} \sin T^{2}
\]

Manifesto hoc idem est, ac si dicas, functionem trium indeterminatarum \(x, y, z\) hanc \((W)\)

\[
a a x x+b b y y+(A A+B B+C C) z z-2 a A x z-2 b B y z
\]

per substitutionem

\[
\begin{aligned}
& x=\alpha u+\alpha^{\prime} u^{\prime}+\alpha^{\prime \prime} u^{\prime \prime} \\
& y=b u+b^{\prime} u^{\prime}+b^{\prime \prime} u^{\prime \prime} \\
& z=\gamma u+\gamma^{\prime} u^{\prime}+\gamma^{\prime \prime} u^{\prime \prime}
\end{aligned}
\]

in functionem indeterminatarum \(u, u^{\prime}, u^{\prime \prime}\) hanc

\[
G u u+G^{\prime} u^{\prime} u^{\prime}+G^{\prime \prime} u^{\prime \prime} u^{\prime \prime}
\]

transire debere. At quum ex his formulis, adiumento aequationum [1], facile sequatur

\[
\begin{aligned}
& u=-\alpha x-b y+\gamma z \\
& u^{\prime}=\alpha^{\prime} x+b^{\prime} y-\gamma^{\prime} z \\
& u^{\prime \prime}=\alpha^{\prime \prime} x+b^{\prime \prime} y-\gamma^{\prime \prime} z
\end{aligned}
\]

manifesto functio \(W\) identica esse debebit cum hac

\[
G(-\alpha x-b y+\gamma z)^{2}+G^{\prime}\left(\alpha^{\prime} x+b^{\prime} y-\gamma^{\prime} z\right)^{2}+G^{\prime \prime}\left(\alpha^{\prime \prime} x+b^{\prime \prime} y-\gamma^{\prime \prime} z\right)^{2}
\]

unde habemus sex aequationes

\[
\left.\begin{array}{rl}
a a & =G \alpha \alpha+G^{\prime} \alpha^{\prime} \alpha^{\prime}+G^{\prime \prime} \alpha^{\prime \prime} \alpha^{\prime \prime} \\
b b & =G b b+G^{\prime} b^{\prime} b^{\prime}+G^{\prime \prime} b^{\prime \prime} b^{\prime \prime} \\
A A+B B+C C & =G \gamma \gamma+G^{\prime} \gamma^{\prime} \gamma^{\prime}+G^{\prime \prime} \gamma^{\prime \prime} \gamma^{\prime \prime} \\
b B & =G b \gamma+G^{\prime} b^{\prime} \gamma^{\prime}+G^{\prime \prime} b^{\prime \prime} \gamma^{\prime \prime} \\
a A & =G \gamma \alpha+G^{\prime} \gamma^{\prime} \alpha^{\prime}+G^{\prime \prime} \gamma^{\prime \prime} \alpha^{\prime \prime} \\
0 & =G \alpha b+G^{\prime} \alpha^{\prime} b^{\prime}+G^{\prime \prime} \alpha^{\prime \prime} b^{\prime \prime}
\end{array}\right\}
\]

Ex his duodecim aequationibus [1] et [2] incognitas nostras \(G, G^{\prime}, G^{\prime \prime}, \alpha, \alpha^{\prime}, \alpha^{\prime \prime}\) etc., determinare oportebit.

5.

E combinatione aequationum [1] et [2] facile derivantur sequentes:

\[
\begin{aligned}
& -\alpha a a+\gamma a A=\alpha G \\
& -b b b+\gamma b B=b G \\
& \gamma(A A+B B+C C)-\alpha a A-b b B=\gamma G
\end{aligned}
\]

unde fit porro

\[
\begin{aligned}
& \alpha=\frac{\gamma a A}{a a+G} \ldots \ldots \ldots \ldots . \ldots[\ldots] \\
& \left.b=\frac{\gamma b B}{b b+G} \ldots \ldots \ldots . \ldots\right] \\
& A A+B B+C C-\frac{a a A A}{a a+G}-\frac{b b B}{b b+G}=G
\end{aligned}
\]

Ultimam sic quoque exhibere possumus

\[
\frac{A A}{a a+G}+\frac{B B}{b b+G}+\frac{C C}{G}=1
\]

Perinde e combinatione aequationum [1] et [2] deducimus

\[
\begin{aligned}
& \alpha^{\prime} a a-\gamma^{\prime} a A=\alpha^{\prime} G^{\prime} \\
& b^{\prime} b b-\gamma^{\prime} b B=b^{\prime} G^{\prime} \\
& -\gamma^{\prime}(A A+B B+C C)+\alpha^{\prime} a A+b^{\prime} b B=\gamma^{\prime} G^{\prime}
\end{aligned}
\]

atque hinc

\[
\begin{aligned}
& \alpha^{\prime}=\frac{\gamma^{\prime} a A}{a a-\overline{G^{\prime}}}
\end{aligned}
\]

\begin{center}
%\includegraphics[max width=\textwidth]{2024_01_11_75975a03bcf8b0416cd0g-180}
\end{center}

\[
\begin{aligned}
& \frac{A A}{a a-G^{\prime}}+\frac{B B}{b b-G^{\prime}}-\frac{C C}{G^{\prime}}=1 \ldots[8]
\end{aligned}
\]

et prorsus simili modo

\[
\begin{aligned}
& \alpha^{\prime \prime}=\frac{\gamma^{\prime \prime} a A}{a a-G^{\prime \prime}} \ldots \ldots \ldots \\
& b^{\prime \prime}=\frac{\gamma^{\prime \prime} b B}{b b-G^{\prime \prime}} \ldots \ldots \ldots \ldots \\
& \frac{A A}{a a-G^{\prime \prime}}+\frac{B B}{b b-G^{\prime \prime}}-\frac{C C}{G^{\prime \prime}}=1
\end{aligned}
\]

Patet itaque, \(G,-G^{\prime},-G^{\prime \prime}\) esse radices aequationis

\[
\frac{A A}{a a+x}+\frac{B B}{b b+x}+\frac{C C}{x}=1
\]

QUAM IN PUNCTUM quODVIS POSITIONIS DATAE EXERCERET PLANETA ETC.

quae rite evoluta ita se habet

\[
\begin{gathered}
x^{3}-(A A+B B+C C-a a-b b) x x+(a a b b-a a B B-a a C C-b b A A-b b C C) x \\
-a a b b C C=0 \ldots \ldots \ldots \ldots[13]
\end{gathered}
\]

6.

Iam de indole huius aequationis cubicae sequentia sunt notanda.

I. Ex aequationis termino ultimo - \(a a b b C C\) concluditur, eam certe habere radicem unam realem, et quidem vel positivam, vel, si \(C=0\), cifrae aequalem. Denotemus hanc radicem realem non negativam per \(g\).

II. Subtrahendo ab aequatione 12, ita exhibita

hanc \(\cdot\)

\[
x=\frac{A A x}{a a+x}+\frac{B B x}{b b+x}+C C
\]

\[
g=\frac{A A g}{a a+g}+\frac{B B g}{b b+g}+C C
\]

et dividendo per \(x-g\), oritur nova, duas reliquas radices complectens

\[
1=\frac{a a A A}{(a a+x)(a a+g)}+\frac{b b B B}{(b b+x)(b b+g)}
\]

quae rite ordinata et soluta suppeditat [14]

\(2 x=\frac{a a A A}{a a+g}+\frac{b b B B}{b b+g}-a a-b b \pm \sqrt{ }\left(\left(a a-b b-\frac{a a A A}{a a+g}+\frac{b b B B}{b b+g}\right)^{2}+\frac{4 a a b b A A B B}{(a a+g)(b b+g)}\right)\)

Haec expressio, quum quantitas sub signo radicali natura sua sit positiva, vel saltem non negativa, monstrat, etiam duas reliquas radices semper fieri reales.

III. Subtrahendo autem ab invicem aequationes istas sic exhibitas

\[
\begin{aligned}
& g x=\frac{A A g x}{a a+x}+\frac{B B g x}{b b+x}+g C C \\
& g x=\frac{A A g x}{a a+g}+\frac{B B g x}{b b+g}+x C C
\end{aligned}
\]

et dividendo per \(g-x\), prodit aequatio duas reliquas radices continens in hacce forma :

\[
0=\frac{A A g x}{(a a+g)(a a+x)}+\frac{B B g x}{(b b+g)(b b+x)}+C C
\]

cui manifesto, si \(g\) est quantitas positiva, per valorem positivum ipsius \(x\) satisfieri nequit. Unde concludimus, aequationem nostram cubicam radices positivas plures quam unam habere non posse.

IV. Quoties itaque 0 non est inter radices aequationis nostrae, aderunt necessario radix una positiva cum duabus negativis. Quoties vero \(C=0\), adeoque 0 una radicum, reliquas complectetur aequatio

\[
x x-(A A+B B-a a-b b) x+a a b b-a a B B-b b A A=0
\]

unde hae radices exprimentur per

\[
\frac{1}{2}(A A+B B-a a-b b) \pm \frac{1}{2} V\left((A A-B B-a a+b b)^{2}+4 A A B B\right)
\]

Tres casus hic iterum distinguere oportebit.

Primo si terminus ultimus \(a a b b-a a B B-b b A A\) est positivus (i. e. si punctum attractum in plano ellipsis attrahentis intra curvam iacet), ambae radices, quum reales esse debeant, eodem signo affectae erunt, adeoque quum simul positivae esse nequeant, necessario erunt negativae. Ceterum hoc etiam independenter ab iis, quae iam demonstrata sunt, inde concludi potest. quod coëfficiens medius, quem ita exhibere licet

\[
(a a b b-a a B B-b b A A)\left(\frac{1}{a a}+\frac{1}{b b}\right)+\frac{b b A A}{a a}+\frac{a a b B}{b b}
\]

manifesto in hoc casu sit positivus.

Secundo, si terminus ultimus est negativus, sive punctum attractum in plano ellipsis extra curvam situm, necessario altera radix positiva erit, altera negativa.

Tertio autem, si terminus ultimus ipse evanesceret, sive punctum attractum in ipsa ellipsis circumferentia iaceret, etiam radix secunda fieret \(=0\), atque tertia

\[
=-\frac{b b A A}{a a}-\frac{a a B B}{b b}
\]

i. e. negativa. Ceterum hunc casum, physice impossibilem, et in quo attractio ipsa infinite magna evaderet, a disquisitione nostra, hocce saltem loco, excludemus.

7.

Ad determinandos coëfficientes \(\gamma, \gamma^{\prime}, \gamma^{\prime \prime}\), ex aequationibus \(1,3,4,6,7,9,10\) invenimus

\begin{center}
%\includegraphics[max width=\textwidth]{2024_01_11_75975a03bcf8b0416cd0g-183}
\end{center}

Ex his aequationibus rite cum 5, 8, 11 combinatis etiam sequitur:

\[
\left.\begin{array}{rl}
\gamma & =\sqrt{\frac{G G}{\left(\frac{A G}{a a+G}\right)^{2}+\left(\frac{B G}{b b+G}\right)^{2}+C C}} \\
\gamma^{\prime} & =\sqrt{\frac{G^{\prime}}{\left(\frac{A G^{\prime}}{a a-G^{\prime}}\right)^{2}+\left(\frac{B G^{\prime}}{b b-G^{\prime}}\right)^{2}+C C}} \\
\gamma^{\prime \prime} & =\sqrt{\frac{G^{\prime \prime}}{\left(\frac{A G^{\prime \prime}}{a a-G^{\prime \prime}}\right)^{2}+\left(\frac{B G^{\prime \prime}}{b b-G^{\prime \prime}}\right)^{2}+C C}}
\end{array}\right\}
\]

Hae posteriores expressiones ostendunt, nullam quantitatum \(G, G^{\prime}, G^{\prime \prime}\) negativam esse posse, siquidem \(\gamma, \gamma^{\prime}, \gamma^{\prime \prime}\) debent esse reales.

In casu itaque eo, ubi non est \(C=0\), necessario \(G\) aequalis statui debet radici positivae aequationis \(B\), patetque adeo, \(-G^{\prime}\) aequalem esse debere alteri radici negativae, atque \(-G^{\prime \prime}\) aequalem alteri \({ }^{*}\) ); utram vero radicem pro - \(G^{\prime}\), utram pro \(-G^{\prime \prime}\) adoptemus, prorsus arbitrarium erit.

Quoties \(C=0\), punctumque attractum intra curvam situm, duas radices negativas aequationis 13 necessario pro \(-G^{\prime}\) et \(-G^{\prime \prime}\) adoptare et proin \(G=0\) statuere oportet. Quoniam vero in hoc casu formula prima in 16 fit indeterminata, formulam primam in 15 eius loco retinebimus, quae suppeditat

\[
\gamma=\frac{1}{\left.\sqrt{\left(1-\frac{A A}{a a}\right.}-\frac{B B}{b b}\right)}
\]

Quoties autem pro \(C=0\) punctum attractum extra ellipsin iacet. aequa-

*) Proprie quidem ex analysi praecedenti tantummodo sequitur, \(-G^{\prime}\) et \(-G^{\prime \prime}\) satisfacere debere aequationi 13, unde dubium esse videtur, annon liceat, utramque \(-G^{\prime}\) et \(-G^{\prime \prime}\) eidem radici negativae aequalem ponere, prorsus neglecta radice tertia. Sed facile perspicietur, siquidem aequationis radix secunda et tertia sint inaequales, ex \(-G^{\prime}=-G^{\prime \prime}\) sequi \(\gamma^{\prime}=\gamma^{\prime \prime}, \alpha^{\prime}=\alpha^{\prime \prime}, \ell^{\prime}=\ell^{\prime \prime}\), et proin \(-a^{\prime} a^{\prime \prime}-b^{\prime} b^{\prime \prime}+\gamma^{\prime} \gamma^{\prime \prime}=-a^{\prime} a^{\prime}-b^{\prime} b^{\prime}+\gamma^{\prime} \gamma^{\prime}=1\), quod aequationi quartae in [1] est contrarium. Conf. quae infra de casu duarum radicum aequalium aequationis 13 dicentur.
tionis 13 radix positiva statuenda est \(=G\), atque vel negativa \(=-G^{\prime}\), et \(G^{\prime \prime}=0\), vel radix negativa \(=-G^{\prime \prime}\), et \(G^{\prime}=0\); coëfficientem \(\gamma^{\prime \prime}\) vel \(\gamma^{\prime}\) vero inveniemus per formulam

\[
\frac{1}{\sqrt{\left(\frac{A A}{a a}+\frac{B B}{b b}-1\right)}}
\]

Ceterum in casu iam excluso, ubi punctum attractum in ipsa circumferentia ellipsis situm supponeretur, coëfficientes \(\gamma\) et \(\gamma^{\prime}\), vel \(\gamma\) et \(\gamma^{\prime \prime}\) evaderent infiniti, quod indicat, transformationem nostram ad hunc casum omnino non esse applicabilem.

8.

Quamquam formulae 15,16 ad determinationem coëfficientium \(\gamma, \gamma^{\prime}, \gamma^{\prime \prime}\) sufficere possent, tamen etiam elegantiores assignare licet. Ad hunc finem multiplicabimus aequationem [5] per \(a a b b-G G\), unde prodit, levi reductione facta, \(\frac{a a A A(b b+G)}{a a+G}-A A G+\frac{b b B B(a a+G)}{b b+G}-B B G+\frac{a a b b C C}{G}-C C G=a a b b-G G\)

Sed e natura aequationis cubicae fit

summa radicum \(G-G^{\prime}-G^{\prime \prime}=A A+B B+C C-a a-b b\) productum radicum \(G G^{\prime} G^{\prime \prime}=a a b b C C\)

Hinc aequatio praecedens transit in sequentem:

\(\frac{a a A A(b b+G)}{a a+G}+\frac{b b B b(a a+G)}{b b+G}+G^{\prime} G^{\prime \prime}-G\left(G-G^{\prime}-G^{\prime \prime}+a a+b b\right)=a a b b-G G\) quam etiam sic exhibere licet

\[
\frac{a a A A(b b+G)}{a a+G}+\frac{b b B B(a a+G)}{b b+G}-(a a+G)(b b+G)+\left(G+G^{\prime}\right)\left(G+G^{\prime \prime}\right)=0
\]

Hinc valor coëfficientis \(\gamma\) e formula prima in [15] transmutatur in sequentem:

\[
\gamma=\sqrt{ } \frac{(a a+G)(b b+G)}{\left(G+G^{\prime}\right)\left(G+G^{\prime \prime}\right)}
\]

Per analysin prorsus similem invenitur

\[
\begin{aligned}
\gamma^{\prime} & =\sqrt{ } \frac{\left(a a-G^{\prime}\right)\left(b b-G^{\prime}\right)}{\left(G+G^{\prime}\right)\left(G^{\prime \prime}-G^{\prime}\right)} \\
\gamma^{\prime \prime} & =\sqrt{ } \frac{\left(a a-G^{\prime \prime}\right)\left(b b-G^{\prime \prime}\right)}{\left(G+G^{\prime \prime}\right)\left(G^{\prime}-G^{\prime \prime}\right)}
\end{aligned}
\]

Postquam coëfficientes \(\gamma, \gamma^{\prime}, \gamma^{\prime \prime}\) inventi sunt, reliqui \(\alpha, b, \alpha^{\prime}, b^{\prime}, \alpha^{\prime \prime}, b^{\prime \prime}\) inde per formulas \(3,4,6,7,9,10\) derivabuntur.

9.

Signa expressionum radicalium, per quas \(\gamma, \gamma^{\prime}, \gamma^{\prime \prime}\) determinavimus, ad lubitum accipi posse facile perspicitur. Operae autem pretium est, inquirere, quomodo signum quantitatis \(\varepsilon\) cum signis istis nexum sit. Ad hunc finem consideremus aequationem tertiam in III art. 3.

\[
\varepsilon \gamma=\alpha^{\prime} b^{\prime \prime}-b^{\prime} \alpha^{\prime \prime}
\]

quae per formulas \(6,7,9,10\) transmutatur in hanc:

\[
\begin{aligned}
\varepsilon \gamma & =\frac{a b A B \gamma^{\prime} \gamma^{\prime \prime}}{\left(a a-G^{\prime}\right)\left(b b-G^{\prime \prime}\right)}-\frac{a b A B \gamma^{\prime} \gamma^{\prime \prime}}{\left(a a-G^{\prime \prime}\right)\left(b b-G^{\prime}\right)} \\
& =\frac{a b(a a-b b) A B\left(G^{\prime \prime}-G^{\prime}\right) \gamma^{\prime} \gamma^{\prime \prime}}{\left(a a-G^{\prime}\right)\left(a a-G^{\prime \prime}\right)\left(b b-G^{\prime}\right)\left(b b-G^{\prime \prime}\right)}
\end{aligned}
\]

Sed e consideratione aequationis 13 facile deducimus

\[
\begin{aligned}
& (a a+G)\left(a a-G^{\prime}\right)\left(a a-G^{\prime \prime}\right)=a a(a a-b b) A A \\
& (b b+G)\left(b b-G^{\prime}\right)\left(b b-G^{\prime \prime}\right)=-b b(a a-b b) B B
\end{aligned}
\]

Hinc aequatio praecedens fit

\[
\varepsilon \gamma=\frac{(a a+G)(b b+G)\left(G^{\prime}-G^{\prime \prime}\right) \gamma^{\prime} \gamma^{\prime \prime}}{a b(a a-b b) A B}
\]

quae combinata cum aequatione 17 suppeditat

\[
\gamma \gamma^{\prime} \gamma^{\prime \prime}=\frac{\varepsilon a b(a a-b b) A B}{\left(G+G^{\prime}\right)\left(G+G^{\prime \prime}\right)\left(G^{\prime}-G^{\prime \prime}\right)}
\]

Hinc patet, si pro \(-G^{\prime}\) electa sit aequationis cubicae radix negativa absolute maior, simulque coëfficientes \(\gamma, \gamma^{\prime}, \gamma^{\prime \prime}\) omnes positive accepti sint, \(\varepsilon\) idem signum nancisci, quod habet \(A B\), idemque evenire, si his quatuor conditionibus, vel omnibus vel duabus ex ipsis, contraria acta sint, oppositum vero, si uni vel tribus conditionibus adversatus fueris. Ceterum sequentes adhuc relationes notare convenit, e praecedentibus facile derivandas:

\[
\begin{aligned}
\alpha \alpha^{\prime} \alpha^{\prime \prime} & =\frac{\varepsilon a a b A A B}{\left(G+G^{\prime}\right)\left(G+G^{\prime \prime}\right)\left(G^{\prime}-G^{\prime \prime}\right)} \\
b b^{\prime} b^{\prime \prime} & =-\frac{\varepsilon a b b A B B}{\left(G+G^{\prime}\right)\left(G+G^{\prime \prime}\right)\left(G^{\prime}-G^{\prime \prime}\right)} \\
\alpha b & =\frac{a b A B}{\left(G+G^{\prime}\right)\left(G+G^{\prime \prime}\right)} \\
\alpha^{\prime} b^{\prime} & =-\frac{a b A B}{\left(G+G^{\prime}\right)\left(G^{\prime}-G^{\prime \prime}\right)} \\
\alpha^{\prime \prime} b^{\prime \prime} & =\frac{a b A B}{\left(G+G^{\prime \prime}\right)\left(G^{\prime}-G^{\prime \prime}\right)}
\end{aligned}
\]

\section*{10.}
Formulae nostrae quibusdam casibus indeterminatae fieri possunt, quos seorsim considerare oportet. Ac primo quidem discutiemus casum eum . ubi aequationis cubicae radices negativae \(-G^{\prime},-G^{\prime \prime}\) aequales fiunt, unde, per formulas 18, 19, coëfficientes \(\gamma, \gamma^{\prime \prime}\) valores infinitos nancisci videntur, qui autem revera sunt indeterminati.

Statuendo in formula 14, \(g=G\), patet, ut duo valores ipsius \(x\), i. e. ut \(-G^{\prime}\) et \(-G^{\prime \prime}\) fiant aequales, necessario esse debere

\[
A B=0, \quad a a-b b-\frac{a a A A}{a a+G}+\frac{b b B B}{b b+G}=0
\]

Hinc facile intelligitur, quum \(a a-b b\) natura sua sit vel quantitas positiva, vel \(=0\), esse debere

\[
\begin{aligned}
& B=0 \\
& a a-b b=\frac{a a A A}{a a+G}, \quad \text { sive } \quad a a+G=\frac{a a A A}{a a-b b}
\end{aligned}
\]

Substituendo hos valores in aequatione 14, fit

\[
G^{\prime}=G^{\prime \prime}=b b
\]

Substituendo porro valorem \(x=-b b\) in aequatione cubica 13, prodit

\[
(a a-b b)(C C+b b)=b b A A
\]

Quoties haec aequatio conditionalis simul cum aequatione \(B=0\) locum habet, casus, quem hic tractamus, adducitur. Et quum fiat

\[
G=\frac{a a A A}{a a-b b}-a a=\frac{a a C C}{b b}
\]

formula 17 suppeditat

\[
\gamma=\sqrt{ } \frac{a a b b A A}{(a a-b b)\left(a a C C+b^{2}\right)}=\sqrt{ } \frac{a a C C+a a b b}{a a C C+b^{4}}
\]

ac dein formulae 3,4

\[
\begin{aligned}
& \alpha=\frac{\gamma(a a-b b)}{a A}=\frac{\gamma b b A}{a(C C+b b)}=\sqrt{ } \frac{b b(a a-b b)}{a a C C+b^{2}}=\sqrt{ } \frac{b^{2} A A}{(C C+b b)\left(a a C C+b^{4}\right)} \\
& b=0
\end{aligned}
\]

Valores coëfficientium \(\gamma^{\prime}, \gamma^{\prime \prime}\) per formulas 18,19 in hoc casu indeterminati manent, atque sic etiam valores coëfficientium reliquorum \(\alpha^{\prime}, b^{\prime}, \alpha^{\prime \prime}, b^{\prime \prime}\). Nihilominus per unum horum coëfficientium omnes quinque reliqui exprimi possunt, e. g. fit per formulam 6

\[
\alpha^{\prime}=\frac{\gamma^{\prime} a A}{a a-b b}
\]

ac dein

\(b^{\prime}=\dot{V}\left(1-\alpha^{\prime} \alpha^{\prime}+\gamma^{\prime} \gamma^{\prime}\right), \gamma^{\prime \prime}=\sqrt{ }\left(\gamma \gamma-1-\gamma^{\prime} \gamma^{\prime}\right), \alpha^{\prime \prime}=\frac{\gamma^{\prime \prime} a A}{a a-b b}, \quad b^{\prime \prime}=\sqrt{ }\left(1-\alpha^{\prime \prime} \alpha^{\prime \prime}+\gamma^{\prime \prime} \gamma^{\prime \prime}\right)\)

Sed concinnius hoc ita perficitur. Ex

sequitur

\[
\gamma \gamma=1+\alpha \alpha, \quad \alpha \alpha^{\prime}=\gamma \gamma^{\prime}, \quad 1=\alpha^{\prime} \alpha^{\prime}+b^{\prime} b^{\prime}-\gamma^{\prime} \gamma^{\prime}
\]

\[
b^{\prime} b^{\prime}+\frac{\gamma^{\prime} \gamma^{\prime}}{\alpha \alpha}=1-\alpha^{\prime} \alpha^{\prime}+\frac{\gamma \gamma \gamma^{\prime} \gamma^{\prime}}{\alpha \alpha}=1
\]

Quapropter statuere possumus

\[
b^{\prime}=\cos f, \quad \gamma^{\prime}=\alpha \sin f, \quad \alpha^{\prime}=\gamma \sin f
\]

Dein vero e formulis

\[
\varepsilon \alpha^{\prime \prime}=b \gamma^{\prime}-\gamma b^{\prime}, \varepsilon b^{\prime \prime}=\gamma \alpha^{\prime}-\alpha \gamma^{\prime}, \quad \varepsilon \gamma^{\prime \prime}=b \alpha^{\prime}-\alpha b^{\prime}, \varepsilon \varepsilon=1
\]

invenimus

\[
\alpha^{\prime \prime}=-\varepsilon \gamma \cos f, b^{\prime \prime}=\varepsilon \sin f, \gamma^{\prime \prime}=-\varepsilon \alpha \cos f
\]

Valor anguli \(f\) hic arbitrarius est, nec non pro lubitu statui poterit vel \(\varepsilon=+1\) vel \(\varepsilon=-1\).

\section*{11.}
Si \(G^{\prime}, G^{\prime \prime}\) sunt inaequales, valores coëfficientium \(\gamma, \gamma^{\prime}, \gamma^{\prime \prime}\) per formulas \(17,18,19\) indeterminati esse nequeunt, sed quoties aliqua quantitatum
\(a a-G^{\prime}, b b-G^{\prime}, a a-G^{\prime \prime}, b b-G^{\prime \prime}\) evanescit, valor coëfficientis \(\alpha^{\prime}, b^{\prime}, \alpha^{\prime \prime}, \gamma^{\prime \prime}\) per formulam \(6,7,9,10\) resp. indeterminatus manere primo aspectu videtur, quod tamen secus se habere levis attentio docebit.

Supponumus e. g., esse \(a a-G^{\prime}=0\), fietque, per aequationem \(18, \gamma^{\prime}=0\), nec non per aequationem \(7, b^{\prime}=0\) (siquidem non fuerit simul \(a a=b b\) ) unde necessario esse debet \(\alpha^{\prime}= \pm 1\). Si vero simul \(a a=b b\), formula, quae praecedit sextam in art. 5 , suppeditat \(\alpha^{\prime} A+b^{\prime} B=0\), quae aequatio cum \(a^{\prime} \alpha^{\prime}+b^{\prime} b^{\prime}=1\) iuncta, producit

\[
\alpha^{\prime}=\frac{B}{\sqrt{(A A+B B)}}, \quad b^{\prime}=\frac{-A}{\sqrt{(A A+B B)}}
\]

Hae expressiones manifesto indeterminatae esse nequeunt, nisi simul fuerit \(A=0, B=0\); tunc vero ad casum in art. praec. iam consideratum delaberemur.

12.

Postquam duodecim quantitates \(G, G^{\prime}, G^{\prime \prime}, \alpha, \alpha^{\prime}, \alpha^{\prime \prime}, b . b^{\prime} . b^{\prime \prime}, \gamma, \gamma^{\prime}, \gamma^{\prime \prime}\) complete determinare docuimus, ad evolutionem differentialis \(\mathrm{d} E\) progredimur. Statuamus

ita ut fiat

\[
t=\gamma+\gamma^{\prime} \cos T+\gamma^{\prime \prime} \sin T \ldots \ldots \ldots[20]
\]

\[
\begin{aligned}
& t \cos E=a+a^{\prime} \cos T+a^{\prime \prime} \sin T \ldots \ldots[21] \\
& t \sin E=b+b^{\prime} \cos T+b^{\prime \prime} \sin T \ldots \ldots[22]
\end{aligned}
\]

Hinc deducimus

\[
\begin{aligned}
t \mathrm{~d} E & =\cos E \mathrm{~d} \cdot t \sin E-\sin E \mathrm{~d} \cdot t \cos E \\
& =\cos E\left(b^{\prime \prime} \cos T-b^{\prime} \sin T\right) \mathrm{d} T-\sin E\left(\alpha^{\prime \prime} \cos T-\alpha^{\prime} \sin T\right) \mathrm{d} T
\end{aligned}
\]

adeoque

\[
\begin{aligned}
t t \mathrm{~d} E & =\left(\alpha b^{\prime \prime}-a^{\prime \prime} b\right) \cos T \mathrm{~d} T+\left(\alpha^{\prime} b-b^{\prime} \alpha\right) \sin T \mathrm{~d} T+\left(\alpha^{\prime} b^{\prime \prime}-b^{\prime} \alpha^{\prime \prime}\right) \mathrm{d} T \\
& =\varepsilon \gamma^{\prime} \cos T \mathrm{~d} T+\varepsilon \gamma^{\prime \prime} \sin T \mathrm{~d} T+\varepsilon \gamma \mathrm{d} T=\varepsilon t \mathrm{~d} T
\end{aligned}
\]

sive

\[
t \mathrm{~d} E=\varepsilon \mathrm{d} T \ldots \ldots . . . . . . . . . . . . . .[23]
\]

Observare convenit, quantitatem \(t\) natura sua semper positivam esse, si coëfficiens \(\gamma\) sit positivus, vel semper negativam, si \(\gamma\) sit negativus. Quum enim sit \(\left(\gamma^{\prime} \cos T+\gamma^{\prime \prime} \sin T\right)^{2}+\left(\gamma^{\prime \prime} \cos T-\gamma^{\prime} \sin T\right)^{2}=\gamma^{\prime} \gamma^{\prime}+\gamma^{\prime \prime} \gamma^{\prime \prime}=\gamma \gamma-1\), erit semper
\(\gamma^{\prime} \cos T+\gamma^{\prime \prime} \sin T\), sine respectu signi, minor quam \(\gamma\). Hinc concludimus, quoties \(\varepsilon \gamma\) sit quantitas positiva, variabiles \(E\) et \(T\) semper simul crescere; quoties autem \(\varepsilon \gamma\) sit quantitas negativa, necessario alteram variabilem semper decrescere, dum altera augeatur.

13.

Nexus inter variabiles \(E\) et \(T\) adhuc melius illustratur per ratiocinia sequentia. Statuendo \(V(\gamma \gamma-1)=\delta\), ita ut fiat \(\delta \delta=\alpha a+b b=\gamma^{\prime} \gamma^{\prime}+\gamma^{\prime \prime} \gamma^{\prime \prime}\), ex aequationibus 20, 21, 22 deducimus

\[
\begin{aligned}
& t(\delta+\alpha \cos E+b \sin E) \\
& \quad=\gamma \delta+\alpha \alpha+b b+\left(\gamma^{\prime} \delta+\alpha \alpha^{\prime}+b b^{\prime}\right) \cos T+\left(\gamma^{\prime \prime} \delta+\alpha \alpha^{\prime \prime}+b b^{\prime \prime}\right) \sin T \\
& \quad=(\gamma+\delta)\left(\delta+\gamma^{\prime} \cos T+\gamma^{\prime \prime} \sin T\right)
\end{aligned}
\]

Perinde ex aequationibus 21, 22 sequitur

\[
t(\alpha \sin E-b \cos E)=\varepsilon\left(\gamma^{\prime} \sin T-\gamma^{\prime \prime} \cos T\right)
\]

Hae aequationes, statuendo

\[
\frac{\alpha}{\delta}=\cos L, \quad \frac{\delta}{\delta}=\sin L, \quad \frac{\gamma^{\prime}}{\delta}=\cos M, \quad \frac{\gamma^{\prime \prime}}{\delta}=\sin M
\]

nanciscuntur formam sequentem:

\[
\begin{aligned}
t(1+\cos (E-L)) & =(\gamma+\delta)(1+\cos (T-M)) \\
t \sin (E-L) & =\varepsilon \sin (T-M)
\end{aligned}
\]

unde fit per divisionem, propter \((\gamma+\delta)(\gamma-\delta)=1\),

\[
\begin{aligned}
& \operatorname{tang} \frac{1}{2}(E-L)=\varepsilon(\gamma-\delta) \operatorname{tang} \frac{1}{2}(T-M) \\
& \operatorname{tang} \frac{1}{2}(T-M)=\varepsilon(\gamma+\delta) \operatorname{tang} \frac{1}{2}(E-L)
\end{aligned}
\]

Hinc non solum eadem conclusio derivatur, ad quam in fine art. praec. deducti sumus, sed insuper etiam patet, si valor ipsius \(\boldsymbol{E}\) crescat 360 gradibus, valorem ipsius \(T\) tantundem vel crescere vel diminui, prout \(\varepsilon \gamma\) sit vel quantitas positiva vel negativa. Ceterum statuendo \(\delta=\operatorname{tang} N, \gamma=\sec N\), manifesto erit

\[
\gamma-\delta=\operatorname{tang}\left(45^{0}-\frac{1}{2} N\right), \quad \gamma+\delta=\operatorname{tang}\left(45^{0}+\frac{1}{2} N\right)
\]

\section*{14.}
E combinatione aequationum 20,21,22 cum aequationibus art. 5 obtinemus :

\[
\begin{aligned}
& a t(A-a \cos E)=\alpha G-\alpha^{\prime} G^{\prime} \cos T-\alpha^{\prime \prime} G^{\prime \prime} \sin T \\
& b t(B-b \sin E)=b G-b^{\prime} G^{\prime} \cos T-b^{\prime \prime} G^{\prime \prime} \sin T
\end{aligned}
\]

Statuendo itaque brevitatis gratia

\[
\begin{gathered}
\left(\alpha G-\alpha^{\prime} G^{\prime} \cos T-\alpha^{\prime \prime} G^{\prime \prime} \sin T\right)\left(\gamma-e \alpha+\left(\gamma^{\prime}-e \alpha^{\prime}\right) \cos T+\left(\gamma^{\prime \prime}-e \alpha^{\prime \prime}\right) \sin T\right)=a X \\
\left(b G-b^{\prime} G^{\prime} \cos T-b^{\prime \prime} G^{\prime \prime} \sin T\right)\left(\gamma-e \alpha+\left(\gamma^{\prime}-e \alpha^{\prime}\right) \cos T+\left(\gamma^{\prime \prime}-e \alpha^{\prime \prime}\right) \sin T\right)=b Y \\
C\left(\gamma+\gamma^{\prime} \cos T+\gamma^{\prime \prime} \sin T\right)\left(\gamma-e \alpha+\left(\gamma^{\prime}-e \alpha^{\prime}\right) \cos T+\left(\gamma^{\prime \prime}-e \alpha^{\prime \prime}\right) \sin T\right)=Z
\end{gathered}
\]

fit

\[
\mathrm{d} \xi=\frac{\varepsilon X \mathrm{~d} T}{2 \pi t^{3} \rho^{3}}, \quad \mathrm{~d} \eta=\frac{\varepsilon Y \mathrm{~d} T}{2 \pi t^{3} \rho^{3}}, \quad \mathrm{~d} \zeta=\frac{\varepsilon Z \mathrm{~d} T}{2 \pi t^{3} \rho^{3}}
\]

Sed habetur

\[
t \rho= \pm \sqrt{ }\left(G+G^{\prime} \cos T^{2}+G^{\prime \prime} \sin T^{2}\right)
\]

signo superiore vel inferiore valente, prout \(t\) est quantitas positiva vel negativa ( \(\rho\) enim natura sua semper positive accipitur), i. e. prout coëfficiens \(\gamma\) est positivus vel negativus. Hinc

\[
\frac{\varepsilon \mathrm{d} T}{2 \pi t^{3} \rho^{3}}= \pm \frac{\mathrm{d} T}{2 \pi\left(G+G^{\prime} \cos T^{2}+G^{\prime} \sin T^{2}\right)^{\frac{3}{2}}}
\]

ubi signum ambiguum a signo quantitatis \(\gamma \varepsilon\) pendet.

Ut iam valores ipsarum \(\xi, \eta, \zeta\) obtineamus, integrationes differentialium exsequi oportet, a valore ipsius \(T\), cui respondet \(E=0\), usque ad valorem, cui respondet \(E=360^{\circ}\), sive etiam (quod manifesto eodem redit) a valore ipsius \(T\), cui respondet valor arbitrarius ipsius \(E\), usque ad valorem, cui respondet valor ipsius \(E\) auctus \(360^{\circ}\); licebit itaque integrare a \(T=0\) usque ad \(T=360^{\circ}\), quoties \(\varepsilon \gamma\) est quantitas positiva, vel a \(T=360^{\circ}\) usque ad \(T=0\), quoties \(\varepsilon_{\gamma}\) est negativa. Manifesto itaque, independenter a signo ipsius \(\varepsilon \gamma\), erit:

\[
\begin{aligned}
& \xi=\int \frac{X \mathrm{~d} T}{2 \pi\left(G+G^{\prime} \cos T^{2}+G^{\prime \prime} \sin T^{2}\right)^{\frac{3}{2}}} \\
& \eta=\int \frac{Y \mathrm{~d} T}{2 \pi\left(G+G^{\prime} \cos T^{2}+G^{\prime \prime} \sin T^{\prime 2}\right)^{\frac{3}{2}}} \\
& \zeta=\int \frac{Z \mathrm{~d} T}{2 \pi\left(G+G^{\prime} \cos T^{2}+G^{\prime \prime} \sin T^{2}\right)^{\frac{3}{2}}}
\end{aligned}
\]

integrationibus a \(T=0\) usque ad \(T=360^{\circ}\) extensis.

Nullo negotio perspicitur, integralia

\[
\begin{aligned}
& \int \frac{\cos T^{\prime} T}{\left(G+G^{\prime} \cos T^{2}+G^{\prime \prime} \sin T^{2}\right)^{\frac{3}{2}}} \\
& \int \frac{\sin T \mathrm{~d} T}{\left(G+G^{\prime} \cos T^{2}+G^{\prime \prime} \sin T^{2}\right)^{\frac{3}{2}}} \\
& \int \frac{\cos T^{2} \sin T \mathrm{~d} T}{\left(G+G^{\prime} \cos T^{2}+G^{\prime \prime} \sin T^{2}\right)^{\frac{3}{2}}}
\end{aligned}
\]

a \(T=180^{\circ}\) usque ad \(T=360^{\circ}\) extensa obtinere valores aequales iis, quos nanciscantur, si a \(T=0\) usque ad \(T=180^{\circ}\) extendantur, sed signis oppositis affectos; quapropter ista integralia a \(T=0\) usque ad \(T=360^{\circ}\) extensa manifesto fiunt \(=0\). Hinc colligimus. esse

\[
\begin{aligned}
& \xi=\int \frac{\left((\gamma-e \alpha) \alpha G-\left(\gamma^{\prime}-e \alpha^{\prime}\right) \alpha^{\prime} G^{\prime} \cos T^{2}-\left(\gamma^{\prime \prime}-e \alpha^{\prime \prime}\right) \alpha^{\prime \prime} G^{\prime \prime} \sin T^{2}\right) \mathrm{d} T}{2 \pi a\left(G+G^{\prime} \cos T^{2}+G^{\prime \prime} \sin T^{2}\right)^{\frac{3}{2}}} \\
& \eta=\int \frac{\left((\gamma-e \alpha) b G-\left(\gamma^{\prime}-e \alpha^{\prime}\right) b^{\prime} G^{\prime} \cos T^{2}-\left(\gamma^{\prime \prime}-e \alpha^{\prime \prime}\right) b^{\prime \prime} G^{\prime \prime} \sin T^{2}\right) \mathrm{d} T}{2 \pi b\left(G+G^{\prime} \cos T^{2}+G^{\prime \prime} \sin T^{2}\right)^{\frac{3}{2}}} \\
& \zeta=\int \frac{\left((\gamma-e \alpha) \gamma+\left(\gamma^{\prime}-e \alpha^{\prime}\right) \gamma^{\prime} \cos T^{2}+\left(\gamma^{\prime \prime}-e \alpha^{\prime \prime}\right) \gamma^{\prime \prime} \sin T^{2}\right) C \mathrm{~d} T^{\prime}}{2 \pi\left(G+G^{\prime} \cos T^{2}+G^{\prime \prime} \sin T^{2}\right)^{\frac{3}{2}}}
\end{aligned}
\]

integralibus a \(T=0\) usque ad \(T=360^{\circ}\) extensis. Quodsi itaque valores integralium, eadem extensione acceptorum,

\[
\begin{aligned}
& \int \frac{\cos T^{2} \mathrm{~d} T}{2 \pi\left(\left(G+G^{\prime}\right) \cos T^{2}+\left(G+G^{\prime \prime}\right) \sin T^{2}\right)^{\frac{3}{2}}} \\
& \int \frac{\sin T^{2} \mathrm{~d} T}{2 \pi\left(\left(G+G^{\prime}\right) \cos T^{2}+\left(G+G^{\prime \prime}\right) \sin T^{2}\right)^{\frac{3}{2}}}
\end{aligned}
\]

per \(P, Q\) denotamus, erit

\[
\begin{aligned}
a \xi & =\left((\gamma-e \alpha) \alpha G-\left(\gamma^{\prime}-e \alpha^{\prime}\right) \alpha^{\prime} G^{\prime}\right) P+\left((\gamma-e \alpha) \alpha G-\left(\gamma^{\prime \prime}-e a^{\prime \prime}\right) \alpha^{\prime \prime} G^{\prime \prime}\right) Q \\
b \eta & =\left((\gamma-e \alpha) b G-\left(\gamma^{\prime}-e \alpha^{\prime}\right) b^{\prime} G^{\prime}\right) P+\left((\gamma-e \alpha) b G-\left(\gamma^{\prime \prime}-e \alpha^{\prime \prime}\right) b^{\prime \prime} G^{\prime \prime}\right) Q \\
\zeta & =\left((\gamma-e \alpha) \gamma+\left(\gamma^{\prime}-e \alpha^{\prime}\right) \gamma^{\prime}\right) C P+\left((\gamma-e \alpha) \gamma+\left(\gamma^{\prime \prime}-e \alpha^{\prime \prime}\right) \gamma^{\prime \prime}\right) C Q
\end{aligned}
\]

quo pacto problema nostrum complete solutum est.

16.

Quod attinet ad quantitates \(P, Q\), manifesto quidem utraque fit

\[
=\frac{1}{2\left(G+G^{\prime}\right)^{\frac{3}{2}}}
\]

quoties \(G^{\prime}=G^{\prime \prime}\), in omnibus vero reliquis casibus ad transscendentes sunt referendae. Quas quomodo per series exprimere liceat, abunde constat. Lectoribus autem gratum fore speramus, si hacce occasione determinationem harum aliarumque transscendentium per algorithmum peculiarem expeditissimum explicemus, quo per multos iam abhinc annos frequenter usi sumus, et de quo alio loco copiosius agere propositum est.

Sint \(m, n\) duae quantitates positivae, statuamusque

\[
m^{\prime}=\frac{1}{2}(m+n), \quad n^{\prime}=\sqrt{ } m n
\]

ita ut \(m^{\prime}, n^{\prime}\) resp. sit medium arithmeticum et geometricum inter \(m\) et \(n\). Medium geometricum semper positive accipi supponemus. Perinde fiat

\[
\begin{array}{ll}
m^{\prime \prime}=\frac{1}{2}\left(m^{\prime}+n^{\prime}\right), & n^{\prime \prime}=\sqrt{ } m^{\prime} n^{\prime} \\
m^{\prime \prime \prime}=\frac{1}{2}\left(m^{\prime \prime}+n^{\prime \prime}\right), & n^{\prime \prime \prime}=\sqrt{ } m^{\prime \prime} n^{\prime \prime}
\end{array}
\]

et sic porro, quo pacto series \(m, m^{\prime}, m^{\prime \prime}, m^{\prime \prime \prime}\) etc., atque \(n, n^{\prime}, n^{\prime \prime}, n^{\prime \prime \prime}\) etc. versus limitem communem rapidissime convergent, quem per \(\mu\) designabimus, atque simpliciter medium arithmetico-geometricum inter \(m\) et \(n\) vocabimus. Iam demonstrabimus, \(\frac{1}{\mu}\) esse valorem integralis

\[
\int \frac{\mathrm{d} T}{\left.2 \pi \sqrt{\left(m m \cos T^{2}+n n \sin \right.} T^{2}\right)}
\]

a \(T=0\) usque ad \(T=360^{\circ}\) extensi.

Demonstr. Supponamus, variabilem \(T\) ita per aliam \(T^{\prime}\) exprimi, ut fiat

\[
\sin T=\frac{2 m \sin T^{\prime}}{(m+n) \cos T^{\prime 2}+2 m \sin T^{\prime 2}}
\]

perspicieturque facile, dum \(T^{\prime}\) a valore 0 usque ad \(90^{\circ}, 180^{\circ}, 270^{\circ}, 360^{\circ}\) augeatur, etiam \(T\) (etsi inaequalibus intervallis) a 0 usque ad \(90^{\circ}, 180^{\circ}, 270^{\circ}, 360^{\circ}\) crescere. Evolutione autem rite facta, invenitur esse

%\includegraphics[max width=\textwidth, center]{2024_01_11_75975a03bcf8b0416cd0g-192}
adeoque valores integralium

\[
\int \frac{\mathrm{d} T}{2 \pi \sqrt{ }\left(m m \cos T^{2}+n n \sin T^{2}\right)}, \quad \int \frac{\mathrm{d} \boldsymbol{T}^{\prime}}{2 \pi \sqrt{\left(m^{\prime} m^{\prime} \cos T^{\prime 2}+n^{\prime} n^{\prime} \sin T^{\prime 2}\right)}}
\]

si utriusque variabilis a valore 0 usque ad valorem \(360^{\circ}\) extenditur, inter se aequales. Et quum perinde ulterius continuare liceat, patet, his valoribus etiam aequalem esse valorem integralis

\[
\int \frac{d \theta}{2 \pi \sqrt{ }\left(\mu \mu \cos \theta^{2}+\mu \mu \sin \theta^{2}\right)}
\]

a \(\theta=0\) usque ad \(\theta=360^{\circ}\), qui manifesto fit \(=\frac{1}{\mu}\). Q. E. D.

\section*{17.}
Ex aequatione, relationem inter \(T\) et \(T^{\prime}\) exhibente,

\[
(m-n) \sin T \cdot \sin T^{\prime 2}=2 m \sin T^{\prime}-(m+n) \sin T
\]

facile deducitur

\[
\begin{aligned}
& \sqrt{ }\left(m m \cos T^{2}+n n \sin T^{2}\right)=m-(m-n) \sin T \cdot \sin T^{\prime} \\
& \sqrt{ }\left(m^{\prime} m^{\prime} \cos T^{2}+n^{\prime} n^{\prime} \sin T^{\prime 2}\right)=m \operatorname{cotang} T \cdot \operatorname{tang} T^{\prime}
\end{aligned}
\]

atque hinc, adiumento eiusdem aequationis,

\[
\begin{aligned}
& \sin T \cdot \sin T^{\prime} \cdot V\left(m m \cos T^{2}+n n \sin T^{2}\right)+m^{\prime}\left(\cos T^{2}-\sin T^{2}\right) \\
= & \cos T \cdot \cos T^{\prime} \cdot V\left(m^{\prime} m^{\prime} \cos T^{\prime 2}+n^{\prime} n^{\prime} \sin T^{\prime 2}\right)-\frac{1}{2}(m-n) \sin T^{2}
\end{aligned}
\]

Multiplicata hac aequatione per

\[
\frac{\mathrm{d} T}{\sqrt{\left(m m \cos T^{2}+n n \sin T^{2}\right)}}=\frac{\mathrm{d} T^{\prime}}{\sqrt{\left(m^{\prime} m^{\prime} \cos T^{\prime 2}+n^{\prime} n^{\prime} \sin T^{\prime 2}\right)}}
\]

prodit

\[
\frac{m^{\prime}\left(\cos T^{2}-\sin T^{2}\right) \mathrm{d} T}{\sqrt{\left(m m \cos T^{2}+n n \sin T^{2}\right)}}=-\frac{\frac{1}{2}(m-n) \sin T^{\prime 2} \mathrm{~d} T^{\prime}}{\sqrt{\left(m^{\prime} m^{\prime} \cos T^{\prime 2}+n^{\prime} n^{\prime} \sin T^{2}\right)}+\mathrm{d} \cdot \sin T^{\prime} \cos T}
\]

Multiplicando hanc aequationem per \(\frac{m-n}{\pi}\), substituendo \(m^{\prime}(m-n)=\frac{1}{2}(m m-n n)\), \((m-n)^{2}=4\left(m^{\prime} m^{\prime}-n^{\prime} n^{\prime}\right), \sin T^{2}=\frac{1}{2}-\frac{1}{2}\left(\cos T^{\prime 2}-\sin T^{\prime 2}\right)\), et integrando, a valoribus \(T\) et \(T^{\prime}=0\) usque ad \(360^{\circ}\), habemus:

\[
\begin{aligned}
& (m m-n n) \int \frac{\left(\cos T^{2}-\sin T^{2}\right) \cdot \mathrm{d} T}{2 \pi \sqrt{\left(m m \cos T^{2}+n n \sin T^{2}\right)}} \\
& =-\frac{2\left(m^{\prime} m^{\prime}-n^{\prime} n^{\prime}\right)}{\mu}+2\left(m^{\prime} m^{\prime}-n^{\prime} n^{\prime}\right) \int \frac{\left(\cos T^{\prime 2}-\sin T^{\prime 2}\right) d T^{\prime}}{2 \pi \sqrt{ }\left(m^{\prime} m^{\prime} \cos T^{\prime 2}+n^{\prime} n^{\prime} \sin T^{\prime 2}\right)}
\end{aligned}
\]

Et quum integrale definitum ad dextram perinde transformare liceat, manifesto integrale

\[
\int \frac{\left(\cos T^{2}-\sin T^{2}\right) d T}{2 \pi \sqrt{\left(m m \cos T^{2}+n n \sin T^{2}\right)}}
\]

exprimetur per seriem infinitam citissime convergentem

\[
-\frac{2\left(m^{\prime} m^{\prime}-n^{\prime} n^{\prime}\right)+4\left(m^{\prime \prime} m^{\prime \prime}-n^{\prime \prime} n^{\prime \prime}\right)+s\left(m^{\prime \prime \prime} m^{\prime \prime \prime}-n^{\prime \prime \prime} n^{\prime \prime \prime}\right)+\text { etc. }}{(m m-n n) \mu}=-\frac{\vee}{\mu}
\]

Calculus numericus commodissime per logarithmos perficitur, si statuimus

\[
\frac{1}{4} V(m m-n n)=\lambda, \quad \frac{1}{4} \sqrt{ }\left(m^{\prime} m^{\prime}-n^{\prime} n^{\prime}\right)=\lambda^{\prime}, \quad \frac{1}{4} V\left(m^{\prime \prime} m^{\prime \prime}-n^{\prime \prime} n^{\prime \prime}\right)=\lambda^{\prime \prime} \text { etc. }
\]

unde erit

\[
\begin{aligned}
& \lambda^{\prime}=\frac{\lambda \lambda}{m^{\prime}}, \quad \lambda^{\prime \prime}=\frac{\lambda^{\prime} \lambda^{\prime}}{m^{\prime \prime}}, \quad \lambda^{\prime \prime \prime}=\frac{\lambda^{\prime \prime} \lambda^{\prime \prime}}{m^{\prime \prime \prime}} \text { etc. atque } \\
& \nu=\frac{2 \lambda^{\prime} \lambda^{\prime}+4 \lambda^{\prime \prime} \lambda^{\prime \prime}+8 \lambda^{\prime \prime \prime} \lambda^{\prime \prime \prime}+\text { etc. }}{\lambda \lambda}
\end{aligned}
\]

18.

Per methodum hic explicatam etiam integralia indefinita (a valore variabilis \(=0\) inchoantia) maxima concinnitate assignare licet. Scilicet, si \(T^{\prime \prime}\) perinde per \(m^{\prime}, n^{\prime}, T^{\prime}\) determinari supponitur, uti \(T^{\prime}\) per \(m, n, T\), ac perinde rursus \(T^{\prime \prime \prime}\) per \(m^{\prime \prime}, n^{\prime \prime}, T^{\prime \prime}\) et sic porro, etiam pro quovis valore determinato ipsius \(T\), valores terminorum serie \(T, T^{\prime}, T^{\prime \prime}, T^{\prime \prime \prime}\) etc. ad limitem \(\theta\) citissime convergent, eritque

\[
\begin{aligned}
& \int \frac{\mathrm{d} T}{\sqrt{\left(m m \cos T^{2}+n n \sin T^{2}\right)}}=\frac{\theta}{\mu} \\
& \int \frac{\left(\cos T^{2}-\sin T^{2}\right) \mathrm{d} T}{\sqrt{\left(m m \cos T^{2}+n n \sin T^{2}\right)}}=-\frac{\nu \theta}{\mu}+\frac{\lambda^{\prime} \cos T \sin T^{\prime}+2 \lambda^{\prime \prime} \cos T^{\prime} \sin T^{\prime \prime}+4 \lambda^{\prime \prime \prime} \cos T^{\prime \prime} \sin T^{\prime \prime \prime}+\text { etc. }}{\lambda \lambda}
\end{aligned}
\]

Sed haec obiter hic addigitavisse sufficiat, quum ad institutum nostrum non sint necessaria.

19.

Quodsi iam statuimus \(m=V\left(G+G^{\prime}\right), n=\sqrt{ }\left(G+G^{\prime \prime}\right)\), valores quantitatum \(P, Q\) facile ad transscendentes \(\mu, \nu\) reducentur. Quum enim \(P, Q \operatorname{sint}\) valores integralium

\[
\int \frac{\cos T^{2} \mathrm{~d} T}{2 \pi\left(m m \cos T^{2}+n n \sin T^{2}\right)^{\frac{3}{2}}}, \quad \int \frac{\sin T^{2} \mathrm{~d} T}{2 \pi\left(m m \cos T^{2}+n n \sin T^{2}\right)^{\frac{3}{2}}}
\]

a \(T=0\) usque ad \(\dot{T}=360^{\circ}\) extensorum, primo statim obvium est, haberi

\[
m m P+n n Q=\frac{1}{\mu}
\]

Porro fit

\[
\begin{aligned}
& \frac{\left(\cos T^{2}-\sin T^{2}\right) \mathrm{d} T}{2 \pi \sqrt{ }\left(m m \cos T^{2}+n n \sin T^{2}\right)}+\frac{\left(m m \cos T^{2}-n n \sin T^{2}\right) \mathrm{d} T}{2 \pi\left(m m \cos T^{2}+n n \sin T^{2}\right)^{\frac{3}{2}}}=\frac{\left(m m \cos T^{2}-n n \sin T^{2}\right) \mathrm{d} T}{\pi\left(m m \cos T^{2}+n n \sin \boldsymbol{T}^{2}\right)^{\frac{3}{2}}} \\
& =\mathrm{d} \cdot \frac{\cos T \sin T}{\pi \sqrt{\left(m m \cos T^{2}+n n \sin T^{2}\right)}}
\end{aligned}
\]

Integrando hanc aequationem a \(T=0\) usque ad \(T=360^{\circ}\), prodit

\[
-\frac{\nu}{\mu}+m m P-n n Q=0 \ldots \ldots[25]
\]

E combinatione aequationum 24,25 denique colligimus

\[
P=\frac{1+\nu}{2 m m \mu}, \quad Q=\frac{1-\nu}{2 n n \mu}
\]


\end{document}