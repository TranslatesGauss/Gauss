\documentclass[14pt]{memoir}
\usepackage{standalone}
\usepackage[dvips,text={6.5truein,9.1truein},left=0.86truein,right=0.8truein,top=1truein]{geometry}
\usepackage{amsmath, amsthm, amsfonts}
\usepackage{titlesec}
\usepackage{enumitem}
% Uncomment to use syncing
%\usepackage{pdfsync}


% Paragraphs
\usepackage{indentfirst}
\parindent=3em
\parskip=0pt

%font
\usepackage{mlmodern}
%\usepackage[T1]{fontenc}% http://ctan.org/pkg/fontenc
\usepackage{microtype}

\titleformat{\section}
 {\centering}{\thesection.}{0em}{}

\titleformat{\subsection}
 {\normalfont\small\centering}{\thesection.}{0em}{}
\titlespacing*{\subsection}
{0pt}{\baselineskip}{0\baselineskip}

%footnotes
\usepackage[perpage]{footmisc}
\usepackage{etoolbox}
\DefineFNsymbols*{asterisks}{{ *}{ **}{ ***}}
\setfnsymbol{asterisks}
\renewcommand{\thefootnote}{\fnsymbol{footnote}}
\makeatletter
\renewcommand{\@makefnmark}{\mbox{\normalfont\@thefnmark})}
\settowidth{\footnotemargin}{***) }
\patchcmd{\@makefntext}{\hss\@makefnmark}{\hss \@makefnmark\ }{}{}
\makeatother

%Line Spacing
\renewcommand{\baselinestretch}{1.1}
\renewcommand{\footnotelayout}{ \baselineskip=1.02\baselineskip }
\setlength{\skip\footins}{2\baselineskip}

\theoremstyle{plain}
\newtheorem*{theorem}{Theorem}
\newtheorem{proposition}{Proposition}
\newtheorem{lemma}{Lemma}
\newtheorem{problem}{Problem}


\theoremstyle{remark}
\newtheorem*{example}{Example}
\newtheorem*{examples}{Examples}

\begin{document}

\setlength{\abovedisplayskip}{0.5\baselineskip}
\setlength{\belowdisplayskip}{0.5\baselineskip}


 \section*{ÜBER EIN NEUES\\[\baselineskip]
  ALLGEMEINES GRUNDGESETZ DER MECHANIK.\\[\baselineskip]
   \rule{0.85in}{0.5pt}}
  
Bekanntlich verwandelt das Princip der virtuellen Geschwindigkeiten die ganze Statik in eine mathematische Aufgabe, und durch \textsc{Dalembert}'s Princip für die Dynamik ist diese wiederum auf die Statik zurückgeführt. Es liegt daher in der Natur der Sache, dass es kein neues Grundprincip für die Bewegungs - und Gleichgewichts - Lehre geben \textit{kann}, welches der Materie nach nicht in jenen beiden schon enthalten und aus ihnen abzuleiten wäre. Inzwischen scheint doch wegen dieses Umstandes noch nicht jedes neue Princip werthlos zu werden. Es wird allezeit interessant und lehrreich bleiben, den Naturgesetzen einen neuen vortheilhaften Gesichtspunkt abzugewinnen, sei es, dass man aus demselben diese oder jene einzelne Aufgabe leichter auflösen könne, oder dass sich aus ihm eine besondere Angemessenheit offenbare. Der grosse Geometer, der das Gebäude der Mechanik auf dem Grunde des Princips der virtuellen Geschwindigkeiten auf eine so glänzende Art aufgeführt hat, hat es nicht verschmäht, \textsc{Maupertuis} Princip der kleinsten Wirkung zu grösserer Bestimmtheit und Allgemeinheit zu erheben, ein Princip, dessen man sich zuweilen mit vielem Vortheil bedienen kann\footnote{Es sei mir jedoch hier die Bemerkung erlaubt, dass ich die Art, wie ein anderer grosser Geometer versucht hat, \textsc{Huyghens} Gesetz für die ausserordentliche Brechung des Lichts in Krystallen von pelter Brechung, vermittelst des Grundsatzes der kleinsten Wirkung zu beweisen, nicht befriedigend finde. In der That ist die Zulässigkeit dieses Grundsatzes wesentlich von dem der Erhaltung der lebendigen Krafte abhängig, nach welchem die Geschwindigkeiten der bewegten materiellen Punkte bloss durch ihre Plätze bedingt werden, ohne dass die Richtung der Bewegung Einfluss darauf haben kann, was doch in dem erwähnten Versuch vorausgesetzt wird. Es scheint mir, dass im Emanationssystem alle Bemühungen, die Erscheinungen der doppelten Brechung an die allgemeinen dynamischen Gesetze anzuknüpfen, so lange erfolglos bleiben müssen, als man die Lichttheilchen bloss wie Punkte betrachtet. }. %pagebreak

Der eigenthümliche Charakter des Princips der virtuellen Geschwindigkeiten besteht darin, dass es eine allgemeine Formel zur Auflösung aller statischen Aufgaben, und so der Stellvertreter aller andern Principe ist, ohne jedoch das Creditiv dazu so unmittelbar aufzuweisen, dass es sich, so wie es nur ausgesprochen wird, schon von selbst als plausibel empföhle. 

In \textit{dieser} Beziehung scheint das Princip, welches ich hier aufstellen werde, den Vorzug zu haben: es hat aber auch noch den zweiten, dass es das Gesetz der Bewegung und der Ruhe auf ganz gleiche Art in grösster Allgemeinheit umfasst. So sehr es in der Ordnung ist, dass bei der allmäligen Ausbildung der Wissenschaft und bei der Belehrung des Individuum das Leichtere dem Schwereren, das Einfachere dem Verwickeltern, das Besondere dem Allgemeinen vorangeht, so fordert doch der Geist, einmal auf dem höhern Standpunkte angelangt, den umgekehrten Gang, wobei die ganze Statik nur als ein ganz specieller Fall der Mechanik erscheine. Selbst der oben erwähnte Geometer scheint darauf Werth zu legen, indem er als einen Vorzug des Princips der kleinsten Wirkung ansieht, dass es das Gleichgewicht und die Bewegung zugleich umfasse, wenn man jenes so ausdrücke, dass die lebendigen Kräfte bei beiden Kleinste seien, eine Bemerkung, die doch mehr witzig als wahr zu sein scheint, da das Minimum in beiden Fällen in ganz verschiedener Beziehung Statt findet. 

Das neue Princip ist nun folgendes:

\textit{Die Bewegung eines Systems materieller, auf was immer für eine Art unter sich verknüpfter Punkte, deren Bewegungen zugleich an was immer für äussere Beschränkungen gebunden sind, geschieht in jedem Augenblick in möglich grösster Übereinstimmung mit der freien Bewegung, oder unter möglich kleinstem Zwange, indem man als Maass des Zwanges, den das ganze System in jedem Zeittheilchen erleidet, die 
Summe der Producte aus dem Quadrate der Ablenkung jedes Punkts von seiner freien Bewegung in seine Masse betrachtet.}

Es seien \(m\), \(m'\), \(m''\) u.s.w. die Massen der Punkte; \(a\), \(a'\), \(a''\) u.s.w. ihre Plätze zur Zeit \(t\); \(b\),\(b'\), \(b''\) u.s.w. die Plätze, welche sie, nach dem unendlich %pagebreak
kleinen Zeittheilchen \(dt\), in Folge der während dieser Zeit auf sie wirkenden Kräfte und der zur Zeit \(t\) erlangten Geschwindigkeiten und Richtungen, einnehmen würden, falls sie alle vollkommen frei wären. Die wirklichen Plätze \(c\), \(c'\), \(c''\) u.s.w. werden dann diejenigen sein, für welche, unter allen mit den Bedingungen des Systems vereinbaren, \(m(bc)^2 + m'(b'c')^2+m''(b''c'')^2\) u.s.w. ein Minimum wird. 

Das Gleichgewicht ist offenbar nur ein einzelner Fall des allgemeinen Gesetzes, und die Bedingung dafür, dass 
\[ m(ab)^2+m'(a'b')^2+m''(a''b'')^2 \text{ u.s.w.}\]
selbst ein Minimum sei, oder dass das Beharren des Systems im Zustande der Ruhe, der freien Bewegung der einzelnen Punkte näher liege, als jedes mögliche Heraustreten aus demselben. 

Die Ableitung unsers Princips aus dem oben angeführten geschieht leicht auf folgende Art. 

Die auf den materiellen Punkt \(m\) wirkende Kraft ist offenbar zusammengesetzt, erstens aus einer, die, in Verbindung mit der zur Zeit \(t\) Statt habenden Geschwindigkeit und Richtung, ihn in der Zeit \(dt\) von \(a\) nach \(c\) führt, und in einer zweiten, die ihn in derselben Zeit aus der Ruhe in \(c\), durch \(cb\) führen würde, wenn man den Punkt als frei betrachtet. Dasselbe gilt von den andern Punkten. Nach \textsc{Dalembert}'s Princip müssen demnach die Punkte \(m\), \(m'\), \(m''\) u.s.w., unter alleiniger Wirkung der zweiten Kräfte, nach \(cb\), \(c’b’\), \(c''b''\) u.s.w., in den Plätzen \(c\), \(c'\), \(c''\) u.s.w. vermöge der Bedingungen des Systems, im Gleichgewicht sein. 

Nach dem Princip der virtuellen Geschwindigkeiten erfordert dies Gleichgewicht, dass die Summe der Producte aus je drei Factoren, nemlich jeder der Massen \(m\), \(m'\), \(m''\) u.s.w., den Linien \(cb\), \(c'b'\), \(c''b''\) u.s.w., und irgend welchen auf letztere resp. projicirten, vermöge der Bedingungen des Systems möglichen Bewegungen jener Punkte, immer = 0 sei, wie man es gewöhnlich ausspricht\footnote{Der gewöhnliche Ausdruck setzt stillschweigend solche Bedingungen voraus, dass die jeder möglichen Bewegung entgegengesetzte gleichfalls möglich sei, wie z. B. dass ein Punkt auf einer bestimmten Fläche zu bleiben genöthigt, dass die Entfernung zweier Punkte von einander unveränderlich sei u. dgl. Allein dies ist eine unnöthige und der Natur nicht immer angemessene Beschränkung. Die Oberfläche eines undurchdringlichen Körpers zwingt einen auf ihr befindlichen materiellen Punkt nicht, auf ihr zu bleiben, sondern verwehrt ihr bloss das Austreten auf die Eine Seite; ein gespannter, nicht ausdehnbarer aber biegsamer Faden zwischen zwei Punkten macht nur die Zunahme, nicht die Abnahme der Entfernung unmöglich u.s.w. Warum wollten wir also das Gesetz der virtuellen Geschwindigkeiten nicht lieber gleich anfangs so ausdrücken, dass es \textit{alle} Fälle umfasst?},%pagebreak
oder richtiger, dass jene Summe niemals positiv werden könne. Sind daher \(\gamma\), \(\gamma'\), \(\gamma''\) u.s.w. von \(c\), \(c'\), \(c''\) u.s. w. verschiedene, aber mit den Bedingungen des Systems verträgliche Plätze; und \(\theta\), \(\theta'\), \(\theta''\) u.s.w. die Winkel, welche \(c\gamma\), \(c'\gamma'\), \(c''\gamma''\) u.s.w. mit \(cb\), \(c'b'\), \(c''b''\) u.s.w. machen, so ist allemal \(\sum m . cb.c\gamma \cos \theta\) entweder \(0\) oder negativ. Da nun 
\[ \gamma b^2=cb^2+c\gamma^2-2cb.c\gamma.\cos \theta\]
so ist klar, dass 
\[ \sum m.\gamma b^2 - \sum m.cb^2 = \sum m.c\gamma^2-2\sum m.cb.c\gamma.\cos\theta\]
folglich immer positiv sein wird, also \(\sum m.\gamma b^2\) immer grösser als \(\sum m.cb^2\), d. i. dass \(\sum m.cb^2\) ein Minimum sein wird. W.Z.B.W. 

Es ist sehr merkwürdig, dass die freien Bewegungen, wenn sie mit den nothwendigen Bedingungen nicht bestehen können, von der Natur gerade auf dieselbe Art modificirt werden, wie der rechnende Mathematiker, nach der Methode der kleinsten Quadrate, Erfahrungen ausgleicht, die sich auf unter einander durch nothwendige Abhängigkeit verknüpfte Grössen beziehen. Diese Analogie liesse sich noch weiter verfolgen, was jedoch gegenwärtig nicht zu meiner Absicht gehört. 

\end{document}