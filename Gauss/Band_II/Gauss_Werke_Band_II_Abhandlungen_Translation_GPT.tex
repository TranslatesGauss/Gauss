\documentclass[twoside,12pt, showframe]{memoir}
\usepackage{standalone}
\usepackage{array}
\usepackage{mlmodern}
\usepackage[utf8]{inputenc}
\usepackage[T1]{fontenc}
\usepackage[dvips,text={6.25in,8.5in},left=1.125truein,top=1.5truein]{geometry}
\renewcommand{\baselinestretch}{1.25}
\usepackage{amsmath}
\usepackage{amsfonts}
\usepackage{amssymb}
\usepackage{graphicx}
\usepackage{CJKutf8}
\usepackage{multirow}
\usepackage{indentfirst}
\usepackage{mathtools}
\newtagform{brackets}{[}{]}
\usetagform{brackets}
\geometry{footnotesep=0.5\baselineskip}
\parindent=3em
\parskip=0pt
\usepackage{titlesec}
\titleformat{\section}
  {\normalfont\centering}{\thesection.}{1em}{}
\titleformat{\subsection}
  {\normalfont\normalsize\centering}{\thesection.}{1em}{}
\titleformat{\subsubsection}
  {\normalfont\normalsize\centering}{\thesection.}{1em}{}
  \titlespacing\subsection{0pt}{12pt plus 6pt minus 6pt}{0pt plus 2pt minus 2pt}
\def\equation{\old@equation\small\hskip\textwidth minus \textwidth}
\def\endequation{\leaders\hbox{ . }\hskip \textwidth minus \textwidth\old@endequation}
\spaceskip=0.67em plus 0.33em minus 0.33em
\renewenvironment{quote}%
  {\list{}{\leftmargin=5em\rightmargin=0em}\item[]}%
  {\endlist}
\renewcommand{\pmod}[1]{\;(\textrm{mod.}\;#1)}
\let\oldfrac\frac
\def\frac#1#2{\mathchoice{\tfrac{#1}{#2}}{\oldfrac{#1}{#2}}{\genfrac{}{}{}{2}{#1}{#2\mathstrut}}{\genfrac{}{}{}{3}{#1}{#2\mathstrut}}}
\thickmuskip=4mu plus 4mu
\medmuskip=3mu plus 1.5mu minus 3mu
\arraycolsep=3pt%\def\arraystretch{1.03}
%\setlength{\tabcolsep}{3pt}
\DeclareMathSizes{12}{12}{9}{6}
\begin{document}
\setlength{\abovedisplayskip}{0.33\baselineskip plus .25\baselineskip minus .25\baselineskip}
\setlength{\belowdisplayskip}{0.33\baselineskip plus .25\baselineskip minus .25\baselineskip}

\begin{center}
{\Huge ARITHMETIC THEOREM}\\[3\baselineskip]
{\large NEW PROOF}\\[4\baselineskip]
{\tiny BY THE AUTHOR}\\[2\baselineskip]
CAROLUS FREDERICUS GAUSS\\[2\baselineskip]
{\scriptsize PRESENTED TO THE ROYAL SOCIETY OF SCIENCE ON JAN. 15, 1504}\\[4\baselineskip]
\rule{3in}{0.5pt}\\[0.5\baselineskip]
{\scriptsize Commentaries of the Royal Society of Science in Göttingen. Vol. \textsc{XVI}.\\
Göttingen \textsc{mdcccviii}.}\\
\rule{3in}{0.5pt}
\end{center}
\clearpage\noindent% 2
\;
\clearpage\noindent% 3
\section*{\;\\[\baselineskip]
{\large ARITHMETIC THEOREM} \\[\baselineskip]
{\small NEW PROOF.} \\[0.5\baselineskip]
\rule{0.75in}{0.5pt}}
%

\subsection*{1.}

Questions from higher arithmetic frequently present singular phenomena, which occur far less frequently in analysis and greatly contribute to the allure of these questions. Namely, in analytic investigations, it is often not permitted to reach new truths unless we have thoroughly grasped the principles on which they are based and have opened up a path to them. On the other hand, in arithmetic, very often by a certain unexpected chance, most elegant new truths leap forth through induction, the demonstrations of which are deeply hidden and enveloped in such darkness that they elude all attempts and deny access to the most rigorous inquiries. Furthermore, there exists such a great and marvelous connection between arithmetic truths, which are seemingly very heterogeneous at first glance, that not infrequently, while we are seeking something far different, we finally arrive, by a much different path than expected, at the much longed-for demonstration, after having vainly sought it through lengthy meditations. However, most of these truths are of such a nature that they can be approached by very different paths, and the shortest paths are not always the ones that present themselves first. Therefore, it is certainly of great value if, after such a truth has been thoroughly explored in vain and then demonstrated, albeit through more obscure detours, we eventually manage to uncover the simplest and most genuine path.
%

\subsection*{2.}

Among the questions we mentioned in the preceding article, an important place is occupied by a theorem which contains almost all of the theory of quadratic residues, which in the \textit{Arithmetic Investigations} (Section IV) is distinguished by the name \textit{Fundamental Theorem}. The illustrious \textsc{Legendre}, without a doubt, should be regarded as the \textit{first} discoverer of this most elegant theorem, since long before the great geometers \textsc{Euler} and \textsc{Lagrange} had already discovered many of its special cases by induction. I do not dwell here on the efforts of these men in enumerating the demonstration; they may proceed to that which has been mentioned just now if they so desire. However, it may be permitted to add, as a confirmation of what was stated in the preceding article, which pertains to my own efforts. I stumbled upon the theorem itself independently in the year 1795, while I was completely ignorant of all things that had already been elaborated in higher arithmetic, and had been completely cut off from literary resources. But it tormented me for a whole year and evaded my most earnest efforts until I finally obtained the demonstration in the fourth section of that work. Afterwards, three other demonstrations presented themselves to me, relying on entirely different principles, one of which I presented in the fifth section, and the others, not inferior in elegance, I will make publicly available on another occasion. But all of these demonstrations, although they seem to leave nothing to be desired in terms of rigor, are derived from principles that are too heterogeneous, except perhaps for the first one, which however proceeds by more laborious reasoning and is burdened with lengthier calculations. Therefore, I have no doubt that a \textit{genuine} demonstration has not yet been provided: let the experts now judge whether the one that I recently succeeded in discovering, and which the following pages present, deserves to be adorned with this name.
%

\subsection*{3.}

\textsc{Theorem.} \textit{Let \(p\) be a positive prime number; \(k\) any integer not divisible by \(p\);
\[\begin{aligned}
&A \text{ a set of numbers }1,2,3 \ldots \frac{1}{2}(p-1)\\
&B \text{ a set of numbers } \frac{1}{2}(p+1), \frac{1}{2}(p+3), \frac{1}{2}(p+5) \ldots p-1
\end{aligned}\]
Let the least positive residues of the products of \(k\) with each number in \(A\) be chosen modulo \(p\), which will clearly all be different, and some will belong to \(A\) and some to \(B\). Now if it is assumed that all \(\mu\) of the residues belong to \(B\), \(k\) will either be a quadratic residue or a non-quadratic residue modulo \(p\), according as \(\mu\) is even or odd.}

\textit{Proof.} Let the residues belonging to \(A\) be \(a\), \(a^{\prime}\), \(a^{\prime \prime} \ldots\), and the remaining residues belonging to \(B\) be \(b\), \(b^{\prime}\), \(b^{\prime \prime} \ldots\), and it is clear that all the complements to these, \(p-b\), \(p-b^{\prime}\), \(p-b^{\prime \prime} \ldots\), are different from the numbers \(a\), \(a^{\prime}\), \(a^{\prime \prime} \ldots\), while together they complete the set \(A\). We therefore have
\[1 . 2 . 3 \ldots \frac{1}{2}(p-1)=a a^{\prime} a^{\prime \prime} \ldots(p-b)(p-b^{\prime})(p-b^{\prime \prime}) \ldots\]
Now the latter product clearly becomes
\[\begin{aligned}
& \equiv(-1)^{\mu} a a^{\prime} a^{\prime \prime} \ldots b b^{\prime} b^{\prime \prime} \ldots \equiv(-1)^{\mu} k . 2 k .3 k \ldots \frac{1}{2}(p-1) k \\
& \equiv(-1)^{\mu} k^{\frac{1}{2}(p-1)} 1.2 .3 \ldots \frac{1}{2}(p-1) \quad\pmod{p}
\end{aligned}\]
Hence we have
\[1 \equiv(-1)^{\mu} k^{\frac{1}{2}(p-1)}\]
or \(k^{\frac{1}{2}(p-1)} \equiv \pm 1\), according as \(\mu\) is even or odd, from which our theorem immediately follows.
%

\subsection*{4.}

It will be very convenient to abbreviate the following calculations by introducing certain appropriate notations. Therefore, let the symbol \((k, p)\) denote the set of residues of the products
\[k, 2k, 3k, \ldots, \frac{1}{2}(p-1)k,\]
whose least positive residues exceed \(\frac{1}{2}p\). Moreover, for any non-integral quantity \(x\), we denote by \([x]\) the greatest integer less than or equal to \(x\), so that \(x-[x]\) is always a positive quantity between 0 and 1. We can now develop the following relations with ease:

I. \([x]+[-x]=-1\).

II. \([x]+h=[x+h]\), whenever \(h\) is an integer.

III. \([x]+[h-x]=h-1\).

IV. If \(x-[x]\) is a fraction smaller than \(\frac{1}{2}\), then \([2x]-2[x]=0\); if it is greater than \(\frac{1}{2}\), then \([2x]-2[x]=1\).

V. If the least positive residue of the integer \(h\) exceeds \(\frac{1}{2}p\) modulo \(p\), then \(\left[\frac{2h}{p}\right]-2\left[\frac{h}{p}\right]=0\); if it is less than or equal to \(\frac{1}{2}p\) modulo \(p\), then \(\left[\frac{2h}{p}\right]-2\left[\frac{h}{p}\right]=1\). 

Hence, it immediately follows that \((k, p)=\)
\[\begin{gathered}
{\left[\frac{2k}{p}\right]+\left[\frac{4k}{p}\right]+\left[\frac{6k}{p}\right] \ldots+\left[\frac{(p-1)k}{p}\right]} \\
-2\left[\frac{k}{p}\right]-2\left[\frac{2k}{p}\right]-2\left[\frac{3k}{p}\right] \ldots-2\left[\frac{\frac{1}{2}(p-1)k}{p}\right].
\end{gathered}\]

VII. From VI and I, we can easily deduce
\[(k, p)+(-k, p)=\frac{1}{2}(p-1)\]
Hence, it follows that \(-k\) has the same or opposite relation to \(p\) (as far as its quadratic residue or non-residue is concerned) as \(+k\), depending on whether \(p\) is of the form \(4n+1\) or \(4n+3\). In the former case, \(-1\) is a residue, while in the latter case, it is a non-residue of \(p\).

VIII. We will transform the formula given in VI as follows. Using III, we have
\[\left[\frac{(p-1)k}{p}\right] \equiv k-1-\left[\frac{k}{p}\right], \left[\frac{(p-3)k}{p}\right]=k-1-\left[\frac{3k}{p}\right], \left[\frac{(p-5)k}{p}\right]=k-1-\left[\frac{5k}{p}\right] \ldots\]
Applying these substitutions to the \(\frac{p \mp 1}{4}\)th terms in the last series expression, we obtain

\textit{Firstly}, if \(p\) is of the form \(4n+1\), then
\[\begin{aligned}
(k, p)= & \frac{1}{4}(k-1)(p-1) \\
&-2\left\{\left[\frac{k}{p}\right]+\left[\frac{3k}{p}\right]+\left[\frac{5k}{p}\right] \ldots+\left[\frac{\frac{1}{2}(p-3)k}{p} \right]\right\} \\
&-\left\{\left[\frac{k}{p}\right]+\left[\frac{2k}{p}\right]+\left[\frac{3k}{p}\right] \ldots+\left[\frac{\frac{1}{2}(p-1)k}{p}\right]\right\}
\end{aligned}\]

\textit{Secondly}, if \(p\) is of the form \(4n+3\), then
\[\begin{aligned}
(k, p)=&\frac{1}{4}(k-1)(p+1) \\
&-2\left\{\left[\frac{k}{p}\right]+\left[\frac{3k}{p}\right]+\left[\frac{5k}{p}\right] \ldots+\left[\frac{\frac{1}{2}(p-1)k}{p}\right]\right\} \\
&-\left\{\left[\frac{k}{p}\right]+\left[\frac{2k}{p}\right]+\left[\frac{3k}{p}\right] \ldots+\left[\frac{\frac{1}{2}(p-1)k}{p}\right]\right\}
\end{aligned}\]

IX. For the special case \(k=+2\), from the formulas given above we get \((2, p)=\frac{1}{4}(p \mp 1)\), with the sign being taken as \(+\) or \(-\) depending on whether \(p\) is of the form \(4n+1\) or \(4n+3\). Thus, \((2, p)\) is even and \(2 \| p\), whenever \(p\) is of the form \(8n+1\) or \(8n+7\); on the contrary, \((2, p)\) is odd and \(2 \nmid p\), whenever \(p\) is of the form \(8n+3\) or \(8n+5\).
%

\subsection*{5.}

\textsc{Theorem.} \textit{Let \(x\) be a positive non-integer quantity, such that no multiple of \(x\), \(2x\), \(3x \ldots\) up to \(nx\) is an integer; let \([nx]=h\), whence it is easily concluded that no integer can be found among the multiples of the reciprocal quantity \(\frac{1}{x}\), \(\frac{2}{x}\), \(\frac{3}{x} \ldots\) up to \(\frac{h}{x}\). Then I say that}
\[\left.\begin{array}{l} 
\phantom{+}[x]+[2x]+[3x]\dots+[nx] \\
+\left[\frac{1}{x}\right]+\left[\frac{2}{x}\right]+\left[\frac{3}{x}\right] \ldots+\left[\frac{h}{x}\right]
\end{array}\right\}=nh\]
 
\textit{Proof.} Let \(\Omega\) represent the series \([x]+[2x]+[3x] \ldots+[nx]\). Then the terms up to \(\left[\frac{1}{x}\right]^{\text{th}}\) term inclusively are clearly all \(=0\); the terms up to \(\left[\frac{2}{x}\right]^{\text{th}}\) term inclusively are all \(=1\); the terms up to \(\left[\frac{3}{x}\right]^{\text{th}}\) term inclusively are all \(=2\) and so on. Hence we have
\[\left.\begin{array}{rl}
{\Omega}&= 0 \times\left[\frac{1}{x}\right] \\
& +1 \times\left\{\left[\frac{2}{x}\right]-\left[\frac{1}{x}\right]\right\} \\
& +2 \times\left\{\left[\frac{3}{x}\right]-\left[\frac{2}{x}\right]\right\} \\
& +3 \times\left\{\left[\frac{4}{x}\right]-\left[\frac{3}{x}\right]\right\} \\
&\qquad \text{etc.} \\
& +(h-1)\left\{\left[\frac{h}{x}\right]-\left[\frac{h-1}{x}\right]\right\} \\
& +h\left\{n-\left[\frac{h}{x}\right]\right\}
\end{array}\right\}=hn-\left[\frac{1}{x}\right]-\left[\frac{2}{x}\right]-\left[\frac{3}{x}\right] \ldots-\left[\frac{h}{x}\right]\]
Q. E. D.
%

\subsection*{6.}
 
\textsc{Theorem.} \textit{Let \(k, p\) be any odd positive integers that are prime to each other. Then}
\[\left.\begin{array}{l} 
\phantom{+} {\left[\frac{k}{p}\right]+\left[\frac{2 k}{p}\right]+\left[\frac{3 k}{p}\right] \ldots+\left[\frac{\frac{1}{2}(p-1) k}{p}\right]} \\
+ {\left[\frac{p}{k}\right]+\left[\frac{2 p}{k}\right]+\left[\frac{3 p}{k}\right] \ldots+\left[\frac{\frac{1}{2}(k-1) p}{k}\right]}
\end{array}\right\}=\frac{1}{4}(k-1)(p-1) .\]
 
\textit{Proof.} Suppose, for the sake of argument, that \(k<p\). Then \(\frac{\frac{1}{2}(p-1) k}{p}\) is smaller than \(\frac{1}{2} k\), but larger than \(\frac{1}{2}(k-1)\), so \(\left[\frac{\frac{1}{2}(p-1) k}{p}\right]=\frac{1}{2}(k-1)\). Hence, it is clear that the current theorem follows immediately from the previous theorem by assuming there \( \frac{k}{p}=x\), \( \frac{1}{2}(p-1)=n\), and therefore \( \frac{1}{2}(k-1)=h\). \clearpage\noindent% 8
%

However, it can be demonstrated in a similar manner, if \(k\) is an \emph{even} number and \(p\) is a prime number, that
\[\left.\begin{array}{l}
 \phantom{+}\left[\frac{k}{p}\right]+\left[\frac{2 k}{p}\right]+\left[\frac{3 k}{p}\right] \ldots+\left[ \frac{\frac{1}{2}(p-1) k}{p}\right] \\
+\left[\frac{p}{k}\right]+\left[\frac{2 p}{k}\right]+\left[\frac{3 p}{k}\right] \ldots+\left[\frac{\frac{1}{2} k p}{k}\right] \end{array} \right\}=\frac{1}{4} k(p-1)\]
However, we do not dwell on this proposition unnecessarily for our purpose.
%

\subsection*{7.}

Now, from the combination of the theorem mentioned above with Proposition VIII, Article 4, the fundamental theorem immediately follows. Indeed, denoting \(k\) and \(p\) as any two unequal positive prime numbers, and letting
\[\begin{aligned}
(k, p)+\left[\frac{k}{p}\right]+\left[\frac{2 k}{p}\right]+\left[\frac{3 k}{p}\right] \ldots+\left[\frac{\frac{1}{2} (p-1) k}{p}\right] &=L\\
(p, k)+\left[\frac{p}{k}\right]+\left[\frac{2 p}{k}\right]+\left[\frac{3 p}{k}\right] \ldots+\left[\frac{\frac{1}{2} (k-1) p}{k}\right] &= M
\end{aligned}\]
by Proposition VIII, Article 4, it is clear that \(L\) and \(M\) are always even. But by Theorem, Article 6, we have
\[L+M=(k, p)+(p, k)+\frac{1}{4}(k-1)(p-1)\]
Therefore, when \(\frac{1}{4}(k-1)(p-1)\) turns out to be even, which occurs if either both \(k\) and \(p\) are of the form \(4n+1\) or one of them is of the form \(4n+1\), it is necessary that both \((k, p)\) and \((p, k)\) are even or both are odd. On the other hand, when \(\frac{1}{4}(k-1)(p-1)\) is odd, which happens if both \(k\) and \(p\) are of the form \(4n+3\), it is necessary that one of the numbers \((k, p)\) and \((p, k)\) is even and the other is odd. In the former case, therefore, the relationship of \(k\) to \(p\) and the relationship of \(p\) to \(k\) (in terms of whether one is the residue or non-residue of the other) will be identical, while in the latter case they will be opposite.

\qquad Q. E. D.
%

\begin{center}
{\large SUMMATION }\\[2\baselineskip]
{\LARGE OF CERTAIN SERIES}\\[2\baselineskip]
{\Large SINGULAR}\\[4\baselineskip]
{\scriptsize BY}\\[1.5\baselineskip]
{CARL FRIEDRICH GAUSS}\\[1.5\baselineskip]
{\scriptsize PRESENTED TO THE SOCIETY ON AUGUST 24, 1808}\\[4\baselineskip]
\rule{4in}{0.5pt}\\[0.5\baselineskip]
{\scriptsize Commentaries of the Royal Society of Sciences of Gottingen. Vol. \textsc{i}.\\
Gottingen, 1811}\\
\rule{4in}{0.5pt}
\end{center}
\clearpage\noindent% 10
\;
\clearpage\noindent% 11
\section*{SUMMATION \\[\baselineskip] 
{\large OF CERTAIN SINGULAR SERIES.}\\[0.5\baselineskip]
\rule{0.75in}{0.5pt}\\}
%

\subsection*{1.}

Among the remarkable truths to which the theory of the division of the circle has opened the way, summation claims for itself no insignificant place. Proposed by Gauss in Disquisitiones Arithmeticae, Article 356, it is not only remarkable for its particular elegance and remarkable fertility, which will provide further opportunities for explanations in the future, but also because its rigorous and complete demonstration is hampered by no ordinary difficulties. These difficulties should certainly have been expected less, since they pertain not so much to the theorem itself, but rather to a limitation of the theorem, which once neglected permits an immediate demonstration and is easily derived from the theory explained in that work. The theorem is presented there in the following form. Assuming \(n\) to be a prime number, denoting indefinitely all the quadratic residues of \(n\) between the limits 1 and \(n-1\) inclusive by \(a\), and denoting all the non-residues between the same limits by \(b\), and finally denoting by \(\omega\) the arc \(\frac{360^{\circ}}{n}\), and by \(k\) any fixed integer not divisible by \(n\), we have

I. for the value of \(n\) which is of the form \(4m+1\),
\[\begin{aligned}
\Sigma \cos a k \omega & =-\frac{1}{2} \pm \frac{1}{2} \sqrt{n} \\
\Sigma \cos b k \omega & =-\frac{1}{2} \mp \frac{1}{2} \sqrt{n}, \text{ and therefore } \\
\Sigma \cos a k \omega -\Sigma \cos b k \omega & = \pm \sqrt{n} \\
\Sigma \sin a k \omega & =0 \\
\Sigma \sin b k \omega & =0
\end{aligned}\]\clearpage\noindent% 12

II. for the value of \(n\) which is of the form \(4m+3\),
\[\begin{aligned}
\Sigma \cos a k \omega & =-\frac{1}{2} \\
\Sigma \cos b k \omega & =-\frac{1}{2} \\
\Sigma \sin a k \omega & = \pm \frac{1}{2} \sqrt{n} \\
\Sigma \sin b k \omega & =\mp \frac{1}{2} \sqrt{n} \\
\Sigma \sin a k \omega -\Sigma \sin b k \omega & = \pm \sqrt{n}
\end{aligned}\]

These sums have been demonstrated with all rigor, and the only remaining difficulty is in determining the \textit{sign} to be assigned to the radical quantity. It can easily be shown that this sign depends on the number \(k\), which must always have the \textit{same} sign for all values of \(k\) that are quadratic residues of \(n\), and conversely must have the opposite sign for all values of \(k\) that are non-residues of \(n\). Therefore, the whole matter revolves around the value \(k=1\), and it is evident that as soon as the sign for this value is known, the signs for all other values of \(k\) will immediately follow. But indeed, in this very question, which at first glance seems to be among the easier ones, we encounter unforeseen difficulties and the method with which we have made progress so far completely denies further help.
%

\subsection*{2.}
 
It will be not inappropriate, before we proceed further, to work out some examples of our summation by numerical calculation: however, it is appropriate to preface this with some general observations.
%

I. If in the case where \(n\) is a prime number of the form \(4m+1\), all quadratic residues of \(n\) lying indefinitely between 1 and \(\frac{1}{2}(n-1)\) (inclusive) are represented by \(a^{\prime}\), and all non-residues between the same limits are represented by \(b^{\prime}\), it follows that all \(n-a^{\prime}\) lie among the \(a\), and all \(n-b^{\prime}\) lie among the \(b\). Therefore, since all \(a^{\prime}\), \(b^{\prime}\), \(n-a^{\prime}\), \(n-b^{\prime}\) clearly exhaust the entire set of numbers 1, 2, 3, ..., \(n-1\), all \(a^{\prime}\) together with all \(n-a^{\prime}\) include all \(a\), and likewise all \(b^{\prime}\) together with all \(n-b^{\prime}\) include all \(b\). Hence we have\clearpage\noindent% 13
\[\begin{aligned}
\Sigma \cos ak\omega &= \Sigma \cos a^{\prime}k\omega + \Sigma \cos (n-a^{\prime})k\omega \\
\Sigma \cos bk\omega &= \Sigma \cos b^{\prime}k\omega + \Sigma \cos (n-b^{\prime})k\omega \\
\Sigma \sin ak\omega &= \Sigma \sin a^{\prime}k\omega + \Sigma \sin (n-a^{\prime})k\omega \\
\Sigma \sin bk\omega &= \Sigma \sin b^{\prime}k\omega + \Sigma \sin (n-b^{\prime})k\omega
\end{aligned}\]
Now since we have \(\cos (n-a^{\prime})k\omega = \cos a^{\prime}k\omega\), \(\cos (n-b^{\prime})k\omega = \cos b^{\prime}k\omega\), \(\sin (n-a^{\prime})k\omega = -\sin a^{\prime}k\omega\), \(\sin (n-b^{\prime})k\omega = -\sin b^{\prime}k\omega\), it is obvious that
\[\begin{aligned}
\Sigma \sin ak\omega &= \Sigma \sin a^{\prime}k\omega - \Sigma \sin a^{\prime}k\omega = 0 \\
\Sigma \sin bk\omega &= \Sigma \sin b^{\prime}k\omega - \Sigma \sin b^{\prime}k\omega = 0
\end{aligned}\]
The summation of cosines, on the other hand, takes on this form
\[\begin{aligned}
\Sigma \cos ak\omega &= 2 \Sigma \cos a^{\prime}k\omega \\
\Sigma \cos bk\omega &= 2 \Sigma \cos b^{\prime}k\omega
\end{aligned}\]
From which it follows that
\[\begin{aligned}
1 + 4\Sigma \cos a^{\prime}k\omega &= \pm \sqrt{n} \\
1 + 4\Sigma \cos b^{\prime}k\omega &= \mp \sqrt{n} \\
2 \sum \cos a^{\prime}k\omega - 2 \Sigma \cos b^{\prime}k\omega &= \pm \sqrt{n}
\end{aligned}\]
%

II. In this case, where \(n\) is of the form \(4m+3\), the complement of any residue \(a\) modulo \(n\) will be a non-residue, and the complement of any \(b\) will be a residue; therefore, all \(n-a\) will coincide with all \(b\), and all \(n-b\) will coincide with all \(a\). Hence we conclude
\[\sum \cos a k \omega=\sum \cos (n-b) k \omega=\sum \cos b k \omega\]
so since all \(a\) and \(b\) together cover all the numbers \(1,2,3, \ldots n-1\), it follows that \(\sum \cos a k \omega+\sum \cos b k \omega=\cos k \omega+\cos 2 k \omega+\cos 3 k \omega+\text{etc.}+\cos (n-1) k \omega=-1\), the summations
\[\begin{aligned}
& \sum \cos a k \omega=-\frac{1}{2} \\
& \sum \cos b k \omega=-\frac{1}{2}
\end{aligned}\]
are therefore obvious. Similarly,
\[\sum \sin a k \omega=\sum \sin (n-b) k \omega=-\sum \sin b k \omega\]\clearpage\noindent% 14
from which it is clear how the summations
\[\begin{aligned}
& 2\sum \sin a k \omega= \pm \sqrt{n} \\
& 2\sum \sin b k \omega= \mp \sqrt{n}
\end{aligned}\]
depend on each other.
%

\subsection*{3.}

Behold, now the numerical computation for some examples:

I. For \(n=5\), there is one value of \(a^{\prime}\), let \(a^{\prime}=1\), and one value of \(b^{\prime}\), let \(b^{\prime}=2\); and it is
\[\cos \omega=+0{,}3090169944 \qquad \qquad \cos 2 \omega=-0{,}8090169944\]
therefore \(1+4 \cos \omega=+\surd 5\), \(1+4 \cos 2 \omega=-\surd 5\).\\
%

II. For \(n=13\), there are three values of \(a^{\prime}\), namely \(1\), \(3\), \(4\), and an equal number of values of \(b^{\prime}\), namely \(2\), \(5\), \(6\), from which we compute
\[ \begin{array}{r} 
\cos \omega\phantom{1} = + 0{,}8854560257 \\ 
\cos 3\omega = + 0{,}1205366803 \\ 
\cos 4 \omega = - 0{,}3546048870 \\ 
\hline \text{Sum} = + 0{,}6513878190 \end{array} 
\qquad \qquad
\begin{array}{r} 
\cos 2\omega = + 0{,}5680647467 \\ 
\cos 3\omega = - 0{,}7485107482 \\ 
\cos 4 \omega = - 0{,}9709418174 \\ 
\hline \text{Sum} = - 1{,}1513878189 \end{array} \]
 
Hence \(1+4 \sum \cos a^{\prime} \omega=+\sqrt{13}\), \(1+4 \Sigma \cos b^{\prime} \omega=-\sqrt{13}\).\\
%

III. For \(n=17\), we have four values of \(a^\prime\), namely 1, 2, 4, 8, and an equal number of values of \(b^\prime\), namely 3, 5, 6, 7. From this, the cosines are computed as follows:
\[\begin{array}{r}
\cos \omega\phantom{1}  =+0{,}9324722294 \\
\cos 2 \omega  =+0{,}7390089172 \\
\cos 4 \omega  =+0{,}0922683595 \\
\cos 8 \omega  =-0{,}9829730997 \\
\hline \text{Sum} =+0{,}7807764064 
\end{array}
\qquad \qquad
\begin{array}{r}
\cos 3 \omega  =+0{,}4457383558 \\
\cos 5 \omega  =-0{,}2736629901 \\
\cos 6 \omega  =-0{,}6026346364 \\
\cos 7 \omega  =-0{,}8502171357 \\
\hline \text{Sum} =-1{,}2807764065
\end{array}\]
Hence, \(1+4 \Sigma \cos a^{\prime} \omega=+\sqrt{17},1+4 \Sigma \cos b^{\prime} \omega=-\sqrt{17}\).
%

IV. For \(n=3\), there is a unique value of \(a\), namely \(a=1\), which corresponds to 
\[\sin \omega=+0.8660254038 \quad\quad \phantom{\sin \omega=+0.8660254038}\]
 
Hence, \(2 \sin \omega=+\sqrt{3}\).\\
%

V. For \(n=7\), there are three values of \(a\), namely 1, 2, 4: hence the sine values are
\[\begin{array}{rl}
 \sin \omega\phantom{1} =+0{,}7818314825 \phantom{,}&\\
\sin 2 \omega  =+0{,}9749279122 \phantom{,}&\\
\sin 4 \omega  =-0{,}4338837391 \phantom{,}&\\
\cline{1-1} \text{Sum}  =+1{,}3228756556,& \text{hence } 2 \Sigma \sin a \omega=+\sqrt{7}.
\end{array}\]
%

VI. For \(n=11\), the values of \(a\) itself are 1,3,4,5,9, for which the corresponding sines are
\[\begin{array}{rl}
\sin \omega\phantom{1}  =+0{,}5406408175 \phantom{,}&\\
\sin 3 \omega  =+0{,}9898214419 \phantom{,}&\\
\sin 4 \omega  =+0{,}7557495744 \phantom{,}&\\
\sin 5 \omega  =+0{,}2817325568 \phantom{,}&\\
\sin 9 \omega  =-0{,}9096319954 \phantom{,}&\\
\cline{1-1} \text{Sum} =+1{,}6583123952,& \text{and thus } 2 \Sigma \sin a \omega=+\sqrt{11}.
\end{array}\]
%

VII. For \(n=19\), the values of \(a\) are \(1,4,5,6,7,9,11,16,17\), for which the corresponding sines are
\[\begin{array}{rl}
 \sin \omega\phantom{11}=+0{,}3246994692 \phantom{,}&\\
 \sin 4 \omega\phantom{1}=+0{,}9694002659 \phantom{,}&\\
 \sin 5 \omega\phantom{1}=+0{,}9965844930 \phantom{,}&\\
\sin 6 \omega\phantom{1}=+0{,}9157733267 \phantom{,}&\\
 \sin 7 \omega\phantom{1}=+0{,}7357239107 \phantom{,}&\\
 \sin 9 \omega\phantom{1}=+0{,}1645945903 \phantom{,}&\\
 \sin 11 \omega=-0{,}4759473930 \phantom{,}&\\
 \sin 16 \omega=-0{,}8371664783 \phantom{,}&\\
 \sin 17 \omega=-0{,}6142127127 \phantom{,}&\\ 
 \cline{1-1} \text{Sum}=+2{,}1794494718,&\text{hence } 2 \Sigma \sin a \omega=+\sqrt{19}.
\end{array}\]\clearpage\noindent% 16
%

\subsection*{4.}

In all these examples, the radical quantity obtains a positive sign, and the same is easily confirmed for larger values of \(n=23\), \(n=29\), etc., from which a strong likelihood emerges that this holds generally. However, the proof of this phenomenon cannot be sought from the principles explained earlier and must be regarded as deserving of a thorough investigation. Therefore, the purpose of this commentary is to present the rigorous proof of this most elegant theorem, which has been attempted in vain in various ways for many years, and finally achieved successfully through careful and subtle considerations. At the same time, we will bring the theorem itself, with its elegance preserved or rather enhanced, to a much greater generality. Finally, in place of a conclusion, we will teach a remarkable and close connection between this summation and another very important arithmetic theorem. We hope that these investigations will not only be pleasing to geometers themselves, but that the methods by which all of this could be accomplished, and which can also be useful in other occasions, will be deemed worthy of their attention.
%

\subsection*{5.}

Our proof relies on the consideration of a specific type of progression, whose terms depend on the expressions \(a b\) given by
\[\frac{(1-x^{m})(1-x^{m-1})(1-x^{m-2}) \ldots(1-x^{m-\mu+1})}{(1-x)(1-x x)(1-x^{3}) \ldots (1-x^{\mu})}\]
For brevity, we will denote such a fraction by \((m, \mu)\), and we will first present some general observations about such functions.
%

I. Whenever \(m\) is a positive integer less than \(\mu\), the function \((m, \mu)\) clearly vanishes, since the numerator involves the factor \(1-x^{0}\). For \(m = \mu\), the factors in the numerator will be identical, but in the reverse order compared to the factors in the denominator, so we have \((\mu, \mu) = 1\): finally, in the case where \(m\) is a positive integer greater than \(\mu\), the following formulas hold
\[\begin{aligned}
& (\mu+1, \mu) = \frac{1-x^{\mu+1}}{1-x} = (\mu+1,1) \\
& (\mu+2, \mu) = \frac{(1-x^{\mu+2})(1-x^{\mu+1})}{(1-x)(1-x x)} = (\mu+2,2) \\
& (\mu+3, \mu) = \frac{(1-x^{\mu+3})(1-x^{\mu+2})(1-x^{\mu+1})}{(1-x)(1-xx)(1-x^{3})} = (\mu+3,3) \text{ etc.}
\end{aligned}\]\clearpage\noindent% 17
or more generally,
\[(m, \mu) = (m, m-\mu)\]
%

II. Furthermore, it is easily confirmed that
\[(m, \mu+1)=(m-1, \mu+1)+x^{m-\mu-1}(m-1, \mu)\]
which is the same as
\[\begin{aligned}
& (m-1, \mu+1)=(m-2, \mu+1)+x^{m-\mu-2}(m-2, \mu) \\
& (m-2, \mu+1)=(m-3, \mu+1)+x^{m-\mu-3}(m-3, \mu) \\
& (m-3, \mu+1)=(m-4, \mu+1)+x^{m-\mu-4}(m-4, \mu) \text{ etc.,}
\end{aligned}\]
which series can be continued until
\[\begin{aligned}
(\mu+2, \mu+1) & =(\mu+1, \mu+1)+x(\mu+1, \mu) \\
& =(\mu, \mu)+x(\mu+1, \mu)
\end{aligned}\]
since \(m\) is a positive integer greater than \(\mu+1\), it will be
\[\begin{aligned}
(m, \mu+1)=(\mu, \mu) & +x(\mu+1, \mu)+x x(\mu+2, \mu)+x^{3}(\mu+3, \mu)+\text{etc.} \\
& +x^{m-\mu-1}(m-1, \mu)
\end{aligned}\]
%

Hence it is evident that, if for any given value of \(\mu\) the function \(({m}, {\mu})\) is integrable for all positive integer values of \(m\), then the function \((m, \mu+1)\) must also be integrable. Therefore, since this assumption holds for \(\mu=1\), it will also hold for \(\mu=2\), and thus for \(\mu=3\) etc., i.e. generally for any positive integer value of \(m\) the function \((m, \mu)\) will be integrable, or the product
\[(1-x^{m})(1-x^{m-1})(1-x^{m-2}) \ldots(1-x^{m-\mu+1})\]
will be divisible by
\[(1-x)(1-x^{2})(1-x^{3}) \ldots(1-x^{\mu})\]
%

\subsection*{6.}

We will now consider two progressions which can both lead to our goal. The first progression is as follows

\[1-\frac{1-x^{m}}{1-x}+\frac{(1-x^{m})(1-x^{m-1})}{(1-x)(1-xx)}-\frac{(1-x^{m})(1-x^{m-1})(1-x^{m-2})}{(1-x)(1-xx)(1-x^{3})}+\ldots\]

or

\[1-(m, 1)+(m, 2)-(m, 3)+(m, 4)-\ldots\]

For brevity, we will denote this series by \(f(x, m)\). Immediately, it is obvious that when \(m\) is a positive integer, this series is \textit{interrupted} after its \({m+1}^{\text{st }}\) term (which is \(= \pm 1\)), and therefore in this case, the sum must be a finite integral function of \(x\). Furthermore, according to Article 5. II., it is clear that generally, for any value of \(m\), we have

\[\begin{aligned}
1 & =1 \\
-(m, 1) & =-(m-1,1)-x^{m-1} \\
+(m, 2) & =+(m-1,2)+x^{m-2}(m-1,1) \\
-(m, 3) & =-(m-1,3)-x^{m-3}(m-1,2) \ldots
\end{aligned}\]

and therefore

\[\begin{aligned}
f(x, m)=1-x^{m-1}&-(1-x^{m-2})(m-1,1)+(1-x^{m-3})(m-1,2) \\
&-(1-x^{m-4})(m-1.3)+\ldots
\end{aligned}\]

But it is clear that

\[\begin{aligned}
& (1-x^{m-2})(m-1,1)=(1-x^{m-1})(m-2,1) \\
& (1-x^{m-3})(m-1,2)=(1-x^{m-1})(m-2,2) \\
& (1-x^{m-4})(m-1,3)=(1-x^{m-1})(m-2,3) \ldots
\end{aligned}\]

from which we deduce the equation

\[f(x, m)=(1-x^{m-1}) f(x, m-2) \tag{1}\]
%

\subsection*{7.}
 
When \(m=0\), \(f(x, m)=1\). By the formula we have just found, we have
\[\begin{aligned}
& f(x, 2)=1-x \\
& f(x, 4)=(1-x)(1-x^{3}) \\
& f(x, 6)=(1-x)(1-x^{3})(1-x^{5}) \\
& f(x, 8)=(1-x)(1-x^{3})(1-x^{5})(1-x^{7}) \text{ etc.}
\end{aligned}\]
or more generally, for any even value of \(m\)
\[f(x, m)=(1-x)(1-x^{3})(1-x^{5}) \ldots (1-x^{m-1}) \tag{2}\] \clearpage\noindent% 19
%

Against the fact that for \(m=1\) we have \(f(x, m) = 0\), we will also have
\[\begin{aligned}
& f(x, 3) = 0 \\
& f(x, 5) = 0 \\
& f(x, 7) = 0 \text{ etc.}
\end{aligned}\]
or, in general, for any odd value of \(m\):
\[f(x, m) = 0\]
%

Moreover, the latter sum could have already been derived from the fact that in the progression
\[1-(m, 1)+(m, 2)-(m, 3)+\text{etc.}+(m, m-1)-(m, m),\]
the last term destroys the first, the penultimate destroys the second, etc.
%

\subsection*{8.}

For our purpose it suffices to consider the case where \(m\) is a positive odd integer. However, due to the uniqueness of the matter, it will not be regrettable to add a few remarks about the cases where \(m\) is fractional or negative. Clearly in these cases our series will not be interrupted, but will diverge to infinity. Moreover, it is easily seen that it diverges whenever the value of \(x\) itself is less than 1, so its summation should be restricted to values of \(x\) greater than 1.
%

According to formula [1] in article 6, we have
\[\begin{aligned}
& f(x,-2)=\frac{1}{1-\frac{1}{x}} \\
& f(x,-4)=\frac{1}{1-\frac{1}{x}} \cdot \frac{1}{1-\frac{1}{x^{3}}} \\
& f(x,-6)=\frac{1}{1-\frac{1}{x}} \cdot \frac{1}{1-\frac{1}{x^{3}}} \cdot \frac{1}{1-\frac{1}{x^{5}}} \text{ etc.}
\end{aligned}\]
so that the value of the function \(f(x, m)\) can also be assigned for a negative even value of \(m\) in finite terms. For the remaining values of \(m\), we will convert the function \(f(x, m)\) into an \textit{infinite product} using the following method.
%

As \(m\) approaches negative infinity, the function \(f(x, m)\) converges to\clearpage\noindent% 20
\[1+\frac{1}{x-1}+\frac{1}{x-1} \cdot \frac{1}{x x-1}+\frac{1}{x-1} \cdot \frac{1}{x x-1} \cdot \frac{1}{x^{3}-1}+\text{etc.}\]
Therefore, this series is equal to the infinite product
\[\frac{1}{1-\frac{1}{x}} \cdot \frac{1}{1-\frac{1}{x^{3}}} \cdot \frac{1}{1-\frac{1}{x^{5}}} \cdot \frac{1}{1-\frac{1}{x^{7}}} \text{etc. to infinity}\]
Moreover, since it is generally true that
\[f(x, m)=f(x, m-2 \lambda) . (1-x^{m-1})(1-x^{m-3})(1-x^{m-5}) \ldots (1-x^{m-2 \lambda+1})\]
we have
\[\begin{aligned}
f(x, m) & =f(x,-\infty) \cdot(1-x^{m-1})(1-x^{m-3})(1-x^{m-5}) \text{ etc. to infinity} \\
& =\frac{1-x^{m-1}}{1-x^{-1}} \cdot \frac{1-x^{m-3}}{1-x^{-3}} \cdot \frac{1-x^{m-5}}{1-x^{-5}} \cdot \frac{1-x^{m-7}}{1-x^{-7}} \text{ etc. to infinity}
\end{aligned}\]
which clearly converge to unity as the factors keep getting closer to unity.
%

The case \(m=-1\) deserves special attention, where we have
\[f(x,-1)=1+x^{-1}+x^{-3}+x^{-6}+x^{-10}+\text{etc.}\]
This series can be expressed as an infinite product
\[\frac{1-x^{-2}}{1-x^{-1}} \cdot \frac{1-x^{-2}}{1-x^{-3}} \cdot \frac{1-x^{-6}}{1-x^{-5}} \text{ etc.}\]
or, by replacing \(x\) with \(x^{-1}\), we have
\[1+x+x^{3}+x^{6}+\text{etc.}=\frac{1-x x}{1-x} \cdot \frac{1-x^{4}}{1-x^{3}} \cdot \frac{1-x^{6}}{1-x^{5}} \cdot \frac{1-x^{8}}{1-x^{7}} \text{ etc.}\]
This equality between two more intricate expressions, to which we will return on another occasion, is indeed very remarkable.
%

\subsection*{9.}

In the second place, we will consider this progression
\[1+x^{\frac{1}{2}} \frac{1-x^{m}}{1-x}+x \frac{(1-x^{m})}{(1-x)} \frac{(1-x^{m-1})}{(1-x x)}+x^{\frac{3}{2}} \frac{(1-x^{m})(1-x^{m-1})(1-x^{m-2})}{(1-x)(1-x x)(1-x^{3})}+\text{etc.}\]
or
\[1+x^{\frac{1}{2}}(m, 1)+x(m, 2)+x^{\frac{3}{2}}(m, 3)+x x(m, 4)+\text{etc.}\]
which we will denote by \({F}(x, m)\). We will restrict this discussion to the case where \(m\) is a positive integer, so that this series is always truncated at the term \(m+1^{\text{to}}\), which is \(=x^{\frac{1}{2} m}(m, m)\). Since
\[(m, m)=1, \quad(m, m-1)=(m, 1), \quad(m, m-2)=(m, 2) \text{ etc.}\]
the progression can also be expressed as:
\[F(x, m)=x^{\frac{1}{2} m}+x^{\frac{1}{2}(m-1)}(m, 1)+x^{\frac{1}{2}(m-2)}(m, 2)+x^{\frac{1}{2}(m-3)}(m, 3)+\text{etc.}\]
Hence we have
\[\begin{aligned}
(1+x^{\frac{1}{2} m+\frac{1}{2}}) F(x, m) =1&+x^{\frac{1}{2}}(m, 1)+x(m, 2)+x^{\frac{3}{2}}(m, 3)+\text{etc.} \\
& +x^{\frac{1}{2}} . x^{m}+x . x^{m-1}(m, 1)+x^{\frac{3}{2}} . x^{m-2}(m, 2)+\text{etc.}
\end{aligned}\]
Therefore, from (1),
\[\begin{aligned}
(m, 1)+x^{m} & =(m+1,1) \\
(m, 2)+x^{m-1}(m, 1) & =(m+1,2) \\
(m, 3)+x^{m-2}(m, 2) & =(m+1,3) \text{ etc.,}
\end{aligned}\]
it follows that
\[(1+x^{\frac{1}{2} m+\frac{1}{2}}) F(x, m)=F(x, m+1)\tag{3}\]
But \(F(x, 0)=1\): therefore we have
\[\begin{aligned}
& F(x, 1)=1+x^{\frac{1}{2}} \\
& F(x, 2)=(1+x^{\frac{1}{2}})(1+x) \\
& F(x, 3)=(1+x^{\frac{1}{2}})(1+x)(1+x^{3}) \text{ etc.,}
\end{aligned}\]
or generally
\[F(x, m)=(1+x^{\frac{1}{2}})(1+x)(1+x^{\frac{3}{2}}) \ldots(1+x^{\frac{1}{2} m})\tag{4}\]
%

\subsection*{10.}

With these preliminary discussions out of the way, let us now proceed closer to our goal. Since the squares \(1, 4, 9, \ldots, (\frac{1}{2}(n-1))^2\) are all incongruent to each other modulo \(n\), it is clear that their residues modulo \(n\) must be identical to the numbers \(a\), and therefore
\[\begin{aligned}
&\Sigma \cos a k \omega=\cos k \omega+\cos 4 k \omega+\cos 9 k \omega+\text{etc.}+\cos (\frac{1}{2}(n-1))^2 k \omega \\
&\Sigma \sin a k \omega=\sin k \omega+\sin 4 k \omega+\sin 9 k \omega+\text{etc.}+\sin (\frac{1}{2}(n-1))^2 k \omega
\end{aligned}\]
Similarly, since the same squares \(1, 4, 9, \ldots, (\frac{1}{2}(n-1))^2\) are congruent to \((\frac{1}{2}(n+1))^2, (\frac{1}{2}(n+3))^2, (\frac{1}{2}(n+5))^2, \ldots, (n-1)^2\) in reverse order, it will also be
\[\begin{aligned}
&\Sigma \cos a k \omega=\cos (\frac{1}{2}(n+1))^2 k \omega+\cos (\frac{1}{2}(n+3))^2 k \omega+\text{etc.}+\cos (n-1)^2 k \omega \\
&\Sigma \sin a k \omega=\sin (\frac{1}{2}(n+1))^2 k \omega+\sin (\frac{1}{2}(n+3))^2 k \omega+\text{etc.}+\sin (n-1)^2 k \omega
\end{aligned}\]
Therefore, assuming
\[\begin{alignedat}[t]{2}
&T=1+\cos k \omega+\cos 4 k \omega+\cos 9 k \omega+\text{etc.}+\cos (n-1)^2 k \omega \\
& U= \sin k \omega+\sin 4 k \omega+\sin 9 k \omega+\text{etc.}+\sin (n-1)^2 k \omega
\end{alignedat}\]
it will be
\[\begin{aligned}
1+2 \Sigma \cos a k \omega & =T \\
2 \Sigma \sin a k \omega & =U
\end{aligned}\]
Hence it is clear that the sums, as proposed in article 1, depend on the summation of the series \(T\) and \(U\). Therefore, by omitting these, we will adapt our investigation and complete it in such a general way that it encompasses not only prime values of \(n\), but also all composite values. However, we assume that the number \(k\) is prime to \(n\): for there is no difficulty in reducing the case where \(k\) and \(n\) have a common divisor to this case.
%

\subsection*{11.}
Let us denote the imaginary quantity $\sqrt{-1}$ by $i$, and let us assume
\[\cos k \omega+i \sin k \omega=r\]
where $r^{n}=1$, or $r$ is a root of the equation $r^{n}-1=0$. It is easy to see that all numbers $k$, $2 k$, $3 k, \ldots, (n-1) k$ that are not divisible by $n$ and are incongruent to each other modulo $n$: therefore, the powers of $r$
\[1, r, r^{2}, r^{3} \ldots r^{n-1}\]
will all be distinct, and each of them will satisfy the equation $x^{n}-1=0$. For this reason, all these powers will represent all the roots of the equation $x^{n}-1=0$.
%

These conclusions would be invalid if \(k\) had a common divisor with \(n\). For if \(\nu\) were such a common divisor, then \(k \cdot \frac{n}{\nu}\) would be divisible by \(n\), and hence it would be a power less than \(r^{n}\), say \(r^{\frac{n}{\nu}}\), equal to the unit. In this case, therefore, the powers of \(r\) up to \(\frac{n}{\nu}\) will exhibit the roots of the equation \(x^{n}-1=0\), and indeed they will present precisely as many distinct roots as there are \(k\), \(n\) for which \(\nu\) is the \textit{greatest} common divisor. In our case, where \(k\) and \(n\) are assumed to be prime to each other, \(r\) can conveniently be called the \textit{primitive root} of the equation \(x^{n}-1=0\): on the other hand, in the other case, where \(k\) and \(n\) have the (greatest) common divisor \(\nu\), \(r\) would be called the \textit{improper root} of that equation, but then \(r\) would be the primitive root of the equation \(x^{\frac{n}{\nu}}-1=0\). The simplest improper root is the unit, and in the case where \(n\) is a prime number, there are no other improper roots at all.
%

\subsection*{12.}

If we now assume 
\[W=1+r+r^{4}+r^{9}+\text{etc.}+r^{(n-1)^{2}}\]
it is clear that \(W=T+i U\) can be done, so that \(T\) is the real part of \(W\) itself, and \(U\) arises from the imaginary part of \(W\) by suppressing the factor \(i\). The whole matter is therefore reduced to finding the sum \(W\): for this purpose either the series considered in article 6, or the one we have shown how to sum in article 9, can be used, although the former is less suitable in the case where \(n\) is an even number. Nevertheless, we hope that it will be pleasing to the readers if we treat the case where \(n\) is odd according to the double method.
%

Let us assume first that \(n\) is an odd number, \(r\) represents the proper root of the equation \(x^{n}-1=0\) of any kind, and in the function \(f(x, m)\) it is stated that \(x=r\) and \(m=n-1\). Hence it is clear that we have
\[\begin{alignedat}[t]{3}
& \frac{1-x^{m}}{1-x}&=\frac{1-r^{-1}}{1-r}&=-r^{-1} \\
& \frac{1-x^{m-1}}{1-x x}&=\frac{1-r^{-2}}{1-r r}&=-r^{-2} \\
& \frac{1-x^{m-2}}{1-x^{3}}&=\frac{1-r^{-3}}{1-r^{3}}&=-r^{-3} \text{ etc.}
\end{alignedat}\]
up to
\[\begin{alignedat}[t]{3}
&\frac{1-x}{1-x^{m}}&=\frac{1-r^{-m}}{1-r^{m}}&=-r^{-m}\phantom{\text{etc.}}
\end{alignedat}\]
(It will not be superfluous to mention that these equations are valid only to the extent that \(r\) is assumed to be a proper root: for if \(r\) were an improper root, the numerator and denominator of some of those fractions would simultaneously vanish, and thus the fractions would become indeterminate).
%

From here, we derive the following equation:

\[
\begin{aligned}
f(r, n-1) & =1+r^{-1}+r^{-3}+r^{-6}+\text{etc.}+r^{-\frac{1}{2}(n-1) n} \\
& =(1-r)(1-r^{3})(1-r^{5}) \ldots(1-r^{n-2})
\end{aligned}
\]
%

The same equation will still hold if we substitute \(r\) with \(r^{\lambda}\), where \(\lambda\) is any arbitrary integer relatively prime to \(n\). Thus, \(r^{\lambda}\) will also be a primitive root of the equation \(x^{n}-1=0\). Let us write \(r^{n-2}\) instead of \(r\), or equivalently \(r^{-2}\), and we have
\[1+r^{2}+r^{6}+r^{12}+\text{etc.}+r^{(n-1) n}=(1-r^{-2})(1-r^{-6})(1-r^{-10}) \ldots(1-r^{-2(n-2)})\]
Now, let's multiply both sides of this equation by
\[r \cdot r^{3} \cdot r^{5} \ldots \cdot r^{(n-2)}=r^{\frac{1}{4}(n-1)^{2}}\]
and we get, due to
\[\begin{alignedat}[t]{4}
r^{2+\frac{1}{4}(n-1)^{2}} & =r^{\frac{1}{4}(n-3)^{2}},\quad & r^{(n-1) n+\frac{1}{4}(n-1)^{2}}&=r^{\frac{1}{4}(n+1)^{2}} \\
r^{6+\frac{1}{4}(n-1)^{2}} & =r^{\frac{1}{4}(n-5)^{2}},\quad & r^{(n-2)(n-1)+\frac{1}{4}(n-1)^{2}}&=r^{\frac{1}{4}(n+3)^{2}} \\
r^{12+\frac{1}{4}(n-1)^{2}} & =r^{\frac{1}{4}(n-7)^{2}},\quad & r^{(n-3)(n-2)+\frac{1}{4}(n-1)^{2}}&=r^{\frac{1}{4}(n+5)^{2}} \text{ etc.}
\end{alignedat}\]
the following equation
\[\begin{aligned}
 r^{\frac{1}{4}(n-1)^{2}}&+r^{\frac{1}{4}(n-3)^{2}}+r^{\frac{1}{4}(n-5)^{2}}+\text{etc.}+r+1 \\
+r^{\frac{1}{4}(n+1)^{2}}&+r^{\frac{1}{4}(n+3)^{2}}+r^{\frac{1}{4}(n+5)^{2}}+\text{etc.}+r^{\frac{1}{4}(2 n-2)^{2}} \\
&=(r-r^{-1})(r^{3}-r^{-3})(r^{5}-r^{-5}) \ldots(r^{n-2}-r^{-n+2})
\end{aligned}\]
or, rearranging the terms on the left-hand side,
\[1+r+r^{4}+\text{etc.}+r^{(n-1)^{2}}=(r-r^{-1})(r^{3}-r^{-3}) \ldots(r^{n-2}-r^{-n+2}) \tag{5}\]
%

\subsection*{13.}

The factors of the second term of the equation [5] can also be represented as follows:
\[\begin{aligned}
r-r^{-1} & =-(r^{n-1}-r^{-n+1}) \\
r^{3}-r^{-3} & =-(r^{n-3}-r^{-n+3}) \\
r^{5}-r^{-5} & =-(r^{n-5}-r^{-n+5}) \text{ etc.}
\end{aligned}\]
up to
\[r^{n-2}-r^{-n+2}=-(r^{2}-r^{-2})\phantom{\text{ etc.}}\]
Thus, this equation takes the following form:

\[W=(-1)^{\frac{1}{2}(n-1)}(r^{2}-r^{-2})(r^{4}-r^{-4})(r^{6}-r^{-6}) \ldots (r^{n-1}-r^{-n+1})\]

Multiplying this equation by [5] in its primitive form, we obtain:

\[W^{2}=(-1)^{\frac{1}{2}(n-1)}(r-r^{-1})(r^{2}-r^{-2})(r^{3}-r^{-3}) \ldots (r^{n-1}-r^{-n+1})\]

where \((-1)^{\frac{1}{2}(n-1)}\) is either \(+1\) or \(-1\), depending on whether \(n\) is of the form \(4 \mu+1\) or \(4 \mu+3\). Therefore,

\[W^{2}= \pm r^{\frac{1}{2} n(n-1)}(1-r^{-2})(1-r^{-4})(1-r^{-6}) \ldots(1-r^{-2(n-1)})\]

But it is clear that \(r^{-2}\), \(r^{-4}\), \(r^{-6} \ldots r^{-2 n+2}\) represent all the roots of the equation \(x^{n}-1=0\), except for the root \(x=1\). Therefore, the following equation must hold:

\[(x-r^{-2})(x-r^{-4})(x-r^{-6}) \ldots(x-r^{-2 n+2})=x^{n-1}+x^{n-2}+x^{n-3}+\text{etc.}+x+1\]

Therefore, by letting \(x=1\), we have:

\[(1-r^{-2})(1-r^{-4})(1-r^{-6}) \ldots(1-r^{-2 n+2})=n\]

And since it is evident that \(r^{\frac{1}{2} n(n-1)}=1\), our equation becomes:

\[W^{2}= \pm n \tag{6}\]

In the case where \(n\) is of the form \(4 \mu+1\), we have:

\[W= \pm \sqrt{n}, \text{ and therefore } T= \pm \sqrt{n}, \quad U=0\]

On the other hand, in the other case where \(n\) is of the form \(4 \mu+3\), we have:

\[W= \pm i \sqrt{n}, \text{ therefore } T=0, \quad U= \pm \sqrt{n}\]
%

\subsection*{14.}

The method assigns only the absolute value of the aggregated \(T\) and \(U\), and leaves undecided whether to establish that \(T\) is equal to \(+\sqrt{n}\) in the prior case and \(U\) is equal to \(-\sqrt{n}\) in the posterior case, or whether to establish the opposite. However, at least for the case where \(k=1\), it will be possible to decide in the following way based on equation [5]. Since it is, for \(k=1\),
\[\begin{aligned}
r-r^{-1} & =2 i \sin \omega \\
r^{3}-r^{-3} & =2 i \sin 3 \omega \\
r^{5}-r^{-5} & =2 i \sin 5 \omega \text{ etc., }
\end{aligned}\]
this equation is transformed into
\[W=(2 i)^{\frac{1}{2}(n-1)} \sin \omega \sin 3 \omega \sin 5 \omega \ldots \sin (n-2) \omega\]
Now, in the case where \(n\) is of the form \(4 \mu+1\), there are found in the series of odd numbers
\[1,3,5,7 \ldots \frac{1}{2}(n-3), \frac{1}{2}(n+1) \ldots(n-2)\]
\(\frac{1}{4}(n-1)\) of them, which are smaller than \(\frac{1}{2} n\), and these clearly correspond to positive sines. On the other hand, the remaining \(\frac{1}{4}(n-1)\) will be larger than \(\frac{1}{2} n\), and these will correspond to negative sines. Therefore, the product of all the sines must be equal to the product of a positive quantity in the multiplier \((-1)^{\frac{1}{4}(n-1)}\), and thus \(W\) will be equal to the product of a positive real quantity in \(i^{n-1}\) or in 1, since \(i^{4}=1\) and \(n-1\) is divisible by 4: i.e. the quantity \(W\) will be a positive real quantity, from which it must necessarily follow that
\[W=+\sqrt{n}, \quad T=+\sqrt{n}\]
%

In the second case, where \(n\) is of the form \(4 \mu+3\) in the series of odd numbers
\[1,3,5,7 \ldots \frac{1}{2}(n-1), \frac{1}{2}(n+3) \ldots(n-2)\]
the first \(\frac{1}{4}(n+1)\) will be smaller than \(\frac{1}{2} n\), the rest \(\frac{1}{4}(n-3)\) on the other hand will be larger. Hence, among the sines of the arcs \(\omega, 3 \omega, 5 \omega \ldots(n-2) \omega\), \(\frac{1}{4}(n-3)\) will be negative, and thus \(W\) will be the product of \(i^{\frac{1}{2}(n-1)}\) with a positive real quantity in \((-1)^{\frac{1}{4}(n-3)}\); the third factor is \(=i^{\frac{1}{2}(n-3)}\), which when combined with the first, gives \(i^{n-2}=i\), since \(i^{n-3}=1\). Therefore it is necessary that
\[W=+i \sqrt{n}, \text{ and } U=+\sqrt{n}\]
%

\subsection*{15.}

Now we will show how the same conclusions considered in article 9 can be deduced from the progression. Let us write in Eq. [4] for \(x^{\frac{1}{2}}\), \(-y^{-1}\), and we have

\[1-y^{-1} \frac{1-y^{-2 m}}{1-y^{-2}}+y^{-2} \frac{(1-y^{-2 m})(1-y^{-2 m+2})}{(1-y^{-2})(1-y^{-4})}-y^{-3} \frac{(1-y^{-2 m})(1-y^{-2 m+2})(1-y^{-2 m+2})}{(1-y^{-2})(1-y^{-4})(1-y^{-6})}+\text{etc.}\]

up to the term \(m+1^{\text{st}}\)

\[=(1-y^{-1})(1+y^{-2})(1-y^{-3})(1+y^{-4}) \ldots(1 \pm y^{-m})\tag{7}\]

If we take \(y\) to be the proper root of the equation \(y^{n}-1=0\), say \(r\), and at the same time we set \(m=n-1\), then we have

\[\begin{aligned}
& \frac{1-y^{-2 m}}{1-y^{-2}}=\frac{1-r^{2}}{1-r^{-2}}=-r^{2} \\
& \frac{1-y^{-2 m+2}}{1-y^{-4}}=\frac{1-r^{4}}{1-r^{-4}}=-r^{4} \\
& \frac{1-y^{-2 m+4}}{1-y^{-6}}=\frac{1-r^{6}}{1-r^{-6}}=-r^{6} \text{ etc.}
\end{aligned}\]

up to

\[\frac{1-y^{-2}}{1-y^{-2 m}}=\frac{1-r^{2 n-2}}{1-r^{-2 n+2}}=-r^{2 n-2}\]

where it should be noted that no denominators \(1-r^{-2}\), \(1-r^{-4}\) etc. become zero. Hence equation [7] takes this form:

\[1+r+r^{4}+r^{9}+\text{etc.}+r^{(n-1)^{2}}=(1-r^{-1})(1+r^{-2})(1-r^{-3}) \ldots(1+r^{-n+1})\]

Multiplying the first term by the last term, the second term by the second-to-last term, etc., of the second member of this equation, we have

\[\begin{aligned}
& (1-r^{-1})(1+r^{-n+1})=r-r^{-1} \\
& (1+r^{-2})(1-r^{-n+2})=r^{n-2}-r^{-n+2} \\
& (1-r^{-3})(1+r^{-n+3})=r^{3}-r^{-3} \\
& (1+r^{-4})(1-r^{-n+4})=r^{n-4}-r^{-n+4} \text{ etc.}
\end{aligned}\]
  
From these products, it is easily seen that the product
  
\[(r-r^{-1})(r^{3}-r^{-3})(r^{5}-r^{-5}) \ldots(r^{n-4}-r^{-n+4})(r^{n-2}-r^{-n+2})\]
  
will be
  
\[=1+r+r^{4}+r^{9}+\text{etc.}+r^{(n-1)^{2}}=W\]

This equation is identical to equation [5] in article 12 from the first derived progression, and the remaining arguments can be constructed in the same way, as in articles 13 and 14.
%

\subsection*{16.}

We move on to another case, where \(n\) is an even number. Let \(n\) first take the form \(4 \mu+2\), or equivalently an odd even number. It is clear that the numbers \(\frac{1}{4} n n\), \((\frac{1}{2} n+1)^{2}-1\), \((\frac{1}{2} n+2)^{2}-4\), etc., or more generally \((\frac{1}{2} n+\lambda)^{2}-\lambda \lambda\) by dividing by \(\frac{1}{2} n\), yield odd quotients, thus congruent to \(\frac{1}{2} n\) modulo \(n\). Hence, if \(r\) is a proper root of the equation \(x^{n}-1=0\), and thus \(r^{\frac{1}{2} n}=-1\), it follows that
\[\begin{aligned}
r^{(\frac{1}{2} n)^{2}} & =-1 \\
r^{(\frac{1}{2} n+1)^{2}} & =-r \\
r^{(\frac{1}{2} n+2)^2} & =-r^{4} \\
r^{(\frac{1}{2} n+3)^{2}} & =-r^{9} \text{ etc.}
\end{aligned}\]
Hence, in the progression
\[1+r+r^{4}+r^{9}+\text{etc.}+r^{(n-1)^{2}}\]
the term \(r^{(\frac{1}{2} n)^{2}}\) destroys the first term, the following term destroys the second term, etc., so we will have
\[W=0, \quad T=0, \quad U=0\]
%

\subsection*{17.}

There remains the case where \(n\) is of the form \(4 \mu\), or equivalently even. Here, in general, \((\frac{1}{2} n+\lambda)^{2}-\lambda \lambda\) will be divisible by \(n\), and therefore
\[r^{(\frac{1}{2} n+\lambda)^{2}}=r^{\lambda \lambda}\]
Hence, in the series
\[1+r+r^{4}+r^{9}+\text{etc.}+r^{(n-1)^{2}}\]
the term \(r^{(\frac{1}{2} n)^{2}}\) will be the first term, the following term will be the second term, and so on, such that
\[W=2(1+r+r^{4}+r^{9}+\text{etc.}+r^{(\frac{1}{2} n-1)^{2}})\]
%

Let us assume, as stated in equation [7] article 15, that \(m=\frac{1}{2} n-1\), and let \(y\) be the positive root of the equation \(y^{n}-1=0\), let's say \(r\). Then, just like in article 15, the equation takes the following form:
\[1+r+r^{4}+\text{etc.}+r^{(\frac{1}{2} n-1)^{2}}=(1-r^{-1})(1+r^{-2})(1-r^{-3}) \ldots(1-r^{-\frac{1}{2} n+1})\]\clearpage\noindent% 29
or
\[W=2(1-r^{-1})(1+r^{-2})(1-r^{-3})(1+r^{-4}) \ldots(1-r^{-\frac{1}{2} n+1}) \tag{8}\]
%

Furthermore, since \(r^{\frac{1}{2} n}=-1\), we have
\[\begin{aligned}
& 1+r^{-2}=-r^{\frac{1}{2} n-2}(1-r^{-\frac{1}{2} n+2}) \\
& 1+r^{-4}=-r^{\frac{1}{2} n-4}(1-r^{-\frac{1}{2} n+4}) \\
& 1+r^{-6}=-r^{\frac{1}{2} n-6}(1-r^{-\frac{1}{2} n+6}) \text{ etc.}
\end{aligned}\]
and the product of the factors \(-r^{\frac{1}{2} n-2}\), \(-r^{\frac{1}{2} n-4}\), \(-r^{\frac{1}{2} n-6}\) etc. up to \(-r^{2}\) becomes \(=(-1)^{\frac{1}{4} n-1} r^{\frac{1}{16} n n-\frac{1}{4} n}\). The previous equation can also be expressed as
\[W=2(-1)^{\frac{1}{4} n-1} r^{\frac{1}{16} n n-\frac{1}{4} n}(1-r^{-1})(1-r^{-2})(1-r^{-3})(1-r^{-4}) \ldots(1-r^{-\frac{1}{2} n+1})\]
%

When we have
\[\begin{aligned}
& 1-r^{-1}=-r^{-1}(1-r^{-n+1}) \\
& 1-r^{-2}=-r^{-2}(1-r^{-n+2}) \\
& 1-r^{-3}=-r^{-3}(1-r^{-n+3}) \text{ etc.}
\end{aligned}\]
it will be
\[\begin{aligned}
 (1-r^{-1})&(1-r^{-2})(1-r^{-3}) \ldots(1-r^{-\frac{1}{2} n+1}) \\
=&(-1)^{\frac{1}{2} n-1} r^{-\frac{1}{8} n n+\frac{1}{4} n}(1-r^{-\frac{1}{2} n-1})(1-r^{-\frac{1}{2} n-2})(1-r^{-\frac{1}{2} n-3}) \ldots(1-r^{-n+1})
\end{aligned}\]
and thus
\[W=2(-1)^{\frac{3}{4} n-2} r^{-\frac{1}{16} n n}(1-r^{-\frac{1}{2} n-1})(1-r^{-\frac{1}{2} n-2})(1-r^{-\frac{1}{2} n-3}) \ldots(1-r^{-n+1})\]
Multiplying this value of \(W\) by the previously found one, and adding the factor \(1-r^{-\frac{1}{2} n}\) on both sides, we get
\[(1-r^{-\frac{1}{2} n}) W^{2}=4(-1)^{n-3} r^{-\frac{1}{4} n}(1-r^{-1})(1-r^{-2})(1-r^{-3}) \ldots(1-r^{-n+1})\]
But we have
\[\begin{gathered}
1-r^{-\frac{1}{2} n}=2 \\
(-1)^{n-3}=-1 \\
r^{-\frac{1}{4} n}=-r^{\frac{1}{4} n} \\
(1-r^{-1})(1-r^{-2})(1-r^{-3}) \ldots(1-r^{-n+1})=n
\end{gathered}\]
From which it finally follows that\clearpage\noindent% 30
\[ W^{2}=2 r^{\frac{1}{4} n} n \tag{9}\]
Now it can be easily seen that \(r^{\frac{1}{4} n}\) is either \(=+i\) or \(=-i\), depending on whether \(k\) is of the form \(4 \mu+1\) or \(4 \mu+3\). And since
\[2 i=(1+i)^{2}, \quad-2 i=(1-i)^{2}\]
it will be in the case where \(k\) is of the form \(4 \mu+1\),
\[W= \pm(1+i) \sqrt{n}, \text{ and thus } \quad T=U= \pm \sqrt{n}\]
in the other case, where \(k\) is of the form \(4 \mu+3\),
\[W= \pm(1-i) \sqrt{n}, \text{ and thus } \quad T=-U= \pm \sqrt{n}\]
%

\subsection*{18.}
 
The method of the preceding article provides the absolute values of the functions \(T\) and \(U\), and assigns conditions under which the signs should be given as equal or opposite: but the signs themselves are not yet determined. For this case, where \(k=1\) is assumed, we will now supply as follows.
 
Let us assume \(\rho=\cos \frac{1}{2} \omega+i \sin \frac{1}{2} \omega\), so that \(r=\rho \rho\), and it is clear that, due to the equation \(\rho^{n}=-1\) [8], it can be expressed as follows,
\[W=2(1+\rho^{n-2})(1+\rho^{-4})(1+\rho^{n-6})(1+\rho^{-8}) \ldots(1+\rho^{-n+4})(1+\rho^{2})\]
or with the factors arranged in a different order,
\[W=2(1+\rho^{2})(1+\rho^{-4})(1+\rho^{6})(1+\rho^{-8}) \ldots(1+\rho^{-n+4})(1+\rho^{n-2})\]
Now we have
\[\begin{aligned}
& 1+\rho^{2}=2 \rho \cos \frac{1}{2} \omega \\
& 1+\rho^{-4}=2 \rho^{-2} \cos \omega \\
& 1+\rho^{+6}=2 \rho^{3} \cos \frac{3}{2} \omega \\
& 1+\rho^{-8}=2 \rho^{-4} \cos 2 \omega \text{ etc.}
\end{aligned}\]
up to
\[\begin{aligned}
& 1+\rho^{-n+4}=2 \rho^{-\frac{1}{2} n+2} \cos (\frac{1}{4} n-1) \omega \\
& 1+\rho^{n-2}=2 \rho^{\frac{1}{2} n-1} \cos (\frac{1}{4} n-\frac{1}{2}) \omega
\end{aligned}\]
Therefore, it holds that\clearpage\noindent% 31
\[W=2^{\frac{1}{2} n} \rho^{\frac{1}{4} n} \cos \frac{1}{2} \omega \cos \omega \cos \frac{3}{2} \omega \ldots \cos (\frac{1}{4} n-\frac{1}{2}) \omega\]
Entering the cosine in this product, it is evident that all are positive, and the factor \(\rho^{\frac{1}{4}n}\) becomes \(=\cos 45^{\circ}+i \sin 45^{\circ}=(1+i) \surd \frac{1}{2}\). Hence we conclude that \(W\) is a product of \(1+i\) and a positive real quantity, so it must necessarily be
\[W=(1+i) \cdot \surd n, \quad T=+\surd n, \quad U=+\surd n\]
%

\subsection*{19.}

It will be worthwhile to gather together here into one view all the summations developed up until now. Specifically, it is generally as follows:
\begin{center}
\begin{tabular}{|c|c|c|}
\hline
\(T=\) & \(U=\) & as \(n\) is of form \\
\hline
\(\pm \surd n\) & \(\pm \surd n\) & \(4 \mu\) \\
\(\pm \surd n\) & 0 & \(4 \mu+1\) \\
0 & 0 & \(4 \mu+2\) \\
0 & \(\pm \surd n\) & \(4 \mu+3\) \\
\hline
\end{tabular}
\end{center}
and in the case where \(k\) is assumed to be \(=1\), the positive sign must be assigned to the radical quantity. Therefore, by strict reasoning, everything that we observed for prime values of \(n\) in article 3 by induction has been demonstrated, and nothing remains except for us to teach how to determine the signs for any values of \(k\) in all cases. But before this task is permitted to be taken up in all generality, it will be necessary to first more closely consider the cases in which \(n\) is either a prime number or a power of a prime number.
%

\subsection*{20.}
 
Let \(n\) be a prime odd number. As we have shown in Article 10, it is evident that \(W=1+2 \Sigma r^{a}=1+2 \Sigma R^{a k}\), if we assume \(R=\cos \omega+i \sin \omega\), where \(a\) denotes all the quadratic residues of \(n\) between 1 and \(n-1\). If we also express by \(b\) all the non-quadratic residues between the same limits, it is easily understood that all the numbers \(a k\) will become congruent modulo \(n\) to either all \(a\) or all \(b\), without respect to order, depending on whether \(k\) is a residue or a non-residue. Therefore, in the former case, we have\clearpage\noindent% 32
\[W=1+2 \sum R^{a}=1+R+R^{4}+R^{9}+\text{etc.}+R^{(n-1)^{2}}\]
and thus \(W=+\sqrt{n}\), if \(n\) is of the form \(4 \mu+1\), and \(W=+i \sqrt{n}\), if \(n\) is of the form \(4 \mu+3\).
%

On the other hand, in the second case where \(k\) is a non-residue of \(n\) itself, we have
\[W=1+2 \Sigma R^{b}\]
Hence, since it is clear that all integers \(a\), \(b\) complete the complex integer numbers \(1\), \(2\), \(3, \ldots\), we have
\[\Sigma R^{a}+\Sigma R^{b}=R+R^{2}+R^{3}+\text{etc.}+R^{n-1}=-1\]
Therefore,
\[W=-1-2 \Sigma R^{a}=-(1+R+R^{4}+R^{9}+\text{etc.}+R^{(n-1)^{2}})\]
Therefore, \(W=-\sqrt{n}\) if \(n\) is of the form \(4\mu+1\), and \(W=-i\sqrt{n}\) if \(n\) is of the form \(4\mu+3\).
%

Hence it is concluded:
\textit{first}, if \(n\) is of the form \(4\mu+1\), and \(k\) is a quadratic residue of \(n\),
\[T=+\sqrt{n}, \quad U=0\]
\textit{second}, if \(n\) is of the form \(4\mu+1\), and \(k\) is a non-residue of \(n\),
\[T=-\sqrt{n}, \quad U=0\]
\textit{third}, if \(n\) is of the form \(4\mu+3\), and \(k\) is a residue of \(n\),
\[T=0, \quad U=+\sqrt{n}\]
\textit{fourth}, if \(n\) is of the form \(4\mu+3\), and \(k\) is a non-residue of \(n\),
\[T=0, \quad U=-\sqrt{n}\]
%

\subsection*{21.}

Let \(n\) be a power of an odd prime \(p\) increased by 2, and let \(n = p^{2 \chi} q\), where \(q\) is either \(1\) or \(p\). First of all, it is important to observe that if \(\lambda\) is any integer not divisible by \(p^{\chi}\), we have\clearpage\noindent% 33
\[\begin{aligned}
& r^{\lambda \lambda}+r^{(\lambda+p^{\chi} q)^{2}}+r^{(\lambda+2 p^{\chi} q)^{2}}+r^{(\lambda+3 p^{\chi} q)^{2}}+\text{etc.}+r^{(\lambda+n-p^{\chi} q)^{2}} \\
& \quad=r^{\lambda \lambda}\left\{1+r^{2 \lambda p^{\chi} q}+r^{4 \lambda p^{\chi} q}+r^{6 \lambda p^{\chi} q}+\text{etc.}+r^{2 \lambda(n-p^{\chi} q)}\right\}=\frac{r^{\lambda \lambda}(1-r^{2 \lambda n})}{1-r^{2 \lambda p^{\chi} q}}=0
\end{aligned}\]
From here, it is easy to see that
\[W=1+r^{p^{2 \chi}}+r^{4 p^{2 \chi}}+r^{9 p^{2 \chi}}+\text{etc.}+r^{(n-p^{\chi})^{2}}\]
The remaining terms of the sequence
\[1+r+r^{4}+r^{9}+\text{etc.}+r^{(n-1)^{2}}\]
can be distributed into \((p^{\chi}-1) q\) partial sequences, each with \(p^{x}\) terms, and by the given transformation, they create vanishing sums.
%

Hence it follows, in the case where \(q=1\), or where \(n\) is a power of a prime number with an even exponent, that
\[W=p^{\chi}=+\sqrt{n}, \text{ and therefore } T=+\sqrt{n}, U=0\]
On the other hand, in the case where \(q=p\), or where \(n\) is a power of a prime number with an odd exponent, let us establish \(r^{p^{2\chi}}=\rho\), where \(\rho\) will be the proper root of the equation \(x^{p}-1=0\), namely \(\rho=\cos \frac{k}{p} 360^{\circ}+i \sin \frac{k}{p} 360^{\circ}\), and then
\[W=1+\rho+\rho^{4}+\rho^{9}+\text{etc.}+\rho^{(p^{\chi+1}-1)^{2}}=p^{\chi}(1+\rho+\rho^{4}+\rho^{9}+\text{etc.}+\rho^{(p-1)^{2}})\]
%

But the sum of the series \(1+p+p^{4}+p^{9}+\) etc. \(+p^{(p-1)^{2}}\) is determined by the preceding article, from which it is concluded spontaneously that it becomes
\[\begin{aligned}
& W= \pm \sqrt{n}=T, \text{ if } p \text{ is of the form } 4 \mu+1 \\
& W= \pm i \sqrt{n}=i U, \text{ if } p \text{ is of the form } 4 \mu+3
\end{aligned}\]
with the positive or negative sign, according to whether \(k\) is a residue or a non-residue of \(p\).
%

\subsection*{22.}

Moreover, an easily derived proposition follows from those things which were explained in articles 20 and 21, which will provide us with a notable application. Let it be supposed
\[W^{\prime}=1+r^{h}+r^{4 h}+r^{9 h}+\text{etc.}+r^{h(n-1)^{2}}\]\clearpage\noindent% 34
where \(h\) is any integer not divisible by \(p\), and in the case where \(n=p\), or where \(n\) is a power of \(p\) with an odd exponent, it will be
\[\begin{aligned}
& W^{\prime}=W, \text{ if } h \text{ is a quadratic residue of } p \\
& W^{\prime}=-W, \text{ if } h \text{ is a quadratic non-residue of } p
\end{aligned}\]
For it is clear that \(W^{\prime}\) arises from \(W\) if we substitute \(kh\) for \(k\); in the former case, however, \(k\) and \(kh\) will be similar, in the latter dissimilar, insofar as they are quadratic residues or non-residues of \(p\).

In the case, however, where \(n\) is a power of \(p\) with an even exponent, it is manifest that \(W^{\prime}=+\sqrt{n}\), and therefore always \(W^{\prime}=W\).
%

\subsection*{23.}
 
In articles 20, 21, 22 we considered odd prime numbers, and their powers: hence, it remains to consider the case where \(n\) is a power of two.
 
For \(n=2\), it is clear that \(W=1+r=0\).
 
For \(n=4\), we obtain \(W=1+r+r^{4}+r^{9}=2+2 r\): hence \(W=2+2 i\), whenever \(k\) is of the form \(4 \mu+1\), and \(W=2-2 i\), whenever \(k\) is of the form \(4 \mu+3\).
 
For \(n=8\), we have \(W=1+r+r^{4}+r^{9}+r^{16}+r^{25}+r^{36}+r^{49}=2+4 r+2 r^{4}\) \(=4 r\). Hence we will have
\[\begin{aligned}
& W=(1+i) \sqrt{8}, \text{ whenever } k \text{ is of the form } 8 \mu+1 \\
& W=(-1+i) \sqrt{8}, \text{ whenever } k \text{ is of the form } 8 \mu+3 \\
& W=(-1-i) \sqrt{8}, \text{ whenever } k \text{ is of the form } 8 \mu+5 \\
& W=(1-i) \sqrt{8}, \text{ whenever } k \text{ is of the form } 8 \mu+7
\end{aligned}\]
%

If \(n\) is greater than a power of two, let \(n=2^{2 \chi} q\), so that \(q\) is either equal to 1 or 2, and \(x\) is greater than 1. Here first of all it should be observed, if \(\lambda\) is any integer not divisible by \(2^{\chi-1}\), we have
\[\begin{aligned}
& r^{\lambda \lambda}+r^{(\lambda+2^{\chi} q)^{2}}+r^{(\lambda+2.2^{\chi} q)^{2}}+r^{(\lambda+3.2^{\chi} q)^{2}}+\text{etc.}+r^{(\lambda+n-2^{\chi} q)^{2}} \\
& \quad=r^{\lambda \lambda}\left\{1+r^{2^{\chi+1} \lambda q}+r^{2.2^{\chi+1} \lambda q}+r^{3.2^{\chi+1} \lambda q}+\text{etc.}+r^{(2 n-2^{\chi+1} q) \lambda}\right\}=\frac{r^{\lambda \lambda}(1-r^{2 \lambda n})}{1-r^{2^{\chi+1} \lambda q}}=0
\end{aligned}\]
Hence it is easily seen that
\[W=1+r^{2^{2 \chi-2}}+r^{4.2^{2 \chi-2}}+r^{9.2^{2 \chi-2}}+\text{etc.}+r^{(n-2^{\chi-1})^{2}}\]
%

We assume \(r^{2^{2 \chi-2}}=\rho\), and \(\rho\) will be a root of the equation \(x^{4 q}-1=0\), specifically \(\rho=\cos \frac{k}{4 q} 360^{\circ}+i \sin \frac{k}{4 q} 360^{\circ};\) then we have
\[\begin{aligned}
W & =1+\rho+\rho^{4}+\rho^{9}+\text{etc.}+\rho^{(2^{\chi+1} q-1)^{2}} \\
& =2^{\chi-1}(1+\rho+\rho^{4}+\rho^{9}+\text{etc.}+\rho^{(4 q-1)^{2}})
\end{aligned}\]
But the sum of the series \(1+\rho+\rho^{4}+\rho^{9}+\text{etc.}+\rho^{(4 q-1)^{2}}\) is determined by what we explained in the cases \(n=4\), \(n=8\), hence we conclude that in the case where \(q=1\), or where \(n\) is a power of 4, we have
\[\begin{aligned}
& W=(1+i) 2^{\chi}=(1+i) \sqrt n, \text{ if } k \text{ is of the form } 4 \mu+1 \\
& W=(1-i) 2^{\chi}=(1-i) \sqrt n, \text{ if } k \text{ is of the form } 4 \mu+3
\end{aligned}\]
which are the exact formulas given for \(n=4\);\\
while in the case where \(q=2\), or where \(n\) is a power of 2 with an odd exponent greater than 3, we have
\[\begin{alignedat}[t]{2}
W&=(1+i) 2^{\chi} \sqrt 2&&=(1+i) \sqrt n, \text{ if } k \text{ is of the form } 8 \mu+1 \\
W&=(-1+i) 2^{\chi} \sqrt 2&&=(-1+i) \sqrt n, \text{ if } k \text{ is of the form } 8 \mu+3 \\
W&=(-1-i) 2^{\chi} \sqrt 2&&=(-1-i) \sqrt n, \text{ if } k \text{ is of the form } 8 \mu+5 \\
W&=(1-i) 2^{\chi} \sqrt 2&&=(1-i) \sqrt n, \text{ if } k \text{ is of the form } 8 \mu+7
\end{alignedat}\]
which also precisely match the formulas we provided for \(n=8\).
%

\subsection*{24.}

It will also be worth our while to determine the ratio of the sum of the series
\[ W' = 1 + r^h + r^{4h} + r^{9h} + \text{etc.} + r^{h(n-1)^2} \]
to \(W\), where \(h\) denotes any odd integer. Since \(W'\) arises from \(W\) by replacing \(k\) with \(kh\), the value of \(W'\) will depend on the form of the number \(kh\), just as \(W\) depends on the form of the number \(k\). Let us assume \(\frac{W'}{W} = l\) and it is evident that this can occur:

I. In the case where \(n = 4\), or any higher power of 2 with an even exponent, we have

\[
\begin{array}{l}
l = 1,\text{ if }h\text{ is of the form }4\mu+1\\ 
l = -i,\text{ if }h\text{ is of the form }4\mu+3,\text{ and }k\text{ is of the form }4\mu+1\\
l = +i,\text{ if }h\text{ is of the form }4\mu+3,\text{ and }k\text{ is of the same form}\\
\end{array}
\]

II. In the case where \(n = 8\), or any higher power of 2 with an odd exponent, we have

\[
\begin{array}{l} 
l = 1, \text{ if }h\text{ is of the form }8\mu+1, \\
l = -1, \text{ if }h\text{ is of the form }8\mu+5, \\
l = +i, \text{ if either }h\text{ is of the form }8\mu+3, \text{ and } k\text{ is of the form }4\mu+1,\\
\phantom{l = +i} \quad \text{or }h\text{ is of the form }8\mu+7, \text{ and } k\text{ is of the form }4\mu+3,\\
l = -i, \text{ if either }h\text{ is of the form }8\mu+3, \text{ and } k\text{ is of the form }4\mu+3, \\
\phantom{l = -i}\quad \text{or }h\text{ is of the form }8\mu+7, \text{ and } k\text{ is of the form }4\mu+1.
\end{array}
\]

The determination of the sum \(W\) for those cases where \(n\) is a prime number or a power of a prime number is complete. It remains, therefore, for us to complete those cases where \(n\) is composed of several prime numbers, for which purpose the following theorem will pave the way.
%

\subsection*{25.}

\textsc{Theorem.} \textit{Let \(n\) be the product of two positive relatively prime integers \(a\) and \(b\). Let}
\[\begin{aligned}
& P=1+r^{a a}+r^{4 a a}+r^{9 a a}+\text{etc.}+r^{(b-1)^{2} a a} \\
& Q=1+r^{b b}+r^{4 b b}+r^{9 b b}+\text{etc.}+r^{(a-1)^{2} b b}
\end{aligned}\]
\textit{Then \(W=PQ\).}

\textit{Proof.} Let \(\alpha\) represent any number among \(0\), \(1\), \(2\), \(3\ldots, a-1\), let \(\beta\) represent any number among \(0\), \(1\), \(2\), \(3\ldots, b-1\), let \(\nu\) represent any number among \(0\), \(1\), \(2\), \(3\ldots, n-1\). Then it is clear that
\[P=\sum r^{a a \beta \beta}, \quad Q=\sum r^{b b \alpha \alpha}, \quad W=\sum r^{\nu \nu}\]
Thus, we have \(PQ=\sum r^{a a \beta \beta +b b \alpha \alpha}\), by substituting for \(\alpha\) and \(\beta\) all possible values, in all possible combinations. Furthermore, because \(2ab\alpha\beta=2\alpha\beta n\), we have \(PQ=\sum r^{(a \beta+b \alpha)^{2}}\). But it is clearly seen, without difficulty, that each value of \(a \beta+b \alpha\) is distinct from the others and equal to some value of \(\nu\). Thus, we have \(PQ=\sum r^{\nu \nu}=W\).

Note also that \(r^{a a}\) is a primitive root of the equation \(x^{b}-1=0\), and \(r^{b b}\) is a primitive root of the equation \(x^{a}-1=0\).\clearpage\noindent% 37
%

\subsection*{26.}

Let \(n\) be a product of three prime numbers \(a\), \(b\), \(c\) with the property that if \(bc = b'\), then \(a\) and \(b'\) are also prime to each other; therefore, \(W\) is a product of two factors:

\[
\begin{aligned}
& 1+r^{aa}+r^{4aa}+r^{9aa}+\text{etc.}+r^{(b'-1)^2aa} \\
& 1+r^{b'b'}+r^{4b'b'}+r^{9b'b'}+\text{etc.}+r^{(a-1)^2b'b'}
\end{aligned}
\]

However, since \(r^{aa}\) is a primitive root of the equation \(x^{bc} - 1 = 0\), the aforementioned factor itself will be the product of

\[
\begin{aligned}
& 1+\rho^{bb}+\rho^{4bb}+\rho^{9bb}+\text{etc.}+\rho^{(c-1)^2bb} \\
& 1+\rho^{cc}+\rho^{4cc}+\rho^{9cc}+\text{etc.}+\rho^{(b-1)^2cc}
\end{aligned}
\]

if we designate \(r^{aa} = \rho\). Hence it is clear that \(W\) is the product of three factors:

\[
\begin{aligned}
& 1+r^{bbcc}+r^{4bbcc}+r^{9bbcc}+\text{etc.}+r^{(a-1)^2bbcc} \\
& 1+r^{aacc}+r^{4aacc}+r^{9aacc}+\text{etc.}+r^{(b-1)^2aacc} \\
& 1+r^{aabb}+r^{4aabb}+r^{9aabb}+\text{etc.}+r^{(c-1)^2aabb}
\end{aligned}
\]

where \(r^{bbcc}\), \(r^{aacc}\), and \(r^{aabb}\) will be the respective primitive roots of the equations \(x^a - 1 = 0\), \(x^b - 1 = 0\), \(x^c - 1 = 0\).
%

\subsection*{27.}

Hence it is easy to conclude in general that, if \(n\) is the product of any prime factors \(a\), \(b\), \(c\), etc., then \(W\) is made up of the same number of factors, which are
\[\begin{aligned}
& 1+r^{\frac{n n}{a a}}+r^{\frac{4 n n}{a a}}+r^{\frac{9 n n}{a a}}+\text{etc.}+r^{\frac{(a-1)^{2} n n}{a a}} \\
& 1+r^{\frac{n n}{b b}}+r^{\frac{4 n n}{b b}}+r^{\frac{9 n n}{b b}}+\text{etc.}+r^{\frac{(b-1)^{2} n n}{b b}} \\
& 1+r^{\frac{n n}{c c}}+r^{\frac{4 n n}{c c}}+r^{\frac{9 n n}{c c}}+\text{etc.}+r^{\frac{(c-1)^{2} n n}{c c}} \text{ etc.}
\end{aligned}\]
where \(r^{\frac{n n}{a b}}, \frac{n n}{r b b}, r^{\frac{n n}{c c}}\) etc. will be the roots of the equations \(x^{a}-1=0, x^{b}-1=0\), \(x^{c}-1=0\) etc.
%

\subsection*{28.}
 
From these principles, the transition to the complete determination of \(W\) for any given value of \(n\) is now obvious. Let \(n\) be decomposed into factors \(a\), \(b\), \(c\), etc., which are either distinct prime numbers or powers of distinct prime numbers. Let \(r^{\frac{n n}{a a}}=A\), \(r^{\frac{n n}{b b}}=B\), \(r^{\frac{n n}{c c}}=C\), etc., and let \(A\), \(B\), \(C\), etc. be the respective roots of the equations \(x^{a}-1=0\), \(x^{b}-1=0\), \(x^{c}-1=0\), etc. Then \(W\) is the product of the factors
\[\begin{aligned}
& 1+A+A^{4}+A^{9}+\text{etc.}+A^{(u-1)^{2}} \\
& 1+B+B^{4}+B^{9}+\text{etc.}+B^{(b-1)^{2}} \\
& 1+C+C^{4}+C^{9}+\text{etc.}+C^{(c-1)^{2}} \text{ etc.}
\end{aligned}\]
But each of these factors can be determined by the methods outlined in articles 20, 21, 23. Hence, the value of the product can also be known. It will be useful to collect the rules for determining these factors here. Since \(A\) is made to be \(=\frac{k n}{a} \cdot \frac{360^{0}}{a}\), the aggregate \(1+A+A^{4}+A^{9}+\text{etc.}+A^{(a-1)^{2}}\), which we will denote by \(L\), will be determined by the number \(\frac{k n}{a}\) in the same way that \(W\) is determined by \(k\) in our general investigation. There are now twelve cases to be distinguished.

I. If \(a\) is a prime number of the form \(4 \mu+1\), say \(=p\), or a power of such a prime number with an odd exponent, and also \(\frac{k n}{a}\) is a quadratic residue of \(p\), then \(L=+\sqrt{a}\).

II. If, for the remaining values, \(\frac{k n}{a}\) is a non-quadratic residue of \(p\), then \(L=-\sqrt{a}\).

III. If \(a\) is a prime number of the form \(4 \mu+3\), say \(=p\), or a power of such a prime number with an odd exponent, and also \(\frac{k n}{a}\) is a quadratic residue of \(p\), then \(L=+i \sqrt{a}\).

IV. If, for the remaining values, \(\frac{k n}{a}\) is a non-quadratic residue of \(p\), then \(L=-i \sqrt{a}\).

V. If \(a\) is a square number or a higher power of a prime number (with an even exponent), then \(L=+\sqrt{a}\).

VI. If \(a=2\), then \(L=0\).

VII. If \(a=4\) or a higher power of the binary number with an even exponent, and also \(\frac{k n}{a}\) is of the form \(4 \mu+1\), then \(L=(1+i) \sqrt{a}\).

VIII. If, for the remaining values, \(\frac{k n}{a}\) is of the form \(4 \mu+3\), then \(L=(1-i) \sqrt{a}\).

IX. If \(a=8\) or a higher power of the binary number with an odd exponent, and also \(\frac{k n}{a}\) is of the form \(8 \mu+1\), then \(L=(1+i) \sqrt{a}\).

X. If, for the remaining values, \(\frac{k n}{a}\) is of the form \(8 \mu+3\), then \(L=(-1+i) \sqrt{a}\).

XI. If, for the remaining values, \(\frac{k n}{a}\) is of the form \(8 \mu+5\), then \(L=(-1-i) \sqrt{a}\).

XII. If, for the remaining values, \(\frac{k n}{a}\) is of the form \(8 \mu+7\), then \(L=(1-i) \sqrt{a}\).
%

\subsection*{29.}

Let's take, for example, \(n=2520=8 . 9 . 5 . 7\) and \(k=13\). In this case, we have
\begin{quote}for \(a=8\), by the case XII, \(L=(1-i) \sqrt{8}\)\\
for the factor 9, by the case \(V\), the sum will be \(=\sqrt{9}\)\\
for the factor 5, by the case II, the sum will be \(=-\sqrt{5}\)\\
for the factor 7, by the case III, the sum will be \(=+i \sqrt{7}\)
\end{quote}

Hence, we get \(W=(1-i) \cdot(-i) \cdot \sqrt{2520}=(-1-i) \sqrt{2520}\).
%

Assuming the value of \( n \) is the same, when \( k = 1 \), the answer is as follows:
\begin{quote}
The factor of 8 yields the sum \((-1+i) \sqrt{8}\)\\
The factor of 9 yields the sum \(\sqrt{9}\)\\
The factor of 5 yields the sum \(\sqrt{5}\)\\
The factor of 7 yields the sum \(-i \sqrt{7}\)
\end{quote}
Hence, the product \(W = (1+i) \sqrt{2520}\) is obtained.
%

\subsection*{30.}
 
Another method of finding the sum \(W\) in a general manner is sought from those things that were explained in articles 22, 24. Let us assume \(\cos \omega+i \sin \omega=\rho\), and
\[\rho^{\frac{n n}{a a}}=\alpha, \quad \rho^{\frac{n n}{b b}}=\beta, \quad \rho^{\frac{n n}{c c}}=\gamma \text{ etc.}\]
so that we have \(r=\rho^{k}\), \(A=\alpha^{k}\), \(B=\beta^{k}\), \(C=\gamma^{k}\) etc. Then there will be
\[1+\rho+\rho^{4}+\rho^{9}+\text{etc.}+\rho^{(n-1)^{2}}\]
a product of factors
\[\begin{aligned}
& 1+\alpha+\alpha^{4}+\alpha^{9}+\text{etc.}+\alpha^{(a-1)^{2}} \\
& 1+\beta+\beta^{4}+\beta^{9}+\text{etc.}+\beta^{(b-1)^{2}} \\
& 1+\gamma+\gamma^{4}+\gamma^{9}+\text{etc.}+\gamma^{(c-1)^{2}} \text{ etc.}
\end{aligned}\]\clearpage\noindent% 40
therefore \(W\) is a product of factors
\[\begin{aligned}
& w=1+\rho+\rho^{4}+\rho^{9}+\text{etc.}+\rho^{(n-1)^{2}} \\
& \mathfrak{A}=\frac{1+A+A^{4}+A^{9}+\text{etc.}+A^{(a-1)^{2}}}{1+\alpha+\alpha^{4}+\alpha^{9}+\text{etc.}+a^{(a-1)^{2}}} \\
& \mathfrak{B}=\frac{1+B+B^{4}+B^{9}+\text{etc.}+B^{(b-1)^{2}}}{1+\beta+\beta^{4}+\beta^{9}+\text{etc.}+\beta^{(b-1)^2}} \\
& \mathfrak{C}=\frac{1+C+C^{4}+C^{9}+\text{etc.}+C^{(c-1)^{2}}}{1+\gamma+\gamma^{4}+\gamma^{9}+\text{etc.}+\gamma^{(c-1)^{2}}} \text{ etc.}
\end{aligned}\]
Now the first factor \(w\) is determined by the above discussion (article 19); the remaining factors \(\mathfrak{A}\), \(\mathfrak{B}\), \(\mathfrak{C}\) etc. come from the formulas of articles 22, 24, which are collected here again so that all can be considered together\footnote{Clearly, what was \(k\) and \(h\) there, will be \(\frac{n}{a}\) and \(k\) with respect to the second factor, \(\frac{n}{b}\) and \(k\) with respect to the third factor etc.}. Twelve cases must be distinguished here, namely\\
 
I. If \(a\) is a prime number (odd) \(=p\), or such a power of a number with an odd exponent, and \(k\) is the quadratic residue of \(p\), then the corresponding factor will be \(\mathfrak{A}=+1\).
 
II. If, with the remaining \(k\), it is the quadratic non-residue of \(p\), then \(\mathfrak{A}=-1\).
 
III. If \(a\) is a square of a prime number or a higher power of it with an even exponent, then \(\mathfrak{A}=+1\).
%

IV. If \(a\) is equal to 4, or a higher power of 2 with an even exponent, and \(k\) is of the form \(4 \mu+1\), then \(\mathfrak{A}=+1\).
 
V. If, with the remaining conditions as in IV, \(k\) is of the form \(4 \mu+3\), and \(\frac{n}{a}\) is of the form \(4 \mu+1\), then \(\mathfrak{A}=-i\).
 
VI. If, with the remaining conditions as in IV, \(k\) is of the form \(4 \mu+3\), and \(\frac{n}{a}\) is of the form \(4 \mu+3\), then \(\mathfrak{A}=+i\).
 
VII. If \(a\) is equal to 8, or a higher power of 2 with an odd exponent, and \(k\) is of the form \(8 \mu+1\), then \(\mathfrak{A}=+1\).
 
VIII. If, with the remaining conditions as in VII, \(k\) is of the form \(8 \mu+5\), then \(\mathfrak{A}=-1\).
 
IX. If, with the remaining conditions as in VII, \(k\) is of the form \(8 \mu+3\), and \(\frac{n}{a}\) is of the form \(4 \mu+1\), then \(\mathfrak{A}=+i\).\clearpage\noindent% 41
 
X. If, with the remaining conditions as in VII, \(k\) is of the form \(8 \mu+3\), and \(\frac{n}{a}\) is of the form \(4 \mu+3\), then \(\mathfrak{A}=-i\).
 
XI. If, with the remaining conditions as in VII, \(k\) is of the form \(8 \mu+7\), and \(\frac{n}{a}\) is of the form \(4 \mu+1\), then \(\mathfrak{A}=-i\).
 
XII. If, with the remaining conditions as in VII, \(k\) is of the form \(8 \mu+7\), and \(\frac{n}{a}\) is of the form \(4 \mu+3\), then \(\mathfrak{A}=+i\).
 
We omit the case where \(a=2\); in this case, indeed \(\mathfrak{A}\) would be \(\frac{0}{0}\), or indeterminate, but then \(W=0\) always.
 
The remaining factors \(\mathfrak{B}\), \(\mathfrak{E}\), etc. depend in the same way on \(b\), \(c\), etc., as \(\mathfrak{A}\) depends on \(a\), insofar as they enter into their determination.
%

\subsection*{31.}

According to this second method, the first example in article 29 is as follows:
\begin{quote}The factor \(w\) becomes \(=(1+i) \surd 2520\)\\
For \(a=8\), the corresponding factor \(\mathfrak{A}\) becomes, by case VIII, \(=-1\)\\
The second factor of \(n\) corresponds to factor \(+1\) (by case III)\\
The factor 5 corresponds to factor \(-1\) (by case II)\\
The factor 7 corresponds to factor \(-1\) (by case II)\end{quote}
Hence, the product \(W=(-1-i) \surd 2520\) is obtained, as in article 29.
%

\subsection*{32.}

Since the value of \(W\) can be determined using two methods, one of which is based on the relations of the numbers \(\frac{n k}{a}\), \(\frac{n k}{b}\), \(\frac{n k}{c}\), etc. with the numbers \(a\), \(b\), \(c\), etc., and the other depends on the relations of \(k\) with the numbers \(a\), \(b\), \(c\), etc., there must be a certain conditional connection between all these relations, so that each of them must be determinable from the others. Let us suppose that all the numbers \(a\), \(b\), \(c\), etc. are odd prime numbers, and let \(k\) be taken \(=1\). Let the factors \(a\), \(b\), \(c\), etc. be distributed into two classes, one of which contains those that are of the form \(4\mu+1\), and which are denoted by \(p\), \(p^{\prime}\), \(p^{\prime\prime}\), etc., and the other consists of those that are of the form \(4\mu+3\), and which are expressed by \(q\), \(q^{\prime}\), \(q^{\prime\prime}\), etc. We will designate the number of the latter by \(m\). Having done this, we observe first that \(n\) becomes of the form \(4\mu+1\), if \(m\) is even (which also applies to the case where the factors of the other class are completely absent, or where \(m=0\)), whereas \(n\) becomes of the form \(4\mu+3\), if \(m\) is odd. Now the determination of \(W\) is achieved by the first method as follows.

Let the numbers \(P\), \(P^{\prime}\), \(P^{\prime\prime}\), etc., \(Q\), \(Q^{\prime}\), \(Q^{\prime\prime}\), etc. be related to the numbers \(p\), \(p^{\prime}\), \(p^{\prime\prime}\), etc., \(q\), \(q^{\prime}\), \(q^{\prime\prime}\), etc., respectively, in such a way that it is stated
\[\begin{aligned}
& P=+1, \text{ if } \frac{n}{p} \text{ is a quadratic residue of } p \\
& P=-1, \text{ if } \frac{n}{p} \text{ is a quadratic non-residue of } p
\end{aligned}\]
and likewise for the rest. Then \(W\) will be the product of the factors \(P \sqrt{p}\), \(P^{\prime} \sqrt{p^{\prime}}\), \(P^{\prime\prime} \sqrt{p^{\prime\prime}}\), etc., \(iQ \sqrt{q}\), \(iQ^{\prime} \sqrt{q^{\prime}}\), \(iQ^{\prime\prime} \sqrt{q^{\prime\prime}}\), etc., and hence
\[W=P P^{\prime} P^{\prime\prime} \ldots Q Q^{\prime} Q^{\prime\prime} \ldots i^{m} \sqrt{n}\]
By the second method, or rather directly by the rules of Article 19, it will be
\begin{quote}\(W=+\sqrt{n}\), if \(n\) is of the form \(4\mu+1\), or equivalently, if \(m\) is even\\
\(W=+i \bigvee n\), if \(n\) is of the form \(4\mu+3\), or if \(m\) is odd\end{quote}
Both cases can be simultaneously encompassed by the following formula:
\[W=i^{m m} \sqrt{n}\]
Hence it follows that
\[P P^{\prime} P^{\prime\prime} \ldots Q Q^{\prime} Q^{\prime\prime} \ldots=i^{m m-m}\]
But \(i^{m m-m}\) becomes \(=1\) whenever \(m\) is of the form \(4\mu\) or \(4\mu+1\), and \(=-1\) whenever \(m\) is of the form \(4\mu+2\) or \(4\mu+3\), from which we deduce the following very elegant
%

\textsc{Theorem.} \textit{Let \(a\), \(b\), \(c\), etc. denote positive odd prime numbers that are not equal to each other, and let their product be denoted \(=n\). Let \(m\) be the number of the form \(4\mu + 3\) among them, and let the other numbers be of the form \(4\mu + 1\). The number of these numbers \(a\), \(b\), \(c\), etc., whose residues are not equal to \(\frac{n}{a}\), \(\frac{n}{b}\), \(\frac{n}{c}\), etc., will be even if \(m\) is of the form \(4\mu\) or \(4\mu + 1\), but odd if \(m\) is of the form \(4\mu + 2\) or \(4\mu + 3\).}
 
For example, if we take \(a=3\), \(b=5\), \(c=7\), \(d=11\), we have three numbers of the form \(4\mu + 3\), namely 3, 7, and 11; and we have \(5 \cdot 7 \cdot 11 \equiv 3\) (mod 3); \(3 \cdot 7 \cdot 11 \equiv 5\) (mod 5); \(3 \cdot 5 \cdot 11 \equiv 7\) (mod 7); \(3 \cdot 5 \cdot 7 \equiv 11\) (mod 11), so only one of \(\frac{n}{d}\) is not a residue of \(d\).
%

\subsection*{33.}

The most famous \textit{fundamental theorem} concerning quadratic residues is nothing but a special case of the theorem just developed. By restricting the set of numbers \(a\), \(b\), \(c\), etc. to \textit{two}, it is evident that if only one of them, or neither, is of the form \(4 \mu+3\), either \(a R b\), \(b R a\), or \(a N b\), \(b N a\) must hold simultaneously; on the other hand, if both are of the form \(4 \mu+3\), one of them must be a non-residue of the other, and vice versa. Thus, we present here the fourth proof of this most weighty theorem, for which we have already given the first and second proofs in Arithmetic Investigations, and the third proof recently in a separate essay (\textit{Commentaries} Vol. XVI). We will present two other proofs in the future, based again on completely different principles. Indeed, it is extremely remarkable that this most elegant theorem, which initially had eluded all attempts so persistently, could be approached later through various methods that were completely different from one another.
%

\subsection*{34.}

Moreover, the remaining theorems, which serve as a supplement to the fundamental theorem, namely those through which prime numbers, whose residues are either non-residues or residues of -1, +2, and -2, can be identified, can be derived from the same principles. We will begin with the residue +2.
%

\[
\text{By setting } n=8a \text{ such that } a \text{ is a prime number, and } k=1, \text{ by method } 28, \text{ } W \text{ will be the product of two factors, of which one will be } +\sqrt{a} \text{ or } +i\sqrt{a} \text{, if 8, or which is the same, 2, is the quadratic residue of } a; \text{ on the contrary } -\sqrt{a} \text{ or } -i\sqrt{a} \text{, if 2 is not a residue of } a. \text{ The second factor is }
\]
\[
\begin{aligned}
& (1+i)\sqrt{8}, \text{ if } a \text{ is of the form } 8\mu+1 \\
& (-1+i)\sqrt{8}, \text{ if } a \text{ is of the form } 8\mu+3 \\
& (-1-i)\sqrt{8}, \text{ if } a \text{ is of the form } 8\mu+5 \\
& (1-i)\sqrt{8}, \text{ if } a \text{ is of the form } 8\mu+7
\end{aligned}
\]
\[
\text{But by method } 18, \text{ it will always be } W=(1+i)\sqrt{n}; \text{ dividing this value by the four values of the second factor, it is evident that the first factor must be}
\]
\[
\begin{aligned}
& +\sqrt{a}, \text{ if } a \text{ is of the form } 8\mu+1 \\
& -i\sqrt{a}, \text{ if } a \text{ is of the form } 8\mu+3 \\
& -\sqrt{a}, \text{ if } a \text{ is of the form } 8\mu+5 \\
& +i\sqrt{a}, \text{ if } a \text{ is of the form } 8\mu+7
\end{aligned}
\]
\[
\text{Hence it follows spontaneously, in the first and fourth cases, that 2 must be the residue of } a, \text{ and in the second and third cases it must not be the residue.}
\]
%

\subsection*{35.}

Prime numbers, whose residue is either non-residue is \(-1\), are easily distinguished with the help of the following theorem, which is also remarkable in itself.

\textsc{Theorem.} \textit{The product of two factors}
\[\begin{aligned}
& W^{\prime}=1+r^{-1}+r^{-4}+\text{etc.}+r^{-(n-1)^{2}} \\
& W=1+r+r^{4}+\text{etc.}+r^{(n-1)^{2}}
\end{aligned}\]
\textit{is \(=n\), if \(n\) is odd; or \(=0\), if \(n\) is odd even; or \(=2 n\), if \(n\) is even even.}

\textit{Proof.} Since it is manifest
\[\begin{aligned}
W & =r+r^{4}+r^{9}+\text{etc.}+r^{n n} \\
& =r^{4}+r^{9}+\text{etc.}+r^{(n+1)^{2}} \\
& =r^{9}+\text{etc.}+r^{(n+2)^{2}} \text{ etc.}
\end{aligned}\]
the product \(W {W}^{\prime}\) can also be exhibited as
\[\begin{aligned}
&\phantom{ +, } 1+r+r^{4}+r^{9}+\text{etc.}+r^{(n-1)^{2}} \\
&+ r^{-1}(r+r^{4}+r^{9}+r^{16}+\text{etc.}+r^{n n}) \\
&+ r^{-4}(r^{4}+r^{9}+r^{16}+r^{25}+\text{etc.}+r^{(n+1)^{2}}) \\
&+ r^{-9}(r^{9}+r^{16}+r^{25}+r^{36}+\text{etc.}+r^{(n+2)^{2}}) \\
&\text{etc.} \\
&+ r^{-(n-1)^{2}}(r^{(n-1)^{2}}+r^{n n}+r^{(n+1)^{2}}+r^{(n+2)^{2}}+\text{etc.}+r^{(2 n-2)^{2}})
\end{aligned}\]
which, when vertically summed, produces
\[\begin{aligned}
&\phantom{+}n \\
&+r(1+r r+r^{4}+r^{6}+\text{etc.}+r^{2 n-2}) \\
&+r^{4}(1+r^{4}+r^{8}+r^{12}+\text{etc.}+r^{4 n-4}) \\
&+r^{9}(1+r^{6}+r^{12}+r^{18}+\text{etc.}+r^{6 n-6}) \\
&+\text{etc.} \\
&+r^{(n-1)^{2}}(1+r^{2 n-2}+r^{4 n-4}+r^{6 n-6}+\text{etc.}+r^{2(n-1)^{2}})
\end{aligned}\]
Now if \(n\) is odd, each part of this sum, except the first \(n\), will be \(=0\); for the second part manifestly becomes \(\frac{r(1-r^{2 n})}{1-r r}\), the third \(\frac{r^{4}(1-r^{4 n})}{1-r^{4}}\) etc. However, when \(n\) is even, it will also be necessary to include the part\clearpage\noindent% 45
\[r^{\frac{1}{4} n n}(1+r^{n}+r^{2 n}+r^{3 n}+\text{etc.}+r^{n n-n})\]
which becomes \(=n r^{\frac{1}{4} n n}\). Therefore, in the former case, \(W W^{\prime}=n\), but in the latter, \(=n+n r^{\frac{1}{4} n n}\); and \(r^{\frac{1}{4} n n}\) becomes \(=+1\), if \(n\) is even even, so in this case it results in \(W W^{\prime}=2 n\); on the contrary, \(r^{\frac{1}{4} n n}=-1\), if \(n\) is odd even, and thus \(W W^{\prime}=0\). Q. E. D.
%

\subsection*{36.}

Now, by article 22, it is clear that if \(n\) is an odd prime number, \(\frac{W^{\prime}}{W}\) will be equal to \(+1\) or \(-1\), according as \(-1\) is a residue or a non-residue of \(n\). Hence in the former case, we must have \(W^{2}=+n\), in the latter \(W^{2}=-n\); wherefore, by article 13, we conclude that the former case can only occur when \(n\) is of the form \(4 \mu+1\), and the latter case when \(n\) is of the form \(4 \mu+3\).
%

\(\text{Finally}, \text{from the combination of conditions for the residues }+2\text{ and }-1\text{, it follows spontaneously, that}\ -2\text{ is the residue of any prime number of the form }8\mu+1\text{ or }8\mu+3\text{, and not the residue of any prime number of the form }8\mu+5\text{ or }8\mu+7\text{.}\)\\[1.5\baselineskip]
\begin{center}\rule{1.5in}{0.5pt}\end{center}
\clearpage\noindent% 46
\;
\clearpage\noindent% 47
%

\begin{center}
\;\\[4\baselineskip]
{\large THE FUNDAMENTAL THEOREM }\\[\baselineskip]
{\tiny IN }\\[\baselineskip]
{\LARGE THE DOCTRINE OF QUADRATIC RESIDUES}\\[2\baselineskip]
{\large NEW DEMONSTRATIONS AND EXPANSIONS}\\[5\baselineskip]
{\tiny BY THE AUTHOR}\\[1.5\baselineskip]
CARL FRIEDRICH GAUSS\\[1.5\baselineskip]
{\scriptsize SUBMITTED TO THE ROYAL SOCIETY OF SCIENCES {\tiny \textsc{1817},} FEBRUARY {\tiny \textsc{10}.}}\\[4\baselineskip]
\rule{4in}{0.5pt}\\[0.5\baselineskip]
{\small Commentaries of the most recent Royal Society of Sciences of Göttingen. Vol. \textsc{iv}.\\
Göttingen \textsc{1818}.}\\
\rule{4in}{0.5pt}
\end{center}
 
\clearpage\noindent% 48
\;
\clearpage\noindent% 49
\section*{\;\\[4\baselineskip]
{\small THE FUNDAMENTAL THEOREM}\\[0.75\baselineskip]
{\scriptsize IN} \\[0.75\baselineskip]
{\large THE DOCTRINE OF QUADRATIC RESIDUES}\\[0.75\baselineskip]
{\small NEW DEMONSTRATIONS AND EXPANSIONS.}\\[0.75\baselineskip]
\rule{0.75in}{0.5pt}\\}
%

\textit{The fundamental theorem of quadratic residues, which is considered among the most beautiful truths of higher arithmetic, has indeed been easily discovered through induction, but has proven to be much more difficult to demonstrate. It often happens in this kind of scenario that the demonstrations of very simple truths, which seem to offer themselves readily to the inquirer through induction, are deeply hidden, and after many futile attempts, a path very different from the one sought may finally come to light. Then it often happens that once one path is discovered, several others are also revealed leading to the same goal, some more briefly and more directly, others seemingly obliquely and beginning from very different principles, among which you would scarcely have suspected any connection to the proposed question. Such a marvelous connection between more abstruse truths not only adds a certain peculiar beauty to these contemplations, but also merits diligent investigation and unraveling, because not infrequently new supports or increments to the very science itself come from there} \footnote{Source: Euler, L. (1738). Theoremata de numeris quibusdam spiralibus. Commentarii academiae scientiarum Petropolitanae, 11, 3-35.}.
%

\footnote{Although the arithmetic theorem, about which this will treat, might seem to be fully completed by previous efforts, which have provided four entirely different proofs\footnote{Two have been set forth in the Disquisitiones Arithmeticae Sections four and five; the third in a separate treatise (Commentt. Soc. Gotting. Vol. XVI), the fourth is included in the treatise: Summation of certain singular series (Commentt. Recentiores, Vol. I).}, yet I return anew to the same argument and add two other demonstrations, which will certainly shed new light on this matter. The first one is in a certain way related to the third, as it proceeds from the same lemma; then, however, it pursues a different path, so that it can well be considered a new proof, which, in its very conciseness, will not seem inferior, if not superior, to that third one. On the other hand, the sixth demonstration is based on a completely different and more subtle principle and provides a new example of the astonishing connection between arithmetical truths that, at first glance, appear to be very far apart. To these two demonstrations is added a very simple new algorithm to determine whether a given integer is a quadratic residue modulo a given prime or not.}
Therefore, although the theorem about which I am going to speak may seem to be fully completed by the previous efforts that have provided four entirely different demonstrations\footnote{Two have been set forth in the Disquisitiones Arithmeticae\enspace Sections four and five; the third in a separate treatise (Commentt. Soc. Gotting. Vol. XVI), the fourth is included in the treatise: Summation of certain singular series (Commentt. Recentiores, Vol. I).}, yet I return anew to the same argument and add two more demonstrations, which will certainly shed new light on this matter. The first is in a certain way related to the third, as it proceeds from the same lemma; then, however, it pursues a different path, so that it can well be considered a new proof, which will not seem inferior, if not superior, to the third one, in its very conciseness. On the other hand, the sixth demonstration is based on a completely different and more subtle principle and provides a new example of the astonishing connection between arithmetic truths that, at first glance, appear to be very far apart. To these two demonstrations, a very simple new algorithm is added to determine whether a given integer is a quadratic residue modulo a given prime or not.
%

\text{Another reason was present, which caused me to publish it now, especially to present new proofs, promised nine years ago.} \text{Namely, when I began to investigate the theory of cubic and biquadratic residues, a far more difficult argument, starting from the year 1805,} \text{I experienced a similar fate as once in the theory of quadratic residues.} \text{Immediately, indeed, those theorems, which entirely exhaust these questions, and in which a remarkable analogy with the theorems pertaining to quadratic residues shines forth, were discovered by induction as soon as an appropriate approach was sought: however, all efforts to obtain them, with their demonstrations perfected in every aspect, remained fruitless for a long time.} \text{This very incentive was the reason for me to endeavor so much to add other and other demonstrations already known about quadratic residues, hoping that, supported by many diverse methods, one or another could contribute something related to the illustration of the argument.} \text{This hope was by no means in vain, and tireless labor was finally followed by successful outcomes.} \text{Soon, the fruits of my vigilance will be allowed into the public light: but before I undertake this arduous work, I decided to return once more to the theory of quadratic residues, to complete everything that still remains to be done concerning the same, and thus to bid farewell to this higher part of arithmetic}.
%

\subsection*{{\scriptsize FUNDAMENTAL THEOREM IN THE THEORY OF QUADRATIC RESIDUES, FIFTH PROOF}\\[0.5\baselineskip]
1.}
 
In the introduction, we have already declared that the fifth proof and the third proceed from the same lemma, which for the sake of convenience, it seemed appropriate to repeat at this point with the signs adapted to the present discussion.
%

\textsc{Lemma.} \textit{Let \(m\) be a prime number (a positive odd integer), \(M\) an integer not divisible by \(m\); let the smallest positive residues of the numbers}
\[M, 2 M, 3 M, 4 M, \ldots \frac{1}{2}(m-1) M\]
\textit{modulo \(m\) be taken, which will be partly less than \(\frac{1}{2} m\) and partly greater: let the number of the latter be \(=n\). Then \(M\) will be a quadratic residue of \(m\), or not, according as \(n\) is even, or odd.}

\textsc{Proof.} Let those residues which are less than \(\frac{1}{2} m\) be denoted by \(a\), \(b\), \(c\), \(d\), etc., and let the rest, greater than \(\frac{1}{2} m\), be denoted by \(a^{\prime}\), \(b^{\prime}\), \(c^{\prime}\), \(d^{\prime}\), etc. The complement of the latter to \(m\), namely \(m-a^{\prime}\), \(m-b^{\prime}\), \(m-c^{\prime}\), \(m-d^{\prime}\), etc., will evidently all be less than \(\frac{1}{2} m\), and will also be different from each other, as well as from the residues \(a\), \(b\), \(c\), \(d\), etc. Thus, when taken together, though in a different order, they will be identical with all the numbers \(1\), \(2\), \(3\), \(4\), \ldots, \(\frac{1}{2}(m-1)\). If we let
\[1 \times 2 \times 3 \times 4 \ldots \frac{1}{2}(m-1)=P\]
then
\[P=a b c d \ldots \times(m-a^{\prime})(m-b^{\prime})(m-c^{\prime})(m-d^{\prime}) \ldots\]
which entails
\[(-1)^{n} P=a b c d \ldots \times(a^{\prime}-m)(b^{\prime}-m)(c^{\prime}-m)(d^{\prime}-m) \ldots\]
%

Then, according to the modulus \(m\),
\[P M^{\frac{1}{2}(m-1)} \equiv a b c d \ldots \times a^{\prime} b^{\prime} c^{\prime} d^{\prime} \ldots \equiv a b c d \ldots \times(a^{\prime}-m)(b^{\prime}-m)(c^{\prime}-m)(d^{\prime}-m) \ldots
\]
so
\[
P M^{\frac{1}{2}(m-1)} \equiv P(-1)^{n}
\]
Hence \(M^{\frac{1}{2}(m-1)} \equiv \pm 1\), taking the upper or lower sign, according to whether \(n\) is even or odd. Therefore, relying on the theorem demonstrated in \textit{Disquisitiones Arithmeticae} art. 106, the truth of the lemma is immediately apparent. 

%

\subsection*{2.}
 
\textsc{Theorem.} \textit{Let \(m\), \(M\) be positive odd integers prime to each other, \(n\) the set of their least positive residues in the sequence}
\[M, 2 M, 3 M \ldots  \frac{1}{2}(m-1) M\]
\textit{modulo \(m\), which are greater than \(\frac{1}{2} m\); and similarly let \(N\) be the set of their least positive residues in the sequence}
\[m, 2 m, 3 m \ldots \frac{1}{2}(M-1) m\]
\textit{modulo \(M\), which are greater than \(\frac{1}{2} M\). Then the three numbers \(n\), \(N\), \(\frac{1}{4}(m-1)(M-1)\) are either all even together, or one is even and the other two are odd.}
 
\textsc{Proof.} Let us denote
\[\begin{array}{l}
\text{by }f\text{ the set of numbers }1,2,3 \ldots \frac{1}{2}(m-1) \\
\text{by }f^{\prime}\text{ the set of numbers }m-1, m-2, m-3 \ldots \frac{1}{2}(m+1) \\
\text{by }F\text{ the set of numbers }1,2,3 \ldots \frac{1}{2}({M}-1) \\
\text{by }F^{\prime}\text{ the set of numbers }M-1, M-2, M-3 \ldots \frac{1}{2}(M+1)\\ 
\end{array}\]
So \(n\) will indicate the number of integers \(M f\) which have their least positive residues modulo \(m\) in the set \(f^{\prime}\), and similarly \(N\) will indicate the number of integers \(m F\) which have their least positive residues modulo \(M\) in the set \(F^{\prime}\). Finally let's denote\\
\[\begin{aligned}
&\varphi\text{ the set of numbers }1, 2, 3 \ldots \frac{1}{2}(m M-1)\\
&\varphi^{\prime}\text{ the set of numbers }m M-1, m M-2, m M-3 \ldots \frac{1}{2}(m M+1)
\end{aligned}\]
Since any integer not divisible by \(m\) should be congruent to a residue from set \(f\) or from set \(f^{\prime}\), and similarly any integer not divisible by \(M\) should be congruent to a residue from set \(F\) or from set \(F^{\prime}\), all the numbers \(\varphi\), among which obviously none is divisible by both \(m\) and \(M\), can be distributed into eight classes in the following way.
%

\(\text{I.}\) In the first class there will be numbers congruent to some number from \(f\) with respect to modulus \(m\), and with respect to modulus \(M\) congruent to some number from \(F\). We will denote the multitude of these numbers by \(\alpha\).\clearpage\noindent% 53

\(\text{II.}\) Numbers congruent to numbers from \(f\), \(F^{\prime}\) with respect to moduli \(m\), \(M\), respectively, of which we will establish the multitude \(=\beta\).

\(\text{III.}\) Numbers congruent to numbers from \(f^{\prime}\), \(F\) with respect to moduli \(m\), \(M\), respectively, of which we will establish the multitude \(=\gamma\).

\(\text{IV.}\) Numbers congruent to numbers from \(f^{\prime}\), \(F^{\prime}\) with respect to moduli \(m\), \(M\), respectively, of which the multitude shall be \(=\delta\).

\(\text{V.}\) Numbers divisible by \(m\), congruent to the residues from \(F\) with respect to modulus \(M\).

\(\text{VI.}\) Numbers divisible by \(m\), congruent to the residues from \(F^{\prime}\) with respect to modulus \(M\).

\(\text{VII.}\) Numbers divisible by \(M\), congruent to the residues from \(f\) with respect to modulus \(m\).

\(\text{VIII.}\) Numbers divisible by \(M\), congruent to the residues from \(f^{\prime}\) with respect to modulus \(m\).

Clearly, classes V and VI taken together will include all numbers \(mF\), the multitude of numbers contained in VI will be \(=N\), and hence the multitude of numbers contained in V will be \(\frac{1}{2}(M-1)-N\). Similarly, classes VII and VIII taken together will contain all numbers \(Mf\), in class VIII there will be \(n\) numbers, while in class VII there will be \(\frac{1}{2}(m-1)-n\).
%

\begin{itemize}
    \item All the numbers \(\varphi^{\prime}\) are distributed into eight classes IX - XVI. In this matter, if we maintain the same order, it will be easily seen that the numbers in the classes
    \begin{center} IX, X, XI, XII, XIII, XIV, XV, XVI \end{center}
    respectively form the complements of the numbers in the classes
    \begin{center} IV, III, II, I, VI, V, VIII, VII \end{center}
    included in the range \(m M\), such that in class IX there are \(\delta\) numbers; in class X, \(\gamma\), and so on. Now it is clear that if all the numbers of the first class are associated with all the numbers of the ninth class, all the numbers below \(m M\) will be obtained, which, according to the modulus \(m\), are congruent to some number from \(f\), and according to the modulus \(M\), are congruent to some number from \(F\), and the equality of their number will be equal to the number of all combinations of each \(f\) with each \(F\), is easily seen. Therefore, we have
    \[\alpha+\delta=\frac{1}{4}(m-1)(M-1)\]
    and by a similar argument
    \[\beta+\gamma=\frac{1}{4}(m-1)(M-1)\]
\end{itemize}
%

\begin{align*}
\text{By joining all numbers of classes II, IV, VI, we will clearly have all numbers} & \text{below}\frac{1}{2} m M\text{, which are congruent to some residue from }F^{\prime}\text{ mod }M.\\
\text{Moreover, these same numbers can also be represented as follows:} & \\
F^{\prime}, M+F^{\prime}, 2 M+F^{\prime}, 3 M+F^{\prime} & \ldots \frac{1}{2}(m-3) M+F^{\prime}\\
\text{where the total number of them will be} & =\frac{1}{4}(m-1)(M-1), \text{ or we will have}\\
\beta+\delta+N=\frac{1}{4}(m-1)(M-1) & \\
\text{Similarly, from the union of all classes} & III, IV, VIII \text{it follows that}\\
\gamma+\delta+n=\frac{1}{4}(m-1)(M-1) & \\
\text{From these four equations arise the following:} & \\
2 \alpha=\frac{1}{4}(m-1)(M-1)+n+N & \\
2 \beta=\frac{1}{4}(m-1)(M-1)+n-N & \\
2 \gamma=\frac{1}{4}(m-1)(M-1)-n+N & \\
2 \delta=\frac{1}{4}(m-1)(M-1)-n-N & \\
\text{each of which demonstrates the truth of the theorem.} &
\end{align*}
%

\subsection*{3.}
 
If we now assume that \(m\) and \(M\) are prime numbers, the combination of the previous theorem with lemma 1 of article 1 immediately yields the fundamental theorem. For it is clear, 
 
I. how many times both \(m\), \(M\), or one of them, are of the form \(4k+1\), the number \(\frac{1}{4}(m-1)(M-1)\) will be even, and therefore \(n\) and \(N\) will be both even or both odd, and thus either both \(m\) and \(M\) are quadratic residues of each other, or both are non-quadratic residues of each other. 
 
II. However, whenever both \(m\), \(M\) are of the form \(4k+3\), \(\frac{1}{4}(m-1)(M-1)\) will be odd, hence one of the numbers \(n\), \(N\) will be even and the other odd, and therefore one of the numbers \(m\), \(M\) will be a quadratic residue of the other, and the other will be a non-quadratic residue of the other. Q. E. D.\clearpage\noindent
%

\subsection*{{\scriptsize FUNDAMENTAL THEOREM IN THE THEORY OF QUADRATIC RESIDUES, PROOF SIXTH.}\\
1.}
 
\textsc{Theorem.} \textit{Let \(p\) denote a prime number (positively odd), \(n\) a positive integer not divisible by \(p\), \(x\) an indeterminate quantity, the function}
\[1+x^{n}+x^{2 n}+x^{3 n}+\text{etc.}+x^{n p-n}\]
\textit{will be divisible by}
\[1+x+x x+x^{3}+\text{etc.}+x^{p-1}\]
 
\textsc{Proof.} Let \(g\) be a positive integer such that \(g n \equiv 1\pmod{p}\), and let \(g n=1+h p\). Then we have
\[\begin{aligned}
\frac{1+x^{n}+x^{2 n}+x^{3 n}+\text{etc.}+x^{n p-n}}{1+x+x x+x^{3}+\text{etc.}+x^{p-1}}=\frac{(1-x^{n p})(1-x)}{(1-x^{n})(1-x^{p})} & =\frac{(1-x^{n p})(1-x^{g n}-x+x^{h p+1})}{(1-x^{n})(1-x^{p})} \\
& =\frac{1-x^{n p}}{1-x^{p}} \cdot \frac{1-x^{g n}}{1-x^{n}}-\frac{x(1-x^{n p})}{1-x^{n}} \cdot \frac{1-x^{h p}}{1-x^{p}}
\end{aligned}\]
thus it's clear that the function is integral. Q. E. D.
 
Therefore, any integral function of \(x\) itself divisible by \(\frac{1-x^{n p}}{1-x^{n}}\), will also be divisible by \(\frac{1-x^{p}}{1-x}\).
%

\subsection*{2.}

Design the \(\alpha\) primitive positive root for the modulus \(p\), i.e., let \(\alpha\) be a positive integer such that the smallest positive residues of the powers \(1\), \(\alpha\), \(\alpha^2\), \(\alpha^{3}\), \(\ldots\), \(\alpha^{p-2}\) with respect to the modulus \(p\) are identical to the numbers \(1\), \(2\), \(3\), \(4\), \(\ldots\), \(p-1\). Furthermore, by denoting the function by \(f(x)\),
\[x+x^{\alpha}+x^{\alpha \alpha}+x^{\alpha^{3}}+\ldots+x^{\alpha^{p-2}}+1\]
it is evident that \(f(x)-1-x-x x-x^{3}-\ldots-x^{p-1}\) will be divisible by \(1-x^{p}\), and therefore, a fortiori by \(\frac{1-x^{p}}{1-x}=1+x+x x+x^{3}+\ldots+x^{p-1}\). Hence it follows that, since \(x\) represents an indeterminate quantity, \(f(x^{n})\) will also be divisible by \(\frac{1-x^{n p}}{1-x^{n}}\) and consequently (by the previous article) also by \(\frac{1-x^{p}}{1-x}\), as long as \(n\) is an integer not divisible by \(p\). Conversely, whenever \(n\) is an integer divisible by \(p\), each part of the function \(f(x^{n})\) reduced by unity will be divisible by \(1-x^{p}\). Therefore in this case, \(f(x^{n})-p\) will also be divisible by \(1-x^{p}\), and consequently also by \(\frac{1-x^{p}}{1-x}\).
%

\subsection*{3.}

\textsc{Theorem.} \textit{By assuming}
\[x-x^{\alpha}+x^{\alpha \alpha}-x^{\alpha^{3}}+x^{\alpha^{4}}-\text{etc.}-x^{\alpha^{p-2}}=\xi\]
\textit{we will have \(\xi \xi \mp p\) divisible by \(\frac{1-x^{p}}{1- x}\), taking the upper sign whenever \(p\) is of the form \(4k+1\) and the lower sign whenever \(p\) is of the form \(4k+3\).}

\textsc{Proof.} It can easily be seen that, from these \(p-1\) functions
\[\begin{aligned}
& +x \xi-x x+x^{\alpha+1}-x^{\alpha \alpha+1}+\text{etc.}+x^{\alpha^{p-2}+1} \\
& -x^{\alpha} \xi-x^{2 \alpha}+x^{\alpha \alpha+\alpha}-x^{\alpha^{3}+\alpha}+\text{etc.}+x^{\alpha^{p-1}+\alpha} \\
& +x^{\alpha \alpha} \xi-x^{2 \alpha \alpha}+x^{\alpha^{3}+\alpha \alpha}-x^{\alpha^{4}+\alpha \alpha}+\text{etc.}+x^{\alpha^{p}+\alpha \alpha} \\
& -x^{\alpha^{3}} \xi-x^{2 \alpha^{3}}+x^{\alpha^{4}+\alpha^{3}}-x^{\alpha^{5}+\alpha^{3}}+\text{etc.}+x^{\alpha^{p+1}+\alpha^{3}}
\end{aligned}\]
etc. all the way to
\[-x^{\alpha^{p-2}} \xi-x^{2 \alpha^{p-2}}+x^{\alpha^{p-1}+\alpha^{p-2}}-x^{\alpha^{p}+\alpha^{p-2}}+\text{etc.}+x^{\alpha^{2 p-4}+\alpha^{p-2}}\]
become \(=0\), and each remaining function is divisible by \(1-x^{p}\). Therefore, the sum of all these will also be divisible by \(1-x^{p}\), which is gathered to be
\[\begin{aligned}
&\begin{gathered}=\xi \xi-(f(x x)-1)+(f(x^{\alpha+1})-1)-(f(x^{\alpha \alpha+1})-1)+(f(x^{\alpha^{3}+1})-1)-\text{etc.}\\
+(f(x^{\alpha^{p-2}+1})-1) \end{gathered}\\
&=\xi \xi-f(x x)+f(x^{\alpha+1})-f(x^{\alpha \alpha+1})+f(x^{\alpha^{3}+1})-\text{etc.}+f(x^{\alpha^{p-2}+1})=\Omega
\end{aligned}\]
Therefore, this expression \(\Omega\) will also be divisible by \(\frac{1-x^{p}}{1-x}\). Now among the exponents \(2\), \(\alpha+1\), \(\alpha \alpha+1\), \(\alpha^{3}+1 \ldots \alpha^{p-2}+1\), only one will be divisible by \(p\), say \(\alpha^{\frac{1}{2}(p-1)}+1\), from the previous article. Therefore, apart from the term \(f(x^{\alpha^{\frac{1}{2}(p-1)}+1})\), every part of the expression \(\Omega\)
\[f(x x), f(x^{\alpha+1}), f(x^{\alpha \alpha+1}), f(x^{\alpha^{3}+1}) \text{ etc.}\]
will be divisible by \(\frac{1-x^{p}}{1-x}\). Thus, we are allowed to delete those parts, so that the function
\[\xi \xi \mp f(x^{\alpha^{\frac{1}{2}(p-1)}+1})\]
will remain divisible by \(\frac{1-x^{p}}{1-x}\). Therefore, this expression \(\xi \xi \mp p\) will also be divisible by \(\frac{1-x^{p}}{1-x}\). Q.E.D.

To avoid any ambiguity from the double sign, we will denote by \(\varepsilon\) the number \(+1\) or \(-1\), according to whether \(p\) is of the form \(4k+1\) or \(4k+3\). Therefore, the function
\[\frac{(1-x)(\xi \xi-\varepsilon p)}{1-x^{p}}\]
is a complete function of \(x\), which we will designate by \(Z\).
%

\subsection*{4.}

Let \(q\) be a positive odd number, and therefore \(\frac{1}{2}(q-1)\) an integer. Thus \((\xi \xi)^{\frac{1}{2}(q-1)}-(\varepsilon p)^{\frac{1}{2}(q-1)}\) will be divisible by \(\xi \xi-\varepsilon p\), and therefore also by \(\frac{1-x^{p}}{1-x}\). Let's assume \(\varepsilon^{\frac{1}{2}(q-1)}=\delta\), and
\[\xi^{q-1}-\delta p^{\frac{1}{2}(q-1)}=\frac{1-x^{p}}{1-x} \cdot Y\]
and \(Y\) will be a whole function of \(x\), and \(\delta=+1\), whenever one of the numbers \(p\), \(q\), or both, is of the form \(4k+1\); conversely, \(\delta=-1\), whenever both \(p\), \(q\) are of the form \(4k+3\).
%

\subsection*{5.}

Now, let \(q\) also be a prime number (different from \(p\)) and it is evident, by theorem 51 demonstrated in the \textit{Disquisitiones Arithmeticae}, that
\[\xi^{q}-(x^{q}-x^{q \alpha}+x^{q \alpha \alpha}-x^{q \alpha^{3}}+\text{etc.}-x^{q \alpha^{p-2}})\]
is divisible by \(q\), or in the form \(qX\), so that \(X\) is a complete function of \(x\), even with respect to numerical coefficients (which is also to be understood for the other complete functions occurring here, \(Z\), \(Y\), \(W\)). Let us designate the index of the number \(q\) by \(\mu\) for the modulus \(p\) and the primitive root \(\alpha\), i.e., let \(q \equiv \alpha^{\mu}\pmod{p}\). Therefore, the numbers \(q\), \(q \alpha\), \(q \alpha \alpha\), \(q \alpha^{3}\ldots q \alpha^{p-2}\) will be congruent to the numbers \(\alpha^{\mu}\), \(\alpha^{\mu+1}\), \(\alpha^{\mu+2}\ldots \alpha^{p-2}\), \(1\), \(\alpha\), \(\alpha \alpha\ldots \alpha^{\mu-1}\) modulo \(p\), and thus
\[\begin{aligned}
&x^{q}-x^{\alpha^{\mu}} \\
&x^{q \alpha}-x^{\alpha^{\mu+1}} \\
&x^{q \alpha \alpha}-x^{\alpha^{\mu+2}} \\
&x^{q \alpha^{3}}-x^{\alpha^{\mu+3}}\\
&\quad \vdots \end{aligned}\]
\[\begin{aligned}
&x^{q \alpha^{p-\mu-2}}-x^{\alpha^{p-2}}\\
&x^{q \alpha^{p-\mu-1}}-x\\
&x^{q \alpha^{p-\mu}}-x^{\alpha}\\
&x^{q \alpha^{p-\mu+1}}-x^{\alpha \alpha}\\
&\quad \vdots\\
&x^{q\alpha^{p-2}}-x^{\alpha^{\mu-1}}
\end{aligned}\]
are divisible by \(1-x^{p}\). Which quantities, taken alternately positive and negative and added, it is clear that the function
\[x^{q}-x^{q \alpha}+x^{q \alpha \alpha}-x^{q \alpha \alpha^{3}}+\text{etc.}-x^{q \alpha^{p-2}}\mp \xi\]
is divisible by \(1-x^{p}\), with the sign above or below, depending on whether \(\mu\) is even or odd, in other words, depending on whether \(q\) is a quadratic residue of \(p\) or a non-residue. Therefore, let us establish
\[x^{q}-x^{q \alpha}+x^{q \alpha}-x^{q \alpha^{3}}+\text{etc.}-x^{q \alpha^{p-2}}-\gamma \xi=(1-x^{p})W\]
making \(\gamma=+1\) or \(\gamma=-1\), depending on whether \(q\) is a quadratic residue of \(p\) or a non-residue, and it is evident that \(W\) becomes a complete function.
%

\subsection*{6.}

Having prepared in this way, from the combination of the preceding equations we deduce
\[q \xi {X}=\varepsilon p(\delta p^{\frac{1}{2}(q-1)}-\gamma)+\frac{1-x^{p}}{1-x} \cdot(Z(\delta p^{\frac{1}{2}(q-1)}-\gamma)+Y \xi \xi-W \xi(1-x))\]
Let us suppose that, from the division of the function \(\xi X\) by
\[x^{p-1}+x^{p-2}+x^{p-3}+\text{etc.}+x+1\]
the quotient \(U\) arises, with remainder \(T\), or the relation
\[\xi X=\frac{1-x^{p}}{1-x} \cdot U+T\]
in such a way that \(U\), \(T\) are integral functions, even with respect to the numerical coefficients, and certainly \(T\) of an order certainly lower than the divisor. Therefore, we will have
\[q T-\varepsilon p(\delta p^{\frac{1}{2}(q-1)}-\gamma)=\frac{1-x^{p}}{1-x} \cdot(Z(\delta p^{\frac{1}{2}(q-1)}-\gamma)+Y \xi \xi-W \xi(1-x)-q U)\]
which equation clearly cannot subsist unless both the left member and the right member vanish separately. It will therefore be that \(\varepsilon p(\delta p^{\frac{1}{2}(q-1)}-\gamma)\) is divisible by \(q\), and also \(\delta p^{\frac{1}{2}(q-1)}-\gamma\), and therefore, by \(\delta \delta=1\), the number \(p^{\frac{1}{2}(q-1)}-\gamma \delta\) will be divisible by \(q\).
%

\(\beta\) is now designated as a positive or negative unit, according to \(p\) being a quadratic residue or non-residue of the number \(q\), we will have \(p^{\frac{1}{2}(q-1)}-\beta\) divisible by \(q\), and also \(\beta-\gamma \delta\), which cannot happen unless \(\beta=\gamma \delta\). Hence, the fundamental theorem follows spontaneously. Namely,

I. as often as either or both \(p\), \(q\), or one of them alone, is of the form \(4 k+1\), and consequently \(\delta=+1\), then \(\beta=\gamma\), and consequently either \(q\) is a quadratic residue of \(p\), and \(p\) is a quadratic residue of \(q\), or both \(q\) is a non-residue of \(p\), and \(p\) is a non-residue of \(q\).

II. as many times as both \(p\), \(q\) are of the form \(4 k+3\), and consequently \(\delta=-1\), then \(\beta=-\gamma\), and accordingly either both \(q\) is a quadratic residue of \(p\), and \(p\) is a non-residue of \(q\), or both \(q\) is a non-residue of \(p\), and \(p\) is a residue of \(q\). Q. E. D.
%

\subsection*{{\scriptsize \textit{A new algorithm for determining whether a given positive integer is a quadratic residue or non-residue modulo a given prime.}}\\
1.}

Before we present a new solution to this problem, we will briefly repeat the solution given in \textit{Disquisitiones Arithmeticae}, which is performed quite expediently with the help of the fundamental theorem and the following well-known theorems:

I. The relation of number \(a\) to number \(b\) (insofar as the former is a quadratic residue or non-residue modulo the latter) is the same as the relation of number \(c\) to \(b\), if \(a \equiv c\pmod{b}\).

II. If \(a\) is a product of factors \(\alpha\), \(\beta\), \(\gamma\), \(\delta\), etc., and \(b\) is a prime number, the relation of \(a\) to \(b\) will depend on the relation of these factors to \(b\), such that \(a\) becomes a quadratic residue or non-residue according to whether there is an even or odd number of such factors which are non-residues modulo \(b\). Thus, whenever a factor is a square, it should not be considered in this examination; but if a factor is a power of an integer with an odd exponent, it will play the role of this integer here.

III. The number 2 is a quadratic residue of any prime number of the form \(8m+1\) or \(8m+7\), and a non-residue of any prime number of the form \(8m+3\) or \(8m+5\).%\clearpage\noindent% 60

Therefore, given the number \(a\) for which its relation to the prime number \(b\) is sought: for \(a\), if it is greater than \(b\), its minimum positive residue modulo \(b\) will be substituted in place of \(a\) first. Then, after resolving this residue into its prime factors, the problem is reduced by theorem II to the determination of the relation of each of these factors to \(b\). The relation of the factor 2 (since it is present once, or thrice, or five times, etc.) is known by theorem III; the relation of the remaining factors depends on the fundamental theorem. Thus, instead of finding the relation of the given number to the prime number \(b\), some relations of the number \(b\) to other odd primes smaller than \(b\) have now been investigated, and these problems, in the same way, will be reduced to smaller moduli, and clearly, these successive reductions will eventually be exhausted.
%

\subsection*{2.}

In order that this solution be illustrated by an example, let us seek the relationship of the number 103 to 379. Since 103 is now smaller than 379, and itself a prime number, it will be immediately necessary to apply the fundamental theorem, which teaches that the sought relationship is the opposite of the relationship of the number 379 to 103. This in turn is equal to the relationship of the number 70 to 103, which depends on the relationships of the numbers 2, 5, 7 to 103. The first of these relationships is revealed by Theorem III. The second, by the fundamental theorem, depends on the relationship of the number 103 to 5, which by Theorem I is equal to the relationship of the number 3 to 5; this, in turn, by the fundamental theorem, depends on the relationship of the number 5 to 3, to which by Theorem I is equal the relationship of the number 2 to 3, as noted by Theorem III. Likewise, the relationship of the number 7 to 103 by the fundamental theorem depends on the relationship of the number 103 to 7, which by Theorem I is equal to the relationship of the number 5 to 7; this, in turn, by the fundamental theorem, depends on the relationship of the number 7 to 5, which is equal by Theorem I to the relationship of the number 2 to 5, as noted by Theorem III. If it is now desired to transform this analysis into a synthesis, the decision of the question will be referred to the fourteen points, which we present here in full, so that the greater conciseness of the new solution may be the more clearly understood.
%

\begin{enumerate}
    \item The number 2 is a quadratic residue of the number 103 (theorem III).
    
    \item The number 2 is a non-quadratic residue of the number 3 (theorem III).
    
    \item The number 5 is a non-quadratic residue of the number 3 (by 1 and 2).
    
    \item The number 3 is a non-quadratic residue of the number 5 (fundamental theorem and 3).
    
    \item The number 103 is a non-quadratic residue of the number 5 (1 and 4).\clearpage\noindent% 61
    
    \item The number 5 is a non-quadratic residue of the number 103 (fundamental theorem and 5).
    
    \item The number 2 is a non-quadratic residue of the number 5 (theorem III).
    
    \item The number 7 is a non-quadratic residue of the number 5 (1 and 7).
    
    \item The number 5 is a non-quadratic residue of the number 7 (fundamental theorem and 8).
    
    \item The number 103 is a non-quadratic residue of the number 7 (1 and 9).
    
    \item The number 7 is a quadratic residue of the number 103 (fundamental theorem and 10).
    
    \item The number 70 is a non-quadratic residue of the number 103 (II, 1, 6, 11).
    
    \item The number 379 is a non-quadratic residue of the number 103 (1 and 12).
    
    \item The number 103 is a quadratic residue of the number 379 (fundamental theorem and 13).
\end{enumerate}
%

\(\)In the following, for the sake of brevity, we will use the notation introduced in \textit{Comment. Gotting. Vol.} XVI. Namely, by \([x]\) we will denote the quantity \(x\) itself, whenever \(x\) is an integer, or the integer closest to \(x\), whenever \(x\) is a fractional quantity, so that \(x-[x]\) always becomes a quantity not negative and less than one unit.
%

\subsection*{3.}
 
\textsc{Problem.} \textit{With \(a\), \(b\) denoting positive integers which are coprime to each other, and putting \(\left[\frac{1}{2} a\right]=a^{\prime}\), find the sum}
\[\left[\frac{b}{a}\right]+\left[\frac{2 b}{a}\right]+\left[\frac{3 b}{a}\right]+\left[\frac{4 b}{a}\right]+\text{etc.}+\left[\frac{a^{\prime} b}{a}\right]\]
 
\textsc{Solution.} For the sake of brevity, let us denote the sum in question by \(\varphi(a, b)\), such that
\[\varphi(b, a)=\left[\frac{a}{b}\right]+\left[\frac{2 a}{b}\right]+\left[\frac{3 a}{b}\right]+\text{etc.}+\left[\frac{b^{\prime} a}{b}\right]\]
if we set \(\left[\frac{1}{2} b\right]=b^{\prime}\). In the demonstration of the third fundamental theorem, it was shown that, in the case where \(a\) and \(b\) are odd, we have
\[\varphi(a, b)+\varphi(b, a)=a^{\prime} b^{\prime}\]
and by easily following the same method, the truth of this proposition is also extended to the case where one of the numbers \(a\), \(b\) is odd, as we already mentioned there. Let \(a\) be divided by \(b\) in a manner analogous to the method by which the greatest common divisor of two integers is investigated, with quotient \(\beta\) and remainder \(c\); then divide \(b\) by \(c\) and so on in a similar manner, such that the equations
\[\begin{aligned}
& a=\beta b+c \\
& b=\gamma c+d \\
& c=\delta d+e \\
& d=\varepsilon e+f \text{ etc.}
\end{aligned}\]
are obtained.
 
In this way, in a series of continuously decreasing numbers \(b\), \(c\), \(d\), \(e\), \(f\) etc., we eventually reach the unit, as hypothesized \(a\) and \(b\) are coprime, so that the last equation becomes
\[k=\lambda l+1\]
%

\[\begin{aligned}
& \text{When it is obvious that} \\
& {\left[\frac{a}{b}\right]=\left[\beta+\frac{c}{b}\right]=\beta+\left[\frac{c}{b}\right]} \\
& {\left[\frac{2 a}{b}\right]=\left[2 \beta+\frac{2 c}{b}\right]=2 \beta+\left[\frac{2 c}{b}\right]} \\
& {\left[\frac{3 a}{b}\right]=\left[3 \beta+\frac{3 c}{b}\right]=3 \beta+\left[\frac{3 c}{b}\right]}
\end{aligned}\]
etc., then
\[\varphi(b, a)=\varphi(b, c)+\frac{1}{2} \beta (b^{\prime} b^{\prime}+b^{\prime})\]
and therefore
\[\varphi(a, b)=a^{\prime} b^{\prime}-\frac{1}{2} \beta (b^{\prime} b^{\prime}+b^{\prime})-\varphi(b, c)\]
By similar reasoning, if we set \(\left[\frac{1}{2} c\right]=c^{\prime}\), \(\left[\frac{1}{2} d\right]=d^{\prime}\), \(\left[\frac{1}{2} e\right]=e^{\prime}\) etc.,
\[\begin{aligned}
& \varphi(b, c)=b^{\prime} c^{\prime}-\frac{1}{2} \gamma(c^{\prime} c^{\prime}+c^{\prime})-\varphi(a, d) \\
& \varphi(c, d)=c^{\prime} d^{\prime}-\frac{1}{2} \delta(d^{\prime} d^{\prime}+d^{\prime})-\varphi(d, e) \\
& \varphi(d, e)=d^{\prime} e^{\prime}-\frac{1}{2} \varepsilon(e^{\prime} e^{\prime}+e^{\prime})-\varphi(e, f)
\end{aligned}\]
etc. until
\[\varphi(k, l)=k^{\prime} l^{\prime}-\frac{1}{2} \lambda(l^{\prime} l^{\prime}+l^{\prime})-\varphi(l, 1)\]
Therefore, since it is obvious that \(\varphi(l, 1)=0\), we obtain the formula
\[\begin{aligned}
& \varphi(a, b)=a^{\prime} b^{\prime}-b^{\prime} c^{\prime}+c^{\prime} d^{\prime}-d^{\prime} e^{\prime}+\text{etc.} \pm k^{\prime} l^{\prime} \\
& \quad-\frac{1}{2} \beta (b^{\prime} b^{\prime}+b^{\prime})+\frac{1}{2} \gamma(c^{\prime} c^{\prime}+c^{\prime})-\frac{1}{2} \delta(d^{\prime} d^{\prime}+d^{\prime})+\frac{1}{2} \varepsilon(e^{\prime} e^{\prime}+e^{\prime})-\text{etc.} \mp \frac{1}{2} \lambda(l^{\prime} l^{\prime}+l^{\prime})
\end{aligned}\]
%

\subsection*{4.}
 
Easily now from those things which were explained in the third proof, one infers that the relation of the number \(b\) to \(a\), whenever \(a\) is a prime number, can spontaneously be known from the value of the aggregate \(\varphi(a, 2 b)\). Namely, just as this aggregate is a even or odd number, \(b\) will be a quadratic residue of \(a\) or a non-residue. Towards the same end the aggregate \(\varphi(a, b)\) itself can also be employed, with the restriction, however, that the case where \(b\) is odd is distinguished from the case where it is even. Namely,
 
I. Whenever \(b\) is odd, \(b\) will be a residue or non-residue of the quadratic of \(a\), according to whether \(\varphi(a, b)\) is even or odd.
 
II. Whenever \(b\) is even, the same rule will hold, if in addition \(a\) is of the form \(8n+1\) or of the form \(8n+7\); if however, for the even value of \(b\) itself, the modulus \(a\) is of the form \(8n+3\) or of the form \(8n+5\), the opposite rule will be applicable, that is, \(b\) will be a quadratic residue of \(a\), if \(\varphi(a, b)\) is odd, but a non-residue if \(\varphi(a, b)\) is even.
 
All of these are very easily derived from article 4 of the third proof.
%

\subsection*{5.}

\textit{Example.} If the ratio of the number 103 to the prime number 379 is sought, then to find the aggregate \(\varphi(379,103)\), we have
\[\begin{array}{r|r|r}
a= 379 & a^{\prime}=189 & \\
b= 103 & b^{\prime}=\phantom{0}51 & \beta=3 \\
c= \phantom{0}70 & c^{\prime}=\phantom{0}35 & \gamma=1 \\
d= \phantom{0}33 & d^{\prime}=\phantom{0}16 & \delta=2 \\
e= \phantom{00}4 & e^{\prime}=\phantom{00}2 & \varepsilon=8
\end{array}\]
hence
\[\varphi(379,103)=9639-1785+560-32-3978+630-272+24=4786\]
whence 103 will be the quadratic residue of the number 379. If we want to apply the same aggregate to \((379,206)\) for the same purpose, we have this pattern:
\begin{center}
\begin{tabular}{r|r|r}
379 & 189 &  \\
206 & 103 & 1 \\
173 & 86 & 1 \\
33 & 16 & 5 \\
8 & 4 & 4 \\
\end{tabular}
\end{center}
from which we deduce
\[\varphi(379,206)=19467-8858+1376-64-5356+3741-680+40=9666\]
therefore 103 is the quadratic residue of the number 379.
%

\subsection*{6.}

When deciding the ratio of the number \(b\) to \(a\), it is not necessary to compute each part of the aggregate \(\varphi(a, b)\), but it is sufficient to know how many of them are odd, our rule can also be expressed as follows:

Let \(a = \beta b+c\), \(b = \gamma c+d\), \(c = \delta d+e\) etc., until the series of numbers \(a\), \(b\), \(c\), \(d\), \(e\) etc. reaches unity. Let \(\left[\frac{1}{2} a\right]=a^{\prime}\), \(\left[\frac{1}{2} b\right]=b^{\prime}\), \(\left[\frac{1}{2} c\right]=c^{\prime}\) etc., and let \(\mu\) be the number of odd numbers in the series \(a^{\prime}\), \(b^{\prime}\), \(c^{\prime}\) etc., immediately followed by an odd number; further, let \(\nu\) be the number of odd numbers in the series \(\beta\), \(\gamma\), \(\delta\) etc., to which a number of the form \(4n+1\) or \(4n+2\) in the series \(b^{\prime}\), \(c^{\prime}\), \(d^{\prime}\) etc., respectively corresponds. With this done, \(b\) will be a quadratic residue or non-residue of \(a\), according to whether \(\mu + \nu\) is even or odd, with the exception of the case when both \(b\) and \(a\) are even, or of the form \(8n+3\) or \(8n+5\), for which the opposite rule holds.

In our example, the series \(a^{\prime}\), \(b^{\prime}\), \(c^{\prime}\), \(d^{\prime}\), \(e^{\prime}\) presents two sequences of odd numbers, hence \(\mu =2\); in the series \(b^{\prime}\), \(c^{\prime}\), \(d^{\prime}\), \(e^{\prime}\), although there are two odd numbers, the corresponding numbers in \(b^{\prime}\), \(c^{\prime}\), \(d^{\prime}\), \(e^{\prime}\) are of the form \(4n+3\), hence \(\nu = 0\). Therefore, \(\mu + \nu\) is even, and so 103 is the quadratic residue of the number 379.
%

\begin{center}
\;\\[4\baselineskip]
{\large THEORY }\\[\baselineskip]
{\LARGE ON BICUADRATIC RESIDUES }\\[3\baselineskip]
{\large FIRST COMMENTARY}\\[2.5\baselineskip]
{\tiny BY}\\[1.5\baselineskip]
CARL FRIEDRICH GAUSS\\[1.5\baselineskip]
{\scriptsize DELIVERED TO THE ROYAL SOCIETY {\tiny \textsc{1825},} APRIL {\tiny \textsc{5}.}}\\[4\baselineskip]
\rule{4in}{0.5pt}\\[0.5\baselineskip]
{\small Recent Commentaries of the Royal Society of Göttingen. Vol. \textsc{vi}.\\
Göttingen \textsc{1827}.}\\
\rule{4in}{0.5pt}
\end{center}
\clearpage\noindent% 66
\;
\clearpage\noindent% 67
\section*{\;\\[4\baselineskip]
{\large THEORY OF BICUADRATIC RESIDUES.}\\[0.75\baselineskip]
FIRST COMMENTARY\\[0.5\baselineskip]
\rule{0.75in}{0.5pt}\\[0.5\baselineskip]}
%

\subsection*{1.}

The theory of quadratic residues reduces to a few fundamental theorems, to be counted among the most beautiful treasures of higher arithmetic. These theorems have first been easily detected by induction, and then it is well known that they have been demonstrated in many ways, such that nothing more desirable is left.

However, the theory of cubic and biquadratic residues is of a much deeper investigation. When we began to investigate this from the year 1805, besides those things which are as if placed at the threshold, indeed some special theorems presented themselves, both on account of their simplicity, and on account of the very remarkable difficulty of their demonstrations. Soon, however, we found that the principles of higher arithmetic hitherto used were by no means sufficient for establishing the general theory, rather they necessarily demand that this be established, so that the field of higher arithmetic may be as if promoted to infinity, which, however, will be most clearly understood in the continuation of these discussions. As soon as we entered this new field, the approach to the knowledge of the simplest theorems exhausting the whole theory immediately appeared through induction: but their demonstrations lay so profoundly hidden that after many vain attempts, they could at last be brought to light.

Now that we are preparing to publish these labors, we will begin from the theory of biquadratic residues, and indeed in this first commentary, we will explain those inquiries which have already been completed within the expanded field of arithmetic. They as if pave the way for it, and at the same time add some new increments to the theory of circle division.
%

\subsection*{2.}

We introduced the concept of a biquadratic residue in \textit{Disquisitiones Arithmeticae} art. 115: namely, an integer \(a\), positive or negative, is called a biquadratic residue mod \(p\) if \(a\) can be made congruent to a biquadrate mod \(p\), and similarly, a non-residue if such a congruence does not exist. In all subsequent discussions, unless explicitly stated otherwise, we will assume \(p\) to be a prime number (positive and odd), and \(a\) to be not divisible by \(p\), since all other cases can easily be reduced to this one.
%

\subsection*{3.}

It is clear that every residue of the biquadratic of the number \(p\) is also a residue of its quadratic, and so every non-residue of the quadratic is also a non-residue of the biquadratic. This proposition can be also inverted whenever \(p\) is a prime number of the form \(4n+3\). For in this case, if \(a\) is a quadratic residue of \(p\), let us assume \(a \equiv b^2 \pmod{p}\), where \(b\) will either be a quadratic residue of \(p\) or a non-residue. In the first case, we assume \(b \equiv c^2\), hence \(a \equiv c^{4}\), that is, \(a\) will be a biquadratic residue of \(p\); in the second case, \(-b\) will be a quadratic residue of \(p\) (since \(-1\) is a non-residue of any prime of the form \(4n+3\)), and making \(-b \equiv c^2\) will as before show that \(a \equiv c^{4}\), and \(a\) is a biquadratic residue of \(p\). At the same time, it will easily be seen that besides these two solutions to the congruence \(x^{4} \equiv a\pmod{p}\), no other solutions exist in this case. Since these propositions clearly encompass the entire theory of biquadratic residues for prime moduli of the form \(4n+3\), we will exclude such moduli entirely from our investigation, or we will limit it to prime moduli of the form \(4n+1\).
%

\subsection*{4.}

Therefore, given that \(p\) is a prime number of the form \(4n+1\), it is not allowed to convert the proposition from the previous article: namely, there can exist quadratic residues that are not at the same time biquadratic residues, which happens whenever the quadratic residue is congruent to the square of a non-quadratic residue. For let \(a \equiv bb\), with \(b\) being a non-\clearpage\noindent% 69 quadratic residue modulo \(p\). If the congruence \(x^4 \equiv a\) could be satisfied, then by the value \(x \equiv c\), we would have \(c^4 \equiv bb\), or the product \((cc-b)(cc+b)\) would be divisible by \(p\), from which \(p\) should measure either the factor \(cc-b\) or the other \(cc+b\), i.e., either \(+b\) or \(-b\) would be the quadratic residue modulo \(p\), and therefore both (since \(-1\) is a quadratic residue), contrary to the hypothesis.
 
Therefore, all integers could be distributed into three classes not divisible by \(p\), the first of which contains biquadratic residues, the second non-biquadratic residues that are at the same time quadratic residues, and the third non-quadratic residues. Clearly, it is sufficient to subjugate only the numbers \(1, 2, 3, \ldots, p-1\) to this classification, transferring half of them to the third class, while the other half is distributed between the first and the second class.
%

\subsection*{5.}

But it will be preferable to establish four classes, in such a way that their nature is as follows.

Let \(A\) be the set of all residues of the biquadratics of \(p\), situated between 1 and \(p-1\) (inclusive), and \(e\) be a non-quadratic residue of \(p\) chosen at will. Let \(B\) be the set of minimal positive residues arising from the products \(e A\) taken with respect to the modulus \(p\), and likewise \(C\) and \(D\) be the set of minimal positive residues arising from the products \(e e A\), \(e^{3} A\) taken with respect to the modulus \(p\). Having done this, it is easy to see that each number in \(B\) will be different from the others, and likewise each number in \(C\) and also each in \(D\); however, a number cannot occur in all of these sets. Furthermore, it is clear that all numbers contained in \(A\) and \(C\) are quadratic residues of \(p\), while all those in \(B\) and \(D\) are non-quadratic residues, so that certainly the sets \(A\), \(C\) cannot have a number in common with the set \(B\) or \(D\). Moreover, neither \(A\) with \(C\) nor \(B\) with \(D\) can have any number in common. For suppose

I. that some number from \(A\), say \(a\), is also found in \(C\), where it arises from the product \(e e a^{\prime}\) being congruent to the same integer, with \(a^{\prime}\) being a number from the set \(A\). Let \(a \equiv \alpha^{4}\), \(a^{\prime} \equiv \alpha^{\prime 4}\), and let an integer \(\Theta\) be chosen such that \(\Theta \alpha^{\prime} \equiv 1\). Having done so, it will be the case that \(e e \alpha^{\prime 4} \equiv \alpha^{4}\), and thus by multiplying by \(\Theta^{4}\), we have
\[e e \equiv \alpha^{4} \Theta^{4}\]
i.e. \(e\) is a biquadratic residue, hence \(e\) is a quadratic residue, contrary to the hypothesis.

II. Similarly assuming that some number common to the sets \(B\), \(D\) arises from the products \(e a\), \(e^{3} a^{\prime}\), with \(a\), \(a^{\prime}\) being numbers from the set \(A\), and from the congruence \(e a \equiv e^{3} a^{\prime}\) it would follow that \(a \equiv e e a^{\prime}\), hence a number would be obtained, arising from the product \(e e a^{\prime}\) belonging to both \(C\) and \(A\), which we have shown to be impossible in the given manner.
%

Moreover, it is easily shown that \textit{all} quadratic residues of \(p\), situated between 1 and \(p-1\) inclusive, must necessarily occur either in \(A\) or in \(C\), and all non-quadratic residues of \(p\) between those limits must necessarily occur either in \(B\) or in \(D\). For
 
I. Every such quadratic residue, which is also a biquadratic residue, is found in \(A\) by hypothesis.
 
II. Let the quadratic residue \(h\) (less than \(p\)), which is also a non-quadratic residue, be set as \(\equiv g g\), where \(g\) is a non-quadratic residue. Let an integer \(\gamma\) be taken such that \(e \gamma \equiv g\), and let \(\gamma\) be a quadratic residue of \(p\), which we set as \(\equiv k k\). Hence
\[h \equiv g g \equiv e e \gamma \gamma \equiv e e k^{4}\]
Therefore, since the minimum residue of \(k^{4}\) is found in \(A\), the number \(h\), which arises from its product by \(e e\), must necessarily be contained in \(C\).
 
III. Let \(h\) denote a non-quadratic residue of \(p\) between the limits 1 and \(p-1\), from which an integer number \(g\) such that \(e g \equiv h\) is derived between the same limits. Thus, \(g\) is a quadratic residue, and therefore is contained in either \(A\) or \(C\): in the former case, \(h\) will clearly be found among the numbers in \(B\), and in the latter case, it will be found among the numbers in \(D\).
%

\(\frac{1}{4}(p-1)\) means that each series should contain one-fourth of the numbers from 1 to p-1.  In this classification, classes A and C possess their numbers essentially, but the distinction between classes B and D is arbitrary to the extent that it depends on the choice of the number e, which is always to be referred to class B; therefore, if another number from class D is adopted in its place, the classes B and D will be permuted with respect to each other.
%

\subsection*{6.}

Since \(-1\) is a quadratic residue of \(p\) itself, we state that \(-1 \equiv f^2\pmod{p}\), from which the four roots of the congruence \(x^{4} \equiv 1\) will be \(1\), \(f\), \(-1\), \(-f\). Therefore, if \(a\) is a biquadratic residue of \(p\), suppose \(\equiv \alpha^{4}\), then the four roots of the congruence \(x^{4} \equiv a\) will be \(\alpha\), \(f \alpha\), \(-\alpha\), \(-f \alpha\), which are easily seen to be incongruous with each other. Hence, it is clear that if the least positive residues of biquadratic residues \(1\), \(16\), \(81\), \(256 \ldots (p-1)^{4}\) are collected, the quartet will always be equal, so that \(\frac{1}{4}(p-1)\) diverse biquadratic residues forming a complex set \(A\) will be obtained. If only the least residues of biquadratic residues up to \((\frac{1}{2} p-\frac{1}{2})^{4}\) are collected, each will occur twice.
%

\subsection*{7.}

The product of two biquadratic residues is clearly a biquadratic residue, so from the multiplication of two numbers of class \(A\) always comes a product whose minimum positive residue belongs to the same class. Similarly, the products of numbers from \(B\) into numbers from \(D\) or numbers from \(C\) into numbers from \(C\) will have their minimum residues in \(A\).

In \(B\), the residues of the products \(A \cdot B\) and \(C \cdot D\) fall; in \(C\) the residues of the products \(A \cdot C\), \(B \cdot B\), and \(D \cdot D\); finally, in \(D\) the residues of the products \(A \cdot D\) and \(B \cdot C\) fall.

The proofs are so obvious that it suffices to have indicated one. Let, for example, \(c\) and \(d\) be numbers from \(C\) and \(D\), and \(c \equiv e ea\), \(d \equiv e^{3} a^{\prime}\), denoting \(a\), \(a^{\prime}\) as numbers from \(A\). Then \(e^{4} a a^{\prime}\) will be a biquadratic residue, that is, its minimum residue will be referred to \(A\): hence, as the product \(c d\) is made \(\equiv e e^{4} a a^{\prime}\), its minimum residue will be contained in \(B\).

At the same time, it can now be easily judged to which class the product of several factors should be referred. Namely, by attributing \(0\), \(1\), \(2\), \(3\) as the character to the class \(A\), \(B\), \(C\), \(D\) respectively, the character of the product or the sum of the characters of the individual factors will be equal, or its minimum residue will be modulo 4.
%

\subsection*{8.}

It seemed worthwhile to develop these elementary propositions without the aid of the theory of residues of powers, as called upon in this way, it is much easier to demonstrate everything thus far.

Let \(g\) be a primitive root for the modulus \(p\), i.e. a number such that in the series of powers \(g, g^2, g^3, \ldots\) no value before \(g^{p-1}\) is congruent to 1 modulo \(p\). Then the minimal residues of the numbers \(1\), \(g\), \(g^2\), \(g^3, \ldots, g^{p-2}\) apart from the order with \(1, 2, 3, \ldots, p-1\) can be conveniently distributed into four classes in the following manner:\clearpage\noindent% 72
\begin{center}
\begin{tabular}{c|l}
to & minimal residues of the numbers \\
\hline
\(A\) & \(1, \phantom{g}g^4, g^8, \phantom{g}g^{12}, \ldots, g^{p-5}\) \\
\(B\) & \(g, \phantom{g}g^5, g^9, \phantom{g}g^{13}, \ldots, g^{p-4}\) \\
\(C\) & \(g^2, g^6, g^{10}, g^{14}, \ldots, g^{p-3}\) \\
\(D\) & \(g^3, g^7, g^{11}, g^{15}, \ldots, g^{p-2}\) \\
\end{tabular}
\end{center}

Hence all the preceding propositions follow naturally.

Moreover, just as here the numbers \(1\), \(2\), \(3, \ldots, p-1\) have been distributed into four classes, the aggregate of which we designate by \(A\), \(B\), \(C\), \(D\), so \textit{any} integer not divisible by \(p\) according to the norm of its minimal residue modulo \(p\) may be assigned to one of these classes.
%

\subsection*{9.}

We will denote by \(f\) the minimal residue of the power \(g^{\frac{1}{4}(p-1)}\) modulo \(p\).  Where it happens that \(f^2 \equiv g^{\frac{1}{2}(p-1)} \equiv -1\) (\textit{Disquis. Arithm.} art. 62), it is clear that the character \(f\) here has the same meaning as in article 6.  Thus, the power \(g^{\frac{1}{4} \lambda(p-1)}\), with \(\lambda\) denoting a positive integer, will be congruent modulo \(p\) to the number \(1\), \(f\), \(-1\), \(-f\), as \(\lambda\) takes the forms \(4m\), \(4m+1\), \(4m+2\), \(4m+3\), or as the minimal residue of \(g^{\lambda}\) is found in \(A\), \(B\), \(C\), \(D\) respectively.  Hence, we obtain a very simple criterion for judging to which class the given number \(h\) not divisible by \(p\) should be referred; namely, \(h\) will belong to \(A\), \(B\), \(C\), or \(D\) as the power \(h^{\frac{1}{4}(p-1)}\) modulo \(p\) turns out congruent to the number \(1\), \(f\), \(-1\), or \(-f\).

As a corollary, it follows from here that \(-1\) is always referred to class \(A\) whenever \(p\) is of the form \(8n+1\), and to class \(C\) whenever \(p\) is of the form \(8n+5\).  The proof of this theorem, independently from the theory of powers of residues, can be easily constructed from what we have shown in \textit{Disquisitionibus Arithmeticis} art. 115, III.
%

\subsection*{10.}
 
Since \textit{all} primitive roots for modulo \(p\) yield residues of powers \(g^{\lambda}\), taking all numbers prime to \(p-1\) for \(\lambda\), it is easily seen that they will be equally distributed between the sets \({B}\) and \({D}\), with the base \(g\) always contained in \({B}\). If, instead of the number \(g\), a different primitive root from the set \(B\) is chosen as the base, the classification will remain the same; however, if a primitive root from the set \(D\) is adopted as the base, the sets \(B\) and \(D\) will be interchanged.
 
If the classification criterion is built upon the previous article's theorem, the distinction between the sets \(B\) and \(D\) will depend on which root of congruence \(x x \equiv-1\pmod{p}\) we adopt as the characteristic number \(f\).
%

\subsection*{11.}
 
In order for the more subtle investigations, which we are about to undertake, to be able to be illustrated by examples, we present here the construction of classes for all modules below 100. We have adopted the primitive root for each minimum.
\[\begin{aligned}
& \begin{array}{c} p=5 \\  g=2, f=2 \end{array} \\
\begin{array}{l} A \\ B \\ C \\D \end{array} & \begin{array}{|r} 1 \\ 2 \\ 4 \\ 3 \end{array} \\
& \begin{array}{c} p=13 \\  g=2, f=8 \end{array} \\ 
\begin{array}{l} A \\ B \\ C \\D \end{array} & \begin{array}{|rrr} 1,& 3,& 9 \\ 2,& 5,&6 \\ 4,& 10,& 12 \\ 7,& 8,& 11 \end{array}\\
& \begin{array}{c} p=17 \\  g=2, f=12 \end{array} \\
\begin{array}{l} A \\ B \\ C \\D \end{array} & 
\begin{array}{|rrrr} 
1,& 4,&13,& 16 \\ 
3,& 5,&12,&14 \\
2,& 8,& 9,&15 \\ 
6,& 7,&10,&11 
\end{array} \\
& \begin{array}{c} p=29 \\  g=2, f=12 \end{array} \\
\begin{array}{l} A \\ B \\ C \\D \end{array} & 
\begin{array}{|rrrrrrr} 1,&7,&16,&20,&23,&24,&25\\ 
2,& 3,&11,&14,&17,&19,&21 \\ 
4,& 5,&6,&9,&13,&22,&28 \\ 
8,& 10,& 12,&15,&18,& 26,&27
\end{array} \\
\end{aligned}\]
%

\[\begin{aligned}
& \begin{array}{c} \text{Prime number: }37 \\  g=2, f=31 \end{array} \\
\begin{array}{l} \text{Set of residues modulo }37 \\ \text{Box A} \\ \text{Box B} \\ \text{Box C} \\ \text{Box D} \end{array} & 
\begin{array}{|rrrrrrrrr} 
1,&7,&9,&10,&12,&16,&26,&33,&34 \\ 
2,& 14,&15,&18,&20,&24,&29,&31,&32 \\ 
3,& 4,& 11,&21,&25,&27,&28,&30,&36 \\ 
5,&6,&8,& 13,& 17,&19,&22,& 23,&35
\end{array} \\
& \begin{array}{c} \text{Prime number: }41 \\  g=6, f=32 \end{array} \\
\begin{array}{l} \text{Set of residues modulo }41 \\ \text{Box A} \\ \text{Box B} \\ \text{Box C} \\ \text{Box D} \end{array} & 
\begin{array}{|rrrrrrrrrr} 
1,&4,&10,&16,&18,&23,&25,&31,&37,&40 \\ 
6,& 14,&15,&17,&19,&22,&24,&26,&27,&35 \\ 
2,& 5,& 8,&9,&20,&21,&32,&33,&36,&39 \\ 
3,&7,&11,& 12,& 13,&28,&29,& 30,&34,&38
\end{array}  \\
& \begin{array}{c} \text{Prime number: }53 \\  g=2, f=30 \end{array} \\
\begin{array}{l} \text{Set of residues modulo }53 \\ \text{Box A} \\ \text{Box B} \\ \text{Box C} \\ \text{Box D} \end{array} & 
\begin{array}{|rrrrrrrrrrrrr} 
1,&10,&13,&15,&16,&24,&28,&36,&42,&44,&46,&47,&49 \\ 
2,&3,& 19,&20,&26,&30,&31,&32,&35,&39,&41,&45,&48 \\ 
4,& 6,& 7,&9,&11,&17,&25,&29,&37,&38,&40,&43,&52 \\ 
5,&8,&12,&14,&18,&21,&22,& 23,&27,&33,&34,&50,&51
\end{array} \\
& \begin{array}{c} \text{Prime number: }61 \\  g=2, f=11 \end{array} \\
\begin{array}{l} \text{Set of residues modulo }61 \\ \text{Box A} \\ \text{Box B} \\ \text{Box C} \\ \text{Box D} \end{array} & 
\begin{array}{|rrrrrrrrrrrrrrr} 
1,&9,&12,&13,&15,&16,&20,&22,&25,&34,&42,&47,&56,&57,&58 \\ 
2,&7,& 18,&23,&24,&26,&30,&32,&33,&40,&44,&50,&51,&53,&55 \\ 
3,& 4,& 5,&14,&19,&27,&36,&39,&41,&45,&46,&48,&49,&52,&60 \\ 
6,&8,&10,&11,&17,&21,&28,& 29,&31,&35,&37,&38,&43,&54,&59
\end{array} \\
& \begin{array}{c} \text{Prime number: }73 \\  g=5, f=27 \end{array} \\
\begin{array}{l} \text{Set of residues modulo }73 \\ \text{Box A} \\ \text{Box B} \\ \text{Box C} \\ \text{Box D} \end{array} & 
\begin{array}{|rrrrrrrrrrrrrrrrrr} 
1,&2,&4,&8,&9,&16,&18,&32,&36,&37,&41,&55,&57,&64,&65,&69,&71,&72 \\ 
5,&7,& 10,&14,&17,&20,&28,&33,&34,&39,&40,&45,&53,&56,&59,&63,&66,&68 \\ 
3,& 6,& 12,&19,&23,&24,&25,&27,&35,&38,&46,&48,&49,&50,&54,&61,&67,&70 \\ 
11,&13,&15,&21,&22,&26,&29,& 30,&31,&42,&43,&44,&47,&51,&52,&58,&60,&62
\end{array}  \\
\end{aligned}\]\clearpage\noindent
%

\[\begin{aligned}
& \begin{array}{c} p=89 \\  g=3, f=34 \end{array} \\
\begin{array}{l} A \\ \vspace{0.1mm}\\ B \\ \hspace{0.1mm}\\ C \\\hspace{0.1mm}\\ D \\ \hspace{0.1mm}\end{array} & 
\begin{array}{|rrrrrrrrrrrrrrrrrrrrrr} 
1,&2,&4,&8,&11,&16,&22,&25,&32,&39,&44,&45,&50,&57,&64,&67,&73,&78,&81,&85,\\
&&87,&88 \\ 
3,&6,&7,& 12,&14,&23,&24,&28,&33,&41,&43,&46,&48,&56,&61,&65,&66,&75,&77,&82 \\ 
&&83,&86 \\
5,& 9,& 10,&17,&18,&20,&21,&34,&36,&40,&42,&47,&49,&53,&55,&68,&69,&71,&72,&79 \\ 
&&80,&84&\\
13,&15,&19,&26,&27,&29,& 30,&31,&35,&37,&38,&51,&52,&54,&58,&59,&60,&62,&63,&70 \\
&&74,&76
\end{array} \\ 
& \begin{array}{c} p=97 \\  g=5, f=22 \end{array} \\
\begin{array}{l} A \\ \vspace{0.1mm}\\ B \\ \hspace{0.1mm}\\ C \\\hspace{0.1mm}\\ D \\ \hspace{0.1mm}\end{array} & 
\begin{array}{|rrrrrrrrrrrrrrrrrrrrrr} 
1,&4,&6,&9,&16,&22,&24,&33,&35,&36,&43,&47,&50,&54,&61,&62,&64,&73,&75,&81,\\
&&88,&91,&93,&96 \\ 
5,&13,&14,& 17,&19,&20,&21,&23,&29,&30,&41,&45,&52,&56,&67,&68,&74,&76,&77,&78 \\ 
&&80,&83,&84,&92 \\
2,& 3,& 8,&11,&12,&18,&25,&27,&31,&32,&44,&48,&49,&53,&65,&66,&70,&72,&79,&85 \\ 
&&86,&89,&94,&95\\
7,&10,&15,&26,&28,&34,& 37,&38,&39,&40,&42,&46,&51,&55,&57,&58,&59,&60,&63,&69 \\
&&71,&82,&87,&90
\end{array} 
\end{aligned}\]
%

\subsection*{12.}
 
Since the number 2 is a quadratic residue of all prime numbers of the form \(8n+1\), and a non-residue of all prime numbers of the form \(8n+5\), it will be found in classes \(A\) or \(C\) for the former form's prime moduli, and in classes \(B\) or \(D\) for the latter form's prime moduli. Since the distinction between classes \(B\) and \(D\) is not essential, indeed depending only on the choice of the number \(f\), we will temporarily set aside the moduli for the form \(8n+5\). By subjecting the moduli for the form \(8n+1\) to \textit{induction}, we find that the number 2 belongs to \(A\) for \(p=73\), 89, 113, 233, 257, 281, 337, 353, etc.; on the contrary, 2 belongs to \(C\) for \(p=17\), 41, 97, 137, 193, 241, 313, 401, 409, 433, 449, 457, etc.
 
Moreover, since the number \(-1\) is a biquadratic residue for a prime modulus of the form \(8n+1\), it is evident that \(-2\) always belongs to the same class as \(+2\).
%

\subsection*{13.}

If the examples of the preceding article are compared with each other, at first glance at least, it seems that no simple criterion presents itself by which the earlier modules could be distinguished from the later ones. Nevertheless, \textit{two} such criteria are given, distinguished by their elegance and remarkable simplicity, to the consideration of which the following observations will pave the way.
%

\(\)
Modulus \(p\) as a prime number of the form\(8 n+1\), will be reducible, and indeed in only one way, to the form \(a a+2 b b\) (\textit{Disquiss. Arithm.} art. 182, II); we will assume the positive roots \(a\), \(b\). Clearly \(a\) will be odd, and \(b\) even; let us assume \(b=2^{\lambda} c\), so that \(c\) is odd. Now we observe

I. when \(p \equiv a a \pmod{{c}}\) that \(p\) is a quadratic residue of \(c\), and therefore also of each prime factor into which \(c\) is resolved: therefore, by the fundamental theorem, each of these prime factors will be a quadratic residue of \(p\), and therefore also their product \(c\) will be a quadratic residue of \(p\). Since this also holds for the number 2, it is clear that \(b\) is a quadratic residue of \(p\), and therefore \(b b\), as well as \(-b b\), is a biquadratic residue.

II. Hence \(-2 b b\) must be referred to the same class in which the number 2 is found; therefore, since \(a a \equiv-2 b b\), it is clear that 2 is found either in class \(A\) or in class \(C\), depending on whether \(a\) is a quadratic residue of \(p\) or not.

III. Now let us suppose that \({a}\) has been resolved into its prime factors, among which those which are of the form \(8 m+1\) or \(8 m+7\) are denoted by \(\alpha\), \(\alpha^{\prime}\), \(\alpha^{\prime \prime}\) etc., and those which are of the form \(8 m+3\) or \(8 m+5\) by \(\beta\), \(\beta^{\prime}\), \(\beta^{\prime \prime}\) etc.: let the number of the latter be \(=\mu\). Since \(p \equiv 2 b b \pmod{a}\), \(p\) will be a quadratic residue of those prime factors of \(a\) of which 2 is a quadratic residue, i.e. the factors \(\alpha\), \(\alpha^{\prime}\), \(\alpha^{\prime \prime}\) etc.; and a non-quadratic residue of those factors of which 2 is not a quadratic residue, i.e. the factors \(\beta\), \(\beta^{\prime}\), \(\beta^{\prime \prime}\) etc. Hence, conversely, by the fundamental theorem, each of \(\alpha\), \(\alpha^{\prime}\), \(\alpha^{\prime}\), \(\alpha^{\prime \prime}\) etc. will be a quadratic residue of \(p\), and each of \(\beta\), \(\beta^{\prime}\), \(\beta^{\prime \prime}\) etc. will be a non-quadratic residue. From this it follows that the product \(a\) will be a quadratic residue of \(p\), or non-residue, depending on whether \(\mu\) is even or odd.

IV. But it is easily confirmed that the product of all \(\alpha\), \(\alpha^{\prime}\), \(\alpha^{\prime \prime}\) etc. will be of the form \(8 m+1\) or \(8 m+7\), and the same holds for the product of all \(\beta\), \(\beta^{\prime}\), \(\beta^{\prime \prime}\) etc., if their number is even, so that in this case the product \(a\) must necessarily be of the form \(8 m+1\) or \(8 m+7\); on the other hand, the product of all \(\beta\), \(\beta^{\prime}\), \(\beta^{\prime \prime}\) etc., whenever their number is odd, will be of the form \(8 m+3\) or \(8 m+5\), and the same holds in this case for the product \(a\).
%

From all these, the elegant theorem is collected:

\textit{As often as \(a\) is of the form \(8m+1\) or \(8m+7\), the number \(2\) will be found in complex \(A\); but as often as \(a\) is of the form \(8m+3\) or \(8m+5\), the number \(2\) will be found in complex \(C\).}

This is confirmed by the examples enumerated in the preceding article; for the former moduli are thus resolved: \(73=1+2\cdot36\), \(89=81+2\cdot4\), \(113=81+2\cdot16\), \(233=225+2\cdot4\), \(257=225+2\cdot16\), \(281=81+2\cdot100\), \(337=49+2\cdot144\), \(353=225+2\cdot64\); but the latter thus: \(17=9+2\cdot4\), \(41=9+2\cdot16\), \(97=25+2\cdot36\), \(137=9+2\cdot64\), \(193=121+2\cdot36\), \(241=169+2\cdot36\), \(313=25+2\cdot144\), \(401=9+2\cdot196\), \(409=121+2\cdot144\), \(433=361+2\cdot36\), \(449=441+2\cdot4\), \(457=169+2\cdot144\).
%

\subsection*{14.}

When the factorization of the number \(p\) into a square and a sum of two squares as remarkable as that with the number 2 has been produced, it seems worthwhile to try whether the factorization into two squares, to which the number \(p\) is equally subject, may provide a similar success. Behold, therefore, the factorizations of the numbers \(p\) for which 2 belongs to the class
\[\begin{array}{c|c}
A & C  \\[2pt]
\hline 
\begin{aligned}
9&+64  \\
25&+64  \\
49&+64 \\
169&+64\\
1&+256 \\
25&+256\\
81&+256\\
289&+64
 \end{aligned}
&
\begin{aligned}
 1&+16\\
 25&+16\\
 81&+16\\
 121&+16\\
 19&+144\\
225&+16\\
169&+144\\
1&+400\\
9&+400\\
289&+144\\
49&+400\\
441&+16 
\end{aligned}
\end{array}\]
%

\begin{enumerate}
    \item First of all we observe that, given \(p\) is divided into two squares, one square must be odd, which we designate \(=a a\), and the other even, which we designate \(=b b\). Since \(a {a}\) takes the form \(8 n+1\), it is clear that the values of \(p\) of the form \(8 n+5\) correspond to the odd values of \(b\), which are excluded by our induction, since they would have the number 2 in class \(B\) or \(D\). For the values of \(p\) of the form \(8 n+1\), the value of \(b\) must also be even, and if we have faith in the induction presented before our eyes, the number 2 must be assigned to class \(A\) for all moduli for which \(b\) takes the form \(8 n\), and to class \(C\) for all moduli for which \(b\) takes the form \(8 n+4\). But this theorem is of much deeper investigation than that which we have unfolded in the preceding article, and multiple preliminary investigations must be undertaken for the demonstration, regarding the order in which the numbers of the sets \(A, B, C, D\) follow each other.
\end{enumerate}
%

\subsection*{15.}

Let us designate the set of numbers from complex \(A\), which is immediately followed by the number from complex \(A\), \(B\), \(C\), \(D\), by \((00)\), \((01)\), \((02)\), \((03)\); similarly, the set of numbers from complex \(B\) that is followed by the numbers from complex \(A\), \(B\), \(C\), \(D\) respectively by \((10)\), \((11)\), \((12)\), \((13)\); and similarly let there be in complex \(C\) the numbers \((20)\), \((21)\), \((22)\), \((23)\), and in complex \(D\) indeed the numbers \((30)\), \((31)\), \((32)\), \((33)\) that are followed by the numbers from complex \(A\), \(B\), \(C\), \(D\). We propose to determine a priori these sixteen sets. In order that the readers can compare general reasoning with examples, it seemed good to add here the numerical values of the terms of the scheme 
\[
\mathcal{S} = 
\begin{array}{c}
\begin{bmatrix}
(00), & (01), & (02), & (03) \\
(10), & (11), & (12), & (13) \\
(20), & (21), & (22), & (23) \\
(30), & (31), & (32), & (33)
\end{bmatrix}
\end{array}
\]
for each modulus for which we have given classifications in article 11.
\[
\mathcal{S} = 
\begin{array}{c|c|c|c}
p=5 & p=13 & p=17 & p=29 \\
0, 1, 0, 0 & 0, 1, 2, 0 & 0, 2, 1, 0 & 2, 3, 0, 2 \\
0, 0, 0, 1 & 1, 1, 0, 1 & 2, 0, 1, 1 & 1, 1, 2, 3 \\
0, 0, 0, 0 & 0, 1, 0, 1 & 1, 1, 1, 1 & 2, 1, 2, 1 \\
0, 0, 1, 0 & 1, 0, 1, 1 & 0, 1, 1, 2 & 1, 2, 3, 1
\end{array}
\]

\[
\mathcal{S} = 
\begin{array}{c|c|c|c}
p=37 & p=41 & p=53 & p=61 \\
2, 1, 2, 4 & 0, 4, 3, 2 & 2, 3, 6, 2 & 4, 3, 2, 6 \\
2, 2, 4, 1 & 4, 2, 2, 2 & 4, 4, 2, 3 & 3, 3, 6, 3 \\
2, 2, 2, 2 & 3, 2, 3, 2 & 2, 4, 2, 4 & 4, 3, 4, 3 \\
2, 4, 1, 2 & 2, 2, 2, 4 & 4, 2, 3, 4 & 3, 6, 3, 3 \\
\hline
p=73 & p=89 & p=97   \\
5, 6, 4, 2 & 3, 8, 6, 4 & 2, 6, 7, 8   \\
6, 2, 5, 5 & 8, 4, 5, 5 & 6, 8, 5, 5   \\
4, 5, 4, 5 & 6, 5, 6, 5 & 7, 5, 7, 5   \\
2, 5, 5, 6 & 4, 5, 5, 8 & 8, 5, 5, 6   \\
\end{array}
\]

Since the moduli of the form \(8n+1\) and \(8n+5\) behave in different ways, each must be treated separately: we will begin with the former.
%

\subsection*{16.}
 
The symbol \((00)\) indicates, in how many different ways the equation \(\alpha+1=\alpha^{\prime}\) can be satisfied, denoting \(\alpha\), \(\alpha^{\prime}\) as indefinite numbers in the complex \(A\). Since for the form of the modulus \(8n+1\), which we understand here, \(\alpha^{\prime}\) and \(p-\alpha^{\prime}\) belong to the same complex, we will say more succinctly that \((00)\) expresses the multitude of different ways to satisfy the equation \(1+\alpha+\alpha^{\prime}=p\): clearly, instead of this equation, the congruence \(1+\alpha+\alpha^{\prime} \equiv 0 \pmod{p}\) can also be used.
%

\begin{center}
\begin{tabular}{ll}
\((01)\) indicates the multitude of solutions of the congruence & \(1+\alpha+\beta \equiv 0\pmod{p}\) \\
\((02)\) the multitude of solutions of the congruence& \(1+\alpha+\gamma \equiv 0\) \\
\((03)\) the multitude of solutions of the congruence& \(1+\alpha+\delta \equiv 0\) \\
\((11)\) the multitude of solutions of the congruence& \(1+\beta+\beta^{\prime} \equiv 0\) etc. \\
\end{tabular}
\end{center}
by expressing indefinitely through \(\beta\) and \(\beta^{\prime}\) the numbers from the complex \(B\), through \(\gamma\) the numbers from the complex \(C\), and through \(\delta\) the numbers from the complex \(D\). Hence we immediately gather the following six equations:
\[(01)=(10), \quad (02)=(20), \quad (03)=(30), \quad (12)=(21), \quad(13)=(31), \quad(23)=(32)\]
%

From any given solution of the congruence \(1+\alpha+\beta \equiv 0\), we seek the solution of the congruence \(1+\delta+\delta^{\prime} \equiv 0\), taking for \(\delta\) a number within the limits \(1 \ldots p-1\)*\footnote{*The limits \((1\ldots p-1)\) refer to the numbers between 1 and \(p-1\).}, such that \(\beta \delta \equiv 1\) (which will clearly be from the complex \(D\)), and for \(\delta^{\prime}\) the minimum residue of the product \(\alpha \delta\) (which will also be from the complex \(D\)); it is accordingly evident that the regression from a given solution of the congruence \(1+\delta+\delta^{\prime} \equiv 0\) to the solution of the congruence \(1+\alpha+\beta \equiv 0\) is apparent, if \(\beta\) is taken in such a way that \(\beta \delta \equiv 1\), and simultaneously \(\alpha \equiv \beta \delta^{\prime}\). Hence, we conclude that both congruences enjoy an equal multitude of solutions, that is, \((01) = (33)\).

In a similar manner, from the congruence \(1+\alpha+\gamma \equiv 0\), we deduce \(\gamma^{\prime}+\gamma^{\prime \prime}+1 \equiv 0\), if \(\gamma^{\prime}\) is taken from the complex \(C\) in such a way that \(\gamma \gamma^{\prime} \equiv 1\), and \(\gamma^{\prime \prime}\) is congruent to the product \(\alpha \gamma^{\prime}\) from the same complex. Hence, we easily infer that these two congruences admit an equal multitude of solutions, that is, \((02) = (22)\).

Similarly, from the congruence \(1+\alpha+\delta \equiv 0\), we deduce \(\beta+\beta^{\prime}+1 \equiv 0\), taking \(\beta\), \(\beta^{\prime}\) in such a way that \(\beta \delta \equiv 1, \beta \alpha \equiv \beta^{\prime}\), and therefore \((03) = (11)\).

Finally, from the congruence \(1+\beta+\gamma \equiv 0\), in a similar manner, we derive both the congruence \(\delta+1+\beta^{\prime} \equiv 0\) and also the congruence \(\gamma^{\prime}+\delta^{\prime}+1 \equiv 0\), and hence we conclude \((12) = (13) = (23)\).

We have thus obtained, among our sixteen unknowns, eleven equations, such that they can be reduced to five, and the scheme \(S\) can thus be exhibited as follows:
\[
\begin{array}{llll}
h & i & k & l \\
i & l & m & m \\
k & m & k & m \\
l & m & m & i
\end{array}
\]
%

\begin{align*}
& \text{Three new conditional equations are easily added. For when any complex number } A \text{, except the last } p-1,\\
& \text{must be followed by a number from among the complex numbers } A, B, C \text{, or } D, \text{we will have}\\
& (00)+(01)+(02)+(03)=2n-1 \text{ and similarly}\\
& \begin{aligned}
&(10)+(11)+(12)+(13)=2n, \\
&(20)+(21)+(22)+(23)=2n, \text{ and} \\
&(30)+(31)+(32)+(33)=2n.
\end{aligned}
\end{align*}
With the signs introduced, the first three equations provide:
\[
\begin{aligned}
h+i+k+l & =2n-1, \\
i+l+2m & =2n, \\
k+m & =n.
\end{aligned}
\]

Fourth becomes identical to the second. Help from these equations allows the elimination of three unknowns, by which means all sixteen are now reduced to two.
%

\subsection*{17.}

But in order to obtain a complete determination, it will be convenient to investigate the number of solutions of the congruence
\[1+\alpha+\beta+\gamma \equiv 0\pmod{p}\]
where \(\alpha\), \(\beta\), \(\gamma\) denote indefinite numbers from the complex \(A\), \(B\), \(C\). Clearly the value \(\alpha=p-1\) is not admissible, since it's not possible to have \(\beta+\gamma \equiv 0\). Therefore, by substituting for \(\alpha\) the remaining values, the values of \(h\), \(i\), \(k\), \(l\) will be obtained for \(1+\alpha\) relating to \(A\), \(B\), \(C\), \(D\) respectively. For any *given* value of \(1+\alpha\) pertaining to \(A\), say for \(1+\alpha=\alpha^{0}\), the congruence \(\alpha^{0}+\beta+\gamma \equiv 0\) will admit as many solutions as the congruence \(1+\beta^{\prime}+\gamma^{\prime} \equiv 0\) does (simply by setting \(\beta \equiv \alpha^{0} \beta^{\prime}\), \(\gamma \equiv \alpha^{0} \gamma^{\prime}\)), i.e. \(m\) solutions. Likewise, for any *given* value of \(1+\alpha\) pertaining to \(B\), say for \(1+\alpha=\beta^{0}\), the congruence \(\beta^{0}+\beta+\gamma \equiv 0\) will have as many solutions as the congruence \(1+\alpha^{\prime}+\beta^{\prime} \equiv 0\) does (by setting \(\beta \equiv \beta^{0} \alpha^{\prime}\), \(\gamma \equiv \beta^{0} \beta^{\prime}\)), i.e. \(i\) solutions. Similarly, for any *given* value of \(1+\alpha\) pertaining to \(C\), say for \(1+\alpha=\gamma^{0}\), the congruence \(\gamma^{0}+\beta+\gamma \equiv 0\) can be solved in as many diverse ways as the congruence \(1+\delta+\alpha^{\prime} \equiv 0\) does (by setting \(\beta \equiv \gamma^{0} \delta\), \(\gamma \equiv \gamma^{0} \alpha^{\prime}\)), i.e. the number of solutions will be \(l\) solutions. Finally, for any *given* value of \(1+\alpha\) pertaining to \(D\), say for \(1+\alpha=\delta^{0}\), the congruence \(\delta^{0}+\beta+\gamma \equiv 0\) will have as many solutions as the congruence \(1+\gamma^{\prime}+\delta^{\prime} \equiv 0\) does (by setting \(\beta \equiv \delta^{0} \gamma^{\prime}\), \(\gamma \equiv \delta^{0} \delta^{\prime}\)), i.e. (23) \(=m\) solutions. Therefore, taking all together, it is clear that the congruence \(1+\alpha+\beta+\gamma \equiv 0\) will admit
\[h m+i i+k l+l m\]
diverse solutions.
%

\(\beta\) singularly produce the sum \(1+\beta\) or \((10)\), \((11)\), \((12)\), \((13)\) or \(i\), \(l\), \(m\), \(m\) as values corresponding to \(A\), \(B\), \(C\), \(D\) and for any given value of \(1+\beta\) corresponding to these complex numbers, we admit diverse congruence \(\alpha+\beta+\gamma \equiv 0\) or \((02)\), \((31)\), \((20)\), \((13)\) or \(k\), \(m\), \(k\), \(m\) solutions, such that the multitude of all solutions becomes
\[=i k+l m+k m+m m\]
We come to the same value if we extend the consideration of the values of the sum \(1+\gamma\).
%

\subsection*{18.}

From this double expression of the same quantity, we obtain the equation:
\[0=h m+i i+k l-i k-k m-m m\]
and hence, by eliminating \(h\) with the aid of the equation \(h=2 m-k-1\),
\[0=(k-m)^{2}+i i+k l-i k-k k-m\]
But the last two equations of article 16 provide \(k=\frac{1}{2}(l+i)\). By substituting this value for \(k\), \(i i+k l-i k-k k\) becomes \(\frac{1}{4}(l-i)^{2}\), and therefore the preceding equation, multiplied by 4, becomes
\[0=4(k-m)^{2}+(l-i)^{2}-4 m\]
Hence, since \(4 m=2(k+m)-2(k-m)=2 n-2(k-m)\), it follows that
\[2 n=4(k-m)^{2}+2(k-m)+(l-i)^{2}\]
or
\[8 n+1=(4(k-m)+1)^{2}+4(l-i)^{2}\]
Therefore, by setting
\[4(k-m)+1=a,\quad 2 l-2 i=b\]
we will have
\[p=a a+b b\]
%

However, it is well known that \(p\) can be expressed as the sum of two squares in only one way, where one of them must be odd and represented by \(a a\), and the other even and represented by \(b b\), so that \(a a\) and \(b b\) are numbers determined from the axis. Also, \(a\) itself will be a completely determined number; for the square root must be taken as positive or negative, according to whether the positive root is of the form \(4 M+1\) or \(4 M+3\). We will soon discuss the determination of the sign of \(b\) itself.
%

\clearpage\noindent% 83
I combine these new equations with the last three from Article 16, so that the five numbers \(h\), \(i\), \(k\), \(l\), \(m\) are completely determined by \(a\), \(b\), and \(n\) in the following way:
\[\begin{aligned}
& 8 h=4 n-3 a-5 \\
& 8 i=4 n+a-2 b-1 \\
& 8 k=4 n+a-1 \\
& 8 l=4 n+a+2 b-1 \\
& 8 m=4 n-a+1
\end{aligned}\]
%

If we want to introduce the modulus \(n\) in place of \(p\), the scheme \({S}\), with each term multiplied by 16 to avoid fractions, is as follows:

\[
\begin{array}{l|l|l|l}
p-6 a-11 & p+2 a-4 b-3 & p+2 a-3 & p+2 a+4 b-3 \\
p+2 a-4 b-3 & p+2 a+4 b-3 & p-2 a+1 & p-2 a+1 \\
p+2 a-3 & p-2 a+1 & p+2 a-3 & p-2 a+1 \\
p+2 a+4 b-3 & p-2 a+1 & p-2 a+1 & p+2 a-4 b-3
\end{array}
\]
%

\subsection*{19.}
It remains for us to show how to assign the sign \(b\) to itself\(^*\). Already above, in article 10, we have pointed out that the distinction between the sets \(B\) and \(D\), which is not essential in itself, depends on the choice of the number \(f\), for which one of the congruence roots \(x x \equiv-1\) must be taken, and they are interchangeable if one of the roots instead of the other is adopted\(^*\). Now, since an inspection of the diagram just presented shows that a similar permutation coheres with the change of the sign \(b\), it is permissible to foresee that there must be a connection between the sign \(b\) and the number \(f\). In order for us to understand this, we first of all observe that if, denoting \(\mu\) as a non-negative integer, all the numbers \(1\), \(2\), \(3 \ldots p-1\) are taken for \(z\) and the operations are performed modulo \(p\), either \(\Sigma z^{\mu} \equiv 0\) or \(\Sigma z^{\mu} \equiv-1\) will occur, according as \(\mu\) is not divisible by \(p-1\) or is divisible. The latter part of the theorem is thereby evident, which states that for the value of \(\mu\) divisible by \(p-1\), we have \(z^{\mu} \equiv 1\); as for the former part, we thus demonstrate it. Denoting \(g\) as the primitive root, all \(z\) agrees with the smallest residues of all \(g^{y}\), taking all numbers \(0\), \(1\), \(2\), \(3 \ldots p-2\) as \(y\), and therefore \(\Sigma z^{\mu} \equiv \Sigma g^{\mu y}\). But we have
\[\Sigma g^{\mu y}=\frac{g^{\mu(p-1)-1}}{g^{\mu}-1}, \text{ hence } (g^{\mu}-1) \Sigma z^{\mu} \equiv g^{\mu(p-1)}-1 \equiv 0\]
Thus, it follows that, since for the value of \(\mu\) not divisible by \(p-1\), \(g^{\mu}\) cannot be congruent to 1, i.e., \(g^{\mu}-1\) cannot be divisible by \(p\), \(\Sigma z^{\mu} \equiv 0\). Q. E. D.\footnote{translated from Latin: *Note: The source text uses footnotes in a manner specific to the conventions of mathematical writing in Latin.}
%

\text{Now, if the power } \((z^{4}+1)^{\frac{1}{4}(p-1)}\) \text{ can be expanded using the binomial theorem, according to the previous lemma, we will have}
\[\Sigma(z^{4}+1)^{\frac{1}{4}(p-1)} \equiv-2\pmod{p}\]
\text{But the minimal residues of all } \(z^{4}\) \text{ show all the numbers } A, \text{ occurring every fourth time. Thus, among the minimal residues of } \(z^{4}+1\)
\[\begin{aligned}
& 4(00) \text{ to } A \\
& 4(01) \text{ to } B \\
& 4(02) \text{ to } C \\
& 4(03) \text{ to } D
\end{aligned}\]
\text{belonging, and all four will be \(=0\) (suppose for } \(z^{4} \equiv p-1\)). \text{Thus, by considering the complex criteria } A, B, C, D, \text{ we deduce}
\[\Sigma(z^{4}+1)^{\frac{1}{4}(p-1)} \equiv 4(00)+4 f(01)-4(02)-4 f(03)\]
\text{and therefore}
\[-2 \equiv 4(00)+4 f(01)-4(02)-4 f(03)\]
\text{or substituting values found in the previous section for } (00), (01) \text{ etc.,}
\[-2 \equiv-2 a-2-2 b f\]
\text{Therefore, we conclude that }\( a+b f \equiv 0 \) \text{ must always be satisfied, or, multiplying by } f,
\[b \equiv a f\]
\text{This congruence serves the determination of the sign of } \(b\) \text{ if the number } \(f\) \text{ has already been chosen, or the determination of the number } \(f\) \text{ if the sign of } \(b\) \text { is prescribed elsewhere.}
%

\subsection*{20.}

After we have completely solved our problem for moduli of the form \(8n+1\), we proceed to the other case, where \(p\) is of the form \(8n+5\): we will be able to complete this more briefly because all the reasoning differs little from the preceding.
%

\begin{center}
\begin{tabular}{c|c}
signum & \begin{tabular}{c}
denote the multiplicity \\
of the solutions of the congruence \\
\end{tabular} \\
\hline
\((00)\) & \(1+\alpha+\gamma \equiv 0\) \\
\((01)\) & \(1+\alpha+\delta \equiv 0\) \\
\((02)\) & \(1+\alpha+\alpha^{\prime} \equiv 0\) \\
\((03)\) & \(1+\alpha+\beta \equiv 0\) \\
\((10)\) & \(1+\beta+\gamma \equiv 0\) \\
\((11)\) & \(1+\beta+\delta \equiv 0\) \\
\((12)\) & \(1+\beta+\alpha \equiv 0\) \\
\((13)\) & \(1+\beta+\beta^{\prime} \equiv 0\) \\
\((20)\) & \(1+\gamma+\gamma^{\prime} \equiv 0\) \\
\((21)\) & \(1+\gamma+\delta \equiv 0\) \\
\((22)\) & \(1+\gamma+\alpha \equiv 0\) \\
\((23)\) & \(1+\gamma+\beta \equiv 0\) \\
\((30)\) & \(1+\delta+\gamma \equiv 0\) \\
\((31)\) & \(1+\delta+\delta^{\prime} \equiv 0\) \\
\((32)\) & \(1+\delta+\alpha \equiv 0\) \\
\((33)\) & \(1+\delta+\beta \equiv 0\) \\
\end{tabular}
\end{center}
hence immediately we have six equations:
\[(00)=(22), \quad (01)=(32), \quad (03)=(12), \quad (10)=(23), \quad (11)=(33), \quad (21)=(30)\]
%

By multiplying the congruence \(1+\alpha+\gamma \equiv 0\) by the number \(\gamma^{\prime}\) and a complex number \(C\) chosen in such a way that \(\gamma \gamma^{\prime} \equiv 1\), and taking for \(\gamma^{\prime \prime}\) the least residue of the product \(\alpha \gamma^{\prime}\), which will evidently also be to be counted as a complex number, we obtain \(\gamma^{\prime}+\gamma^{\prime \prime}+1 \equiv 0\), from which we conclude \((00)=(20)\).

Equations \((01)=(13)\), \((03)=(31)\), \((10)=(11)=(21)\) are obtained in a completely similar manner.
%

\[\begin{array}{llll}
h,&  i,& k,& l \\
m,& m,& l,& i \\
h,& m,& h,& m \\
m,& l,&  i,&  m
\end{array}\]

Furthermore, we have the equations
\[\begin{aligned}
& (00)+(01)+(02)+(03)=2 n+1 \\
& (10)+(11)+(12)+(13)=2 n+1 \\
& (20)+(21)+(22)+(23)=2 n \\
& (30)+(31)+(32)+(33)=2 n+1
\end{aligned}\]
or, using the introduced signs, these three (I):
\[\begin{aligned}
h+i+k+l & =2 n+1 \\
2 m+i+l & =2 n+1 \\
h+m & =n
\end{aligned}\]
Therefore, with the help of these equations, we can now reduce our unknowns to two.
%

\text{We will derive the remaining equations from consideration of the multiplicity of congruence solutions }\(\alpha\), \(\beta\), \(\gamma \equiv 0\)\text{, representing} \(A\), \(B\), \(C\) \text{as indefinite numbers from complex numbers }A, B, C\text{. Namely, by considering} \textit{firstly}, \(1+\alpha\) \text{ yielding} h, i, k, l \text{numbers corresponding to} \(A\), \(B\), \(C\), \(D\) \text{, and for any given value of} \(\alpha\) \text{, in these four cases, solutions} \(m\), \(l\), \(i\), \(m\) \text{will be obtained, the total number of solutions will be} 
\[\begin{aligned}
& =hm+il+ik+lm
\end{aligned}\]
\text{Secondly, as} \(1+\beta\) \text{ yields} \(m\), \(m\), \(l\), \(i\) \text{numbers corresponding to} \(A\), \(B\), \(C\), \(D\) \text{, and for any given value of} \(\beta\) \text{, in these four cases, solutions} \(h\), \(m\), \(h\), \(m\) \text{exist, the total number of solutions will be} 
\[\begin{aligned}
& =hm+mm+hl+im
\end{aligned}\]
\text{from which we derive the equation} 
\[\begin{aligned}
& 0=mm+hl+im-il-ik-lm
\end{aligned}\]
\text{which, with the help of the equation} k=2m-h\text{, from (I), transforms into this:} 
\[\begin{aligned}
& 0=mm+hl+hi-il-im-lm
\end{aligned}\]
\text{Now, from the equations in I, we also have} l+i=1+2h\text{, whence}
\[\begin{aligned}
& 2i=1+2h+(i-l) \\
& 2l=1+2h-(i-l)
\end{aligned}\]
\text{Substituting these values into the preceding equation, we get:} 
\[\begin{aligned}
& 0=4mm-4m-1-8hm+4h^2+(i-l)^2
\end{aligned}\]
\text{If finally we substitute }4m\text{ here with }2(h+m)-2(h-m)\text{, or, due to the last equation in I, }2n-2(h-m)\text{, we obtain:} 
\[\begin{aligned}
& 0=4(h-m)^2-2n+2(h-m)-1+(i-l)^2
\end{aligned}\]
\text{and thus} 
\[\begin{aligned}
& 8n+5=(4(h-m)+1)^2+4(i-l)^2
\end{aligned}\]
\text{Setting therefore} 
\[\begin{aligned}
& 4(h-m)+1=a, \quad 2i-2l=b
\end{aligned}\]
\text{will be} 
\[\begin{aligned}
& p=a^2+b^2
\end{aligned}\]
%

\text{When, in this case, only one prime number \(p\) can be partitioned in two squares in only one way: one even, the other odd, \(a\) and \(b\) will be completely determined numbers; for it is evident that \(a\) must be the square of an odd number, and \(b\) of an even number. Moreover, the} \textit{sign} \text{of \(a\) must be established in such a way that \(a \equiv 1\pmod{4}\), and the sign of \(b\) in such a way that \(b \equiv af\pmod{p}\), as is easily proven by reasonings similar to those which we employed in the previous article.}
%

\[\begin{aligned}
&h = \frac{1}{8}(4n + a - 1) \\
&i = \frac{1}{8}(4n + a + 2b + 3) \\
&k = \frac{1}{8}(4n - 3a + 3) \\
&l = \frac{1}{8}(4n + a - 2b + 3) \\
&m = \frac{1}{8}(4n - a + 1)
\end{aligned}\]
or if we prefer expressions in terms of \(p\), the terms of scheme \(S\) multiplied by 16 will be as follows:
\[
\begin{array}{l|l|l|l}
p+2a-7 & p+2a+4b+1 & p-6a+1 & p+2a-4b+1 \\
p-2a-3 & p-2a-3 & p+2a-4b+1 & p+2a+4b+1 \\
p+2a-7 & p-2a-3 & p+2a-7 & p-2a-3 \\
p-2a-3 & p+2a-4b+1 & p+2a+4b+1 & p-2a-3
\end{array}
\]
%

\subsection*{21}

After solving our problem, we return to the main investigation, embracing the complete determination to which the number 2 belongs, now to be approached.

I. Whenever \(p\) is of the form \(8n+1\), it is already clear that the number 2 is found either in the complex \(A\) or in the complex \(C\). In the former case, it is easily seen that the numbers \(\frac{1}{2}(p-1)\) and \(\frac{1}{2}(p+1)\) also belong to \(A\), and in the latter case, they belong to \(C\). Now consider, if \(\alpha\) and \(\alpha+1\) are consecutive numbers in the complex \(A\), then \(p-\alpha-1\) and \(p-\alpha\) are also such numbers, or, which is the same, numbers of the complex \(A\) followed by a number from the same complex, are always associated in pairs, \((\alpha\) and \(p-1-\alpha)\). The number of such numbers, therefore, \((00)\), will always be even, unless there exists a number associated with itself, i.e. unless \(\frac{1}{2}(p-1)\) belongs to \(A\), in which case this quantity will be odd. Hence we conclude that \((00)\) is odd whenever 2 belongs to the complex \(A\), and even whenever 2 belongs to \(C\). But we have
\[16(00)=a a+b b-6 a-11\]
or, setting \(a=4q+1\), \(b=4r\) (see art. 14),
\[(00)=q q-q+r r-1\]
Therefore, since \(q q-q\) is clearly always even, \((00)\) will be odd or even, according as \(r\) is even or odd, and thus 2 will belong to \(A\) or \(C\), according as \(b\) is of the form \(8m\) or \(8m+4\). Which is the theorem itself, found by induction in art. 14.

II. We can also complete the other case, where \(p\) is of the form \(8n+5\). The number 2 here belongs to either \(B\) or \(D\), and it is easily seen that in the former case \(\frac{1}{2}(p-1)\) belongs to \(B\), \(\frac{1}{2}(p+1)\) to \(D\), and in the latter case \(\frac{1}{2}(p-1)\) belongs to \(D\), \(\frac{1}{2}(p+1)\) to \(B\). Now consider, if \(\beta\) is a number in \(B\) such that it is followed by a number in \(D\), then the number \(p-\beta-1\) will also be in \(B\) and \(p-\beta\) in \(D\), i.e. numbers of this property always appear in pairs. Therefore the quantity of these will be (13), even, except in the case when one of them is associated with itself, i.e. when \(\frac{1}{2}(p-1)\) belongs to \(B\) and \(\frac{1}{2}(p+1)\) to \(D\); then of course (13) will be odd. Hence we conclude that (13) is even whenever 2 belongs to \(D\), and odd whenever 2 belongs to \(B\). But we have
\[16(13)=a a+b b+2 a+4 b+1\]
or, setting \(a=4q+1\), \(b=4r+2\),
\[(13)=q q+q+r r+2 r+1\]
Therefore, (13) will be odd whenever \(r\) is even; conversely, (13) will be even whenever \(r\) is odd: hence we conclude that 2 belongs to \(B\) whenever \(b\) is of the form \(8m+2\), and to \(D\) whenever \(b\) is of the form \(8m+6\).
%

The sum of these investigations can be stated as follows:

\textit{The number 2 belongs to the set \(A\), \(B\), \(C\), or \(D\), according to whether the number \(\frac{1}{2} b\) is of the form \(4m\), \(4m+1\), \(4m+2\), or \(4m+3\).}
%

\subsection*{22.}

In the \textit{Disquisitionibus Arithmeticis} we explained the general theory of the division of the circle, as well as the solution of the equation \(x^{p}-1=0\), and among other things, we taught that if \(\mu\) is a divisor of the number \(p-1\), the function \(\frac{x^{p}-1}{x-1}\) can be resolved into \(\mu\) factors of order \(\frac{p-1}{\mu}\), with the help of an auxiliary equation of order \(\mu\). In addition to the general theory of this resolution, we separately considered special cases where \(\mu=2\) or \(\mu=3\) in that work, in articles \(356-358\), and we showed how to assign the auxiliary equation a priori, i.e. without the development of the scheme of the least residues of any primitive root for the modulus \(p\). Now, without us reminding the attentive readers, it will be easily perceived the closest connection of this theory to the next case, for example, for \(\mu=4\), with the investigations explained here in articles \(15-20\), by the aid of which, it can also be completed without difficulty. But we reserve this treatment for another occasion, so we also preferred to complete the disquisition in a purely arithmetic form in the present essay, without mixing in the theory of the equation \(x^{p}-1=0\). On the other hand, we will add some new purely arithmetic theorems, closely connected to the previously treated argument, in place of the crown still.
%

\subsection*{23.}
 
If the power \((x^{4}+1)^{\frac{1}{2}(p-1)}\) is expanded using the binomial theorem, three terms will appear, in which the exponent of \(x\) is divisible by \(p-1\), namely
\[x^{2(p-1)}, P x^{p-1} \text{ and } 1\]
denoting by \(P\) the middle coefficient
\[\frac{\frac{1}{2}(p-1) \cdot \frac{1}{2}(p-3) \cdot \frac{1}{2}(p-5) \ldots \frac{1}{2}(p+3)}{1 \cdot 2 \cdot 3 \ldots \frac{1}{4}(p-1)}\]
Substituting then for \(x\) in turn the numbers \(1\), \(2\), \(3 \ldots p-1\), we obtain by lemma art. 19
\[\Sigma(x^{4}+1)^{\frac{1}{2}(p-1)} \equiv-2-P\]
But by considering what we exposed in art. 19 and additionally, that the numbers of the complex \(A\), \(B\), \(C\), \(D\), raised to the power of exponent \(\frac{1}{2}(p-1)\) are congruent, modulo \(p\), to the numbers \(+1\), \(-1\), \(+1\), \(-1\) respectively, it is easily understood to occur
\[\Sigma(x^{4}+1)^{\frac{1}{2}(p-1)} \equiv 4(00)-4(01)+4(02)-4(03)\]
and thus by the diagrams given at the end of arts. 18, 20
\[\Sigma(x^{4}+1)^{\frac{1}{2}(p-1)} \equiv-2 a-2\]
The comparison of these two values supplies the most elegant theorem: namely we have
\[P \equiv 2 a\pmod{p}\]
%

Denoting the four products as
\[\begin{aligned}
& 1 \cdot 2 \cdot 3 \ldots \frac{1}{4}(p-1) \\
& \frac{1}{4}(p+3) \cdot \frac{1}{4}(p+7) \cdot \frac{1}{4}(p+11) \ldots  \frac{1}{2}(p-1) \\
& \frac{1}{2}(p+1) \cdot \frac{1}{2}(p+3) \cdot \frac{1}{2}(p+5) \ldots  \frac{3}{4}(p-1) \\
& \frac{1}{4}(3p+1) \cdot \frac{1}{4}(3p+5) \cdot \frac{1}{4}(3p+9) \ldots (p-1)
\end{aligned}\]
respectively by \(q\), \(r\), \(s\), \(t\), the preceding theorem is exhibited as follows:
\[2a \equiv \frac{r}{q}\pmod{p}\]
Since each factor of \(q\) has its complement to \(p\) in \(t\), will be \(q \equiv t\pmod{p}\) whenever the multiplicity of factors is even, i.e. whenever \(p\) is of the form \(8n+1\); on the contrary, \(q \equiv -t\pmod{p}\) whenever the multiplicity of factors is odd, or \(p\) is of the form \(8n+5\). Similarly, in the former case, \(r \equiv s\), in the latter, \(r \equiv -s\). In each case, \(qr \equiv st\), and because it is clear that \(qrst \equiv -1\), it will be \(qqrr \equiv -1\), and consequently \(qr \equiv \pm f\pmod{p}\). Combining this congruence with the theorem just found, we obtain \(rr \equiv \pm 2af\), and therefore, by articles 19, 20
\[2b \equiv \pm rr\pmod{p}\footnote{and \(\{(a \mp b) q\}^{2} \equiv a \equiv(\frac{r-qr r}{2})^{2}\)}\]
It is very notable that the factorization of the number \(p\) into two squares can be found by completely direct operations; namely, the square root of an odd square will be the absolute minimum residue of \(\frac{r}{2q}\), and the square root of an even square will be the absolute minimum residue of \(\frac{1}{2}rr\) modulo \(p\). For the value of \(\frac{r}{2q}\), which becomes \(=1\) for \(p=5\), it can also be exhibited for larger values of \(p\) as:
\[\frac{6\cdot 10\cdot 14\cdot 18 \ldots (p-3)}{2\cdot 3\cdot 4\cdot 5 \ldots \frac{1}{4}(p-1)}\]
But since we furthermore know by which sign this formula for the square root of an odd number is affected, namely, so that it always takes the form \(4m+1\), it is highly noteworthy that a similar general criterion with respect to the sign of the square root of an even number has not yet been found. If anyone finds it and communicates it to us, they will do us a great favor. Meanwhile, it seemed appropriate to include here the values of the numbers \(a\), \(b\), \(f\), which arise for values of \(p\) less than 200 from the minimum residues of the expressions \(\frac{r}{2q}\), \(\frac{1}{2}rr\), \(qr\).
%

\[
\begin{array}{r|r|r|r}
p & a & b & f \\
\hline
5 & +1 & +2 & 2 \\
13 & +3 & -2 & 5 \\
17 & +1& -4 & 13 \\
29 & +5 & +2 & 12 \\
37 & +1 & -6 & 31 \\
41 & +5 & +4 & 9 \\
53 & -7 & -2 & 23 \\
61 & +5 & -6 & 11 \\
73 & -3 & -8 & 27 \\
89 & +5 & -8 & 34 \\
97 & +9 & +4 & 22 \\
101 & +1 & -10 & 91 \\
109 & -3 & +10 & 33 \\
113 & -7 & +8 & 15 \\
137 & -11 & +4 & 37 \\
149 & -7 & -10 & 44 \\
157 & -11 & -6 & 129 \\
173 & +13 & +2 & 80 \\
181 & +9 & +10 & 162 \\
193 & -7 & +12 & 81 \\
197 & +1 & -14 & 183 
\end{array}
\]\\[\baselineskip]
\begin{center} 
\rule{2in}{0.5pt}
\end{center}\clearpage

%

\begin{center}
\;\\[4\baselineskip]
{\large THEORY}\\[\baselineskip]
{\LARGE ON THE RESIDUES OF BICUADRATIC EQUATIONS}\\[3\baselineskip]
{\large SECOND DISSERTATION}\\[2.5\baselineskip]
{\tiny BY}\\[1.5\baselineskip]
CARL FRIEDRICH GAUSS\\[1.5\baselineskip]
{\scriptsize DELIVERED TO THE ROYAL SOCIETY {\tiny \textsc{1831},} APR. {\tiny \textsc{15}.}}\\[4\baselineskip]
\rule{4in}{0.5pt}\\[0.5\baselineskip]
{\small Recent dissertations of the Royal Society of Sciences in Göttingen. Vol. \textsc{vii}.\\
Göttingen \textsc{1832}.}\\
\rule{4in}{0.5pt}
\end{center}
\clearpage\noindent% 94
\;
\clearpage\noindent% 95
\section*{\;\\[3\baselineskip]{\large THEORY OF THE RESIDUES OF BICUADRATIC EQUATIONS.} \\[0.75\baselineskip]
SECOND DISSERTATION \\[0.5\baselineskip]
\rule{0.75in}{0.5pt}\\[0.5\baselineskip]}
%

\subsection*{24.}

In the first commentary, those things which are required for the classification of the biquadratic number \(+2\) are completely determined. That is, while we conceive all numbers distributed across the four complexes \(A\), \(B\), \(C\), \(D\) [such that they are] not divisible by the modulus \(p\) (which is assumed to be a prime number of the form \(4n+1\), according to how each one when raised to the power of the exponent \(\frac{1}{4}(p-1)\) becomes congruent to \(+1\), \(+f\), \(-1\), \(-f\) modulo \(p\), where \(f\) denotes one of the roots of the congruence \(f^2 \equiv -1 \pmod{p}\): we find that the judgment of which complex the number \(+2\) should be assigned to depends on the resolution of the number \(p\) into two squares, namely in such a way that if it is stipulated \(p=a^2+b^2\), with \(a^2\) being an odd square, and \(b^2\) being an even square, then further, \textit{assuming their signs} \(a\), \(b\) are taken in such a way that we have \(a \equiv 1\pmod{4}\), \(b \equiv af\pmod{p}\), the number \(+2\) should belong to the complex \(A\), \(B\), \(C\), \(D\) according to whether \(\frac{1}{2}b\) is of the form \(4n\), \(4n+1\), \(4n+2\), \(4n+3\), respectively.
%

\(\text{Sponte quoque hinc demanat regula classificationi numeri } -2 \text{ inserviens. Scilicet quum } -1 \text{ pertineat ad classem } A \text{ pro valore pari ipsius } \frac{1}{2} b\text{, ad classem } C \text{ vero pro impari: pertinebit, per theorema art. 7, numerus } -2 \text{ ad classem } A\text{, }B\text{, }C\text{, }D\text{, prout } \frac{1}{2} b \text{ est formae } 4n\text{, }4n+3\text{, }4n+2\text{, }4n+1\text{ resp.}\)
%

\begin{center}
These theorems can also be expressed in the following way:
\end{center}
\clearpage
\begin{center}
\begin{tabular}{c|c|c|}
Pertinet & \(+2\) & \(-2\) \\
\hline
ad complexum & \multicolumn{1}{r}{if \(b\) is congruent to} & \multicolumn{1}{l|}{8 with respect to modulus} \\
\hline
\(A\) & \(0\) & 0 \\
\(B\) & \(2 a\) & \(6 a\) \\
\(C\) & \(4 a\) & \(4 a\) \\
\(D\) & \(6 a\) & \(2 a\) \\
\end{tabular}
\end{center}
It is easily understood that the theorems thus stated no longer depend on the condition \(a \equiv 1\pmod{4}\), but still hold if \(a \equiv 3\pmod{4}\), provided that the other condition, \(a f \equiv b\pmod{p}\), is preserved.
%

It can be easily seen that the sum of these theorems can be elegantly condensed into a single formula, namely:
 
\textit{if \(a\) and \(b\) are assumed to be positive, then it always holds that}
\[b^{\frac{1}{2} a b} \equiv a^{\frac{1}{2} a b} 2^{\frac{1}{4}(p-1)}\pmod{p}\]
%

\subsection*{25.}

Let us now see to what extent the induction indicates the classification of the number 3. The table in article 11, further continued (always adopting the minimum primitive root), shows \(+3\) to belong
\[\begin{array}{cccc}
\multicolumn{4}{c}{\text{to the complex}}\\
\begin{array}[t]{r|r|r} 
\multicolumn{3}{c}{ A \text{ for} }\\
p & a & b \\
13 & -3 & +2 \\[-4pt]
109 & -3 & +10 \\[-4pt]
181 & +9 & +10 \\[-4pt]
193 & -7 & -12 \\[-4pt]
229 & -15 & +2 \\[-4pt]
277 & +9 & +14 
\end{array} & 
\begin{array}[t]{||r|r|r||} 
\multicolumn{3}{||c||}{ B \text{ for} }\\
p & a & b \\
17&+1&-4\\[-4pt]
29&+5&+2\\[-4pt]
53&-7&+2\\[-4pt]
89&+5&-8\\[-4pt]
101&+1&+10\\[-4pt]
113&-7&-8\\[-4pt]
137&-11&-4\\[-4pt]
197&+1&-14\\[-4pt]
233&+13&+8\\[-4pt]
257&+1&-16\\[-4pt]
269&+13&+10\\[-4pt]
281&+5&+16\\[-4pt]
293&+17&+2
\end{array}&
\begin{array}[t]{r|r|r||}
\multicolumn{3}{c||}{ C \text{ for} }\\
p & a & b \\
37&+1&-6\\[-4pt]
61&+5&-6\\[-4pt]
73&-3&-8\\[-4pt]
97&+9&+4\\[-4pt]
157&-11&-6\\[-4pt] 
241&-15&-4 
\end{array}
&
\begin{array}[t]{r|r|r} 
\multicolumn{3}{c}{ D \text{ for} }\\
p & a & b \\
5&+1&+2\\[-4pt]
41&+5&-4\\[-4pt]
149&-7&+10\\[-4pt]
173&+13&+2
\end{array}
\end{array}\]
%

\begin{quote}
  At first glance, we do not observe a simple connection between the values of the numbers \(a, b\) that correspond to the same complex.\footnote{The original \LaTeX{} code "resp. idem complexus" is translated as "correspond to the same complex."} However, if we consider a similar judgment in the theory of quadratic residues to be completed by a simpler rule with respect to the number \(-3\), compared to the number \(+3\), there is hope for equally successful outcomes in the theory of biquadratic residues. We find, however, that \(-3\) belongs to the complex
  \[\begin{array}{cccc}
    \begin{array}[t]{r|r|r} 
      \multicolumn{3}{c}{A\text{ for }}\\
      p & a & b \\[-4pt]
      37&+1&-6\\[-4pt]
      61&+5&-6\\[-4pt]
      157&-11&-6\\[-4pt]
      193 & -7 & -12 
    \end{array} & 
    \begin{array}[t]{||r|r|r||} 
      \multicolumn{3}{c}{B\text{ for }}\\
      p & a & b \\
      5&+1&+2\\[-4pt]
      17&+1&-4\\[-4pt]
      89&+5&-8\\[-4pt]
      113&-7&-8\\[-4pt]
      137&-11&-4\\[-4pt]
      149&-7&+10\\[-4pt]
      173&+13&+2\\[-4pt]
      233&+13&+8\\[-4pt]
      257&+1&-16\\[-4pt]
      281&+5&+16
    \end{array}&
    \begin{array}[t]{r|r|r||} 
      \multicolumn{3}{c}{C\text{ for }}\\
      p & a & b \\
      13 & -3 & +2 \\[-4pt]
      73&-3&-8\\[-4pt]
      97&+9&+4\\[-4pt]
      109 & -3 & +10 \\[-4pt]
      181 & +9 & +10 \\[-4pt]
      229 & -15 & +2 \\[-4pt]
      241&-15&-4 \\[-4pt]
      277 & +9 & +14 
    \end{array}
    &
    \begin{array}[t]{r|r|r} 
      \multicolumn{3}{c}{D\text{ for }}\\
      p & a & b \\
      29&+5&+2\\[-4pt]
      41&+5&-4\\[-4pt]
      53&-7&+2\\[-4pt]
      101&+1&+10\\[-4pt]
      197&+1&-14\\[-4pt]
      269&+13&+10\\[-4pt]
      293&+17&+2 
    \end{array}
  \end{array}\]
  where the law of induction presents itself spontaneously. Namely, \(-3\) belongs to the complex
  \begin{quote}
    \(A\), as often as \(b\) is divisible by 3, or \(b \equiv 0\pmod{3}\)\\
    \(B\), as often as \(a+b\) is divisible by 3, or \(b \equiv 2 a\pmod{3}\)\\
    \(C\), as often as \(a\) is divisible by 3, or \(a \equiv 0\pmod{3}\)\\
    \(D\), as often as \(a-b\) is divisible by 3, or \(b \equiv a\pmod{3}\)
  \end{quote}
\end{quote}
%

\subsection*{26.}

We find the number \(+5\) to be assigned to the complex
\begin{quote}
\(A\) for \(p=101\), \(109\), \(149\), \(181\), \(269\) \\
\(B\) for \(p=13\), \(17\), \(73\), \(97\), \(157\), \(193\), \(197\), \(233\), \(277\), \(293\) \\
\(C\) for \(p=29\), \(41\), \(61\), \(89\), \(229\), \(241\), \(281\) \\
\(D\) for \(p=37\), \(53\), \(113\), \(137\), \(173\), \(257\)\end{quote}
After calling the values of the numbers \(a, b\) corresponding to each \(p\) into consideration, the law here is just as easily grasped as it is for the classification of the number \(-3\). That is, we come across the complex
\[\begin{array}{l}
A, \text{ whenever } b \equiv 0\pmod{5} \\
B, \text{ whenever } b \equiv a \\
C, \text{ whenever } a \equiv 0 \\
D, \text{ whenever } b \equiv 4 a
\end{array}\]
It is clear that these rules encompass all cases, since for \(b \equiv 2 a\) or \( b \equiv 3 a\pmod{5}\), it would be \(a a+b b \equiv 0\), Q.E.A., since by hypothesis \(p\) is a prime number different from 5.
%

\subsection*{27.}

Induction applied in the same way to the numbers \(-7\), \(-11\), \(+13\), \(+17\), \(-19\), \(-23\) and sufficiently produced, indicates the following rules:
\begin{center}
\begin{tabular}{l|l}
\multicolumn{2}{c}{For the number \(-7\)}\\
\(A\) & \(a \equiv 0\), or \(b \equiv 0\pmod{7}\) \\
\(B\) & \(b \equiv 4 a\), or \(b \equiv 5 a\) \\
\(C\) & \(b \equiv a\), or \(b \equiv 6 a\) \\
\(D\) & \(b \equiv 2 a\), or \(b \equiv 3 a\) \\
\multicolumn{2}{c}{For the number \(-11\).}\\
\(A\) & \(b \equiv 0\), \(5 a\), or \(6 a\pmod{11}\) \\
\(B\) & \(b \equiv a\), \(3 a\) or \(4 a\) \\
\(C\) & \(a \equiv 0\), or \(b \equiv 2 a\) or \(9 a\) \\
\(D\) & \(b \equiv 7 a\), \(8 a\) or \(10 a\) \\
\multicolumn{2}{c}{For the number \(+13\).}\\
\(A\) & \(b \equiv 0\), \(4a\), \(9 a\pmod{13}\) \\
\(B\) & \(b \equiv 6 a\), \(11 a\), \(12 a\) \\
\(C\) & \(a \equiv 0\); \(b \equiv 3 a\), \(10 a\) \\
\(D\) & \(b \equiv a\), \(2 a\), \(7 a\) \\
\multicolumn{2}{c}{For the number \(+17\).}\\
\(A\) & \(a \equiv 0\); \(b \equiv 0\), \(a\), \(16 a\pmod{17}\)\\
\(B\) & \(b \equiv 2 a\), \(6 a\), \(8 a\), \(14 a\)\\
\(C\) & \(b \equiv 5 a\), \(7 a\), \(10 a\), \(12 a\)\\
\(D\) & \(b \equiv 3 a\), \(9 a\), \(11 a\), \(15 a\)
\end{tabular}
\end{center}\clearpage\noindent% 99
\begin{center}
\begin{tabular}{l|l}
\multicolumn{2}{c}{For the number \(-19\).}\\
\(A\) & \(b \equiv 0\), \(2 a\), \(5 a\), \(14 a\), \(17 a\pmod{19} \)\\
\(B\) & \(b \equiv 3 a\), \(7 a\), \(11 a\), \(13 a\), \(18 a \)\\
\(C\) & \(a \equiv 0\); \(b \equiv 4 a\), \(9 a\), \(10 a\), \(15 a \)\\
\(D\) & \(b \equiv a\), \(6 a\), \(8 a\), \(12 a\), \(16 a\)\\
\multicolumn{2}{c}{For the number \(-23\).}\\
\(A\) & \(a \equiv 0\); \(b \equiv 0\), \(7 a\), \(10 a\), \(13 a\), \(16 a\pmod{23} \)\\
\(B\) & \(b \equiv 2 a\), \(3 a\), \(4 a\), \(11 a\), \(15 a\), \(17 a \)\\
\(C\) & \(b \equiv a\), \(5 a\), \(9 a\), \(14 a\), \(18 a\), \(22 a \)\\
\(D\) & \(b \equiv 6 a\), \(8 a\), \(12 a\), \(19 a\), \(20 a\), \(21 a\)
\end{tabular}
\end{center}
%

\subsection*{28.}

Special theorems are confirmed to have been derived by induction in this way, as long as this continues, and they reveal the most beautiful form of criteria. However, if they are compared with each other, so that general conclusions may be derived from them, the following observations immediately present themselves at first sight.
%

\textbf{Criteria for adjudication}, to which complex number the prime number \(\pm q\) should be referred (by taking the sign above or below, as \(q\) is of the form \(4n+1\) or \(4n+3\)), depend on the forms of the numbers \(a\), \(b\) compared with respect to the modulus \(q\). Namely

I. Whenever \(a \equiv 0\pmod{q}\), \(\pm q\) belongs to a specific complex, which is \(A\) for \(q=7\), \(17\), \(23\), as well as \(C\) for \(q=3\), \(11\), \(13\), \(19\). From this arises the conjecture that the former case generally holds whenever \(q\) is of the form \(8n\pm1\), and the latter holds whenever \(q\) is of the form \(8n\pm3\). However, the complexes \(B\) and \(D\) are already excluded without induction for the value of \(a\) divisible by \(q\), where \(p \equiv b^2\pmod{q}\), that is, where \(p\) is the quadratic residue of \(q\), hence by the fundamental theorem, \(\pm q\) must be a quadratic residue of \(p\).

II. However, if \(a\) is not divisible by \(q\), the criterion depends on the value of the expression \(\frac{b}{a}\pmod{q}\). This expression indeed admits different values of \(q\), such as \(0\), \(1\), \(2\), \(3\ldots q-1\), but whenever \(q\) is of the form \(4n+1\), two values of the expression \(\sqrt{-1}\pmod{q}\) are to be excluded, which obviously cannot be values of the expression \(\frac{b}{a}\pmod{q}\), since \(p=a^2+b^2\) is always assumed to be a prime number different from \(q\). Therefore, the number of admissible values of the expression \(\frac{b}{a}\pmod{q}\) is \(=q-2\) for \(q \equiv 1\pmod{4}\), while it remains \(=q\) for \(q \equiv 3\pmod{4}\).
%

When these values are distributed into four classes, for example, let some people, denoted indefinitely by \(\alpha\), correspond to the complex \(A\); let others, denoted by \(\beta\), correspond to the complex \(B\); let \(\gamma\) correspond to \(C\); finally, let the rest correspond to \(\delta\) to the complex \(D\), so that \(\pm q\) is to be assigned to the complexes \(A\), \(B\), \(C\), \(D\) according as \(b \equiv \alpha a\), \(b \equiv \beta a\), \(b \equiv \gamma a\), \(c \equiv \delta a\pmod{q}\).
%

\[\text{The distribution law seems more abstruse than it actually is, even though some general observations can be made promptly. The same number is found in the three classes, namely, }\frac{1}{4}(q-1) \text{ or } \frac{1}{4}(q+1)\text{, while in a unit (specifically the one corresponding with the criterion }a \equiv 0\text{), the number is one less, so that the number of different criteria relative to each complex becomes the same, namely, }\frac{1}{4}(q-1) \text{ or } \frac{1}{4}(q+1)\text{. Furthermore, we note that 0 is always found in the first class (among } \alpha \text{), as well as the complements of the numbers }\alpha, \beta, \gamma, \delta \text{ to } q, \text{ namely } q-\alpha, q-\beta, q-\gamma, q-\delta \text{ in the first, fourth, third, second class, respectively. Finally, we see that the values of the expressions }\frac{1}{a}, \frac{1}{\beta}, \frac{1}{\gamma}, \frac{1}{\delta} \pmod{q} \text{ belong to the first, fourth, third, second class, whenever the criterion }a \equiv 0 \text{ corresponds to the complex }A\text{; to the third, second, first, fourth class, however, whenever the criterion }a \equiv 0 \text{ is referred to the complex }C\text{. But these are almost all the observations that can be reached by induction, unless we presumptuously dare to anticipate those more boldly, which will be derived below from genuine sources.}\]
%

\subsection*{29}
Before we proceed further, it is worth noting that the criteria for prime numbers (taken positive if they are of the form \(4n+1\), and negative if of the form \(4n+3\)) suffice for the determination of all other numbers, provided that the theorem in article 7 and the criteria for \(-1\) and \(\pm 2\) are called upon to assist. Thus, for example, if criteria for the number \(+3\) are desired, the criteria stated in article 25, which refer to \(-3\), will still apply for \(+3\) whenever \(\frac{1}{2}b\) is an even number; on the other hand, the complex numbers \(A\), \(B\), \(C\), \(D\) should be interchanged with the complex numbers \(C\), \(D\), \(A\), \(B\) whenever \(\frac{1}{2}b\) is odd. From this, the following instructions follow:

\(\begin{array}{c|l}
\multicolumn{1}{c}{}&\multicolumn{1}{l}{+3\text{ pertains}}\\
\text{to complex}&\text{if}\\
A & b \equiv 0\pmod{12}; \text{ or at the same time } a \equiv 0\pmod{3}, b \equiv 2\pmod{4} \\
B & b \equiv 8 a \text{ or } 10 a\pmod{12} \\
C & b \equiv 6 a\pmod{12}; \text{ or at the same time } a \equiv 0\pmod{3}, b \equiv 0\pmod{4} \\
D & b \equiv 2 a \text{ or } 4 a\pmod{12}
\end{array}\)
%

\(\)Similarly, the criteria for \(\pm 6\) are sought from the combination of criteria for \(\mp 2\) and \(-3\); namely\\
\(\begin{array}{c|l}
\multicolumn{1}{c}{}&\multicolumn{1}{l}{+6\text{ applies}}\\
\text{to the complex}&\text{if}\\
A & b \equiv 0, \phantom{00a}2a, 22a \pmod{24}; \text{ or simultaneously } a \equiv 0 \pmod{3}, b \equiv 4a \pmod{8} \\
B & b \equiv 4a, \phantom{00}6a, \phantom{0}8a \pmod{24}; \text{ or simultaneously } a \equiv 0 \pmod{3}, b \equiv 2a \pmod{8} \\
C & b \equiv 10a, 12a, 14a \pmod{24}; \text{ or simultaneously } a \equiv 0 \pmod{3}, b \equiv 0\phantom{a} \pmod{8} \\
D & b \equiv 16a, 18a, 20a \pmod{24}; \text{ or simultaneously } a \equiv 0 \pmod{3}, b \equiv 6a \pmod{8}
\end{array}\)\\
\(\begin{array}{c|l}
\multicolumn{1}{c}{}&\multicolumn{1}{l}{-6\text{ applies}}\\
\text{to the complex}&\text{if}\\
A & b \equiv 0,\phantom{a} 10a, 14a \pmod{24}; \text{ or simultaneously } a \equiv 0 \pmod{3}, b \equiv 4a \pmod{8} \\
B & b \equiv 4a, \phantom{0}8a, 18a \pmod{24}; \text{ or simultaneously } a \equiv 0 \pmod{3}, b \equiv 6a \pmod{8} \\
C & b \equiv 2a, 12a, 22a \pmod{24}; \text{ or simultaneously } a \equiv 0 \pmod{3}, b \equiv 0\phantom{a} \pmod{8} \\
D & b \equiv 6a, 16a, 20a \pmod{24}; \text{ or simultaneously } a \equiv 0 \pmod{3}, b \equiv 2a \pmod{8}
\end{array}\)\\

In a similar way, the criteria for the number \(+21\) will be put together from the criteria for \(-3\) and \(-7\); the criteria for \(-105\) from the criteria for \(-1\), \(-3\), \(+5\), \(-7\), etc.
%

\subsection*{30.}

Induction therefore opens up a very abundant harvest of special theorems, such as the theorem for the number 2 of linear forms: but a common link is desired, rigorous demonstrations are desired, since the method, by which we completed the number 2 in the first commentary, does not allow further application. Indeed, there are various methods by which it would be possible to obtain demonstrations for particular cases, especially those which concern the distribution of quadratic residues among the complex \(A\), \(C\), with which, however, we do not linger; since the theory should encompass *all* general cases. When we started dedicating our thoughts to this matter since 1805, we soon became aware that the genuine source of the general theory in the field of arithmetic was to be sought, as we have already mentioned in art. 1.

As indeed higher arithmetic, in the questions hitherto treated, concerns only real integral numbers, so the theorems about biquadratic residues shine forth in the highest simplicity and genuine beauty only when the field of arithmetic is extended to imaginary quantities, so that, without restriction, the object itself consists of numbers of the form \(a+b i\), denoting \(i\) as the usual imaginary quantity \(\sqrt{-1}\), and \(a\), \(b\) indefinite all real integral numbers between \(-\infty\) and \(+\infty\). We shall call such numbers *complex integral* numbers, so that they are not opposed to real numbers, but are considered to be contained as species under these. The present essay will present both the elementary doctrine of complex numbers and the initial elements of the theory of biquadratic residues, which we will undertake to render perfect in every respect in the subsequent continuation\footnote{By the way, it is appropriate to note at least here, that the field defined in this manner is adapted primarily to the theory of biquadratic residues. The theory of cubic residues is to be built in a similar way for the consideration of numbers of the form \(a+b h\), where \(h\) is the imaginary root of the equation \(h^{3}-1=0\), for instance \(h=-\frac{1}{2}+\sqrt{\frac{3}{4}} \cdot i\); and similarly, the theory of higher powers of residues will require the introduction of other imaginary quantities.}.
%

\subsection*{31.}

First of all we introduce some notations, by the introduction of which brevity and clarity will be served.

The field of complex numbers \(a+bi\) contains\\
I. real numbers, where \(b=0\), and, among these, depending on the nature of \(a\),

1) the digits

2) positive numbers

3) negative numbers\\
II. imaginary numbers, where \(b\) is a non-zero number. Here again we distinguish

1) purely imaginary numbers, i.e. where \(a=0\)

2) imaginary numbers with a real part, where neither \(b\) nor \(a\) equals 0.\\
If you like, the former can be called pure imaginary numbers, the latter can be called mixed imaginary numbers.\clearpage
%

\begin{enumerate}
  \item We use four units in this doctrine: \(+1\), \(-1\), \(+i\), \(-i\), which simply belong to the positive, negative, positive imaginary, and negative imaginary.
  
  \item We will call the products of any complex number by \(-1\), \(+i\), \(-i\) its \textit{companions} or \textit{numbers associated with it}. Except for the number itself (which is associated with itself), the associated numbers are always four unequal numbers.
  
  \item On the other hand, we call a complex number \textit{conjugate} if it arises from a permutation of \(i\) with \(-i\). Therefore, among imaginary numbers, any two unequal numbers are always conjugate, while real numbers are conjugate to themselves, if it is pleasing to extend the denomination to them.
  
  \item We call the product of a complex number by its conjugate the \textit{norm} of it. So, for the norm of a real number, its square is to be taken.
  
  \item Generally, we have eight connected numbers, for example
  \[
  \begin{array}{r|r}
  a+bi & a-bi \\
  -b+ai & -b-ai \\
  -a-bi & -a+bi \\
  b-ai & b+ai
  \end{array}
  \]
  where we see two quaternions of associated numbers, four binions of conjugates, and the common norm of all is \(a^2+b^2\). But eight numbers are reduced to four unequals, whether \(a= \pm b\), or one of the numbers \(a\), \(b=0\).
  
  \item The following immediately follow from the given definitions:
  \begin{itemize}
    \item The product of two complex numbers is conjugate to the product of the numbers associated with them.
    \item The same holds for products with several factors, as well as for quotients.
    \item The norm of the product of two complex numbers is equal to the product of their norms.
    \item This theorem also extends to products with any number of factors and to quotients.
  \end{itemize}
  
  \item The norm of any complex number (except for the number itself, which is usually tacitly understood from now on) is a \textit{positive} number.
  
  \item However, there is nothing preventing our definitions from extending to fractional or even irrational values of \(a\), \(b\); but \(a+bi\) then only constitutes an integer complex number when \textit{both} \(a\), \(b\) are integers, and it is only rational when \textit{both} \(a\), \(b\) are rational.
\end{enumerate}
%

\subsection*{32.}
 
The algorithm of arithmetic operations on complex numbers, commonly known, is: division, through the introduction of the norm, is reduced to multiplication, since we have
\[\frac{a+b i}{c+d i}=(a+b i) \cdot \frac{c-d i}{c c+d d}=\frac{a c+b d}{c c+d d}+\frac{b c-a d}{c c+d d} \cdot i\]
 
The extraction of the square root is accomplished with the help of the formula
\[\surd(a+b i)= \pm(\surd \frac{\surd(a a+b b)+a}{2}+i \surd \frac{\surd(a a+b b)-a}{2})\]
if \(b\) is a positive number, or with this
\[\surd(a+b i)= \pm(\surd \frac{\surd(a a+b b)+a}{2}-i \surd \frac{\surd(a a+b b)-a}{2})\]
if \(b\) is a negative number. The use of the transformation of the complex quantity \(a+b i\) into \(r(\cos \varphi+i \sin \varphi)\) for the purpose of facilitating calculations is not necessary to dwell on here.
%

\subsection*{33.}

We call a complex integer which can be resolved into two factors different from unity, a composite complex number; conversely, a complex number is said to be prime if it does not admit such a resolution into factors. Hence it immediately follows that any real composite number also is a composite complex number. But a prime real number could be a composite complex number, and indeed this holds for the number \(2\) and for all positive real prime numbers of the form \(4n+1\) (except for the number 1), since it is known that they can be decomposed into two positive squares; for example, \(2=(1+i)(1-i)\), \(5=(1+2i)(1-2i)\), \(13=(3+2i)(3-2i)\), \(17=(1+4i)(1-4i)\), etc.

On the other hand, positive real prime numbers of the form \(4n+3\) are always complex prime numbers. For if such a number \(q\) were \(=(a+bi)(\alpha+\beta i)\), it would also be \(q=(a-bi)(\alpha-\beta i)\), and therefore \(qq=(a^2+b^2)(\alpha^2+\beta^2)\). But \(qq\) can only be resolved into positive factors greater than unity in a unique way, that is into \(q\times q\), from which it should be \(q=a^2+b^2=\alpha^2+\beta^2\), Q.E.D.; since the sum of two squares cannot be of the form \(4n+3\).
%

Real negative numbers clearly retain the same designations as positive numbers, and the same holds for pure imaginary numbers.

Thus it remains for us to teach how to distinguish between mixed imaginary numbers, which are composed either of primes or composite numbers, as is done by the following

\textsc{Theorem.} \textit{Any mixed imaginary integer \(a+b i\) is either a complex prime number or a composite number, depending on whether its norm is a prime real number or a composite number.}

\textit{Proof.} I. Since the norm of composite complex numbers is always a composite number, it is clear that a complex number whose norm is a prime real number must necessarily be a complex prime number. Q. E. D.

II. If the norm \(a a+b b\) is a composite number, let \(p\) be a positive real prime number that measures it. There are now two distinct cases to consider.

1) If \(p\) is of the form \(4 n+3\), it is clear that \(a a+b b\) cannot be divisible by \(p\) unless \(p\) also measures \(a\) and \(b\), so \(a+b i\) will be a composite number.

2) If \(p\) is not of the form \(4 n+3\), it can definitely be decomposed into two squares: so let us assume that \(p=\alpha \alpha+\beta \beta\). When we have
\[(a \alpha+b \beta)(a \alpha-b \beta)=a a(\alpha \alpha+\beta \beta)-\beta \beta(a a+b b)\]
and so it is divisible by \(p\), it can certainly measure either of the factors \(a \alpha+b \beta\), \(a \alpha-b \beta\), and in addition, as
\[(a \alpha+b \beta)^{2}+(b \alpha-a \beta)^{2}=(a \alpha-b \beta)^{2}+(b \alpha+a \beta)^{2}=(a a+b b)(\alpha \alpha+\beta \beta)\]
and so is divisible by \(p p\), it is clear that in the first case \(b \alpha-a \beta\) must also be divisible by \(p\), while in the latter case \(b \alpha+a \beta\) must also be divisible by \(p\). Therefore, in the first case
\[\frac{a+b i}{\alpha+\beta i}=\frac{a \alpha+b \beta}{p}+\frac{b \alpha-a \beta}{p} \cdot i\]
will be a complex integer, and in the latter
\[\frac{a+b i}{\alpha-\beta i}=\frac{a \alpha-b \beta}{p}+\frac{b \alpha+a \beta}{p} \cdot i\]
will be an integer. Therefore, since the given number is divisible either by \(\alpha+\beta i\) or by \(\alpha-\beta i\), and since the norm of the quotient \(=\frac{a a+b b}{p}\) is different from 1 by the hypothesis, it is clear that \(a+b i\) is a composite complex number in both cases. Q. E. F.
%

\subsection*{34.}

So the entire set of prime numbers can be exhausted by the following four types:

1) the four units, \(1\), \(+i\), \(-1\), \(-i\), which, however, we will usually understand to be excluded when discussing prime numbers.

2) the number \(1+i\) with its three associates \(-1+i\), \(-1-i\), \(1-i\).

3) positive real prime numbers of the form \(4n+3\) along with their three associates.

4) complex numbers, the norms of which are real prime numbers of the form \(4n+1\) greater by unity, and indeed for any given norm there will always be exactly eight such prime complex numbers, since a norm of this kind can be decomposed into only two squares in a unique way.
%

\subsection*{35.}

Just as the integers are distributed into evens and odds, and the evens are further divided into evenly even and oddly even, so too does an equally essential distinction present itself among complex numbers: namely,

\textit{either} they are not divisible by \(1+i\), for example the numbers \(a+bi\), where one of the numbers \(a\), \(b\) is odd and the other even;

\textit{or} they are not divisible by \(1+i\) but are divisible by 2, whenever both \(a\), \(b\) are odd;

\textit{or} they are divisible by 2, whenever both \(a\), \(b\) are even.

The numbers of the first class can conveniently be called odd complex numbers, those of the second semiodd, and those of the third even.

The product of multiple complex factors will always be odd, provided all factors are odd; semiodd, whenever one factor is semiodd and the rest are odd; and even, whenever among the factors, either two semiodd are found, or at least one even.

The norm of any odd complex number is of the form \(4n+1\); the norm of a semiodd number is of the form \(8n+2\); finally, the norm of an even number is the product of a number of the form \(4n+1\) raised to the power 4 or higher.
%

\subsection*{36.}
 
When the connection between four complex numbers partners is analogous to the connection between two opposite real numbers (i.e. affected equally and in opposite directions), and among them, the positive number is usually considered as the primary one, the question arises whether a similar distinction can be established for four complex numbers partners, and whether it should be considered useful. In order to decide this, we must consider that the principle of distinction should be such that the product of two numbers, which are considered as primary among their partners, always becomes a primary number among their partners. But soon we are assured that such a principle does not exist at all unless the distinction is restricted to integers: so much so that the \textit{useful} distinction should be limited to odd numbers. For these purposes, the proposed goal can be achieved in two ways. Namely,
 
I. The product of two numbers \(a+b i\), \(a^{\prime}+b^{\prime} i\) being compared in such a way that \(a\), \(a^{\prime}\) are of the form \(4 n+1\), and \(b\), \(b^{\prime}\) are even, will enjoy the same property, that the real part becomes \(\equiv 1\pmod{4}\), and the imaginary part is even. And it will be easily seen that among the associated odd quaternions, only one is contained under that form.
 
II. If the number \(a+b i\) is compared in such a way that \(a-1\) and \(b\) are either both even or both odd, then its product with a complex number of the same form will enjoy the same form, and it is easily seen that among the associated odd quaternions, only one is contained under this form.
%

Ex his duobus principiis aeque fere idoneis posterius adoptabimus, scilicet inter quaternos numeros complexos impares associatos eum pro primario habebimus, qui secundum modulum $2+2 i$ unitati positivae fit congruus: hoc pacto plura insignia theoremata maiori concinnitate enunciare licebit. Ita e.g. sunt numeri primi complexi primarii $-1+2 i$, $-1-2 i$, $+3+2 i$, $+3-2 i$, $+1+4 i$, $+1-4 i$ etc., nec non reales $-3$, $-7$, $-11$, $-19$ etc. manifesto semper signo negativo afficiendi. Numero complexo impari primario coniunctus quoque primarius erit.

Pro numeris semiparibus et paribus in genere similis distinctio nimis arbitraria parumque utilis foret. E numeris primis associatis $1+i$, $1-i$, $-1+i$, $-1-i$ unum quidem prae reliquis pro primario eligere possumus, sed ad compositos talem distinctionem non extendemus.
%

\subsection*{37.}

If among the factors of a composite number complex numbers are found, which are themselves composite, and these again are resolved into their factors, clearly we will eventually descend to prime factors, i.e., any composite number is resolvable into prime factors. If any non-prime numbers are found, substitute in place of each of them the product of the associated prime by \(i\), \(-1\) or \(-i\). In this way, it is clear that any composite complex number \(M\) can be reduced to the form
\[M=i^{\mu} A^{\alpha} B^{\beta} C^{\gamma} \ldots\]
such that \(A\), \(B\), \(C\) etc. are distinct prime complex numbers, and \(\mu=0\), \(1\), \(2\) or \(3\). Concerning this resolution, a theorem presents itself, that it can only be done in one way, a theorem that might appear obvious in passing, but certainly requires a demonstration. To which the following sets out to provide a path
%

\textsc{Theorem.} \textit{The product \(M=A^{\alpha} B^{\beta} C^{\gamma} \ldots\), with \(A\), \(B\), \(C\) denoting \(n u\) meros primos complexos primarios diversos, cannot be divisible by any numerus primus complexum primarium, which is not found among \(A\), \(B\), \(C\), etc.}
 
\textit{Proof.} Let \(P\) be a numerus primus complexus primarius not contained among \(A\), \(B\), \(C\), etc., and let \(p\), \(a\), \(b\), \(c\), etc. be the norms of the numbers \(P\), \(A\), \(B\), \(C\), etc. It is easily seen that the norm of the number \(M\) will be \(=a^{\alpha} b^{\beta} c^{\gamma}\) etc., from which it follows that if \(M\) were divisible by \({P}\), it should also be divisible by \(p\). Since the norms are either real prime numbers (from the series \(2\), \(5\), \(13\), \(17\) etc.), or squares of real prime numbers (from the series \(9\), \(49\), \(121\) etc.), it is clear that this cannot occur, unless \(p\) is identical to some norm \(a\), \(b\), \(c\), etc.: we thus suppose \(p=a\). But since \(P\) and \(A\) are assumed to be distinct numeri primi complexi primarii, it is easy to see that these cannot simultaneously hold, unless \(P\) and \(A\) are imaginary complex numbers, and therefore \(p=a\) is an odd real prime number (not the square of a prime number): we assume \(A=k+l i\), \(P=k-l i\). Hence (by extending the concept and sign of congruence to complex integers), we have \(A \equiv 2 k\pmod{P}\), from which it is easily deduced
\[M \equiv 2^{\alpha} k^{\alpha} B^{\beta} C^{\gamma} \ldots\pmod{P}\]
Therefore, while \(M\) is supposed to be divisible by \(P\), it will also be
\[2^{\alpha} k^{\alpha} B^{\beta} C^{\gamma} \ldots\]
divisible by \(P\), and hence the norm of this number, which becomes
\[=2^{2 \alpha} k^{2 \alpha} b^{\beta} c^{\gamma} \ldots\]
is divisible by \(p\). But since 2 and \(k\) are not divisible by \(p\), it follows that \(p\) must be identical to some of the numbers \(b\), \(c\), etc.: let's say \(p=b\). From this, we conclude that either \(B=k+l i\), or \(B=k-l i\), i.e. either \(B=A\), or \(B=P\), both of which contradict the hypothesis.
 
From this theorem, another one is easily derived, namely that the resolution into prime factors can only be accomplished in a single way, and this follows using reasoning entirely analogous to that which we used for real numbers in the \textit{Disquisitiones Arithmeticis} (art. 16); therefore, it would be superfluous to dwell on it here.
%

\subsection*{38.}

We now proceed to the congruence of numbers according to complex modulus. But at the beginning of this discussion, it is convenient to indicate how the representation of complex quantities can be subjected to intuition.

Just as any real quantity can be expressed by taking an arbitrary part of a line extending infinitely in both directions from an arbitrary starting point, and evaluating it according to an arbitrary segment taken as the unit, and can thus be represented by another point, so that points from one side of the starting point represent positive quantities and from the other side represent negative quantities, so any complex quantity can be represented by some point in the infinite plane, in which a line is determined for real quantities, namely the complex quantity \(x+i y\) by the point whose abscissa is \(=x\) and ordinate (taken as positive from one side of the line of abscissas, and negative from the other) is \(=y\). In this way, it can be said that any complex quantity can be measured by the distance between the position of the referred point and the position of the initial point, with a positive unit denoting a determined arbitrary deflection towards a determined arbitrary direction; a negative unit denoting an equally large deflection towards the opposite direction; and finally imaginary units denoting equally large deflections towards two normal lateral directions.
%

\emph{In this way the metaphysics of quantities, which we call imaginary, is remarkably elucidated. If the initial point is denoted by \((0)\), and two complex quantities \(m\), \(m^{\prime}\) are referred to the points \(M\), \(M^{\prime}\), expressing their relative position to \((0)\), the difference \(m-m^{\prime}\) will be nothing but the position of the point \(M\) relative to the point \(M^{\prime}\); likewise, by representing the product \(m m^{\prime}\) as the position of the point \(N\) relative to \((0)\), you will easily see that this position is determined just as much by the position of the point \(M\) to \((0)\), as the position of the point \(M^{\prime}\) is determined by the position of the point corresponding to the positive unit, so that it is not inappropriate to say that the positions of the points corresponding to the complex quantities \(m m^{\prime}\), \(m\), \(m^{\prime}\), \(1\) form a \textit{proportion}. But we reserve a more extensive treatment of this matter for another occasion. The difficulties by which the theory of imaginary quantities is considered involved largely derived from unsuitable denominations (since, indeed, some have used the inappropriate name of impossible quantities). If, starting from the concepts presented by the variations of two dimensions (such as are seen in their greatest purity in the intuitions of space), we had called positive quantities direct, negative quantities inverse, and imaginary quantities lateral, simplicity would have succeeded obscurity.}
%

\subsection*{39.}

The things that were brought forth in the preceding article are referred to continuous complex quantities: in arithmetic, which deals only with integers, the schema of complex numbers will be a system of equidistant points and lines arranged in such a way that the infinite plane is divided into infinitely many equal squares. All numbers divisible by a given complex number \(a+bi=m\) will also form infinitely many squares, whose sides \(=\sqrt{a^2+b^2}\) or areas \(=a^2+b^2\); the later squares will be inclined to the former whenever neither of the numbers \(a\), \(b\) is \(=0\). To every number not divisible by the modulus \(m\), a point will correspond, either situated inside such a square, or in a side adjacent to two squares; however, the latter case cannot occur unless \(a\), \(b\) have a common divisor. Furthermore, it is clear that numbers congruent according to the modulus \(m\) will occupy congruent positions in their squares. Hence, it is easily concluded that, if all numbers situated within a determined square, as well as all those which may lie on two of its non-opposite sides, are collected, and finally to these is ascribed a number divisible by \(m\), we have a complete system of incongruent residues according to the modulus \(m\), i.e. that any integer should be congruent to only one among those. It would not be hard to show that the number of these residues is equal to the modulus norm, for example \(=a^2+b^2\). But it seems advisable to demonstrate this weighty theorem in a purely arithmetic way.

%

\subsection*{40.}
 
\textsc{Theorem.} \textit{According to the given complex modulus \(m=a+b i\), with norm \(a a+b b=p\), and for which \(a\), \(b\) are prime numbers, any integer complex number will be congruent to some residue from the series \(0\), \(1\), \(2\), \(3 \ldots p-1\), and not to more than one.}
 
\textit{Proof.} I. Let \(\alpha\), \(\beta\) be such integers that satisfy \(\alpha a+\beta b=1\), consequently we have
\[i=\alpha b- \beta a+m(\beta+\alpha i)\]
Therefore, for a given integer complex number \(A+B i\), we obtain
\[A+B i=A+(\alpha b-\beta a) B+m(\beta B+\alpha B i)\]
Hence, denoting by \(h\) the smallest positive residue of the number \(A+(\alpha b-\beta a) B\) according to the modulus \(p\), and setting
\[A+(\alpha b-\beta a) B=h+k p=h+m(a k-b k i)\]
we get
\[A+B i=h+m(\beta B+a k+(\alpha B-b k) i)\]
or
\[A+B i \equiv h\pmod{m} . \quad \text{Q. E. D.}\]
 
II. Whenever the same complex number is congruent to two real numbers \(h\), \(h^{\prime}\) according to the modulus \(m\), they will also be congruent to each other. Therefore, let \(h-h^{\prime}=m(c+d i)\), then we have
\[(h-h^{\prime})(a-b i)=p(c+d i)\]
and hence
\[(h-h^{\prime}) a=p c, \quad (h-h^{\prime}) b=-p d\]
and also, due to \(a \alpha+b \beta=1\),
\[h-h^{\prime}=p(c \alpha-d \beta), \text{ i.e. } h \equiv h^{\prime}\pmod{p}\]
 
Therefore, \(h\) and \(h^{\prime}\), since they are unequal, cannot both be included in the complex of numbers \(0\), \(1\), \(2\), \(3 \ldots p-1\). Q. E. S.
%

\subsection*{41.}

\textsc{Theorem.} \textit{According to the complex modulus \(m=a+b i\), whose norm is \(a a+b b=p\), and for which \(a, b\) are not relatively prime, but have the maximum common divisor \(\lambda\) (which we assume to be positive), any complex number is congruent to the residue \(x+y i\) such that \(x\) is any of the numbers \(0,1,2,3 \ldots \frac{p}{\lambda}-1\), and \(y\) is any of the numbers \(0\), \(1\), \(2\), \(3 \ldots \lambda-1\), and indeed only one among all the \(p\) residues which enjoy this form.}

\textit{Proof.} I. By taking integers \(\alpha, \beta\) such that \(\alpha a+\beta b=\lambda\), we have \(\lambda i=\alpha b-\beta a+m(\beta+\alpha i)\). Now let \(A+B i\) be the given complex number, \(y\) the minimum positive residue of \(B\) according to the modulus \(\lambda\), and \(x\) the minimum positive residue of \(A+(\alpha b-\beta a) \cdot \frac{B-y}{\lambda}\) according to the modulus \(\frac{p}{\lambda}\). Let it be established:
\[A+(\alpha b-\beta a) \cdot \frac{B-y}{\lambda}=x+\frac{p}{\lambda} \cdot k\]
Then:
\[\begin{aligned}
A+B i-(x+y i) & =\frac{p}{\lambda} \cdot k+(B-y) i-(\alpha b-\beta a) \frac{B-y}{\lambda} \\
& =\frac{p}{\lambda} \cdot k+\frac{B-y}{\lambda} \cdot m(\beta+\alpha i) \\
& =(\frac{a}{\lambda}-\frac{b}{\lambda} \cdot i) k m+\frac{B-y}{\lambda}(\beta+\alpha i) m
\end{aligned}\]
i.e. divisible by \(m\), or \(A+B i \equiv x+y i\pmod{m}\ \) Q. E. D.

II. Let's suppose that two complex numbers \(x+y i\), \(x^{\prime}+y^{\prime} i\) are congruent to the same complex modulus \(m\), so they will also be congruent to each other according to the modulus \(m\). Therefore, they will also be congruent according to the modulus \(\lambda\), and thus \(y \equiv y^{\prime}\pmod{\lambda}\). Therefore, if both \(y\), \(y^{\prime}\) are assumed to be among the numbers \(0\), \(1\), \(2\), \(3 \ldots \lambda-1\), then it must necessarily be \(y=y^{\prime}\). In this way, it will also be \(x \equiv x^{\prime}\pmod{m}\), i.e. \(x-x^{\prime}\) is divisible by \(m\), and therefore \(\frac{x-x^{\prime}}{\lambda}\) is an integer divisible by \(\frac{a}{\lambda}+\frac{b}{\lambda} \cdot i\), i.e.
\[\frac{x-x^{\prime}}{\lambda} \equiv 0\pmod{\frac{a}{\lambda}+\frac{b}{\lambda} \cdot i}\]
From this, since \(\frac{a}{\lambda}\), \(\frac{b}{\lambda}\) are relatively prime, it is concluded by the second part of the theorem of the previous article that \(\frac{x-x^{\prime}}{\lambda}\) is also divisible by the norm of the number \(\frac{a}{\lambda}+\frac{b}{\lambda} \cdot i\), i.e. by the number \(\frac{p}{\lambda \lambda}\), and therefore \(x-x^{\prime}\) is divisible by \(\frac{p}{\lambda}\). Therefore, if both \(x\), \(x^{\prime}\) are assumed to be among the complex numbers \(0\), \(1\), \(2\), \(3 \ldots \frac{p}{\lambda}-1\), then it must necessarily be \(x=x^{\prime}\), i.e. the residues \(x+y i\), \(x^{\prime}+y^{\prime} i\) are identical. Q.E.D.
%

\(\) But it is clear that it is necessary to refer to this case as well, where the modulus is a real number, for example \(b=0\), and then for \(\lambda= \pm a\), and also where it is a pure imaginary number, for example \(a=0\), and then for \(\lambda= \pm b\). In each case, we have \(\frac{p}{\lambda}=\lambda\).
%

\subsection*{42.}

Therefore, by arranging all complex numbers according to a given modulus into congruence classes with each other, and incongruent to different classes, there will be exactly \(p\) classes exhaustively covering the entire domain of integral numbers, denoting \(p\) as the norm of the modulus. Each class consisting of as many numbers as taken from individual classes will present a complete system of incongruent residues, as we have described in articles 40, 41. And in this system, the choice of residue classes, as if representing their own, was based on the principle, that for any class, a residue \(x+yi\) should be adopted, such that \(y\) has the minimum value, and among all those for which the same minimum value of \(y\) exists, that for which the value of \(x\) is the minimum, excluding negative values for both \(x\) and \(y\). But for other purposes, it will be suitable to use different principles, and in particular, the method where residues are adopted which, when divided by the modulus, yield simple quotients. Clearly, if \(\alpha+\beta i\), \(\alpha^{\prime}+\beta^{\prime} i\), \(\alpha^{\prime \prime}+\beta^{\prime \prime} i\) etc. are quotients resulting from the division of congruent numbers by the modulus, then the differences of both quantities \(\alpha\), \(\alpha^{\prime}\), \(\alpha^{\prime \prime}\) etc. between themselves will be whole numbers, and the differences between the quantities \(\beta\), \(\beta^{\prime}\), \(\beta^{\prime \prime}\) etc., and it is clear that there is always one residue for which \(\alpha\) and \(\beta\) lie between the limits 0 and 1, the former including and the latter excluding the limit: we simply call such a residue the minimum residue. If preferable, instead of those limits, \( -\frac{1}{2} \) and \( +\frac{1}{2} \) (with one included and the other excluded) can also be adopted: and we will call a residue corresponding to such a limitation the \textit{absolute minimum}.

Around these minimum residues, the following problems present themselves.
%

\subsection*{43.}

The minimum residue of the given complex number \(A+Bi\) with respect to the modulus \(a+bi\), whose norm \(=p\), is found in the following way. If \(x+yi\) is the sought minimum residue, then \((x+yi)(a-bi)\) will be the minimum residue of the product \((A+Bi)(a-bi)\) with respect to the modulus \((a+bi)(a-bi)\), i.e. with respect to the modulus \(p\). Therefore, assuming
\[a A+ b B = F p + f, \quad a B - b A = G p + g,\]
so that \(f\), \(g\) are the minimum residues of the numbers \(aA + bB\), \(aB - bA\) with respect to the modulus \(p\), then
\[x+yi=\frac{f+gi}{a-bi}\]
or
\[\begin{aligned}
& x=\frac{af-bg}{p}=A-aF+bG \\
& y=\frac{ag+bf}{p}=B-aG-bF
\end{aligned}\]
Clearly, the minimum residues \(f\), \(g\) should be taken either within the limits 0 and \(p-1\), or within \(-\frac{1}{2}p\) and \(\frac{1}{2}p\), depending on whether the complex numbers or the minimum residue itself is desired to be minimum or absolute minimum.
%

\subsection*{44.}

The construction of the complete system of least residues for a given modulus can be accomplished in several ways. The first method proceeds by first determining the limits within which the real terms must lie, and then assigning limits for the imaginary parts for each value within these limits. The general criterion for the least residue \(x+yi\) for modulus \(a+bi\) consists in the conditions that both \(ax+by=\xi\) and \(ay-bx=\eta\) lie within the limits \(0\) and \(aa+bb\), whenever we deal with simply least residues, or lie within the limits \(-\frac{1}{2}(aa+bb)\) and \(\frac{1}{2}(aa+bb)\) whenever absolute least residues are desired, with one of the limits excluded. Specific rules for distinguishing cases that the variety of signs of the numbers \(a\), \(b\) brings, are required, however, we shall refrain from lingering here, as the solution of this, since it presents no difficulty, has been deferred: it suffices to have set forth the nature of the method through a single example.
%

For the modulus \(5+2i\), the residues \(x+yi\) must be compared in such a way that both \(5x+2y=\xi\) and \(5y-2x=\eta\) are equal to some of the numbers \(0, 1, 2, 3, \ldots, 28\). The equation \(29x=5\xi-2\eta\) shows that the positive values of \(x\) cannot exceed \(\frac{5 \cdot 28}{29}\), and by considering the sign, the negative values cannot exceed \(\frac{2 \cdot 28}{29}\). Therefore, all admissible values of \(x\) are \(-1\), \( 0\), \(1\), \(2\), \(3\), \(4\). For \(x=-1\), \(2y\) must be equal to some of the numbers \(5\), \(6\), \(7 \ldots 33\), and \(5y\) to some of these \(-2\), \(-1\), \(0\), \(1 \ldots 26\). Hence, the minimum value of \(y\) is \(+3\), and the maximum is \(+5\). Treating the remaining values of \(x\) similarly, the following scheme of all minimum residues arises:

\begin{center}
\begin{tabular}{r|l}
\multicolumn{1}{c|}{\(x\)} & \multicolumn{1}{c}{\(y\)} \\
\hline
\(-1\) & \(3\), \(4\), \(5\) \\
\(0\) & \(0\), \(1\), \(2\), \(3\), \(4\), \(5\) \\
\(1\) & \(1\), \(2\), \(3\), \(4\), \(5\), \(6\) \\
\(2\) & \(1\), \(2\), \(3\), \(4\), \(5\), \(6\) \\
\(3\) & \(2\), \(3\), \(4\), \(5\), \(6\) \\
\(4\) & \(2\), \(3\), \(4\) \\
\end{tabular}
\end{center}
%

In a similar manner, for the absolute minimum residues, \(\xi\) and \(\eta\) must be equal to some of the numbers \(-14\), \(-13\), \(-12 \ldots+14\); hence \(29 x\) cannot be outside the limits \(-7.14\) and \(+7.14\), and therefore \(x\) must be equal to some of the numbers \(-3\), \(-2\), \(-1\), \(0\), \(1\), \(2\), \(3\). For \(x=-3\) will have \(2 y=\xi-5 x=\xi+15\) equal to some of the numbers \(1\), \(2\), \(3 \ldots 29\), \(5 y=\eta+2 x=\eta-6\) however will be equal to some of these \(-20\), \(-19\), \(-18 \ldots+8\): hence it follows for \(y\) a unique value \(+1\). Proceeding in the same way for the other values of \(x\), we have the scheme of all absolute minimum residues:
\begin{center}
\begin{tabular}{r|l}
\multicolumn{1}{c|}{\(x\)} & \multicolumn{1}{c}{\(y\)} \\
\hline
\(-3\) & \(+1\) \\
\(-2\) & \(-2\), \(-1\), \(0\), \(+1\), \(+2\) \\
\(-1\) & \(-3\), \(-2\), \(-1\), \(0\), \(+1\), \(+2\) \\
\(0\) & \(-2\), \(-1\), \(0\), \(+1\), \(+2\) \\
\(+1\) & \(-2\), \(-1\), \(0\), \(+1\), \(+2\), \(+3\) \\
\(+2\) & \(-2\), \(-1\), \(0\), \(+1\), \(+2\) \\
\(+3\) & \(-1\) \\
\end{tabular}
\end{center}
%

\subsection*{45.}
In the application of the second method, it will be convenient to distinguish two cases.

In the first case, where \(a\) and \(b\) do not have a common divisor, let \(\alpha a+\beta b=1\), and let \(k\) be the minimum positive residue of \(\beta a-\alpha b\) modulo \(p\). Hence, the identical equations
\[a(\beta a-\alpha b)=\beta p-b(\alpha a+\beta b), \quad b(\beta a-\alpha b)=-\alpha p+a(\alpha a+\beta b)\]
show that \(a k \equiv-b\), \(b k \equiv a \pmod{p}\). Therefore, by assuming \(a x+b y=\xi\) as above, \(a y-b x=\eta\), we have \(\eta \equiv k \xi\), \(\xi \equiv -k \eta \pmod{p}\). So all numbers \(\xi+\eta i\), for which the residues simply correspond to the minimum \(x+y i\), are obtained, while values \(0\), \(1\), \(2\), \(3 \ldots p-1\) are successively taken for \(\xi\), and for \(\eta\) the minimum positive residues of the products \(k \xi\) modulo \(p\), or in a different order for \(\eta\) those values and for \(\xi\) the minimum residues of the products \(-k \eta\). From each \(\xi+\eta i\) then corresponding \(x+y i\) are found by the formula
\[x+y i=\frac{\xi+\eta i}{a-b i}=\frac{a \xi-b \eta}{p}+\frac{a \eta+b \xi}{p} \cdot i\]
However, it is obvious that as \(\eta\) increases by unity, \(\xi\) will undergo an increase of \(k\) or a decrease of \(p-k\), and thus \(x+y i\)
\[ \text{will undergo a change to} \frac{a-k b}{p}+\frac{a k+b}{p} \cdot i \text{or to this} \frac{a-k b}{p}+b+(\frac{a k+b}{p}-a) \cdot i \]
This observation serves to facilitate the construction.
%

Finally, it is clear that if the absolute minimum residues \(x+yi\) are desired, these instructions are only to be changed in such a way that the values of \(\xi\) are subsequently assigned to be between the limits \(-\frac{1}{2} p\) and \(+\frac{1}{2} p\), while for \(\eta\) one should obtain the absolute minimum residues of the products \(k\xi\). Here is a view of the absolute minimum residues for the modulus \(5+2i\) arranged in this way:
\begin{center}
Simply minimum residues.\\
\begin{tabular}{r|r||r|r||r|r}
\(\xi+\eta i\) & \(x+yi\) & \(\xi+\eta i\) & \(x+yi\) & \(\xi+\eta i\) & \(x+yi\) \\
\hline
0 & 0 & \(10+25i\) & \(+5i\) & \(20+21i\) & \(+2+5i\) \\
\(1+17i\) & \(-1+3i\) & \(11+13i\) & \(+1+3i\) & \(21+9i\) & \(+3+3i\) \\
\(2+5i\) & \(+i\) & \(12+i\) & \(+2+i\) & \(22+26i\) & \(+2+6i\) \\
\(3+22i\) & \(+1+4i\) & \(13+18i\) & \(+1+4i\) & \(23+14i\) & \(+3+4i\) \\
\(4+10i\) & \(+2i\) & \(14+6i\) & \(+2+2i\) & \(24+2i\) & \(+4+2i\) \\
\(5+27i\) & \(-1+5i\) & \(15+23i\) & \(+1+5i\) & \(25+19i\) & \(+3+5i\) \\
\(6+15i\) & \(+3i\) & \(16+11i\) & \(+2+3i\) & \(26+7i\) & \(+4+3i\) \\
\(7+3i\) & \(+i\) & \(17+28i\) & \(+1+6i\) & \(27+24i\) & \(+3+6i\) \\
\(8+20i\) & \(+4i\) & \(18+16i\) & \(+2+4i\) & \(28+12i\) & \(+4+4i\) \\
\(9+8i\) & \(+1+2i\) & \(19+4i\) & \(+3+2i\) &\multicolumn{2}{c}{} \\
\end{tabular}
\end{center}\clearpage\noindent% 117
\begin{center}
Absolute minimum residues.\\
\begin{tabular}{r|r||r|r||r|r}
\(\xi+\eta i\) & \(x+yi\) & \(\xi+\eta i\) & \(x+yi\) & \(\xi+\eta i\) & \(x+yi\) \\
\hline
\(-14-6i\) & \(-2-2i\) & \(-4-10i\) & \(-2i\) & \(+5-2i\) & \(+1\) \\
\(-13+11i\) & \(-3+i\) & \(-3+7i\) & \(-1+i\) & \(+6-14i\) & \(+2-2i\) \\
\(-12-i\) & \(-2-i\) & \(-2-5\) & \(-i\) & \(+7+3i\) & \(+1+i\) \\
\(-11-13i\) & \(-1-3i\) & \(-1+12i\) & \(-1+2i\) & \(+8-9i\) & \(+2-i\) \\
\(-10+4i\) & \(-2\) & 0 & 0 & \(+9+8i\) & \(+1+2i\) \\
\(-9-8i\) & \(-1-2i\) & \(+1-12i\) & \(+1-2i\) & \(+10-4i\) & \(+2\) \\
\(-8+9i\) & \(-2+i\) & \(+2+5i\) & \(+i\) & \(+11+13i\) & \(+1+3i\) \\
\(-7-3i\) & \(-1-i\) & \(+3-7i\) & \(+1-i\) & \(+12+i\) & \(+2+i\) \\
\(-6+14i\) & \(-2+2i\) & \(+4+10i\) & \(+2i\) & \(+13-11i\) & \(+3-i\) \\
\(-5+2i\) & \(-1\) &\multicolumn{2}{c||}{} & \(+14+6i\) & \(+2+2i\) \\
\end{tabular}
\end{center}
%

In the second case, where \(a\), \(b\) are not coprime, it is easy to reduce to the previous case. Let \(\lambda\) be the greatest common divisor of the numbers \(a\), \(b\), and \(a=\lambda a^{\prime}\), \(b=\lambda b^{\prime}\). Let \(F\) denote the indefinite minimum residue for the modulus \(\lambda\), insofar as it is considered as a complex number, i.e., it represents an indefinite number \(x+yi\) such that \(x, y\) are either between 0 and \(\lambda\) or between \(-\frac{1}{2}\lambda\) and \(\frac{1}{2}\lambda\) (depending on whether it is about residues simply or absolutely minimal). Let \(F^{\prime}\) denote the indefinite minimum residue for the modulus \(a^{\prime}+b^{\prime}i\). Then \((a^{\prime}+b^{\prime}i)F+F^{\prime}\) will be the indefinite minimum residue for the modulus \(a+bi\), and the complete system of these residues will emerge as all \(F\) are combined with all \(F^{\prime}\).
%

\subsection*{46.}

Two complex numbers are said to be prime to each other if, besides units, they do not admit any other common divisors. But whenever such common divisors are present, they are called the greatest common divisors, whose norm is maximum.

If the resolution of two proposed numbers into prime factors is given, the determination of the greatest common divisor is carried out entirely in the same way as for real numbers (\textit{Disquiss. Ar.} art. 18). At the same time it becomes clear from this that all the common divisors of the two given numbers must be measured by this greatest common divisor found in this way. Therefore, since it is already obvious, it is always the case that three numbers are also common divisors of these associates, and no more. Hence, the common divisors of the four numbers, and no more, will be called the greatest, and their norm will be a multiple of the norm of any other common divisor.
%

\(\text{If the resolution of two given numbers into prime factors is not present, the greatest common divisor is extracted with the help of a similar algorithm, as for real numbers. Let }m, m'\text{ be the two given numbers, and let the series }m'', m''', \text{ etc., be formed by repeated division so that }m''\text{ is the absolute minimum remainder of }m\text{ with respect to }m', \text{ then }m'''\text{ is the absolute minimum remainder of }m'\text{ with respect to }m''\text{, and so on. Denoting the norms of the numbers }m, m', m'', m''', \text{ etc., by }p, p', p'', p''', \text{ etc., we have }\frac{p''}{p'}\text{ as the norm of the quotient }\frac{m''}{m'}\text{, and thus, by the definition of the absolute minimum remainder, it is certainly not greater than }\frac{1}{2}; \text{ the same holds for }\frac{p'''}{p''}\text{, etc. Therefore, positive real integers }p', p'', p''', \text{ etc., will form a continuously decreasing series, and hence must necessarily reach the limit }0\text{, or equivalently, in the series }m, m', m'', m''', \text{ etc., we will eventually reach a limit that measures the preceding one without remainder. Let this be }m^{(n+1)}, \text{ and let us assume}\)
\[
\begin{aligned}
& m=k m'+m'' \\
& m'=k' m''+m''' \\
& m''=k'' m'''+m'''' \\
\end{aligned}
\]
\text{etc., until}
\[m^{(n)}=k^{(n)} m^{(n+1)}\]
\[\text{By going through these equations in reverse order, it becomes clear that }m^{(n+1)}\text{ measures each preceding term }m^{(n)} \ldots m'', m', m; \text{ on the other hand, going through the same equations in direct order, it is evident that any common divisor of the numbers }m, m', m''\text{ also measures each subsequent one. The former conclusion indicates that }m^{(n+1)}\text{ is a common divisor of the numbers }m, m'; \text{ the latter, that this divisor is the greatest.}\]
 
\(\text{However, whenever the ultimate remainder }m^{(n+1)}\text{ happens to be equal to one of the four units }1, -1, i, -i\text{, this will be an indication that }m\text{ and }m'\text{ are prime to each other.}\)

%

\subsection*{47.}

If the equations of the foregoing article, omitting the last one, are combined in such a way that \(m^{\prime \prime}, m^{\prime \prime \prime}\), \(m^{\prime \prime \prime \prime} \ldots m^{(n)}\) are eliminated, there arises an equation of the form
\[m^{(n+1)}=h m+h^{\prime} m^{\prime}\]

where \(h\), \(h^{\prime}\) will be integers, and indeed, if we use the designation introduced in \textit{Disquiss. Ar.} art. 27
\[\begin{aligned}
& h= \pm\left[k^{\prime}, k^{\prime \prime}, k^{\prime \prime \prime} \ldots k^{(n-1)}\right]= \pm\left[k^{(n-1)}, k^{(n-2)} \ldots k^{\prime \prime}, k^{\prime}\right] \\
& h^{\prime}=\mp\left[k, k^{\prime}, k^{\prime \prime}, k^{\prime \prime \prime} \ldots k^{(n-1)}\right]=\mp\left[k^{(n-1)}, k^{n-2)} \ldots k^{\prime \prime}, k^{\prime}, k\right]
\end{aligned}\]
holding the upper or lower signs, as \(n\) is even or odd. We state this theorem as follows:

\textit{The greatest common divisor of two complex numbers \(m\), \(m^{\prime}\) can be reduced to the form \(h m+h^{\prime} m^{\prime}\), in such a way that \(h, h^{\prime}\) are integers.}
%

\(\text{For this manifesto holds not only for the greatest common divisor to which the previous algorithm led, but also for three associated with it, for which in place of the coefficients }h\text{, }h^{\prime}\text{ we ought to take either }h i\text{, }h^{\prime} i\text{ or }-h\text{, }-h^{\prime}\text{ or }-h i\text{, }-h^{\prime} i\text{.}\)

\(\text{Therefore, whenever the numbers }m\text{, }m^{\prime}\text{ are relatively prime, the equation}\)
\[1=h m+h^{\prime} m^{\prime}\]
\(\text{can be satisfied.}\)
%

\[\text{Let the numbers be e.g. } 31+6i=m, 11-20i=m', \text{ Here we find } \]
\[
\begin{array}{clccl}
k&=\phantom{+1-\;} i,&\quad& m''&=+11-5i \\
k'&=+1-i, &\quad& m'''&=+\phantom{0}5-4i \\
k''&=+2, &\quad& m''''&=+\phantom{0}1+3i \\
k'''&=-1-2i, &\quad& m'''''&=\phantom{+00}+i \\
k'''''&=+3-i &\quad& \\
\end{array}
\]
\[\text{and thus } 
\begin{aligned}
& {\left[k', k'', k'''\right]=-6-5i} \\
& {\left[k, k', k'', k'''\right]=+4-10i}
\end{aligned}
\]
\[\text{and therefore} \quad m''''''=i=(6+5i)m+(4-10i)m', \]
\[\text{as well as} \quad 1=(5-6i)m+(-10-4i)m', \text{ which is verified by calculation.}\]
%

\subsection*{48.}

By all the foregoing, everything required for the theory of congruences of the first degree in the arithmetic of complex numbers has been prepared; but since it does not essentially differ from that which holds for the arithmetic of real numbers, and which is copiously set out in the \textit{Disquisitiones Arithmetic}, it will suffice to have set down the principal points here.

I. The congruence \(m t \equiv 1\pmod{m^{\prime}}\) is equivalent to the indeterminate equation \(m t+m^{\prime} u=1\), and if this is satisfied by the values \(t=h, u=h^{\prime}\), its general solution is exhibited by \(t \equiv h\pmod{m^{\prime}}\); the condition for solvability is that the modulus \(m^{\prime}\) with coefficient \(m\) does not have a common divisor.

II. The solution of the congruence \(a x+b \equiv c\pmod{M}\) in the case where \(a\), \(M\) are relatively prime, depends on the solution of this
\[a t \equiv 1\pmod{M}\]
which, if satisfied by \(t=h\), its general solution is contained in the formula
\[x \equiv(c-b) h\pmod{M}\]

III. The congruence \(a x+b \equiv c\pmod{M}\) in the case where \(a\), \(M\) have a common divisor \(\lambda\), is equivalent to this
\[\frac{a}{\lambda} \cdot x \equiv \frac{c-b}{\lambda}\pmod{\frac{M}{\lambda}}\]
Therefore, when the largest common divisor of the numbers \(a . M\) is adopted for \(\lambda\), the solution of the proposed congruence is reduced to the preceding case, and it is clear that for solubility to be required and sufficient, \(\lambda\) also measures the difference \(c-b\).
%

\subsection*{49.}

So far we have only touched on elementary matters, yet it was not permissible to omit the cause of the connections. In more advanced investigations, the arithmetic of complex numbers is similar to the arithmetic of real numbers, in that more elegant and simpler theorems emerge, while we only consider such moduli which are prime numbers: in fact, their extension to composite moduli is usually more lengthy than difficult, and involves more labor than skill. Therefore, in the following, we will primarily deal with prime moduli.
%

\subsection*{50.}

Denoting by \(X\) an undetermined function of \(x\) of the form
\[A x^{n}+B x^{n-1}+C x^{n-2}+\text{etc.}+M x+N\]
where \(n\) is a positive real integer, \(A\), \(B\), \(C\), etc. are real or imaginary integers, and \(m\) is a complex integer, we will also call here a \textit{root} of the congruence \(X \equiv 0\pmod{m}\) any integer that, when substituted for \(x\), yields a value for \(X\) divisible by \(m\). Solutions by roots congruent to the modulus will not be considered as different ones.

Whenever the modulus is a prime number, such a congruence of order \(n\) cannot admit more than \(n\) distinct solutions. Denoting \(\alpha\) as any given integer (complex), \(X\) can be indefinitely divided by \(x-\alpha\) to the form \(X=(x-\alpha) X^{\prime}+h\), so that \(h\) becomes a given integer and \(X^{\prime}\) becomes a function of order \(n-1\) with integer coefficients. Now, whenever \(\alpha\) is a root of the congruence \(X \equiv 0\pmod{m}\), evidently \(h\) will be divisible by \(m\), therefore we obtain indefinitely \(X \equiv(x-\alpha) X^{\prime}\pmod{m}\).
%

If \(\beta\) is a given integer, and \(X^{\prime}\) is reduced to the form \((x-\beta) X^{\prime \prime}+h^{\prime}\), then \(X^{\prime \prime}\) will be a function of order \(n-2\) with integral coefficients. However, if \(\beta\) is assumed to be a root of congruence \({X} \equiv 0\), it must also satisfy \((\beta-\alpha) X^{\prime} \equiv 0\) and \(X^{\prime} \equiv 0\), since the roots \(\alpha\), \(\beta\) are incongruent. Hence, it follows that \(h^{\prime}\) must be divisible by \(m\), or indefinitely, \(X \equiv(x-\alpha)(x-\beta) X^{\prime \prime}\pmod{m}\).

Similarly, with the introduction of a third root \(\gamma\) incongruent to the previous ones, we will have indefinitely \(X \equiv(x-\alpha)(x-\beta)(x-\gamma) X^{\prime \prime \prime}\), such that \(X^{\prime \prime \prime}\) is a function of order \(n-3\) with integral coefficients. This process can be further extended, and it is evident that the coefficient of the highest term in each function is \(=A\), which is assumed to be indivisible by \(m\). Otherwise, the congruence \(X \equiv 0\) would essentially be referred to a lower order. Therefore, whenever there are \(n\) incongruent roots, say \(\alpha, \beta, \gamma \ldots \nu\), we will have indefinitely
\[X \equiv A(x-\alpha)(x-\beta)(x-\gamma) \ldots(x-\nu)\pmod{m}\]
Hence, the substitution of new values for each \(\alpha, \beta, \gamma \ldots \nu\) incongruent will reconcile \(X\) to a value not divisible by \(m\), and the truth of the theorem follows naturally.

Moreover, this demonstration essentially agrees with that which we presented in \textit{Disq. Ar.} art. 43, and each moment of it is equally valid for complex numbers as for real numbers.
%

\subsection*{51.}

The results presented in the third section of the \textit{Disquisitiones Arithmeticae} concerning residues of powers, for the most part, hold true with slight modifications, even in the arithmetic of complex numbers. Indeed, the proofs of the theorems can often be retained. Nevertheless, in order to provide a complete account, we will present the main theorems established with concise proofs, where it should always be understood that the modulus is a prime number.
%

\textsc{Theorem.} \textit{Let \(k\) denote an integer by modulus \(m\), whose norm \(=p\), not divisible, then \(k^{p-1} \equiv 1 \pmod{m}\).}
 
\textit{Proof.} Let \(a\), \(b\), \(c\), etc. form a complete system of incongruous residues for modulus \(m\), such that the residue divisible by \(m\) is omitted, and thus the number of those numbers, whose set we denote by \(C\), is \(=p-1\). Let \(C^{\prime}\) be the set of products \(k a\), \(k b\), \(k c\), etc. Since none of these products will be divisible by \(m\) under the hypothesis, each will have congruous residues in the set \(C\), so it can happen that \(a k \equiv a^{\prime}\), \(b k \equiv b^{\prime}\), \(c k \equiv c^{\prime}\), etc. \(\pmod{m}\), so that the numbers \(a^{\prime}\), \(b^{\prime}\), \(c^{\prime}\), etc. are found in the set \(C\): let us denote the set of numbers \(a^{\prime}\), \(b^{\prime}\), \(c^{\prime}\), etc. by \(C^{\prime \prime}\). Let \(P\), \(P^{\prime}\), \(P^{\prime \prime}\) be the products of individual numbers of the sets \(C\), \(C^{\prime}\), \(C^{\prime \prime}\), respectively, that is,
\[\begin{aligned}
& P=a b c \ldots \\
& P^{\prime}=k^{p-1} a b c \ldots=k^{p-1} P \\
& P^{\prime \prime}=a^{\prime} b^{\prime} c^{\prime} \ldots
\end{aligned}\]
Since the numbers of the set \(C^{\prime \prime}\) are congruous to the numbers of the set \(C^{\prime}\), \(P^{\prime \prime} \equiv P^{\prime}\) or \(P^{\prime \prime} \equiv k^{p-1} P\). But since it is easy to see that any two numbers of the set \(C^{\prime \prime}\) are incongruous with each other, and thus all of them are different from each other, necessarily the set of numbers \(C^{\prime \prime}\) completely agrees with the set of numbers \(C\), with only the order changed, whence \(P^{\prime \prime}=P\). Thus, \((k^{p-1}-1) P\) will be divisible by \(m\), so, since \(m\) is a prime number that does not measure any of its factors, \(k^{p-1}-1\) will necessarily have to be divisible by \(m\). Q. E. D.
%

\subsection*{52.}

\textsc{Theorem.} \textit{Denoting \(k\), as in the preceding article, an integer not divisible by the modulus \(m\), and \(t\) the smallest exponent (other than 0) for which \(k^{t} \equiv 1\pmod{m}\), then \(t\) will be a divisor of any other exponent \(u\) for which \(k^{u} \equiv 1\pmod{m}\).}

\textit{Proof.} If \(t\) were not a divisor of \(u\), let \(g t\) be the multiple of \(u\) just greater than \(u\), so \(g t-u\) is a positive integer less than \(t\). From \(k^{t} \equiv 1\), \(k^{u} \equiv 1\), it follows that \(0 \equiv k^{g t}-k^{u} \equiv k^{u}(k^{g t-u}-1)\), so \(k^{g t-u} \equiv 1\), that is, the power of \(k\) with an exponent less than \(t\) equivalent to 1, contrary to the assumption.

As a corollary, it follows that \(t\) certainly measures the number \(p-1\).

We will call such numbers \(k\), for which \(t=p-1\), also here \textit{primitive roots} for the modulus \(m\): we will actually show that they exist.
%

\subsection*{53.}

Let the number \(p-1\) be resolved into its prime factors, so that we have
\[p-1=a^{\alpha} b^{\beta} c^{\gamma} \ldots\]
where \(a\), \(b\), \(c\), etc. denote real positive prime numbers that are not equal. Let \(A\), \(B\), \(C\), etc. be integers (complex) not divisible by \(m\), and let them satisfy the congruences
\[x^{\frac{p-1}{a}} \equiv 1, x^{\frac{p-1}{b}} \equiv 1, x^{\frac{p-1}{c}} \equiv 1 \text{ etc.}\]
modulo \(m\), such as those given in theorem 50. Finally, let \(h\) be congruent modulo \(m\) to the product
\[A^{\frac{p-1}{a^{\alpha}}} B^{\frac{p-1}{b^{\beta}}} C^{\frac{p-1}{c^{\gamma}}} \ldots\]
Then I say, \(h\) will be a primitive root.

\textit{Proof.} Denoting by \(t\) the exponent of the lowest power congruent to \(h^{t}\), if \(h\) were not a primitive root, \(t\) would be a proper divisor of \(p-1\), that is \(\frac{p-1}{t}\) would be an integer greater than unity. Evidently, this integer will have its real prime factors among \(a\), \(b\), \(c\), etc.: let us suppose, therefore (which is allowed), that \(\frac{p-1}{t}\) is divisible by \(a\), and let's assume \(p-1=a t u\). Then, because of \(h^{t} \equiv 1\), even \(h^{t u} \equiv 1\) or
\[A^{\frac{p-1}{a^{\alpha}} \cdot \frac{p-1}{a}} B^{\frac{p-1}{b^{\beta}} \cdot \frac{p-1}{a}} C^{\frac{p-1}{a^{\gamma}} \cdot \frac{p-1}{a}} \ldots \equiv 1\]
But evidently \(\frac{p-1}{a b^{\beta}}\) is an integer, therefore
\[B^{\frac{p-1}{b^{\beta}} \cdot \frac{p-1}{a}}=(B^{p-1})^{\frac{p-1}{a b^{\beta}}} \equiv 1\]
similarly also
\[C^{\frac{p-1}{c^{\gamma}} \cdot \frac{p-1}{a}} \equiv 1,\text{ and so on; hence it must be }A^{\frac{p-1}{a^{\alpha}} \cdot \frac{p-1}{a}} \equiv 1\]
Next, let a positive integer \(\lambda\) be determined so that
\(\lambda b^{\beta} c^{\gamma} \ldots \equiv 1\pmod{a}\)
which can be done, since the prime number \(a\) does not measure \(b^{\beta} c^{\gamma} \ldots\) itself, and let \(\lambda b^{\beta} c^{\gamma} \ldots=1+a \mu\). It evidently follows that
\[ A^{\lambda \cdot \frac{p-1}{a^{\alpha}} \cdot \frac{p-1}{a}} \equiv 1,\text{ thus, because }\lambda \cdot \frac{p-1}{a^{\alpha}} \cdot \frac{p-1}{a} = (1+a \mu) \frac{p-1}{a} = (p-1) \mu + \frac{p-1}{a}\]
we have \(A^{(p-1) \mu} \cdot A^{\frac{p-1}{a}} \equiv 1\), and hence, since \(A^{(p-1) \mu} \equiv 1\) naturally, also \(A^{\frac{p-1}{a}} \equiv 1\), which is against the hypothesis. Therefore, the assumption that \(t\) is a proper divisor of \(p-1\) cannot stand, and so \(h\) must necessarily be a primitive root.
%

\subsection*{54.}

Let \(h\) denote the primitive root for modulus \(m\), with norm \(=p\), then the terms of the progression
\[1, h, h^{2}, h^{3}, \ldots, h^{p-2}\]
will be incongruent to each other. Hence we easily conclude that any integer not divisible by the modulus must be congruent to one of these, in other words, it must exhibit the complete system of incongruent residues excluding zero. The exponent of the power to which a given number is congruent can be called the \textit{index} of this, while \(h\) is considered as the \textit{base}. Here are some examples in which we have given the absolute minimium residue for each index.
\begin{center}
\textit{First Example.}\\
\(m=5+4 i, \quad p=41, \quad h=1+2 i\)\\
\begin{tabular}{c|c||c|c||c|c||c|c||c|c}
Ind. & Residue & Ind. &Residue & Ind. & Residue & Ind. & Residue & Ind. & Residue \\
\hline
\(0\) & \(+1\phantom{\;+0i}\) & \(8\) & \(-4\phantom{\;+0i}\) & \(16 \)& \(-2+2 i\) & \(24\) & \(\phantom{+0}+2 i\) & \(32 \)& \(+1+\phantom{1}i\) \\
\(1\) & \(+1+2 i\) & \(9\) & \(-3+\phantom{1}i\) & \(17\) & \(-1+2 i\) & \(25\) & \(\phantom{+0}-3 i\) & \(33\) & \(+1+3 i\) \\
\(2\) & \(+1-\phantom{1}i\) & \(10\) & \(\phantom{+0}-\phantom{1}i\) & \(18\) & \(\phantom{+0}+4 i\) & \(26\) & \(+2+2 i\) & \(34\) & \(+2\phantom{\;+0i}\) \\
\(3\) & \(+3+\phantom{1}i\) & \(11\) & \(+2-\phantom{1}i\) & \(19\) & \(+1+3 i\) & \(27\) & \(+2+\phantom{1}i\) & \(35\) & \(-3\phantom{\;+0i}\)\\
\(4\) & \(\phantom{+0}-2 i\) & \(12\) & \(-1-\phantom{1}i\) & \(20\) & \(-1\phantom{\;+0i}\) & \(28\) & \(+4\phantom{\;+0i}\) & \(36\) & \(+2-2 i\) \\
\(5\) & \(\phantom{+0}+3 i\) & \(13\) & \(+1-3 i\) & \(21\) & \(-1-2 i\) & \(29\) & \(+3-\phantom{1}i\) & \(37\) & \(+1-2 i\) \\
\(6\) & \(-2-2 i\) & \(14\) & \(-2\phantom{\;+0i}\) & \(22\) & \(-1+\phantom{1}i\) & \(30\) & \(\phantom{+0}+\phantom{1}i\) & \(38\) & \(\phantom{+0}-4 i\) \\
\(7\) & \(-2-\phantom{1}i\) & \(15\) & \(+3\phantom{\;+0i}\) & \(23\) & \(-3-\phantom{1}i\) & \(31\) & \(-2+\phantom{1}i\) & \(39\) & \(-1-3 i\) \\
\end{tabular}
\end{center}
%

\begin{center}
\textit{Example 2.}\\
\(m=7, p=49, h=1+2 i\)\\
\begin{tabular}{c|c||c|c||c|c||c|c||c|c}
Index & Residue & Index & Residue & Index & Residue & Index & Residue & Index & Residue \\
\hline
\(0\) & \(+1\phantom{\;+0i}\) & \(10\) & \(-1-\phantom{1}i\) & \(20\) & \(\phantom{+0}+2 i\) & \(30\) & \(+2-2 i\) & \(40\) & \(+3\phantom{\;+0i} \)\\[-4pt]
\(1\) & \(+1+2 i\) & \(11\) & \(+1-3 i\) & \(21\) & \(+3+2 i\) & \(31\) & \(\) & \(41\) & \(+3-\phantom{1}i \)\\[-4pt]
\(2\) & \(-3-3 i\) & \(12\) & \(\phantom{+0}-\phantom{1}i\) & \(22\) & \(-1+\phantom{1}i\) & \(32\) & \(+2\phantom{\;+0i}\) & \(42\) & \(-2-2 i \)\\[-4pt]
\(3\) & \(+3-2 i\) & \(13\) & \(+2-\phantom{1}i\) & \(23\) & \(-3-\phantom{1}i\) & \(33\) & \(\phantom{+0}-3 i\) & \(43\) & \(+2+\phantom{1}i \)\\[-4pt]
\(4\) & \(\phantom{+0}-3 i\) & \(14\) & \(-3+3 i\) & \(24\) & \(-1\phantom{\;+0i}\) & \(34\) & \(+1+\phantom{1}i\) & \(44\) & \(\phantom{+0}-2 i \)\\[-4pt]
\(5\) & \(-1-3 i\) & \(15\) & \(-2-3 i\) & \(25\) & \(-1-2 i\) & \(35\) & \(-1+3 i\) & \(45\) & \(-3-2 i \)\\[-4pt]
\(6\) & \(-2+2i\) & \(16\) & \(-3\phantom{\;+0i}\) & \(26\) & \(+3\phantom{\;+0i}\) & \(36\) & \(\) & \(46\) & \(+1-\phantom{1}i \)\\[-4pt]
\(7\) & \(+1-2 i\) & \(17\) & \(-3+\phantom{1}i\) & \(27\) & \(-3+2 i\) & \(37\) & \(-2+\phantom{1}i\) & \(47\) & \(+3+\phantom{1}i \)\\[-4pt]
\(8\) & \(-2\phantom{\;+0i}\) & \(18\) & \(+2+2 i\) & \(28\) & \(\phantom{+0}+3 i\) & \(38\) & \(+3-3 i\) & \multicolumn{2}{c}{}\\[-4pt]
\(9\) & \(-2+3 i\) & \(19\) & \(-2-\phantom{1}i\) & \(29\) & \(+1+3 i\) & \(39\) & \(+2+3 i\) & \multicolumn{2}{c}{}
\end{tabular}
\end{center}
%

\subsection*{55.}

We add some observations about primitive roots and indices algorithm, omitting the proofs for the sake of simplicity.

I. Indices congruent to \(p-1\) modulus in a given system correspond to residues congruent to \(m\) modulus and vice versa.

II. The residues corresponding to the indices to the \(p-1\) primes are also primitive roots and vice versa.

III. If the primitive root \(h\) is accepted as the base, the index of another primitive root \(h^{\prime}\) is \(t\), and vice versa, \(t^{\prime}\) is the index of \(h\) itself when \(h^{\prime}\) is taken as the base, then \(t t^{\prime} \equiv 1\pmod{p-1}\); and if the same indices of any other number in these two systems are \(u\), \(u^{\prime}\) respectively, then \(t u^{\prime} \equiv u\), \(t^{\prime} u \equiv u^{\prime}\pmod{p-1}\).

IV. While the numbers \(1\), \(1+i\) and their three companions (as too meager) are excluded from the moduli we consider, the prime numbers remaining are those we placed in articles 34 in the third and fourth place. The norms of the latter will be prime numbers of the form \(4n+1\); the norms of the former will be the squares of prime numbers of real form: in both cases, therefore, \(p-1\) is divisible by 4.
%

\(\text{V.}\) By denoting the index of the number \(-1\) as \(u\), it will be \(2 u \equiv 0\pmod{p-1}\), and therefore either \(u \equiv 0\) or \(u \equiv \frac{1}{2}(p-1)\): but since the index 0 corresponds to the residue \(+1\), the index of the number \(-1\) must necessarily be \(\frac{1}{2}(p-1)\).

\(\text{VI.}\) Likewise, by denoting the index of the number \(i\) as \(u\), it will be \(2 u \equiv \frac{1}{2}(p-1) \pmod{p-1}\), and therefore either \(u \equiv \frac{1}{4}(p-1)\) or \(u \equiv \frac{3}{4}(p-1)\). But this ambiguity depends on the choice of the primitive root. Namely, if the primitive root \(h\) is taken as the base and the index of the number \(i\) is \(\frac{1}{4}(p-1\), then the index will become \(\frac{3}{4}(p-1)\) when \(h^{\mu}\) is taken as the base, where \(\mu\) denotes the first positive integer of the form \(4 n+3\) less than \(p-1\), for example the number \(p-2\), and vice versa. Therefore, by alternating the choice of primitive roots, the number \(i\) will have the index \(\frac{1}{4}(p-1)\) for one base, and the index \(\frac{3}{4}(p-1)\) for the other, and it is clear that for these bases, \(-i\) will have the index \(\frac{3}{4}(p-1)\), and for these, it will have the index \(\frac{1}{4}(p-1)\).

\(\text{VII.}\) Whenever the modulus is a positive real prime of the form \(4 n+3\), say \(=q\), and thus \(p=q q\), the indices of all real numbers will be divisible by \(q+1\); for denoting the index of the real number \(k\) as \(t\), it will be, due to \(k^{q-1} \equiv 1\pmod{q}\), \((q-1) t \equiv 0\pmod{q q-1}\), and therefore \(\frac{t}{q+1}\) is an integer. Likewise, the indices of purely imaginary numbers like \(ki\) will be divisible by \(\frac{1}{2}(q+1)\). It is therefore clear that primitive roots for such moduli must be sought among only mixed numbers.

\(\text{VIII.}\) On the contrary, for the modulus \(m\) which is a prime complex mixed number (whose norm \(p\) is a prime real number of the form \(4 n+1\)), any primitive roots can be chosen among real numbers, among which a complete system of incongruous residues can be demonstrated (art. 40). However, any real number which is a primitive root for the complex modulus \(m\) will at the same time be a primitive root in the arithmetic of real numbers for the modulus \(p\), and vice versa.
%

\subsection*{56.}
 
Although the theory of residues and non-residues of quadratic numbers is contained within the theory of residues of biquadratic numbers in the arithmetic of complex numbers, before we transition to this, we will present separately its remarkable theorems here. For the sake of brevity, however, we will speak here only about the principal case, where the modulus is a complex prime number (odd).
%

Let \(m\) be such a modulus, and \(p\) its norm. Manifestly, any integer (always understood to be indivisible by \(m\)) can either be congruent to a quadratic residue modulo \(m\) or not, depending on whether its index, taken with some primitive root as a base, is even or odd. In the former case, that integer is called the quadratic residue of \(m\), and in the latter, it is a non-residue. It is concluded from this that among the \(p-1\) numbers that constitute a complete system of incongruent residues (indivisible by \(m\)), half are quadratic residues and the other half are non-quadratic residues. For any other number outside this system, the same characteristic is attributed as the number that belongs to the system in this respect.

Furthermore, it follows from this that the product of two quadratic residues, as well as the product of two non-residues, is a quadratic residue. Conversely, the product of a quadratic residue with a non-residue results in a non-residue. Generally, the product of any number of factors is a quadratic residue or a non-residue, depending on whether the number of non-residues among the factors is even or odd.
%

Let's start with the number \(1+i\), which is found to be a quadratic residue modulo\\
\(-1+2 i\), \(+3-2 i\), \(-5-2 i\), \(-1-6 i\), \(+5+4 i\), \(+5-4 i\), \(-7\), \(+7+2 i\), \(-5+6 i\), etc.\\
While the following numbers are non-quadratic residues:\\
\(-1-2 i\), \(-3\), \(+3+2 i\), \(+1+4 i\), \(+1-4 i\), \(-5+2 i\), \(-1+6 i\), \(+7-2 i\), \(-5-6 i\), \(-3+8 i\), \(-3-8 i\), \(+5+8 i\), \(+5-8 i\), \(+9+4 i\), \(+9-4 i\), etc.
%

\subsection*{57.}

It is indeed easy, given a modulus, to complete the system of incongruous residues into two classes, namely quadratic residues and non-residues, in such a way that at the same time its classes are spontaneously assigned to all the remaining numbers. However, a much more profound inquiry is the question of criteria for distinguishing the moduli for which a given number is a quadratic residue from those for which it is a non-residue.

As regards the real units \(+1\) and \(-1\), these in the arithmetic of complex numbers are actually squares, and therefore also quadratic residues for \textit{any} modulus. Equally easily from the criterion in the preceding article it follows that the number \(i\) (and similarly \(-i\)) is a quadratic residue for any modulus whose norm is of the form \(8n+1\), and a non-residue for any modulus whose norm is of the form \(8n+5\). Since clearly it makes no difference whether the number \(m\) or any of the associated numbers \(im\), \(-m\), \(-im\) is adopted as the modulus, it may be assumed, according to article 36, II, that the modulus is prime-associated, and hence by stipulating the modulus \(=a+bi\), that \(a\) is odd and \(b\) is even. Thus, since it is always the case that \(a^2\equiv 1\pmod{8}\), and \(b^2\) is either \(\equiv 0\) or \(\equiv 4\pmod{8}\), depending on whether \(b\) is also even or is odd, it is clear that the numbers \(+i\) and \(-i\) are quadratic residues of the modulus in the former case, and non-residues in the latter.
%

\subsection*{58.}
 
When determining the character of a composite number, whether it is a quadratic residue or non-residue, depends on the characters of the factors, it will be sufficient to limit the development of criteria for distinguishing the moduli for which a given number \(k\) is a quadratic residue, from those for which it is a non-residue, to such values of \(k\) which are prime numbers, and moreover to those among them which are associated primes. In this investigation, \textit{induction} immediately provides particularly elegant theorems.
%

\text{Let's start with the number } 1+i \text{, which is found to be the residue of the squares of the moduli}\\
-1+2 i, +3-2 i, -5-2 i, -1-6 i, +5+4 i, +5-4 i, -7, +7+2 i, -5+6 i, \text{etc.}\\
\text{but not residue of the squares of the following}\\
-1-2 i, -3, +3+2 i, +1+4 i, +1-4 i, -5+2 i, -1+6 i, +7-2 i, -5-6 i, -3+8 i, -3-8 i, +5+8 i, +5-8 i, +9+4 i, +9-4 i \text{etc.}
%

\text{If we examine this view, in which we have always *primarium* associated with the four modules, carefully, we easily notice that the modules $a+bi$ are all such in the former class, for which $a+b \equiv +1 \pmod{8}$, and in the latter class, for which $a+b \equiv -3 \pmod{8}$. Clearly this criterion, if we adopt the associated $-m$ instead of the primitive module $m$, must be modified in such a way that for the modules of the former class it should be $a+b \equiv -1$, and for the modules of the latter class $\equiv +3 \pmod{8}$. Therefore, if the induction has not failed, in general, designating the prime number by $a+bi$, where $a$ is odd, $b$ is even, $1+i$ will become its quadratic residue or non-quadratic residue, as $a+b \equiv \pm 1$, or $\equiv \pm 3 \pmod{8}$.}
%

\(\text{The same rule holds for the number } -1 - i \text{ as for } 1 + i. \text{ However, by considering } 1 - i \text{ as a product of } -i \text{ and } 1 + i, \text{ it is clear that the number } 1 - i \text{ has the same characteristic as that to be assigned to } 1 + i \text{ whenever } b \text{ is also even, and the opposite whenever } b \text{ is odd, from which it is easily inferred that } 1 - i \text{ is the quadratic residue of the prime number } a + bi \text{ whenever } a - b \equiv \pm 1, \text{ and nonresidue whenever } a - b \equiv \pm 3 \pmod{8}, \text{ always assuming that } a \text{ is odd and } b \text{ is even.}\)
%

Moreover, this second proposition can also be deduced from the previous one, with the help of the following more general theorem, which we state as follows:


\textit{In the theory of quadratic residues, the characteristic of the number \(\alpha+\beta i\) with respect to the modulus \(a+b i\) is the same as that of the number \(\alpha-\beta i\) with respect to the modulus \(a-b i\).}

\textit{The proof of this theorem is sought from the fact that each modulus has the same norm \(p\), and as many times as \((\alpha+\beta i)^{\frac{1}{2}(p-1)}-1\) is divisible by \(a+b i\), \((\alpha-\beta i)^{\frac{1}{2}(p-1)}-1\) must also be divisible by \(a-b i\); and as many times as \((\alpha+\beta i)^{\frac{1}{2}(p-1)}+1\) is divisible by \(a+b i\), \((\alpha-\beta i)^{\frac{1}{2}(p-1)}+1\) must also be divisible by \(a-b i\).}
%

\subsection*{59.}
Let us proceed to odd prime numbers.

We find that the number \(-1+2 i\) is a quadratic residue of moduli \(+3+2 i\), \(+1-4 i\), \(-5+2 i\), \(-5-2 i\), \(-1-6 i\), \(+7-2 i\), \(-3+8 i\), \(+5+8 i\), \(+5-8 i\), \(+9+4 i\), etc.

And it is a non-residue of moduli \(-1-2 i\), \(-3\), \(+3-2 i\), \(+1+4 i\), \(-1+6 i\), \(+5+4 i\), \(+5-4 i\), \(-7\), \(+7+2 i\), \(-5+6 i\), \(-5-6 i\), \(-3-8 i\), \(+9-4 i\), etc.

By reducing the moduli of the former class to their residues, absolute minimum with respect to the modulus \(-1+2 i\) is found. We find only \(+1\) and \(-1\), shown as follows: \(+3+2 i \equiv -1\), \(+1-4 i \equiv -1\), \(-5+2 i \equiv +1\), \(-5-2 i \equiv -1\), etc.

On the other hand, all moduli of the latter class are found to be congruent to \(+i\) or \(-i\) with respect to the modulus \(-1+2 i\).

Now, the numbers \(+1\) and \(-1\) are residues of the quadratic modulus \(-1+2 i\), while \(+i\) and \(-i\) are non-residues. Hence, as far as induction is concerned, the theorem follows: The number \(-1+2 i\) is a quadratic residue or non-residue of the prime number \(a+b i\), depending on whether it is a quadratic residue or non-residue of \(-1+2 i\), provided that \(a+b i\) is a member of the associated quaternions, or rather, if \(a\) is odd and \(b\) is even.

Furthermore, from this theorem, analogous theorems follow naturally concerning the numbers \(+1-2 i\), \(-1-2 i\), \(+1+2 i\).
%

\subsection*{60.}

By establishing a similar induction around the number \(-3\) or \(+3\), we find that each of the following are quadratic residues of modulus \(+3+2 i\), \(+3-2 i\), \(-1+6 i\), \(-1-6 i\), \(-7\), \(-5+6 i\), \(-5-6 i\), \(-3+8 i\), \(-3-8 i\), \(+9+4 i\), \(+9-4 i\) etc.

However, none of the following are residues: \(-1+2 i\), \(-1-2 i\), \(+1+4 i\), \(+1-4 i\), \(-5+2 i\), \(-5-2 i\), \(+5+4 i\), \(+5-4 i\), \(+7+2 i\), \(+7-2 i\), \(+5+8 i\), \(+5-8 i\) etc.

The former set, according to the modulus of \(3\), is congruent to one of the four numbers \(+1\), \(-1\), \(+i\), \(-i\); while the latter is congruent to one of \(+1+i\), \(+1-i\), \(-1+i\), \(-1-i\). The former are the exact quadratic residues of the number \(3\), the latter are not.

Therefore, this induction teaches that the prime number \(a+b i\), assuming \(a\) is odd and \(b\) is even, has the same relationship with the number \(-3\) (and also with \(+3\)) as it does with it, specifically whether each is the quadratic residue of the other or not.
%

Extending a similar induction to other prime numbers, everywhere we have found this most elegant law of reciprocity confirmed, and we are brought to this fundamental theorem concerning quadratic residues in the arithmetic of complex numbers. 
 
\textit{Letting \(a+b i\), \( A+B i\) be prime numbers, such that \(a\), \(A\) are odd, \(b\), \( B\) are even: either both of them are quadratic residues of the other, or both of them are non-residues of the other.} 
 
Despite the great simplicity of the theorem, its proof is pressed by great difficulties, which, however, we do not dwell on here, since the theorem itself is only a special case of a more general theorem, almost exhausting the sum of the theory of biquadratic residues. Let us now move on to this.
%

\subsection*{61.}

The concepts expounded in article 2 of the previous treatise regarding the notion of quadratic residue and nonresidue are also extended to the arithmetic of complex numbers, and similarly, here as there, our examination is limited to moduli that are prime numbers; furthermore, it will generally be understood tacitly that the modulus should be taken such that it is prime to the associated numbers, for example, \(\equiv 1\) modulo \(2+2i\), and also that the numbers, with regard to their characteristics as quadratic residues or nonresidues, are not divisible by the modulus.

Thus, given a modulus, the numbers not divisible by it could be divided into three classes, the first of which would contain quadratic residues, the second non-residues which are quadratic residues, and the third non-quadratic residues.
%

But it is better to establish two in place of the third class, so that in total there are four.

Given any primitive root as the base, the residues of biquadratics will have exponents divisible by 4 or of the form \(4n\); the non-residues which are residues of quadratics will have exponents of the form \(4n+2\); finally, the exponents of non-quadratic non-residues will be partly of the form \(4n+1\), partly of the form \(4n+3\). In this way, four classes indeed arise, but the distinction between the two posterior classes would not be absolute, but would depend on the choice of the assumed primitive root; for it is easy to see that, given a non-quadratic residue, one can reconcile the exponent of the form \(4n+1\) with half of the primitive roots, and the other half with the exponent of the form \(4n+3\). In order to remove this ambiguity, we will always suppose that such a primitive root is adopted, for which the exponent \(\frac{1}{4}(p-1)\) corresponds to the number \(+i\) (cf. art. 55, VI). In this way, a classification arises, which we can more concisely enunciate independently of the primitive roots.

The \textit{first} class contains the numbers \(k\) for which \(k^{\frac{1}{4}(p-1)} \equiv 1\); these numbers are the moduli of the residues of biquadratics.

The \textit{second} class contains those for which \(k^{\frac{1}{4}(p-1)} \equiv i\).

The \textit{third} class contains those for which \(k^{\frac{1}{4}(p-1)} \equiv-1\).

Lastly, the \textit{fourth} class contains those for which \(k^{\frac{1}{4}(p-1)} \equiv-i\).

The third class will include non-quadratic residues which are quadratic residues; the non-quadratic residues will be distributed between the second and fourth.

We assign the respective numbers of these classes as \textit{biquadratic characters} \(0\), \(1\), \(2\), \(3\). If we define the character \(\lambda\) of the number \(k\) according to the modulus \(m\) as the exponent of the power of \(i\) to which the number \(k^{\frac{1}{2}(p-1)}\) is congruent, then it is evident that characters congruent to 4 modulo the equivalent are to be considered equivalent. However, this notion is limited for the time being to moduli which are prime numbers: in the continuation of these discussions, we will show how it can also be adapted to composite moduli.
%

\subsection*{62.}
 
To make it easier to construct a comprehensive induction about the properties of numbers, we attach a concise table here, by the help of which the character of any given number with respect to a modulus, whose norm does not exceed the value 157, can be easily obtained, provided attention is paid to the following observations.
 
Since the character of a composite number is equal to the aggregate of the characters of its individual factors (i.e., congruent modulo 4), it suffices if we can assign characters of prime numbers for the given modulus. Moreover, since the characters of the units \(-1\), \( i\), \(-i\) are clearly congruent to the numbers \(\frac{1}{2}(p-1)\), \( \frac{1}{4}(p-1)\), \( \frac{3}{4}(p-1)\) modulo 4, it is also sufficient to have exhibited the characters of numbers among the associated primes. Furthermore, it suffices to include the characters of those numbers in the table which are contained in the system of absolutely minimal residues and are congruent to the same character modulo \(m\). Moreover, by the same reasoning as demonstrated in art. 58, if the character of the number \(A+B i\) is \(\lambda\) for the modulus \(a+b i\), and the character of the number \(A-B i\) is \(\lambda^{\prime}\) for the modulus \(a-b i\), then it is always \(\lambda \equiv-\lambda^{\prime}\pmod{4}\), or equivalently, \(\lambda+\lambda^{\prime}\) is divisible by 4. Therefore, it suffices to include the moduli in the table in which \(b\) is either 0 or positive.
%

Thus, for example, if we seek the characteristic of the number \(11-6 i\) with respect to the modulus \(-5-6 i\), we substitute these numbers in place of the given numbers, namely \(11+6 i,-5+6 i\); then we determine (art. 43)\footnote{Article 43 is not provided in this excerpt.} the absolute minimum residue of the number \(11+6 i\) with respect to the modulus \(-5+6 i\), which becomes \(-1-4 i=-1 \times(1+4 i)\); therefore, since the characteristic of \(-1\) with respect to the modulus \(-5+6 i\) is 30, and the characteristic of the number \(1+4 i\), from the table, is 2, therefore the characteristic of the number \(11+6 i\) with respect to the modulus \(-5+6 i\) will be 32 or 0, and consequently, by the ultimate observation, the characteristic of the number \(11-6 i\) with respect to the modulus \(-5-6 i\) will also be. Similarly, if we seek the characteristic of the number \(-5+6 i\) with respect to the modulus \(11+6 i\), its absolute minimum residue \(1-5 i\) is resolved into the factors \(-i\), \(1+i\), \(3-2 i\), which correspond to the characteristics \(117\), \(0\), \(1\), whence the sought characteristic will be 118 or 2; the same characteristic is also to be attributed to the number \(-5-6 i\) with respect to the modulus \(11-6 i\).
\begin{center}
\begin{tabular}{l|c|l}
Modulus. & Characteristic. & Number. \\
\hline
\(-3\) & 3 & \(\phantom{+}1+i\) \\
\(+3+2 i\) & 3 & \(\phantom{+}1+i\) \\
\(+1+4 i\) & 1 & \(-1+2 i\) \\
 & 3 & \(\phantom{+}1+i\) \\
\(-5+2 i\)& 0 & \(-1-2 i\) \\
 & 1 & \(\phantom{+}1+i\) \\
& 2 & \(-1+2 i\) \\
\(-1+6 i\) & 0 & \(-3\) \\
 &1&\(1+i\), \(-1+2i\)
\end{tabular}
\end{center}
%

\begin{center}
\begin{tabular}{c|c|l}
Modulus. & Character. & Numeri. \\
\hline
\(-1+6 i\) & 2 & \(-1-2 i\) \\
 \(+5+4 i\)& 0 & \(\phantom{+}1+i\) \\
 & 1 & \(-3\) \\
 & 3 & \(-1+2 i\), \(-1-2 i\) \\
\(-7\)  & 0 & \(-3\) \\
 & 1 & \(-1+2 i\), \(-3-2 i\) \\
 & 2 & \(\phantom{+}1+i\) \\
 & 3 & \(-1-2 i\) \\
\(+7+2 i\) & 0 & \(\phantom{+}1+i\), \( 3+2 i\), \( 3-2 i\), \( 1-4 i\) \\
 & 1 & \(-3\) \\
 & 2 & \(-1-2 i\), \( 1+4 i\) \\
 & 3 & \(-1+2 i\) \\
\(-5+6 i\) & 0 & \(\phantom{+}1+i\), \(-3\), \(3+2 i\), \( 3-2 i\) \\
 & 1 & \(\phantom{+}1-4 i\) \\
 & 2 & \(\phantom{+}1+4 i\) \\
 & 3 & \(-1+2 i\), \(-1-2 i\) \\
\(-3+8 i\) & 0 & \(-1+2 i\), \( 3-2 i\), \( 1-4 i\) \\
 & 1 & \(\phantom{+}1+i\), \( 3+2 i\) \\
 & 2 & \(-3\) \\
 & 3 & \(-1-2 i\), \( 1+4 i\), \(-5+2 i\) \\
\(+5+8 i\) & 0 & \(-1-2 i\) \\
 & 1 & \(-5-2 i\), \(-1+6 i\) \\
 & 2 & \(-1+2 i\), \( 3-2 i\) \\
 & 3 & \(\phantom{+}1+i\), \(-3\), \(3+2 i\), \( 1+4 i\), \( 1-4 i\) \\
\(+9+4 i\) & 0 & \(-1+2 i\), \( 3+2 i\) \\
 & 1 & \(\phantom{+}1+i\), \(-1-2 i\), \( 3-2 i\) \\
 & 2 & \(-3\), \(1+4 i\) \\
 & 3 & \(\phantom{+}1-4 i\), \(-5+2 i\) \\
\(-1+10 i\) & 0 & \(\phantom{+}1+i\), \(-1+2 i\), \(-1-2 i\), \( 3+2 i\) \\
 & 1 & \(-3\) \\
 & 2 & \(\phantom{+}3-2 i\), \(-5+2 i\), \( 5-4 i\) \\
 & 3 & \(\phantom{+}1+4 i\), \( 1-4 i\) \\
\end{tabular}
\end{center}
%

\begin{center}
\begin{tabular}{l|c|l}
Modulus. & Character. & Numeri. \\
\hline
\(+3+10 i\) & 1 & \(\phantom{+}1+i\), \(-1-2 i\), \(1-4 i\) \\
 & 2 & \(-3\), \(3+2 i\), \(1+4 i\), \(-5-2 i\) \\
 & 3 & \(-1+2 i\), \(3-2 i\) \\
\(-7+8 i\) & 0 & \(\phantom{+}1+i\), \(-7\) \\
 & 1 & \(\phantom{+}3+2 i\), \(3-2 i\), \(1-4 i\), \(-5-2 i\) \\
 & 2 & \(-1-2 i\), \(1+4 i\), \(-5+2 i\), \(-1-6 i\) \\
 & 3 & \(-1+2 i\), \(-3\), \(-1+6 i\) \\
\(-11\) & 0 & \(-3\) \\
 & 1 & \(\phantom{+}1+i\), \(3-2 i\), \(1+4 i\), \(-5+2 i\), \(5+4 i\) \\
 & 2 & \(-1+2 i\), \(-1-2 i\) \\
 & 3 & \(\phantom{+}3+2 i\), \(1-4 i\), \(-5-2 i\), \(5-4 i\) \\
\(-11+4 i\) & 0 & \(\phantom{+}1+i\), \(-1+2 i\), \(3+2 i\), \(5+4 i\) \\
 & 1 & \(-1-2 i\), \(-1+6 i\) \\
 & 2 & \(-5+2 i\) \\
 & 3 & \(-3\), \(3-2 i\), \(1+4 i\), \(1-4 i\), \(-5-2 i\) \\
\(+7+10 i\) & 0 & \(\phantom{+}1+4 i\), \(1-4 i\), \(-1+6 i\), \(-1-6 i\) \\
 & 1 & \(-1+2 i\), \(3+2 i\), \(-5+2 i\) \\
 & 2 & \(\phantom{+}1+i\), \(3-2 i\) \\
 & 3 & \(-1-2 i\), \(-3\), \(-5-2 i\) \\
\(+11+6 i\) & 0 & \(\phantom{+}1+i\), \(-1+2 i\), \(-3\), \(1+4 i\), \(1-4 i\), \(-7\) \\
 & 1 & \(-1-2 i\), \(3+2 i\), \(3-2 i\) \\
 & 2 & \(-5-2 i\), \(-1+6 i\), \(-5-4 i\) \\
 & 3 & \(-5+2 i\), \(5+4 i\), \(7-2 i\) \\
\end{tabular}
\end{center}
%

\subsection*{63.}
We will now give attention to detecting common criteria for moduli, for which a given prime number has the same character, by induction. We always assume prime moduli among associates, for example such as \(a+b i\), for which either \(a \equiv 1\), \( b \equiv 0\), or \(a \equiv 3\), \( b \equiv 2\pmod{4}\).

With respect to the number \(1+i\), from which we begin the induction, the law is more easily grasped if we separate the moduli of the prior type (for which \(a \equiv 1\), \( b \equiv 0\)) from the moduli of the latter type (for which \(a \equiv 3\), \( b \equiv 2\)). With the help of the table in the previous article, we find the answer\clearpage\noindent\begin{center}
\begin{tabular}{c|l}
character & \multicolumn{1}{c}{moduli of the first type} \\
\hline
0 & \(\phantom{+}5+4 i\), \(-7+8 i\), \(-7-8 i\), \(-11+4 i\) \\
1 & \(\phantom{+}1-4 i\), \(-3+8 i\), \(-3-8 i\), \( 9+4i\), \(-11\) \\
2 & \(\phantom{+}5-4 i\), \(-7\), \(-11-4 i\) \\
3 & \(-3\), \(1+4 i\), \( 5+8 i\), \( 5-8 i\), \( 9-4 i\) \\
\end{tabular}
\end{center}
If we consider these seventeen examples attentively, we find in all of them the character \(\equiv \frac{1}{4}(a-b-1)\pmod{4}\).
%

\begin{center}
\begin{tabular}{c|l}
character & \multicolumn{1}{c}{second type moduli.} \\
\hline
0 & \(\phantom{+}3-2 i\), \(-1-6 i\), \( 7+2 i\), \(-5+6 i\), \(-1+10 i\), \( 11+6 i\) \\
1 & \(-5+2 i\), \(-1+6 i\), \( 7-2 i\), \(-1-10 i\), \( 3+10 i\) \\
2 & \(-1+2 i\), \(-5-2 i\), \( 3-10 i\), \( 7+10 i\) \\
3 & \(-1-2 i\), \( 3+2 i\), \(-5-6 i\), \( 7-10 i\), \( 11-6 i\) \\
\end{tabular}
\end{center}

In all of these twenty examples, with a little attention, the character is found to satisfy \(\equiv \frac{1}{4}(a-b-5)\pmod{4}\).
%

Easily, one can condense two rules into one able to be applied to both types of moduli, if we consider \(\frac{1}{4} b b\) to be \( \equiv 0\pmod{4}\) for moduli of the former type, and \( \equiv 1\pmod{4}\) for moduli of the latter type. Thus, the character of the number \(1+i\) with respect to any prime modulus is \(\equiv \frac{1}{4}(a-b-1-b b)\pmod{4}\).

By the way, it is convenient to note here, since \((b+1)^{2}\) is always of the form \(8 n+1\), i.e., \(\frac{1}{4}(2 b+b b)\) is even, this character will always be even or odd, according to whether \(\frac{1}{4}(a+b-1)\) is even or odd, which squares with the rule for the quadratic character stated in article 58.

Since \(\frac{1}{4}(a-b-1)\), \( \frac{1}{4}(a-b+3)\) are integers, of which one is even and the other odd, their product will be even, so \(\frac{1}{8}(a-b-1)(a-b+3) \equiv 0\pmod{4}\). Hence, in place of the expression for the biquadratic character, this can also be adopted
\[\frac{1}{4}(a-b-1-b b)-\frac{1}{8}(a-b-1)(a-b+3)=\frac{1}{8}(-a a+2 a b-3 b b+1)\]
which form also recommends itself by the fact that it is not restricted to prime moduli, but only assumes that \(a\) is odd, \(b\) is even; indeed, in this assumption, either \(a+b i\), or \(-a-b i\) will be a prime associated number, and the value of this formula will be the same for both moduli.\\
%

\subsection*{64.}

Departing from the last rule extracted in the previous article, we find that there are
\begin{center}
\begin{tabular}{c|l}
numbers & \multicolumn{1}{c}{character \(\equiv\)} \\
\hline
\(-1+i\) & \(\frac{1}{8}(a a+2 a b-b b-1)\) \\
\(-1-i\) & \(\frac{1}{8}(-a a+2 a b+b b+1)\) \\
\(+1-i\) & \(\frac{1}{8}(a a+2 a b+3 b b-1)\) \\
\end{tabular}
\end{center}
This immediately implies that the character of \(i\) is \(\frac{1}{4}(a a+b b-1)\), and the character of \(-1\) is\(\frac{1}{2}(a a+b b-1) \equiv \frac{1}{2} b b\), since \(a a-1\) is always of the form \(8n\). Clearly these four rules, even if they have so far been borrowed from induction, are so interconnected that as soon as the demonstration of one is complete, the other three are demonstrated simultaneously. There is scarcely any need to mention that in these rules we only assume \(a\) to be odd and \(b\) to be even.
%


If you do not mind using formulas restricted to primary modules, we can use them in the following way. It is
\begin{center}
\begin{tabular}{c|l}
numbers &\multicolumn{1}{c}{character \(\equiv\)} \\
\hline
\(-1+i\) & \(\frac{1}{4}(-a-b+1-b b)\) \\
\(-1-i\) & \(\frac{1}{4}(a-b-1+b b)\) \\
\(+1-i\) & \(\frac{1}{4}(-a-b+1+b b)\) \\
\end{tabular}
\end{center}
The simplest formulas emerge if, as we did at the beginning of our induction, we distinguish between primary and secondary modules. That is, the character is
\begin{center}
\begin{tabular}{c|c|c}
numbers & for primary modules & for secondary modules \\
\hline
\(-1+i\) & \(\frac{1}{4}(-a-b+1)\) & \(\frac{1}{4}(-a-b-3)\) \\
\(-1-i\) & \(\frac{1}{4}(a-b-1)\) & \(\frac{1}{4}(a-b+3)\) \\
\(+1-i\) & \(\frac{1}{4}(-a-b+1)\) & \(\frac{1}{4}(-a-b+5)\) \\
\end{tabular}
\end{center}
%

\subsection*{65.}

For the number \(-1+2 i\), to which we now proceed, we will use the same distinction between the moduli \(a+b i\) for which \(a \equiv 1\), \( b \equiv 0\), and those for which \(a \equiv 3\), \( b \equiv 2\). Table in article 62 shows, with respect to this number, the correspondence to\clearpage
\begin{center}
\begin{tabular}{c|l}
character & \multicolumn{1}{c}{moduli of the first kind} \\
\hline
0 & \(-3+8 i\), \(+5-8 i\), \(+9+4 i\), \(-11+4 i\) \\
1 & \(+1+4 i\), \(+5-4 i\), \(-7\), \(-3-8 i\) \\
2 & \(+1-4 i\), \(+5+8 i\), \(-7-8 i\), \(-11\) \\
3 & \(-3\), \(+5+4 i\), \(+9-4 i\), \(-7+8 i\), \(-11-4 i\) \\
\end{tabular}
\end{center}

Examining each of these moduli with respect to the smallest absolute residue according to the modulus \(-1+2 i\), we observe that all those corresponding to character 0 are congruent to \(1\); those corresponding to character 1 are congruent to \(i\); those with character 2 become congruent to \(-1\); finally, all those with character 3 become congruent to \(-i\). Now the characters of the numbers \(1\), \( i\), \(-1\), \(-i\) with respect to the modulus \(-1+2 i\) are themselves \(0\), \(1\), \(2\), \(3\) respectively, thus in all these 17 examples the character of the number \(-1+2 i\) with respect to the modulus of the first kind \(a+b i\) is identical to the character of this number with respect to the modulus \(-1+2 i\).
%

\begin{center}
\begin{tabular}{c|l}
character & \multicolumn{1}{c}{second-order residues} \\
\hline
0 & \(+3+2 i\), \(-5-2 i\), \(-1+10 i\), \(-1-10 i\), \(+11+6 i\) \\
1 & \(+3-2 i\), \(-1+6 i\), \(-5-6 i\), \(+7+10 i\), \(+7-10 i\) \\
2 & \(-5+2 i\), \(-1-6 i\), \(+7-2 i\) \\
3 & \(-1-2 i\), \(+7+2 i\), \(-5+6 i\), \(+3+10 i\), \(+3-10 i\), \(+11-6 i\) \\
\end{tabular}
\end{center}
 
By recalling these residues to the minimum with respect to the modulus \(-1+2 i\), all, to which the characters \(0\), \(1\), \(2\), \(3\) correspond, are found to be congruent to the numbers \(-1\), \(-i\), \(+1\), \(+i\); to these same numbers, however, if \(-1+2 i\) is adopted as modulus, the characters \(2\), \(3\), \(0\), \(1\) correspond. Therefore, in all these 19 examples, the character number \(-1+2 i\) differs from the character of this number with respect to the number \(-1+2 i\) as modulus by two units.
 
Moreover, it is easily understood that the situations will be completely similar with respect to the number \(-1-2 i\).
%

\subsection*{66.}

We omit the distinction between the modules of the first and second kind for the number \(-3\), since experience shows that it is unnecessary here. The response is thus
\begin{center}
\begin{tabular}{c|l}
character & \multicolumn{1}{c}{modules} \\
\hline
0 & \(-1+6 i\), \(-1-6 i\), \(-7\), \(-5+6 i\), \(-5-6 i\), \(-11\), \(11+6 i\), \( 11-6 i\) \\
1 & \(-1-2 i\), \( 1-4 i\), \(-5+2 i\), \( 5+4 i\), \( 7+2 i\), \( 5-8 i\), \(-1+10 i\), \(-7-8 i\),  \\
 & \quad \(-11-4 i\), \( 7-10 i\) \\
2 & \(3+2 i\), \( 3-2 i\), \(-3+8 i\), \(-3-8 i\), \( 9+4 i\), \( 3+10 i\), \( 3-10 i\) \\
3 & \(-1+2 i\), \( 1+4 i\), \(-5-2 i\), \( 5-4 i\), \( 7-2 i\), \( 5+8 i\), \(-1-10 i\), \(-7+8 i\), \\
 & \quad \(-11+4 i\), \( 7+10 i\) \\
\end{tabular}
\end{center}

Recalling these modules to the smallest residues according to the modulus \(3\), we see that the ones corresponding to character 0 become either \(\equiv 1\) or \(\equiv -1\); those with character 1 become either \(\equiv 1-i\) or \(\equiv -1+i\); those with character 2 become either \(\equiv i\) or \(\equiv -i\); and finally, those with character 3 become either \(\equiv 1+i\) or \(\equiv -1-i\). From this induction, we conclude that the character of the number \(-3\) for a modulus, which is a prime number among the associated primorials, is identical to the character of the number itself, when 3, or equivalently \(-3\), is considered as the modulus.
%

\subsection*{67.}

By a similar induction carried out with respect to other prime numbers, we find that the numbers \(3 \pm 2i\), \(-1 \pm 6i\), \(7 \pm 2i\), \(-5 \pm 6i\), etc., supply similar theorems, to which in Article 65 we came with respect to the number \(-1+2i\); on the other hand, the numbers \(1 \pm 4i\), \(5 \pm 4i\), \(-3 \pm 8i\), \(5 \pm 8i\), \(9 \pm 4i\), etc., behave just like the number \(-3\). Therefore, induction leads to the most elegant theorem, which, following the theory of quadratic residues in the arithmetic of real numbers, may be called \textsc{The Fundamental Theorem} of the theory of biquadratic residues, namely:

\textit{Denoting by \(a+bi\), \(a^{\prime}+b^{\prime}i\) distinct prime numbers among their associates, i.e., congruent to the unit according to the modulus \(2+2i\), the biquadratic character of the number \(a+bi\) with respect to the modulus \(a^{\prime}+b^{\prime}i\) will be identical with the character of the number \(a^{\prime}+b^{\prime}i\) with respect to the modulus \(a+bi\), if either both of the numbers \(a+bi\), \(a^{\prime}+b^{\prime}i\), or at least one, is referred to the first genus, i.e., is congruent to the unit according to the modulus 4: on the contrary, the characters will be different by two units, if neither of the numbers \(a+bi\), \(a^{\prime}+b^{\prime}i\) is referred to the first genus, i.e., if both are congruent to the number \(3+2i\) according to the modulus 4.}\clearpage\noindent

However, despite the simplicity of this theorem, the proof of it should be referred to among the most hidden mysteries of the higher arithmetic, so that, at least for now, it can be unravelled only through the most subtle investigations, which would far exceed the limits of the present discussion. Therefore, we reserve the publication of this proof, as well as the development of the connection between this theorem and those which we began to establish by induction at the beginning of this discussion, for a third discussion. In the place of a conclusion, however, we will now present what is required for the proof of the theorems proposed in Articles 63, 64.
%

\subsection*{68.}

We begin with the prime numbers \(a+b i\), for which \(b=0\) (third kind of Art. 34), where, therefore (so that the number is prime among its associates), \(a\) must be a negative real prime number of the form \(-(4n+3)\), for which we write \(-q\), such as \(-3\), \(-7\), \(-11\), \(-19\) etc. By denoting with \(\lambda\) the character of the number \(1+i\), its modulus must be
\[i^{\lambda} \equiv (1+i)^{\frac{1}{4}(qq-1)} \equiv 2^{\frac{1}{8}(qq-1)} \cdot i^{\frac{1}{8}(qq-1)}\pmod{q}\]
But it is known that 2 is a quadratic residue or non-residue of \(q\) itself, depending on whether \(q\) is of the form \(8n+7\) or of the form \(8n+3\), from which we infer, in general,
\[2^{\frac{1}{2}(q-1)} \equiv (-1)^{\frac{1}{4}(q+1)} \equiv i^{\frac{1}{2}(q+1)}\pmod{q}\]
and so elevating to the power of the exponent \(\frac{1}{4}(q+1)\)
\[2^{\frac{1}{8}(qq-1)} \equiv i^{\frac{1}{8}(q+1)^{2}}\pmod{q}\]
Therefore, the preceding equation takes this form
\[i^{\lambda} \equiv i^{\frac{1}{8}(q+1)^{2}+\frac{1}{8}(qq-1)} \equiv i^{\frac{1}{4}(qq+q)}\pmod{q}\]
whence it follows
\[\lambda \equiv \frac{1}{4}(qq+q) \equiv \frac{1}{4}(q+1)^{2}-\frac{1}{4}(q+1)\pmod{4}\]
or since we have \(\frac{1}{4}(q+1)^{2} \equiv 0\pmod{4}\), \(\lambda \equiv -\frac{1}{4}(q+1) \equiv \frac{1}{4}(a-1)\pmod{4}\).
Which is the theorem itself of Art. 63 for the case \(b=0\).
%

\subsection*{69.}

Far more difficult are the cases of the moduli \(a+b i\) for which \(b\) is not equal to 0 (numbers of the fourth kind, art. 34), and various investigations need to be carried out beforehand. We will denote the norm \(a^2+b^2\), which will be a prime number in the form \(4n+1\), by \(p\).

Let \({S}\) be the set of all simply minimal residues for the modulus \(a+bi=m\), excluding 0, such that the number of numbers contained in \(S\) is \(=p-1\). Let \(x+yi\) denote an indefinite number of this system, and let \(ax+by=\xi\), \(ay-bx=\eta\) be assumed. Then \(\xi, \eta\) will be integers contained between the limits 0 and \(p\) \textit{exclusively}: in the present case, where \(a\), \(b\) are prime to one another, the formulas of art. 45, like \(\eta \equiv k \xi\), \(\xi \equiv-k \eta\pmod{p}\), show that neither of the numbers \(\xi\), \(\eta\) can be \(=0\) unless the other simultaneously vanishes, and thus \(x=0\), \(y=0\), a combination which we have already dismissed. Therefore, the criterion for the number \(x+yi\) to be contained in \(S\) is that the four numbers \(\xi\), \(\eta\), \(p-\xi\), \(p-\eta\) are positive.
%

Furthermore, we observe that for no such numbers can \(\xi=\eta\) hold; for it would then follow that \(p(x+y)=a(\xi+\eta)+b(\xi-\eta)=2a\xi\), which is absurd, as none of the factors \(2\), \(a\), \(\xi\) is divisible by \(p\). By a similar reasoning, the equation \(p(x-y+a+b)=2a\xi+(a+b)(p-\xi-\eta)\) shows that \(\xi+\eta\) cannot be equal to \(p\). Therefore, since the numbers \(\xi-\eta\), \(p-\xi-\eta\) must be either positive or negative, we seek the subdivision of the system \(S\) into four complexes \(C\), \(C^{\prime}\), \(C^{\prime\prime}\), \(C^{\prime\prime\prime}\), as it were, to be conjectured.
\begin{center}
\begin{tabular}{c|c}
the complex & contains the numbers for which \\
\hline
\(C\) & \(\xi-\eta\) is positive, \(p-\xi-\eta\) is positive \\
\(C^{\prime}\) & \(\xi-\eta\) is positive, \(p-\xi-\eta\) is negative \\
\(C^{\prime \prime}\) & \(\xi-\eta\) is negative, \(p-\xi-\eta\) is negative \\
\(C^{\prime \prime \prime}\) & \(\xi-\eta\) is negative, \(p-\xi-\eta\) is positive \\
\end{tabular}
\end{center}
Therefore, the criterion for the complex number \(C\) is properly sixfold, namely, six numbers \(\xi\), \(\eta\), \(p-\xi\), \(p-\eta\), \(\xi-\eta\), \(p-\xi-\eta\) must be positive; but clearly, conditions 2, 5, and 6 already imply the remaining ones. Similar considerations apply to the complexes \(C^{\prime}\), \(C^{\prime\prime}\), \(C^{\prime\prime\prime}\), so that the complete criteria are threefold, namely,
\begin{center}
\begin{tabular}{c|lll}
for the complex & \multicolumn{3}{c}{these numbers must be positive}   \\
\hline
\(C\) & \(\eta\), & \(\xi-\eta\), & \(p-\xi-\eta\) \\
\(C^{\prime}\) & \(p-\xi\) & \(\xi-\eta\), & \(\xi+\eta-p\)   \\
\(C^{\prime \prime}\) & \(p-\eta\)& \(\eta-\xi\), & \(\xi+\eta-p\)   \\
\(C^{\prime \prime \prime}\) & \(\xi\), & \(\eta-\xi\), & \(p-\xi-\eta\) \\
\end{tabular}
\end{center}
%

Moreover, even without our guidance, anyone will easily understand that, in the graphical representation of complex numbers (see art. 39), the numbers of the system \(S\) are contained within a square, whose sides connect points representing the numbers \(0\), \(a+b i\), \((1+i)(a+b i)\), \(i(a+b i)\), and the subdivision of the system \(S\) corresponds to the partition of the square by diagonal lines. However, in this place, we prefer to use purely arithmetic reasoning, leaving the illustration through figurative intuition to the knowledgeable reader for the sake of brevity.
%

\subsection*{70.}

If four complex numbers \(r=x+yi\), \(r^{\prime}=x^{\prime}+y^{\prime}i\), \(r^{\prime \prime}=x^{\prime \prime}+y^{\prime \prime}i\), \(r^{\prime \prime \prime}=x^{\prime \prime \prime}+y^{\prime \prime \prime}i\) are connected in such a way that \(r^{\prime}=m+ir\), \(r^{\prime \prime}=m+ir^{\prime}\) \(=(1+i)m-r\), \(r^{\prime \prime \prime}=m+ir^{\prime \prime}=im-ir\), and it is assumed that the first \(r\) belongs to the complex \(C\), the remaining \(r^{\prime}\), \(r^{\prime \prime}\), \(r^{\prime \prime \prime}\) respectively belong to the complexes \(C^{\prime}\), \(C^{\prime \prime}\), \(C^{\prime
\prime \prime}\). For if we assume \(\xi=ax+by\), \( \eta=ay-bx\), \(\xi^{\prime}=ax^{\prime}+by^{\prime}\), \( \eta^{\prime}=ay^{\prime}-bx^{\prime}\), \(\xi^{\prime \prime}=ax^{\prime \prime}+by^{\prime \prime}\), \( \eta^{\prime \prime}=ay^{\prime \prime}-bx^{\prime \prime}\), \(\xi^{\prime \prime \prime}=ax^{\prime \prime \prime}+by^{\prime \prime \prime}\), \( \eta^{\prime \prime \prime}=ay^{\prime \prime \prime}-bx^{\prime \prime \prime}\), we find
\[\begin{gathered}
\eta=p-\xi^{\prime}=p-\eta^{\prime \prime}=\xi^{\prime \prime \prime} \\
\xi-\eta=\xi^{\prime}+\eta^{\prime}-p=\eta^{\prime \prime}-\xi^{\prime \prime}=p-\xi^{\prime \prime \prime}-\eta^{\prime \prime \prime} \\
p-\xi-\eta=\xi^{\prime}-\eta^{\prime}=\xi^{\prime \prime}+\eta^{\prime \prime}-p=\eta^{\prime \prime \prime}-\xi^{\prime \prime \prime}
\end{gathered}\]
whence, with the help of criteria of the theorem, the truth naturally emerges. And when, moreover, \(r=m+ir^{\prime \prime \prime}\) is done again, it will be easily seen, if \(r\) is assumed to belong to \(C^{\prime}\), the numbers \(r^{\prime}\), \(r^{\prime \prime}\), \(r^{\prime \prime \prime}\) respectively belong to \(C^{\prime \prime}\), \(C^{\prime \prime \prime}\), \(C\); if it belongs to \(C^{\prime \prime}\), they belong to \(C^{\prime \prime \prime}\), \(C\), \(C^{\prime}\); finally, if it belongs to \(C^{\prime \prime \prime}\), they belong to \(C\), \(C^{\prime}\), \(C^{\prime \prime}\).

At the same time, it is inferred from here, that in each of the complexes \(C\), \(C^{\prime}\), \(C^{\prime \prime}\), \(C^{\prime \prime \prime}\) an equal number of numbers is found, namely \(\frac{1}{4}(p-1)\).
%

\subsection*{71.}
 
\textsc{Theorem.} \textit{If, denoting by \(k\) a whole number not divisible by \(m\), each of the sets of complex numbers \(C\) is multiplied by \(k\), and the residues of the products are simply minimized in terms of the modulus \(m\) among the sets \(C\), \( C^{\prime}\), \( C^{\prime \prime}\), \( C^{\prime \prime \prime}\) separated, the quantity related to each of these sets, respectively denoted by \(c\), \( c^{\prime}\), \( c^{\prime \prime}\), \( c^{\prime \prime \prime}\): the characteristic of the number \(k\) with respect to the modulus \(m\) will be \(\equiv c^{\prime}+2 c^{\prime \prime}+3 c^{\prime \prime \prime}\pmod{4}\).}
%

\textit{Proof.} Let \(c\) be the smallest residues related to \(C\), such as \(\alpha\), \( \beta\), \( \gamma\), \( \delta\), etc.; then let \(c^{\prime}\) be the residues related to \(C^{\prime}\), such as \(m+i \alpha^{\prime}\), \( m+i \beta^{\prime}\), \( m+i \gamma^{\prime}\), \( m+i \delta^{\prime}\), etc.; furthermore, let \(c^{\prime \prime}\) be the residues related to \(C^{\prime \prime}\), such as \((1+i) m-\alpha^{\prime \prime}\), \((1+i) m-\beta^{\prime \prime}\), \((1+i) m-\gamma^{\prime \prime}\), \((1+i) m-\delta^{\prime \prime}\), etc.; finally, let \(c^{\prime \prime \prime}\) be the residues related to \(C^{\prime \prime \prime}\), such as \(i m-i \alpha^{\prime \prime \prime}\), \( i m-i \beta^{\prime \prime \prime}\), \(i m-i \gamma^{\prime \prime \prime}\), \( i m-i \delta^{\prime \prime \prime}\), etc. Now let us consider four products, namely
\begin{enumerate}
\item[1)] the product of all \(\frac{1}{4}(p-1)\) complex numbers constituting \(C\);
\item[2)] the product of products arising from the multiplication of each of these numbers by \(k\);
\item[3)] the product of the smallest residues of these products, i.e., of numbers \(\alpha\), \( \beta\), \( \gamma\), \( \delta\), etc., \(m+i \alpha^{\prime}\), \( m+i \beta^{\prime}\) etc. etc.;
\item[4)] the product of all \(c+c^{\prime}+c^{\prime \prime}+c^{\prime \prime \prime}\) numbers \(\alpha\), \( \beta\), \( \gamma\), \( \delta\) etc., \(\alpha^{\prime}\), \( \beta^{\prime}\), \( \gamma^{\prime}\), \( \delta^{\prime}\) etc., \(\alpha^{\prime \prime}\), \( \beta^{\prime \prime}\), \( \gamma^{\prime \prime}\), \( \delta^{\prime \prime}\) etc., \(\alpha^{\prime \prime \prime}\), \( \beta^{\prime \prime \prime}\), \( \gamma^{\prime \prime \prime}\), \( \delta^{\prime \prime \prime}\) etc.
\end{enumerate}
Denoting these four products in their order by \(P\), \( P^{\prime} \), \( P^{\prime \prime}\), \( P^{\prime \prime \prime}\), it will be clear that
\[P^{\prime}=k^{\frac{1}{4}(p-1)} P, \quad P^{\prime} \equiv P^{\prime \prime}, \quad P^{\prime \prime} \equiv P^{\prime \prime \prime} i^{c^{\prime}+2 c^{\prime \prime}+3 c^{\prime \prime \prime}}\pmod{m}\]
and thus
\[P k^{\frac{1}{4}(p-1)} \equiv P^{\prime \prime \prime} i^{c^{\prime}+2 c^{\prime \prime}+3 c^{\prime \prime \prime}}\pmod{m}\]
But it will be easily seen that numbers \(\alpha^{\prime}\), \( \beta^{\prime}\), \( \gamma^{\prime}\), \( \delta^{\prime}\), etc., \(\alpha^{\prime \prime}\), \( \beta^{\prime \prime}\), \( \gamma^{\prime \prime}\), \( \delta^{\prime \prime}\), etc., \(\alpha^{\prime \prime \prime}\), \( \beta^{\prime \prime \prime}\), \( \gamma^{\prime \prime \prime}\), \( \delta^{\prime \prime \prime}\), etc. all belong to complex number \(C\), and both among themselves and from the numbers \(\alpha\), \( \beta\), \( \gamma\), \( \delta\) etc. they are different, just as these very numbers are different among themselves. Therefore, all these numbers taken together, and disregarding order, must be entirely identical with all the complex numbers constituting \(C\, \)\footnote{For each prime number \(p\) (p-1) factors will occur}. From this we deduce \(P=P^{\prime \prime \prime}\), and therefore
\[P k^{\frac{1}{4}(p-1)} \equiv P i^{c^{\prime}+2 c^{\prime \prime}+3 c^{\prime \prime \prime}}\pmod{m}\]
Finally, since each factor of the product \(P\) is not divisible by \(m\), we may conclude
\[k^{\frac{1}{4}(p-1)} \equiv i^{c^{\prime}+2 c^{\prime \prime}+3 c^{\prime \prime \prime}}\pmod{m}\]
whence \(c^{\prime}+2 c^{\prime \prime}+3 c^{\prime \prime \prime}\) will be the character of the number \(k\) with respect to the modulus \(m\). Q. E. D.
%

\subsection*{72.}

To apply the general theorem of the preceding article to the number \(1+i\), it is necessary to subdivide the complex \(C\) again into two smaller complexes \(G\) and \(G^{\prime}\), and indeed we will refer to the complex \(G\) the numbers \(x+yi\), for which \(ax+by = \xi\) is less than \(\frac{1}{2}p\), and to \(G^{\prime}\) those for which \(\xi\) is greater than \(\frac{1}{2}p\); we will denote the number of numbers contained in the complexes \(G\), \(G^{\prime}\) respectively by \(g\), \(g^{\prime}\), whence it will be \(g+g^{\prime} = \frac{1}{4}(p-1)\).

The complete criterion for the numbers pertaining to \(G\) will therefore be that three numbers \(\eta\), \(\xi-\eta\), \(p-2\xi\) are positive: for the third condition for the complex \(C\), according to which \(p-\xi-\eta\) must be positive, is implicitly contained in them, since \(p-\xi-\eta = (\xi-\eta) + (p-2\xi)\). Similarly, the complete criterion for the numbers pertaining to \(G^{\prime}\) will consist in the positive values of three numbers \(\eta\), \(p-\xi-\eta\), \(2\xi-p\).

Hence it is easily concluded that the product of any number of the complex \(G\) pertains to the complex \(C^{\prime\prime\prime}\); for if it is established
\[
(x+yi)(1+i) = x^{\prime}+y^{\prime}i, \text{ and } ax^{\prime}+by^{\prime}=\xi^{\prime}, ay^{\prime}-bx^{\prime}=\eta^{\prime}, \text{ is found }
\]
\[
\xi^{\prime} = \xi-\eta, \eta^{\prime}-\xi^{\prime}=2\eta, p-\xi^{\prime}-\eta^{\prime}=p-2\xi
\]
i.e. the criterion for the number \(x+yi\) subject to the complex \(G\) is identical to the criterion for the number \(x^{\prime}+y^{\prime}i\) pertaining to the complex \(C^{\prime\prime\prime}\).

It is completely similarly shown that the product of any number of the complex \(G^{\prime}\) by \(1+i\) pertains to the complex \(C^{\prime\prime}\).

Therefore, if in the preceding article we assign the value \(1+i\) to \(k\) itself, \(c=0\), \(c^{\prime}=0\), \(c^{\prime\prime}=g^{\prime}\), \(c^{\prime\prime\prime}=g\), and therefore for the character of the number \(1+i\) it will be \(3g+2g^{\prime} = \frac{1}{2}(p-1)+g\). And whereas the characters of the numbers \(i\), \(-1\), are \(\frac{1}{4}(p-1)\), \(\frac{1}{2}(p-1)\), the characters of the numbers \(-1+i\), \(-1-i\), \(1-i\) respectively will be \(\frac{3}{4}(p-1)+g\), \(g\), \(\frac{1}{4}(p-1)+g\). Therefore, the whole essence of the matter now turns on the investigation of the number \(g\).
%

\subsection*{73.}

What we have explained in articles 69-72 is properly independent of the assumption that \(m\) is a prime number: from now on, however, we will at least assume that \(a\) is odd and \(b\) is even, and further that \(a, b\), and \(a-b\) are positive numbers. First of all, it is necessary to establish the limits of the values of \(x\) in the complex \(G\).

By setting \(a y-b x=\eta\), \((a+b) x-(a-b) y=\zeta\), \(p-2 a x-2 b y=\theta\), the criterion for the numbers \(x+yi\) to belong to the complex \(G\) consists of three conditions: that \(\eta\), \(\zeta\), and \(\theta\) are positive numbers. Since \(p x=(a-b) \eta+a \zeta\), \(p(a-2 x)=a \theta+2 b \eta\), it is clear that \(x\) and \(2a-x\) must be positive numbers, i.e., \(x\) should be equal to one of the numbers \(1\), \(2\), \(3\ldots \frac{1}{2}(a-1)\). Furthermore, since \((a-b) \theta=2 b \zeta+p(a-b-2 x)\), it is evident that as long as \(x\) is less than \(\frac{1}{2}(a-b)\), the second condition (that \(\zeta\) must be positive) already implies the third condition (that \(\theta\) must be positive); conversely, whenever \(x\) is greater than \(\frac{1}{2}(a-b)\), the second condition is already contained in the third condition. Therefore, for the values of \(x\) being one of \(1\), \(2\), \(3\ldots \frac{1}{2}(a-b-1)\), it is only necessary to ensure that \(\eta\) and \(\zeta\) are positive, i.e., that \(y\) is greater than \(\frac{b x}{a}\) and less than \(\frac{(a+b) x}{a-b}\). Therefore, for such a given value of \(x\), the numbers \(x+yi\) will be present
\[\left[\frac{(a+b) x}{a-b}\right]-\left[\frac{b x}{a}\right]\]
if we use parentheses in the same sense as we have already used elsewhere (Compare \textit{Theorematis arithm. dem. nova} art. 4 and \textit{Theorematis fund. in doctr. de residuis quadr. etc. Algorithm. nov.} art. 3). On the other hand, for the values of \(x\) being \(\frac{1}{2}(a-b+1)\), \(\frac{1}{2}(a-b+3)\ldots \frac{1}{2}(a-1)\), it will suffice to reconcile the positive values of \(\eta\) and \(\theta\), i.e., that \(y\) is greater than \(\frac{b x}{a}\) and less than \(\frac{p-2 a x}{2 b}\) or \(\frac{1}{2} b+\frac{a a-2 a x}{2 b}\). Therefore, for such a given value of \(x\), the numbers \(x+yi\) will be present
\[\left[\frac{1}{2} b+\frac{a a-2 a x}{2 b}\right]-\left[\frac{b x}{a}\right]\]
Hence, we conclude that the set of numbers of the complex \(G\) is
\[g=\Sigma\left[\frac{(a+b) x}{a-b}\right]+\Sigma\left[\frac{1}{2} b+\frac{a a-2 a x}{2 b}\right]-\Sigma\left[\frac{b x}{a}\right]\]
where, in the first term, the summation should extend over all integral values of \(x\) from 1 to \(\frac{1}{2}(a-b-1)\), in the second from \(\frac{1}{2}(a-b+1)\) to \(\frac{1}{2}(a-1)\), and in the third from 1 to \(\frac{1}{2}(a-1)\).
%

\begin{enumerate}
    \item If we use the characteristic function \(\varphi\) in the same sense as used in the cited place (cf. \textit{Theorematis fund.} etc. \textit{Algor. nov.} art. 3), for example, if it is
    \[\varphi(t, u)=\left[\frac{u}{t}\right]+\left[\frac{2 u}{t}\right]+\left[\frac{3 u}{t}\right] \ldots+\left[\frac{t^{\prime} u}{t}\right]\]
    denoting \(t, u\) as any positive numbers, and \(t^{\prime}\) being the number \(\left[\frac{1}{2} t\right]\), then the first term is \(=\varphi(a-b, a+b)\), the third \(=-\varphi(a, b)\); but the second is
    \[ =\frac{1}{4} b b+\sum\left[\frac{a a-2 a x}{2 b}\right]\]
    But, by writing the terms in reverse order,
    \[\Sigma\left[\frac{a a-2 a x}{2 b}\right]=\left[\frac{a}{2 b}\right]+\left[\frac{3 a}{2 b}\right]+\left[\frac{5 a}{2 b}\right]+\ldots+\left[\frac{(b-1) a}{2 b}\right]=\varphi(2 b, a)-\varphi(b, a)\]
    Therefore, our formula takes the following form:
    \[g=\varphi(a-b, a+b)+\varphi(2 b . a)-\varphi(a, b)-\varphi(b, a)+\frac{1}{4} b b\]
    
    \item Let us consider the first term \(\varphi(a-b, a+b)\), which is immediately transformed into \(p(a-b, 2 b)+1+2+3+\text{etc.}+\frac{1}{2}(a-b-1)\) or into
    \[\varphi(a-b, 2 b)+\frac{1}{8}((a-b)^{2}-1)\]
    
    Then, since by the general theorem we have \(\varphi(t, u)+\varphi(u, t)=\left[\frac{1}{2} t\right] \cdot\left[\frac{1}{2} u\right]\) when \(t, u\) are positive integers prime to each other, we have
    \[\varphi(a-b, 2 b)=\frac{1}{2} b(a-b-1)-\varphi(2 b, a-b)\]
    and thus
    \[\varphi(a-b, a+b)=\frac{1}{8}(a a+2 a b-3 b b-4 b-1)-\varphi(2 b . a-b)\]
\end{enumerate}
%

\(\text{Let us arrange the parts of} \, \varphi(2 b, a-b) \, \text{in the following manner}\)
\[\begin{aligned}
& {\left[\frac{a-b}{2 b}\right]+\left[\frac{3(a-b)}{2 b}\right]+\left[\frac{5(a-b)}{2 b}\right]+\text{etc.}+\left[\frac{(b-1)(a-b)}{2 b}\right] } \\
+ & {\left[\frac{a-b}{b}\right]+\left[\frac{2(a-b)}{b}\right]+\left[\frac{3(a-b)}{b}\right]+\text{etc.}+\left[\frac{\frac{1}{2} b(a-b)}{b}\right] }
\end{aligned}\]
\text{The second series is evidently given by}
\[=\varphi(b, a-b)=\varphi(b, a)-1-2-3-\text{etc.}-\frac{1}{2} b=\varphi(b, a)-\frac{1}{8}(b^2+2b)d\]
\text{We represent the first series in reverse order of terms as follows:}
\[\left[\frac{1}{2}(a+1-b)-\frac{a}{2 b}\right]+\left[\frac{1}{2}(a+3-b)-\frac{3a}{2 b}\right]+\left[\frac{1}{2}(a+5-b)-\frac{5a}{2b}\right]+\text{etc.}+\left[\frac{1}{2}(a-1)-\frac{(b-1)a}{2b}\right]\]
\text{This expression, denoting} \, t \, \text{an integer,} \, u \, \text{a fraction, is generally} \, [t-u]=t-1-[u], \, \text{is transformed into the following}
\[\begin{aligned}
& \frac{1}{8}b(2a-4-b)-\left[\frac{a}{2b}\right]-\left[\frac{3a}{2b}\right]-\left[\frac{5a}{2b}\right]-\text{etc.}-\left[\frac{(b-1)a}{2b}\right] \\
= & \frac{1}{8}b(2a-4-b)-\varphi(2b,a)+\varphi(b,a)
\end{aligned}\]
%

\emph{Hence,}
\[\varphi(2 b, a-b)=2 \varphi(b, a)-\varphi(2 b, a)+\frac{1}{4} b(a-3-b)\]
\emph{and therefore}
\[\varphi(a-b, a+b)=\varphi(2 b, a)-2 \varphi(b, a)+\frac{1}{8}(a a-b b+2 b-1)\]
\emph{Substituting this value into the formula for \(g\) given above, and also using the fact that \(\varphi(a , b)+\varphi(b, a)=\frac{1}{4} b(a-1)\), we obtain}
\[g=2 \varphi(2 b, a)-2 \varphi(b, a)+\frac{1}{8}(a a-2 a b+b b+4 b-1)\]
%

\subsection*{74.}

The case is completely resolved by very similar reasoning, where \(a\), \(b\) remain positive and \(a-b\) is negative or \(b-a\) is positive. The equations \(p(a-2x) = 2b\eta + a\theta\), \(p(b-a+2x) = 2b\zeta + (b-a)\theta\) show that \(\frac{1}{2}a-x\) and \(x+\frac{1}{2}(b-a)\) are positive, and so \(x\) must be equal to one of the numbers \(-\frac{1}{2}(b-a-1)\), \(-\frac{1}{2}(b-a-3)\), \(-\frac{1}{2}(b-a-5) \ldots +\frac{1}{2}(a-1)\). Furthermore, from the equation \(px+(b-a)\eta = a\zeta\), it follows that for negative values of \(x\), the condition for \(\eta\) to be positive, is already contained in the condition for \(\zeta\) to be positive, but the contrary happens whenever a positive value is assigned to \(x\). Hence, the values of \(y\) for a given negative value of \(x\) must lie between \(\frac{(a+b)x}{a-b}\) and \(\frac{p-2ax}{2b}\), while for a positive value of \(x\), they must lie between \(\frac{bx}{a}\) and \(\frac{p-2ax}{2b}\). Here, it is evident that for \(x=0\) these limits are 0 and \(\frac{p-2ax}{2b}\), with the value \(y=0\) itself excluded. Thus, we deduce

\[g = -\sum\left[\frac{(a+b)x}{a-b}\right]+\sum\left[\frac{1}{2}b + \frac{a a-2ax}{2b}\right]-\sum\left[\frac{bx}{a}\right]\]

where in the first term, the summation extends over all negative values of \(x\) from \(a-1\) down to \(-\frac{1}{2}(b-a-1)\); in the second term, over all values of \(x\) from \(a-\frac{1}{2}(b-a-1)\) to \(\frac{1}{2}(a-1)\); and in the third, over all positive values of \(x\) from 1 to \(\frac{1}{2}(a-1)\). Thus, as a result of the first summation, it follows that \(-\varphi(b-a, b+a)\), from the second as in the preceding article, \(\frac{1}{4}b b + \varphi(2b, a) - \varphi(b, a)\), and finally from the third, \(-\varphi(a, b)\), giving us

\[g = -\varphi(b-a, b+a) + \varphi(2b, a) - \varphi(b, a) - \varphi(a, b) + \frac{1}{4}b b\]

In a similar manner as in the previous article, we find

\[\begin{aligned}
\varphi(b-a, b+a) & = \varphi(b-a, 2b) - \frac{1}{8}\left((b-a)^2-1\right) \\
& = \frac{1}{8}\left(3b b - 2ab - a a - 4b + 1\right) - \varphi(2b, b-a)
\end{aligned}\]

and also

\[\varphi(2b, b-a)=\varphi(2b, a)-2\varphi(b, a)+\frac{1}{4}b(b-1-a)\]

and hence

\[\varphi(b-a, b+a) = 2\varphi(b, a) - \varphi(2b, a) + \frac{1}{8}\left(b b - a a - 2b + 1\right)\]

and finally

\[g = 2\varphi(2b, a) - 2\varphi(b, a) + \frac{1}{8}\left(a a - 2ab + b b + 4b - 1\right)\]

It has therefore been shown that the same formula holds for \(g\), whether \(a-b\) is positive or negative, provided that \(a\), \(b\) are positive.
%

\subsection*{75.}

In order to obtain further reduction, we establish
\[\begin{aligned}
& L=\left[\frac{a}{2 b}\right]+\left[\frac{2 a}{2 b}\right]+\left[\frac{3 a}{2 b}\right]+\text{etc.}+\left[\frac{\frac{1}{2} b a}{2 b}\right] \\
& M=\left[\frac{(\frac{1}{2} b+1) a}{2 b}\right]+\left[\frac{(\frac{1}{2} b+2) a}{2 b}\right]+\left[\frac{(\frac{1}{2} b+3) a}{2 b}\right]+\text{etc.}+\left[\frac{b a}{2 b}\right] \\
& N=\left[\frac{a+b}{2 b}\right]+\left[\frac{2 a+b}{2 b}\right]+\left[\frac{3 a+b}{2 b}\right]+\text{etc.}+\left[\frac{\frac{1}{2} b a+b}{2 b}\right]
\end{aligned}\]
Since it is easily seen that in general, \([u]+\left[u+\frac{1}{2}\right]=[2 u]\), for whatever real quantity \(u\) denotes, we have \(L+N=\varphi(b, a)\), and since it is manifest that \(L+M=\varphi(2 b, a)\), it will be
\[\varphi(2 b, a)-\varphi(b, a)=M-N\]
Moreover, it is obvious that the sum of the first term of the series \(N\) with the penultimate term of the series \(M\), for example \(\left[\frac{a+b}{2 b}\right]+\left[\frac{(b-1) a}{2 b}\right]\) becomes \(=\frac{1}{2}(a-1)\), and the same sum is produced by the second term of the series \(N\) with the antepenultimate series \(M\), and so on. Therefore, since the ultimate term of the series \(M\) also becomes \(=\frac{1}{2}(a-1)\), and the ultimate term of the series \(N\) will be \(=\left[\frac{a+2}{4}\right]=\frac{1}{4}(a \mp 1)\), with the upper or lower sign depending on whether \(a\) is of the form \(4 n+1\) or \(4 n-1\): we have
\[M+N=\frac{1}{4}(a-1) b+\frac{1}{4}(a \mp 1)\]
and therefore
\[\varphi(2 b, a)-\varphi(b, a)=\frac{1}{4}(a-1) b+\frac{1}{4}(a \mp 1)-2 N\]\clearpage\noindent% 148
Thus, the formula for \(g\) found in articles 73 and 74 passes into the following
\[g=\frac{1}{8}((a+b)^{2}-1)+2 n-4 N\]
assuming \(a \mp 1=4 n\), where \(n\) is an integer. But since we obtain \(1=16 n n-8 a n+a a\) from here, this formula can also be expressed in the following way:
\[g=\frac{1}{8}(-a a+2 a b+b b+1)+4(\frac{1}{2}(a+1) n-n n-N)\]
Therefore, since \(g\) is the character of the number \(-1-i\) for the modulus \(a+b i\), this character becomes \(\equiv \frac{1}{8}(-a a+2 a b+b b+1)\pmod{4}\), which is the theorem above (art.64) obtained by induction, and hence the theorems concerning the characters of the numbers \(1+i\), \(1-i\), \(-1+i\) naturally follow. Therefore, these four theorems, for the case where \(a\) and \(b\) are positive, are now rigorously demonstrated.
%

\subsection*{76.}

If \(a\) remains positive and \(b\) is negative, let \(b=-b^{\prime}\), so that \(b^{\prime}\) becomes positive. Since it has already been proved, the character of the number \(-1-i\) for the modulus \(a+b^{\prime} i\) is \(\equiv \frac{1}{8}(-a a+2 a b^{\prime}+b^{\prime} b^{\prime}+1)\pmod{4}\), by the theorem in art. 62 the character of the number \(-1+i\) for the modulus \(a-b^{\prime} i\) will be \(\equiv \frac{1}{8}(a a-2 a b^{\prime}-b^{\prime} b^{\prime}-1)\), that is, the character of the number \(-1+i\) for the modulus \(a+b i\) becomes \(\equiv \frac{1}{8}(a a+2 a b-b b-1)\): but this is the same theorem as that mentioned in art. 64, from which the three remaining concerning the characters of the numbers \(1+i\), \(1-i\), \(-1-i\) are spontaneously removed. Therefore, these theorems have also been proved for the case where \(b\) is negative, namely for all cases where \(a\) is positive.

Finally, if \(a\) is negative, let \(a=-a^{\prime}, b=-b^{\prime}\). Therefore, since by what has already been proved, the character of the number \(1+i\) with respect to the modulus \(a^{\prime}+b^{\prime} i\) is \(\equiv \frac{1}{8}(-a^{\prime} a^{\prime}+2 a^{\prime} b^{\prime}-3 b^{\prime} b^{\prime}+1)\pmod{4}\), and it makes no difference as to whether we have the number \(a^{\prime}+b^{\prime} i\) or its opposite \(-a^{\prime}-b^{\prime} i\) in place of the modulus; it is manifest that the character of the number \(1+i\) with respect to the modulus \(a+b i\) is \(\equiv \frac{1}{8}(-a a+2 a b-3 b b+1)\), and the same is valid for the characters of the numbers \(1-i\), \(-1+i\), \(-1-i\).

From these, it is clear that the demonstration of the theorems concerning the characters of the numbers \(1+i\), \(1-i\), \(-1+i\), \(-1-i\) (arts. 63. 64) is no longer subject to any limitation.

\begin{center}\rule{1.5in}{0.5pt}\end{center}
\end{document}