\documentclass[12pt]{memoir}
\usepackage{standalone}
\usepackage[dvips,text={6.5truein,9.1truein},left=0.86truein,right=0.8truein,top=1truein]{geometry}
\usepackage{amsmath, amsthm, amsfonts}
\usepackage{titlesec}

% Uncomment to use syncing
%\usepackage{pdfsync}


% Paragraphs
\usepackage{indentfirst}
\parindent=3em
\parskip=0pt

%font
\usepackage{mlmodern}
%\usepackage[T1]{fontenc}% http://ctan.org/pkg/fontenc
\usepackage{microtype}

\titleformat{\section}
 {\centering}{\thesection.}{0em}{}

\titleformat{\subsection}
 {\normalfont\small\centering}{\thesection.}{0em}{}
\titlespacing*{\subsection}
{0pt}{\baselineskip}{0\baselineskip}

%footnotes
\usepackage[perpage]{footmisc}
\usepackage{etoolbox}
\DefineFNsymbols*{asterisks}{{ *}{ **}{ ***}}
\setfnsymbol{asterisks}
\renewcommand{\thefootnote}{\fnsymbol{footnote}}
\makeatletter
\renewcommand{\@makefnmark}{\mbox{\normalfont\@thefnmark})}
\settowidth{\footnotemargin}{***) }
\patchcmd{\@makefntext}{\hss\@makefnmark}{\hss \@makefnmark\ }{}{}
\makeatother

%Line Spacing
\renewcommand{\baselinestretch}{1.1}
\renewcommand{\footnotelayout}{ \baselineskip=1.02\baselineskip }
\setlength{\skip\footins}{2\baselineskip}

\theoremstyle{plain}
\newtheorem*{theorem}{Theorem}
\newtheorem{proposition}{Proposition}
\newtheorem{lemma}{Lemma}
\newtheorem{problem}{Problem}


\theoremstyle{remark}
\newtheorem*{example}{Example}
\newtheorem*{examples}{Examples}

\begin{document}
\section*{THIRD PROOF OF THE THEOREM ON THE RESOLVABILITY
OF INTEGRAL ALGEBRAIC FUNCTIONS INTO REAL FACTORS}

After the previous commentary had already been expressed in type, repeated meditations on the same topic led to a new demonstration of the theorem, which is indeed just as purely analytical as the previous one, but is based on completely different principles, and seems to be far preferable to it in terms of simplicity. The following pages are therefore dedicated to this third demonstration.

\subsection*{1.}

Let the function of the indeterminate \(x\) be given by
\[X = x^m + Ax^{m-1} + Bx^{m-2} + Cx^{m-3} + \text{ etc. } + Lx + M,\]
where the coefficients \(A,\) \(B,\) \(C\) etc. are fixed real quantities. Let \(r,\) \(\varphi\) be indeterminates, and consider
\[\begin{array}{rl}
r^m\cos m\varphi + A r^{m-1}\cos(m-1)\varphi + Br^{m-2}\cos(m-2)\varphi & \\
+ Cr^{m-3}\cos(m-3)\varphi + \text{ etc. } + Lr\cos\varphi + M &= t\\
r^m\sin m\varphi + Ar^{m-1}\sin(m-1)\varphi + Br^{m-2}\sin(m-2)\varphi & \\
+ Cr^{m-3}\sin(m-3)\varphi + \text{ etc. } + Lr\sin\varphi &= u\\
mr^m\cos m\varphi +(m-1)Ar^{m-1}\cos(m-1)\varphi+(m-2)Br^{m-2}\cos(m-2)\varphi &\\
+ (m-3)Cr^{m-3}\cos(m-3)\varphi + \text{ etc. } + Lr\cos\varphi &= t'\\
mr^m\sin m\varphi +(m-1)Ar^{m-1}\sin(m-1)\varphi + (m-2)Br^{m-2}\sin(m-2)\varphi &\\
+ (m-3)Cr^{m-3}\sin(m-3)\varphi + \text{ etc. } + Lr\sin\varphi &= u'\\
mmr^m\cos m\varphi+(m-1)^2Ar^{m-1}\cos(m-1)\varphi+(m-2)^2Br^{m-2}\cos(m-2)\varphi &\\
+ (m-3)^2Cr^{m-3}\cos(m-3)\varphi + \text{ etc. } + Lr\cos \varphi &= t''\\
mmr^m\sin m\varphi+(m-1)^2Ar^{m-1}\sin(m-1)\varphi+(m-2)^2Br^{m-2}\sin(m-2)\varphi &\\
+ (m-3)^2Cr^{m-3}\sin(m-3)\varphi + \text{ etc. } + Lr\sin\varphi &= u''
\end{array}\]
\[\frac{(tt+uu)(tt''+uu'')+(tu'-ut')^2-(tt'+uu')^2}{r(tt+uu)^2} = y.\]
The factor \(r\) can obviously be removed from the numerator and denominator of the final formula,  since it divides \(t',\) \(u',\) \(t'',\) \(u''\). Finally, let \(R\) be a positive quantity, arbitrarily chosen but greater than the maximum of the following quantities:
\[mA\sqrt{2}, \sqrt{mB\sqrt{2}}, \sqrt[3]{mC\sqrt{2}}, \sqrt[4]{mD\sqrt{2}},\]
where the signs of the quantities \(A,\) \(B,\) \(C\) etc. are excluded, i.e. the negatives, if any, have been changed to positives. These preparations being made, I say that \(tt'+uu'\) obtains a certain positive value when \(r = R,\) for any real value assigned to \(\varphi.\)

\textit{Proof.} Let us set
\begin{align*}
R^m\cos 45^o + A R^{m-1}\cos(45^o+\varphi)+BR^{m-2}\cos(45^o+2\varphi) &\\
+CR^{m-3}\cos(45^o+3\varphi)+\text{ etc. }+LR\cos(45^o+(m-1)\varphi)+M\cos(45^o+m\varphi) &= T\\
R^m\sin45^o+AR^{m-1}\sin(45^o+\varphi)+BR^{m-2}\sin(45^o+2\varphi)&\\
+CR^{m-3}\sin(45^o+3\varphi)+\text{etc.}+LR\sin(45^o+m\varphi)&= U \\
mR^m\cos 45^o +(m-1)AR^{m-1}\cos(45^o+\varphi)+(m-2)BR^{m-2}\cos(45^o+2\varphi)&\\
+(m-3)CR^{m-3}\cos(45^o+3\varphi)+\text{ etc. }+LR\cos(45^o+(m-1)\varphi) &= T'\\
mR^m\sin45^o +(m-1)AR^{m-1}\sin(45^o+\varphi) + (m-2)BR^{m-2}\sin(45^o+2\varphi)&\\
+(m-3)CR^{m-3}\sin(45^o+3\varphi)+\text{ etc. }+Lr\sin(45^o+(m-1)\varphi) &= U'.\end{align*}
Then it is clear that

I. \(T\) is composed of the parts
\begin{align*}
&\frac{R^{m-1}}{m\sqrt{2}}\left[R+mA\sqrt{2}.\cos(45^o+\varphi)\right]\\
+&\frac{R^{m-2}}{m\sqrt{2}}\left[RR+mB\sqrt{2}.\cos(45^o+2\varphi)\right]\\
+&\frac{R^{m-3}}{m\sqrt{2}}\left[R^3+mC\sqrt{2}.\cos(45^o+3\varphi)\right]\\
+&\frac{R^{m-4}}{m\sqrt{2}}\left[R^4+mD\sqrt{2}.\cos(45^o+4\varphi)\right]\\
+&\text{ etc.,} 
\end{align*}
each of which, for any determined real value of \(\varphi,\) is easily seen to be positive.  Hence, \(T\) necessarily takes a positive value. Similarly, it can be shown that \(U,\) \(T',\) \(U'\) are positive, and thus \(TT'+UU'\) is necessarily a positive quantity.

II. For \(r=R,\) the functions \(t,\) \(u,\) \(t',\) \(u'\) respectively become
\[\begin{array}{c}
T\cos(45^o+m\varphi)+U\sin(45^o+m\varphi)\\
T\sin(45^o+m\varphi)-U\cos(45^o+m\varphi)\\
T'\cos(45^o+m\varphi)+U'\sin(45^o+m\varphi)\\
T'\sin(45^o+m\varphi)-U'\cos(45^o+m\varphi)\end{array}\]
as can be easily proven by expanding. Thus for \(r=R\), the value of the function \(tt'+uu'\)  is derived to be \(=TT'+UU',\) and thus it is a positive quantity. Q.E.D.

Moreover, from the same formulas, we infer that for \(r=R\) the value of the function \(tt+uu\) is \(TT+UU,\) and therefore it is positive. Hence, we conclude that for no value of \(r,\) which is greater than \(mA\sqrt{2},\) \(\sqrt{mB\sqrt{2}},\) \(\sqrt[3]{mC\sqrt{2}}\) etc. simultaneously, is it possible to have \(t=0,\) \(u=0.\)

\subsection*{2.}

\textsc{Theorem.} \textit{Within the limits \(r=0\) and \(r=R,\) and \(\varphi=0\) and \(\varphi=360^o,\) there exist certain values of the indeterminates \(r,\) \(\varphi,\) for which \(t=0\) and \(u=0\) simultaneously.}

\textit{Proof.} Let us suppose that the theorem is not true.  Then it is evident that for all values of the indeterminates within the assigned limits, the value of \(tt+uu\)  must be a positive quantity, and therefore the value of \(y\) must always be finite. Let us consider the double integral
\[\iint y \,dr\,d\varphi\]
from \(r=0\) to \(r=R,\) and from \(\varphi=0\) to \(\varphi=360^o,\) which has a fully determined finite value.  This value, which we denote by \(\Omega,\) must be the same whether the integration is first carried out with respect to \(\varphi\) and then with respect to \(r,\) or in the reverse order. However, we have \textit{indefinitely}, considering \(r\) as constant,
\[\int y \,d\varphi = \frac{tu'-ut'}{r(tt+uu)},\]
as is easily confirmed by differentiation with respect to \(\varphi\).  A constant need not be added, assuming that the integral begins at \(\varphi=0,\) since for \(\varphi=0,\) we have \(\frac{tu'-ut'}{r(tt+uu)}=0\). Therefore, since \(\frac{tu'-ut'}{r(tt+uu)}\) clearly also vanishes for \(\varphi=360^o,\) the integral \(\displaystyle \int y\,d\varphi\) from \(\varphi=0\) to \(\varphi=360^o\) becomes \(=0,\) with \(r\) remaining indefinite.  It follows from this that \(\Omega=0.\)

On the other hand, considering \(\varphi\) as a constant, we have indefinitely
\[\int y \,dr = \frac{tt'+uu'}{tt+uu},\]
as is easily confirmed by differentiation with respect to \(r\). Here too, a constant need not be added, assuming that the integral starts at \(r=0\). Therefore, since the integral from \(r=0\) to \(r=R\) is carried out by what has been demonstrated in the previous article, it is \(=\frac{TT'+UU'}{TT+UU},\) and therefore, by the theorem in the previous article, it is always a positive quantity for any real value of \(\varphi\). Hence \(\Omega,\) i.e., the value of the integral
\[\int\tfrac{TT'+UU'}{TT+UU}d\varphi\]
from \(\varphi=0\) to \(\varphi=360^o,\) is necessarily a positive quantity\footnote{As is now self-evident. However, the indefinite integral is easily shown to be \(= m\varphi+45^o-\mathrm{arc.tang}\tfrac{U}{T}\), and it can be shown \textit{elsewhere} (since it is not immediately obvious which value from the infinitely many possible values of the multiform function \(\mathrm{arc.tang}\tfrac{U}{T}\) should be adopted for \(\varphi=360^o\)), that the definite integral from \(\varphi=0\) to \(\varphi=360^o\) will be \(=m\times360^o\) or \(=2 m\pi\). However, this is not necessary for our purpose.}. This is absurd, as we previously found that the same quantity is \(=0.\) Thus the assumption cannot hold, and the truth of the theorem is established.

\subsection*{3.}

The function \(X\) is transformed into \(t+u\sqrt{-1}\) by the substitution \(x = r(\cos\varphi + \sin\varphi \sqrt{-1}),\) and likewise it is transformed into \(t-u\sqrt{-1}\) by the substitution \(x=r(\cos\varphi-\sin\varphi.\sqrt{-1}).\)  Therefore, if for determined values of \(r\) and \(\varphi\), say for \(r=g\), \(\varphi=G\), it simultaneously results in \(t=0,\) \(u=0\) (as demonstrated in the previous article), then \(X\) obtains the value \(0\) for both substitutions
\[ x=g(\cos G+\sin G.\sqrt{-1}),\quad x=g(\cos G - \sin G.\sqrt{-1})\]
Consequently, it is indefinitely divisible by
\[ x-g(\cos G+\sin G\sqrt{-1}),\quad x-g(\cos G - \sin G.\sqrt{-1})\]
Whenever \(\sin G\) is not equal to 0, nor \(g=0\), these divisors are unequal. Thus, \(X\) is divisible by their product
\[xx-2g\cos G .x+gg\]
Whenever either \(\sin G =0\), and hence \(\cos G=\pm1\), or \(g=0\), these factors are identical, namely \(=x\mp g\). It is therefore certain that the function \(X\) involves a real divisor of the second or first order. Since the same conclusion holds for the quotient, \(X\) can be completely resolved into such factors. Q.E.D.

\subsection*{4.}

Although we have fully dealt with the matter proposed in the preceding section, it will not be superfluous to add some further reasoning about art. 2. Starting from the assumption that \(t\) and \(u\) vanish for any values of the indeterminates \(r, \varphi\) within the assigned limits, we have fallen into an inevitable contradiction, from which we concluded the falsity of the assumption itself. Therefore, this contradiction must cease if there are indeed values of \(r,\) \(\varphi,\) for which \(t\) and \(u\) simultaneously become \(=0.\) To illustrate this more clearly, we observe that for such values, \(tt+uu=0,\) and consequently, \(y\) becomes infinite. Hence, it will no longer be permissible to treat the double integral \(\iint y drd\varphi\) as an assignable quantity.

In general terms, denoting \(\xi,\) \(\eta,\) \(\zeta,\) as indefinite coordinates of points in space, the integral \(\iint ydrd\varphi\) represents the volume of a solid contained between five planes with equations:
\[ \xi=0\text{, }\eta=0\text{, }\zeta=0\text{, }\xi=R\text{, }\eta=360^o \]
and the surface with equation \(\zeta=y,\) considering those parts as negative where the \(\zeta\) coordinates are negative. However, it is implicitly understood here that the sixth surface is \textit{continuous}. When this condition ceases, and \(y\) becomes infinite, it is indeed possible that the concept lacks meaning. In such a case, it is impossible to speak about the integral \(\iint y drd\varphi\), and it is not surprising that analytical operations applied blindly to empty calculations lead to absurdities.

The integral \(\int y d\varphi = \frac{tu' - ut'}{r(tt + uu)}\) is a true integration, i.e., a summation, as long as within the limits over which it extends, \(y\) is everywhere a finite quantity. It becomes absurd, however, if \(y\) becomes infinite somewhere within those limits. For an integral like \(\int \eta d\xi,\) which generally represents the area between the abscissa line and the curve with the ordinate \(=\eta\) for the abscissa \(\xi,\) when we evaluate it according to usual rules, often ignoring continuity, we are frequently entangled in contradictions. For example, assuming \(\eta = \frac{1}{\xi\xi},\) the analysis provides an integral \(= C - \frac{1}{\xi},\) by which the area is correctly defined as long as the curve maintains continuity. However, if it is interrupted at \(\xi = 0,\) and someone incorrectly inquires about the magnitude of the area from the negative abscissa to the positive one, the formula will yield an absurd answer, stating that it is negative. We will explore the meaning of these and similar paradoxes of analysis more extensively on another occasion.

Let me add a final observation. In the \textit{unrestricted} statements of the questions, which may turn out to be absurd in certain cases, consulting analysis often leads to ambiguous answers. Thus, the value of the integral \(\iint y \, dr \, d\varphi\) from \(r=e\) to \(r=f\) and from \(\varphi = E\) to \(\varphi = F,\) if the value of \(\frac{u}{t}\)
\[ \begin{array}{ccccc} \text{for} & r=e,& \varphi = E ,& \text{is denoted by} &\theta \\
& r=e, & \varphi = F,& .\dots. &\theta' \\
& r=f, & \varphi = E,& .\dots. &\theta'' \\
& r=f, & \varphi = F,& .\dots. &\theta''' \end{array} \]
can be extended, through analytical operations, and is easily obtained as
\[ \mathrm{Arc.tang }\theta - \mathrm{Arc.tang }\theta' -\mathrm{Arc.tang }\theta''+\mathrm{Arc.tang }\theta''' \]
Of course, the integral can only have a definite value whenever \(y\) remains finite within the assigned limits. This value, given by the formula, is certainly satisfactory, but it is not yet fully determined through it, as \(\mathrm{Arc.tang}\) is a multiform function. It will be necessary to decide which values of the function should be used in the specific case, through other considerations (which are not difficult). On the other hand, whenever \(y\) becomes infinite somewhere within the assigned limits, the question of the value of the integral \(\iint y \, dr \, d\varphi\) is absurd. Despite this, if you insist on extracting an answer from the analysis, a variety of methods will lead you to the same result, in one way or another, each of which will be contained under the previously given general formula.

\end{document}







\begin{center}
THEOREMATIS DE RESOLUBILITATE

FUNCTIONUM ALGEBRAICARUM INTEGRARUM IN FACTORES REALES

DEMONSTRATIO TERTIA.
 
SUPPLEMENTUM COMMENTATIONIS PRAECEDENTIS.

 \end{center}

Postquam commentatio praecedens typis iam expressa esset, iteratae de eodem argumento meditationes ad novam theorematis demonstrationem perduxerunt, quae perinde quidem ac praecedens pure analytica est, sed principiis prorsus diversis innititur, et respectu simplicitatis illi longissime praeferenda videtur. Huic itaque tertiae demonstrationi pagellae sequentes dicatae sunto. 

\subsection*{1.} 

Proposita sit functio indeterminatae x haecce: 
\[X = x^m + Ax^{m-1}+Bx^{m-2}+Cx^{m-3}+\text{ etc. }+Lx+M\]
in qua coefficientes \(A,\) \(B,\) \(C\) etc. sunt quantitates reales determinatae. Sint \(r,\) \(\varphi\) alia indeterminatae, statuamusque 
\[\begin{array}{rl}
r^m\cos m\varphi + A r^{m-1}\cos(m-1)\varphi+Br^{m-2}\cos(m-2)\varphi & \\
+Cr^{m-3}\cos(m-3)\varphi+\text{ etc. }+Lr\cos\varphi+M&=t\\
r^m\sin m\varphi+Ar^{m-1}\sin(m-1)\varphi+Br^{m-2}\sin(m-2)\varphi&\\
+Cr^{m-3}\sin(m-3)\varphi+\text{etc.}+Lr\sin\varphi&=u \\
mr^m\cos m\varphi +(m-1)Ar^{m-1}\cos(m-1)\varphi+(m-2)Br^{m-2}\cos(m-2)\varphi&\\ +(m-3)Cr^{m-3}\cos(m-3)\varphi+\text{ etc. }+Lr\cos\varphi &= t'\\
mr^m\sin m\varphi +(m-1)Ar^{m-1}\sin(m-1)\varphi + (m-2)Br^{m-2}\sin(m-2)\varphi&\\ +(m-3)Cr^{m-3}\sin(m-3)\varphi+\text{ etc. }+Lr\sin\varphi &= u'\\
mmr^m\cos m\varphi+(m-1)^2Ar^{m-1}\cos(m-1)\varphi+(m-2)^2Br^{m-2}\cos(m-2)\varphi&\\ +(m-3)^2Cr^{m-3}\cos(m-3)\varphi +\text{ etc. }+Lr\cos \varphi &= t''\\
mmr^m\sin m\varphi+(m-1)^2Ar^{m-1}\sin(m-1)\varphi+(m-2)^2Br^{m-2}\sin(m-2)\varphi&\\ + (m-3)^2Cr^{m-3}\sin(m-3)\varphi+\text{ etc. }+Lr\sin\varphi&=u''\end{array}\] 
\[ \frac{(tt+uu)(tt''+uu'')+(tu'-ut')^2-(tt'+uu')^2}{r(tt+uu)^2} = y \]
Factorem \(r\) manifesto e denominatore formulae ultimae tollere licet, quum \(t',\) \(u',\) \(t'',\) \(u''\) per illum sint divisibiles. Denique sit \(R\) quantitas positiva determinata, arbitraria quidem, attamen maior maxima quantitatum 
\[ mA\surd{2} , \; \sqrt{mB\surd{2}}, \; \sqrt[3]{(mC\surd{2})}, \; \sqrt[4]{mD\surd{2}} \] 
abstrahendo a signis quantitatum \(A,\) \(B,\) \(C\) etc., i.e. mutatis negativis, si quae adsint, in positivas. His ita praeparatis, dico, \(tt'+uu'\) certo nancisci valorem 
positivum, si statuatur \(r=R,\) quicunque valor (realis) ipsi \(\varphi\) tribuatur. 

\textit{Demonstratio.} Statuamus 
\[\begin{array}{rl}
R^m\cos 45^o + A R^{m-1}\cos(45^o+\varphi)+BR^{m-2}\cos(45^o+2\varphi) &\\ +CR^{m-3}\cos(45^o+3\varphi)+\text{ etc. }+LR\cos(45^o+(m-1)\varphi)+M\cos(45^o+m\varphi) &=T\\
R^m\sin45^o+AR^{m-1}\sin(45^o+\varphi)+BR^{m-2}\sin(45^o+2\varphi)&\\+CR^{m-3}\sin(45^o+3\varphi)+\text{etc.}+LR\sin(45^o+m\varphi)&=U \\
mR^m\cos 45^o +(m-1)AR^{m-1}\cos(45^o+\varphi)+(m-2)BR^{m-2}\cos(45^o+2\varphi)&\\+(m-3)CR^{m-3}\cos(45^o+3\varphi)+\text{ etc. }+LR\cos(45^o+(m-1)\varphi) &= T'\\
mR^m\sin45^o +(m-1)AR^{m-1}\sin(45^o+\varphi) + (m-2)BR^{m-2}\sin(45^o+2\varphi)&\\+(m-3)Cr^{m-3}\sin(45^o+3\varphi)+\text{ etc. }+Lr\sin(45^o+(m-1)\varphi) &= U'\end{array}\]
patetque 

I. \(T\) compositam esse e partibus \[\begin{array}{rl}
&\frac{R^{m-1}{m\surd{2}\left[R+mA\surd{2}.\cos(45^o+\phi)\right]\\
+&\frac{R^{m-2}{m\surd{2}\left[RR+mB\surd{2}.\cos(45^o+2\phi)\right]\\
+&\frac{R^{m-3}{m\surd{2}\left[R^3+mC\surd{2}.\cos(45^o+3\phi)\right]\\
+&\frac{R^{m-4}{m\surd{2}\left[R^4+mD\surd{2}.\cos(45^o+4\phi)\right]\\
+&\text{etc.} \end{array}\]
quas singulas, pro valore quolibet determinato reali ipsius \(\varphi\), positivas evadere facile perspicitur: hinc \(T\) necessario valorem positivum obtinet. Simili modo probatur, etiam \(U, \) \(T',\) \(U'\) fieri positivas, unde etiam \(TT'+UU'\) necessario 
fit quantitas positiva. 

II. Pro \(r=R\) functiones \(t,\) \(u,\) \(t',\) \(u'\) resp. transeunt in 
\[\begin{array}{c}
T\cos(45^o+m\varphi)+U\sin(45^o+m\varphi)\\
T\sin(45^o+m\varphi)-U\cos(45^o+m\varphi)\\
T\cos(45^o+m\varphi)+U\sin(45^o+m\varphi)\\
T\sin(45^o+m\varphi)-U\cos(45^o+m\varphi)\end{array}\]
uti evolutione facta facile probatur. Hinc vero valor functionis \(tt'+uu'\), pro \(r=R,\) derivatur \(=TT'+UU',\) adeoque est quantitas positiva. Q.E.D. 

Ceterum ex iisdem formulis colligimus, valorem functionis \(tt+uu,\) pro \(r=R,\) esse \(TT+UU,\) adeoque positivum, unde concludimus, pro nullo valore ipsius \(r,\) singulis \(mA\surd{2},\) \(\sqrt{mB\surd{2},\) \(\sqrt[3]{mC\surd{2}\) etc. maiori, simul fieri posse \(t=0,\) \(u=0.\) 

\subsection*{2.} 

\textsc{Theorema.} \textit{Intra limites \(r=0\) et \(r=R,\) atque \(\varphi =0\) et \(\varphi = 360^o\) certo exstant valores tales indeterminatarum \(r,\) \(\varphi,\) pro quibus fiat simul \(t=0\) et \(u=0.\) }

\textit{Demonstratio.} Supponamus theorema non esse verum, patetque, valorem ipsius \(tt+uu\) pro cunctis valoribus indeterminatarum intra limites assignatos fieri debere quantitatem positivam, et proin valorem ipsius \(y\) semper finitum. Consideremus integrale duplex \[\iint y dr d\varphi \] ab \(r=0\) usque ad \(r=R,\) atque a \(\varphi =0\) usque ad \(\varphi = 360^o\) extensum, quod igitur valorem finitum plene determinatum nanciscitur. Hic valor, quem per \(\Omega\) denotabimus, idem prodire debebit, sive integratio primo instituatur secundum \(\varphi\) ac dein secundum \(r,\) sive ordine inverso. At habemus \textit{indefinite}, considerando \(r\) tamquam constantem, 
\[\int y d\varphi = \tfrac{tu'-ut'}{r(tt+uu)}\]
uti per differentiationem secundum \(\varphi\) facile confirmatur. Constans non adiicienda, siquidem integrale a \(\varphi=0\) incipiendum supponamus, quoniam pro \(\varphi=0\) fit \(\tfrac{tu'-ut'}{r(tt+uu)}=0\). Quare quum manifesto \(\tfrac{tu'-ut'}{r(tt+uu)}\) etiam evanescat pro \(\varphi=360^o,\) integrale \(\int yd\varphi\) a \(\varphi=0\) usque ad \(\varphi = 360^o\) fit \(=0,\) manente \(r\) 
indefinita. Hinc autem sequitur \(\Omega=0.\) 

Perinde habemus indefinite, considerando \(\varphi\) tamquam constantem, 
\[\int ydr = \frac{tt'+uu'}{tt+uu}\]
uti aeque facile per differentiationem secundum \(r\) confirmatur; hic quoque constans non adiicienda, integrali ab \(r=0\) incipiente. Quapropter integrale ab \(r=0\) usque ad \(r=R\) extensum fit per ea, quae in art. praec. demonstrata sunt, \(=\frac{TT'+UU'}{TT+UU}\) adeoque per theorema art. praec. semper quantitas positiva pro quolibet valore reali ipsius \(\varphi\). Hinc etiam \(\Omega\), i. e. valor integralis \[\int\tfrac{TT'+UU'}{TT+UU}d\varphi \] a \(\varphi =0\) usque ad \(\varphi = 360^o,\) necessario fit quantitas positiva<ref>Uti iam per se manifestum est. Ceterum integrale indefinitum facile eruitur \(= m\varphi+45^o-\mathrm{arc.tang}\tfrac{U}{T'}\) atque \textit{aliunde} demonstrari potest (per se enim nondum obvium est, quemnam valorem ex infinite multis functioni multiformi \(\mathrm{arc.tang}\tfrac{U}{T}\) competentibus pro \(\varphi = 360^o\) adoptare oporteat), huius valorem usque ad \(\varphi = 360^o\) extensum statui debere \(=m\times360^o\) sive \(=2 m\pi\). Sed hoc ad institutum nostrum non est necessarium.</ref>. Quod est absurdum, quoniam eandem quantitatem antea invenimus \(=0:\) suppositio itaque consistere nequit, theorematisque veritas hinc evicta est. 

\subsection*{3.} 

Functio \(X\) per substitutionem \(x = r(\cos\varphi + \sin\varphi \surd{-1})\) transit in \(t+u\surd{-1}\) nec non per substitutionem \(x=r(\cos\varphi-\sin\varphi.\surd{-1})\) in \(t-u\surd{-1}.\) Quodsi igitur pro valoribus determinatis ipsarum \(r,\) \(\varphi,\) puta pro \(r=g\), \(\varphi=G\), simul provenit \(t=0,\) \(a=0\) (quales valores exstare in art. praec. demonstratum est), \(X\) per utramque substitutionem \[ x=g(\cos G+\sin G.\surd{-1})\text{, }x=g(\cos G - \sin G.\surd{-1})\]
valorem 0 obtinet, et proin indefinite per 
\[ x-g(\cos G+\sin G\surd{-1})\text{, nec non per }x-g(\cos G - \sin G.\surd{-1})\]
divisibilis erit. Quoties non est \(\sin G=0,\) neque \(g=0,\) hi divisores sunt 
inaequales, et proin \(X\) etiam per illorum productum 
\[xx-2g\cos G .x+gg\]
divisibilis erit, quoties autem vel \(\sin G =0\) adeoque \(\cos G=\pm1,\) vel \(g=0,\) illi factores sunt identici scilicet \(=x\mp g\). Certum itaque est, functionem \(X\) involvere divisorem realem secundi vel primi ordinis, et quum eadem conclusio rursus de quotiente valeat, \(X\) in tales factores complete resolubilis erit. Q.E.D. 

\subsection*{4.} 
Quamquam in praecedentibus negotio quod propositum erat, iam plene perfuncti simus, tamen haud superfluum erit, adhuc quaedam de ratiocinatione art. 2 
adiicere. A suppositione, \(t\) et \(u\) pro nullis valoribus indeterminatarum \(r, \varphi\) intra limites illic assignatos simul evanescere, ad contradictionem inevitabilem delapsi sumus, unde ipsius suppositionis falsitatem conclusimus. Haec igitur contradictio cessare debet, si revera adsunt valores ipsarum \(r,\) \(\varphi,\) pro quibus \(t\) et \(u\) simul fiunt \(=0.\) Quod ut magis illustretur, observamus, pro talibus valoribus fieri \(tt+uu=0,\) adeoque ipsam \(y\) infinitam, unde haud amplius licebit, integrale duplex \(\iint y drd\varphi\) tamquam quantitatem assignabilem tractare. Generaliter quidem loquendo, denotantibus \(\xi,\) \(\eta,\) \(\zeta,\) indefinite coordinatas punctorum in spatio, integrale \(\iint ydrd\varphi\) exhibet volumen solidi, quod continetur inter quinque plana, quorum aequationes sunt 
\[ \xi=0\text{, }\eta=0\text{, }\zeta=0\text{, }\xi=R\text{, }\eta=360^o \]
atque superficiem, cuius aequatio \(\zeta=y,\) considerando eas partes tamquam negativas, in quibus coordinatae \(\zeta\) sunt negativae. Sed tacite hic subintelligitur, superficiem sextam esse \textit{continuam} qua conditione cessante, dum \(y\) evadit infinita, utique fieri potest, ut conceptus ille sensu careat. In tali casu de integrali \(\iint y drd\varphi\) colligendo sermo esse nequit, neque adeo mirandum est, operationes analyticas coeco calculo ad inania applicatas ad absurda perducere.

Integratio \(\int y d\varphi = \tfrac{tu'-ut'}{r(tt+uu)} \) eatenus tantum est integratio vera, i.e. summatio, quatenus inter limites, per quos extenditur, \(y\) ubique est quantitas finita, absurda autem, si inter illos limites \(y\) alicubi infinita evadit. Si integrale tale \(\int \eta d\xi,\) quod generaliter loquendo exhibet aream inter lineam abscissarum atque curvam, cuius ordinata \(=\eta\) pro abscissa \(\xi,\) secundum regulas suetas evolvimus, continuitatis immemores, saepissime contradictionibus implicamur. E.g. statuendo 
\(\eta = \frac{1}{\xi\xi},\) analysis suppeditat integrale \(= C-\frac{1}{\xi},\) quo area recte definitur, quamdiu curva continuitatem servat; qua pro \(\xi =0\) interrupta, si quis magnitudinem areae inde ab abscissa negativa usque ad positivam inepte rogat, responsum absurdum a formula feret, eam esse negativam. Quid autem sibi velint haec similiaque analyseos paradoxa, alia occasione fusius persequemur. 

Hic unicam observationem adiicere liceat. Propositis \textit{absque restrictione} quaestionibus , quae certis casibus absurdae evadere possunt, saepissime ita sibi consulit analysis, ut responsum ex parte vagum reddat. Ita pro valore integralis \(\iint y dr d\varphi\) ab \(r=e\) usque ad \(r=f,\) atque a \(\varphi = E\) usque ad \(\varphi = F\) extendendi, si valor ipsius \(\tfrac{u}{t}\)
\[ \begin{array}{ccccc}
\text{pro} & r=e,& \varphi=E,& \text{designatur per} &\theta \\
& r=e,& \varphi=F,& .\dots. &\theta' \\
& r=f,& \varphi=E,& .\dots. &\theta'' \\
& r=f,& \varphi=F,& .\dots. &\theta''' \end{array} \]
per operationes analyticas facile obtinetur 
\[ \mathrm{Arc.tang }\theta - \mathrm{Arc.tang }\theta' -\mathrm{Arc.tang }\theta''+\mathrm{Arc.tang }\theta''' \]
Revera quidem integrale tunc tantum valorem certum habere potest, quoties \(y\) inter limites assignatos semper manet finita: hic valor sub formula tradita utique contentus, tamen per eam nondum ex asse definitur, quoniam \(\mathrm{Arc.tang.}\) est functio multiformis, seorsimque per alias considerationes (haud quidem difficiles) decidere oportebit, quinam potissimum functionis valores in casu determinato sint adhibendi. Contra quoties \(y\) alicubi inter limites assignatos infinita evadit, quaestio 
de valore integralis \(\iint y dr d\varphi\) absurda est: quo non obstante si responsum ab analysi extorquere obstinaveris, pro methodorum diversitate modo hoc modo illud reddetur, quae tamen singula sub formula generali ante tradita contenta erunt. 





